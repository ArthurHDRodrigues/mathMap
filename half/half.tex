\documentclass[12pt]{article}
\usepackage[utf8]{inputenc}
\usepackage{pgf,tikz,pgfplots}
\pgfplotsset{compat=1.15}
\usepackage{mathrsfs}
\usetikzlibrary{arrows}
\usepackage{fontspec}
\setmainfont[Renderer=ICU,Mapping=tex-text]{Cousine}
\usepackage{amssymb}
\usepackage[paperwidth=453.4600000000002cm,paperheight=152.89999999999998cm,left=0.1cm,right=0.1cm,top=0.1cm,bottom=0.1cm]{geometry}
\begin{document}\begin{tikzpicture}[line cap=round,line join=round,>=triangle 45,x=1cm,y=1cm]
\clip(0, 0)rectangle(449.4600000000002, -148.89999999999998);

\draw(0, 0) node[anchor=north west,align=left] {\Huge half};
\draw (0, 0) rectangle (449.4600000000002,-148.89999999999998);
\draw(1, -1) node[anchor=north west,align=left] {\LARGE Partial differential equations};
\draw (1, -1) rectangle (76.86999999999999,-52.4);
\draw(2, -2) node[anchor=north west,align=left] {\large General higher-order partial differential equations and systems of higher-order partial differential equations};
\draw (2, -2) rectangle (42.6,-8.4);
\draw(3, -3) node[anchor=north west,align=left] {Initial value\\ problems \\ for linear \\ higher-order PDEs};
\draw (3, -3) rectangle (7.85,-5.1);
\draw(7.949999999999999, -3) node[anchor=north west,align=left] {Boundary value\\ problems for\\ linear \\ higher-order PDEs};
\draw (7.949999999999999, -3) rectangle (12.799999999999999,-5.1);
\draw(12.899999999999999, -3) node[anchor=north west,align=left] {Initial-boundary\\ value \\ problems for\\ linear \\ higher-order PDEs};
\draw (12.899999999999999, -3) rectangle (17.75,-5.6);
\draw(17.849999999999998, -3) node[anchor=north west,align=left] {Initial \\ value problems\\ for \\ nonlinear \\ higher-order PDEs};
\draw (17.849999999999998, -3) rectangle (22.699999999999996,-5.6);
\draw(22.799999999999997, -3) node[anchor=north west,align=left] {Boundary \\ value problems\\ for nonlinear\\ higher-order PDEs};
\draw (22.799999999999997, -3) rectangle (27.65,-5.6);
\draw(27.749999999999996, -3) node[anchor=north west,align=left] {Initial-boundary\\ value \\ problems for\\ nonlinear \\ higher-order PDEs};
\draw (27.749999999999996, -3) rectangle (32.599999999999994,-5.6);
\draw(32.699999999999996, -3) node[anchor=north west,align=left] {Systems\\ of linear\\ higher-order PDEs};
\draw (32.699999999999996, -3) rectangle (37.55,-5.1);
\draw(37.65, -3) node[anchor=north west,align=left] {Initial value\\ problems \\ for systems \\ of linear \\ higher-order PDEs};
\draw (37.65, -3) rectangle (42.5,-5.6);
\draw(3, -5.7) node[anchor=north west,align=left] {Boundary value\\ problems\\ for systems\\ of linear \\ higher-order PDEs};
\draw (3, -5.7) rectangle (7.85,-8.3);
\draw(7.949999999999999, -5.7) node[anchor=north west,align=left] {Initial-boundary\\ value problems\\ for systems\\ of linear \\ higher-order PDEs};
\draw (7.949999999999999, -5.7) rectangle (12.799999999999999,-8.3);
\draw(12.899999999999999, -5.7) node[anchor=north west,align=left] {Systems \\ of nonlinear\\ higher-order PDEs};
\draw (12.899999999999999, -5.7) rectangle (17.75,-7.800000000000001);
\draw(17.849999999999998, -5.7) node[anchor=north west,align=left] {Initial value\\ problems for\\ systems of \\ nonlinear \\ higher-order PDEs};
\draw (17.849999999999998, -5.7) rectangle (22.699999999999996,-8.3);
\draw(22.799999999999997, -5.7) node[anchor=north west,align=left] {Boundary value\\ problems for\\ systems of\\ nonlinear \\ higher-order PDEs};
\draw (22.799999999999997, -5.7) rectangle (27.65,-8.3);
\draw(27.749999999999996, -5.7) node[anchor=north west,align=left] {Initial-boundary\\ value problems\\ for systems\\ of nonlinear\\ higher-order PDEs};
\draw (27.749999999999996, -5.7) rectangle (32.599999999999994,-8.3);
\draw(32.699999999999996, -5.7) node[anchor=north west,align=left] {Linear \\ higher-order\\ PDEs};
\draw (32.699999999999996, -5.7) rectangle (36.3,-7.300000000000001);
\draw(36.4, -5.7) node[anchor=north west,align=left] {Nonlinear\\ higher-order\\ PDEs};
\draw (36.4, -5.7) rectangle (40.0,-7.300000000000001);
\draw(42.7, -2) node[anchor=north west,align=left] {\large Partial differential equations and systems of partial differential equations with constant coefficients};
\draw (42.7, -2) rectangle (75.23,-6.199999999999999);
\draw(43.7, -3) node[anchor=north west,align=left] {Convexity \\ properties of \\ solutions to PDEs\\ and systems of\\ PDEs with \\ constant coefficients};
\draw (43.7, -3) rectangle (49.550000000000004,-6.1);
\draw(49.650000000000006, -3) node[anchor=north west,align=left] {Initial value \\ problems for PDEs\\ and systems \\ of PDEs with \\ constant coefficients};
\draw (49.650000000000006, -3) rectangle (55.50000000000001,-5.6);
\draw(55.6, -3) node[anchor=north west,align=left] {General theory\\ of PDEs and\\ systems of \\ PDEs with \\ constant coefficients};
\draw (55.6, -3) rectangle (61.45,-5.6);
\draw(61.550000000000004, -3) node[anchor=north west,align=left] {Fundamental \\ solutions to PDEs\\ and systems of\\ PDEs with constant\\ coefficients};
\draw (61.550000000000004, -3) rectangle (66.65,-5.6);
\draw(2, -8.5) node[anchor=north west,align=left] {\large General first-order partial differential equations and systems of first-order partial differential equations};
\draw (2, -8.5) rectangle (41.6,-14.9);
\draw(3, -9.5) node[anchor=north west,align=left] {Hamilton-Jacobiequations};
\draw (3, -9.5) rectangle (9.6,-11.1);
\draw(9.7, -9.5) node[anchor=north west,align=left] {Initial value\\ problems \\ for linear \\ first-order PDEs};
\draw (9.7, -9.5) rectangle (14.299999999999999,-11.6);
\draw(14.399999999999999, -9.5) node[anchor=north west,align=left] {Boundary value\\ problems\\ for linear \\ first-order PDEs};
\draw (14.399999999999999, -9.5) rectangle (19.0,-11.6);
\draw(19.1, -9.5) node[anchor=north west,align=left] {Initial-boundary\\ value\\ problems \\ for linear \\ first-order PDEs};
\draw (19.1, -9.5) rectangle (23.700000000000003,-12.1);
\draw(23.8, -9.5) node[anchor=north west,align=left] {Initial value\\ problems for\\ nonlinear \\ first-order PDEs};
\draw (23.8, -9.5) rectangle (28.4,-11.6);
\draw(28.5, -9.5) node[anchor=north west,align=left] {Boundary \\ value problems\\ for \\ nonlinear \\ first-order PDEs};
\draw (28.5, -9.5) rectangle (33.1,-12.1);
\draw(33.2, -9.5) node[anchor=north west,align=left] {Initial-boundary\\ value \\ problems for\\ nonlinear \\ first-order PDEs};
\draw (33.2, -9.5) rectangle (37.800000000000004,-12.1);
\draw(37.9, -9.5) node[anchor=north west,align=left] {Systems \\ of nonlinear\\ first-order\\ PDEs};
\draw (37.9, -9.5) rectangle (41.5,-11.6);
\draw(3, -12.2) node[anchor=north west,align=left] {Initial value\\ problems \\ for systems \\ of linear \\ first-order PDEs};
\draw (3, -12.2) rectangle (7.6,-14.799999999999999);
\draw(7.699999999999999, -12.2) node[anchor=north west,align=left] {Boundary value\\ problems\\ for systems\\ of linear \\ first-order PDEs};
\draw (7.699999999999999, -12.2) rectangle (12.299999999999999,-14.799999999999999);
\draw(12.399999999999999, -12.2) node[anchor=north west,align=left] {Initial-boundary\\ value problems\\ for systems\\ of linear \\ first-order PDEs};
\draw (12.399999999999999, -12.2) rectangle (17.0,-14.799999999999999);
\draw(17.099999999999998, -12.2) node[anchor=north west,align=left] {Initial value\\ problems \\ for systems \\ of nonlinear\\ first-order PDEs};
\draw (17.099999999999998, -12.2) rectangle (21.699999999999996,-14.799999999999999);
\draw(21.799999999999997, -12.2) node[anchor=north west,align=left] {Boundary value\\ problems for\\ systems of\\ nonlinear \\ first-order PDEs};
\draw (21.799999999999997, -12.2) rectangle (26.4,-14.799999999999999);
\draw(26.499999999999996, -12.2) node[anchor=north west,align=left] {Initial-boundary\\ value problems\\ for systems\\ of nonlinear\\ first-order PDEs};
\draw (26.499999999999996, -12.2) rectangle (31.099999999999994,-14.799999999999999);
\draw(31.199999999999996, -12.2) node[anchor=north west,align=left] {Linear\\ first-order\\ PDEs};
\draw (31.199999999999996, -12.2) rectangle (34.55,-13.799999999999999);
\draw(34.65, -12.2) node[anchor=north west,align=left] {Nonlinear\\ first-order\\ PDEs};
\draw (34.65, -12.2) rectangle (38.0,-13.799999999999999);
\draw(38.1, -12.2) node[anchor=north west,align=left] {Systems\\ of linear\\ first-order\\ PDEs};
\draw (38.1, -12.2) rectangle (41.45,-14.299999999999999);
\draw(41.7, -8.5) node[anchor=north west,align=left] {\large Qualitative properties of solutions to partial differential equations};
\draw (41.7, -8.5) rectangle (67.7,-21.5);
\draw(42.7, -9.5) node[anchor=north west,align=left] {Oscillation, \\ zeros of solutions,\\ mean value \\ theorems, etc. \\ in context of PDEs};
\draw (42.7, -9.5) rectangle (48.050000000000004,-12.1);
\draw(48.150000000000006, -9.5) node[anchor=north west,align=left] {Critical points\\ of functionals\\ in context of\\ PDEs (e.g., \\ energy functionals)};
\draw (48.150000000000006, -9.5) rectangle (53.50000000000001,-12.1);
\draw(53.6, -9.5) node[anchor=north west,align=left] {Homogenization\\ in context of\\ PDEs; PDEs in\\ media with \\ periodic structure};
\draw (53.6, -9.5) rectangle (58.7,-12.1);
\draw(58.800000000000004, -9.5) node[anchor=north west,align=left] {Dependence of \\ solutions to PDEs\\ on initial \\ and/or boundary \\ data and/or on \\ parameters of PDEs};
\draw (58.800000000000004, -9.5) rectangle (63.900000000000006,-12.6);
\draw(64.0, -9.5) node[anchor=north west,align=left] {Bifurcations\\ in context\\ of PDEs};
\draw (64.0, -9.5) rectangle (67.6,-11.1);
\draw(42.7, -12.7) node[anchor=north west,align=left] {Almost and \\ pseudo-almost\\ periodic \\ solutions to PDEs};
\draw (42.7, -12.7) rectangle (47.550000000000004,-14.799999999999999);
\draw(47.650000000000006, -12.7) node[anchor=north west,align=left] {Liouville \\ theorems and \\ Phragmén-Lindelöf\\ theorems in\\ context of PDEs};
\draw (47.650000000000006, -12.7) rectangle (52.50000000000001,-15.299999999999999);
\draw(52.6, -12.7) node[anchor=north west,align=left] {Continuation\\ and prolongation\\ of \\ solutions to PDEs};
\draw (52.6, -12.7) rectangle (57.45,-14.799999999999999);
\draw(57.55, -12.7) node[anchor=north west,align=left] {Smoothness\\ and regularity\\ of \\ solutions to PDEs};
\draw (57.55, -12.7) rectangle (62.4,-14.799999999999999);
\draw(62.5, -12.7) node[anchor=north west,align=left] {Symmetries,\\ invariants,\\ etc. in \\ context of PDEs};
\draw (62.5, -12.7) rectangle (66.85,-14.799999999999999);
\draw(42.7, -15.4) node[anchor=north west,align=left] {Perturbations\\ in \\ context of PDEs};
\draw (42.7, -15.4) rectangle (47.050000000000004,-17.0);
\draw(47.150000000000006, -15.4) node[anchor=north west,align=left] {Singular \\ perturbations\\ in context\\ of PDEs};
\draw (47.150000000000006, -15.4) rectangle (51.00000000000001,-17.5);
\draw(51.1, -15.4) node[anchor=north west,align=left] {Asymptotic\\ behavior\\ of solutions\\ to PDEs};
\draw (51.1, -15.4) rectangle (54.7,-17.5);
\draw(54.800000000000004, -15.4) node[anchor=north west,align=left] {Critical\\ exponents\\ in context\\ of PDEs};
\draw (54.800000000000004, -15.4) rectangle (57.900000000000006,-17.5);
\draw(58.0, -15.4) node[anchor=north west,align=left] {Resonance\\ in context\\ of PDEs};
\draw (58.0, -15.4) rectangle (61.1,-17.0);
\draw(61.2, -15.4) node[anchor=north west,align=left] {Stability\\ in context\\ of PDEs};
\draw (61.2, -15.4) rectangle (64.3,-17.0);
\draw(64.4, -15.4) node[anchor=north west,align=left] {Pattern \\ formations\\ in context\\ of PDEs};
\draw (64.4, -15.4) rectangle (67.5,-17.5);
\draw(42.7, -17.6) node[anchor=north west,align=left] {Attractors};
\draw (42.7, -17.6) rectangle (45.800000000000004,-18.700000000000003);
\draw(45.900000000000006, -17.6) node[anchor=north west,align=left] {Blow-up\\ in context\\ of PDEs};
\draw (45.900000000000006, -17.6) rectangle (49.00000000000001,-19.200000000000003);
\draw(49.1, -17.6) node[anchor=north west,align=left] {A priori\\ estimates\\ in context\\ of PDEs};
\draw (49.1, -17.6) rectangle (52.2,-19.700000000000003);
\draw(52.300000000000004, -17.6) node[anchor=north west,align=left] {Maximum \\ principles\\ in context\\ of PDEs};
\draw (52.300000000000004, -17.6) rectangle (55.400000000000006,-19.700000000000003);
\draw(55.5, -17.6) node[anchor=north west,align=left] {Comparison\\ principles\\ in context\\ of PDEs};
\draw (55.5, -17.6) rectangle (58.6,-19.700000000000003);
\draw(58.7, -17.6) node[anchor=north west,align=left] {Axially\\ symmetric\\ solutions\\ to PDEs};
\draw (58.7, -17.6) rectangle (61.550000000000004,-19.700000000000003);
\draw(61.650000000000006, -17.6) node[anchor=north west,align=left] {Entire \\ solutions\\ to PDEs};
\draw (61.650000000000006, -17.6) rectangle (64.5,-19.200000000000003);
\draw(64.6, -17.6) node[anchor=north west,align=left] {Positive\\ solutions\\ to PDEs};
\draw (64.6, -17.6) rectangle (67.44999999999999,-19.200000000000003);
\draw(42.7, -19.8) node[anchor=north west,align=left] {Periodic\\ solutions\\ to PDEs};
\draw (42.7, -19.8) rectangle (45.550000000000004,-21.400000000000002);
\draw(45.650000000000006, -19.8) node[anchor=north west,align=left] {Inertial\\ manifolds};
\draw (45.650000000000006, -19.8) rectangle (48.50000000000001,-21.400000000000002);
\draw(2, -15.0) node[anchor=north west,align=left] {\large Representations of solutions to partial differential equations};
\draw (2, -15.0) rectangle (25.450000000000003,-19.9);
\draw(3, -16.0) node[anchor=north west,align=left] {Integral \\ representations\\ of solutions\\ to PDEs};
\draw (3, -16.0) rectangle (7.35,-18.1);
\draw(7.449999999999999, -16.0) node[anchor=north west,align=left] {Trigonometric\\ solutions\\ to PDEs};
\draw (7.449999999999999, -16.0) rectangle (11.299999999999999,-17.6);
\draw(11.4, -16.0) node[anchor=north west,align=left] {Self-similar\\ solutions\\ to PDEs};
\draw (11.4, -16.0) rectangle (15.0,-17.6);
\draw(15.1, -16.0) node[anchor=north west,align=left] {Asymptotic\\ expansions\\ of solutions\\ to PDEs};
\draw (15.1, -16.0) rectangle (18.7,-18.1);
\draw(18.8, -16.0) node[anchor=north west,align=left] {Solutions\\ to PDEs in\\ closed form};
\draw (18.8, -16.0) rectangle (22.150000000000002,-17.6);
\draw(22.25, -16.0) node[anchor=north west,align=left] {Polynomial\\ solutions\\ to PDEs};
\draw (22.25, -16.0) rectangle (25.35,-17.6);
\draw(3, -18.2) node[anchor=north west,align=left] {Traveling\\ wave \\ solutions};
\draw (3, -18.2) rectangle (5.85,-19.8);
\draw(5.95, -18.2) node[anchor=north west,align=left] {Soliton\\ solutions};
\draw (5.95, -18.2) rectangle (8.8,-19.8);
\draw(8.9, -18.2) node[anchor=north west,align=left] {Series \\ solutions\\ to PDEs};
\draw (8.9, -18.2) rectangle (11.75,-19.8);
\draw(2, -20.0) node[anchor=north west,align=left] {\large History of partial \\ differential equations};
\draw (2, -20.0) rectangle (9.42,-21.1);
\draw(67.8, -8.5) node[anchor=north west,align=left] {\large Close-to-elliptic equations};
\draw (67.8, -8.5) rectangle (76.77,-12.9);
\draw(68.8, -9.5) node[anchor=north west,align=left] {Quasiellipticequations};
\draw (68.8, -9.5) rectangle (74.89999999999999,-11.1);
\draw(68.8, -11.2) node[anchor=north west,align=left] {Hypoelliptic\\ equations};
\draw (68.8, -11.2) rectangle (72.39999999999999,-12.799999999999999);
\draw(72.5, -11.2) node[anchor=north west,align=left] {Subelliptic\\ equations};
\draw (72.5, -11.2) rectangle (75.85,-12.799999999999999);
\draw(2, -21.6) node[anchor=north west,align=left] {\large General topics in partial differential equations};
\draw (2, -21.6) rectangle (20.35,-36.1);
\draw(3, -22.6) node[anchor=north west,align=left] {Inequalities \\ applied to PDEs \\ involving derivatives,\\ differential\\ and integral \\ operators, or integrals};
\draw (3, -22.6) rectangle (9.35,-25.700000000000003);
\draw(9.45, -22.6) node[anchor=north west,align=left] {Cauchy-Kovalevskaya\\ theorems};
\draw (9.45, -22.6) rectangle (14.799999999999999,-24.200000000000003);
\draw(14.899999999999999, -22.6) node[anchor=north west,align=left] {Microlocal \\ methods and methods\\ of sheaf \\ theory and \\ homological algebra\\ applied to PDEs};
\draw (14.899999999999999, -22.6) rectangle (20.25,-25.700000000000003);
\draw(3, -25.8) node[anchor=north west,align=left] {Existence problems\\ for PDEs: \\ global existence,\\ local existence,\\ non-existence};
\draw (3, -25.8) rectangle (8.1,-28.400000000000002);
\draw(8.2, -25.8) node[anchor=north west,align=left] {Geometric \\ theory, \\ characteristics, \\ transformations \\ in context of PDEs};
\draw (8.2, -25.8) rectangle (13.299999999999999,-28.400000000000002);
\draw(13.399999999999999, -25.8) node[anchor=north west,align=left] {Uniqueness \\ problems for \\ PDEs: global \\ uniqueness, local\\ uniqueness,\\ non-uniqueness};
\draw (13.399999999999999, -25.8) rectangle (18.25,-28.900000000000002);
\draw(3, -29.0) node[anchor=north west,align=left] {Topological \\ and monotonicity\\ methods \\ applied to PDEs};
\draw (3, -29.0) rectangle (7.6,-31.1);
\draw(7.699999999999999, -29.0) node[anchor=north west,align=left] {Transform \\ methods (e.g.,\\ integral \\ transforms) \\ applied to PDEs};
\draw (7.699999999999999, -29.0) rectangle (12.049999999999999,-31.6);
\draw(12.149999999999999, -29.0) node[anchor=north west,align=left] {Methods of\\ ordinary \\ differential\\ equations \\ applied to PDEs};
\draw (12.149999999999999, -29.0) rectangle (16.5,-31.6);
\draw(16.599999999999998, -29.0) node[anchor=north west,align=left] {Parametrices\\ in context\\ of PDEs};
\draw (16.599999999999998, -29.0) rectangle (20.2,-30.6);
\draw(3, -31.700000000000003) node[anchor=north west,align=left] {Other \\ special methods\\ applied\\ to PDEs};
\draw (3, -31.700000000000003) rectangle (7.35,-33.800000000000004);
\draw(7.449999999999999, -31.700000000000003) node[anchor=north west,align=left] {Theoretical\\ approximation\\ in context\\ of PDEs};
\draw (7.449999999999999, -31.700000000000003) rectangle (11.299999999999999,-33.800000000000004);
\draw(11.4, -31.700000000000003) node[anchor=north west,align=left] {Fundamental\\ solutions\\ to PDEs};
\draw (11.4, -31.700000000000003) rectangle (14.75,-33.300000000000004);
\draw(14.85, -31.700000000000003) node[anchor=north west,align=left] {Variational\\ methods\\ applied\\ to PDEs};
\draw (14.85, -31.700000000000003) rectangle (18.2,-33.800000000000004);
\draw(3, -33.900000000000006) node[anchor=north west,align=left] {Analyticity\\ in context\\ of PDEs};
\draw (3, -33.900000000000006) rectangle (6.35,-35.50000000000001);
\draw(6.449999999999999, -33.900000000000006) node[anchor=north west,align=left] {Singularity\\ in context\\ of PDEs};
\draw (6.449999999999999, -33.900000000000006) rectangle (9.799999999999999,-35.50000000000001);
\draw(9.899999999999999, -33.900000000000006) node[anchor=north west,align=left] {Wave front\\ sets\\ in context\\ of PDEs};
\draw (9.899999999999999, -33.900000000000006) rectangle (12.999999999999998,-36.00000000000001);
\draw(13.1, -33.900000000000006) node[anchor=north west,align=left] {Classical\\ solutions\\ to PDEs};
\draw (13.1, -33.900000000000006) rectangle (15.95,-35.50000000000001);
\draw(20.450000000000003, -21.6) node[anchor=north west,align=left] {\large Generalized solutions to partial differential equations};
\draw (20.450000000000003, -21.6) rectangle (38.10000000000001,-24.3);
\draw(21.450000000000003, -22.6) node[anchor=north west,align=left] {Weak \\ solutions\\ to PDEs};
\draw (21.450000000000003, -22.6) rectangle (24.300000000000004,-24.200000000000003);
\draw(24.400000000000002, -22.6) node[anchor=north west,align=left] {Strong \\ solutions\\ to PDEs};
\draw (24.400000000000002, -22.6) rectangle (27.250000000000004,-24.200000000000003);
\draw(27.35, -22.6) node[anchor=north west,align=left] {Viscosity\\ solutions\\ to PDEs};
\draw (27.35, -22.6) rectangle (30.200000000000003,-24.200000000000003);
\draw(20.450000000000003, -24.400000000000002) node[anchor=north west,align=left] {\large Hyperbolic equations and hyperbolic systems};
\draw (20.450000000000003, -24.400000000000002) rectangle (36.55,-32.0);
\draw(21.450000000000003, -25.400000000000002) node[anchor=north west,align=left] {Initial value\\ problems\\ for first-order\\ hyperbolic equations};
\draw (21.450000000000003, -25.400000000000002) rectangle (27.050000000000004,-28.000000000000004);
\draw(27.150000000000002, -25.400000000000002) node[anchor=north west,align=left] {Initial-boundary\\ value \\ problems for \\ first-order \\ hyperbolic equations};
\draw (27.150000000000002, -25.400000000000002) rectangle (32.75,-28.000000000000004);
\draw(32.85, -25.400000000000002) node[anchor=north west,align=left] {Second-order\\ hyperbolic\\ equations};
\draw (32.85, -25.400000000000002) rectangle (36.45,-27.500000000000004);
\draw(21.450000000000003, -28.1) node[anchor=north west,align=left] {Initial-boundary\\ value \\ problems for \\ second-order \\ hyperbolic equations};
\draw (21.450000000000003, -28.1) rectangle (27.050000000000004,-30.700000000000003);
\draw(27.150000000000002, -28.1) node[anchor=north west,align=left] {Initial value\\ problems \\ for second-order\\ hyperbolic\\ equations};
\draw (27.150000000000002, -28.1) rectangle (31.75,-30.700000000000003);
\draw(31.85, -28.1) node[anchor=north west,align=left] {First-order\\ hyperbolic\\ equations};
\draw (31.85, -28.1) rectangle (35.2,-29.700000000000003);
\draw(21.450000000000003, -30.800000000000004) node[anchor=north west,align=left] {Wave \\ equation};
\draw (21.450000000000003, -30.800000000000004) rectangle (24.050000000000004,-31.900000000000006);
\draw(38.20000000000001, -21.6) node[anchor=north west,align=left] {\large Parabolic equations and parabolic systems};
\draw (38.20000000000001, -21.6) rectangle (53.80000000000001,-49.6);
\draw(39.20000000000001, -22.6) node[anchor=north west,align=left] {Unilateral problems\\ for nonlinear\\ parabolic \\ equations and variational\\ inequalities\\ with nonlinear\\ parabolic operators};
\draw (39.20000000000001, -22.6) rectangle (46.05000000000001,-26.200000000000003);
\draw(46.150000000000006, -22.6) node[anchor=north west,align=left] {Unilateral problems\\ for linear parabolic\\ equations and\\ variational \\ inequalities with linear\\ parabolic operators};
\draw (46.150000000000006, -22.6) rectangle (52.75000000000001,-25.700000000000003);
\draw(39.20000000000001, -26.3) node[anchor=north west,align=left] {Initial value\\ problems\\ for \\ second-order \\ parabolic equations};
\draw (39.20000000000001, -26.3) rectangle (44.55000000000001,-28.900000000000002);
\draw(44.650000000000006, -26.3) node[anchor=north west,align=left] {Initial-boundary\\ value \\ problems for \\ second-order \\ parabolic equations};
\draw (44.650000000000006, -26.3) rectangle (50.00000000000001,-28.900000000000002);
\draw(50.10000000000001, -26.3) node[anchor=north west,align=left] {Second-order\\ parabolic\\ equations};
\draw (50.10000000000001, -26.3) rectangle (53.70000000000001,-27.900000000000002);
\draw(39.20000000000001, -29.0) node[anchor=north west,align=left] {Initial value\\ problems\\ for \\ higher-order \\ parabolic equations};
\draw (39.20000000000001, -29.0) rectangle (44.55000000000001,-31.6);
\draw(44.650000000000006, -29.0) node[anchor=north west,align=left] {Initial-boundary\\ value \\ problems for \\ higher-order \\ parabolic equations};
\draw (44.650000000000006, -29.0) rectangle (50.00000000000001,-31.6);
\draw(50.10000000000001, -29.0) node[anchor=north west,align=left] {Higher-order\\ parabolic\\ equations};
\draw (50.10000000000001, -29.0) rectangle (53.70000000000001,-30.6);
\draw(39.20000000000001, -31.700000000000003) node[anchor=north west,align=left] {Nonlinear initial,\\ boundary and\\ initial-boundary\\ value problems\\ for linear\\ parabolic equations};
\draw (39.20000000000001, -31.700000000000003) rectangle (44.55000000000001,-34.800000000000004);
\draw(44.650000000000006, -31.700000000000003) node[anchor=north west,align=left] {Nonlinear initial,\\ boundary and\\ initial-boundary\\ value problems\\ for nonlinear \\ parabolic equations};
\draw (44.650000000000006, -31.700000000000003) rectangle (50.00000000000001,-34.800000000000004);
\draw(50.10000000000001, -31.700000000000003) node[anchor=north west,align=left] {Second-order\\ parabolic\\ systems};
\draw (50.10000000000001, -31.700000000000003) rectangle (53.70000000000001,-33.300000000000004);
\draw(39.20000000000001, -34.900000000000006) node[anchor=north west,align=left] {Unilateral problems\\ for parabolic\\ systems and systems\\ of variational\\ inequalities with\\ parabolic operators};
\draw (39.20000000000001, -34.900000000000006) rectangle (44.55000000000001,-38.00000000000001);
\draw(44.650000000000006, -34.900000000000006) node[anchor=north west,align=left] {Semilinear \\ parabolic equations\\ with Laplacian,\\ bi-Laplacian\\ or poly-Laplacian};
\draw (44.650000000000006, -34.900000000000006) rectangle (50.00000000000001,-37.50000000000001);
\draw(50.10000000000001, -34.900000000000006) node[anchor=north west,align=left] {Higher-order\\ parabolic\\ systems};
\draw (50.10000000000001, -34.900000000000006) rectangle (53.70000000000001,-36.50000000000001);
\draw(39.20000000000001, -38.1) node[anchor=north west,align=left] {Reaction-diffusion\\ equations};
\draw (39.20000000000001, -38.1) rectangle (44.30000000000001,-39.7);
\draw(44.400000000000006, -38.1) node[anchor=north west,align=left] {Initial value\\ problems\\ for \\ second-order \\ parabolic systems};
\draw (44.400000000000006, -38.1) rectangle (49.25000000000001,-40.7);
\draw(49.35000000000001, -38.1) node[anchor=north west,align=left] {Quasilinear \\ parabolic \\ equations with \\ \(p\)-Laplacian};
\draw (49.35000000000001, -38.1) rectangle (53.70000000000001,-40.2);
\draw(39.20000000000001, -40.8) node[anchor=north west,align=left] {Initial value\\ problems\\ for \\ higher-order \\ parabolic systems};
\draw (39.20000000000001, -40.8) rectangle (44.05000000000001,-43.4);
\draw(44.150000000000006, -40.8) node[anchor=north west,align=left] {Initial-boundary\\ value \\ problems for \\ second-order \\ parabolic systems};
\draw (44.150000000000006, -40.8) rectangle (49.00000000000001,-43.4);
\draw(49.10000000000001, -40.8) node[anchor=north west,align=left] {Ultraparabolic\\ equations,\\ pseudoparabolic \\ equations, etc.};
\draw (49.10000000000001, -40.8) rectangle (53.70000000000001,-43.4);
\draw(39.20000000000001, -43.5) node[anchor=north west,align=left] {Initial-boundary\\ value \\ problems for \\ higher-order \\ parabolic systems};
\draw (39.20000000000001, -43.5) rectangle (44.05000000000001,-46.1);
\draw(44.150000000000006, -43.5) node[anchor=north west,align=left] {Quasilinear\\ parabolic \\ equations with\\ mean curvature\\ operator};
\draw (44.150000000000006, -43.5) rectangle (48.25000000000001,-46.1);
\draw(48.35000000000001, -43.5) node[anchor=north west,align=left] {Parabolic\\ Monge-Ampère\\ equations};
\draw (48.35000000000001, -43.5) rectangle (51.95000000000001,-45.1);
\draw(39.20000000000001, -46.2) node[anchor=north west,align=left] {Quasilinear\\ parabolic\\ equations};
\draw (39.20000000000001, -46.2) rectangle (42.55000000000001,-47.800000000000004);
\draw(42.650000000000006, -46.2) node[anchor=north west,align=left] {Semilinear\\ parabolic\\ equations};
\draw (42.650000000000006, -46.2) rectangle (45.75000000000001,-47.800000000000004);
\draw(45.85000000000001, -46.2) node[anchor=north west,align=left] {Degenerate\\ parabolic\\ equations};
\draw (45.85000000000001, -46.2) rectangle (48.95000000000001,-47.800000000000004);
\draw(49.05000000000001, -46.2) node[anchor=north west,align=left] {Nonlinear\\ parabolic\\ equations};
\draw (49.05000000000001, -46.2) rectangle (51.90000000000001,-47.800000000000004);
\draw(39.20000000000001, -47.9) node[anchor=north west,align=left] {Singular\\ parabolic\\ equations};
\draw (39.20000000000001, -47.9) rectangle (42.05000000000001,-49.5);
\draw(42.15000000000001, -47.9) node[anchor=north west,align=left] {Abstract\\ parabolic\\ equations};
\draw (42.15000000000001, -47.9) rectangle (45.000000000000014,-49.5);
\draw(45.10000000000001, -47.9) node[anchor=north west,align=left] {Heat \\ equation};
\draw (45.10000000000001, -47.9) rectangle (47.70000000000001,-49.0);
\draw(47.80000000000001, -47.9) node[anchor=north west,align=left] {Heat\\ kernel};
\draw (47.80000000000001, -47.9) rectangle (49.90000000000001,-49.0);
\draw(53.900000000000006, -21.6) node[anchor=north west,align=left] {\large Elliptic equations and elliptic systems};
\draw (53.900000000000006, -21.6) rectangle (68.75,-52.3);
\draw(54.900000000000006, -22.6) node[anchor=north west,align=left] {Unilateral problems\\ for linear elliptic\\ equations and\\ variational \\ inequalities with linear\\ elliptic operators};
\draw (54.900000000000006, -22.6) rectangle (61.50000000000001,-25.700000000000003);
\draw(61.60000000000001, -22.6) node[anchor=north west,align=left] {Unilateral problems\\ for nonlinear\\ elliptic equations\\ and variational\\ inequalities\\ with nonlinear\\ elliptic operators};
\draw (61.60000000000001, -22.6) rectangle (66.95,-26.200000000000003);
\draw(54.900000000000006, -26.3) node[anchor=north west,align=left] {Unilateral problems\\ for elliptic \\ systems and systems\\ of variational\\ inequalities with\\ elliptic operators};
\draw (54.900000000000006, -26.3) rectangle (60.25000000000001,-29.400000000000002);
\draw(60.35000000000001, -26.3) node[anchor=north west,align=left] {Boundary \\ value problems\\ for \\ second-order \\ elliptic equations};
\draw (60.35000000000001, -26.3) rectangle (65.45,-28.900000000000002);
\draw(65.55000000000001, -26.3) node[anchor=north west,align=left] {Semilinear\\ elliptic\\ equations};
\draw (65.55000000000001, -26.3) rectangle (68.65,-27.900000000000002);
\draw(54.900000000000006, -29.5) node[anchor=north west,align=left] {Boundary \\ value problems\\ for \\ higher-order \\ elliptic equations};
\draw (54.900000000000006, -29.5) rectangle (60.00000000000001,-32.1);
\draw(60.10000000000001, -29.5) node[anchor=north west,align=left] {Nonlinear \\ boundary value\\ problems for\\ linear \\ elliptic equations};
\draw (60.10000000000001, -29.5) rectangle (65.2,-32.1);
\draw(65.30000000000001, -29.5) node[anchor=north west,align=left] {Schrödinger\\ operator,\\ Schrödinger\\ equation};
\draw (65.30000000000001, -29.5) rectangle (68.65,-31.6);
\draw(54.900000000000006, -32.2) node[anchor=north west,align=left] {Nonlinear \\ boundary value \\ problems for \\ nonlinear \\ elliptic equations};
\draw (54.900000000000006, -32.2) rectangle (60.00000000000001,-34.800000000000004);
\draw(60.10000000000001, -32.2) node[anchor=north west,align=left] {Semilinear \\ elliptic equations\\ with Laplacian,\\ bi-Laplacian \\ or poly-Laplacian};
\draw (60.10000000000001, -32.2) rectangle (65.2,-34.800000000000004);
\draw(65.30000000000001, -32.2) node[anchor=north west,align=left] {First-order\\ elliptic\\ systems};
\draw (65.30000000000001, -32.2) rectangle (68.65,-33.800000000000004);
\draw(54.900000000000006, -34.900000000000006) node[anchor=north west,align=left] {Quasilinear\\ elliptic \\ equations \\ with mean \\ curvature operator};
\draw (54.900000000000006, -34.900000000000006) rectangle (60.00000000000001,-37.50000000000001);
\draw(60.10000000000001, -34.900000000000006) node[anchor=north west,align=left] {Elliptic \\ equations \\ with \\ infinity-Laplacian};
\draw (60.10000000000001, -34.900000000000006) rectangle (65.2,-37.00000000000001);
\draw(65.30000000000001, -34.900000000000006) node[anchor=north west,align=left] {Quasilinear\\ elliptic\\ equations};
\draw (65.30000000000001, -34.900000000000006) rectangle (68.65,-36.50000000000001);
\draw(54.900000000000006, -37.6) node[anchor=north west,align=left] {Laplace operator,\\ Helmholtz \\ equation (reduced\\ wave equation),\\ Poisson equation};
\draw (54.900000000000006, -37.6) rectangle (59.75000000000001,-40.2);
\draw(59.85000000000001, -37.6) node[anchor=north west,align=left] {Boundary \\ value problems\\ for \\ first-order \\ elliptic systems};
\draw (59.85000000000001, -37.6) rectangle (64.45,-40.2);
\draw(64.55000000000001, -37.6) node[anchor=north west,align=left] {Green’s \\ functions for\\ elliptic\\ equations};
\draw (64.55000000000001, -37.6) rectangle (68.4,-39.7);
\draw(54.900000000000006, -40.3) node[anchor=north west,align=left] {Boundary \\ value problems\\ for \\ second-order \\ elliptic systems};
\draw (54.900000000000006, -40.3) rectangle (59.50000000000001,-42.9);
\draw(59.60000000000001, -40.3) node[anchor=north west,align=left] {Boundary \\ value problems\\ for \\ higher-order \\ elliptic systems};
\draw (59.60000000000001, -40.3) rectangle (64.2,-42.9);
\draw(64.30000000000001, -40.3) node[anchor=north west,align=left] {Quasilinear\\ elliptic \\ equations with\\ \(p\)-Laplacian};
\draw (64.30000000000001, -40.3) rectangle (68.65,-42.4);
\draw(54.900000000000006, -43.0) node[anchor=north west,align=left] {Boundary values\\ of solutions\\ to elliptic \\ equations and \\ elliptic systems};
\draw (54.900000000000006, -43.0) rectangle (59.50000000000001,-45.6);
\draw(59.60000000000001, -43.0) node[anchor=north west,align=left] {Second-order\\ elliptic\\ equations};
\draw (59.60000000000001, -43.0) rectangle (63.20000000000001,-44.6);
\draw(63.300000000000004, -43.0) node[anchor=north west,align=left] {Variational\\ methods for\\ second-order\\ elliptic\\ equations};
\draw (63.300000000000004, -43.0) rectangle (66.9,-45.6);
\draw(54.900000000000006, -45.7) node[anchor=north west,align=left] {Higher-order\\ elliptic\\ equations};
\draw (54.900000000000006, -45.7) rectangle (58.50000000000001,-47.300000000000004);
\draw(58.60000000000001, -45.7) node[anchor=north west,align=left] {Variational\\ methods for\\ higher-order\\ elliptic\\ equations};
\draw (58.60000000000001, -45.7) rectangle (62.20000000000001,-48.300000000000004);
\draw(62.300000000000004, -45.7) node[anchor=north west,align=left] {Second-order\\ elliptic\\ systems};
\draw (62.300000000000004, -45.7) rectangle (65.9,-47.300000000000004);
\draw(54.900000000000006, -48.4) node[anchor=north west,align=left] {Higher-order\\ elliptic\\ systems};
\draw (54.900000000000006, -48.4) rectangle (58.50000000000001,-50.0);
\draw(58.60000000000001, -48.4) node[anchor=north west,align=left] {Variational\\ methods\\ for elliptic\\ systems};
\draw (58.60000000000001, -48.4) rectangle (62.20000000000001,-50.5);
\draw(62.300000000000004, -48.4) node[anchor=north west,align=left] {Monge-Ampère\\ equations};
\draw (62.300000000000004, -48.4) rectangle (65.9,-50.0);
\draw(54.900000000000006, -50.599999999999994) node[anchor=north west,align=left] {Degenerate\\ elliptic\\ equations};
\draw (54.900000000000006, -50.599999999999994) rectangle (58.00000000000001,-52.199999999999996);
\draw(58.10000000000001, -50.599999999999994) node[anchor=north west,align=left] {Nonlinear\\ elliptic\\ equations};
\draw (58.10000000000001, -50.599999999999994) rectangle (60.95000000000001,-52.199999999999996);
\draw(61.050000000000004, -50.599999999999994) node[anchor=north west,align=left] {Singular\\ elliptic\\ equations};
\draw (61.050000000000004, -50.599999999999994) rectangle (63.900000000000006,-52.199999999999996);
\draw(76.96999999999998, -1) node[anchor=north west,align=left] {\LARGE Ordinary differential equations};
\draw (76.96999999999998, -1) rectangle (137.07,-77.5);
\draw(77.96999999999998, -2) node[anchor=north west,align=left] {\large Functional-differential equations (including equations with delayed, advanced or state-dependent argument)};
\draw (77.96999999999998, -2) rectangle (117.66999999999999,-23.400000000000002);
\draw(78.96999999999998, -3) node[anchor=north west,align=left] {Functional-differentialinclusions};
\draw (78.96999999999998, -3) rectangle (87.81999999999998,-5.1);
\draw(87.91999999999999, -3) node[anchor=north west,align=left] {Functional-differentialequations\\ with fractional\\ derivatives};
\draw (87.91999999999999, -3) rectangle (96.51999999999998,-5.6);
\draw(96.61999999999998, -3) node[anchor=north west,align=left] {Functional-differentialequations\\ with state-dependent\\ arguments};
\draw (96.61999999999998, -3) rectangle (105.21999999999997,-5.6);
\draw(105.31999999999998, -3) node[anchor=north west,align=left] {Transformation\\ and reduction \\ of functional-differential\\ equations and systems,\\ normal forms};
\draw (105.31999999999998, -3) rectangle (112.41999999999997,-6.1);
\draw(78.96999999999998, -6.2) node[anchor=north west,align=left] {Symmetries,\\ invariants\\ of \\ functional-differential\\ equations};
\draw (78.96999999999998, -6.2) rectangle (85.31999999999998,-8.8);
\draw(85.41999999999999, -6.2) node[anchor=north west,align=left] {General theory\\ of \\ functional-differential\\ equations};
\draw (85.41999999999999, -6.2) rectangle (91.76999999999998,-8.3);
\draw(91.86999999999998, -6.2) node[anchor=north west,align=left] {Linear \\ functional-differential\\ equations};
\draw (91.86999999999998, -6.2) rectangle (98.21999999999997,-8.3);
\draw(98.32, -6.2) node[anchor=north west,align=left] {Theoretical \\ approximation of \\ solutions to \\ functional-differential\\ equations};
\draw (98.32, -6.2) rectangle (104.66999999999999,-8.8);
\draw(104.76999999999998, -6.2) node[anchor=north west,align=left] {Spectral theory\\ of \\ functional-differential\\ operators};
\draw (104.76999999999998, -6.2) rectangle (111.11999999999998,-8.3);
\draw(111.21999999999998, -6.2) node[anchor=north west,align=left] {Boundary value\\ problems\\ for \\ functional-differential\\ equations};
\draw (111.21999999999998, -6.2) rectangle (117.56999999999998,-8.8);
\draw(78.96999999999998, -8.9) node[anchor=north west,align=left] {Oscillation\\ theory of\\ functional-differential\\ equations};
\draw (78.96999999999998, -8.9) rectangle (85.31999999999998,-11.5);
\draw(85.41999999999999, -8.9) node[anchor=north west,align=left] {Growth, \\ boundedness, \\ comparison of \\ solutions to \\ functional-differential\\ equations};
\draw (85.41999999999999, -8.9) rectangle (91.76999999999998,-12.0);
\draw(91.86999999999998, -8.9) node[anchor=north west,align=left] {Periodic \\ solutions to\\ functional-differential\\ equations};
\draw (91.86999999999998, -8.9) rectangle (98.21999999999997,-11.5);
\draw(98.32, -8.9) node[anchor=north west,align=left] {Almost and \\ pseudo-almost periodic\\ solutions to \\ functional-differential\\ equations};
\draw (98.32, -8.9) rectangle (104.66999999999999,-11.5);
\draw(104.76999999999998, -8.9) node[anchor=north west,align=left] {Heteroclinic \\ and homoclinic\\ orbits of \\ functional-differential\\ equations};
\draw (104.76999999999998, -8.9) rectangle (111.11999999999998,-11.5);
\draw(111.21999999999998, -8.9) node[anchor=north west,align=left] {Bifurcation\\ theory of\\ functional-differential\\ equations};
\draw (111.21999999999998, -8.9) rectangle (117.56999999999998,-11.5);
\draw(78.96999999999998, -12.100000000000001) node[anchor=north west,align=left] {Invariant \\ manifolds of\\ functional-differential\\ equations};
\draw (78.96999999999998, -12.100000000000001) rectangle (85.31999999999998,-14.700000000000001);
\draw(85.41999999999999, -12.100000000000001) node[anchor=north west,align=left] {Stability \\ theory of \\ functional-differential\\ equations};
\draw (85.41999999999999, -12.100000000000001) rectangle (91.76999999999998,-14.200000000000001);
\draw(91.86999999999998, -12.100000000000001) node[anchor=north west,align=left] {Stationary\\ solutions \\ of \\ functional-differential\\ equations};
\draw (91.86999999999998, -12.100000000000001) rectangle (98.21999999999997,-14.700000000000001);
\draw(98.32, -12.100000000000001) node[anchor=north west,align=left] {Complex (chaotic)\\ behavior of\\ solutions to \\ functional-differential\\ equations};
\draw (98.32, -12.100000000000001) rectangle (104.66999999999999,-14.700000000000001);
\draw(104.76999999999998, -12.100000000000001) node[anchor=north west,align=left] {Synchronization\\ of \\ functional-differential\\ equations};
\draw (104.76999999999998, -12.100000000000001) rectangle (111.11999999999998,-14.200000000000001);
\draw(111.21999999999998, -12.100000000000001) node[anchor=north west,align=left] {Asymptotic\\ theory of\\ functional-differential\\ equations};
\draw (111.21999999999998, -12.100000000000001) rectangle (117.56999999999998,-14.700000000000001);
\draw(78.96999999999998, -14.8) node[anchor=north west,align=left] {Singular \\ perturbations\\ of \\ functional-differential\\ equations};
\draw (78.96999999999998, -14.8) rectangle (85.31999999999998,-17.400000000000002);
\draw(85.41999999999999, -14.8) node[anchor=north west,align=left] {Perturbations\\ of \\ functional-differential\\ equations};
\draw (85.41999999999999, -14.8) rectangle (91.76999999999998,-16.900000000000002);
\draw(91.86999999999998, -14.8) node[anchor=north west,align=left] {Inverse \\ problems for\\ functional-differential\\ equations};
\draw (91.86999999999998, -14.8) rectangle (98.21999999999997,-17.400000000000002);
\draw(98.32, -14.8) node[anchor=north west,align=left] {Functional-differential\\ equations in \\ abstract spaces};
\draw (98.32, -14.8) rectangle (104.66999999999999,-16.900000000000002);
\draw(104.76999999999998, -14.8) node[anchor=north west,align=left] {Lattice \\ functional-differential\\ equations};
\draw (104.76999999999998, -14.8) rectangle (111.11999999999998,-16.900000000000002);
\draw(111.21999999999998, -14.8) node[anchor=north west,align=left] {Implicit \\ functional-differential\\ equations};
\draw (111.21999999999998, -14.8) rectangle (117.56999999999998,-16.900000000000002);
\draw(78.96999999999998, -17.5) node[anchor=north west,align=left] {Averaging \\ for \\ functional-differential\\ equations};
\draw (78.96999999999998, -17.5) rectangle (85.31999999999998,-19.6);
\draw(85.41999999999999, -17.5) node[anchor=north west,align=left] {Hybrid systems\\ of \\ functional-differential\\ equations};
\draw (85.41999999999999, -17.5) rectangle (91.76999999999998,-19.6);
\draw(91.86999999999998, -17.5) node[anchor=north west,align=left] {Control \\ problems for\\ functional-differential\\ equations};
\draw (91.86999999999998, -17.5) rectangle (98.21999999999997,-20.1);
\draw(98.32, -17.5) node[anchor=north west,align=left] {Fuzzy \\ functional-differential\\ equations};
\draw (98.32, -17.5) rectangle (104.66999999999999,-19.6);
\draw(104.76999999999998, -17.5) node[anchor=north west,align=left] {Functional-differential\\ inequalities};
\draw (104.76999999999998, -17.5) rectangle (111.11999999999998,-19.6);
\draw(111.21999999999998, -17.5) node[anchor=north west,align=left] {Discontinuous\\ functional-differential\\ equations};
\draw (111.21999999999998, -17.5) rectangle (117.56999999999998,-19.6);
\draw(78.96999999999998, -20.2) node[anchor=north west,align=left] {Neutral \\ functional-differential\\ equations};
\draw (78.96999999999998, -20.2) rectangle (85.31999999999998,-22.3);
\draw(85.41999999999999, -20.2) node[anchor=north west,align=left] {Functional-differential\\ equations\\ in the \\ complex domain};
\draw (85.41999999999999, -20.2) rectangle (91.76999999999998,-22.8);
\draw(91.86999999999998, -20.2) node[anchor=north west,align=left] {Functional-differential\\ equations on \\ time scales or\\ measure chains};
\draw (91.86999999999998, -20.2) rectangle (98.21999999999997,-22.8);
\draw(98.32, -20.2) node[anchor=north west,align=left] {Functional-differential\\ equations\\ with impulses};
\draw (98.32, -20.2) rectangle (104.66999999999999,-22.3);
\draw(104.76999999999998, -20.2) node[anchor=north west,align=left] {Stochastic\\ functional-differential\\ equations};
\draw (104.76999999999998, -20.2) rectangle (111.11999999999998,-22.3);
\draw(111.21999999999998, -20.2) node[anchor=north west,align=left] {Qualitative \\ investigation and\\ simulation of \\ models involving\\ functional-differential\\ equations};
\draw (111.21999999999998, -20.2) rectangle (117.56999999999998,-23.3);
\draw(117.76999999999998, -2) node[anchor=north west,align=left] {\large General theory for ordinary differential equations};
\draw (117.76999999999998, -2) rectangle (136.86999999999998,-20.700000000000003);
\draw(118.76999999999998, -3) node[anchor=north west,align=left] {Analytical theory\\ of ordinary \\ differential \\ equations: series, \\ transformations, \\ transforms, \\ operational calculus, etc.};
\draw (118.76999999999998, -3) rectangle (125.86999999999998,-6.6);
\draw(125.96999999999998, -3) node[anchor=north west,align=left] {Initial value problems,\\ existence, \\ uniqueness, continuous\\ dependence and \\ continuation of solutions\\ to ordinary \\ differential equations};
\draw (125.96999999999998, -3) rectangle (132.82,-6.6);
\draw(132.92, -3) node[anchor=north west,align=left] {Discontinuous\\ ordinary\\ differential\\ equations};
\draw (132.92, -3) rectangle (136.76999999999998,-5.1);
\draw(118.76999999999998, -6.7) node[anchor=north west,align=left] {Generalized \\ ordinary differential\\ equations \\ (measure-differential\\ equations, \\ set-valued differential\\ equations, etc.)};
\draw (118.76999999999998, -6.7) rectangle (125.11999999999998,-10.3);
\draw(125.21999999999998, -6.7) node[anchor=north west,align=left] {Fractional \\ ordinary \\ differential equations\\ and \\ fractional differential\\ inclusions};
\draw (125.21999999999998, -6.7) rectangle (131.57,-9.8);
\draw(131.67, -6.7) node[anchor=north west,align=left] {Inverse \\ problems involving\\ ordinary\\ differential\\ equations};
\draw (131.67, -6.7) rectangle (136.76999999999998,-9.3);
\draw(118.76999999999998, -10.4) node[anchor=north west,align=left] {Implicit ordinary\\ differential\\ equations,\\ differential-algebraic\\ equations};
\draw (118.76999999999998, -10.4) rectangle (124.86999999999998,-13.0);
\draw(124.96999999999998, -10.4) node[anchor=north west,align=left] {Theoretical \\ approximation of\\ solutions to\\ ordinary \\ differential equations};
\draw (124.96999999999998, -10.4) rectangle (131.07,-13.0);
\draw(131.17, -10.4) node[anchor=north west,align=left] {Geometric \\ methods in ordinary\\ differential\\ equations};
\draw (131.17, -10.4) rectangle (136.51999999999998,-12.5);
\draw(118.76999999999998, -13.100000000000001) node[anchor=north west,align=left] {Explicit \\ solutions, first \\ integrals of \\ ordinary differential\\ equations};
\draw (118.76999999999998, -13.100000000000001) rectangle (124.61999999999998,-15.700000000000001);
\draw(124.71999999999998, -13.100000000000001) node[anchor=north west,align=left] {Nonlinear \\ ordinary differential\\ equations\\ and systems,\\ general theory};
\draw (124.71999999999998, -13.100000000000001) rectangle (130.57,-15.700000000000001);
\draw(130.67, -13.100000000000001) node[anchor=north west,align=left] {Differential \\ inequalities \\ involving functions\\ of a single\\ real variable};
\draw (130.67, -13.100000000000001) rectangle (136.01999999999998,-15.700000000000001);
\draw(118.76999999999998, -15.8) node[anchor=north west,align=left] {Linear \\ ordinary \\ differential \\ equations and \\ systems, general};
\draw (118.76999999999998, -15.8) rectangle (123.36999999999998,-18.400000000000002);
\draw(123.46999999999998, -15.8) node[anchor=north west,align=left] {Ordinary \\ differential \\ equations of \\ infinite order};
\draw (123.46999999999998, -15.8) rectangle (127.56999999999998,-17.900000000000002);
\draw(127.66999999999999, -15.8) node[anchor=north west,align=left] {Hybrid systems\\ of ordinary\\ differential\\ equations};
\draw (127.66999999999999, -15.8) rectangle (131.76999999999998,-17.900000000000002);
\draw(131.86999999999998, -15.8) node[anchor=north west,align=left] {Ordinary \\ differential\\ equations \\ with impulses};
\draw (131.86999999999998, -15.8) rectangle (135.71999999999997,-17.900000000000002);
\draw(118.76999999999998, -18.5) node[anchor=north west,align=left] {Fuzzy \\ ordinary \\ differential\\ equations};
\draw (118.76999999999998, -18.5) rectangle (122.36999999999998,-20.6);
\draw(122.46999999999998, -18.5) node[anchor=north west,align=left] {Ordinary\\ lattice \\ differential\\ equations};
\draw (122.46999999999998, -18.5) rectangle (126.06999999999998,-20.6);
\draw(126.16999999999999, -18.5) node[anchor=north west,align=left] {Ordinary \\ differential\\ inclusions};
\draw (126.16999999999999, -18.5) rectangle (129.76999999999998,-20.1);
\draw(77.96999999999998, -23.500000000000004) node[anchor=north west,align=left] {\large Ordinary differential equations in the complex domain};
\draw (77.96999999999998, -23.500000000000004) rectangle (98.07,-44.7);
\draw(78.96999999999998, -24.500000000000004) node[anchor=north west,align=left] {Singular perturbation\\ problems for \\ ordinary differential\\ equations in the \\ complex domain (complex\\ WKB, turning \\ points, steepest descent)};
\draw (78.96999999999998, -24.500000000000004) rectangle (85.81999999999998,-28.100000000000005);
\draw(85.91999999999999, -24.500000000000004) node[anchor=north west,align=left] {Algebraic aspects \\ (differential-algebraic,\\ hypertranscendence,\\ group-theoretical)\\ of ordinary \\ differential equations\\ in the complex domain};
\draw (85.91999999999999, -24.500000000000004) rectangle (92.51999999999998,-28.100000000000005);
\draw(92.61999999999998, -24.500000000000004) node[anchor=north west,align=left] {Oscillation, \\ growth of solutions\\ to ordinary\\ differential\\ equations in\\ the complex domain};
\draw (92.61999999999998, -24.500000000000004) rectangle (97.96999999999997,-27.600000000000005);
\draw(78.96999999999998, -28.200000000000003) node[anchor=north west,align=left] {Painlevé and other\\ special ordinary\\ differential equations\\ in the complex\\ domain; classification,\\ hierarchies};
\draw (78.96999999999998, -28.200000000000003) rectangle (85.31999999999998,-31.300000000000004);
\draw(85.41999999999999, -28.200000000000003) node[anchor=north west,align=left] {Singularities, \\ monodromy and local \\ behavior of solutions\\ to ordinary \\ differential equations\\ in the complex \\ domain, normal forms};
\draw (85.41999999999999, -28.200000000000003) rectangle (91.51999999999998,-31.800000000000004);
\draw(91.61999999999998, -28.200000000000003) node[anchor=north west,align=left] {Entire and \\ meromorphic solutions\\ to ordinary\\ differential\\ equations in\\ the complex domain};
\draw (91.61999999999998, -28.200000000000003) rectangle (97.46999999999997,-31.300000000000004);
\draw(78.96999999999998, -31.900000000000006) node[anchor=north west,align=left] {Formal solutions\\ and transform \\ techniques for \\ ordinary differential\\ equations in\\ the complex domain};
\draw (78.96999999999998, -31.900000000000006) rectangle (84.81999999999998,-35.00000000000001);
\draw(84.91999999999999, -31.900000000000006) node[anchor=north west,align=left] {Stokes phenomena\\ and connection \\ problems (linear and\\ nonlinear) for\\ ordinary differential\\ equations in\\ the complex domain};
\draw (84.91999999999999, -31.900000000000006) rectangle (90.76999999999998,-35.50000000000001);
\draw(90.86999999999998, -31.900000000000006) node[anchor=north west,align=left] {Inverse problems \\ (Riemann-Hilbert, \\ inverse differential\\ Galois, etc.) for\\ ordinary differential\\ equations in\\ the complex domain};
\draw (90.86999999999998, -31.900000000000006) rectangle (96.71999999999997,-35.50000000000001);
\draw(78.96999999999998, -35.60000000000001) node[anchor=north west,align=left] {Topological \\ structure of \\ trajectories of \\ ordinary differential\\ equations in\\ the complex domain};
\draw (78.96999999999998, -35.60000000000001) rectangle (84.81999999999998,-38.70000000000001);
\draw(84.91999999999999, -35.60000000000001) node[anchor=north west,align=left] {Ordinary \\ differential \\ equations on complex\\ manifolds};
\draw (84.91999999999999, -35.60000000000001) rectangle (90.51999999999998,-37.70000000000001);
\draw(90.61999999999998, -35.60000000000001) node[anchor=north west,align=left] {Nonlinear ordinary\\ differential\\ equations and\\ systems in the\\ complex domain};
\draw (90.61999999999998, -35.60000000000001) rectangle (95.71999999999997,-38.20000000000001);
\draw(78.96999999999998, -38.800000000000004) node[anchor=north west,align=left] {Spectral theory\\ for ordinary\\ differential\\ operators in \\ the complex domain};
\draw (78.96999999999998, -38.800000000000004) rectangle (84.06999999999998,-41.400000000000006);
\draw(84.16999999999999, -38.800000000000004) node[anchor=north west,align=left] {Asymptotics and\\ summation methods\\ for ordinary\\ differential\\ equations in the\\ complex domain};
\draw (84.16999999999999, -38.800000000000004) rectangle (89.01999999999998,-41.900000000000006);
\draw(89.11999999999998, -38.800000000000004) node[anchor=north west,align=left] {Isomonodromic\\ deformations \\ for ordinary \\ differential \\ equations in the\\ complex domain};
\draw (89.11999999999998, -38.800000000000004) rectangle (93.71999999999997,-41.900000000000006);
\draw(78.96999999999998, -42.0) node[anchor=north west,align=left] {Linear ordinary\\ differential\\ equations and\\ systems in the\\ complex domain};
\draw (78.96999999999998, -42.0) rectangle (83.31999999999998,-44.6);
\draw(98.16999999999999, -23.500000000000004) node[anchor=north west,align=left] {\large Control problems including ordinary differential equations};
\draw (98.16999999999999, -23.500000000000004) rectangle (118.22,-27.200000000000003);
\draw(99.16999999999999, -24.500000000000004) node[anchor=north west,align=left] {Chaos control\\ for problems\\ involving \\ ordinary \\ differential equations};
\draw (99.16999999999999, -24.500000000000004) rectangle (105.26999999999998,-27.100000000000005);
\draw(105.36999999999999, -24.500000000000004) node[anchor=north west,align=left] {Control \\ problems involving\\ ordinary\\ differential\\ equations};
\draw (105.36999999999999, -24.500000000000004) rectangle (110.46999999999998,-27.100000000000005);
\draw(110.57, -24.500000000000004) node[anchor=north west,align=left] {Stabilization\\ of solutions\\ to ordinary\\ differential\\ equations};
\draw (110.57, -24.500000000000004) rectangle (114.41999999999999,-27.100000000000005);
\draw(114.51999999999998, -24.500000000000004) node[anchor=north west,align=left] {Bifurcation\\ control\\ of ordinary\\ differential\\ equations};
\draw (114.51999999999998, -24.500000000000004) rectangle (118.11999999999998,-27.100000000000005);
\draw(98.16999999999999, -27.300000000000004) node[anchor=north west,align=left] {\large Dynamic equations on time scales or measure chains};
\draw (98.16999999999999, -27.300000000000004) rectangle (114.26999999999998,-30.500000000000004);
\draw(99.16999999999999, -28.300000000000004) node[anchor=north west,align=left] {Dynamic \\ equations on time\\ scales or\\ measure chains};
\draw (99.16999999999999, -28.300000000000004) rectangle (104.01999999999998,-30.400000000000006);
\draw(98.16999999999999, -30.600000000000005) node[anchor=north west,align=left] {\large Ordinary differential operators};
\draw (98.16999999999999, -30.600000000000005) rectangle (111.57,-44.400000000000006);
\draw(99.16999999999999, -31.600000000000005) node[anchor=north west,align=left] {Eigenfunctions, \\ eigenfunction \\ expansions, completeness\\ of eigenfunctions\\ of ordinary\\ differential operators};
\draw (99.16999999999999, -31.600000000000005) rectangle (105.76999999999998,-34.7);
\draw(105.86999999999999, -31.600000000000005) node[anchor=north west,align=left] {General \\ spectral theory\\ of ordinary\\ differential\\ operators};
\draw (105.86999999999999, -31.600000000000005) rectangle (110.21999999999998,-34.2);
\draw(99.16999999999999, -34.800000000000004) node[anchor=north west,align=left] {Asymptotic distribution\\ of eigenvalues,\\ asymptotic \\ theory of eigenfunctions\\ for ordinary \\ differential operators};
\draw (99.16999999999999, -34.800000000000004) rectangle (105.76999999999998,-37.900000000000006);
\draw(105.86999999999999, -34.800000000000004) node[anchor=north west,align=left] {Nonlinear\\ ordinary \\ differential\\ operators};
\draw (105.86999999999999, -34.800000000000004) rectangle (109.46999999999998,-36.900000000000006);
\draw(99.16999999999999, -38.00000000000001) node[anchor=north west,align=left] {Numerical approximation\\ of eigenvalues\\ and of other \\ parts of the spectrum\\ of ordinary \\ differential operators};
\draw (99.16999999999999, -38.00000000000001) rectangle (105.51999999999998,-41.10000000000001);
\draw(105.61999999999999, -38.00000000000001) node[anchor=north west,align=left] {Particular \\ ordinary differential\\ operators\\ (Dirac, \\ one-dimensional \\ Schrödinger, etc.)};
\draw (105.61999999999999, -38.00000000000001) rectangle (111.46999999999998,-41.10000000000001);
\draw(99.16999999999999, -41.2) node[anchor=north west,align=left] {Eigenvalues, \\ estimation of \\ eigenvalues, upper and\\ lower bounds \\ of ordinary \\ differential operators};
\draw (99.16999999999999, -41.2) rectangle (105.26999999999998,-44.300000000000004);
\draw(105.36999999999999, -41.2) node[anchor=north west,align=left] {Scattering theory,\\ inverse \\ scattering involving\\ ordinary \\ differential operators};
\draw (105.36999999999999, -41.2) rectangle (111.46999999999998,-43.800000000000004);
\draw(118.32, -23.500000000000004) node[anchor=north west,align=left] {\large Differential equations in abstract spaces};
\draw (118.32, -23.500000000000004) rectangle (131.63,-26.700000000000003);
\draw(119.32, -24.500000000000004) node[anchor=north west,align=left] {Linear \\ differential \\ equations in \\ abstract spaces};
\draw (119.32, -24.500000000000004) rectangle (123.66999999999999,-26.600000000000005);
\draw(123.77, -24.500000000000004) node[anchor=north west,align=left] {Nonlinear \\ differential \\ equations in\\ abstract spaces};
\draw (123.77, -24.500000000000004) rectangle (128.12,-26.600000000000005);
\draw(128.22, -24.500000000000004) node[anchor=north west,align=left] {Evolution\\ inclusions};
\draw (128.22, -24.500000000000004) rectangle (131.32,-26.100000000000005);
\draw(77.96999999999998, -44.80000000000001) node[anchor=north west,align=left] {\large Boundary value problems for ordinary differential equations};
\draw (77.96999999999998, -44.80000000000001) rectangle (97.57,-63.500000000000014);
\draw(78.96999999999998, -45.80000000000001) node[anchor=north west,align=left] {Linear boundary\\ value problems \\ for ordinary \\ differential equations\\ with nonlinear\\ dependence on\\ the spectral parameter};
\draw (78.96999999999998, -45.80000000000001) rectangle (85.06999999999998,-49.40000000000001);
\draw(85.16999999999999, -45.80000000000001) node[anchor=north west,align=left] {Parameter dependent\\ boundary \\ value problems \\ for ordinary \\ differential equations};
\draw (85.16999999999999, -45.80000000000001) rectangle (91.26999999999998,-48.40000000000001);
\draw(91.36999999999998, -45.80000000000001) node[anchor=north west,align=left] {Boundary \\ eigenvalue \\ problems for \\ ordinary \\ differential equations};
\draw (91.36999999999998, -45.80000000000001) rectangle (97.46999999999997,-48.40000000000001);
\draw(78.96999999999998, -49.500000000000014) node[anchor=north west,align=left] {Nonlocal and\\ multipoint \\ boundary value\\ problems for\\ ordinary \\ differential equations};
\draw (78.96999999999998, -49.500000000000014) rectangle (85.06999999999998,-52.600000000000016);
\draw(85.16999999999999, -49.500000000000014) node[anchor=north west,align=left] {Nonlinear \\ boundary value \\ problems for \\ ordinary \\ differential equations};
\draw (85.16999999999999, -49.500000000000014) rectangle (91.26999999999998,-52.100000000000016);
\draw(91.36999999999998, -49.500000000000014) node[anchor=north west,align=left] {Singular nonlinear\\ boundary \\ value problems for\\ ordinary \\ differential equations};
\draw (91.36999999999998, -49.500000000000014) rectangle (97.46999999999997,-52.100000000000016);
\draw(78.96999999999998, -52.70000000000001) node[anchor=north west,align=left] {Positive solutions\\ to nonlinear\\ boundary value\\ problems for\\ ordinary \\ differential equations};
\draw (78.96999999999998, -52.70000000000001) rectangle (85.06999999999998,-55.80000000000001);
\draw(85.16999999999999, -52.70000000000001) node[anchor=north west,align=left] {Boundary value\\ problems with\\ impulses for\\ ordinary \\ differential equations};
\draw (85.16999999999999, -52.70000000000001) rectangle (91.26999999999998,-55.30000000000001);
\draw(91.36999999999998, -52.70000000000001) node[anchor=north west,align=left] {Boundary value\\ problems on\\ infinite \\ intervals for \\ ordinary \\ differential equations};
\draw (91.36999999999998, -52.70000000000001) rectangle (97.46999999999997,-55.80000000000001);
\draw(78.96999999999998, -55.90000000000001) node[anchor=north west,align=left] {Boundary value\\ problems on\\ graphs and \\ networks for \\ ordinary \\ differential equations};
\draw (78.96999999999998, -55.90000000000001) rectangle (85.06999999999998,-59.000000000000014);
\draw(85.16999999999999, -55.90000000000001) node[anchor=north west,align=left] {Applications\\ of boundary \\ value problems\\ involving \\ ordinary \\ differential equations};
\draw (85.16999999999999, -55.90000000000001) rectangle (91.26999999999998,-59.000000000000014);
\draw(91.36999999999998, -55.90000000000001) node[anchor=north west,align=left] {Linear boundary\\ value \\ problems for \\ ordinary differential\\ equations};
\draw (91.36999999999998, -55.90000000000001) rectangle (97.21999999999997,-58.500000000000014);
\draw(78.96999999999998, -59.10000000000001) node[anchor=north west,align=left] {Weyl theory and\\ its generalizations\\ for \\ ordinary differential\\ equations};
\draw (78.96999999999998, -59.10000000000001) rectangle (84.81999999999998,-61.70000000000001);
\draw(84.91999999999999, -59.10000000000001) node[anchor=north west,align=left] {Green’s functions\\ for \\ ordinary differential\\ equations};
\draw (84.91999999999999, -59.10000000000001) rectangle (90.76999999999998,-61.20000000000001);
\draw(90.86999999999998, -59.10000000000001) node[anchor=north west,align=left] {Special ordinary\\ differential\\ equations\\ (Mathieu, \\ Hill, Bessel, etc.)};
\draw (90.86999999999998, -59.10000000000001) rectangle (96.21999999999997,-61.70000000000001);
\draw(78.96999999999998, -61.80000000000001) node[anchor=north west,align=left] {Sturm-Liouville\\ theory};
\draw (78.96999999999998, -61.80000000000001) rectangle (83.31999999999998,-63.40000000000001);
\draw(97.66999999999999, -44.80000000000001) node[anchor=north west,align=left] {\large Qualitative theory for ordinary differential equations};
\draw (97.66999999999999, -44.80000000000001) rectangle (117.26999999999998,-67.70000000000002);
\draw(98.66999999999999, -45.80000000000001) node[anchor=north west,align=left] {Ordinary differential\\ equations and \\ connections with real \\ algebraic geometry \\ (fewnomials, desingularization,\\ zeros of \\ abelian integrals, etc.)};
\draw (98.66999999999999, -45.80000000000001) rectangle (107.01999999999998,-49.40000000000001);
\draw(107.11999999999999, -45.80000000000001) node[anchor=north west,align=left] {Theory of limit cycles\\ of polynomial and \\ analytic vector fields\\ (existence, uniqueness,\\ bounds, Hilbert’s\\ 16th problem and \\ ramifications) for ordinary\\ differential equations};
\draw (107.11999999999999, -45.80000000000001) rectangle (114.46999999999998,-49.90000000000001);
\draw(98.66999999999999, -50.000000000000014) node[anchor=north west,align=left] {Topological structure\\ of integral\\ curves, singular\\ points, limit \\ cycles of ordinary \\ differential equations};
\draw (98.66999999999999, -50.000000000000014) rectangle (104.76999999999998,-53.100000000000016);
\draw(104.86999999999999, -50.000000000000014) node[anchor=north west,align=left] {Oscillation theory,\\ zeros, \\ disconjugacy and \\ comparison theory for\\ ordinary \\ differential equations};
\draw (104.86999999999999, -50.000000000000014) rectangle (110.96999999999998,-53.100000000000016);
\draw(111.07, -50.000000000000014) node[anchor=north west,align=left] {Nonlinear \\ oscillations and\\ coupled \\ oscillators for \\ ordinary \\ differential equations};
\draw (111.07, -50.000000000000014) rectangle (117.16999999999999,-53.100000000000016);
\draw(98.66999999999999, -53.20000000000001) node[anchor=north west,align=left] {Almost and \\ pseudo-almost periodic\\ solutions \\ to ordinary \\ differential equations};
\draw (98.66999999999999, -53.20000000000001) rectangle (104.76999999999998,-55.80000000000001);
\draw(104.86999999999999, -53.20000000000001) node[anchor=north west,align=left] {Homoclinic and\\ heteroclinic\\ solutions to \\ ordinary \\ differential equations};
\draw (104.86999999999999, -53.20000000000001) rectangle (110.96999999999998,-55.80000000000001);
\draw(111.07, -53.20000000000001) node[anchor=north west,align=left] {Equivalence and\\ asymptotic \\ equivalence of\\ ordinary \\ differential equations};
\draw (111.07, -53.20000000000001) rectangle (117.16999999999999,-55.80000000000001);
\draw(98.66999999999999, -55.900000000000006) node[anchor=north west,align=left] {Growth and \\ boundedness of \\ solutions to \\ ordinary differential\\ equations};
\draw (98.66999999999999, -55.900000000000006) rectangle (104.51999999999998,-58.50000000000001);
\draw(104.61999999999999, -55.900000000000006) node[anchor=north west,align=left] {Transformation\\ and reduction\\ of ordinary \\ differential \\ equations and \\ systems, normal forms};
\draw (104.61999999999999, -55.900000000000006) rectangle (110.46999999999998,-59.00000000000001);
\draw(110.57, -55.900000000000006) node[anchor=north west,align=left] {Periodic \\ solutions to \\ ordinary differential\\ equations};
\draw (110.57, -55.900000000000006) rectangle (116.41999999999999,-58.00000000000001);
\draw(98.66999999999999, -59.10000000000001) node[anchor=north west,align=left] {Complex behavior\\ and chaotic\\ systems of \\ ordinary differential\\ equations};
\draw (98.66999999999999, -59.10000000000001) rectangle (104.51999999999998,-61.70000000000001);
\draw(104.61999999999999, -59.10000000000001) node[anchor=north west,align=left] {Averaging \\ method for ordinary\\ differential\\ equations};
\draw (104.61999999999999, -59.10000000000001) rectangle (109.96999999999998,-61.20000000000001);
\draw(110.07, -59.10000000000001) node[anchor=north west,align=left] {Monotone \\ systems involving\\ ordinary\\ differential\\ equations};
\draw (110.07, -59.10000000000001) rectangle (114.91999999999999,-61.70000000000001);
\draw(98.66999999999999, -61.80000000000001) node[anchor=north west,align=left] {Qualitative \\ investigation\\ and simulation\\ of ordinary\\ differential\\ equation models};
\draw (98.66999999999999, -61.80000000000001) rectangle (103.01999999999998,-64.9);
\draw(103.11999999999999, -61.80000000000001) node[anchor=north west,align=left] {Multifrequency\\ systems\\ of ordinary\\ differential\\ equations};
\draw (103.11999999999999, -61.80000000000001) rectangle (107.21999999999998,-64.4);
\draw(107.32, -61.80000000000001) node[anchor=north west,align=left] {Invariant \\ manifolds for\\ ordinary\\ differential\\ equations};
\draw (107.32, -61.80000000000001) rectangle (111.16999999999999,-64.4);
\draw(111.26999999999998, -61.80000000000001) node[anchor=north west,align=left] {Symmetries,\\ invariants\\ of ordinary\\ differential\\ equations};
\draw (111.26999999999998, -61.80000000000001) rectangle (114.86999999999998,-64.4);
\draw(98.66999999999999, -65.0) node[anchor=north west,align=left] {Bifurcation\\ theory \\ for ordinary\\ differential\\ equations};
\draw (98.66999999999999, -65.0) rectangle (102.26999999999998,-67.6);
\draw(102.36999999999999, -65.0) node[anchor=north west,align=left] {Relaxation\\ oscillations\\ for ordinary\\ differential\\ equations};
\draw (102.36999999999999, -65.0) rectangle (105.96999999999998,-67.6);
\draw(106.07, -65.0) node[anchor=north west,align=left] {Ordinary \\ differential\\ equations \\ and systems\\ on manifolds};
\draw (106.07, -65.0) rectangle (109.66999999999999,-67.6);
\draw(109.76999999999998, -65.0) node[anchor=north west,align=left] {Hysteresis\\ for ordinary\\ differential\\ equations};
\draw (109.76999999999998, -65.0) rectangle (113.36999999999998,-67.1);
\draw(77.96999999999998, -63.600000000000016) node[anchor=north west,align=left] {\large Ordinary differential equations and systems with randomness};
\draw (77.96999999999998, -63.600000000000016) rectangle (96.85999999999999,-67.30000000000001);
\draw(78.96999999999998, -64.60000000000002) node[anchor=north west,align=left] {Bifurcation of \\ solutions to \\ ordinary differential\\ equations \\ involving randomness};
\draw (78.96999999999998, -64.60000000000002) rectangle (84.81999999999998,-67.20000000000002);
\draw(84.91999999999999, -64.60000000000002) node[anchor=north west,align=left] {Resonance phenomena\\ for ordinary\\ differential\\ equations \\ involving randomness};
\draw (84.91999999999999, -64.60000000000002) rectangle (90.51999999999998,-67.20000000000002);
\draw(90.61999999999998, -64.60000000000002) node[anchor=north west,align=left] {Ordinary \\ differential\\ equations \\ and systems\\ with randomness};
\draw (90.61999999999998, -64.60000000000002) rectangle (94.96999999999997,-67.20000000000002);
\draw(117.36999999999999, -44.80000000000001) node[anchor=north west,align=left] {\large Stability theory for ordinary differential equations};
\draw (117.36999999999999, -44.80000000000001) rectangle (136.97,-57.10000000000001);
\draw(118.36999999999999, -45.80000000000001) node[anchor=north west,align=left] {Structural \\ stability and analogous\\ concepts \\ of solutions to\\ ordinary \\ differential equations};
\draw (118.36999999999999, -45.80000000000001) rectangle (124.71999999999998,-48.90000000000001);
\draw(124.82, -45.80000000000001) node[anchor=north west,align=left] {Asymptotic \\ properties of \\ solutions to \\ ordinary \\ differential equations};
\draw (124.82, -45.80000000000001) rectangle (130.92,-48.40000000000001);
\draw(131.01999999999998, -45.80000000000001) node[anchor=north west,align=left] {Synchronization\\ of solutions\\ to \\ ordinary differential\\ equations};
\draw (131.01999999999998, -45.80000000000001) rectangle (136.86999999999998,-48.40000000000001);
\draw(118.36999999999999, -49.000000000000014) node[anchor=north west,align=left] {Characteristic\\ and Lyapunov\\ exponents of \\ ordinary \\ differential equations};
\draw (118.36999999999999, -49.000000000000014) rectangle (124.46999999999998,-51.600000000000016);
\draw(124.57, -49.000000000000014) node[anchor=north west,align=left] {Dichotomy, \\ trichotomy of \\ solutions to \\ ordinary \\ differential equations};
\draw (124.57, -49.000000000000014) rectangle (130.67,-51.600000000000016);
\draw(130.76999999999998, -49.000000000000014) node[anchor=north west,align=left] {Global stability\\ of \\ solutions to \\ ordinary \\ differential equations};
\draw (130.76999999999998, -49.000000000000014) rectangle (136.86999999999998,-51.600000000000016);
\draw(118.36999999999999, -51.70000000000001) node[anchor=north west,align=left] {Stability of \\ manifolds of \\ solutions to \\ ordinary differential\\ equations};
\draw (118.36999999999999, -51.70000000000001) rectangle (124.21999999999998,-54.30000000000001);
\draw(124.32, -51.70000000000001) node[anchor=north west,align=left] {Perturbations\\ of ordinary\\ differential\\ equations};
\draw (124.32, -51.70000000000001) rectangle (128.17,-53.80000000000001);
\draw(128.26999999999998, -51.70000000000001) node[anchor=north west,align=left] {Singular \\ perturbations\\ of ordinary\\ differential\\ equations};
\draw (128.26999999999998, -51.70000000000001) rectangle (132.11999999999998,-54.30000000000001);
\draw(132.22, -51.70000000000001) node[anchor=north west,align=left] {Stability \\ of solutions\\ to ordinary\\ differential\\ equations};
\draw (132.22, -51.70000000000001) rectangle (135.82,-54.30000000000001);
\draw(118.36999999999999, -54.40000000000001) node[anchor=north west,align=left] {Attractors\\ of solutions\\ to ordinary\\ differential\\ equations};
\draw (118.36999999999999, -54.40000000000001) rectangle (121.96999999999998,-57.000000000000014);
\draw(77.96999999999998, -67.80000000000001) node[anchor=north west,align=left] {\large Asymptotic theory for ordinary differential equations};
\draw (77.96999999999998, -67.80000000000001) rectangle (97.57,-77.4);
\draw(78.96999999999998, -68.80000000000001) node[anchor=north west,align=left] {Asymptotic \\ expansions of \\ solutions to \\ ordinary \\ differential equations};
\draw (78.96999999999998, -68.80000000000001) rectangle (85.06999999999998,-71.4);
\draw(85.16999999999999, -68.80000000000001) node[anchor=north west,align=left] {Perturbations,\\ asymptotics \\ of solutions to\\ ordinary \\ differential equations};
\draw (85.16999999999999, -68.80000000000001) rectangle (91.26999999999998,-71.4);
\draw(91.36999999999998, -68.80000000000001) node[anchor=north west,align=left] {Singular \\ perturbations, general\\ theory for\\ ordinary \\ differential equations};
\draw (91.36999999999998, -68.80000000000001) rectangle (97.46999999999997,-71.4);
\draw(78.96999999999998, -71.50000000000001) node[anchor=north west,align=left] {Singular \\ perturbations, turning\\ point theory,\\ WKB methods for\\ ordinary \\ differential equations};
\draw (78.96999999999998, -71.50000000000001) rectangle (85.06999999999998,-74.60000000000001);
\draw(85.16999999999999, -71.50000000000001) node[anchor=north west,align=left] {Canard solutions\\ to \\ ordinary differential\\ equations};
\draw (85.16999999999999, -71.50000000000001) rectangle (91.01999999999998,-73.60000000000001);
\draw(91.11999999999998, -71.50000000000001) node[anchor=north west,align=left] {Methods of \\ nonstandard \\ analysis for \\ ordinary differential\\ equations};
\draw (91.11999999999998, -71.50000000000001) rectangle (96.96999999999997,-74.10000000000001);
\draw(78.96999999999998, -74.70000000000002) node[anchor=north west,align=left] {Multiple \\ scale methods\\ for ordinary\\ differential\\ equations};
\draw (78.96999999999998, -74.70000000000002) rectangle (82.81999999999998,-77.30000000000001);
\draw(97.66999999999999, -67.80000000000001) node[anchor=north west,align=left] {\large History of \\ ordinary differential\\ equations};
\draw (97.66999999999999, -67.80000000000001) rectangle (104.77999999999999,-69.4);
\draw(1, -52.5) node[anchor=north west,align=left] {\LARGE General algebraic systems};
\draw (1, -52.5) rectangle (22.0,-77.2);
\draw(2, -53.5) node[anchor=north west,align=left] {\large Algebraic structures};
\draw (2, -53.5) rectangle (12.649999999999999,-68.2);
\draw(3, -54.5) node[anchor=north west,align=left] {Heterogeneousalgebras};
\draw (3, -54.5) rectangle (8.85,-56.1);
\draw(8.95, -54.5) node[anchor=north west,align=left] {Subalgebras,\\ congruence\\ relations};
\draw (8.95, -54.5) rectangle (12.549999999999999,-56.6);
\draw(3, -56.7) node[anchor=north west,align=left] {Applications\\ of universal\\ algebra in \\ computer science};
\draw (3, -56.7) rectangle (7.6,-58.800000000000004);
\draw(7.699999999999999, -56.7) node[anchor=north west,align=left] {Operations and\\ polynomials\\ in algebraic\\ structures,\\ primal algebras};
\draw (7.699999999999999, -56.7) rectangle (12.049999999999999,-59.300000000000004);
\draw(3, -59.4) node[anchor=north west,align=left] {Automorphisms\\ and \\ endomorphisms \\ of algebraic\\ structures};
\draw (3, -59.4) rectangle (7.1,-62.0);
\draw(7.199999999999999, -59.4) node[anchor=north west,align=left] {Word problems\\ (aspects\\ of algebraic\\ structures)};
\draw (7.199999999999999, -59.4) rectangle (11.049999999999999,-61.5);
\draw(3, -62.1) node[anchor=north west,align=left] {Relational\\ systems,\\ laws of\\ composition};
\draw (3, -62.1) rectangle (6.35,-64.2);
\draw(6.449999999999999, -62.1) node[anchor=north west,align=left] {Equational\\ compactness};
\draw (6.449999999999999, -62.1) rectangle (9.799999999999999,-63.7);
\draw(9.899999999999999, -62.1) node[anchor=north west,align=left] {Partial\\ algebras};
\draw (9.899999999999999, -62.1) rectangle (12.499999999999998,-63.2);
\draw(3, -64.3) node[anchor=north west,align=left] {Structure\\ theory of\\ algebraic\\ structures};
\draw (3, -64.3) rectangle (6.1,-66.39999999999999);
\draw(6.199999999999999, -64.3) node[anchor=north west,align=left] {Infinitary\\ algebras};
\draw (6.199999999999999, -64.3) rectangle (9.299999999999999,-65.89999999999999);
\draw(9.399999999999999, -64.3) node[anchor=north west,align=left] {Fuzzy \\ algebraic\\ structures};
\draw (9.399999999999999, -64.3) rectangle (12.499999999999998,-65.89999999999999);
\draw(3, -66.5) node[anchor=north west,align=left] {Unary \\ algebras};
\draw (3, -66.5) rectangle (5.6,-67.6);
\draw(5.7, -66.5) node[anchor=north west,align=left] {Finitary\\ algebras};
\draw (5.7, -66.5) rectangle (8.3,-68.1);
\draw(12.749999999999998, -53.5) node[anchor=north west,align=left] {\large Other classes of algebras};
\draw (12.749999999999998, -53.5) rectangle (21.9,-58.4);
\draw(13.749999999999998, -54.5) node[anchor=north west,align=left] {Quasivarieties};
\draw (13.749999999999998, -54.5) rectangle (17.849999999999998,-55.6);
\draw(17.949999999999996, -54.5) node[anchor=north west,align=left] {Natural \\ dualities for\\ classes\\ of algebras};
\draw (17.949999999999996, -54.5) rectangle (21.799999999999997,-56.6);
\draw(13.749999999999998, -56.7) node[anchor=north west,align=left] {Categories\\ of\\ algebras};
\draw (13.749999999999998, -56.7) rectangle (16.849999999999998,-58.300000000000004);
\draw(16.949999999999996, -56.7) node[anchor=north west,align=left] {Axiomatic\\ model\\ classes};
\draw (16.949999999999996, -56.7) rectangle (19.799999999999997,-58.300000000000004);
\draw(2, -68.3) node[anchor=north west,align=left] {\large Varieties};
\draw (2, -68.3) rectangle (12.149999999999999,-75.89999999999999);
\draw(3, -69.3) node[anchor=north west,align=left] {Products, \\ amalgamated products,\\ and other\\ kinds of limits\\ and colimits};
\draw (3, -69.3) rectangle (8.85,-71.89999999999999);
\draw(8.95, -69.3) node[anchor=north west,align=left] {Equational\\ logic,\\ Mal’tsev\\ conditions};
\draw (8.95, -69.3) rectangle (12.049999999999999,-71.39999999999999);
\draw(3, -72.0) node[anchor=north west,align=left] {Congruence \\ modularity, \\ congruence \\ distributivity};
\draw (3, -72.0) rectangle (7.1,-74.1);
\draw(7.199999999999999, -72.0) node[anchor=north west,align=left] {Subdirect \\ products and\\ subdirect\\ irreducibility};
\draw (7.199999999999999, -72.0) rectangle (11.299999999999999,-74.1);
\draw(3, -74.2) node[anchor=north west,align=left] {Injectives,\\ projectives};
\draw (3, -74.2) rectangle (6.35,-75.8);
\draw(6.449999999999999, -74.2) node[anchor=north west,align=left] {Lattices\\ of \\ varieties};
\draw (6.449999999999999, -74.2) rectangle (9.299999999999999,-75.8);
\draw(9.399999999999999, -74.2) node[anchor=north west,align=left] {Free \\ algebras};
\draw (9.399999999999999, -74.2) rectangle (11.999999999999998,-75.3);
\draw(12.249999999999998, -68.3) node[anchor=north west,align=left] {\large Computational methods for\\ problems pertaining to\\ general algebraic systems};
\draw (12.249999999999998, -68.3) rectangle (20.599999999999998,-69.89999999999999);
\draw(2, -76.0) node[anchor=north west,align=left] {\large History of general\\ algebraic systems};
\draw (2, -76.0) rectangle (8.18,-77.1);
\draw(1, -77.6) node[anchor=north west,align=left] {\LARGE Several complex variables and analytic spaces};
\draw (1, -77.6) rectangle (64.4,-141.1);
\draw(2, -78.6) node[anchor=north west,align=left] {\large Non-Archimedean analysis (should also be assigned at least one other classification number from Section 32-XX describing the type of problem)};
\draw (2, -78.6) rectangle (46.31,-83.3);
\draw(3, -79.6) node[anchor=north west,align=left] {Non-Archimedean \\ analysis (should also be\\ assigned at least \\ one other classification\\ number from Section\\ 32-XX describing\\ the type of problem)};
\draw (3, -79.6) rectangle (9.6,-83.19999999999999);
\draw(46.410000000000004, -78.6) node[anchor=north west,align=left] {\large Geometric convexity in several complex variables};
\draw (46.410000000000004, -78.6) rectangle (62.760000000000005,-84.5);
\draw(47.410000000000004, -79.6) node[anchor=north west,align=left] {Analytical \\ consequences of \\ geometric convexity\\ (vanishing\\ theorems, etc.)};
\draw (47.410000000000004, -79.6) rectangle (52.760000000000005,-82.19999999999999);
\draw(52.86, -79.6) node[anchor=north west,align=left] {Other notions\\ of convexity\\ in relation\\ to several \\ complex variables};
\draw (52.86, -79.6) rectangle (57.71,-82.19999999999999);
\draw(57.81, -79.6) node[anchor=north west,align=left] {Invariant \\ metrics and \\ pseudodistances \\ in several \\ complex variables};
\draw (57.81, -79.6) rectangle (62.660000000000004,-82.19999999999999);
\draw(47.410000000000004, -82.3) node[anchor=north west,align=left] {\(q\)-convexity,\\ \(q\)-concavity};
\draw (47.410000000000004, -82.3) rectangle (52.010000000000005,-83.89999999999999);
\draw(52.11, -82.3) node[anchor=north west,align=left] {Finite-type\\ conditions \\ for the boundary\\ of a domain};
\draw (52.11, -82.3) rectangle (56.71,-84.39999999999999);
\draw(56.81, -82.3) node[anchor=north west,align=left] {Topological\\ consequences\\ of geometric\\ convexity};
\draw (56.81, -82.3) rectangle (60.410000000000004,-84.39999999999999);
\draw(2, -84.6) node[anchor=north west,align=left] {\large Holomorphic functions of several complex variables};
\draw (2, -84.6) rectangle (21.1,-109.69999999999999);
\draw(3, -85.6) node[anchor=north west,align=left] {Other generalizations\\ of function theory\\ of one complex \\ variable (should also be\\ assigned at least \\ one classification \\ number from Section 30-XX)};
\draw (3, -85.6) rectangle (10.1,-89.19999999999999);
\draw(10.2, -85.6) node[anchor=north west,align=left] {Other spaces of \\ holomorphic functions of\\ several complex \\ variables (e.g., bounded\\ mean oscillation \\ (BMOA), vanishing mean\\ oscillation (VMOA))};
\draw (10.2, -85.6) rectangle (16.799999999999997,-89.19999999999999);
\draw(16.9, -85.6) node[anchor=north west,align=left] {Hyperfunctions};
\draw (16.9, -85.6) rectangle (21.0,-86.69999999999999);
\draw(3, -89.3) node[anchor=north west,align=left] {Integral \\ representations, \\ constructed kernels\\ (e.g., \\ Cauchy, Fantappiè-type\\ kernels)};
\draw (3, -89.3) rectangle (9.1,-92.39999999999999);
\draw(9.2, -89.3) node[anchor=north west,align=left] {Normal families \\ of holomorphic \\ functions, mappings\\ of several complex\\ variables, and\\ related topics \\ (taut manifolds etc.)};
\draw (9.2, -89.3) rectangle (15.049999999999999,-92.89999999999999);
\draw(15.149999999999999, -89.3) node[anchor=north west,align=left] {Functional analysis\\ techniques \\ applied to functions\\ of several \\ complex variables};
\draw (15.149999999999999, -89.3) rectangle (20.75,-91.89999999999999);
\draw(3, -93.0) node[anchor=north west,align=left] {\(H^p\)-spaces,\\ Nevanlinna \\ spaces of functions\\ in several \\ complex variables};
\draw (3, -93.0) rectangle (8.35,-95.6);
\draw(8.45, -93.0) node[anchor=north west,align=left] {Banach algebra\\ techniques applied\\ to functions\\ of several \\ complex variables};
\draw (8.45, -93.0) rectangle (13.549999999999999,-95.6);
\draw(13.649999999999999, -93.0) node[anchor=north west,align=left] {Power series,\\ series of\\ functions of\\ several \\ complex variables};
\draw (13.649999999999999, -93.0) rectangle (18.5,-95.6);
\draw(3, -95.69999999999999) node[anchor=north west,align=left] {Polynomials\\ and rational\\ functions \\ of several \\ complex variables};
\draw (3, -95.69999999999999) rectangle (7.85,-98.29999999999998);
\draw(7.949999999999999, -95.69999999999999) node[anchor=north west,align=left] {Holomorphic\\ functions of\\ several \\ complex variables};
\draw (7.949999999999999, -95.69999999999999) rectangle (12.799999999999999,-97.79999999999998);
\draw(12.899999999999999, -95.69999999999999) node[anchor=north west,align=left] {Special \\ families of \\ functions of\\ several \\ complex variables};
\draw (12.899999999999999, -95.69999999999999) rectangle (17.75,-98.29999999999998);
\draw(3, -98.39999999999999) node[anchor=north west,align=left] {Bloch functions,\\ normal\\ functions of\\ several \\ complex variables};
\draw (3, -98.39999999999999) rectangle (7.85,-100.99999999999999);
\draw(7.949999999999999, -98.39999999999999) node[anchor=north west,align=left] {Meromorphic\\ functions of\\ several \\ complex variables};
\draw (7.949999999999999, -98.39999999999999) rectangle (12.799999999999999,-100.49999999999999);
\draw(12.899999999999999, -98.39999999999999) node[anchor=north west,align=left] {Nevanlinna \\ theory; growth \\ estimates; other\\ inequalities\\ of several \\ complex variables};
\draw (12.899999999999999, -98.39999999999999) rectangle (17.75,-101.49999999999999);
\draw(3, -101.6) node[anchor=north west,align=left] {Algebras of\\ holomorphic\\ functions of\\ several \\ complex variables};
\draw (3, -101.6) rectangle (7.85,-104.19999999999999);
\draw(7.949999999999999, -101.6) node[anchor=north west,align=left] {Boundary behavior\\ of holomorphic\\ functions\\ of several \\ complex variables};
\draw (7.949999999999999, -101.6) rectangle (12.799999999999999,-104.19999999999999);
\draw(12.899999999999999, -101.6) node[anchor=north west,align=left] {Zero sets of\\ holomorphic\\ functions \\ of several \\ complex variables};
\draw (12.899999999999999, -101.6) rectangle (17.75,-104.19999999999999);
\draw(3, -104.3) node[anchor=north west,align=left] {Integral \\ representations;\\ canonical \\ kernels (Szegő,\\ Bergman, etc.)};
\draw (3, -104.3) rectangle (7.6,-106.89999999999999);
\draw(7.699999999999999, -104.3) node[anchor=north west,align=left] {Multifunctions\\ of \\ several complex\\ variables};
\draw (7.699999999999999, -104.3) rectangle (12.049999999999999,-106.39999999999999);
\draw(12.149999999999999, -104.3) node[anchor=north west,align=left] {Entire \\ functions of \\ several complex\\ variables};
\draw (12.149999999999999, -104.3) rectangle (16.5,-106.39999999999999);
\draw(16.599999999999998, -104.3) node[anchor=north west,align=left] {Bergman \\ spaces of \\ functions in \\ several complex\\ variables};
\draw (16.599999999999998, -104.3) rectangle (20.949999999999996,-106.89999999999999);
\draw(3, -107.0) node[anchor=north west,align=left] {Harmonic \\ analysis of \\ several complex\\ variables};
\draw (3, -107.0) rectangle (7.35,-109.1);
\draw(7.449999999999999, -107.0) node[anchor=north west,align=left] {Singular \\ integrals of \\ functions in \\ several complex\\ variables};
\draw (7.449999999999999, -107.0) rectangle (11.799999999999999,-109.6);
\draw(11.899999999999999, -107.0) node[anchor=north west,align=left] {Residues\\ for several\\ complex\\ variables};
\draw (11.899999999999999, -107.0) rectangle (15.249999999999998,-109.1);
\draw(21.200000000000003, -84.6) node[anchor=north west,align=left] {\large Differential operators in several variables};
\draw (21.200000000000003, -84.6) rectangle (36.35,-93.19999999999999);
\draw(22.200000000000003, -85.6) node[anchor=north west,align=left] {\(\overline\partial_b\)\\ and\\ \(\overline\partial_b\)-Neumann\\ operators};
\draw (22.200000000000003, -85.6) rectangle (30.550000000000004,-88.19999999999999);
\draw(30.650000000000002, -85.6) node[anchor=north west,align=left] {Other partial \\ differential \\ equations of complex\\ analysis in\\ several variables};
\draw (30.650000000000002, -85.6) rectangle (36.25,-88.19999999999999);
\draw(22.200000000000003, -88.3) node[anchor=north west,align=left] {\(\overline\partial\)\\ and\\ \(\overline\partial\)-Neumann\\ operators};
\draw (22.200000000000003, -88.3) rectangle (30.050000000000004,-90.89999999999999);
\draw(30.150000000000002, -88.3) node[anchor=north west,align=left] {Pseudodifferential\\ operators in \\ several complex\\ variables};
\draw (30.150000000000002, -88.3) rectangle (35.25,-90.89999999999999);
\draw(22.200000000000003, -91.0) node[anchor=north west,align=left] {Complex \\ Monge-Ampère\\ operators};
\draw (22.200000000000003, -91.0) rectangle (25.800000000000004,-92.6);
\draw(25.900000000000002, -91.0) node[anchor=north west,align=left] {Heat kernels\\ in several\\ complex\\ variables};
\draw (25.900000000000002, -91.0) rectangle (29.500000000000004,-93.1);
\draw(21.200000000000003, -93.3) node[anchor=north west,align=left] {\large Holomorphic mappings and correspondences};
\draw (21.200000000000003, -93.3) rectangle (35.8,-106.6);
\draw(22.200000000000003, -94.3) node[anchor=north west,align=left] {Holomorphic \\ mappings, (holomorphic)\\ embeddings \\ and related questions\\ in several\\ complex variables};
\draw (22.200000000000003, -94.3) rectangle (28.550000000000004,-97.39999999999999);
\draw(28.650000000000002, -94.3) node[anchor=north west,align=left] {Proper \\ holomorphic mappings,\\ finiteness\\ theorems};
\draw (28.650000000000002, -94.3) rectangle (34.5,-96.39999999999999);
\draw(22.200000000000003, -97.5) node[anchor=north west,align=left] {Iteration of \\ holomorphic maps, \\ fixed points of \\ holomorphic maps\\ and related \\ problems for several\\ complex variables};
\draw (22.200000000000003, -97.5) rectangle (27.800000000000004,-101.1);
\draw(27.900000000000002, -97.5) node[anchor=north west,align=left] {Picard-type \\ theorems and \\ generalizations \\ for several \\ complex variables};
\draw (27.900000000000002, -97.5) rectangle (32.75,-100.1);
\draw(22.200000000000003, -101.2) node[anchor=north west,align=left] {Value \\ distribution \\ theory in higher\\ dimensions};
\draw (22.200000000000003, -101.2) rectangle (26.800000000000004,-103.3);
\draw(26.900000000000002, -101.2) node[anchor=north west,align=left] {Meromorphic\\ mappings in\\ several complex\\ variables};
\draw (26.900000000000002, -101.2) rectangle (31.25,-103.3);
\draw(31.35, -101.2) node[anchor=north west,align=left] {Boundary \\ uniqueness of \\ mappings in \\ several complex\\ variables};
\draw (31.35, -101.2) rectangle (35.7,-103.8);
\draw(22.200000000000003, -103.9) node[anchor=north west,align=left] {Boundary \\ regularity of \\ mappings in \\ several complex\\ variables};
\draw (22.200000000000003, -103.9) rectangle (26.550000000000004,-106.5);
\draw(21.200000000000003, -106.69999999999999) node[anchor=north west,align=left] {\large Computational methods \\ for problems pertaining\\ to several complex \\ variables and analytic spaces};
\draw (21.200000000000003, -106.69999999999999) rectangle (30.790000000000003,-108.79999999999998);
\draw(36.45, -84.6) node[anchor=north west,align=left] {\large Complex spaces with a group of automorphisms};
\draw (36.45, -84.6) rectangle (50.85,-93.69999999999999);
\draw(37.45, -85.6) node[anchor=north west,align=left] {Hermitian symmetric\\ spaces, bounded\\ symmetric \\ domains, Jordan \\ algebras (complex-analytic\\ aspects)};
\draw (37.45, -85.6) rectangle (44.550000000000004,-88.69999999999999);
\draw(44.650000000000006, -85.6) node[anchor=north west,align=left] {Complex vector\\ fields, \\ holomorphic \\ foliations, \\ \(\mathbb{C}\)-actions};
\draw (44.650000000000006, -85.6) rectangle (50.75000000000001,-88.19999999999999);
\draw(37.45, -88.8) node[anchor=north west,align=left] {Automorphism\\ groups of \\ \(\mathbb{C}^n\)\\ and affine\\ manifolds};
\draw (37.45, -88.8) rectangle (42.050000000000004,-91.39999999999999);
\draw(42.150000000000006, -88.8) node[anchor=north west,align=left] {Complex Lie\\ groups, group\\ actions on\\ complex spaces};
\draw (42.150000000000006, -88.8) rectangle (46.25000000000001,-90.89999999999999);
\draw(46.35, -88.8) node[anchor=north west,align=left] {Automorphism\\ groups\\ of other \\ complex spaces};
\draw (46.35, -88.8) rectangle (50.45,-90.89999999999999);
\draw(37.45, -91.5) node[anchor=north west,align=left] {Homogeneous\\ complex\\ manifolds};
\draw (37.45, -91.5) rectangle (40.800000000000004,-93.1);
\draw(40.900000000000006, -91.5) node[anchor=north west,align=left] {Almost \\ homogeneous\\ manifolds\\ and spaces};
\draw (40.900000000000006, -91.5) rectangle (44.25000000000001,-93.6);
\draw(50.95, -84.6) node[anchor=north west,align=left] {\large Deformations of analytic structures};
\draw (50.95, -84.6) rectangle (64.3,-96.39999999999999);
\draw(51.95, -85.6) node[anchor=north west,align=left] {Moduli and \\ deformations for \\ ordinary differential\\ equations (e.g.,\\ Knizhnik-Zamolodchikov\\ equation)};
\draw (51.95, -85.6) rectangle (58.050000000000004,-88.69999999999999);
\draw(58.150000000000006, -85.6) node[anchor=north west,align=left] {Moduli of Riemann\\ surfaces, \\ Teichmüller theory\\ (complex-analytic\\ aspects in\\ several variables)};
\draw (58.150000000000006, -85.6) rectangle (63.25000000000001,-88.69999999999999);
\draw(51.95, -88.8) node[anchor=north west,align=left] {Complex-analytic\\ moduli problems};
\draw (51.95, -88.8) rectangle (56.550000000000004,-90.39999999999999);
\draw(56.650000000000006, -88.8) node[anchor=north west,align=left] {Period matrices,\\ variation\\ of Hodge\\ structure; \\ degenerations};
\draw (56.650000000000006, -88.8) rectangle (61.25000000000001,-91.39999999999999);
\draw(51.95, -91.5) node[anchor=north west,align=left] {Applications\\ of deformations\\ of analytic\\ structures\\ to the sciences};
\draw (51.95, -91.5) rectangle (56.300000000000004,-94.1);
\draw(56.400000000000006, -91.5) node[anchor=north west,align=left] {Deformations\\ of \\ fiber bundles};
\draw (56.400000000000006, -91.5) rectangle (60.25000000000001,-93.1);
\draw(60.35, -91.5) node[anchor=north west,align=left] {Deformations\\ of \\ submanifolds \\ and subspaces};
\draw (60.35, -91.5) rectangle (64.2,-93.6);
\draw(51.95, -94.19999999999999) node[anchor=north west,align=left] {Deformations\\ of\\ complex\\ structures};
\draw (51.95, -94.19999999999999) rectangle (55.550000000000004,-96.29999999999998);
\draw(55.650000000000006, -94.19999999999999) node[anchor=north west,align=left] {Deformations\\ of special\\ (e.g., CR)\\ structures};
\draw (55.650000000000006, -94.19999999999999) rectangle (59.25000000000001,-96.29999999999998);
\draw(36.45, -93.8) node[anchor=north west,align=left] {\large History of several\\ complex variables\\ and analytic spaces};
\draw (36.45, -93.8) rectangle (42.940000000000005,-95.39999999999999);
\draw(2, -109.8) node[anchor=north west,align=left] {\large Complex singularities};
\draw (2, -109.8) rectangle (16.099999999999998,-128.0);
\draw(3, -110.8) node[anchor=north west,align=left] {Monodromy; \\ relations with \\ differential equations\\ and \(D\)-modules\\ (complex-analytic aspects)};
\draw (3, -110.8) rectangle (10.1,-113.89999999999999);
\draw(10.2, -110.8) node[anchor=north west,align=left] {Mixed Hodge \\ theory of \\ singular varieties\\ (complex-analytic\\ aspects)};
\draw (10.2, -110.8) rectangle (15.299999999999999,-113.39999999999999);
\draw(3, -114.0) node[anchor=north west,align=left] {Topological aspects\\ of complex \\ singularities: Lefschetz\\ theorems, \\ topological \\ classification, invariants};
\draw (3, -114.0) rectangle (10.1,-117.1);
\draw(10.2, -114.0) node[anchor=north west,align=left] {Singularities\\ of \\ holomorphic vector\\ fields \\ and foliations};
\draw (10.2, -114.0) rectangle (15.299999999999999,-116.6);
\draw(3, -117.2) node[anchor=north west,align=left] {Stratifications;\\ constructible\\ sheaves; \\ intersection cohomology\\ (complex-analytic\\ aspects)};
\draw (3, -117.2) rectangle (9.35,-120.3);
\draw(9.45, -117.2) node[anchor=north west,align=left] {Modifications;\\ resolution\\ of singularities\\ (complex-analytic\\ aspects)};
\draw (9.45, -117.2) rectangle (14.299999999999999,-119.8);
\draw(3, -120.4) node[anchor=north west,align=left] {Equisingularity\\ (topological \\ and analytic)};
\draw (3, -120.4) rectangle (7.35,-122.5);
\draw(7.449999999999999, -120.4) node[anchor=north west,align=left] {Relations\\ with \\ arrangements of\\ hyperplanes};
\draw (7.449999999999999, -120.4) rectangle (11.799999999999999,-122.5);
\draw(11.899999999999999, -120.4) node[anchor=north west,align=left] {Global theory\\ of complex\\ singularities;\\ cohomological\\ properties};
\draw (11.899999999999999, -120.4) rectangle (15.999999999999998,-123.0);
\draw(3, -123.1) node[anchor=north west,align=left] {Deformations\\ of complex\\ singularities;\\ vanishing\\ cycles};
\draw (3, -123.1) rectangle (7.1,-125.69999999999999);
\draw(7.199999999999999, -123.1) node[anchor=north west,align=left] {Milnor \\ fibration; \\ relations with\\ knot theory};
\draw (7.199999999999999, -123.1) rectangle (11.299999999999999,-125.19999999999999);
\draw(11.399999999999999, -123.1) node[anchor=north west,align=left] {Local \\ complex \\ singularities};
\draw (11.399999999999999, -123.1) rectangle (15.249999999999998,-124.69999999999999);
\draw(3, -125.8) node[anchor=north west,align=left] {Complex \\ surface and \\ hypersurface\\ singularities};
\draw (3, -125.8) rectangle (6.85,-127.89999999999999);
\draw(6.949999999999999, -125.8) node[anchor=north west,align=left] {Other \\ operations on\\ complex \\ singularities};
\draw (6.949999999999999, -125.8) rectangle (10.799999999999999,-127.89999999999999);
\draw(10.9, -125.8) node[anchor=north west,align=left] {Invariants\\ of \\ analytic \\ local rings};
\draw (10.9, -125.8) rectangle (14.25,-127.89999999999999);
\draw(16.199999999999996, -109.8) node[anchor=north west,align=left] {\large Generalizations of analytic spaces};
\draw (16.199999999999996, -109.8) rectangle (28.849999999999994,-115.7);
\draw(17.199999999999996, -110.8) node[anchor=north west,align=left] {Differentiable\\ functions \\ on analytic \\ spaces, \\ differentiable spaces};
\draw (17.199999999999996, -110.8) rectangle (23.049999999999997,-113.39999999999999);
\draw(23.149999999999995, -110.8) node[anchor=north west,align=left] {Holomorphic\\ maps with \\ infinite-dimensional\\ arguments\\ or values};
\draw (23.149999999999995, -110.8) rectangle (28.749999999999993,-113.39999999999999);
\draw(17.199999999999996, -113.5) node[anchor=north west,align=left] {Formal and\\ graded \\ complex spaces};
\draw (17.199999999999996, -113.5) rectangle (21.299999999999997,-115.1);
\draw(21.399999999999995, -113.5) node[anchor=north west,align=left] {Banach \\ analytic \\ manifolds \\ and spaces};
\draw (21.399999999999995, -113.5) rectangle (24.499999999999996,-115.6);
\draw(16.199999999999996, -115.8) node[anchor=north west,align=left] {\large Holomorphic convexity};
\draw (16.199999999999996, -115.8) rectangle (28.599999999999994,-125.89999999999999);
\draw(17.199999999999996, -116.8) node[anchor=north west,align=left] {Holomorphic, \\ polynomial and rational\\ approximation, and\\ interpolation in\\ several complex \\ variables; Runge pairs};
\draw (17.199999999999996, -116.8) rectangle (23.549999999999997,-119.89999999999999);
\draw(23.649999999999995, -116.8) node[anchor=north west,align=left] {Holomorphically\\ convex\\ complex \\ spaces, reduction\\ theory};
\draw (23.649999999999995, -116.8) rectangle (28.499999999999993,-119.39999999999999);
\draw(17.199999999999996, -120.0) node[anchor=north west,align=left] {Polynomial \\ convexity, rational\\ convexity, \\ meromorphic convexity\\ in several\\ complex variables};
\draw (17.199999999999996, -120.0) rectangle (23.049999999999997,-123.1);
\draw(23.149999999999995, -120.0) node[anchor=north west,align=left] {Stein \\ spaces, Stein\\ manifolds};
\draw (23.149999999999995, -120.0) rectangle (26.999999999999996,-121.6);
\draw(17.199999999999996, -123.2) node[anchor=north west,align=left] {Global boundary\\ behavior of \\ holomorphic functions\\ of several\\ complex variables};
\draw (17.199999999999996, -123.2) rectangle (23.049999999999997,-125.8);
\draw(23.149999999999995, -123.2) node[anchor=north west,align=left] {The Levi\\ problem};
\draw (23.149999999999995, -123.2) rectangle (25.749999999999996,-124.3);
\draw(28.949999999999996, -109.8) node[anchor=north west,align=left] {\large Complex manifolds};
\draw (28.949999999999996, -109.8) rectangle (41.599999999999994,-126.7);
\draw(29.949999999999996, -110.8) node[anchor=north west,align=left] {Kähler-Einsteinmanifolds};
\draw (29.949999999999996, -110.8) rectangle (36.55,-112.39999999999999);
\draw(36.64999999999999, -110.8) node[anchor=north west,align=left] {Calabi-Yau\\ theory \\ (complex-analytic\\ aspects)};
\draw (36.64999999999999, -110.8) rectangle (41.49999999999999,-112.89999999999999);
\draw(29.949999999999996, -113.0) node[anchor=north west,align=left] {Special domains\\ (Reinhardt, \\ Hartogs, circular, \\ tube, etc.) in \\ \(\mathbb{C}^n\) \\ and complex manifolds};
\draw (29.949999999999996, -113.0) rectangle (35.8,-116.1);
\draw(35.89999999999999, -113.0) node[anchor=north west,align=left] {Pseudoholomorphic\\ curves};
\draw (35.89999999999999, -113.0) rectangle (40.74999999999999,-114.6);
\draw(29.949999999999996, -116.2) node[anchor=north west,align=left] {Complex \\ manifolds as \\ subdomains of \\ Euclidean space};
\draw (29.949999999999996, -116.2) rectangle (34.3,-118.3);
\draw(34.39999999999999, -116.2) node[anchor=north west,align=left] {Uniformization\\ of complex\\ manifolds};
\draw (34.39999999999999, -116.2) rectangle (38.49999999999999,-118.3);
\draw(38.599999999999994, -116.2) node[anchor=north west,align=left] {Negative\\ curvature\\ complex\\ manifolds};
\draw (38.599999999999994, -116.2) rectangle (41.449999999999996,-118.3);
\draw(29.949999999999996, -118.4) node[anchor=north west,align=left] {Classification\\ theorems\\ for complex\\ manifolds};
\draw (29.949999999999996, -118.4) rectangle (34.05,-120.5);
\draw(34.14999999999999, -118.4) node[anchor=north west,align=left] {Hyperbolic\\ and Kobayashi\\ hyperbolic\\ manifolds};
\draw (34.14999999999999, -118.4) rectangle (37.99999999999999,-120.5);
\draw(38.099999999999994, -118.4) node[anchor=north west,align=left] {Notions of\\ stability\\ for complex\\ manifolds};
\draw (38.099999999999994, -118.4) rectangle (41.449999999999996,-120.5);
\draw(29.949999999999996, -120.6) node[anchor=north west,align=left] {Oka principle\\ and Oka\\ manifolds};
\draw (29.949999999999996, -120.6) rectangle (33.8,-122.19999999999999);
\draw(33.89999999999999, -120.6) node[anchor=north west,align=left] {Embedding\\ theorems \\ for complex\\ manifolds};
\draw (33.89999999999999, -120.6) rectangle (37.24999999999999,-122.69999999999999);
\draw(37.349999999999994, -120.6) node[anchor=north west,align=left] {Topological\\ aspects\\ of complex\\ manifolds};
\draw (37.349999999999994, -120.6) rectangle (40.699999999999996,-122.69999999999999);
\draw(29.949999999999996, -122.8) node[anchor=north west,align=left] {Positive\\ curvature\\ complex\\ manifolds};
\draw (29.949999999999996, -122.8) rectangle (32.8,-124.89999999999999);
\draw(32.9, -122.8) node[anchor=north west,align=left] {Kähler \\ manifolds};
\draw (32.9, -122.8) rectangle (35.75,-123.89999999999999);
\draw(35.849999999999994, -122.8) node[anchor=north west,align=left] {Stein \\ manifolds};
\draw (35.849999999999994, -122.8) rectangle (38.699999999999996,-123.89999999999999);
\draw(29.949999999999996, -125.0) node[anchor=north west,align=left] {Almost \\ complex \\ manifolds};
\draw (29.949999999999996, -125.0) rectangle (32.8,-126.6);
\draw(16.199999999999996, -115.8) node[anchor=north west,align=left] {\large Local analytic geometry};
\draw (16.199999999999996, -115.8) rectangle (27.599999999999994,-124.39999999999999);
\draw(17.199999999999996, -116.8) node[anchor=north west,align=left] {Triangulation and\\ topological \\ properties of \\ semi-analytic \\ andsubanalytic sets, \\ and related questions};
\draw (17.199999999999996, -116.8) rectangle (23.049999999999997,-119.89999999999999);
\draw(23.149999999999995, -116.8) node[anchor=north west,align=left] {Germs of \\ analytic sets,\\ local \\ parametrization};
\draw (23.149999999999995, -116.8) rectangle (27.499999999999993,-118.89999999999999);
\draw(17.199999999999996, -120.0) node[anchor=north west,align=left] {Analytic \\ algebras and\\ generalizations,\\ preparation theorems};
\draw (17.199999999999996, -120.0) rectangle (22.799999999999997,-122.6);
\draw(22.899999999999995, -120.0) node[anchor=north west,align=left] {Semi-analytic\\ sets, \\ subanalytic \\ sets, and \\ generalizations};
\draw (22.899999999999995, -120.0) rectangle (27.249999999999993,-122.6);
\draw(17.199999999999996, -122.7) node[anchor=north west,align=left] {Analytic \\ subsets of\\ affine space};
\draw (17.199999999999996, -122.7) rectangle (20.799999999999997,-124.3);
\draw(41.699999999999996, -109.8) node[anchor=north west,align=left] {\large Analytic spaces};
\draw (41.699999999999996, -109.8) rectangle (51.849999999999994,-128.4);
\draw(42.699999999999996, -110.8) node[anchor=north west,align=left] {The Levi \\ problem in complex\\ spaces;\\ generalizations};
\draw (42.699999999999996, -110.8) rectangle (47.8,-112.89999999999999);
\draw(47.89999999999999, -110.8) node[anchor=north west,align=left] {Real-analytic\\ manifolds,\\ real-analytic\\ spaces};
\draw (47.89999999999999, -110.8) rectangle (51.74999999999999,-112.89999999999999);
\draw(42.699999999999996, -113.0) node[anchor=north west,align=left] {Applications\\ of analytic \\ spaces to physics\\ and other\\ areas of science};
\draw (42.699999999999996, -113.0) rectangle (47.55,-115.6);
\draw(47.64999999999999, -113.0) node[anchor=north west,align=left] {Real-analytic\\ sets,\\ complex \\ Nash functions};
\draw (47.64999999999999, -113.0) rectangle (51.74999999999999,-115.1);
\draw(42.699999999999996, -115.7) node[anchor=north west,align=left] {Integration\\ on analytic\\ sets and \\ spaces, currents};
\draw (42.699999999999996, -115.7) rectangle (47.3,-117.8);
\draw(47.39999999999999, -115.7) node[anchor=north west,align=left] {Local \\ cohomology\\ of \\ analytic spaces};
\draw (47.39999999999999, -115.7) rectangle (51.74999999999999,-117.8);
\draw(42.699999999999996, -117.9) node[anchor=north west,align=left] {Analytic\\ sheaves \\ and cohomology\\ groups};
\draw (42.699999999999996, -117.9) rectangle (46.8,-120.0);
\draw(46.89999999999999, -117.9) node[anchor=north west,align=left] {Sheaves of \\ differential \\ operators and \\ their modules,\\ \(D\)-modules};
\draw (46.89999999999999, -117.9) rectangle (50.99999999999999,-120.5);
\draw(42.699999999999996, -120.6) node[anchor=north west,align=left] {Embedding\\ of \\ real-analytic\\ manifolds};
\draw (42.699999999999996, -120.6) rectangle (46.55,-122.69999999999999);
\draw(46.64999999999999, -120.6) node[anchor=north west,align=left] {Complex\\ supergeometry};
\draw (46.64999999999999, -120.6) rectangle (50.49999999999999,-122.19999999999999);
\draw(42.699999999999996, -122.8) node[anchor=north west,align=left] {Analytic\\ subsets\\ and \\ submanifolds};
\draw (42.699999999999996, -122.8) rectangle (46.3,-124.89999999999999);
\draw(46.39999999999999, -122.8) node[anchor=north west,align=left] {Duality \\ theorems \\ for analytic\\ spaces};
\draw (46.39999999999999, -122.8) rectangle (49.99999999999999,-124.89999999999999);
\draw(42.699999999999996, -125.0) node[anchor=north west,align=left] {Topology\\ of analytic\\ spaces};
\draw (42.699999999999996, -125.0) rectangle (46.05,-126.6);
\draw(46.14999999999999, -125.0) node[anchor=north west,align=left] {Embedding\\ of analytic\\ spaces};
\draw (46.14999999999999, -125.0) rectangle (49.49999999999999,-126.6);
\draw(42.699999999999996, -126.69999999999999) node[anchor=north west,align=left] {Normal\\ analytic\\ spaces};
\draw (42.699999999999996, -126.69999999999999) rectangle (45.3,-128.29999999999998);
\draw(45.4, -126.69999999999999) node[anchor=north west,align=left] {Complex\\ spaces};
\draw (45.4, -126.69999999999999) rectangle (47.75,-127.79999999999998);
\draw(51.949999999999996, -109.8) node[anchor=north west,align=left] {\large Pseudoconvex domains};
\draw (51.949999999999996, -109.8) rectangle (62.099999999999994,-116.89999999999999);
\draw(52.949999999999996, -110.8) node[anchor=north west,align=left] {Geometric and\\ analytic \\ invariants on \\ weakly pseudoconvex\\ boundaries};
\draw (52.949999999999996, -110.8) rectangle (58.3,-113.39999999999999);
\draw(58.39999999999999, -110.8) node[anchor=north west,align=left] {Strongly\\ pseudoconvex\\ domains};
\draw (58.39999999999999, -110.8) rectangle (61.99999999999999,-112.39999999999999);
\draw(52.949999999999996, -113.5) node[anchor=north west,align=left] {Finite-typedomains};
\draw (52.949999999999996, -113.5) rectangle (58.05,-115.1);
\draw(58.14999999999999, -113.5) node[anchor=north west,align=left] {Domains\\ of \\ holomorphy};
\draw (58.14999999999999, -113.5) rectangle (61.24999999999999,-115.1);
\draw(52.949999999999996, -115.2) node[anchor=north west,align=left] {Exhaustion\\ functions};
\draw (52.949999999999996, -115.2) rectangle (56.05,-116.8);
\draw(56.14999999999999, -115.2) node[anchor=north west,align=left] {Peak \\ functions};
\draw (56.14999999999999, -115.2) rectangle (58.99999999999999,-116.3);
\draw(59.099999999999994, -115.2) node[anchor=north west,align=left] {Worm \\ domains};
\draw (59.099999999999994, -115.2) rectangle (61.449999999999996,-116.3);
\draw(2, -128.5) node[anchor=north west,align=left] {\large Compact analytic spaces};
\draw (2, -128.5) rectangle (11.899999999999999,-138.8);
\draw(3, -129.5) node[anchor=north west,align=left] {Transcendental\\ methods of \\ algebraic geometry\\ (complex-analytic\\ aspects)};
\draw (3, -129.5) rectangle (8.1,-132.1);
\draw(8.2, -129.5) node[anchor=north west,align=left] {Applications\\ of compact\\ analytic\\ spaces to \\ the sciences};
\draw (8.2, -129.5) rectangle (11.799999999999999,-132.1);
\draw(3, -132.2) node[anchor=north west,align=left] {Compact \\ Kähler manifolds:\\ generalizations, \\ classification};
\draw (3, -132.2) rectangle (7.85,-134.79999999999998);
\draw(7.949999999999999, -132.2) node[anchor=north west,align=left] {Compact \\ complex \\ \(3\)-folds};
\draw (7.949999999999999, -132.2) rectangle (11.299999999999999,-133.79999999999998);
\draw(3, -134.9) node[anchor=north west,align=left] {Compactification\\ of \\ analytic spaces};
\draw (3, -134.9) rectangle (7.6,-137.0);
\draw(7.699999999999999, -134.9) node[anchor=north west,align=left] {Compact \\ complex \\ \(n\)-folds};
\draw (7.699999999999999, -134.9) rectangle (11.049999999999999,-136.5);
\draw(3, -137.1) node[anchor=north west,align=left] {Algebraic\\ dependence\\ theorems};
\draw (3, -137.1) rectangle (6.1,-138.7);
\draw(6.199999999999999, -137.1) node[anchor=north west,align=left] {Compact\\ complex\\ surfaces};
\draw (6.199999999999999, -137.1) rectangle (8.799999999999999,-138.7);
\draw(11.999999999999998, -128.5) node[anchor=north west,align=left] {\large Pluripotential theory};
\draw (11.999999999999998, -128.5) rectangle (21.4,-141.0);
\draw(12.999999999999998, -129.5) node[anchor=north west,align=left] {Plurisubharmonic\\ functions\\ and \\ generalizations};
\draw (12.999999999999998, -129.5) rectangle (17.599999999999998,-131.6);
\draw(17.699999999999996, -129.5) node[anchor=north west,align=left] {Currents};
\draw (17.699999999999996, -129.5) rectangle (20.299999999999997,-130.6);
\draw(12.999999999999998, -131.7) node[anchor=north west,align=left] {Plurisubharmonic\\ exhaustion\\ functions};
\draw (12.999999999999998, -131.7) rectangle (17.599999999999998,-133.79999999999998);
\draw(17.699999999999996, -131.7) node[anchor=north west,align=left] {Lelong\\ numbers};
\draw (17.699999999999996, -131.7) rectangle (20.049999999999997,-132.79999999999998);
\draw(12.999999999999998, -133.9) node[anchor=north west,align=left] {Plurisubharmonic\\ extremal\\ functions, \\ pluricomplex \\ Green functions};
\draw (12.999999999999998, -133.9) rectangle (17.599999999999998,-136.5);
\draw(12.999999999999998, -136.6) node[anchor=north west,align=left] {Capacity\\ theory\\ and \\ generalizations};
\draw (12.999999999999998, -136.6) rectangle (17.349999999999998,-138.7);
\draw(12.999999999999998, -138.8) node[anchor=north west,align=left] {General \\ pluripotential\\ theory};
\draw (12.999999999999998, -138.8) rectangle (17.099999999999998,-140.4);
\draw(17.199999999999996, -138.8) node[anchor=north west,align=left] {Removable\\ sets in\\ pluripotential\\ theory};
\draw (17.199999999999996, -138.8) rectangle (21.299999999999997,-140.9);
\draw(21.5, -128.5) node[anchor=north west,align=left] {\large Automorphic functions};
\draw (21.5, -128.5) rectangle (30.65,-134.4);
\draw(22.5, -129.5) node[anchor=north west,align=left] {General theory\\ of automorphic\\ functions\\ of several \\ complex variables};
\draw (22.5, -129.5) rectangle (27.35,-132.1);
\draw(22.5, -132.2) node[anchor=north west,align=left] {Automorphic\\ forms in \\ several complex\\ variables};
\draw (22.5, -132.2) rectangle (26.85,-134.29999999999998);
\draw(26.95, -132.2) node[anchor=north west,align=left] {Automorphic\\ functions\\ in symmetric\\ domains};
\draw (26.95, -132.2) rectangle (30.55,-134.29999999999998);
\draw(30.75, -128.5) node[anchor=north west,align=left] {\large CR manifolds};
\draw (30.75, -128.5) rectangle (39.9,-138.3);
\draw(31.75, -129.5) node[anchor=north west,align=left] {Extension of\\ functions and\\ other analytic\\ objects \\ from CR manifolds};
\draw (31.75, -129.5) rectangle (36.6,-132.1);
\draw(36.7, -129.5) node[anchor=north west,align=left] {Embeddings\\ of CR\\ manifolds};
\draw (36.7, -129.5) rectangle (39.800000000000004,-131.1);
\draw(31.75, -132.2) node[anchor=north west,align=left] {CR structures,\\ CR operators,\\ and \\ generalizations};
\draw (31.75, -132.2) rectangle (36.1,-134.29999999999998);
\draw(36.2, -132.2) node[anchor=north west,align=left] {CR manifolds\\ as \\ boundaries\\ of domains};
\draw (36.2, -132.2) rectangle (39.800000000000004,-134.29999999999998);
\draw(31.75, -134.4) node[anchor=north west,align=left] {Finite-type\\ conditions\\ on \\ CR manifolds};
\draw (31.75, -134.4) rectangle (35.35,-136.5);
\draw(35.45, -134.4) node[anchor=north west,align=left] {Real \\ submanifolds\\ in complex\\ manifolds};
\draw (35.45, -134.4) rectangle (39.050000000000004,-136.5);
\draw(31.75, -136.6) node[anchor=north west,align=left] {CR \\ functions};
\draw (31.75, -136.6) rectangle (34.6,-137.7);
\draw(34.7, -136.6) node[anchor=north west,align=left] {Analysis\\ on CR \\ manifolds};
\draw (34.7, -136.6) rectangle (37.550000000000004,-138.2);
\draw(40.0, -128.5) node[anchor=north west,align=left] {\large Analytic continuation};
\draw (40.0, -128.5) rectangle (48.9,-135.6);
\draw(41.0, -129.5) node[anchor=north west,align=left] {Removable \\ singularities in\\ several complex\\ variables};
\draw (41.0, -129.5) rectangle (45.6,-131.6);
\draw(45.7, -129.5) node[anchor=north west,align=left] {Domains\\ of \\ holomorphy};
\draw (45.7, -129.5) rectangle (48.800000000000004,-131.1);
\draw(41.0, -131.7) node[anchor=north west,align=left] {Continuation\\ of analytic\\ objects in\\ several complex\\ variables};
\draw (41.0, -131.7) rectangle (45.35,-134.29999999999998);
\draw(45.45, -131.7) node[anchor=north west,align=left] {Envelopes\\ of \\ holomorphy};
\draw (45.45, -131.7) rectangle (48.550000000000004,-133.29999999999998);
\draw(41.0, -134.4) node[anchor=north west,align=left] {Riemann\\ domains};
\draw (41.0, -134.4) rectangle (43.35,-135.5);
\draw(49.0, -128.5) node[anchor=north west,align=left] {\large Holomorphic fiber spaces};
\draw (49.0, -128.5) rectangle (57.9,-140.3);
\draw(50.0, -129.5) node[anchor=north west,align=left] {Twistor theory,\\ double \\ fibrations \\ (complex-analytic\\ aspects)};
\draw (50.0, -129.5) rectangle (54.85,-132.1);
\draw(54.95, -129.5) node[anchor=north west,align=left] {Bundle \\ convexity};
\draw (54.95, -129.5) rectangle (57.800000000000004,-130.6);
\draw(50.0, -132.2) node[anchor=north west,align=left] {Holomorphic\\ bundles\\ and \\ generalizations};
\draw (50.0, -132.2) rectangle (54.35,-134.29999999999998);
\draw(54.45, -132.2) node[anchor=north west,align=left] {Vanishing\\ theorems};
\draw (54.45, -132.2) rectangle (57.300000000000004,-133.79999999999998);
\draw(50.0, -134.4) node[anchor=north west,align=left] {Sheaves and \\ cohomology of\\ sections of \\ holomorphic \\ vector bundles,\\ general results};
\draw (50.0, -134.4) rectangle (54.35,-137.5);
\draw(50.0, -137.6) node[anchor=north west,align=left] {Applications\\ of holomorphic\\ fiber\\ spaces to\\ the sciences};
\draw (50.0, -137.6) rectangle (54.1,-140.2);
\draw(64.5, -77.6) node[anchor=north west,align=left] {\LARGE Associative rings and algebras};
\draw (64.5, -77.6) rectangle (121.30000000000001,-124.69999999999999);
\draw(65.5, -78.6) node[anchor=north west,align=left] {\large Chain conditions, growth conditions, and other forms of finiteness for associative rings and algebras};
\draw (65.5, -78.6) rectangle (98.95,-85.5);
\draw(66.5, -79.6) node[anchor=north west,align=left] {Chain conditions\\ on \\ annihilators and \\ summands: \\ Goldie-type conditions};
\draw (66.5, -79.6) rectangle (72.6,-82.19999999999999);
\draw(72.7, -79.6) node[anchor=north west,align=left] {Noetherian \\ rings and \\ modules (associative\\ rings\\ and algebras)};
\draw (72.7, -79.6) rectangle (78.3,-82.19999999999999);
\draw(78.4, -79.6) node[anchor=north west,align=left] {Chain conditions\\ on other classes\\ of submodules,\\ ideals, subrings,\\ etc.; coherence\\ (associative\\ rings and algebras)};
\draw (78.4, -79.6) rectangle (83.75,-83.19999999999999);
\draw(83.85, -79.6) node[anchor=north west,align=left] {Finite rings\\ and \\ finite-dimensional\\ associative\\ algebras};
\draw (83.85, -79.6) rectangle (88.94999999999999,-82.19999999999999);
\draw(89.05, -79.6) node[anchor=north west,align=left] {Artinian \\ rings and \\ modules \\ (associative rings\\ and algebras)};
\draw (89.05, -79.6) rectangle (94.14999999999999,-82.19999999999999);
\draw(94.25, -79.6) node[anchor=north west,align=left] {Localization\\ and \\ associative \\ Noetherian rings};
\draw (94.25, -79.6) rectangle (98.85,-81.69999999999999);
\draw(66.5, -83.3) node[anchor=north west,align=left] {Growth \\ rate, \\ Gelfand-Kirillov\\ dimension};
\draw (66.5, -83.3) rectangle (71.1,-85.39999999999999);
\draw(99.05000000000001, -78.6) node[anchor=north west,align=left] {\large Associative rings and algebras with additional structure};
\draw (99.05000000000001, -78.6) rectangle (119.80000000000001,-87.69999999999999);
\draw(100.05000000000001, -79.6) node[anchor=north west,align=left] {Actions of \\ groups and \\ semigroups; invariant\\ theory \\ (associative \\ rings and algebras)};
\draw (100.05000000000001, -79.6) rectangle (105.9,-82.69999999999999);
\draw(106.00000000000001, -79.6) node[anchor=north west,align=left] {Valuations, \\ completions, formal \\ power series and \\ related constructions\\ (associative \\ rings and algebras)};
\draw (106.00000000000001, -79.6) rectangle (111.85000000000001,-82.69999999999999);
\draw(111.95000000000002, -79.6) node[anchor=north west,align=left] {Filtered \\ associative \\ rings; filtrational\\ and \\ graded techniques};
\draw (111.95000000000002, -79.6) rectangle (117.30000000000001,-82.19999999999999);
\draw(100.05000000000001, -82.8) node[anchor=north west,align=left] {Rings with \\ involution; Lie,\\ Jordan and other\\ nonassociative\\ structures};
\draw (100.05000000000001, -82.8) rectangle (104.65,-85.39999999999999);
\draw(104.75000000000001, -82.8) node[anchor=north west,align=left] {Automorphisms\\ and \\ endomorphisms};
\draw (104.75000000000001, -82.8) rectangle (108.60000000000001,-84.39999999999999);
\draw(108.70000000000002, -82.8) node[anchor=north west,align=left] {Graded rings\\ and modules\\ (associative\\ rings\\ and algebras)};
\draw (108.70000000000002, -82.8) rectangle (112.55000000000001,-85.39999999999999);
\draw(112.65, -82.8) node[anchor=north west,align=left] {Derivations,\\ actions\\ of Lie\\ algebras};
\draw (112.65, -82.8) rectangle (116.25,-84.89999999999999);
\draw(116.35000000000001, -82.8) node[anchor=north west,align=left] {“Super” \\ (or “skew”)\\ structure};
\draw (116.35000000000001, -82.8) rectangle (119.7,-84.39999999999999);
\draw(100.05000000000001, -85.5) node[anchor=north west,align=left] {Topological\\ and ordered\\ rings\\ and modules};
\draw (100.05000000000001, -85.5) rectangle (103.4,-87.6);
\draw(65.5, -85.6) node[anchor=north west,align=left] {\large History of associative\\ rings and algebras};
\draw (65.5, -85.6) rectangle (72.92,-86.69999999999999);
\draw(65.5, -87.8) node[anchor=north west,align=left] {\large Associative rings and algebras arising under various constructions};
\draw (65.5, -87.8) rectangle (90.30000000000001,-98.6);
\draw(66.5, -88.8) node[anchor=north west,align=left] {Rings arising\\ from \\ noncommutative algebraic\\ geometry};
\draw (66.5, -88.8) rectangle (73.1,-90.89999999999999);
\draw(73.2, -88.8) node[anchor=north west,align=left] {Associative rings\\ determined by \\ universal properties \\ (free algebras, \\ coproducts, adjunction\\ of inverses, etc.)};
\draw (73.2, -88.8) rectangle (79.3,-91.89999999999999);
\draw(79.4, -88.8) node[anchor=north west,align=left] {Associative \\ rings of \\ functions, subdirect\\ products,\\ sheaves of rings};
\draw (79.4, -88.8) rectangle (85.0,-91.39999999999999);
\draw(85.1, -88.8) node[anchor=north west,align=left] {Finite generation,\\ finite \\ presentability,\\ normal forms \\ (diamond lemma,\\ term-rewriting)};
\draw (85.1, -88.8) rectangle (90.19999999999999,-91.89999999999999);
\draw(66.5, -92.0) node[anchor=north west,align=left] {Rings of \\ differential \\ operators \\ (associative \\ algebraic aspects)};
\draw (66.5, -92.0) rectangle (71.6,-94.6);
\draw(71.7, -92.0) node[anchor=north west,align=left] {Torsion theories;\\ radicals on\\ module categories\\ (associative\\ algebraic aspects)};
\draw (71.7, -92.0) rectangle (76.8,-94.6);
\draw(76.9, -92.0) node[anchor=north west,align=left] {Twisted and\\ skew group\\ rings, \\ crossed products};
\draw (76.9, -92.0) rectangle (81.5,-94.1);
\draw(81.6, -92.0) node[anchor=north west,align=left] {Ordinary \\ and skew \\ polynomial \\ rings and \\ semigroup rings};
\draw (81.6, -92.0) rectangle (85.94999999999999,-94.6);
\draw(86.05, -92.0) node[anchor=north west,align=left] {Extensions\\ of associative\\ rings\\ by ideals};
\draw (86.05, -92.0) rectangle (90.14999999999999,-94.1);
\draw(66.5, -94.7) node[anchor=north west,align=left] {Associative\\ rings of \\ fractions and \\ localizations};
\draw (66.5, -94.7) rectangle (70.6,-96.8);
\draw(70.7, -94.7) node[anchor=north west,align=left] {Centralizing\\ and \\ normalizing\\ extensions};
\draw (70.7, -94.7) rectangle (74.3,-96.8);
\draw(74.4, -94.7) node[anchor=north west,align=left] {Universal \\ enveloping \\ algebras of\\ Lie algebras};
\draw (74.4, -94.7) rectangle (78.0,-96.8);
\draw(78.1, -94.7) node[anchor=north west,align=left] {Smash \\ products of\\ general \\ Hopf actions};
\draw (78.1, -94.7) rectangle (81.69999999999999,-96.8);
\draw(81.8, -94.7) node[anchor=north west,align=left] {Endomorphism\\ rings;\\ matrix rings};
\draw (81.8, -94.7) rectangle (85.39999999999999,-96.3);
\draw(85.5, -94.7) node[anchor=north west,align=left] {Deformations\\ of\\ associative\\ rings};
\draw (85.5, -94.7) rectangle (89.1,-96.8);
\draw(66.5, -96.9) node[anchor=north west,align=left] {Quadratic\\ and Koszul\\ algebras};
\draw (66.5, -96.9) rectangle (69.6,-98.5);
\draw(69.7, -96.9) node[anchor=north west,align=left] {Leavitt\\ path \\ algebras};
\draw (69.7, -96.9) rectangle (72.3,-98.5);
\draw(72.4, -96.9) node[anchor=north west,align=left] {Group\\ rings};
\draw (72.4, -96.9) rectangle (74.25,-98.0);
\draw(90.4, -87.8) node[anchor=north west,align=left] {\large Radicals and radical properties of associative rings};
\draw (90.4, -87.8) rectangle (109.7,-91.5);
\draw(91.4, -88.8) node[anchor=north west,align=left] {Jacobson\\ radical,\\ quasimultiplication};
\draw (91.4, -88.8) rectangle (96.75,-90.89999999999999);
\draw(96.85000000000001, -88.8) node[anchor=north west,align=left] {Nil and \\ nilpotent \\ radicals, sets,\\ ideals, \\ associative rings};
\draw (96.85000000000001, -88.8) rectangle (101.7,-91.39999999999999);
\draw(101.80000000000001, -88.8) node[anchor=north west,align=left] {General \\ radicals \\ and associative\\ rings};
\draw (101.80000000000001, -88.8) rectangle (106.15,-90.89999999999999);
\draw(106.25, -88.8) node[anchor=north west,align=left] {Prime and\\ semiprime\\ associative\\ rings};
\draw (106.25, -88.8) rectangle (109.6,-90.89999999999999);
\draw(90.4, -91.6) node[anchor=north west,align=left] {\large Associative algebras and orders};
\draw (90.4, -91.6) rectangle (102.05000000000001,-97.0);
\draw(91.4, -92.6) node[anchor=north west,align=left] {Separable \\ algebras (e.g., \\ quaternion algebras,\\ Azumaya \\ algebras, etc.)};
\draw (91.4, -92.6) rectangle (97.0,-95.19999999999999);
\draw(97.10000000000001, -92.6) node[anchor=north west,align=left] {Commutativeorders};
\draw (97.10000000000001, -92.6) rectangle (101.95,-94.19999999999999);
\draw(91.4, -95.3) node[anchor=north west,align=left] {Orders in\\ separable\\ algebras};
\draw (91.4, -95.3) rectangle (94.25,-96.89999999999999);
\draw(94.35000000000001, -95.3) node[anchor=north west,align=left] {Lattices\\ over\\ orders};
\draw (94.35000000000001, -95.3) rectangle (96.95,-96.89999999999999);
\draw(109.80000000000001, -87.8) node[anchor=north west,align=left] {\large Rings with polynomial identity};
\draw (109.80000000000001, -87.8) rectangle (121.20000000000002,-96.89999999999999);
\draw(110.80000000000001, -88.8) node[anchor=north west,align=left] {Other kinds of\\ identities \\ (generalized \\ polynomial, rational,\\ involution)};
\draw (110.80000000000001, -88.8) rectangle (116.65,-91.39999999999999);
\draw(116.75000000000001, -88.8) node[anchor=north west,align=left] {Functional\\ identities\\ (associative\\ rings\\ and algebras)};
\draw (116.75000000000001, -88.8) rectangle (120.60000000000001,-91.39999999999999);
\draw(110.80000000000001, -91.5) node[anchor=north west,align=left] {\(T\)-ideals,\\ identities,\\ varieties of\\ associative \\ rings and algebras};
\draw (110.80000000000001, -91.5) rectangle (115.9,-94.1);
\draw(116.00000000000001, -91.5) node[anchor=north west,align=left] {Trace rings \\ and invariant\\ theory \\ (associative rings\\ and algebras)};
\draw (116.00000000000001, -91.5) rectangle (121.10000000000001,-94.1);
\draw(110.80000000000001, -94.2) node[anchor=north west,align=left] {Semiprime p.i.\\ rings, rings\\ embeddable in\\ matrices over\\ commutative rings};
\draw (110.80000000000001, -94.2) rectangle (115.65,-96.8);
\draw(115.75000000000001, -94.2) node[anchor=north west,align=left] {Identities \\ other than \\ those of matrices\\ over \\ commutative rings};
\draw (115.75000000000001, -94.2) rectangle (120.60000000000001,-96.8);
\draw(90.4, -91.6) node[anchor=north west,align=left] {\large Division rings and semisimple Artin rings};
\draw (90.4, -91.6) rectangle (105.5,-94.8);
\draw(91.4, -92.6) node[anchor=north west,align=left] {Infinite-dimensional\\ and general\\ division rings};
\draw (91.4, -92.6) rectangle (97.0,-94.69999999999999);
\draw(97.10000000000001, -92.6) node[anchor=north west,align=left] {Finite-dimensional\\ division\\ rings};
\draw (97.10000000000001, -92.6) rectangle (102.2,-94.69999999999999);
\draw(102.30000000000001, -92.6) node[anchor=north west,align=left] {Brauer \\ groups \\ (algebraic\\ aspects)};
\draw (102.30000000000001, -92.6) rectangle (105.4,-94.69999999999999);
\draw(65.5, -98.69999999999999) node[anchor=north west,align=left] {\large Modules, bimodules and ideals in associative algebras};
\draw (65.5, -98.69999999999999) rectangle (84.6,-109.29999999999998);
\draw(66.5, -99.69999999999999) node[anchor=north west,align=left] {Structure and \\ classification for \\ modules, bimodules and\\ ideals (except \\ as in 16Gxx), direct\\ sum decomposition\\ and cancellation\\ in associative algebras)};
\draw (66.5, -99.69999999999999) rectangle (73.1,-103.79999999999998);
\draw(73.2, -99.69999999999999) node[anchor=north west,align=left] {Infinite-dimensional\\ simple \\ rings (except\\ as in 16Kxx)};
\draw (73.2, -99.69999999999999) rectangle (78.8,-102.29999999999998);
\draw(78.9, -99.69999999999999) node[anchor=north west,align=left] {Free, projective,\\ and flat\\ modules and\\ ideals in \\ associative algebras};
\draw (78.9, -99.69999999999999) rectangle (84.5,-102.29999999999998);
\draw(66.5, -103.89999999999999) node[anchor=north west,align=left] {Simple and \\ semisimple \\ modules, primitive\\ rings and \\ ideals in \\ associative algebras};
\draw (66.5, -103.89999999999999) rectangle (72.1,-106.99999999999999);
\draw(72.2, -103.89999999999999) node[anchor=north west,align=left] {Other classes\\ of modules\\ and ideals\\ in \\ associative algebras};
\draw (72.2, -103.89999999999999) rectangle (77.8,-106.49999999999999);
\draw(77.9, -103.89999999999999) node[anchor=north west,align=left] {Injective \\ modules, \\ self-injective \\ associative rings};
\draw (77.9, -103.89999999999999) rectangle (82.75,-105.99999999999999);
\draw(66.5, -107.1) node[anchor=north west,align=left] {General \\ module theory\\ in associative\\ algebras};
\draw (66.5, -107.1) rectangle (70.6,-109.19999999999999);
\draw(70.7, -107.1) node[anchor=north west,align=left] {Module \\ categories in\\ associative\\ algebras};
\draw (70.7, -107.1) rectangle (74.55,-109.19999999999999);
\draw(74.65, -107.1) node[anchor=north west,align=left] {Bimodules\\ in \\ associative\\ algebras};
\draw (74.65, -107.1) rectangle (78.0,-109.19999999999999);
\draw(78.1, -107.1) node[anchor=north west,align=left] {Ideals in\\ associative\\ algebras};
\draw (78.1, -107.1) rectangle (81.44999999999999,-108.69999999999999);
\draw(84.7, -98.69999999999999) node[anchor=north west,align=left] {\large Representation theory of associative rings and algebras};
\draw (84.7, -98.69999999999999) rectangle (103.05000000000001,-104.6);
\draw(85.7, -99.69999999999999) node[anchor=north west,align=left] {Auslander-Reiten\\ sequences (almost\\ split sequences)\\ and \\ Auslander-Reiten quivers};
\draw (85.7, -99.69999999999999) rectangle (92.3,-102.29999999999998);
\draw(92.4, -99.69999999999999) node[anchor=north west,align=left] {Representation\\ type (finite,\\ tame, wild, \\ etc.) of associative\\ algebras};
\draw (92.4, -99.69999999999999) rectangle (98.0,-102.29999999999998);
\draw(98.1, -99.69999999999999) node[anchor=north west,align=left] {Representations\\ of orders,\\ lattices, \\ algebras over \\ commutative rings};
\draw (98.1, -99.69999999999999) rectangle (102.94999999999999,-102.29999999999998);
\draw(85.7, -102.39999999999999) node[anchor=north west,align=left] {Representations\\ of \\ associative \\ Artinian rings};
\draw (85.7, -102.39999999999999) rectangle (90.05,-104.49999999999999);
\draw(90.15, -102.39999999999999) node[anchor=north west,align=left] {Representations\\ of quivers\\ and partially\\ ordered sets};
\draw (90.15, -102.39999999999999) rectangle (94.5,-104.49999999999999);
\draw(94.6, -102.39999999999999) node[anchor=north west,align=left] {Cohen-Macaulay\\ modules\\ in associative\\ algebras};
\draw (94.6, -102.39999999999999) rectangle (98.69999999999999,-104.49999999999999);
\draw(84.7, -104.69999999999999) node[anchor=north west,align=left] {\large Computational aspects of associative rings};
\draw (84.7, -104.69999999999999) rectangle (98.32000000000001,-108.39999999999999);
\draw(85.7, -105.69999999999999) node[anchor=north west,align=left] {Computational\\ aspects\\ of associative\\ rings \\ (general theory)};
\draw (85.7, -105.69999999999999) rectangle (90.3,-108.29999999999998);
\draw(90.4, -105.69999999999999) node[anchor=north west,align=left] {Gröbner-Shirshov\\ bases};
\draw (90.4, -105.69999999999999) rectangle (95.0,-107.29999999999998);
\draw(103.15, -98.69999999999999) node[anchor=north west,align=left] {\large Homological methods in associative algebras};
\draw (103.15, -98.69999999999999) rectangle (119.5,-111.49999999999999);
\draw(104.15, -99.69999999999999) node[anchor=north west,align=left] {Homological conditions\\ on associative\\ rings (generalizations\\ of regular,\\ Gorenstein, \\ Cohen-Macaulay rings, etc.)};
\draw (104.15, -99.69999999999999) rectangle (111.5,-102.79999999999998);
\draw(111.60000000000001, -99.69999999999999) node[anchor=north west,align=left] {Homological \\ functors on modules\\ (Tor, Ext,\\ etc.) in \\ associative algebras};
\draw (111.60000000000001, -99.69999999999999) rectangle (117.2,-102.29999999999998);
\draw(104.15, -102.89999999999999) node[anchor=north west,align=left] {(Co)homology \\ of rings and \\ associative \\ algebras (e.g., \\ Hochschild, cyclic,\\ dihedral, etc.)};
\draw (104.15, -102.89999999999999) rectangle (109.5,-105.99999999999999);
\draw(109.60000000000001, -102.89999999999999) node[anchor=north west,align=left] {Grothendieck\\ groups,\\ \(K\)-theory, etc.};
\draw (109.60000000000001, -102.89999999999999) rectangle (114.7,-104.99999999999999);
\draw(114.80000000000001, -102.89999999999999) node[anchor=north west,align=left] {Derived \\ categories \\ and associative\\ algebras};
\draw (114.80000000000001, -102.89999999999999) rectangle (119.15,-104.99999999999999);
\draw(104.15, -106.1) node[anchor=north west,align=left] {Differential \\ graded algebras \\ and applications\\ (associative \\ algebraic aspects)};
\draw (104.15, -106.1) rectangle (109.25,-108.69999999999999);
\draw(109.35000000000001, -106.1) node[anchor=north west,align=left] {von Neumann \\ regular rings and\\ generalizations\\ (associative \\ algebraic aspects)};
\draw (109.35000000000001, -106.1) rectangle (114.45,-108.69999999999999);
\draw(114.55000000000001, -106.1) node[anchor=north west,align=left] {Semihereditary\\ and hereditary\\ rings, free ideal\\ rings, Sylvester\\ rings, etc.};
\draw (114.55000000000001, -106.1) rectangle (119.4,-108.69999999999999);
\draw(104.15, -108.79999999999998) node[anchor=north west,align=left] {Syzygies, \\ resolutions,\\ complexes\\ in associative\\ algebras};
\draw (104.15, -108.79999999999998) rectangle (108.25,-111.39999999999998);
\draw(108.35000000000001, -108.79999999999998) node[anchor=north west,align=left] {Homological\\ dimension\\ in associative\\ algebras};
\draw (108.35000000000001, -108.79999999999998) rectangle (112.45,-110.89999999999998);
\draw(84.7, -104.69999999999999) node[anchor=north west,align=left] {\large Hopf algebras, quantum groups and related topics};
\draw (84.7, -104.69999999999999) rectangle (101.0,-110.1);
\draw(85.7, -105.69999999999999) node[anchor=north west,align=left] {Ring-theoretic\\ aspects\\ of quantum\\ groups};
\draw (85.7, -105.69999999999999) rectangle (89.8,-107.79999999999998);
\draw(89.9, -105.69999999999999) node[anchor=north west,align=left] {Connections\\ of Hopf \\ algebras with\\ combinatorics};
\draw (89.9, -105.69999999999999) rectangle (93.75,-107.79999999999998);
\draw(93.85000000000001, -105.69999999999999) node[anchor=north west,align=left] {Hopf \\ algebras and\\ their \\ applications};
\draw (93.85000000000001, -105.69999999999999) rectangle (97.45,-107.79999999999998);
\draw(97.55, -105.69999999999999) node[anchor=north west,align=left] {Yang-Baxter\\ equations};
\draw (97.55, -105.69999999999999) rectangle (100.89999999999999,-107.29999999999998);
\draw(85.7, -107.89999999999999) node[anchor=north west,align=left] {Bialgebras};
\draw (85.7, -107.89999999999999) rectangle (88.8,-108.99999999999999);
\draw(88.9, -107.89999999999999) node[anchor=north west,align=left] {Coalgebras\\ and \\ comodules;\\ corings};
\draw (88.9, -107.89999999999999) rectangle (92.0,-109.99999999999999);
\draw(65.5, -111.6) node[anchor=north west,align=left] {\large Conditions on elements};
\draw (65.5, -111.6) rectangle (76.4,-124.6);
\draw(66.5, -112.6) node[anchor=north west,align=left] {Integral \\ domains (associative\\ rings\\ and algebras)};
\draw (66.5, -112.6) rectangle (72.1,-114.69999999999999);
\draw(72.2, -112.6) node[anchor=north west,align=left] {Ore rings, \\ multiplicative\\ sets, Ore\\ localization};
\draw (72.2, -112.6) rectangle (76.3,-114.69999999999999);
\draw(66.5, -114.8) node[anchor=north west,align=left] {Idempotent \\ elements \\ (associative rings\\ and algebras)};
\draw (66.5, -114.8) rectangle (71.6,-116.89999999999999);
\draw(71.7, -114.8) node[anchor=north west,align=left] {Divisibility,\\ noncommutative\\ UFDs};
\draw (71.7, -114.8) rectangle (75.8,-116.89999999999999);
\draw(66.5, -117.0) node[anchor=north west,align=left] {Center, normalizer\\ (invariant\\ elements) \\ (associative rings\\ and algebras)};
\draw (66.5, -117.0) rectangle (71.6,-119.6);
\draw(71.7, -117.0) node[anchor=north west,align=left] {Units, groups\\ of units\\ (associative\\ rings and\\ algebras)};
\draw (71.7, -117.0) rectangle (75.55,-119.6);
\draw(66.5, -119.69999999999999) node[anchor=north west,align=left] {Generalizations\\ of commutativity\\ (associative rings\\ and algebras)};
\draw (66.5, -119.69999999999999) rectangle (71.6,-122.29999999999998);
\draw(66.5, -122.39999999999999) node[anchor=north west,align=left] {Generalized \\ inverses \\ (associative rings\\ and algebras)};
\draw (66.5, -122.39999999999999) rectangle (71.6,-124.49999999999999);
\draw(76.5, -111.6) node[anchor=north west,align=left] {\large Local rings and generalizations};
\draw (76.5, -111.6) rectangle (86.71,-115.3);
\draw(77.5, -112.6) node[anchor=north west,align=left] {Quasi-Frobenius\\ rings};
\draw (77.5, -112.6) rectangle (81.85,-114.19999999999999);
\draw(81.95, -112.6) node[anchor=north west,align=left] {Noncommutative\\ local \\ and semilocal\\ rings, \\ perfect rings};
\draw (81.95, -112.6) rectangle (86.05,-115.19999999999999);
\draw(76.5, -115.4) node[anchor=north west,align=left] {\large General and miscellaneous};
\draw (76.5, -115.4) rectangle (84.85,-121.80000000000001);
\draw(77.5, -116.4) node[anchor=north west,align=left] {Category-theoretic\\ methods\\ and results\\ in associative\\ algebras \\ (except as in 16D90)};
\draw (77.5, -116.4) rectangle (83.1,-119.5);
\draw(77.5, -119.60000000000001) node[anchor=north west,align=left] {Applications\\ of logic\\ in associative\\ algebras};
\draw (77.5, -119.60000000000001) rectangle (81.6,-121.7);
\draw(86.81, -111.6) node[anchor=north west,align=left] {\large Generalizations};
\draw (86.81, -111.6) rectangle (92.06,-117.89999999999999);
\draw(87.81, -112.6) node[anchor=north west,align=left] {Hyperrings};
\draw (87.81, -112.6) rectangle (90.91,-113.69999999999999);
\draw(87.81, -113.8) node[anchor=north west,align=left] {Near-rings};
\draw (87.81, -113.8) rectangle (90.91,-114.89999999999999);
\draw(87.81, -115.0) node[anchor=north west,align=left] {\(\Gamma\)\\ and fuzzy\\ structures};
\draw (87.81, -115.0) rectangle (90.91,-116.6);
\draw(87.81, -116.69999999999999) node[anchor=north west,align=left] {Semirings};
\draw (87.81, -116.69999999999999) rectangle (90.66,-117.79999999999998);
\draw(121.4, -77.6) node[anchor=north west,align=left] {\LARGE History and biography};
\draw (121.4, -77.6) rectangle (138.1,-103.19999999999999);
\draw(122.4, -78.6) node[anchor=north west,align=left] {\large History of mathematics and mathematicians};
\draw (122.4, -78.6) rectangle (138.0,-103.1);
\draw(123.4, -79.6) node[anchor=north west,align=left] {Bibliographicstudies};
\draw (123.4, -79.6) rectangle (129.0,-81.19999999999999);
\draw(129.1, -79.6) node[anchor=north west,align=left] {History of \\ mathematics of the\\ indigenous \\ cultures of Africa,\\ Asia, and Oceania};
\draw (129.1, -79.6) rectangle (134.45,-82.19999999999999);
\draw(134.55, -79.6) node[anchor=north west,align=left] {History of\\ mathematics\\ in Ancient\\ Babylon};
\draw (134.55, -79.6) rectangle (137.9,-81.69999999999999);
\draw(123.4, -82.3) node[anchor=north west,align=left] {History of \\ mathematics of the\\ indigenous \\ cultures of Europe\\ (pre-Greek, etc.)};
\draw (123.4, -82.3) rectangle (128.5,-84.89999999999999);
\draw(128.6, -82.3) node[anchor=north west,align=left] {Ethnomathematics,\\ general};
\draw (128.6, -82.3) rectangle (133.45,-83.89999999999999);
\draw(133.55, -82.3) node[anchor=north west,align=left] {History of\\ mathematics\\ in Paleolithic\\ and \\ Neolithic times};
\draw (133.55, -82.3) rectangle (137.9,-84.89999999999999);
\draw(123.4, -85.0) node[anchor=north west,align=left] {History of \\ mathematics \\ in late antiquity\\ and \\ medieval Europe};
\draw (123.4, -85.0) rectangle (128.25,-87.6);
\draw(128.35, -85.0) node[anchor=north west,align=left] {History of \\ mathematics at\\ institutions\\ and academies\\ (non-university)};
\draw (128.35, -85.0) rectangle (132.95,-87.6);
\draw(133.05, -85.0) node[anchor=north west,align=left] {History of \\ mathematics \\ in Ancient \\ Greece and Rome};
\draw (133.05, -85.0) rectangle (137.4,-87.1);
\draw(123.4, -87.69999999999999) node[anchor=north west,align=left] {History of \\ mathematics in\\ the 15th and\\ 16th centuries,\\ Renaissance};
\draw (123.4, -87.69999999999999) rectangle (127.75,-90.29999999999998);
\draw(127.85000000000001, -87.69999999999999) node[anchor=north west,align=left] {Collected or\\ selected works;\\ reprintings\\ or translations\\ of classics};
\draw (127.85000000000001, -87.69999999999999) rectangle (132.20000000000002,-90.29999999999998);
\draw(132.3, -87.69999999999999) node[anchor=north west,align=left] {History of \\ mathematics of\\ the indigenous\\ cultures of\\ the Americas};
\draw (132.3, -87.69999999999999) rectangle (136.4,-90.29999999999998);
\draw(123.4, -90.39999999999999) node[anchor=north west,align=left] {History \\ of mathematics\\ in \\ Ancient Egypt};
\draw (123.4, -90.39999999999999) rectangle (127.5,-92.49999999999999);
\draw(127.60000000000001, -90.39999999999999) node[anchor=north west,align=left] {History of\\ mathematics\\ in \\ Southeast Asia};
\draw (127.60000000000001, -90.39999999999999) rectangle (131.70000000000002,-92.49999999999999);
\draw(131.8, -90.39999999999999) node[anchor=north west,align=left] {Biographies,\\ obituaries,\\ personalia,\\ bibliographies};
\draw (131.8, -90.39999999999999) rectangle (135.9,-92.49999999999999);
\draw(123.4, -92.6) node[anchor=north west,align=left] {Sociology\\ (and \\ profession) of\\ mathematics};
\draw (123.4, -92.6) rectangle (127.5,-94.69999999999999);
\draw(127.60000000000001, -92.6) node[anchor=north west,align=left] {Historiography};
\draw (127.60000000000001, -92.6) rectangle (131.70000000000002,-93.69999999999999);
\draw(131.8, -92.6) node[anchor=north west,align=left] {History of \\ mathematics \\ in the Golden\\ Age of Islam};
\draw (131.8, -92.6) rectangle (135.65,-94.69999999999999);
\draw(123.4, -94.8) node[anchor=north west,align=left] {General \\ histories, \\ source books};
\draw (123.4, -94.8) rectangle (127.0,-96.39999999999999);
\draw(127.10000000000001, -94.8) node[anchor=north west,align=left] {History of\\ mathematics\\ in the\\ 17th century};
\draw (127.10000000000001, -94.8) rectangle (130.70000000000002,-96.89999999999999);
\draw(130.8, -94.8) node[anchor=north west,align=left] {History of\\ mathematics\\ in the\\ 18th century};
\draw (130.8, -94.8) rectangle (134.4,-96.89999999999999);
\draw(134.5, -94.8) node[anchor=north west,align=left] {History of\\ mathematics\\ in China};
\draw (134.5, -94.8) rectangle (137.85,-96.39999999999999);
\draw(123.4, -97.0) node[anchor=north west,align=left] {History of\\ mathematics\\ in the\\ 19th century};
\draw (123.4, -97.0) rectangle (127.0,-99.1);
\draw(127.10000000000001, -97.0) node[anchor=north west,align=left] {History of\\ mathematics\\ in the\\ 20th century};
\draw (127.10000000000001, -97.0) rectangle (130.70000000000002,-99.1);
\draw(130.8, -97.0) node[anchor=north west,align=left] {History of\\ mathematics\\ in the\\ 21st century};
\draw (130.8, -97.0) rectangle (134.4,-99.1);
\draw(134.5, -97.0) node[anchor=north west,align=left] {History of\\ mathematics\\ in Japan};
\draw (134.5, -97.0) rectangle (137.85,-98.6);
\draw(123.4, -99.19999999999999) node[anchor=north west,align=left] {Development\\ of \\ contemporary\\ mathematics};
\draw (123.4, -99.19999999999999) rectangle (127.0,-101.29999999999998);
\draw(127.10000000000001, -99.19999999999999) node[anchor=north west,align=left] {Future \\ perspectives\\ in \\ mathematics};
\draw (127.10000000000001, -99.19999999999999) rectangle (130.70000000000002,-101.29999999999998);
\draw(130.8, -99.19999999999999) node[anchor=north west,align=left] {History of\\ mathematics\\ at specific\\ universities};
\draw (130.8, -99.19999999999999) rectangle (134.4,-101.29999999999998);
\draw(134.5, -99.19999999999999) node[anchor=north west,align=left] {History of\\ mathematics\\ in India};
\draw (134.5, -99.19999999999999) rectangle (137.85,-100.79999999999998);
\draw(123.4, -101.39999999999999) node[anchor=north west,align=left] {Schools\\ of \\ mathematics};
\draw (123.4, -101.39999999999999) rectangle (126.75,-102.99999999999999);
\draw(138.20000000000002, -1) node[anchor=north west,align=left] {\LARGE Functions of a complex variable};
\draw (138.20000000000002, -1) rectangle (187.85000000000002,-50.6);
\draw(139.20000000000002, -2) node[anchor=north west,align=left] {\large Entire and meromorphic functions of one complex variable, and related topics};
\draw (139.20000000000002, -2) rectangle (167.20000000000002,-9.4);
\draw(140.20000000000002, -3) node[anchor=north west,align=left] {Functional equations\\ in the complex\\ plane, iteration\\ and composition\\ of analytic\\ functions of one\\ complex variable};
\draw (140.20000000000002, -3) rectangle (145.8,-6.6);
\draw(145.9, -3) node[anchor=north west,align=left] {Representations\\ of entire\\ functions of\\ one complex \\ variable by \\ series and integrals};
\draw (145.9, -3) rectangle (151.5,-6.1);
\draw(151.60000000000002, -3) node[anchor=north west,align=left] {Special classes\\ of entire functions\\ of one complex\\ variable and\\ growth estimates};
\draw (151.60000000000002, -3) rectangle (156.95000000000002,-5.6);
\draw(157.05, -3) node[anchor=north west,align=left] {Value distribution\\ of meromorphic\\ functions\\ of one complex\\ variable, \\ Nevanlinna theory};
\draw (157.05, -3) rectangle (162.15,-6.1);
\draw(162.25000000000003, -3) node[anchor=north west,align=left] {Cluster \\ sets, prime\\ ends, \\ boundary behavior};
\draw (162.25000000000003, -3) rectangle (167.10000000000002,-5.1);
\draw(140.20000000000002, -6.7) node[anchor=north west,align=left] {Entire \\ functions of one\\ complex \\ variable, \\ general theory};
\draw (140.20000000000002, -6.7) rectangle (144.8,-9.3);
\draw(144.9, -6.7) node[anchor=north west,align=left] {Normal \\ functions of one\\ complex \\ variable, \\ normal families};
\draw (144.9, -6.7) rectangle (149.5,-9.3);
\draw(149.60000000000002, -6.7) node[anchor=north west,align=left] {Quasi-analytic\\ and other \\ classes of \\ functions of one\\ complex variable};
\draw (149.60000000000002, -6.7) rectangle (154.20000000000002,-9.3);
\draw(154.3, -6.7) node[anchor=north west,align=left] {Meromorphic\\ functions of\\ one complex\\ variable, \\ general theory};
\draw (154.3, -6.7) rectangle (158.4,-9.3);
\draw(167.3, -2) node[anchor=north west,align=left] {\large Series expansions of functions of one complex variable};
\draw (167.3, -2) rectangle (187.10000000000002,-8.4);
\draw(168.3, -3) node[anchor=north west,align=left] {Boundary behavior\\ of power \\ series in one complex\\ variable; \\ over-convergence};
\draw (168.3, -3) rectangle (174.15,-5.6);
\draw(174.25, -3) node[anchor=north west,align=left] {Completeness \\ problems, closure\\ of a system of\\ functions of \\ one complex variable};
\draw (174.25, -3) rectangle (179.85,-5.6);
\draw(179.95000000000002, -3) node[anchor=north west,align=left] {Dirichlet series,\\ exponential\\ series and other\\ series in one\\ complex variable};
\draw (179.95000000000002, -3) rectangle (184.8,-5.6);
\draw(168.3, -5.7) node[anchor=north west,align=left] {Power series\\ (including\\ lacunary \\ series) in one\\ complex variable};
\draw (168.3, -5.7) rectangle (172.9,-8.3);
\draw(173.0, -5.7) node[anchor=north west,align=left] {Random power\\ series\\ in one \\ complex variable};
\draw (173.0, -5.7) rectangle (177.6,-7.800000000000001);
\draw(177.70000000000002, -5.7) node[anchor=north west,align=left] {Analytic \\ continuation\\ of functions\\ of one \\ complex variable};
\draw (177.70000000000002, -5.7) rectangle (182.3,-8.3);
\draw(182.4, -5.7) node[anchor=north west,align=left] {Continued \\ fractions; \\ complex-analytic\\ aspects};
\draw (182.4, -5.7) rectangle (187.0,-7.800000000000001);
\draw(139.20000000000002, -9.5) node[anchor=north west,align=left] {\large Spaces and algebras of analytic functions of one complex variable};
\draw (139.20000000000002, -9.5) rectangle (161.70000000000002,-14.9);
\draw(140.20000000000002, -10.5) node[anchor=north west,align=left] {Spaces of \\ bounded analytic\\ functions\\ of one complex\\ variable};
\draw (140.20000000000002, -10.5) rectangle (144.8,-13.1);
\draw(144.9, -10.5) node[anchor=north west,align=left] {Algebras \\ of analytic\\ functions\\ of one \\ complex variable};
\draw (144.9, -10.5) rectangle (149.5,-13.1);
\draw(149.60000000000002, -10.5) node[anchor=north west,align=left] {de \\ Branges-Rovnyak\\ spaces};
\draw (149.60000000000002, -10.5) rectangle (153.95000000000002,-12.1);
\draw(154.05, -10.5) node[anchor=north west,align=left] {Besov spaces\\ and \\ \(Q_p\)-spaces};
\draw (154.05, -10.5) rectangle (158.15,-12.1);
\draw(158.25, -10.5) node[anchor=north west,align=left] {Nevanlinna\\ spaces\\ and Smirnov\\ spaces};
\draw (158.25, -10.5) rectangle (161.6,-12.6);
\draw(140.20000000000002, -13.2) node[anchor=north west,align=left] {Bergman \\ spaces and \\ Fock spaces};
\draw (140.20000000000002, -13.2) rectangle (143.55,-14.799999999999999);
\draw(143.65, -13.2) node[anchor=north west,align=left] {BMO-spaces};
\draw (143.65, -13.2) rectangle (146.75,-14.299999999999999);
\draw(146.85000000000002, -13.2) node[anchor=north west,align=left] {Corona\\ theorems};
\draw (146.85000000000002, -13.2) rectangle (149.45000000000002,-14.299999999999999);
\draw(149.55, -13.2) node[anchor=north west,align=left] {Zygmund\\ spaces};
\draw (149.55, -13.2) rectangle (151.9,-14.299999999999999);
\draw(152.00000000000003, -13.2) node[anchor=north west,align=left] {Hardy\\ spaces};
\draw (152.00000000000003, -13.2) rectangle (154.10000000000002,-14.299999999999999);
\draw(154.20000000000002, -13.2) node[anchor=north west,align=left] {Bloch\\ spaces};
\draw (154.20000000000002, -13.2) rectangle (156.3,-14.299999999999999);
\draw(161.8, -9.5) node[anchor=north west,align=left] {\large Miscellaneous topics of analysis in the complex plane};
\draw (161.8, -9.5) rectangle (181.60000000000002,-15.9);
\draw(162.8, -10.5) node[anchor=north west,align=left] {Integration, \\ integrals of Cauchy \\ type, integral \\ representations of \\ analytic functions\\ in the complex plane};
\draw (162.8, -10.5) rectangle (168.4,-13.6);
\draw(168.5, -10.5) node[anchor=north west,align=left] {Moment problems\\ and \\ interpolation \\ problems in the\\ complex plane};
\draw (168.5, -10.5) rectangle (172.85,-13.1);
\draw(172.95000000000002, -10.5) node[anchor=north west,align=left] {Asymptotic\\ representations\\ in the\\ complex plane};
\draw (172.95000000000002, -10.5) rectangle (177.3,-12.6);
\draw(177.4, -10.5) node[anchor=north west,align=left] {Boundary \\ value problems\\ in the \\ complex plane};
\draw (177.4, -10.5) rectangle (181.5,-12.6);
\draw(162.8, -13.7) node[anchor=north west,align=left] {Approximation\\ in\\ the \\ complex plane};
\draw (162.8, -13.7) rectangle (166.65,-15.799999999999999);
\draw(181.70000000000002, -9.5) node[anchor=north west,align=left] {\large History of \\ functions of a \\ complex variable};
\draw (181.70000000000002, -9.5) rectangle (187.26000000000002,-11.1);
\draw(139.20000000000002, -16.0) node[anchor=north west,align=left] {\large Universal holomorphic functions of one complex variable};
\draw (139.20000000000002, -16.0) rectangle (158.25000000000003,-19.2);
\draw(140.20000000000002, -17.0) node[anchor=north west,align=left] {Universal \\ Taylor series\\ in one \\ complex variable};
\draw (140.20000000000002, -17.0) rectangle (144.8,-19.1);
\draw(144.9, -17.0) node[anchor=north west,align=left] {Universal \\ Dirichlet series\\ in one \\ complex variable};
\draw (144.9, -17.0) rectangle (149.5,-19.1);
\draw(149.60000000000002, -17.0) node[anchor=north west,align=left] {Universal\\ functions\\ of one \\ complex variable};
\draw (149.60000000000002, -17.0) rectangle (154.20000000000002,-19.1);
\draw(154.3, -17.0) node[anchor=north west,align=left] {Compositional\\ universality};
\draw (154.3, -17.0) rectangle (158.15,-18.6);
\draw(158.35000000000002, -16.0) node[anchor=north west,align=left] {\large General properties of functions of one complex variable};
\draw (158.35000000000002, -16.0) rectangle (176.00000000000003,-19.7);
\draw(159.35000000000002, -17.0) node[anchor=north west,align=left] {Monogenic \\ and polygenic\\ functions\\ of one \\ complex variable};
\draw (159.35000000000002, -17.0) rectangle (163.95000000000002,-19.6);
\draw(164.05, -17.0) node[anchor=north west,align=left] {Inequalities\\ in\\ the complex\\ plane};
\draw (164.05, -17.0) rectangle (167.65,-19.1);
\draw(176.10000000000002, -16.0) node[anchor=north west,align=left] {\large Riemann surfaces};
\draw (176.10000000000002, -16.0) rectangle (187.75000000000003,-28.5);
\draw(177.10000000000002, -17.0) node[anchor=north west,align=left] {Conformal \\ metrics \\ (hyperbolic, Poincaré,\\ distance\\ functions)};
\draw (177.10000000000002, -17.0) rectangle (183.20000000000002,-19.6);
\draw(183.3, -17.0) node[anchor=north west,align=left] {Ideal \\ boundary theory\\ for Riemann\\ surfaces};
\draw (183.3, -17.0) rectangle (187.65,-19.1);
\draw(177.10000000000002, -19.7) node[anchor=north west,align=left] {Fuchsian groups\\ and automorphic\\ functions (aspects\\ of compact \\ Riemann surfaces \\ and uniformization)};
\draw (177.10000000000002, -19.7) rectangle (182.45000000000002,-22.8);
\draw(182.55, -19.7) node[anchor=north west,align=left] {Kleinian groups\\ (aspects of\\ compact Riemann\\ surfaces and\\ uniformization)};
\draw (182.55, -19.7) rectangle (186.9,-22.3);
\draw(177.10000000000002, -22.9) node[anchor=north west,align=left] {Compact \\ Riemann \\ surfaces and \\ uniformization};
\draw (177.10000000000002, -22.9) rectangle (181.20000000000002,-25.0);
\draw(181.3, -22.9) node[anchor=north west,align=left] {Classification\\ theory\\ of Riemann\\ surfaces};
\draw (181.3, -22.9) rectangle (185.4,-25.0);
\draw(177.10000000000002, -25.1) node[anchor=north west,align=left] {Differentials\\ on\\ Riemann\\ surfaces};
\draw (177.10000000000002, -25.1) rectangle (180.95000000000002,-27.200000000000003);
\draw(181.05, -25.1) node[anchor=north west,align=left] {Teichmüller\\ theory\\ for Riemann\\ surfaces};
\draw (181.05, -25.1) rectangle (184.4,-27.200000000000003);
\draw(184.50000000000003, -25.1) node[anchor=north west,align=left] {Harmonic\\ functions\\ on Riemann\\ surfaces};
\draw (184.50000000000003, -25.1) rectangle (187.60000000000002,-27.200000000000003);
\draw(177.10000000000002, -27.3) node[anchor=north west,align=left] {Klein \\ surfaces};
\draw (177.10000000000002, -27.3) rectangle (179.70000000000002,-28.400000000000002);
\draw(158.35000000000002, -19.8) node[anchor=north west,align=left] {\large Computational methods for\\ problems pertaining to \\ functions of a complex variable};
\draw (158.35000000000002, -19.8) rectangle (168.56000000000003,-21.400000000000002);
\draw(158.35000000000002, -21.5) node[anchor=north west,align=left] {\large Analysis on metric spaces};
\draw (158.35000000000002, -21.5) rectangle (167.50000000000003,-26.4);
\draw(159.35000000000002, -22.5) node[anchor=north west,align=left] {Quasiconformal\\ mappings\\ in \\ metric spaces};
\draw (159.35000000000002, -22.5) rectangle (163.45000000000002,-24.6);
\draw(163.55, -22.5) node[anchor=north west,align=left] {Geometric\\ embeddings\\ of \\ metric spaces};
\draw (163.55, -22.5) rectangle (167.4,-24.6);
\draw(159.35000000000002, -24.7) node[anchor=north west,align=left] {Inequalities\\ in \\ metric spaces};
\draw (159.35000000000002, -24.7) rectangle (163.20000000000002,-26.3);
\draw(139.20000000000002, -28.6) node[anchor=north west,align=left] {\large Geometric function theory};
\draw (139.20000000000002, -28.6) rectangle (152.60000000000002,-50.5);
\draw(140.20000000000002, -29.6) node[anchor=north west,align=left] {Schwarz-Christoffel-type\\ mappings};
\draw (140.20000000000002, -29.6) rectangle (146.8,-31.700000000000003);
\draw(146.9, -29.6) node[anchor=north west,align=left] {Maximum principle,\\ Schwarz’s \\ lemma, Lindelöf \\ principle, analogues\\ and generalizations;\\ subordination};
\draw (146.9, -29.6) rectangle (152.5,-32.7);
\draw(140.20000000000002, -32.800000000000004) node[anchor=north west,align=left] {Extremal problems\\ for conformal\\ and \\ quasiconformal mappings,\\ other methods};
\draw (140.20000000000002, -32.800000000000004) rectangle (146.8,-35.400000000000006);
\draw(146.9, -32.800000000000004) node[anchor=north west,align=left] {Extremal problems\\ for conformal\\ and quasiconformal\\ mappings, \\ variational methods};
\draw (146.9, -32.800000000000004) rectangle (152.25,-35.400000000000006);
\draw(140.20000000000002, -35.5) node[anchor=north west,align=left] {Zeros of polynomials,\\ rational functions,\\ and other analytic\\ functions of one\\ complex variable\\ (e.g., zeros of \\ functions with bounded\\ Dirichlet integral)};
\draw (140.20000000000002, -35.5) rectangle (146.3,-39.6);
\draw(146.4, -35.5) node[anchor=north west,align=left] {Special classes \\ of univalent and \\ multivalent functions\\ of one complex\\ variable (starlike,\\ convex, bounded\\ rotation, etc.)};
\draw (146.4, -35.5) rectangle (152.25,-39.1);
\draw(140.20000000000002, -39.7) node[anchor=north west,align=left] {Coefficient \\ problems for \\ univalent and \\ multivalent functions\\ of one \\ complex variable};
\draw (140.20000000000002, -39.7) rectangle (146.05,-42.800000000000004);
\draw(146.15, -39.7) node[anchor=north west,align=left] {Quasiconformal\\ mappings in\\ \(\mathbb{R}^n\),\\ other \\ generalizations};
\draw (146.15, -39.7) rectangle (151.0,-42.300000000000004);
\draw(140.20000000000002, -42.900000000000006) node[anchor=north west,align=left] {Polynomials\\ and rational\\ functions\\ of one \\ complex variable};
\draw (140.20000000000002, -42.900000000000006) rectangle (144.8,-45.50000000000001);
\draw(144.9, -42.900000000000006) node[anchor=north west,align=left] {Kernel \\ functions in one\\ complex \\ variable and\\ applications};
\draw (144.9, -42.900000000000006) rectangle (149.5,-45.50000000000001);
\draw(149.60000000000002, -42.900000000000006) node[anchor=north west,align=left] {General \\ theory of\\ conformal\\ mappings};
\draw (149.60000000000002, -42.900000000000006) rectangle (152.45000000000002,-45.00000000000001);
\draw(140.20000000000002, -45.6) node[anchor=north west,align=left] {General theory\\ of univalent and\\ multivalent \\ functions of one\\ complex variable};
\draw (140.20000000000002, -45.6) rectangle (144.8,-48.2);
\draw(144.9, -45.6) node[anchor=north west,align=left] {Covering \\ theorems in \\ conformal \\ mapping theory};
\draw (144.9, -45.6) rectangle (149.0,-47.7);
\draw(149.10000000000002, -45.6) node[anchor=north west,align=left] {Conformal\\ mappings\\ of special\\ domains};
\draw (149.10000000000002, -45.6) rectangle (152.20000000000002,-47.7);
\draw(140.20000000000002, -48.3) node[anchor=north west,align=left] {Quasiconformal\\ mappings\\ in the \\ complex plane};
\draw (140.20000000000002, -48.3) rectangle (144.3,-50.4);
\draw(144.4, -48.3) node[anchor=north west,align=left] {Capacity and\\ harmonic \\ measure in the\\ complex plane};
\draw (144.4, -48.3) rectangle (148.5,-50.4);
\draw(152.70000000000002, -28.6) node[anchor=north west,align=left] {\large Generalized function theory};
\draw (152.70000000000002, -28.6) rectangle (164.10000000000002,-38.2);
\draw(153.70000000000002, -29.6) node[anchor=north west,align=left] {Finely \\ holomorphic functions\\ and \\ topological \\ function theory};
\draw (153.70000000000002, -29.6) rectangle (159.55,-32.2);
\draw(159.65, -29.6) node[anchor=north west,align=left] {Non-Archimedean\\ function theory};
\draw (159.65, -29.6) rectangle (164.0,-31.200000000000003);
\draw(153.70000000000002, -32.300000000000004) node[anchor=north west,align=left] {Generalizations\\ of Bers and \\ Vekua type \\ (pseudoanalytic, \\ \(p\)-analytic, etc.)};
\draw (153.70000000000002, -32.300000000000004) rectangle (159.55,-34.900000000000006);
\draw(159.65, -32.300000000000004) node[anchor=north west,align=left] {Functions of\\ hypercomplex\\ variables\\ and generalized\\ variables};
\draw (159.65, -32.300000000000004) rectangle (164.0,-34.900000000000006);
\draw(153.70000000000002, -35.0) node[anchor=north west,align=left] {Other \\ generalizations of \\ analytic functions\\ (including\\ abstract-valued\\ functions)};
\draw (153.70000000000002, -35.0) rectangle (159.05,-38.1);
\draw(159.15, -35.0) node[anchor=north west,align=left] {Discrete\\ analytic\\ functions};
\draw (159.15, -35.0) rectangle (162.0,-36.6);
\draw(152.70000000000002, -38.300000000000004) node[anchor=north west,align=left] {\large Function theory on the disc};
\draw (152.70000000000002, -38.300000000000004) rectangle (161.67000000000002,-43.2);
\draw(153.70000000000002, -39.300000000000004) node[anchor=north west,align=left] {Singular inner\\ functions\\ of one complex\\ variable};
\draw (153.70000000000002, -39.300000000000004) rectangle (157.8,-41.400000000000006);
\draw(157.9, -39.300000000000004) node[anchor=north west,align=left] {Inner \\ functions of\\ one complex\\ variable};
\draw (157.9, -39.300000000000004) rectangle (161.5,-41.400000000000006);
\draw(153.70000000000002, -41.50000000000001) node[anchor=north west,align=left] {Blaschke\\ products};
\draw (153.70000000000002, -41.50000000000001) rectangle (156.3,-43.10000000000001);
\draw(187.95000000000002, -1) node[anchor=north west,align=left] {\LARGE Number theory};
\draw (187.95000000000002, -1) rectangle (235.68000000000004,-113.69999999999999);
\draw(188.95000000000002, -2) node[anchor=north west,align=left] {\large Probabilistic theory: distribution modulo \(1\); metric theory of algorithms};
\draw (188.95000000000002, -2) rectangle (215.60000000000002,-8.9);
\draw(189.95000000000002, -3) node[anchor=north west,align=left] {Normal numbers,\\ radix expansions,\\ Pisot \\ numbers, Salem \\ numbers, good \\ lattice points, etc.};
\draw (189.95000000000002, -3) rectangle (195.55,-6.1);
\draw(195.65, -3) node[anchor=north west,align=left] {Metric theory\\ of other \\ algorithms and \\ expansions; \\ measure and \\ Hausdorff dimension};
\draw (195.65, -3) rectangle (201.0,-6.1);
\draw(201.10000000000002, -3) node[anchor=north west,align=left] {Harmonic analysis\\ and almost\\ periodicity \\ in probabilistic\\ number theory};
\draw (201.10000000000002, -3) rectangle (205.95000000000002,-5.6);
\draw(206.05, -3) node[anchor=north west,align=left] {Well-distributed\\ sequences\\ and \\ other variations};
\draw (206.05, -3) rectangle (210.65,-5.1);
\draw(210.75000000000003, -3) node[anchor=north west,align=left] {Irregularities\\ of \\ distribution,\\ discrepancy};
\draw (210.75000000000003, -3) rectangle (214.85000000000002,-5.1);
\draw(189.95000000000002, -6.2) node[anchor=north west,align=left] {Diophantine\\ approximation\\ in \\ probabilistic \\ number theory};
\draw (189.95000000000002, -6.2) rectangle (194.05,-8.8);
\draw(194.15, -6.2) node[anchor=north west,align=left] {Pseudo-random\\ numbers;\\ Monte \\ Carlo methods};
\draw (194.15, -6.2) rectangle (198.0,-8.3);
\draw(198.10000000000002, -6.2) node[anchor=north west,align=left] {Arithmetic \\ functions in\\ probabilistic\\ number theory};
\draw (198.10000000000002, -6.2) rectangle (201.95000000000002,-8.3);
\draw(202.05, -6.2) node[anchor=north west,align=left] {General \\ theory of \\ distribution\\ modulo \(1\)};
\draw (202.05, -6.2) rectangle (205.65,-8.3);
\draw(205.75000000000003, -6.2) node[anchor=north west,align=left] {Continuous,\\ \(p\)-adic\\ and abstract\\ analogues};
\draw (205.75000000000003, -6.2) rectangle (209.35000000000002,-8.3);
\draw(209.45000000000002, -6.2) node[anchor=north west,align=left] {Metric \\ theory of \\ continued\\ fractions};
\draw (209.45000000000002, -6.2) rectangle (212.55,-8.3);
\draw(212.65, -6.2) node[anchor=north west,align=left] {Special\\ sequences};
\draw (212.65, -6.2) rectangle (215.5,-7.800000000000001);
\draw(215.70000000000002, -2) node[anchor=north west,align=left] {\large Arithmetic algebraic geometry (Diophantine geometry)};
\draw (215.70000000000002, -2) rectangle (235.50000000000003,-13.3);
\draw(216.70000000000002, -3) node[anchor=north west,align=left] {\(L\)-functions\\ of varieties\\ over global\\ fields; \\ Birch-Swinnerton-Dyer\\ conjecture};
\draw (216.70000000000002, -3) rectangle (222.55,-6.1);
\draw(222.65, -3) node[anchor=north west,align=left] {Drinfel’d \\ modules; \\ higher-dimensional\\ motives, etc.};
\draw (222.65, -3) rectangle (227.75,-5.1);
\draw(227.85000000000002, -3) node[anchor=north west,align=left] {Arithmetic \\ aspects of dessins\\ d’enfants,\\ Belyĭ theory};
\draw (227.85000000000002, -3) rectangle (232.95000000000002,-5.1);
\draw(233.05, -3) node[anchor=north west,align=left] {Heights};
\draw (233.05, -3) rectangle (235.4,-4.1);
\draw(216.70000000000002, -6.2) node[anchor=north west,align=left] {Complex \\ multiplication\\ and \\ moduli of \\ abelian varieties};
\draw (216.70000000000002, -6.2) rectangle (221.55,-8.8);
\draw(221.65, -6.2) node[anchor=north west,align=left] {Arithmetic \\ aspects of \\ modular and \\ Shimura varieties};
\draw (221.65, -6.2) rectangle (226.5,-8.3);
\draw(226.60000000000002, -6.2) node[anchor=north west,align=left] {Curves of \\ arbitrary genus\\ or genus \\ \(\ne~1\) over\\ global fields};
\draw (226.60000000000002, -6.2) rectangle (230.95000000000002,-8.8);
\draw(231.05, -6.2) node[anchor=north west,align=left] {Polylogarithms\\ and \\ relations with\\ \(K\)-theory};
\draw (231.05, -6.2) rectangle (235.15,-8.3);
\draw(216.70000000000002, -8.9) node[anchor=north west,align=left] {Elliptic\\ curves\\ over \\ global fields};
\draw (216.70000000000002, -8.9) rectangle (220.55,-11.0);
\draw(220.65, -8.9) node[anchor=north west,align=left] {Elliptic\\ and \\ modular units};
\draw (220.65, -8.9) rectangle (224.5,-10.5);
\draw(224.60000000000002, -8.9) node[anchor=north west,align=left] {Varieties\\ over \\ global fields};
\draw (224.60000000000002, -8.9) rectangle (228.45000000000002,-10.5);
\draw(228.55, -8.9) node[anchor=north west,align=left] {Abelian \\ varieties\\ of dimension\\ \(>~1\)};
\draw (228.55, -8.9) rectangle (232.15,-11.0);
\draw(232.25000000000003, -8.9) node[anchor=north west,align=left] {Elliptic\\ curves\\ over local\\ fields};
\draw (232.25000000000003, -8.9) rectangle (235.35000000000002,-11.0);
\draw(216.70000000000002, -11.100000000000001) node[anchor=north west,align=left] {Curves \\ over finite\\ and \\ local fields};
\draw (216.70000000000002, -11.100000000000001) rectangle (220.3,-13.200000000000001);
\draw(220.4, -11.100000000000001) node[anchor=north west,align=left] {Varieties\\ over \\ finite and \\ local fields};
\draw (220.4, -11.100000000000001) rectangle (224.0,-13.200000000000001);
\draw(224.10000000000002, -11.100000000000001) node[anchor=north west,align=left] {Geometric\\ class \\ field theory};
\draw (224.10000000000002, -11.100000000000001) rectangle (227.70000000000002,-12.700000000000001);
\draw(227.8, -11.100000000000001) node[anchor=north west,align=left] {Arithmetic\\ mirror\\ symmetry};
\draw (227.8, -11.100000000000001) rectangle (230.9,-12.700000000000001);
\draw(188.95000000000002, -9.0) node[anchor=north west,align=left] {\large Miscellaneous applications of number theory};
\draw (188.95000000000002, -9.0) rectangle (202.88000000000002,-12.2);
\draw(189.95000000000002, -10.0) node[anchor=north west,align=left] {Miscellaneous\\ applications of\\ number theory};
\draw (189.95000000000002, -10.0) rectangle (194.3,-12.1);
\draw(188.95000000000002, -13.4) node[anchor=north west,align=left] {\large Finite fields and commutative rings (number-theoretic aspects)};
\draw (188.95000000000002, -13.4) rectangle (210.00000000000003,-19.8);
\draw(189.95000000000002, -14.4) node[anchor=north west,align=left] {Structure \\ theory for finite\\ fields and\\ commutative \\ rings \\ (number-theoretic aspects)};
\draw (189.95000000000002, -14.4) rectangle (197.05,-17.5);
\draw(197.15, -14.4) node[anchor=north west,align=left] {Algebraic \\ coding theory;\\ cryptography\\ (number-theoretic\\ aspects)};
\draw (197.15, -14.4) rectangle (202.0,-17.0);
\draw(202.10000000000002, -14.4) node[anchor=north west,align=left] {Polynomials\\ over \\ finite fields};
\draw (202.10000000000002, -14.4) rectangle (205.95000000000002,-16.0);
\draw(206.05, -14.4) node[anchor=north west,align=left] {Arithmetic\\ theory of \\ polynomial \\ rings over \\ finite fields};
\draw (206.05, -14.4) rectangle (209.9,-17.0);
\draw(189.95000000000002, -17.6) node[anchor=north west,align=left] {Exponential\\ sums};
\draw (189.95000000000002, -17.6) rectangle (193.3,-18.700000000000003);
\draw(193.4, -17.6) node[anchor=north west,align=left] {Finite \\ upper \\ half-planes};
\draw (193.4, -17.6) rectangle (196.75,-19.200000000000003);
\draw(196.85000000000002, -17.6) node[anchor=north west,align=left] {Other \\ character\\ sums and\\ Gauss sums};
\draw (196.85000000000002, -17.6) rectangle (199.95000000000002,-19.700000000000003);
\draw(200.05, -17.6) node[anchor=north west,align=left] {Cyclotomy};
\draw (200.05, -17.6) rectangle (202.9,-18.700000000000003);
\draw(210.10000000000002, -13.4) node[anchor=north west,align=left] {\large Diophantine approximation, transcendental number theory};
\draw (210.10000000000002, -13.4) rectangle (230.85000000000002,-26.4);
\draw(211.10000000000002, -14.4) node[anchor=north west,align=left] {Transcendence\\ theory\\ of elliptic\\ and \\ abelian functions};
\draw (211.10000000000002, -14.4) rectangle (215.95000000000002,-17.0);
\draw(216.05, -14.4) node[anchor=north west,align=left] {Transcendence\\ theory \\ of other \\ special functions};
\draw (216.05, -14.4) rectangle (220.9,-16.5);
\draw(221.00000000000003, -14.4) node[anchor=north west,align=left] {Small \\ fractional parts\\ of polynomials\\ and \\ generalizations};
\draw (221.00000000000003, -14.4) rectangle (225.60000000000002,-17.0);
\draw(225.70000000000002, -14.4) node[anchor=north west,align=left] {Number-theoretic\\ analogues\\ of methods \\ in Nevanlinna\\ theory (work\\ of Vojta et al.)};
\draw (225.70000000000002, -14.4) rectangle (230.3,-17.5);
\draw(211.10000000000002, -17.6) node[anchor=north west,align=left] {Markov and\\ Lagrange \\ spectra and \\ generalizations};
\draw (211.10000000000002, -17.6) rectangle (215.45000000000002,-19.700000000000003);
\draw(215.55, -17.6) node[anchor=north west,align=left] {Approximation\\ in \\ non-Archimedean\\ valuations};
\draw (215.55, -17.6) rectangle (219.9,-19.700000000000003);
\draw(220.00000000000003, -17.6) node[anchor=north west,align=left] {Continued\\ fractions\\ and \\ generalizations};
\draw (220.00000000000003, -17.6) rectangle (224.35000000000002,-19.700000000000003);
\draw(224.45000000000002, -17.6) node[anchor=north west,align=left] {Homogeneous\\ approximation \\ to one number};
\draw (224.45000000000002, -17.6) rectangle (228.55,-19.700000000000003);
\draw(228.65000000000003, -17.6) node[anchor=north west,align=left] {Metric\\ theory};
\draw (228.65000000000003, -17.6) rectangle (230.75000000000003,-18.700000000000003);
\draw(211.10000000000002, -19.8) node[anchor=north west,align=left] {Simultaneous\\ homogeneous \\ approximation,\\ linear forms};
\draw (211.10000000000002, -19.8) rectangle (215.20000000000002,-21.900000000000002);
\draw(215.3, -19.8) node[anchor=north west,align=left] {Irrationality;\\ linear\\ independence\\ over a field};
\draw (215.3, -19.8) rectangle (219.4,-21.900000000000002);
\draw(219.50000000000003, -19.8) node[anchor=north west,align=left] {Linear forms\\ in \\ logarithms; \\ Baker’s method};
\draw (219.50000000000003, -19.8) rectangle (223.60000000000002,-21.900000000000002);
\draw(223.70000000000002, -19.8) node[anchor=north west,align=left] {Approximation\\ by \\ numbers from\\ a fixed field};
\draw (223.70000000000002, -19.8) rectangle (227.55,-21.900000000000002);
\draw(227.65000000000003, -19.8) node[anchor=north west,align=left] {Results\\ involving\\ abelian\\ varieties};
\draw (227.65000000000003, -19.8) rectangle (230.50000000000003,-21.900000000000002);
\draw(211.10000000000002, -22.0) node[anchor=north west,align=left] {Inhomogeneous\\ linear forms};
\draw (211.10000000000002, -22.0) rectangle (214.95000000000002,-23.6);
\draw(215.05, -22.0) node[anchor=north west,align=left] {Approximation\\ to\\ algebraic\\ numbers};
\draw (215.05, -22.0) rectangle (218.9,-24.1);
\draw(219.00000000000003, -22.0) node[anchor=north west,align=left] {Transcendence\\ (general\\ theory)};
\draw (219.00000000000003, -22.0) rectangle (222.85000000000002,-23.6);
\draw(222.95000000000002, -22.0) node[anchor=north west,align=left] {Measures of\\ irrationality\\ and of\\ transcendence};
\draw (222.95000000000002, -22.0) rectangle (226.8,-24.1);
\draw(226.90000000000003, -22.0) node[anchor=north west,align=left] {Algebraic\\ independence;\\ Gel’fond’s\\ method};
\draw (226.90000000000003, -22.0) rectangle (230.75000000000003,-24.1);
\draw(211.10000000000002, -24.200000000000003) node[anchor=north west,align=left] {Transcendence\\ theory of\\ Drinfel’d and\\ \(t\)-modules};
\draw (211.10000000000002, -24.200000000000003) rectangle (214.95000000000002,-26.300000000000004);
\draw(215.05, -24.200000000000003) node[anchor=north west,align=left] {Diophantine\\ inequalities};
\draw (215.05, -24.200000000000003) rectangle (218.65,-25.800000000000004);
\draw(218.75000000000003, -24.200000000000003) node[anchor=north west,align=left] {Distribution\\ modulo one};
\draw (218.75000000000003, -24.200000000000003) rectangle (222.35000000000002,-25.800000000000004);
\draw(222.45000000000002, -24.200000000000003) node[anchor=north west,align=left] {Schmidt \\ Subspace \\ Theorem and \\ applications};
\draw (222.45000000000002, -24.200000000000003) rectangle (226.05,-26.300000000000004);
\draw(188.95000000000002, -19.900000000000002) node[anchor=north west,align=left] {\large Connections of number theory and logic};
\draw (188.95000000000002, -19.900000000000002) rectangle (201.33,-25.300000000000004);
\draw(189.95000000000002, -20.900000000000002) node[anchor=north west,align=left] {Decidability\\ (number-theoretic\\ aspects)};
\draw (189.95000000000002, -20.900000000000002) rectangle (194.8,-23.000000000000004);
\draw(194.9, -20.900000000000002) node[anchor=north west,align=left] {Ultraproducts\\ (number-theoretic\\ aspects)};
\draw (194.9, -20.900000000000002) rectangle (199.75,-23.000000000000004);
\draw(189.95000000000002, -23.1) node[anchor=north west,align=left] {Model \\ theory \\ (number-theoretic\\ aspects)};
\draw (189.95000000000002, -23.1) rectangle (194.8,-25.200000000000003);
\draw(194.9, -23.1) node[anchor=north west,align=left] {Nonstandard\\ arithmetic \\ (number-theoretic\\ aspects)};
\draw (194.9, -23.1) rectangle (199.75,-25.200000000000003);
\draw(230.95000000000002, -13.4) node[anchor=north west,align=left] {\large History of \\ number theory};
\draw (230.95000000000002, -13.4) rectangle (235.58,-14.5);
\draw(188.95000000000002, -26.5) node[anchor=north west,align=left] {\large Algebraic number theory: local and \(p\)-adic fields};
\draw (188.95000000000002, -26.5) rectangle (208.75000000000003,-36.8);
\draw(189.95000000000002, -27.5) node[anchor=north west,align=left] {Non-Archimedeandynamical\\ systems};
\draw (189.95000000000002, -27.5) rectangle (196.55,-29.6);
\draw(196.65, -27.5) node[anchor=north west,align=left] {Other analytic\\ theory (analogues\\ of beta and\\ gamma functions,\\ \(p\)-adic \\ integration, etc.)};
\draw (196.65, -27.5) rectangle (201.75,-30.6);
\draw(201.85000000000002, -27.5) node[anchor=north west,align=left] {Langlands-Weil\\ conjectures,\\ nonabelian class\\ field theory};
\draw (201.85000000000002, -27.5) rectangle (206.45000000000002,-30.1);
\draw(206.55, -27.5) node[anchor=north west,align=left] {Galois\\ theory};
\draw (206.55, -27.5) rectangle (208.65,-28.6);
\draw(189.95000000000002, -30.7) node[anchor=north west,align=left] {Integral\\ representations};
\draw (189.95000000000002, -30.7) rectangle (194.3,-32.3);
\draw(194.4, -30.7) node[anchor=north west,align=left] {Zeta \\ functions \\ and \\ \(L\)-functions};
\draw (194.4, -30.7) rectangle (198.75,-32.8);
\draw(198.85000000000002, -30.7) node[anchor=north west,align=left] {Algebras and\\ orders, \\ and their \\ zeta functions};
\draw (198.85000000000002, -30.7) rectangle (202.95000000000002,-32.8);
\draw(203.05, -30.7) node[anchor=north west,align=left] {Prehomogeneous\\ vector spaces};
\draw (203.05, -30.7) rectangle (207.15,-32.3);
\draw(189.95000000000002, -32.9) node[anchor=north west,align=left] {Class field\\ theory; \\ \(p\)-adic \\ formal groups};
\draw (189.95000000000002, -32.9) rectangle (193.8,-35.0);
\draw(193.9, -32.9) node[anchor=north west,align=left] {Ramification\\ and\\ extension\\ theory};
\draw (193.9, -32.9) rectangle (197.5,-35.0);
\draw(197.60000000000002, -32.9) node[anchor=north west,align=left] {\(K\)-theory\\ of \\ local fields};
\draw (197.60000000000002, -32.9) rectangle (201.20000000000002,-34.5);
\draw(201.3, -32.9) node[anchor=north west,align=left] {Polynomials};
\draw (201.3, -32.9) rectangle (204.65,-34.0);
\draw(204.75000000000003, -32.9) node[anchor=north west,align=left] {Other \\ nonanalytic\\ theory};
\draw (204.75000000000003, -32.9) rectangle (208.10000000000002,-34.5);
\draw(189.95000000000002, -35.1) node[anchor=north west,align=left] {Galois\\ cohomology};
\draw (189.95000000000002, -35.1) rectangle (193.05,-36.7);
\draw(208.85000000000002, -26.5) node[anchor=north west,align=left] {\large Discontinuous groups and automorphic forms};
\draw (208.85000000000002, -26.5) rectangle (224.95000000000002,-55.5);
\draw(209.85000000000002, -27.5) node[anchor=north west,align=left] {Automorphic forms on\\ \(\mbox{GL}(2)\); \\ Hilbert and Hilbert-Siegel\\ modular groups\\ and their modular \\ and automorphic forms;\\ Hilbert modular surfaces};
\draw (209.85000000000002, -27.5) rectangle (216.95000000000002,-31.1);
\draw(217.05, -27.5) node[anchor=north west,align=left] {Representation-theoretic\\ methods;\\ automorphic\\ representations\\ over local and\\ global fields};
\draw (217.05, -27.5) rectangle (223.65,-30.6);
\draw(209.85000000000002, -31.2) node[anchor=north west,align=left] {Special values \\ of automorphic \\ \(L\)-series, periods\\ of automorphic\\ forms, cohomology,\\ modular symbols};
\draw (209.85000000000002, -31.2) rectangle (215.70000000000002,-34.3);
\draw(215.8, -31.2) node[anchor=north west,align=left] {Langlands \\ \(L\)-functions;\\ one variable\\ Dirichlet \\ series and \\ functional equations};
\draw (215.8, -31.2) rectangle (221.4,-34.3);
\draw(221.50000000000003, -31.2) node[anchor=north west,align=left] {Modular\\ and \\ automorphic\\ functions};
\draw (221.50000000000003, -31.2) rectangle (224.85000000000002,-33.3);
\draw(209.85000000000002, -34.4) node[anchor=north west,align=left] {Other groups \\ and their modular\\ and automorphic\\ forms \\ (several variables)};
\draw (209.85000000000002, -34.4) rectangle (215.20000000000002,-37.0);
\draw(215.3, -34.4) node[anchor=north west,align=left] {Hecke-Petersson\\ operators,\\ differential\\ operators \\ (several variables)};
\draw (215.3, -34.4) rectangle (220.65,-37.0);
\draw(220.75000000000003, -34.4) node[anchor=north west,align=left] {Relations \\ with algebraic\\ geometry\\ and topology};
\draw (220.75000000000003, -34.4) rectangle (224.85000000000002,-36.5);
\draw(209.85000000000002, -37.1) node[anchor=north west,align=left] {Dirichlet series\\ in several complex\\ variables \\ associated to \\ automorphic forms; \\ Weyl group multiple\\ Dirichlet series};
\draw (209.85000000000002, -37.1) rectangle (215.20000000000002,-40.7);
\draw(215.3, -37.1) node[anchor=north west,align=left] {Siegel modular\\ groups; Siegel\\ and Hilbert-Siegel\\ modular and\\ automorphic forms};
\draw (215.3, -37.1) rectangle (220.4,-39.7);
\draw(220.50000000000003, -37.1) node[anchor=north west,align=left] {Holomorphic\\ modular \\ forms of \\ integral weight};
\draw (220.50000000000003, -37.1) rectangle (224.85000000000002,-39.2);
\draw(209.85000000000002, -40.8) node[anchor=north west,align=left] {Structure of\\ modular groups\\ and \\ generalizations; \\ arithmetic groups};
\draw (209.85000000000002, -40.8) rectangle (214.70000000000002,-43.4);
\draw(214.8, -40.8) node[anchor=north west,align=left] {Automorphic\\ forms and \\ their relations\\ with \\ perfectoid spaces};
\draw (214.8, -40.8) rectangle (219.65,-43.4);
\draw(219.75000000000003, -40.8) node[anchor=north west,align=left] {Modular \\ correspondences,\\ etc.};
\draw (219.75000000000003, -40.8) rectangle (224.35000000000002,-42.4);
\draw(209.85000000000002, -43.5) node[anchor=north west,align=left] {Relationship\\ to Lie algebras\\ and finite\\ simple groups};
\draw (209.85000000000002, -43.5) rectangle (214.20000000000002,-45.6);
\draw(214.3, -43.5) node[anchor=north west,align=left] {Hecke-Petersson\\ operators,\\ differential\\ operators\\ (one variable)};
\draw (214.3, -43.5) rectangle (218.65,-46.1);
\draw(218.75000000000003, -43.5) node[anchor=north west,align=left] {Theta series;\\ Weil \\ representation;\\ theta \\ correspondences};
\draw (218.75000000000003, -43.5) rectangle (223.10000000000002,-46.1);
\draw(209.85000000000002, -46.2) node[anchor=north west,align=left] {Congruences \\ for modular and\\ \(p\)-adic\\ modular forms};
\draw (209.85000000000002, -46.2) rectangle (214.20000000000002,-48.300000000000004);
\draw(214.3, -46.2) node[anchor=north west,align=left] {Galois\\ representations};
\draw (214.3, -46.2) rectangle (218.65,-47.800000000000004);
\draw(218.75000000000003, -46.2) node[anchor=north west,align=left] {Fourier \\ coefficients\\ of automorphic\\ forms};
\draw (218.75000000000003, -46.2) rectangle (222.85000000000002,-48.300000000000004);
\draw(209.85000000000002, -48.4) node[anchor=north west,align=left] {Forms of \\ half-integer\\ weight; \\ nonholomorphic\\ modular forms};
\draw (209.85000000000002, -48.4) rectangle (213.95000000000002,-51.0);
\draw(214.05, -48.4) node[anchor=north west,align=left] {Dedekind\\ eta function,\\ Dedekind sums};
\draw (214.05, -48.4) rectangle (217.9,-50.5);
\draw(218.00000000000003, -48.4) node[anchor=north west,align=left] {Modular forms\\ associated\\ to Drinfel’d\\ modules};
\draw (218.00000000000003, -48.4) rectangle (221.85000000000002,-50.5);
\draw(221.95000000000002, -48.4) node[anchor=north west,align=left] {Jacobi\\ forms};
\draw (221.95000000000002, -48.4) rectangle (224.05,-49.5);
\draw(209.85000000000002, -51.099999999999994) node[anchor=north west,align=left] {Spectral \\ theory; trace\\ formulas \\ (e.g., that\\ of Selberg)};
\draw (209.85000000000002, -51.099999999999994) rectangle (213.70000000000002,-53.699999999999996);
\draw(213.8, -51.099999999999994) node[anchor=north west,align=left] {Cohomology\\ of arithmetic\\ groups};
\draw (213.8, -51.099999999999994) rectangle (217.65,-52.699999999999996);
\draw(217.75000000000003, -51.099999999999994) node[anchor=north west,align=left] {Automorphic\\ forms, \\ one variable};
\draw (217.75000000000003, -51.099999999999994) rectangle (221.35000000000002,-52.699999999999996);
\draw(209.85000000000002, -53.8) node[anchor=north west,align=left] {\(p\)-adic\\ theory, \\ local fields};
\draw (209.85000000000002, -53.8) rectangle (213.45000000000002,-55.4);
\draw(188.95000000000002, -36.9) node[anchor=north west,align=left] {\large Zeta and \(L\)-functions: analytic theory};
\draw (188.95000000000002, -36.9) rectangle (204.3,-49.2);
\draw(189.95000000000002, -37.9) node[anchor=north west,align=left] {Selberg zeta functions\\ and regularized \\ determinants; applications\\ to spectral theory,\\ Dirichlet series, \\ Eisenstein series, etc.\\ (explicit formulas)};
\draw (189.95000000000002, -37.9) rectangle (197.05,-41.5);
\draw(197.15, -37.9) node[anchor=north west,align=left] {Nonreal zeros \\ of \(\zeta~(s)\)\\ and \(L(s,~\chi)\);\\ Riemann \\ and other hypotheses};
\draw (197.15, -37.9) rectangle (202.75,-40.5);
\draw(189.95000000000002, -41.6) node[anchor=north west,align=left] {Real zeros\\ of \(L(s,~\chi)\);\\ results on \\ \(L(1,~\chi)\)};
\draw (189.95000000000002, -41.6) rectangle (195.05,-44.2);
\draw(195.15, -41.6) node[anchor=north west,align=left] {Zeta and \\ \(L\)-functions\\ in characteristic\\ \(p\)};
\draw (195.15, -41.6) rectangle (200.0,-43.7);
\draw(200.10000000000002, -41.6) node[anchor=north west,align=left] {\(\zeta~(s)\)\\ and \\ \(L(s,~\chi)\)};
\draw (200.10000000000002, -41.6) rectangle (204.20000000000002,-43.2);
\draw(189.95000000000002, -44.3) node[anchor=north west,align=left] {Multiple \\ Dirichlet series\\ and zeta \\ functions and \\ multizeta values};
\draw (189.95000000000002, -44.3) rectangle (194.55,-46.9);
\draw(194.65, -44.3) node[anchor=north west,align=left] {Other \\ Dirichlet series\\ and zeta\\ functions};
\draw (194.65, -44.3) rectangle (199.25,-46.4);
\draw(199.35000000000002, -44.3) node[anchor=north west,align=left] {Hurwitz and\\ Lerch \\ zeta functions};
\draw (199.35000000000002, -44.3) rectangle (203.45000000000002,-45.9);
\draw(189.95000000000002, -47.0) node[anchor=north west,align=left] {Relations\\ with \\ noncommutative\\ geometry};
\draw (189.95000000000002, -47.0) rectangle (194.05,-49.1);
\draw(194.15, -47.0) node[anchor=north west,align=left] {Relations\\ with random\\ matrices};
\draw (194.15, -47.0) rectangle (197.5,-48.6);
\draw(197.60000000000002, -47.0) node[anchor=north west,align=left] {Tauberian\\ theorems};
\draw (197.60000000000002, -47.0) rectangle (200.45000000000002,-48.6);
\draw(188.95000000000002, -49.300000000000004) node[anchor=north west,align=left] {\large Polynomials and matrices};
\draw (188.95000000000002, -49.300000000000004) rectangle (197.85000000000002,-52.50000000000001);
\draw(189.95000000000002, -50.300000000000004) node[anchor=north west,align=left] {Polynomials\\ in \\ number theory};
\draw (189.95000000000002, -50.300000000000004) rectangle (193.8,-51.900000000000006);
\draw(193.9, -50.300000000000004) node[anchor=north west,align=left] {Matrices,\\ determinants\\ in \\ number theory};
\draw (193.9, -50.300000000000004) rectangle (197.75,-52.400000000000006);
\draw(225.05, -26.5) node[anchor=north west,align=left] {\large Sequences and sets};
\draw (225.05, -26.5) rectangle (235.45000000000002,-42.4);
\draw(226.05, -27.5) node[anchor=north west,align=left] {Farey sequences;\\ the \\ sequences \\ \(1^k,~2^k,~\dots\)};
\draw (226.05, -27.5) rectangle (231.4,-29.6);
\draw(231.5, -27.5) node[anchor=north west,align=left] {Other \\ combinatorial\\ number\\ theory};
\draw (231.5, -27.5) rectangle (235.35,-29.6);
\draw(226.05, -29.7) node[anchor=north west,align=left] {Arithmetic \\ combinatorics;\\ higher \\ degree uniformity};
\draw (226.05, -29.7) rectangle (230.9,-31.8);
\draw(231.0, -29.7) node[anchor=north west,align=left] {Fibonacci and\\ Lucas numbers\\ and \\ polynomials and\\ generalizations};
\draw (231.0, -29.7) rectangle (235.35,-32.3);
\draw(226.05, -32.4) node[anchor=north west,align=left] {Binomial \\ coefficients; \\ factorials; \\ \(q\)-identities};
\draw (226.05, -32.4) rectangle (230.65,-34.5);
\draw(230.75, -32.4) node[anchor=north west,align=left] {Representation\\ functions};
\draw (230.75, -32.4) rectangle (234.85,-34.0);
\draw(226.05, -34.6) node[anchor=north west,align=left] {Arithmetic\\ progressions};
\draw (226.05, -34.6) rectangle (229.65,-36.2);
\draw(229.75, -34.6) node[anchor=north west,align=left] {Recurrences};
\draw (229.75, -34.6) rectangle (233.1,-35.7);
\draw(226.05, -36.3) node[anchor=north west,align=left] {Bernoulli\\ and Euler\\ numbers and\\ polynomials};
\draw (226.05, -36.3) rectangle (229.4,-38.4);
\draw(229.5, -36.3) node[anchor=north west,align=left] {Special\\ sequences\\ and \\ polynomials};
\draw (229.5, -36.3) rectangle (232.85,-38.4);
\draw(226.05, -38.5) node[anchor=north west,align=left] {Additive\\ bases,\\ including\\ sumsets};
\draw (226.05, -38.5) rectangle (228.9,-40.6);
\draw(229.0, -38.5) node[anchor=north west,align=left] {Sequences\\ (mod\\ \(m\))};
\draw (229.0, -38.5) rectangle (231.85,-40.1);
\draw(231.95000000000002, -38.5) node[anchor=north west,align=left] {Automata\\ sequences};
\draw (231.95000000000002, -38.5) rectangle (234.8,-40.1);
\draw(226.05, -40.7) node[anchor=north west,align=left] {Density,\\ gaps,\\ topology};
\draw (226.05, -40.7) rectangle (228.65,-42.300000000000004);
\draw(228.75, -40.7) node[anchor=north west,align=left] {Bell and\\ Stirling\\ numbers};
\draw (228.75, -40.7) rectangle (231.35,-42.300000000000004);
\draw(188.95000000000002, -55.6) node[anchor=north west,align=left] {\large Algebraic number theory: global fields};
\draw (188.95000000000002, -55.6) rectangle (203.55,-77.4);
\draw(189.95000000000002, -56.6) node[anchor=north west,align=left] {PV-numbers and\\ generalizations;\\ other special\\ algebraic \\ numbers; Mahler measure};
\draw (189.95000000000002, -56.6) rectangle (196.3,-59.2);
\draw(196.4, -56.6) node[anchor=north west,align=left] {Integral \\ representations related\\ to algebraic \\ numbers; Galois \\ module structure \\ of rings of integers};
\draw (196.4, -56.6) rectangle (202.75,-59.7);
\draw(189.95000000000002, -59.800000000000004) node[anchor=north west,align=left] {Zeta functions\\ and \\ \(L\)-functions of \\ function fields};
\draw (189.95000000000002, -59.800000000000004) rectangle (195.3,-61.900000000000006);
\draw(195.4, -59.800000000000004) node[anchor=north west,align=left] {Polynomials\\ (irreducibility,\\ etc.)};
\draw (195.4, -59.800000000000004) rectangle (200.0,-61.900000000000006);
\draw(200.10000000000002, -59.800000000000004) node[anchor=north west,align=left] {Other \\ abelian and\\ metabelian\\ extensions};
\draw (200.10000000000002, -59.800000000000004) rectangle (203.45000000000002,-61.900000000000006);
\draw(189.95000000000002, -62.0) node[anchor=north west,align=left] {Langlands-Weil\\ conjectures,\\ nonabelian class\\ field theory};
\draw (189.95000000000002, -62.0) rectangle (194.55,-64.6);
\draw(194.65, -62.0) node[anchor=north west,align=left] {Zeta functions\\ and \\ \(L\)-functions \\ of number fields};
\draw (194.65, -62.0) rectangle (199.25,-64.1);
\draw(199.35000000000002, -62.0) node[anchor=north west,align=left] {Algebraic \\ numbers; rings\\ of algebraic\\ integers};
\draw (199.35000000000002, -62.0) rectangle (203.45000000000002,-64.1);
\draw(189.95000000000002, -64.7) node[anchor=north west,align=left] {Other algebras\\ and orders,\\ and their\\ zeta and \\ \(L\)-functions};
\draw (189.95000000000002, -64.7) rectangle (194.3,-67.3);
\draw(194.4, -64.7) node[anchor=north west,align=left] {Arithmetic\\ theory of \\ algebraic \\ function fields};
\draw (194.4, -64.7) rectangle (198.75,-66.8);
\draw(198.85000000000002, -64.7) node[anchor=north west,align=left] {Cyclotomic \\ function fields\\ (class groups,\\ Bernoulli\\ objects, etc.)};
\draw (198.85000000000002, -64.7) rectangle (203.20000000000002,-67.3);
\draw(189.95000000000002, -67.4) node[anchor=north west,align=left] {Class \\ numbers, class\\ groups, \\ discriminants};
\draw (189.95000000000002, -67.4) rectangle (194.05,-69.5);
\draw(194.15, -67.4) node[anchor=north west,align=left] {Quaternion and\\ other division\\ algebras:\\ arithmetic,\\ zeta functions};
\draw (194.15, -67.4) rectangle (198.25,-70.0);
\draw(198.35000000000002, -67.4) node[anchor=north west,align=left] {Units \\ and \\ factorization};
\draw (198.35000000000002, -67.4) rectangle (202.20000000000002,-69.0);
\draw(189.95000000000002, -70.1) node[anchor=north west,align=left] {Class groups\\ and \\ Picard groups\\ of orders};
\draw (189.95000000000002, -70.1) rectangle (193.8,-72.19999999999999);
\draw(193.9, -70.1) node[anchor=north west,align=left] {\(K\)-theory\\ of \\ global fields};
\draw (193.9, -70.1) rectangle (197.75,-71.69999999999999);
\draw(197.85000000000002, -70.1) node[anchor=north west,align=left] {Distribution\\ of \\ prime ideals};
\draw (197.85000000000002, -70.1) rectangle (201.45000000000002,-71.69999999999999);
\draw(189.95000000000002, -72.3) node[anchor=north west,align=left] {Quadratic\\ extensions};
\draw (189.95000000000002, -72.3) rectangle (193.05,-73.89999999999999);
\draw(193.15, -72.3) node[anchor=north west,align=left] {Cubic and\\ quartic\\ extensions};
\draw (193.15, -72.3) rectangle (196.25,-73.89999999999999);
\draw(196.35000000000002, -72.3) node[anchor=north west,align=left] {Cyclotomic\\ extensions};
\draw (196.35000000000002, -72.3) rectangle (199.45000000000002,-73.89999999999999);
\draw(199.55, -72.3) node[anchor=north west,align=left] {Galois\\ cohomology};
\draw (199.55, -72.3) rectangle (202.65,-73.89999999999999);
\draw(189.95000000000002, -74.0) node[anchor=north west,align=left] {Adèle \\ rings and\\ groups};
\draw (189.95000000000002, -74.0) rectangle (192.8,-75.6);
\draw(192.9, -74.0) node[anchor=north west,align=left] {Density\\ theorems};
\draw (192.9, -74.0) rectangle (195.5,-75.1);
\draw(195.60000000000002, -74.0) node[anchor=north west,align=left] {Other \\ analytic\\ theory};
\draw (195.60000000000002, -74.0) rectangle (198.20000000000002,-75.6);
\draw(198.3, -74.0) node[anchor=north west,align=left] {Iwasawa\\ theory};
\draw (198.3, -74.0) rectangle (200.65,-75.1);
\draw(200.75000000000003, -74.0) node[anchor=north west,align=left] {Totally\\ real\\ fields};
\draw (200.75000000000003, -74.0) rectangle (203.10000000000002,-75.6);
\draw(189.95000000000002, -75.7) node[anchor=north west,align=left] {Other\\ number\\ fields};
\draw (189.95000000000002, -75.7) rectangle (192.05,-77.3);
\draw(192.15, -75.7) node[anchor=north west,align=left] {Galois\\ theory};
\draw (192.15, -75.7) rectangle (194.25,-76.8);
\draw(194.35000000000002, -75.7) node[anchor=north west,align=left] {Class\\ field\\ theory};
\draw (194.35000000000002, -75.7) rectangle (196.45000000000002,-77.3);
\draw(203.65, -55.6) node[anchor=north west,align=left] {\large Exponential sums and character sums};
\draw (203.65, -55.6) rectangle (216.75,-63.2);
\draw(204.65, -56.6) node[anchor=north west,align=left] {Estimates\\ on \\ exponential sums};
\draw (204.65, -56.6) rectangle (209.25,-58.2);
\draw(209.35, -56.6) node[anchor=north west,align=left] {Gauss and\\ Kloosterman\\ sums; \\ generalizations};
\draw (209.35, -56.6) rectangle (213.7,-58.7);
\draw(213.8, -56.6) node[anchor=north west,align=left] {Sums over\\ primes};
\draw (213.8, -56.6) rectangle (216.65,-57.7);
\draw(204.65, -58.800000000000004) node[anchor=north west,align=left] {Jacobsthal\\ and Brewer\\ sums; other\\ complete \\ character sums};
\draw (204.65, -58.800000000000004) rectangle (208.75,-61.400000000000006);
\draw(208.85, -58.800000000000004) node[anchor=north west,align=left] {Trigonometric\\ and \\ exponential\\ sums, general};
\draw (208.85, -58.800000000000004) rectangle (212.7,-60.900000000000006);
\draw(212.8, -58.800000000000004) node[anchor=north west,align=left] {Estimates\\ on character\\ sums};
\draw (212.8, -58.800000000000004) rectangle (216.4,-60.400000000000006);
\draw(204.65, -61.5) node[anchor=north west,align=left] {Sums over\\ arbitrary\\ intervals};
\draw (204.65, -61.5) rectangle (207.5,-63.1);
\draw(207.6, -61.5) node[anchor=north west,align=left] {Weyl\\ sums};
\draw (207.6, -61.5) rectangle (209.2,-62.6);
\draw(203.65, -63.300000000000004) node[anchor=north west,align=left] {\large Additive number theory; partitions};
\draw (203.65, -63.300000000000004) rectangle (216.5,-73.60000000000001);
\draw(204.65, -64.30000000000001) node[anchor=north west,align=left] {Partition \\ identities;\\ identities\\ of \\ Rogers-Ramanujan type};
\draw (204.65, -64.30000000000001) rectangle (210.5,-66.9);
\draw(210.6, -64.30000000000001) node[anchor=north west,align=left] {Inverse \\ problems of \\ additive number\\ theory, \\ including sumsets};
\draw (210.6, -64.30000000000001) rectangle (215.45,-66.9);
\draw(204.65, -67.0) node[anchor=north west,align=left] {Goldbach-type\\ theorems; \\ other additive\\ questions \\ involving primes};
\draw (204.65, -67.0) rectangle (209.25,-69.6);
\draw(209.35, -67.0) node[anchor=north west,align=left] {Applications\\ of the\\ Hardy-Littlewood\\ method};
\draw (209.35, -67.0) rectangle (213.95,-69.1);
\draw(204.65, -69.7) node[anchor=north west,align=left] {Partitions; \\ congruences and\\ congruential\\ restrictions};
\draw (204.65, -69.7) rectangle (209.0,-71.8);
\draw(209.1, -69.7) node[anchor=north west,align=left] {Waring’s\\ problem \\ and variants};
\draw (209.1, -69.7) rectangle (212.7,-71.3);
\draw(212.8, -69.7) node[anchor=north west,align=left] {Lattice\\ points \\ in specified\\ regions};
\draw (212.8, -69.7) rectangle (216.4,-71.8);
\draw(204.65, -71.9) node[anchor=north west,align=left] {Elementary\\ theory of\\ partitions};
\draw (204.65, -71.9) rectangle (207.75,-73.5);
\draw(207.85, -71.9) node[anchor=north west,align=left] {Analytic\\ theory of\\ partitions};
\draw (207.85, -71.9) rectangle (210.95,-73.5);
\draw(216.85000000000002, -55.6) node[anchor=north west,align=left] {\large Multiplicative number theory};
\draw (216.85000000000002, -55.6) rectangle (229.75000000000003,-72.80000000000001);
\draw(217.85000000000002, -56.6) node[anchor=north west,align=left] {Distribution \\ functions \\ associated with \\ additive and \\ positive multiplicative\\ functions};
\draw (217.85000000000002, -56.6) rectangle (224.20000000000002,-59.7);
\draw(224.3, -56.6) node[anchor=north west,align=left] {Primes represented\\ by \\ polynomials; other \\ multiplicative\\ structures of\\ polynomial values};
\draw (224.3, -56.6) rectangle (229.65,-59.7);
\draw(217.85000000000002, -59.800000000000004) node[anchor=north west,align=left] {Other results \\ on the distribution\\ of values \\ or the characterization\\ of \\ arithmetic functions};
\draw (217.85000000000002, -59.800000000000004) rectangle (224.20000000000002,-62.900000000000006);
\draw(224.3, -59.800000000000004) node[anchor=north west,align=left] {Asymptotic \\ results on \\ counting functions\\ for algebraic\\ and topological\\ structures};
\draw (224.3, -59.800000000000004) rectangle (229.4,-62.900000000000006);
\draw(217.85000000000002, -63.0) node[anchor=north west,align=left] {Applications \\ of automorphic\\ functions and\\ forms to \\ multiplicative problems};
\draw (217.85000000000002, -63.0) rectangle (224.20000000000002,-65.6);
\draw(224.3, -63.0) node[anchor=north west,align=left] {Distribution\\ of integers\\ in special\\ residue classes};
\draw (224.3, -63.0) rectangle (228.65,-65.1);
\draw(217.85000000000002, -65.7) node[anchor=north west,align=left] {Distribution\\ of integers \\ with specified\\ multiplicative\\ constraints};
\draw (217.85000000000002, -65.7) rectangle (221.95000000000002,-68.3);
\draw(222.05, -65.7) node[anchor=north west,align=left] {Applications\\ of \\ sieve methods};
\draw (222.05, -65.7) rectangle (225.9,-67.3);
\draw(226.00000000000003, -65.7) node[anchor=north west,align=left] {Distribution\\ of primes};
\draw (226.00000000000003, -65.7) rectangle (229.60000000000002,-67.3);
\draw(217.85000000000002, -68.4) node[anchor=north west,align=left] {Asymptotic\\ results \\ on arithmetic\\ functions};
\draw (217.85000000000002, -68.4) rectangle (221.70000000000002,-70.5);
\draw(221.8, -68.4) node[anchor=north west,align=left] {Generalized\\ primes \\ and integers};
\draw (221.8, -68.4) rectangle (225.4,-70.0);
\draw(225.50000000000003, -68.4) node[anchor=north west,align=left] {Primes in\\ congruence\\ classes};
\draw (225.50000000000003, -68.4) rectangle (228.60000000000002,-70.0);
\draw(217.85000000000002, -70.6) node[anchor=north west,align=left] {Rate of \\ growth of\\ arithmetic\\ functions};
\draw (217.85000000000002, -70.6) rectangle (220.95000000000002,-72.69999999999999);
\draw(221.05, -70.6) node[anchor=north west,align=left] {Turán\\ theory};
\draw (221.05, -70.6) rectangle (223.15,-71.69999999999999);
\draw(223.25000000000003, -70.6) node[anchor=north west,align=left] {Sieves};
\draw (223.25000000000003, -70.6) rectangle (225.35000000000002,-71.69999999999999);
\draw(188.95000000000002, -77.5) node[anchor=north west,align=left] {\large Forms and linear algebraic groups};
\draw (188.95000000000002, -77.5) rectangle (201.55,-95.9);
\draw(189.95000000000002, -78.5) node[anchor=north west,align=left] {Analytic theory\\ (Epstein zeta\\ functions; \\ relations with\\ automorphic \\ forms and functions)};
\draw (189.95000000000002, -78.5) rectangle (195.55,-81.6);
\draw(195.65, -78.5) node[anchor=north west,align=left] {Sums of squares\\ and representations\\ by other\\ particular\\ quadratic forms};
\draw (195.65, -78.5) rectangle (201.0,-81.1);
\draw(189.95000000000002, -81.7) node[anchor=north west,align=left] {General ternary\\ and quaternary \\ quadratic forms;\\ forms of more \\ than two variables};
\draw (189.95000000000002, -81.7) rectangle (195.05,-84.3);
\draw(195.15, -81.7) node[anchor=north west,align=left] {General \\ binary quadratic\\ forms};
\draw (195.15, -81.7) rectangle (199.75,-83.3);
\draw(189.95000000000002, -84.4) node[anchor=north west,align=left] {Galois \\ cohomology of \\ linear algebraic\\ groups};
\draw (189.95000000000002, -84.4) rectangle (194.55,-86.5);
\draw(194.65, -84.4) node[anchor=north west,align=left] {Algebraic \\ theory of \\ quadratic \\ forms; Witt \\ groups and rings};
\draw (194.65, -84.4) rectangle (199.25,-87.0);
\draw(189.95000000000002, -87.1) node[anchor=north west,align=left] {Class numbers\\ of quadratic\\ and \\ Hermitian forms};
\draw (189.95000000000002, -87.1) rectangle (194.3,-89.19999999999999);
\draw(194.4, -87.1) node[anchor=north west,align=left] {\(K\)-theory\\ of quadratic\\ and \\ Hermitian forms};
\draw (194.4, -87.1) rectangle (198.75,-89.19999999999999);
\draw(189.95000000000002, -89.3) node[anchor=north west,align=left] {Quadratic\\ forms\\ over \\ general fields};
\draw (189.95000000000002, -89.3) rectangle (194.05,-91.39999999999999);
\draw(194.15, -89.3) node[anchor=north west,align=left] {Bilinear\\ and Hermitian\\ forms};
\draw (194.15, -89.3) rectangle (198.0,-90.89999999999999);
\draw(198.10000000000002, -89.3) node[anchor=north west,align=left] {Quadratic\\ forms over\\ local rings\\ and fields};
\draw (198.10000000000002, -89.3) rectangle (201.45000000000002,-91.39999999999999);
\draw(189.95000000000002, -91.5) node[anchor=north west,align=left] {Forms of \\ degree higher\\ than two};
\draw (189.95000000000002, -91.5) rectangle (193.8,-93.1);
\draw(193.9, -91.5) node[anchor=north west,align=left] {Quadratic \\ forms over \\ global rings\\ and fields};
\draw (193.9, -91.5) rectangle (197.5,-93.6);
\draw(197.60000000000002, -91.5) node[anchor=north west,align=left] {\(p\)-adic\\ theory};
\draw (197.60000000000002, -91.5) rectangle (200.70000000000002,-93.1);
\draw(189.95000000000002, -93.7) node[anchor=north west,align=left] {Forms \\ over real\\ fields};
\draw (189.95000000000002, -93.7) rectangle (192.8,-95.3);
\draw(192.9, -93.7) node[anchor=north west,align=left] {Classical\\ groups};
\draw (192.9, -93.7) rectangle (195.75,-94.8);
\draw(195.85000000000002, -93.7) node[anchor=north west,align=left] {Quadratic\\ spaces;\\ Clifford\\ algebras};
\draw (195.85000000000002, -93.7) rectangle (198.70000000000002,-95.8);
\draw(201.65, -77.5) node[anchor=north west,align=left] {\large Elementary number theory};
\draw (201.65, -77.5) rectangle (212.55,-87.8);
\draw(202.65, -78.5) node[anchor=north west,align=left] {Multiplicative\\ structure; \\ Euclidean algorithm;\\ greatest\\ common divisors};
\draw (202.65, -78.5) rectangle (208.25,-81.1);
\draw(208.35, -78.5) node[anchor=north west,align=left] {Factorization;\\ primality};
\draw (208.35, -78.5) rectangle (212.45,-80.1);
\draw(202.65, -81.2) node[anchor=north west,align=left] {Arithmetic\\ functions;\\ related \\ numbers; inversion\\ formulas};
\draw (202.65, -81.2) rectangle (207.75,-83.8);
\draw(207.85, -81.2) node[anchor=north west,align=left] {Congruences;\\ primitive\\ roots; \\ residue systems};
\draw (207.85, -81.2) rectangle (212.2,-83.3);
\draw(202.65, -83.9) node[anchor=north west,align=left] {Radix \\ representation;\\ digital\\ problems};
\draw (202.65, -83.9) rectangle (207.0,-86.0);
\draw(207.1, -83.9) node[anchor=north west,align=left] {Other \\ number \\ representations};
\draw (207.1, -83.9) rectangle (211.45,-85.5);
\draw(202.65, -86.1) node[anchor=north west,align=left] {Power \\ residues, \\ reciprocity};
\draw (202.65, -86.1) rectangle (206.0,-87.69999999999999);
\draw(206.1, -86.1) node[anchor=north west,align=left] {Continued\\ fractions};
\draw (206.1, -86.1) rectangle (208.95,-87.69999999999999);
\draw(209.05, -86.1) node[anchor=north west,align=left] {Primes};
\draw (209.05, -86.1) rectangle (211.15,-87.19999999999999);
\draw(212.65, -77.5) node[anchor=north west,align=left] {\large Computational number theory};
\draw (212.65, -77.5) rectangle (223.05,-90.0);
\draw(213.65, -78.5) node[anchor=north west,align=left] {Continued \\ fraction \\ calculations \\ (number-theoretic\\ aspects)};
\draw (213.65, -78.5) rectangle (218.5,-81.1);
\draw(218.6, -78.5) node[anchor=north west,align=left] {Factorization};
\draw (218.6, -78.5) rectangle (222.45,-79.6);
\draw(213.65, -81.2) node[anchor=north west,align=left] {Number-theoretic\\ algorithms;\\ complexity};
\draw (213.65, -81.2) rectangle (218.25,-83.3);
\draw(218.35, -81.2) node[anchor=north west,align=left] {Evaluation\\ of \\ number-theoretic\\ constants};
\draw (218.35, -81.2) rectangle (222.95,-83.3);
\draw(213.65, -83.4) node[anchor=north west,align=left] {Analytic\\ computations};
\draw (213.65, -83.4) rectangle (217.25,-85.0);
\draw(217.35, -83.4) node[anchor=north west,align=left] {Algebraic\\ number \\ theory \\ computations};
\draw (217.35, -83.4) rectangle (220.95,-85.5);
\draw(213.65, -85.6) node[anchor=north west,align=left] {Computer \\ solution of\\ Diophantine\\ equations};
\draw (213.65, -85.6) rectangle (217.0,-87.69999999999999);
\draw(217.1, -85.6) node[anchor=north west,align=left] {Calculation\\ of integer\\ sequences};
\draw (217.1, -85.6) rectangle (220.45,-87.19999999999999);
\draw(213.65, -87.8) node[anchor=north west,align=left] {Values of\\ arithmetic\\ functions;\\ tables};
\draw (213.65, -87.8) rectangle (216.75,-89.89999999999999);
\draw(216.85, -87.8) node[anchor=north west,align=left] {Primality};
\draw (216.85, -87.8) rectangle (219.7,-88.89999999999999);
\draw(223.15, -77.5) node[anchor=north west,align=left] {\large Geometry of numbers};
\draw (223.15, -77.5) rectangle (233.05,-87.8);
\draw(224.15, -78.5) node[anchor=north west,align=left] {Lattices and\\ convex bodies\\ (number-theoretic\\ aspects)};
\draw (224.15, -78.5) rectangle (229.0,-80.6);
\draw(229.1, -78.5) node[anchor=north west,align=left] {Relations\\ with \\ coding theory};
\draw (229.1, -78.5) rectangle (232.95,-80.1);
\draw(224.15, -80.7) node[anchor=north west,align=left] {Lattice \\ packing and \\ covering \\ (number-theoretic\\ aspects)};
\draw (224.15, -80.7) rectangle (229.0,-83.3);
\draw(229.1, -80.7) node[anchor=north west,align=left] {Automorphism\\ groups\\ of lattices};
\draw (229.1, -80.7) rectangle (232.7,-82.3);
\draw(224.15, -83.4) node[anchor=north west,align=left] {Quadratic \\ forms (reduction\\ theory,\\ extreme \\ forms, etc.)};
\draw (224.15, -83.4) rectangle (228.75,-86.0);
\draw(228.85, -83.4) node[anchor=north west,align=left] {Mean value\\ and transfer\\ theorems};
\draw (228.85, -83.4) rectangle (232.45,-85.0);
\draw(224.15, -86.1) node[anchor=north west,align=left] {Nonconvex\\ bodies};
\draw (224.15, -86.1) rectangle (227.0,-87.19999999999999);
\draw(227.1, -86.1) node[anchor=north west,align=left] {Products\\ of linear\\ forms};
\draw (227.1, -86.1) rectangle (229.95,-87.69999999999999);
\draw(230.05, -86.1) node[anchor=north west,align=left] {Minima\\ of forms};
\draw (230.05, -86.1) rectangle (232.65,-87.19999999999999);
\draw(188.95000000000002, -96.0) node[anchor=north west,align=left] {\large Diophantine equations};
\draw (188.95000000000002, -96.0) rectangle (198.10000000000002,-113.6);
\draw(189.95000000000002, -97.0) node[anchor=north west,align=left] {Higher \\ degree equations;\\ Fermat’s\\ equation};
\draw (189.95000000000002, -97.0) rectangle (194.8,-99.1);
\draw(194.9, -97.0) node[anchor=north west,align=left] {Rational\\ numbers \\ as sums of\\ fractions};
\draw (194.9, -97.0) rectangle (198.0,-99.1);
\draw(189.95000000000002, -99.2) node[anchor=north west,align=left] {Counting \\ solutions \\ of Diophantine\\ equations};
\draw (189.95000000000002, -99.2) rectangle (194.05,-101.3);
\draw(194.15, -99.2) node[anchor=north west,align=left] {\(p\)-adic\\ and \\ power \\ series fields};
\draw (194.15, -99.2) rectangle (198.0,-101.3);
\draw(189.95000000000002, -101.4) node[anchor=north west,align=left] {Multiplicative\\ and\\ norm form\\ equations};
\draw (189.95000000000002, -101.4) rectangle (194.05,-103.5);
\draw(194.15, -101.4) node[anchor=north west,align=left] {Quadratic \\ and bilinear\\ Diophantine\\ equations};
\draw (194.15, -101.4) rectangle (197.75,-103.5);
\draw(189.95000000000002, -103.6) node[anchor=north west,align=left] {Diophantine\\ equations\\ in \\ many variables};
\draw (189.95000000000002, -103.6) rectangle (194.05,-105.69999999999999);
\draw(194.15, -103.6) node[anchor=north west,align=left] {Diophantine\\ inequalities};
\draw (194.15, -103.6) rectangle (197.75,-105.19999999999999);
\draw(189.95000000000002, -105.8) node[anchor=north west,align=left] {Representation\\ problems};
\draw (189.95000000000002, -105.8) rectangle (194.05,-107.39999999999999);
\draw(194.15, -105.8) node[anchor=north west,align=left] {Linear \\ Diophantine\\ equations};
\draw (194.15, -105.8) rectangle (197.5,-107.39999999999999);
\draw(189.95000000000002, -107.5) node[anchor=north west,align=left] {Cubic and\\ quartic\\ Diophantine\\ equations};
\draw (189.95000000000002, -107.5) rectangle (193.3,-109.6);
\draw(193.4, -107.5) node[anchor=north west,align=left] {Thue-Mahler\\ equations};
\draw (193.4, -107.5) rectangle (196.75,-109.1);
\draw(189.95000000000002, -109.7) node[anchor=north west,align=left] {Exponential\\ Diophantine\\ equations};
\draw (189.95000000000002, -109.7) rectangle (193.3,-111.8);
\draw(193.4, -109.7) node[anchor=north west,align=left] {Congruences\\ in many\\ variables};
\draw (193.4, -109.7) rectangle (196.75,-111.3);
\draw(189.95000000000002, -111.9) node[anchor=north west,align=left] {The \\ Frobenius\\ problem};
\draw (189.95000000000002, -111.9) rectangle (192.8,-113.5);
\draw(235.78000000000003, -1) node[anchor=north west,align=left] {\LARGE Group theory and generalizations};
\draw (235.78000000000003, -1) rectangle (281.63000000000005,-68.6);
\draw(236.78000000000003, -2) node[anchor=north west,align=left] {\large Groupoids (i.e. small categories in which all morphisms are isomorphisms)};
\draw (236.78000000000003, -2) rectangle (260.01000000000005,-5.7);
\draw(237.78000000000003, -3) node[anchor=north west,align=left] {Groupoids (i.e.\\ small categories\\ in which\\ all morphisms\\ are isomorphisms)};
\draw (237.78000000000003, -3) rectangle (242.63000000000002,-5.6);
\draw(260.11, -2) node[anchor=north west,align=left] {\large Structure and classification of infinite or finite groups};
\draw (260.11, -2) rectangle (281.41,-11.1);
\draw(261.11, -3) node[anchor=north west,align=left] {Free products of \\ groups, free products\\ with amalgamation,\\ Higman-Neumann-Neumann\\ extensions,\\ and generalizations};
\draw (261.11, -3) rectangle (267.21000000000004,-6.1);
\draw(267.31, -3) node[anchor=north west,align=left] {Chains and \\ lattices of \\ subgroups, subnormal\\ subgroups};
\draw (267.31, -3) rectangle (272.91,-5.1);
\draw(273.01, -3) node[anchor=north west,align=left] {Extensions,\\ wreath products,\\ and other\\ compositions\\ of groups};
\draw (273.01, -3) rectangle (277.61,-5.6);
\draw(277.71000000000004, -3) node[anchor=north west,align=left] {Groups \\ with a \\ \(BN\)-pair;\\ buildings};
\draw (277.71000000000004, -3) rectangle (281.31000000000006,-5.1);
\draw(261.11, -6.2) node[anchor=north west,align=left] {Residual \\ properties and \\ generalizations;\\ residually\\ finite groups};
\draw (261.11, -6.2) rectangle (265.71000000000004,-8.8);
\draw(265.81, -6.2) node[anchor=north west,align=left] {Automorphisms\\ of \\ infinite groups};
\draw (265.81, -6.2) rectangle (270.16,-7.800000000000001);
\draw(270.26, -6.2) node[anchor=north west,align=left] {Quasivarieties\\ and\\ varieties\\ of groups};
\draw (270.26, -6.2) rectangle (274.36,-8.3);
\draw(274.46000000000004, -6.2) node[anchor=north west,align=left] {Free \\ nonabelian\\ groups};
\draw (274.46000000000004, -6.2) rectangle (277.56000000000006,-7.800000000000001);
\draw(277.66, -6.2) node[anchor=north west,align=left] {Local \\ properties\\ of groups};
\draw (277.66, -6.2) rectangle (280.76000000000005,-7.800000000000001);
\draw(261.11, -8.9) node[anchor=north west,align=left] {General \\ structure\\ theorems\\ for groups};
\draw (261.11, -8.9) rectangle (264.21000000000004,-11.0);
\draw(264.31, -8.9) node[anchor=north west,align=left] {Conjugacy\\ classes\\ for groups};
\draw (264.31, -8.9) rectangle (267.41,-10.5);
\draw(267.51, -8.9) node[anchor=north west,align=left] {Subgroup\\ theorems;\\ subgroup\\ growth};
\draw (267.51, -8.9) rectangle (270.36,-11.0);
\draw(270.46000000000004, -8.9) node[anchor=north west,align=left] {Limits,\\ profinite\\ groups};
\draw (270.46000000000004, -8.9) rectangle (273.31000000000006,-10.5);
\draw(273.41, -8.9) node[anchor=north west,align=left] {Maximal\\ subgroups};
\draw (273.41, -8.9) rectangle (276.26000000000005,-10.5);
\draw(276.36, -8.9) node[anchor=north west,align=left] {Groups\\ acting\\ on trees};
\draw (276.36, -8.9) rectangle (278.96000000000004,-10.5);
\draw(279.06, -8.9) node[anchor=north west,align=left] {Simple\\ groups};
\draw (279.06, -8.9) rectangle (281.16,-10.0);
\draw(236.78000000000003, -5.8) node[anchor=north west,align=left] {\large Connections of group theory with homological algebra and category theory};
\draw (236.78000000000003, -5.8) rectangle (259.70000000000005,-9.0);
\draw(237.78000000000003, -6.8) node[anchor=north west,align=left] {Homological\\ methods\\ in \\ group theory};
\draw (237.78000000000003, -6.8) rectangle (241.38000000000002,-8.9);
\draw(241.48000000000002, -6.8) node[anchor=north west,align=left] {Cohomology\\ of groups};
\draw (241.48000000000002, -6.8) rectangle (244.58,-8.4);
\draw(244.68000000000004, -6.8) node[anchor=north west,align=left] {Category\\ of\\ groups};
\draw (244.68000000000004, -6.8) rectangle (247.28000000000003,-8.4);
\draw(236.78000000000003, -9.1) node[anchor=north west,align=left] {\large Computational methods\\ for problems \\ pertaining to group theory};
\draw (236.78000000000003, -9.1) rectangle (245.44000000000003,-10.7);
\draw(236.78000000000003, -11.2) node[anchor=north west,align=left] {\large Special aspects of infinite or finite groups};
\draw (236.78000000000003, -11.2) rectangle (253.63000000000002,-31.8);
\draw(237.78000000000003, -12.2) node[anchor=north west,align=left] {Word problems, \\ other decision \\ problems, connections\\ with logic \\ and automata \\ (group-theoretic aspects)};
\draw (237.78000000000003, -12.2) rectangle (244.63000000000002,-15.299999999999999);
\draw(244.73000000000002, -12.2) node[anchor=north west,align=left] {Cancellation\\ theory of \\ groups; application\\ of van \\ Kampen diagrams};
\draw (244.73000000000002, -12.2) rectangle (250.08,-14.799999999999999);
\draw(250.18000000000004, -12.2) node[anchor=north west,align=left] {Groups of\\ finite \\ Morley rank};
\draw (250.18000000000004, -12.2) rectangle (253.53000000000003,-13.799999999999999);
\draw(237.78000000000003, -15.399999999999999) node[anchor=north west,align=left] {Generators,\\ relations, \\ and presentations\\ of groups};
\draw (237.78000000000003, -15.399999999999999) rectangle (242.63000000000002,-17.5);
\draw(242.73000000000002, -15.399999999999999) node[anchor=north west,align=left] {Representations\\ of groups\\ as automorphism\\ groups of\\ algebraic systems};
\draw (242.73000000000002, -15.399999999999999) rectangle (247.58,-18.0);
\draw(247.68000000000004, -15.399999999999999) node[anchor=north west,align=left] {Algebraic \\ geometry over \\ groups; equations\\ over groups};
\draw (247.68000000000004, -15.399999999999999) rectangle (252.53000000000003,-17.5);
\draw(237.78000000000003, -18.1) node[anchor=north west,align=left] {Generalizations\\ of solvable\\ and \\ nilpotent groups};
\draw (237.78000000000003, -18.1) rectangle (242.38000000000002,-20.200000000000003);
\draw(242.48000000000002, -18.1) node[anchor=north west,align=left] {Fundamental \\ groups and their\\ automorphisms\\ (group-theoretic\\ aspects)};
\draw (242.48000000000002, -18.1) rectangle (247.08,-20.700000000000003);
\draw(247.18000000000004, -18.1) node[anchor=north west,align=left] {Reflection\\ and Coxeter\\ groups \\ (group-theoretic\\ aspects)};
\draw (247.18000000000004, -18.1) rectangle (251.78000000000003,-20.700000000000003);
\draw(237.78000000000003, -20.8) node[anchor=north west,align=left] {Ordered \\ groups \\ (group-theoretic\\ aspects)};
\draw (237.78000000000003, -20.8) rectangle (242.38000000000002,-22.900000000000002);
\draw(242.48000000000002, -20.8) node[anchor=north west,align=left] {Derived series,\\ central\\ series, and\\ generalizations\\ for groups};
\draw (242.48000000000002, -20.8) rectangle (246.83,-23.400000000000002);
\draw(246.93000000000004, -20.8) node[anchor=north west,align=left] {Other classes\\ of groups\\ defined by \\ subgroup chains};
\draw (246.93000000000004, -20.8) rectangle (251.28000000000003,-22.900000000000002);
\draw(237.78000000000003, -23.5) node[anchor=north west,align=left] {FC-groups\\ and \\ their \\ generalizations};
\draw (237.78000000000003, -23.5) rectangle (242.13000000000002,-25.6);
\draw(242.23000000000002, -23.5) node[anchor=north west,align=left] {Solvable\\ groups, \\ supersolvable\\ groups};
\draw (242.23000000000002, -23.5) rectangle (246.08,-25.6);
\draw(246.18000000000004, -23.5) node[anchor=north west,align=left] {Periodic\\ groups; \\ locally \\ finite groups};
\draw (246.18000000000004, -23.5) rectangle (250.03000000000003,-25.6);
\draw(250.13000000000002, -23.5) node[anchor=north west,align=left] {Associated\\ Lie \\ structures \\ for groups};
\draw (250.13000000000002, -23.5) rectangle (253.48000000000002,-25.6);
\draw(237.78000000000003, -25.7) node[anchor=north west,align=left] {Hyperbolic \\ groups and \\ nonpositively\\ curved groups};
\draw (237.78000000000003, -25.7) rectangle (241.63000000000002,-27.8);
\draw(241.73000000000002, -25.7) node[anchor=north west,align=left] {Automorphism\\ groups\\ of groups};
\draw (241.73000000000002, -25.7) rectangle (245.33,-27.3);
\draw(245.43000000000004, -25.7) node[anchor=north west,align=left] {Braid \\ groups; \\ Artin groups};
\draw (245.43000000000004, -25.7) rectangle (249.03000000000003,-27.3);
\draw(249.13000000000002, -25.7) node[anchor=north west,align=left] {Other groups\\ related \\ to topology\\ or analysis};
\draw (249.13000000000002, -25.7) rectangle (252.73000000000002,-27.8);
\draw(237.78000000000003, -27.9) node[anchor=north west,align=left] {Commutator\\ calculus};
\draw (237.78000000000003, -27.9) rectangle (240.88000000000002,-29.5);
\draw(240.98000000000002, -27.9) node[anchor=north west,align=left] {Formations\\ of groups,\\ Fitting\\ classes};
\draw (240.98000000000002, -27.9) rectangle (244.08,-30.0);
\draw(244.18000000000004, -27.9) node[anchor=north west,align=left] {Engel \\ conditions};
\draw (244.18000000000004, -27.9) rectangle (247.28000000000003,-29.0);
\draw(247.38000000000002, -27.9) node[anchor=north west,align=left] {Asymptotic\\ properties\\ of groups};
\draw (247.38000000000002, -27.9) rectangle (250.48000000000002,-29.5);
\draw(250.58000000000004, -27.9) node[anchor=north west,align=left] {Nilpotent\\ groups};
\draw (250.58000000000004, -27.9) rectangle (253.43000000000004,-29.0);
\draw(237.78000000000003, -30.099999999999998) node[anchor=north west,align=left] {Geometric\\ group\\ theory};
\draw (237.78000000000003, -30.099999999999998) rectangle (240.63000000000002,-31.7);
\draw(253.73000000000002, -11.2) node[anchor=north west,align=left] {\large Linear algebraic groups and related topics};
\draw (253.73000000000002, -11.2) rectangle (269.83000000000004,-24.2);
\draw(254.73000000000002, -12.2) node[anchor=north west,align=left] {Quantum groups\\ (quantized \\ function algebras)\\ and their \\ representations};
\draw (254.73000000000002, -12.2) rectangle (259.83000000000004,-14.799999999999999);
\draw(259.93, -12.2) node[anchor=north west,align=left] {Linear algebraic\\ groups\\ over adèles\\ and other \\ rings and schemes};
\draw (259.93, -12.2) rectangle (264.78000000000003,-14.799999999999999);
\draw(264.88, -12.2) node[anchor=north west,align=left] {Exceptionalgroups};
\draw (264.88, -12.2) rectangle (269.73,-13.799999999999999);
\draw(254.73000000000002, -14.899999999999999) node[anchor=north west,align=left] {Representation\\ theory for\\ linear \\ algebraic groups};
\draw (254.73000000000002, -14.899999999999999) rectangle (259.33000000000004,-17.0);
\draw(259.43, -14.899999999999999) node[anchor=north west,align=left] {Structure \\ theory for \\ linear algebraic\\ groups};
\draw (259.43, -14.899999999999999) rectangle (264.03000000000003,-17.0);
\draw(264.13, -14.899999999999999) node[anchor=north west,align=left] {Cohomology\\ theory for\\ linear \\ algebraic groups};
\draw (264.13, -14.899999999999999) rectangle (268.73,-17.0);
\draw(254.73000000000002, -17.1) node[anchor=north west,align=left] {Linear \\ algebraic groups\\ over \\ arbitrary fields};
\draw (254.73000000000002, -17.1) rectangle (259.33000000000004,-19.200000000000003);
\draw(259.43, -17.1) node[anchor=north west,align=left] {Linear algebraic\\ groups over\\ the reals, the\\ complexes, \\ the quaternions};
\draw (259.43, -17.1) rectangle (264.03000000000003,-19.700000000000003);
\draw(264.13, -17.1) node[anchor=north west,align=left] {Linear algebraic\\ groups\\ over local \\ fields and \\ their integers};
\draw (264.13, -17.1) rectangle (268.73,-19.700000000000003);
\draw(254.73000000000002, -19.8) node[anchor=north west,align=left] {Linear algebraic\\ groups\\ over global\\ fields and \\ their integers};
\draw (254.73000000000002, -19.8) rectangle (259.33000000000004,-22.400000000000002);
\draw(259.43, -19.8) node[anchor=north west,align=left] {Linear \\ algebraic \\ groups over \\ finite fields};
\draw (259.43, -19.8) rectangle (263.28000000000003,-21.900000000000002);
\draw(263.38, -19.8) node[anchor=north west,align=left] {Applications\\ of linear\\ algebraic\\ groups to \\ the sciences};
\draw (263.38, -19.8) rectangle (266.98,-22.400000000000002);
\draw(254.73000000000002, -22.5) node[anchor=north west,align=left] {Schur and\\ \(q\)-Schur\\ algebras};
\draw (254.73000000000002, -22.5) rectangle (258.08000000000004,-24.1);
\draw(258.18, -22.5) node[anchor=north west,align=left] {Kac-Moody\\ groups};
\draw (258.18, -22.5) rectangle (261.03000000000003,-23.6);
\draw(253.73000000000002, -24.3) node[anchor=north west,align=left] {\large Probabilistic methods in group theory};
\draw (253.73000000000002, -24.3) rectangle (265.8,-27.5);
\draw(254.73000000000002, -25.3) node[anchor=north west,align=left] {Probabilistic\\ methods\\ in \\ group theory};
\draw (254.73000000000002, -25.3) rectangle (258.58000000000004,-27.400000000000002);
\draw(253.73000000000002, -27.6) node[anchor=north west,align=left] {\large History of\\ group theory};
\draw (253.73000000000002, -27.6) rectangle (258.05,-28.700000000000003);
\draw(269.93, -11.2) node[anchor=north west,align=left] {\large Other generalizations of groups};
\draw (269.93, -11.2) rectangle (281.53000000000003,-17.299999999999997);
\draw(270.93, -12.2) node[anchor=north west,align=left] {Sets with a\\ single \\ binary operation\\ (groupoids)};
\draw (270.93, -12.2) rectangle (275.53000000000003,-14.299999999999999);
\draw(275.63, -12.2) node[anchor=north west,align=left] {Ternary systems\\ (heaps,\\ semiheaps, \\ heapoids, etc.)};
\draw (275.63, -12.2) rectangle (279.98,-14.299999999999999);
\draw(270.93, -14.399999999999999) node[anchor=north west,align=left] {\(n\)-ary\\ systems \\ \((n\ge~3)\)};
\draw (270.93, -14.399999999999999) rectangle (274.53000000000003,-15.999999999999998);
\draw(274.63, -14.399999999999999) node[anchor=north west,align=left] {Loops,\\ quasigroups};
\draw (274.63, -14.399999999999999) rectangle (277.98,-15.999999999999998);
\draw(278.08, -14.399999999999999) node[anchor=north west,align=left] {Hypergroups};
\draw (278.08, -14.399999999999999) rectangle (281.43,-15.499999999999998);
\draw(270.93, -16.1) node[anchor=north west,align=left] {Fuzzy\\ groups};
\draw (270.93, -16.1) rectangle (273.03000000000003,-17.200000000000003);
\draw(236.78000000000003, -31.900000000000002) node[anchor=north west,align=left] {\large Representation theory of groups};
\draw (236.78000000000003, -31.900000000000002) rectangle (250.43000000000004,-49.300000000000004);
\draw(237.78000000000003, -32.900000000000006) node[anchor=north west,align=left] {Group rings of\\ infinite groups\\ and their \\ modules \\ (group-theoretic aspects)};
\draw (237.78000000000003, -32.900000000000006) rectangle (244.63000000000002,-35.50000000000001);
\draw(244.73000000000002, -32.900000000000006) node[anchor=north west,align=left] {Representations\\ of \\ infinite symmetric\\ groups};
\draw (244.73000000000002, -32.900000000000006) rectangle (249.83,-35.00000000000001);
\draw(237.78000000000003, -35.6) node[anchor=north west,align=left] {Group rings of\\ finite groups\\ and their \\ modules (group-theoretic\\ aspects)};
\draw (237.78000000000003, -35.6) rectangle (244.38000000000002,-38.2);
\draw(244.48000000000002, -35.6) node[anchor=north west,align=left] {Applications of\\ group representations\\ to physics\\ and other \\ areas of science};
\draw (244.48000000000002, -35.6) rectangle (250.33,-38.2);
\draw(237.78000000000003, -38.300000000000004) node[anchor=north west,align=left] {Representationsof\\ sporadic\\ groups};
\draw (237.78000000000003, -38.300000000000004) rectangle (242.63000000000002,-40.400000000000006);
\draw(242.73000000000002, -38.300000000000004) node[anchor=north west,align=left] {Representations\\ of \\ finite \\ symmetric groups};
\draw (242.73000000000002, -38.300000000000004) rectangle (247.33,-40.400000000000006);
\draw(237.78000000000003, -40.5) node[anchor=north west,align=left] {Hecke \\ algebras and \\ their \\ representations};
\draw (237.78000000000003, -40.5) rectangle (242.13000000000002,-42.6);
\draw(242.23000000000002, -40.5) node[anchor=north west,align=left] {Integral \\ representations\\ of \\ finite groups};
\draw (242.23000000000002, -40.5) rectangle (246.58,-42.6);
\draw(237.78000000000003, -42.7) node[anchor=north west,align=left] {\(p\)-adic\\ representations\\ of\\ finite groups};
\draw (237.78000000000003, -42.7) rectangle (242.13000000000002,-44.800000000000004);
\draw(242.23000000000002, -42.7) node[anchor=north west,align=left] {Integral \\ representations\\ of \\ infinite groups};
\draw (242.23000000000002, -42.7) rectangle (246.58,-44.800000000000004);
\draw(237.78000000000003, -44.900000000000006) node[anchor=north west,align=left] {Ordinary\\ representations\\ and\\ characters};
\draw (237.78000000000003, -44.900000000000006) rectangle (242.13000000000002,-47.00000000000001);
\draw(242.23000000000002, -44.900000000000006) node[anchor=north west,align=left] {Modular \\ representations\\ and\\ characters};
\draw (242.23000000000002, -44.900000000000006) rectangle (246.58,-47.00000000000001);
\draw(237.78000000000003, -47.1) node[anchor=north west,align=left] {Projective\\ representations\\ and\\ multipliers};
\draw (237.78000000000003, -47.1) rectangle (242.13000000000002,-49.2);
\draw(242.23000000000002, -47.1) node[anchor=north west,align=left] {Representations\\ of \\ finite groups\\ of Lie type};
\draw (242.23000000000002, -47.1) rectangle (246.58,-49.2);
\draw(250.53000000000003, -31.900000000000002) node[anchor=north west,align=left] {\large Permutation groups};
\draw (250.53000000000003, -31.900000000000002) rectangle (261.93,-43.400000000000006);
\draw(251.53000000000003, -32.900000000000006) node[anchor=north west,align=left] {Finite automorphism\\ groups of\\ algebraic, \\ geometric, or \\ combinatorial structures};
\draw (251.53000000000003, -32.900000000000006) rectangle (258.13000000000005,-35.50000000000001);
\draw(258.23, -32.900000000000006) node[anchor=north west,align=left] {Infinite\\ automorphism\\ groups};
\draw (258.23, -32.900000000000006) rectangle (261.83000000000004,-34.50000000000001);
\draw(251.53000000000003, -35.6) node[anchor=north west,align=left] {General \\ theory for \\ finite permutation\\ groups};
\draw (251.53000000000003, -35.6) rectangle (256.63000000000005,-37.7);
\draw(256.73, -35.6) node[anchor=north west,align=left] {General \\ theory for \\ infinite \\ permutation groups};
\draw (256.73, -35.6) rectangle (261.83000000000004,-37.7);
\draw(251.53000000000003, -37.800000000000004) node[anchor=north west,align=left] {Characterization\\ theorems\\ for permutation\\ groups};
\draw (251.53000000000003, -37.800000000000004) rectangle (256.13000000000005,-39.900000000000006);
\draw(256.23, -37.800000000000004) node[anchor=north west,align=left] {Multiply\\ transitive\\ infinite groups};
\draw (256.23, -37.800000000000004) rectangle (260.58000000000004,-39.900000000000006);
\draw(251.53000000000003, -40.0) node[anchor=north west,align=left] {Subgroups\\ of symmetric\\ groups};
\draw (251.53000000000003, -40.0) rectangle (255.13000000000002,-41.6);
\draw(255.23000000000002, -40.0) node[anchor=north west,align=left] {Multiply\\ transitive\\ finite\\ groups};
\draw (255.23000000000002, -40.0) rectangle (258.33000000000004,-42.1);
\draw(258.43, -40.0) node[anchor=north west,align=left] {Primitive\\ groups};
\draw (258.43, -40.0) rectangle (261.28000000000003,-41.1);
\draw(251.53000000000003, -42.2) node[anchor=north west,align=left] {Symmetric\\ groups};
\draw (251.53000000000003, -42.2) rectangle (254.38000000000002,-43.300000000000004);
\draw(250.53000000000003, -43.50000000000001) node[anchor=north west,align=left] {\large Foundations};
\draw (250.53000000000003, -43.50000000000001) rectangle (259.43,-48.900000000000006);
\draw(251.53000000000003, -44.50000000000001) node[anchor=north west,align=left] {Metamathematical\\ considerations \\ in group theory};
\draw (251.53000000000003, -44.50000000000001) rectangle (256.13000000000005,-46.60000000000001);
\draw(251.53000000000003, -46.70000000000001) node[anchor=north west,align=left] {Axiomatics\\ and elementary\\ properties\\ of groups};
\draw (251.53000000000003, -46.70000000000001) rectangle (255.63000000000002,-48.80000000000001);
\draw(255.73000000000002, -46.70000000000001) node[anchor=north west,align=left] {Applications\\ of \\ logic to \\ group theory};
\draw (255.73000000000002, -46.70000000000001) rectangle (259.33000000000004,-48.80000000000001);
\draw(262.03000000000003, -31.900000000000002) node[anchor=north west,align=left] {\large Abstract finite groups};
\draw (262.03000000000003, -31.900000000000002) rectangle (273.18,-48.10000000000001);
\draw(263.03000000000003, -32.900000000000006) node[anchor=north west,align=left] {Finite solvable\\ groups, theory \\ of formations, \\ Schunck classes, \\ Fitting classes,\\ \(\pi\)-length, ranks};
\draw (263.03000000000003, -32.900000000000006) rectangle (268.88000000000005,-36.00000000000001);
\draw(268.98, -32.900000000000006) node[anchor=north west,align=left] {Finite simple\\ groups \\ and their \\ classification};
\draw (268.98, -32.900000000000006) rectangle (273.08000000000004,-35.00000000000001);
\draw(263.03000000000003, -36.1) node[anchor=north west,align=left] {Arithmetic and\\ combinatorial\\ problems \\ involving abstract\\ finite groups};
\draw (263.03000000000003, -36.1) rectangle (268.13000000000005,-38.7);
\draw(268.23, -36.1) node[anchor=north west,align=left] {Sylow subgroups,\\ Sylow \\ properties, \\ \(\pi\)-groups, \\ \(\pi\)-structure};
\draw (268.23, -36.1) rectangle (273.08000000000004,-38.7);
\draw(263.03000000000003, -38.800000000000004) node[anchor=north west,align=left] {Simple \\ groups: sporadic\\ groups};
\draw (263.03000000000003, -38.800000000000004) rectangle (267.63000000000005,-40.400000000000006);
\draw(267.73, -38.800000000000004) node[anchor=north west,align=left] {Simple groups:\\ alternating\\ groups\\ and groups\\ of Lie type};
\draw (267.73, -38.800000000000004) rectangle (271.83000000000004,-41.400000000000006);
\draw(263.03000000000003, -41.5) node[anchor=north west,align=left] {Special \\ subgroups \\ (Frattini, \\ Fitting, etc.)};
\draw (263.03000000000003, -41.5) rectangle (267.13000000000005,-43.6);
\draw(267.23, -41.5) node[anchor=north west,align=left] {Subnormal \\ subgroups of\\ abstract \\ finite groups};
\draw (267.23, -41.5) rectangle (271.08000000000004,-43.6);
\draw(263.03000000000003, -43.7) node[anchor=north west,align=left] {Products of\\ subgroups\\ of abstract\\ finite groups};
\draw (263.03000000000003, -43.7) rectangle (266.88000000000005,-45.800000000000004);
\draw(266.98, -43.7) node[anchor=north west,align=left] {Automorphisms\\ of \\ abstract \\ finite groups};
\draw (266.98, -43.7) rectangle (270.83000000000004,-45.800000000000004);
\draw(263.03000000000003, -45.900000000000006) node[anchor=north west,align=left] {Finite \\ nilpotent \\ groups, \\ \(p\)-groups};
\draw (263.03000000000003, -45.900000000000006) rectangle (266.63000000000005,-48.00000000000001);
\draw(266.73, -45.900000000000006) node[anchor=north west,align=left] {Series and\\ lattices\\ of subgroups};
\draw (266.73, -45.900000000000006) rectangle (270.33000000000004,-47.50000000000001);
\draw(236.78000000000003, -49.400000000000006) node[anchor=north west,align=left] {\large Semigroups};
\draw (236.78000000000003, -49.400000000000006) rectangle (247.43000000000004,-68.5);
\draw(237.78000000000003, -50.400000000000006) node[anchor=north west,align=left] {Semigroup \\ rings, \\ multiplicative \\ semigroups of rings};
\draw (237.78000000000003, -50.400000000000006) rectangle (243.13000000000002,-52.50000000000001);
\draw(243.23000000000002, -50.400000000000006) node[anchor=north west,align=left] {Ideal \\ theory for \\ semigroups};
\draw (243.23000000000002, -50.400000000000006) rectangle (246.58,-52.00000000000001);
\draw(237.78000000000003, -52.60000000000001) node[anchor=north west,align=left] {Representation\\ of \\ semigroups; \\ actions of \\ semigroups on sets};
\draw (237.78000000000003, -52.60000000000001) rectangle (242.88000000000002,-55.20000000000001);
\draw(242.98000000000002, -52.60000000000001) node[anchor=north west,align=left] {Varieties\\ and \\ pseudovarieties\\ of semigroups};
\draw (242.98000000000002, -52.60000000000001) rectangle (247.33,-54.70000000000001);
\draw(237.78000000000003, -55.300000000000004) node[anchor=north west,align=left] {Semigroups\\ of \\ transformations, \\ relations, \\ partitions, etc.};
\draw (237.78000000000003, -55.300000000000004) rectangle (242.63000000000002,-57.900000000000006);
\draw(242.73000000000002, -55.300000000000004) node[anchor=north west,align=left] {Free semigroups,\\ generators and \\ relations, \\ word problems};
\draw (242.73000000000002, -55.300000000000004) rectangle (247.33,-57.900000000000006);
\draw(237.78000000000003, -58.00000000000001) node[anchor=north west,align=left] {Semigroups \\ in automata \\ theory, \\ linguistics, etc.};
\draw (237.78000000000003, -58.00000000000001) rectangle (242.63000000000002,-60.10000000000001);
\draw(242.73000000000002, -58.00000000000001) node[anchor=north west,align=left] {Algebraicmonoids};
\draw (242.73000000000002, -58.00000000000001) rectangle (247.33,-59.60000000000001);
\draw(237.78000000000003, -60.2) node[anchor=north west,align=left] {Connections of\\ semigroups \\ with homological\\ algebra and \\ category theory};
\draw (237.78000000000003, -60.2) rectangle (242.38000000000002,-62.800000000000004);
\draw(242.48000000000002, -60.2) node[anchor=north west,align=left] {Generalizations\\ of\\ semigroups};
\draw (242.48000000000002, -60.2) rectangle (246.83,-61.800000000000004);
\draw(237.78000000000003, -62.900000000000006) node[anchor=north west,align=left] {Commutative\\ semigroups};
\draw (237.78000000000003, -62.900000000000006) rectangle (241.13000000000002,-64.5);
\draw(241.23000000000002, -62.900000000000006) node[anchor=north west,align=left] {General \\ structure\\ theory for\\ semigroups};
\draw (241.23000000000002, -62.900000000000006) rectangle (244.33,-65.0);
\draw(237.78000000000003, -65.10000000000001) node[anchor=north west,align=left] {Radical \\ theory for\\ semigroups};
\draw (237.78000000000003, -65.10000000000001) rectangle (240.88000000000002,-66.7);
\draw(240.98000000000002, -65.10000000000001) node[anchor=north west,align=left] {Arithmetic\\ theory of\\ semigroups};
\draw (240.98000000000002, -65.10000000000001) rectangle (244.08,-66.7);
\draw(244.18000000000004, -65.10000000000001) node[anchor=north west,align=left] {Mappings\\ of \\ semigroups};
\draw (244.18000000000004, -65.10000000000001) rectangle (247.28000000000003,-66.7);
\draw(237.78000000000003, -66.80000000000001) node[anchor=north west,align=left] {Regular\\ semigroups};
\draw (237.78000000000003, -66.80000000000001) rectangle (240.88000000000002,-68.4);
\draw(240.98000000000002, -66.80000000000001) node[anchor=north west,align=left] {Inverse\\ semigroups};
\draw (240.98000000000002, -66.80000000000001) rectangle (244.08,-68.4);
\draw(244.18000000000004, -66.80000000000001) node[anchor=north west,align=left] {Orthodox\\ semigroups};
\draw (244.18000000000004, -66.80000000000001) rectangle (247.28000000000003,-68.4);
\draw(247.53000000000003, -49.400000000000006) node[anchor=north west,align=left] {\large Other groups of matrices};
\draw (247.53000000000003, -49.400000000000006) rectangle (257.43,-58.50000000000001);
\draw(248.53000000000003, -50.400000000000006) node[anchor=north west,align=left] {Unimodular \\ groups, congruence\\ subgroups\\ (group-theoretic\\ aspects)};
\draw (248.53000000000003, -50.400000000000006) rectangle (253.63000000000002,-53.00000000000001);
\draw(253.73000000000002, -50.400000000000006) node[anchor=north west,align=left] {Other matrix\\ groups\\ over fields};
\draw (253.73000000000002, -50.400000000000006) rectangle (257.33000000000004,-52.00000000000001);
\draw(248.53000000000003, -53.10000000000001) node[anchor=north west,align=left] {Fuchsian groups\\ and their\\ generalizations\\ (group-theoretic\\ aspects)};
\draw (248.53000000000003, -53.10000000000001) rectangle (253.13000000000002,-55.70000000000001);
\draw(253.23000000000002, -53.10000000000001) node[anchor=north west,align=left] {Other \\ matrix groups\\ over \\ finite fields};
\draw (253.23000000000002, -53.10000000000001) rectangle (257.08000000000004,-55.20000000000001);
\draw(248.53000000000003, -55.800000000000004) node[anchor=north west,align=left] {Other geometric\\ groups,\\ including\\ crystallographic\\ groups};
\draw (248.53000000000003, -55.800000000000004) rectangle (253.13000000000002,-58.400000000000006);
\draw(253.23000000000002, -55.800000000000004) node[anchor=north west,align=left] {Other matrix\\ groups\\ over rings};
\draw (253.23000000000002, -55.800000000000004) rectangle (256.83000000000004,-57.400000000000006);
\draw(257.53000000000003, -49.400000000000006) node[anchor=north west,align=left] {\large Abelian groups};
\draw (257.53000000000003, -49.400000000000006) rectangle (266.68,-64.10000000000001);
\draw(258.53000000000003, -50.400000000000006) node[anchor=north west,align=left] {Direct sums,\\ direct products,\\ etc. for\\ abelian groups};
\draw (258.53000000000003, -50.400000000000006) rectangle (263.13000000000005,-52.50000000000001);
\draw(263.23, -50.400000000000006) node[anchor=north west,align=left] {Subgroups\\ of abelian\\ groups};
\draw (263.23, -50.400000000000006) rectangle (266.33000000000004,-52.00000000000001);
\draw(258.53000000000003, -52.60000000000001) node[anchor=north west,align=left] {Torsion groups,\\ primary\\ groups and \\ generalized \\ primary groups};
\draw (258.53000000000003, -52.60000000000001) rectangle (262.88000000000005,-55.20000000000001);
\draw(262.98, -52.60000000000001) node[anchor=north west,align=left] {Torsion-free\\ groups,\\ finite rank};
\draw (262.98, -52.60000000000001) rectangle (266.58000000000004,-54.20000000000001);
\draw(258.53000000000003, -55.300000000000004) node[anchor=north west,align=left] {Automorphisms,\\ homomorphisms,\\ endomorphisms,\\ etc. for\\ abelian groups};
\draw (258.53000000000003, -55.300000000000004) rectangle (262.63000000000005,-57.900000000000006);
\draw(262.73, -55.300000000000004) node[anchor=north west,align=left] {Torsion-free\\ groups, \\ infinite rank};
\draw (262.73, -55.300000000000004) rectangle (266.58000000000004,-57.400000000000006);
\draw(258.53000000000003, -58.00000000000001) node[anchor=north west,align=left] {Homological\\ and \\ categorical \\ methods for\\ abelian groups};
\draw (258.53000000000003, -58.00000000000001) rectangle (262.63000000000005,-60.60000000000001);
\draw(262.73, -58.00000000000001) node[anchor=north west,align=left] {Extensions\\ of abelian\\ groups};
\draw (262.73, -58.00000000000001) rectangle (265.83000000000004,-59.60000000000001);
\draw(258.53000000000003, -60.7) node[anchor=north west,align=left] {Topological\\ methods\\ for \\ abelian groups};
\draw (258.53000000000003, -60.7) rectangle (262.63000000000005,-62.800000000000004);
\draw(262.73, -60.7) node[anchor=north west,align=left] {Finite\\ abelian\\ groups};
\draw (262.73, -60.7) rectangle (265.08000000000004,-62.300000000000004);
\draw(258.53000000000003, -62.900000000000006) node[anchor=north west,align=left] {Mixed\\ groups};
\draw (258.53000000000003, -62.900000000000006) rectangle (260.63000000000005,-64.0);
\draw(235.78000000000003, -68.69999999999999) node[anchor=north west,align=left] {\LARGE Category theory; homological algebra};
\draw (235.78000000000003, -68.69999999999999) rectangle (278.23,-130.5);
\draw(236.78000000000003, -69.69999999999999) node[anchor=north west,align=left] {\large Homological algebra in category theory, derived categories and functors};
\draw (236.78000000000003, -69.69999999999999) rectangle (263.28000000000003,-78.79999999999998);
\draw(237.78000000000003, -70.69999999999999) node[anchor=north west,align=left] {\(A_{\infty}\)-categories,\\ relations \\ with homological\\ mirror symmetry};
\draw (237.78000000000003, -70.69999999999999) rectangle (244.88000000000002,-73.29999999999998);
\draw(244.98000000000002, -70.69999999999999) node[anchor=north west,align=left] {Relative homological\\ algebra,\\ projective classes\\ (category-theoretic\\ aspects)};
\draw (244.98000000000002, -70.69999999999999) rectangle (250.58,-73.29999999999998);
\draw(250.68000000000004, -70.69999999999999) node[anchor=north west,align=left] {Projectives\\ and injectives\\ (category-theoretic\\ aspects)};
\draw (250.68000000000004, -70.69999999999999) rectangle (256.03000000000003,-73.29999999999998);
\draw(256.13000000000005, -70.69999999999999) node[anchor=north west,align=left] {Resolutions;\\ derived \\ functors \\ (category-theoretic\\ aspects)};
\draw (256.13000000000005, -70.69999999999999) rectangle (261.4800000000001,-73.29999999999998);
\draw(237.78000000000003, -73.39999999999999) node[anchor=north west,align=left] {Ext and Tor, \\ generalizations,\\ Künneth formula\\ (category-theoretic\\ aspects)};
\draw (237.78000000000003, -73.39999999999999) rectangle (243.13000000000002,-75.99999999999999);
\draw(243.23000000000002, -73.39999999999999) node[anchor=north west,align=left] {Homological\\ dimension \\ (category-theoretic\\ aspects)};
\draw (243.23000000000002, -73.39999999999999) rectangle (248.58,-75.49999999999999);
\draw(248.68000000000004, -73.39999999999999) node[anchor=north west,align=left] {Chain complexes\\ (category-theoretic\\ aspects),\\ dg categories};
\draw (248.68000000000004, -73.39999999999999) rectangle (254.03000000000003,-75.99999999999999);
\draw(254.13000000000002, -73.39999999999999) node[anchor=north west,align=left] {Nonabelian\\ homological\\ algebra \\ (category-theoretic\\ aspects)};
\draw (254.13000000000002, -73.39999999999999) rectangle (259.48,-75.99999999999999);
\draw(259.58000000000004, -73.39999999999999) node[anchor=north west,align=left] {Derived \\ categories,\\ triangulated\\ categories};
\draw (259.58000000000004, -73.39999999999999) rectangle (263.18000000000006,-75.49999999999999);
\draw(237.78000000000003, -76.1) node[anchor=north west,align=left] {Other \\ (co)homology \\ theories \\ (category-theoretic\\ aspects)};
\draw (237.78000000000003, -76.1) rectangle (243.13000000000002,-78.69999999999999);
\draw(243.23000000000002, -76.1) node[anchor=north west,align=left] {2-groups, \\ crossed \\ modules, crossed\\ complexes};
\draw (243.23000000000002, -76.1) rectangle (247.83,-78.19999999999999);
\draw(247.93000000000004, -76.1) node[anchor=north west,align=left] {Spectral\\ sequences,\\ hypercohomology};
\draw (247.93000000000004, -76.1) rectangle (252.28000000000003,-78.19999999999999);
\draw(252.38000000000002, -76.1) node[anchor=north west,align=left] {Simplicial\\ modules and\\ Dold-Kan \\ correspondence};
\draw (252.38000000000002, -76.1) rectangle (256.48,-78.19999999999999);
\draw(256.58000000000004, -76.1) node[anchor=north west,align=left] {Stable \\ module \\ categories};
\draw (256.58000000000004, -76.1) rectangle (259.68000000000006,-77.69999999999999);
\draw(259.78000000000003, -76.1) node[anchor=north west,align=left] {Graph \\ complexes\\ and graph\\ homology};
\draw (259.78000000000003, -76.1) rectangle (262.63000000000005,-78.19999999999999);
\draw(263.38000000000005, -69.69999999999999) node[anchor=north west,align=left] {\large Categories in geometry and topology};
\draw (263.38000000000005, -69.69999999999999) rectangle (277.53000000000003,-83.19999999999999);
\draw(264.38000000000005, -70.69999999999999) node[anchor=north west,align=left] {Presheaves and\\ sheaves, stacks,\\ descent \\ conditions \\ (category-theoretic aspects)};
\draw (264.38000000000005, -70.69999999999999) rectangle (271.9800000000001,-73.29999999999998);
\draw(272.08000000000004, -70.69999999999999) node[anchor=north west,align=left] {Abstract \\ manifolds and fiber\\ bundles \\ (category-theoretic\\ aspects)};
\draw (272.08000000000004, -70.69999999999999) rectangle (277.43000000000006,-73.29999999999998);
\draw(264.38000000000005, -73.39999999999999) node[anchor=north west,align=left] {Synthetic \\ differential geometry,\\ tangent \\ categories, differential\\ categories};
\draw (264.38000000000005, -73.39999999999999) rectangle (270.9800000000001,-75.99999999999999);
\draw(271.08000000000004, -73.39999999999999) node[anchor=north west,align=left] {Algebraic \\ \(K\)-theory and\\ \(L\)-theory\\ (category-theoretic\\ aspects)};
\draw (271.08000000000004, -73.39999999999999) rectangle (276.43000000000006,-75.99999999999999);
\draw(264.38000000000005, -76.1) node[anchor=north west,align=left] {Grothendieck\\ groups \\ (category-theoretic\\ aspects)};
\draw (264.38000000000005, -76.1) rectangle (269.7300000000001,-78.19999999999999);
\draw(269.83000000000004, -76.1) node[anchor=north west,align=left] {Grothendieck\\ topologies\\ and \\ Grothendieck topoi};
\draw (269.83000000000004, -76.1) rectangle (274.93000000000006,-78.19999999999999);
\draw(264.38000000000005, -78.29999999999998) node[anchor=north west,align=left] {Frames and \\ locales, pointfree\\ topology,\\ Stone duality};
\draw (264.38000000000005, -78.29999999999998) rectangle (269.4800000000001,-80.39999999999998);
\draw(269.58000000000004, -78.29999999999998) node[anchor=north west,align=left] {Categories\\ of topological\\ spaces\\ and continuous\\ mappings};
\draw (269.58000000000004, -78.29999999999998) rectangle (273.68000000000006,-80.89999999999998);
\draw(273.78000000000003, -78.29999999999998) node[anchor=north west,align=left] {Local \\ categories \\ and functors};
\draw (273.78000000000003, -78.29999999999998) rectangle (277.38000000000005,-79.89999999999998);
\draw(264.38000000000005, -80.99999999999999) node[anchor=north west,align=left] {Goodwillie\\ calculus\\ and functor\\ calculus};
\draw (264.38000000000005, -80.99999999999999) rectangle (267.7300000000001,-83.09999999999998);
\draw(267.83000000000004, -80.99999999999999) node[anchor=north west,align=left] {Quantales};
\draw (267.83000000000004, -80.99999999999999) rectangle (270.68000000000006,-82.09999999999998);
\draw(236.78000000000003, -78.89999999999999) node[anchor=north west,align=left] {\large Computational methods\\ for problems pertaining\\ to category theory};
\draw (236.78000000000003, -78.89999999999999) rectangle (244.51000000000002,-80.49999999999999);
\draw(236.78000000000003, -80.6) node[anchor=north west,align=left] {\large History of \\ category theory};
\draw (236.78000000000003, -80.6) rectangle (242.03000000000003,-81.69999999999999);
\draw(236.78000000000003, -83.29999999999998) node[anchor=north west,align=left] {\large Higher categories and homotopical algebra};
\draw (236.78000000000003, -83.29999999999998) rectangle (252.43000000000004,-97.79999999999998);
\draw(237.78000000000003, -84.29999999999998) node[anchor=north west,align=left] {\((\infty,~n)\)-categories\\ and \\ \((\infty,\infty)\)-categories};
\draw (237.78000000000003, -84.29999999999998) rectangle (245.88000000000002,-86.89999999999998);
\draw(245.98000000000002, -84.29999999999998) node[anchor=north west,align=left] {Categories of \\ fibrations, \\ relations to \\ \(K\)-theory, relations\\ to type theory};
\draw (245.98000000000002, -84.29999999999998) rectangle (252.33,-86.89999999999998);
\draw(237.78000000000003, -86.99999999999999) node[anchor=north west,align=left] {\((\infty,1)\)-categories\\ (quasi-categories,\\ Segal \\ spaces, etc.); \\ \(\infty\)-topoi, stable\\ \(\infty\)-categories};
\draw (237.78000000000003, -86.99999999999999) rectangle (244.63000000000002,-90.09999999999998);
\draw(244.73000000000002, -86.99999999999999) node[anchor=north west,align=left] {Localizations\\ (e.g., \\ simplicial localization,\\ Bousfield\\ localization)};
\draw (244.73000000000002, -86.99999999999999) rectangle (251.33,-89.59999999999998);
\draw(237.78000000000003, -90.19999999999999) node[anchor=north west,align=left] {Tricategories,\\ weak \\ \(n\)-categories, \\ coherence, \\ semi-strictification};
\draw (237.78000000000003, -90.19999999999999) rectangle (243.38000000000002,-92.79999999999998);
\draw(243.48000000000002, -90.19999999999999) node[anchor=north west,align=left] {\(\infty\)-operads\\ and higher\\ algebra};
\draw (243.48000000000002, -90.19999999999999) rectangle (248.58,-92.29999999999998);
\draw(248.68000000000004, -90.19999999999999) node[anchor=north west,align=left] {Simplicial\\ sets,\\ simplicial\\ objects};
\draw (248.68000000000004, -90.19999999999999) rectangle (251.78000000000003,-92.29999999999998);
\draw(237.78000000000003, -92.89999999999998) node[anchor=north west,align=left] {Strict \\ omega-categories,\\ computads,\\ polygraphs};
\draw (237.78000000000003, -92.89999999999998) rectangle (242.63000000000002,-94.99999999999997);
\draw(242.73000000000002, -92.89999999999998) node[anchor=north west,align=left] {Categorification};
\draw (242.73000000000002, -92.89999999999998) rectangle (247.33,-93.99999999999997);
\draw(247.43000000000004, -92.89999999999998) node[anchor=north west,align=left] {2-categories,\\ bicategories,\\ double\\ categories};
\draw (247.43000000000004, -92.89999999999998) rectangle (251.28000000000003,-94.99999999999997);
\draw(237.78000000000003, -95.09999999999998) node[anchor=north west,align=left] {2-dimensional\\ monad theory};
\draw (237.78000000000003, -95.09999999999998) rectangle (241.63000000000002,-96.69999999999997);
\draw(241.73000000000002, -95.09999999999998) node[anchor=north west,align=left] {Homotopical\\ algebra, \\ Quillen model\\ categories,\\ derivators};
\draw (241.73000000000002, -95.09999999999998) rectangle (245.58,-97.69999999999997);
\draw(252.53000000000003, -83.29999999999998) node[anchor=north west,align=left] {\large General theory of categories and functors};
\draw (252.53000000000003, -83.29999999999998) rectangle (267.88000000000005,-99.29999999999998);
\draw(253.53000000000003, -84.29999999999998) node[anchor=north west,align=left] {Factorization \\ systems, substructures,\\ quotient \\ structures, \\ congruences, amalgams};
\draw (253.53000000000003, -84.29999999999998) rectangle (259.88000000000005,-86.89999999999998);
\draw(259.98, -84.29999999999998) node[anchor=north west,align=left] {Limits and colimits\\ (products, sums,\\ directed limits,\\ pushouts, fiber\\ products, \\ equalizers, kernels, \\ ends and coends, etc.)};
\draw (259.98, -84.29999999999998) rectangle (266.08000000000004,-87.89999999999998);
\draw(253.53000000000003, -87.99999999999999) node[anchor=north west,align=left] {Categories \\ admitting limits \\ (complete categories),\\ functors\\ preserving limits,\\ completions};
\draw (253.53000000000003, -87.99999999999999) rectangle (259.63000000000005,-91.09999999999998);
\draw(259.73, -87.99999999999999) node[anchor=north west,align=left] {Special \\ properties of \\ functors (faithful,\\ full, etc.)};
\draw (259.73, -87.99999999999999) rectangle (265.08000000000004,-90.09999999999998);
\draw(253.53000000000003, -91.19999999999999) node[anchor=north west,align=left] {Adjoint functors\\ (universal \\ constructions, \\ reflective \\ subcategories, Kan \\ extensions, etc.)};
\draw (253.53000000000003, -91.19999999999999) rectangle (258.88000000000005,-94.29999999999998);
\draw(258.98, -91.19999999999999) node[anchor=north west,align=left] {Foundations,\\ relations to\\ logic and \\ deductive systems};
\draw (258.98, -91.19999999999999) rectangle (263.83000000000004,-93.29999999999998);
\draw(263.93, -91.19999999999999) node[anchor=north west,align=left] {Graphs, \\ diagram \\ schemes, \\ precategories};
\draw (263.93, -91.19999999999999) rectangle (267.78000000000003,-93.29999999999998);
\draw(253.53000000000003, -94.39999999999998) node[anchor=north west,align=left] {Definitions\\ and \\ generalizations \\ in theory of\\ categories};
\draw (253.53000000000003, -94.39999999999998) rectangle (258.13000000000005,-96.99999999999997);
\draw(258.23, -94.39999999999998) node[anchor=north west,align=left] {Epimorphisms,\\ monomorphisms,\\ special classes\\ of morphisms,\\ null morphisms};
\draw (258.23, -94.39999999999998) rectangle (262.58000000000004,-96.99999999999997);
\draw(262.68, -94.39999999999998) node[anchor=north west,align=left] {Functor \\ categories,\\ comma \\ categories};
\draw (262.68, -94.39999999999998) rectangle (266.03000000000003,-96.49999999999997);
\draw(253.53000000000003, -97.09999999999998) node[anchor=north west,align=left] {Natural \\ morphisms,\\ dinatural\\ morphisms};
\draw (253.53000000000003, -97.09999999999998) rectangle (256.63000000000005,-99.19999999999997);
\draw(256.73, -97.09999999999998) node[anchor=north west,align=left] {Graded \\ categories\\ (general)};
\draw (256.73, -97.09999999999998) rectangle (259.83000000000004,-98.69999999999997);
\draw(267.98, -83.29999999999998) node[anchor=north west,align=left] {\large Categories and theories};
\draw (267.98, -83.29999999999998) rectangle (278.13,-95.09999999999998);
\draw(268.98, -84.29999999999998) node[anchor=north west,align=left] {Monads (= standard\\ construction, \\ triple or triad), \\ algebras for monads,\\ homology and derived\\ functors for monads};
\draw (268.98, -84.29999999999998) rectangle (274.58000000000004,-87.39999999999998);
\draw(274.68, -84.29999999999998) node[anchor=north west,align=left] {Accessible\\ and locally\\ presentable\\ categories};
\draw (274.68, -84.29999999999998) rectangle (278.03000000000003,-86.39999999999998);
\draw(268.98, -87.49999999999999) node[anchor=north west,align=left] {Theories \\ (e.g., algebraic\\ theories),\\ structure,\\ and semantics};
\draw (268.98, -87.49999999999999) rectangle (273.58000000000004,-90.09999999999998);
\draw(273.68, -87.49999999999999) node[anchor=north west,align=left] {Eilenberg-Moore\\ and Kleisli\\ constructions\\ for monads};
\draw (273.68, -87.49999999999999) rectangle (278.03000000000003,-89.59999999999998);
\draw(268.98, -90.19999999999999) node[anchor=north west,align=left] {Sketches\\ and \\ generalizations};
\draw (268.98, -90.19999999999999) rectangle (273.33000000000004,-91.79999999999998);
\draw(273.43, -90.19999999999999) node[anchor=north west,align=left] {Structured\\ objects \\ in a category\\ (group\\ objects, etc.)};
\draw (273.43, -90.19999999999999) rectangle (277.53000000000003,-92.79999999999998);
\draw(268.98, -92.89999999999998) node[anchor=north west,align=left] {Categorical\\ semantics\\ of formal\\ languages};
\draw (268.98, -92.89999999999998) rectangle (272.33000000000004,-94.99999999999997);
\draw(272.43, -92.89999999999998) node[anchor=north west,align=left] {Equational\\ categories};
\draw (272.43, -92.89999999999998) rectangle (275.53000000000003,-94.49999999999997);
\draw(236.78000000000003, -99.39999999999999) node[anchor=north west,align=left] {\large Monoidal categories and operads};
\draw (236.78000000000003, -99.39999999999999) rectangle (249.18000000000004,-119.5);
\draw(237.78000000000003, -100.39999999999999) node[anchor=north west,align=left] {Polycategories/dioperads,\\ properads, PROPs, \\ cyclic operads,\\ modular operads};
\draw (237.78000000000003, -100.39999999999999) rectangle (244.63000000000002,-102.99999999999999);
\draw(244.73000000000002, -100.39999999999999) node[anchor=north west,align=left] {Algebraic \\ operads, \\ cooperads, and \\ Koszul duality};
\draw (244.73000000000002, -100.39999999999999) rectangle (249.08,-102.49999999999999);
\draw(237.78000000000003, -103.1) node[anchor=north west,align=left] {Braided \\ monoidal categories\\ and ribbon\\ categories};
\draw (237.78000000000003, -103.1) rectangle (243.13000000000002,-105.19999999999999);
\draw(243.23000000000002, -103.1) node[anchor=north west,align=left] {Fusion \\ categories, modular\\ tensor \\ categories, \\ modular functors};
\draw (243.23000000000002, -103.1) rectangle (248.58,-105.69999999999999);
\draw(237.78000000000003, -105.8) node[anchor=north west,align=left] {Dagger \\ categories, \\ categorical quantum\\ mechanics};
\draw (237.78000000000003, -105.8) rectangle (243.13000000000002,-107.89999999999999);
\draw(243.23000000000002, -105.8) node[anchor=north west,align=left] {Monoidal \\ categories, \\ symmetric monoidal\\ categories};
\draw (243.23000000000002, -105.8) rectangle (248.33,-107.89999999999999);
\draw(237.78000000000003, -108.0) node[anchor=north west,align=left] {Traced monoidal\\ categories,\\ compact \\ closed categories,\\ star-autonomous\\ categories};
\draw (237.78000000000003, -108.0) rectangle (242.88000000000002,-111.1);
\draw(242.98000000000002, -108.0) node[anchor=north west,align=left] {Categories\\ of networks\\ and \\ processes, \\ compositionality};
\draw (242.98000000000002, -108.0) rectangle (247.58,-110.6);
\draw(237.78000000000003, -111.19999999999999) node[anchor=north west,align=left] {Non-symmetric\\ operads, \\ multicategories,\\ generalized\\ multicategories};
\draw (237.78000000000003, -111.19999999999999) rectangle (242.38000000000002,-113.79999999999998);
\draw(242.48000000000002, -111.19999999999999) node[anchor=north west,align=left] {Bimonoidal,\\ skew-monoidal,\\ duoidal\\ categories};
\draw (242.48000000000002, -111.19999999999999) rectangle (246.58,-113.29999999999998);
\draw(237.78000000000003, -113.89999999999999) node[anchor=north west,align=left] {Species, \\ Hopf monoids,\\ operads in\\ combinatorics};
\draw (237.78000000000003, -113.89999999999999) rectangle (241.63000000000002,-115.99999999999999);
\draw(241.73000000000002, -113.89999999999999) node[anchor=north west,align=left] {String \\ diagrams and\\ graphical\\ calculi};
\draw (241.73000000000002, -113.89999999999999) rectangle (245.33,-115.99999999999999);
\draw(245.43000000000004, -113.89999999999999) node[anchor=north west,align=left] {Categorical\\ aspects\\ of \\ linear logic};
\draw (245.43000000000004, -113.89999999999999) rectangle (249.03000000000003,-115.99999999999999);
\draw(237.78000000000003, -116.1) node[anchor=north west,align=left] {Topological\\ and\\ simplicial\\ operads};
\draw (237.78000000000003, -116.1) rectangle (241.13000000000002,-118.19999999999999);
\draw(241.23000000000002, -116.1) node[anchor=north west,align=left] {Tannakian\\ categories};
\draw (241.23000000000002, -116.1) rectangle (244.33,-117.69999999999999);
\draw(244.43000000000004, -116.1) node[anchor=north west,align=left] {Operads\\ (general)};
\draw (244.43000000000004, -116.1) rectangle (247.28000000000003,-117.69999999999999);
\draw(237.78000000000003, -118.29999999999998) node[anchor=north west,align=left] {Globular\\ operads};
\draw (237.78000000000003, -118.29999999999998) rectangle (240.38000000000002,-119.39999999999998);
\draw(249.28000000000003, -99.39999999999999) node[anchor=north west,align=left] {\large Categorical structures};
\draw (249.28000000000003, -99.39999999999999) rectangle (260.18,-110.69999999999999);
\draw(250.28000000000003, -100.39999999999999) node[anchor=north west,align=left] {Proarrow equipments,\\ Yoneda \\ structures, KZ \\ doctrines (lax \\ idempotent monads)};
\draw (250.28000000000003, -100.39999999999999) rectangle (255.88000000000002,-102.99999999999999);
\draw(255.98000000000002, -100.39999999999999) node[anchor=north west,align=left] {Enriched \\ categories \\ (over closed\\ or monoidal\\ categories)};
\draw (255.98000000000002, -100.39999999999999) rectangle (259.58000000000004,-102.99999999999999);
\draw(250.28000000000003, -103.1) node[anchor=north west,align=left] {Actions of a\\ monoidal \\ category, \\ tensorial strength};
\draw (250.28000000000003, -103.1) rectangle (255.38000000000002,-105.19999999999999);
\draw(255.48000000000002, -103.1) node[anchor=north west,align=left] {Profunctors\\ (= \\ correspondences,\\ distributors,\\ modules)};
\draw (255.48000000000002, -103.1) rectangle (260.08000000000004,-105.69999999999999);
\draw(250.28000000000003, -105.8) node[anchor=north west,align=left] {Closed categories\\ (closed \\ monoidal and \\ Cartesian closed\\ categories, etc.)};
\draw (250.28000000000003, -105.8) rectangle (255.13000000000002,-108.39999999999999);
\draw(255.23000000000002, -105.8) node[anchor=north west,align=left] {Fibered\\ categories};
\draw (255.23000000000002, -105.8) rectangle (258.33000000000004,-107.39999999999999);
\draw(250.28000000000003, -108.5) node[anchor=north west,align=left] {Internal\\ categories\\ and\\ groupoids};
\draw (250.28000000000003, -108.5) rectangle (253.38000000000002,-110.6);
\draw(253.48000000000002, -108.5) node[anchor=north west,align=left] {Formal\\ category\\ theory};
\draw (253.48000000000002, -108.5) rectangle (256.08000000000004,-110.1);
\draw(260.28000000000003, -99.39999999999999) node[anchor=north west,align=left] {\large Categorical algebra};
\draw (260.28000000000003, -99.39999999999999) rectangle (270.43,-111.39999999999999);
\draw(261.28000000000003, -100.39999999999999) node[anchor=north west,align=left] {Protomodular\\ categories, \\ semi-abelian \\ categories, \\ Mal’tsev categories};
\draw (261.28000000000003, -100.39999999999999) rectangle (266.63000000000005,-102.99999999999999);
\draw(266.73, -100.39999999999999) node[anchor=north west,align=left] {Preadditive,\\ additive\\ categories};
\draw (266.73, -100.39999999999999) rectangle (270.33000000000004,-101.99999999999999);
\draw(261.28000000000003, -103.1) node[anchor=north west,align=left] {Definable \\ subcategories\\ and \\ connections with\\ model theory};
\draw (261.28000000000003, -103.1) rectangle (265.88000000000005,-105.69999999999999);
\draw(265.98, -103.1) node[anchor=north west,align=left] {Localization\\ of categories,\\ calculus\\ of fractions};
\draw (265.98, -103.1) rectangle (270.08000000000004,-105.19999999999999);
\draw(261.28000000000003, -105.8) node[anchor=north west,align=left] {Abelian \\ categories,\\ Grothendieck\\ categories};
\draw (261.28000000000003, -105.8) rectangle (264.88000000000005,-107.89999999999999);
\draw(264.98, -105.8) node[anchor=north west,align=left] {Regular \\ categories,\\ Barr-exact\\ categories};
\draw (264.98, -105.8) rectangle (268.33000000000004,-107.89999999999999);
\draw(261.28000000000003, -108.0) node[anchor=north west,align=left] {Categorical\\ embedding\\ theorems};
\draw (261.28000000000003, -108.0) rectangle (264.63000000000005,-109.6);
\draw(264.73, -108.0) node[anchor=north west,align=left] {Categorical\\ Galois\\ theory};
\draw (264.73, -108.0) rectangle (268.08000000000004,-109.6);
\draw(261.28000000000003, -109.69999999999999) node[anchor=north west,align=left] {Torsion\\ theories,\\ radicals};
\draw (261.28000000000003, -109.69999999999999) rectangle (264.13000000000005,-111.29999999999998);
\draw(236.78000000000003, -119.6) node[anchor=north west,align=left] {\large Special categories};
\draw (236.78000000000003, -119.6) rectangle (246.68000000000004,-130.4);
\draw(237.78000000000003, -120.6) node[anchor=north west,align=left] {Categories\\ of sets,\\ characterizations};
\draw (237.78000000000003, -120.6) rectangle (242.63000000000002,-122.69999999999999);
\draw(242.73000000000002, -120.6) node[anchor=north west,align=left] {Extensive, \\ distributive,\\ and adhesive\\ categories};
\draw (242.73000000000002, -120.6) rectangle (246.58,-122.69999999999999);
\draw(237.78000000000003, -122.8) node[anchor=north west,align=left] {Categories\\ of spans/cospans,\\ relations, or \\ partial maps};
\draw (237.78000000000003, -122.8) rectangle (242.63000000000002,-125.39999999999999);
\draw(242.73000000000002, -122.8) node[anchor=north west,align=left] {Categories\\ of machines,\\ automata};
\draw (242.73000000000002, -122.8) rectangle (246.33,-124.39999999999999);
\draw(237.78000000000003, -125.5) node[anchor=north west,align=left] {Preorders, \\ orders, domains\\ and lattices\\ (viewed\\ as categories)};
\draw (237.78000000000003, -125.5) rectangle (242.13000000000002,-128.1);
\draw(242.23000000000002, -125.5) node[anchor=north west,align=left] {Groupoids, \\ semigroupoids,\\ semigroups, \\ groups (viewed\\ as categories)};
\draw (242.23000000000002, -125.5) rectangle (246.33,-128.1);
\draw(237.78000000000003, -128.2) node[anchor=north west,align=left] {Embedding\\ theorems,\\ universal\\ categories};
\draw (237.78000000000003, -128.2) rectangle (240.88000000000002,-130.29999999999998);
\draw(240.98000000000002, -128.2) node[anchor=north west,align=left] {Topoi};
\draw (240.98000000000002, -128.2) rectangle (242.83,-128.79999999999998);
\draw(281.7300000000001, -1) node[anchor=north west,align=left] {\LARGE Measure and integration};
\draw (281.7300000000001, -1) rectangle (323.2700000000001,-32.300000000000004);
\draw(282.7300000000001, -2) node[anchor=north west,align=left] {\large Set functions, measures and integrals with values in abstract spaces};
\draw (282.7300000000001, -2) rectangle (304.4100000000001,-6.199999999999999);
\draw(283.7300000000001, -3) node[anchor=north west,align=left] {Set-valued set \\ functions and \\ measures; integration\\ of set-valued\\ functions; \\ measurable selections};
\draw (283.7300000000001, -3) rectangle (289.5800000000001,-6.1);
\draw(289.68000000000006, -3) node[anchor=north west,align=left] {Group- or \\ semigroup-valued\\ set \\ functions, measures\\ and integrals};
\draw (289.68000000000006, -3) rectangle (295.0300000000001,-5.6);
\draw(295.13000000000005, -3) node[anchor=north west,align=left] {Vector-valued\\ set functions,\\ measures\\ and integrals};
\draw (295.13000000000005, -3) rectangle (299.2300000000001,-5.1);
\draw(299.3300000000001, -3) node[anchor=north west,align=left] {Set functions,\\ measures and\\ integrals \\ with values in\\ ordered spaces};
\draw (299.3300000000001, -3) rectangle (303.4300000000001,-5.6);
\draw(304.5100000000001, -2) node[anchor=north west,align=left] {\large Miscellaneous topics in measure theory};
\draw (304.5100000000001, -2) rectangle (316.8900000000001,-5.2);
\draw(305.5100000000001, -3) node[anchor=north west,align=left] {Other \\ connections \\ with logic \\ and set theory};
\draw (305.5100000000001, -3) rectangle (309.6100000000001,-5.1);
\draw(309.7100000000001, -3) node[anchor=north west,align=left] {Nonstandard\\ measure\\ theory};
\draw (309.7100000000001, -3) rectangle (313.0600000000001,-4.6);
\draw(313.1600000000001, -3) node[anchor=north west,align=left] {Fuzzy\\ measure\\ theory};
\draw (313.1600000000001, -3) rectangle (315.5100000000001,-4.6);
\draw(316.99000000000007, -2) node[anchor=north west,align=left] {\large History of measure\\ and integration};
\draw (316.99000000000007, -2) rectangle (323.1700000000001,-3.1);
\draw(282.7300000000001, -6.299999999999999) node[anchor=north west,align=left] {\large Set functions and measures on spaces with additional structure};
\draw (282.7300000000001, -6.299999999999999) rectangle (302.5800000000001,-13.7);
\draw(283.7300000000001, -7.299999999999999) node[anchor=north west,align=left] {Set functions and\\ measures and integrals\\ in \\ infinite-dimensional spaces\\ (Wiener measure, \\ Gaussian measure, etc.)};
\draw (283.7300000000001, -7.299999999999999) rectangle (291.0800000000001,-10.399999999999999);
\draw(291.18000000000006, -7.299999999999999) node[anchor=north west,align=left] {Integration theory\\ via linear \\ functionals (Radon \\ measures, Daniell \\ integrals, etc.),\\ representing set\\ functions and measures};
\draw (291.18000000000006, -7.299999999999999) rectangle (297.2800000000001,-10.899999999999999);
\draw(297.38000000000005, -7.299999999999999) node[anchor=north west,align=left] {Set functions and\\ measures on \\ topological groups\\ or semigroups,\\ Haar measures, \\ invariant measures};
\draw (297.38000000000005, -7.299999999999999) rectangle (302.4800000000001,-10.399999999999999);
\draw(283.7300000000001, -11.0) node[anchor=north west,align=left] {Set functions \\ and measures on\\ topological \\ spaces (regularity\\ of measures, etc.)};
\draw (283.7300000000001, -11.0) rectangle (288.8300000000001,-13.6);
\draw(302.68000000000006, -6.299999999999999) node[anchor=north west,align=left] {\large Measure-theoretic ergodic theory};
\draw (302.68000000000006, -6.299999999999999) rectangle (314.33000000000004,-12.2);
\draw(303.68000000000006, -7.299999999999999) node[anchor=north west,align=left] {General groups\\ of \\ measure-preserving \\ transformations};
\draw (303.68000000000006, -7.299999999999999) rectangle (309.0300000000001,-9.399999999999999);
\draw(309.13000000000005, -7.299999999999999) node[anchor=north west,align=left] {Measure-preserving\\ transformations};
\draw (309.13000000000005, -7.299999999999999) rectangle (314.2300000000001,-9.399999999999999);
\draw(303.68000000000006, -9.5) node[anchor=north west,align=left] {One-parameter\\ continuous \\ families of \\ measure-preserving\\ transformations};
\draw (303.68000000000006, -9.5) rectangle (308.7800000000001,-12.1);
\draw(308.88000000000005, -9.5) node[anchor=north west,align=left] {Entropy \\ and other\\ invariants};
\draw (308.88000000000005, -9.5) rectangle (311.9800000000001,-11.1);
\draw(314.43000000000006, -6.299999999999999) node[anchor=north west,align=left] {\large Computational methods for\\ problems pertaining to\\ measure and integration};
\draw (314.43000000000006, -6.299999999999999) rectangle (322.7800000000001,-7.899999999999999);
\draw(282.7300000000001, -13.799999999999999) node[anchor=north west,align=left] {\large Classical measure theory};
\draw (282.7300000000001, -13.799999999999999) rectangle (293.88000000000005,-32.2);
\draw(283.7300000000001, -14.799999999999999) node[anchor=north west,align=left] {Measurable and\\ nonmeasurable \\ functions, sequences\\ of measurable\\ functions,\\ modes of convergence};
\draw (283.7300000000001, -14.799999999999999) rectangle (289.3300000000001,-17.9);
\draw(289.43000000000006, -14.799999999999999) node[anchor=north west,align=left] {Contents, \\ measures, \\ outer measures,\\ capacities};
\draw (289.43000000000006, -14.799999999999999) rectangle (293.7800000000001,-16.9);
\draw(283.7300000000001, -18.0) node[anchor=north west,align=left] {Classes of sets\\ (Borel fields, \\ \(\sigma\)-rings,\\ etc.), measurable\\ sets, Suslin\\ sets, analytic sets};
\draw (283.7300000000001, -18.0) rectangle (289.0800000000001,-21.1);
\draw(289.18000000000006, -18.0) node[anchor=north west,align=left] {Measures \\ on Boolean\\ rings, \\ measure algebras};
\draw (289.18000000000006, -18.0) rectangle (293.7800000000001,-20.1);
\draw(283.7300000000001, -21.2) node[anchor=north west,align=left] {Abstract \\ differentiation \\ theory, \\ differentiation of\\ set functions};
\draw (283.7300000000001, -21.2) rectangle (288.8300000000001,-23.8);
\draw(288.93000000000006, -21.2) node[anchor=north west,align=left] {Real- or\\ complex-valued\\ set\\ functions};
\draw (288.93000000000006, -21.2) rectangle (293.0300000000001,-23.3);
\draw(283.7300000000001, -23.9) node[anchor=north west,align=left] {Integration\\ and \\ disintegration\\ of measures};
\draw (283.7300000000001, -23.9) rectangle (287.8300000000001,-26.0);
\draw(287.93000000000006, -23.9) node[anchor=north west,align=left] {Length, area,\\ volume, other\\ geometric \\ measure theory};
\draw (287.93000000000006, -23.9) rectangle (292.0300000000001,-26.0);
\draw(283.7300000000001, -26.1) node[anchor=north west,align=left] {Integration\\ with respect\\ to measures\\ and other \\ set functions};
\draw (283.7300000000001, -26.1) rectangle (287.5800000000001,-28.700000000000003);
\draw(287.68000000000006, -26.1) node[anchor=north west,align=left] {Measures \\ and integrals\\ in product\\ spaces};
\draw (287.68000000000006, -26.1) rectangle (291.5300000000001,-28.200000000000003);
\draw(283.7300000000001, -28.799999999999997) node[anchor=north west,align=left] {Spaces of\\ measures,\\ convergence\\ of measures};
\draw (283.7300000000001, -28.799999999999997) rectangle (287.0800000000001,-30.9);
\draw(287.18000000000006, -28.799999999999997) node[anchor=north west,align=left] {Hausdorff\\ and packing\\ measures};
\draw (287.18000000000006, -28.799999999999997) rectangle (290.5300000000001,-30.4);
\draw(290.63000000000005, -28.799999999999997) node[anchor=north west,align=left] {Fractals};
\draw (290.63000000000005, -28.799999999999997) rectangle (293.2300000000001,-29.9);
\draw(283.7300000000001, -31.0) node[anchor=north west,align=left] {Lifting\\ theory};
\draw (283.7300000000001, -31.0) rectangle (286.0800000000001,-32.1);
\draw(281.7300000000001, -32.400000000000006) node[anchor=north west,align=left] {\LARGE Algebraic geometry};
\draw (281.7300000000001, -32.400000000000006) rectangle (322.3300000000001,-129.5);
\draw(282.7300000000001, -33.400000000000006) node[anchor=north west,align=left] {\large Arithmetic problems in algebraic geometry; Diophantine geometry};
\draw (282.7300000000001, -33.400000000000006) rectangle (306.5300000000001,-42.50000000000001);
\draw(283.7300000000001, -34.400000000000006) node[anchor=north west,align=left] {Zeta functions \\ and related questions\\ in algebraic\\ geometry (e.g.,\\ Birch-Swinnerton-Dyer\\ conjecture)};
\draw (283.7300000000001, -34.400000000000006) rectangle (289.5800000000001,-37.50000000000001);
\draw(289.68000000000006, -34.400000000000006) node[anchor=north west,align=left] {Hasse principle,\\ weak and \\ strong approximation,\\ Brauer-Manin\\ obstruction};
\draw (289.68000000000006, -34.400000000000006) rectangle (295.5300000000001,-37.00000000000001);
\draw(295.63000000000005, -34.400000000000006) node[anchor=north west,align=left] {Universal profinite\\ groups (relationship\\ to moduli\\ spaces, projective\\ and moduli towers,\\ Galois theory)};
\draw (295.63000000000005, -34.400000000000006) rectangle (301.2300000000001,-37.50000000000001);
\draw(301.3300000000001, -34.400000000000006) node[anchor=north west,align=left] {Positive \\ characteristic\\ ground \\ fields in \\ algebraic geometry};
\draw (301.3300000000001, -34.400000000000006) rectangle (306.4300000000001,-37.00000000000001);
\draw(283.7300000000001, -37.60000000000001) node[anchor=north west,align=left] {Other \\ nonalgebraically \\ closed ground \\ fields in algebraic\\ geometry};
\draw (283.7300000000001, -37.60000000000001) rectangle (289.0800000000001,-40.20000000000001);
\draw(289.18000000000006, -37.60000000000001) node[anchor=north west,align=left] {Applications\\ to coding \\ theory and \\ cryptography of \\ arithmetic geometry};
\draw (289.18000000000006, -37.60000000000001) rectangle (294.5300000000001,-40.20000000000001);
\draw(294.63000000000005, -37.60000000000001) node[anchor=north west,align=left] {Arithmetic\\ varieties \\ and schemes;\\ Arakelov \\ theory; heights};
\draw (294.63000000000005, -37.60000000000001) rectangle (298.9800000000001,-40.20000000000001);
\draw(299.0800000000001, -37.60000000000001) node[anchor=north west,align=left] {Perfectoid\\ spaces and\\ mixed \\ characteristic};
\draw (299.0800000000001, -37.60000000000001) rectangle (303.1800000000001,-39.70000000000001);
\draw(303.2800000000001, -37.60000000000001) node[anchor=north west,align=left] {Rational\\ points};
\draw (303.2800000000001, -37.60000000000001) rectangle (305.8800000000001,-38.70000000000001);
\draw(283.7300000000001, -40.300000000000004) node[anchor=north west,align=left] {Finite \\ ground fields\\ in algebraic\\ geometry};
\draw (283.7300000000001, -40.300000000000004) rectangle (287.5800000000001,-42.400000000000006);
\draw(287.68000000000006, -40.300000000000004) node[anchor=north west,align=left] {Global \\ ground fields\\ in algebraic\\ geometry};
\draw (287.68000000000006, -40.300000000000004) rectangle (291.5300000000001,-42.400000000000006);
\draw(291.63000000000005, -40.300000000000004) node[anchor=north west,align=left] {Local ground\\ fields\\ in algebraic\\ geometry};
\draw (291.63000000000005, -40.300000000000004) rectangle (295.2300000000001,-42.400000000000006);
\draw(295.3300000000001, -40.300000000000004) node[anchor=north west,align=left] {Modular \\ and Shimura\\ varieties};
\draw (295.3300000000001, -40.300000000000004) rectangle (298.6800000000001,-41.900000000000006);
\draw(298.7800000000001, -40.300000000000004) node[anchor=north west,align=left] {Rigid \\ analytic\\ geometry};
\draw (298.7800000000001, -40.300000000000004) rectangle (301.3800000000001,-41.900000000000006);
\draw(306.6300000000001, -33.400000000000006) node[anchor=north west,align=left] {\large (Co)homology theory in algebraic geometry};
\draw (306.6300000000001, -33.400000000000006) rectangle (322.23000000000013,-47.400000000000006);
\draw(307.6300000000001, -34.400000000000006) node[anchor=north west,align=left] {Other algebro-geometric\\ (co)homologies\\ (e.g., \\ intersection, equivariant,\\ Lawson, Deligne\\ (co)homologies)};
\draw (307.6300000000001, -34.400000000000006) rectangle (314.73000000000013,-37.50000000000001);
\draw(314.8300000000001, -34.400000000000006) node[anchor=north west,align=left] {Differentials\\ and other special\\ sheaves; \\ D-modules; \\ Bernstein-Sato \\ ideals and polynomials};
\draw (314.8300000000001, -34.400000000000006) rectangle (320.9300000000001,-37.50000000000001);
\draw(307.6300000000001, -37.60000000000001) node[anchor=north west,align=left] {Derived categories\\ of sheaves,\\ dg categories,\\ and related \\ constructions in\\ algebraic geometry};
\draw (307.6300000000001, -37.60000000000001) rectangle (312.73000000000013,-40.70000000000001);
\draw(312.8300000000001, -37.60000000000001) node[anchor=north west,align=left] {Homotopy \\ theory and \\ fundamental groups\\ in algebraic\\ geometry};
\draw (312.8300000000001, -37.60000000000001) rectangle (317.9300000000001,-40.20000000000001);
\draw(318.0300000000001, -37.60000000000001) node[anchor=north west,align=left] {Étale and \\ other \\ Grothendieck \\ topologies and\\ (co)homologies};
\draw (318.0300000000001, -37.60000000000001) rectangle (322.1300000000001,-40.20000000000001);
\draw(307.6300000000001, -40.800000000000004) node[anchor=north west,align=left] {Classical \\ real and complex\\ (co)homology\\ in algebraic\\ geometry};
\draw (307.6300000000001, -40.800000000000004) rectangle (312.23000000000013,-43.400000000000006);
\draw(312.3300000000001, -40.800000000000004) node[anchor=north west,align=left] {Motivic \\ cohomology;\\ motivic \\ homotopy theory};
\draw (312.3300000000001, -40.800000000000004) rectangle (316.6800000000001,-42.900000000000006);
\draw(316.7800000000001, -40.800000000000004) node[anchor=north west,align=left] {de Rham \\ cohomology \\ and algebraic\\ geometry};
\draw (316.7800000000001, -40.800000000000004) rectangle (320.6300000000001,-42.900000000000006);
\draw(307.6300000000001, -43.50000000000001) node[anchor=north west,align=left] {Sheaves \\ in algebraic\\ geometry};
\draw (307.6300000000001, -43.50000000000001) rectangle (311.23000000000013,-45.10000000000001);
\draw(311.3300000000001, -43.50000000000001) node[anchor=north west,align=left] {Vanishing\\ theorems \\ in algebraic\\ geometry};
\draw (311.3300000000001, -43.50000000000001) rectangle (314.9300000000001,-45.60000000000001);
\draw(315.0300000000001, -43.50000000000001) node[anchor=north west,align=left] {Topological\\ properties\\ in algebraic\\ geometry};
\draw (315.0300000000001, -43.50000000000001) rectangle (318.6300000000001,-45.60000000000001);
\draw(318.73000000000013, -43.50000000000001) node[anchor=north west,align=left] {\(p\)-adic\\ cohomology,\\ crystalline\\ cohomology};
\draw (318.73000000000013, -43.50000000000001) rectangle (322.08000000000015,-45.60000000000001);
\draw(307.6300000000001, -45.7) node[anchor=north west,align=left] {Multiplier\\ ideals};
\draw (307.6300000000001, -45.7) rectangle (310.73000000000013,-47.300000000000004);
\draw(310.8300000000001, -45.7) node[anchor=north west,align=left] {Brauer \\ groups of\\ schemes};
\draw (310.8300000000001, -45.7) rectangle (313.6800000000001,-47.300000000000004);
\draw(282.7300000000001, -42.60000000000001) node[anchor=north west,align=left] {\large History of \\ algebraic geometry};
\draw (282.7300000000001, -42.60000000000001) rectangle (288.9100000000001,-43.70000000000001);
\draw(282.7300000000001, -47.50000000000001) node[anchor=north west,align=left] {\large Projective and enumerative algebraic geometry};
\draw (282.7300000000001, -47.50000000000001) rectangle (299.5800000000001,-57.10000000000001);
\draw(283.7300000000001, -48.50000000000001) node[anchor=north west,align=left] {Gromov-Witten \\ invariants, quantum \\ cohomology, Gopakumar-Vafa\\ invariants,\\ Donaldson-Thomas\\ invariants \\ (algebro-geometric aspects)};
\draw (283.7300000000001, -48.50000000000001) rectangle (291.0800000000001,-52.10000000000001);
\draw(291.18000000000006, -48.50000000000001) node[anchor=north west,align=left] {Enumerative \\ problems \\ (combinatorial \\ problems) in \\ algebraic geometry};
\draw (291.18000000000006, -48.50000000000001) rectangle (296.2800000000001,-51.10000000000001);
\draw(296.38000000000005, -48.50000000000001) node[anchor=north west,align=left] {Varieties\\ of \\ low degree};
\draw (296.38000000000005, -48.50000000000001) rectangle (299.4800000000001,-50.10000000000001);
\draw(283.7300000000001, -52.20000000000001) node[anchor=north west,align=left] {Secant \\ varieties, tensor\\ rank, \\ varieties of\\ sums of powers};
\draw (283.7300000000001, -52.20000000000001) rectangle (288.5800000000001,-54.80000000000001);
\draw(288.68000000000006, -52.20000000000001) node[anchor=north west,align=left] {Configurations\\ and \\ arrangements of \\ linear subspaces};
\draw (288.68000000000006, -52.20000000000001) rectangle (293.2800000000001,-54.30000000000001);
\draw(293.38000000000005, -52.20000000000001) node[anchor=north west,align=left] {Projective\\ techniques\\ in algebraic\\ geometry};
\draw (293.38000000000005, -52.20000000000001) rectangle (296.9800000000001,-54.30000000000001);
\draw(283.7300000000001, -54.900000000000006) node[anchor=north west,align=left] {Adjunction\\ problems};
\draw (283.7300000000001, -54.900000000000006) rectangle (286.8300000000001,-56.50000000000001);
\draw(286.93000000000006, -54.900000000000006) node[anchor=north west,align=left] {Classical\\ problems,\\ Schubert\\ calculus};
\draw (286.93000000000006, -54.900000000000006) rectangle (289.7800000000001,-57.00000000000001);
\draw(299.68000000000006, -47.50000000000001) node[anchor=north west,align=left] {\large Families, fibrations in algebraic geometry};
\draw (299.68000000000006, -47.50000000000001) rectangle (315.7800000000001,-60.30000000000001);
\draw(300.68000000000006, -48.50000000000001) node[anchor=north west,align=left] {Applications of \\ vector bundles and\\ moduli spaces in\\ mathematical \\ physics (twistor \\ theory, instantons,\\ quantum field theory)};
\draw (300.68000000000006, -48.50000000000001) rectangle (306.5300000000001,-52.10000000000001);
\draw(306.63000000000005, -48.50000000000001) node[anchor=north west,align=left] {Structure \\ of families\\ (Picard-Lefschetz,\\ monodromy, etc.)};
\draw (306.63000000000005, -48.50000000000001) rectangle (311.7300000000001,-51.10000000000001);
\draw(311.83000000000004, -48.50000000000001) node[anchor=north west,align=left] {Fine and\\ coarse \\ moduli spaces};
\draw (311.83000000000004, -48.50000000000001) rectangle (315.68000000000006,-50.10000000000001);
\draw(300.68000000000006, -52.20000000000001) node[anchor=north west,align=left] {Fibrations,\\ degenerations\\ in \\ algebraic geometry};
\draw (300.68000000000006, -52.20000000000001) rectangle (305.7800000000001,-54.30000000000001);
\draw(305.88000000000005, -52.20000000000001) node[anchor=north west,align=left] {Variation \\ of Hodge \\ structures \\ (algebro-geometric\\ aspects)};
\draw (305.88000000000005, -52.20000000000001) rectangle (310.9800000000001,-54.80000000000001);
\draw(311.08000000000004, -52.20000000000001) node[anchor=north west,align=left] {Algebraic \\ moduli problems,\\ moduli of\\ vector bundles};
\draw (311.08000000000004, -52.20000000000001) rectangle (315.68000000000006,-54.30000000000001);
\draw(300.68000000000006, -54.900000000000006) node[anchor=north west,align=left] {Formal methods\\ and deformations\\ in \\ algebraic geometry};
\draw (300.68000000000006, -54.900000000000006) rectangle (305.7800000000001,-57.00000000000001);
\draw(305.88000000000005, -54.900000000000006) node[anchor=north west,align=left] {Geometric \\ Langlands \\ program \\ (algebro-geometric\\ aspects)};
\draw (305.88000000000005, -54.900000000000006) rectangle (310.9800000000001,-57.50000000000001);
\draw(311.08000000000004, -54.900000000000006) node[anchor=north west,align=left] {Stacks \\ and moduli\\ problems};
\draw (311.08000000000004, -54.900000000000006) rectangle (314.18000000000006,-56.50000000000001);
\draw(300.68000000000006, -57.60000000000001) node[anchor=north west,align=left] {Arithmetic ground\\ fields (finite,\\ local, global)\\ and families\\ or fibrations};
\draw (300.68000000000006, -57.60000000000001) rectangle (305.5300000000001,-60.20000000000001);
\draw(282.7300000000001, -60.400000000000006) node[anchor=north west,align=left] {\large Surfaces and higher-dimensional varieties};
\draw (282.7300000000001, -60.400000000000006) rectangle (298.3300000000001,-77.30000000000001);
\draw(283.7300000000001, -61.400000000000006) node[anchor=north west,align=left] {Arithmetic \\ ground fields \\ for surfaces \\ or higher-dimensional\\ varieties};
\draw (283.7300000000001, -61.400000000000006) rectangle (289.5800000000001,-64.0);
\draw(289.68000000000006, -61.400000000000006) node[anchor=north west,align=left] {Topology of \\ surfaces (Donaldson\\ polynomials,\\ Seiberg-Witten\\ invariants)};
\draw (289.68000000000006, -61.400000000000006) rectangle (295.0300000000001,-64.0);
\draw(295.13000000000005, -61.400000000000006) node[anchor=north west,align=left] {Surfaces\\ of general\\ type};
\draw (295.13000000000005, -61.400000000000006) rectangle (298.2300000000001,-63.00000000000001);
\draw(283.7300000000001, -64.10000000000001) node[anchor=north west,align=left] {Moduli, \\ classification: \\ analytic theory;\\ relations \\ with modular forms};
\draw (283.7300000000001, -64.10000000000001) rectangle (288.8300000000001,-66.7);
\draw(288.93000000000006, -64.10000000000001) node[anchor=north west,align=left] {Singularities\\ of surfaces\\ or \\ higher-dimensional\\ varieties};
\draw (288.93000000000006, -64.10000000000001) rectangle (294.0300000000001,-66.7);
\draw(294.13000000000005, -64.10000000000001) node[anchor=north west,align=left] {Hypersurfaces\\ and\\ algebraic\\ geometry};
\draw (294.13000000000005, -64.10000000000001) rectangle (297.9800000000001,-66.2);
\draw(283.7300000000001, -66.80000000000001) node[anchor=north west,align=left] {Elliptic \\ surfaces, elliptic\\ or Calabi-Yau\\ fibrations};
\draw (283.7300000000001, -66.80000000000001) rectangle (288.8300000000001,-68.9);
\draw(288.93000000000006, -66.80000000000001) node[anchor=north west,align=left] {Calabi-Yau \\ manifolds \\ (algebro-geometric\\ aspects)};
\draw (288.93000000000006, -66.80000000000001) rectangle (294.0300000000001,-68.9);
\draw(294.13000000000005, -66.80000000000001) node[anchor=north west,align=left] {Relationships\\ with physics};
\draw (294.13000000000005, -66.80000000000001) rectangle (297.9800000000001,-68.4);
\draw(283.7300000000001, -69.0) node[anchor=north west,align=left] {Mirror \\ symmetry \\ (algebro-geometric\\ aspects)};
\draw (283.7300000000001, -69.0) rectangle (288.8300000000001,-71.1);
\draw(288.93000000000006, -69.0) node[anchor=north west,align=left] {Automorphisms\\ of surfaces\\ and \\ higher-dimensional\\ varieties};
\draw (288.93000000000006, -69.0) rectangle (294.0300000000001,-71.6);
\draw(294.13000000000005, -69.0) node[anchor=north west,align=left] {\(K3\) \\ surfaces and\\ Enriques\\ surfaces};
\draw (294.13000000000005, -69.0) rectangle (297.7300000000001,-71.1);
\draw(283.7300000000001, -71.7) node[anchor=north west,align=left] {Vector bundles\\ on surfaces and\\ higher-dimensional\\ varieties,\\ and their moduli};
\draw (283.7300000000001, -71.7) rectangle (288.8300000000001,-74.3);
\draw(288.93000000000006, -71.7) node[anchor=north west,align=left] {Families, \\ moduli, \\ classification: \\ algebraic theory};
\draw (288.93000000000006, -71.7) rectangle (293.5300000000001,-73.8);
\draw(293.63000000000005, -71.7) node[anchor=north west,align=left] {Holomorphic\\ symplectic\\ varieties,\\ hyper-Kähler\\ varieties};
\draw (293.63000000000005, -71.7) rectangle (297.2300000000001,-74.3);
\draw(283.7300000000001, -74.4) node[anchor=north west,align=left] {\(3\)-folds};
\draw (283.7300000000001, -74.4) rectangle (287.0800000000001,-75.5);
\draw(287.18000000000006, -74.4) node[anchor=north west,align=left] {\(4\)-folds};
\draw (287.18000000000006, -74.4) rectangle (290.5300000000001,-75.5);
\draw(290.63000000000005, -74.4) node[anchor=north west,align=left] {\(n\)-folds\\ (\(n>4\))};
\draw (290.63000000000005, -74.4) rectangle (293.9800000000001,-76.0);
\draw(294.0800000000001, -74.4) node[anchor=north west,align=left] {Rational\\ and ruled\\ surfaces};
\draw (294.0800000000001, -74.4) rectangle (296.9300000000001,-76.0);
\draw(283.7300000000001, -76.10000000000001) node[anchor=north west,align=left] {Fano \\ varieties};
\draw (283.7300000000001, -76.10000000000001) rectangle (286.5800000000001,-77.2);
\draw(286.68000000000006, -76.10000000000001) node[anchor=north west,align=left] {Special\\ surfaces};
\draw (286.68000000000006, -76.10000000000001) rectangle (289.2800000000001,-77.2);
\draw(298.43000000000006, -60.400000000000006) node[anchor=north west,align=left] {\large Computational aspects in algebraic geometry};
\draw (298.43000000000006, -60.400000000000006) rectangle (314.0300000000001,-68.5);
\draw(299.43000000000006, -61.400000000000006) node[anchor=north west,align=left] {Geometric \\ aspects of \\ numerical algebraic\\ geometry};
\draw (299.43000000000006, -61.400000000000006) rectangle (304.7800000000001,-63.50000000000001);
\draw(304.88000000000005, -61.400000000000006) node[anchor=north west,align=left] {Computational\\ aspects of \\ higher-dimensional\\ varieties};
\draw (304.88000000000005, -61.400000000000006) rectangle (309.9800000000001,-63.50000000000001);
\draw(310.08000000000004, -61.400000000000006) node[anchor=north west,align=left] {Computational\\ aspects\\ of algebraic\\ curves};
\draw (310.08000000000004, -61.400000000000006) rectangle (313.93000000000006,-63.50000000000001);
\draw(299.43000000000006, -63.60000000000001) node[anchor=north west,align=left] {Effectivity, \\ complexity and\\ computational\\ aspects of \\ algebraic geometry};
\draw (299.43000000000006, -63.60000000000001) rectangle (304.5300000000001,-66.2);
\draw(304.63000000000005, -63.60000000000001) node[anchor=north west,align=left] {Computational\\ aspects\\ of algebraic\\ surfaces};
\draw (304.63000000000005, -63.60000000000001) rectangle (308.4800000000001,-65.7);
\draw(308.58000000000004, -63.60000000000001) node[anchor=north west,align=left] {Computational\\ algebraic\\ geometry over\\ arithmetic\\ ground fields};
\draw (308.58000000000004, -63.60000000000001) rectangle (312.43000000000006,-66.2);
\draw(299.43000000000006, -66.30000000000001) node[anchor=north west,align=left] {Computational\\ real\\ algebraic\\ geometry};
\draw (299.43000000000006, -66.30000000000001) rectangle (303.2800000000001,-68.4);
\draw(314.13000000000005, -60.400000000000006) node[anchor=north west,align=left] {\large Tropical geometry};
\draw (314.13000000000005, -60.400000000000006) rectangle (322.03000000000003,-70.7);
\draw(315.13000000000005, -61.400000000000006) node[anchor=north west,align=left] {Foundations\\ of tropical\\ geometry\\ and relations\\ with algebra};
\draw (315.13000000000005, -61.400000000000006) rectangle (318.9800000000001,-64.0);
\draw(315.13000000000005, -64.10000000000001) node[anchor=north west,align=left] {Combinatorial\\ aspects\\ of tropical\\ varieties};
\draw (315.13000000000005, -64.10000000000001) rectangle (318.9800000000001,-66.2);
\draw(315.13000000000005, -66.30000000000001) node[anchor=north west,align=left] {Applications\\ of\\ tropical\\ geometry};
\draw (315.13000000000005, -66.30000000000001) rectangle (318.7300000000001,-68.4);
\draw(315.13000000000005, -68.5) node[anchor=north west,align=left] {Geometric\\ aspects\\ of tropical\\ varieties};
\draw (315.13000000000005, -68.5) rectangle (318.4800000000001,-70.6);
\draw(318.58000000000004, -68.5) node[anchor=north west,align=left] {Arithmetic\\ aspects \\ of tropical\\ varieties};
\draw (318.58000000000004, -68.5) rectangle (321.93000000000006,-70.6);
\draw(282.7300000000001, -77.4) node[anchor=north west,align=left] {\large Real algebraic and real-analytic geometry};
\draw (282.7300000000001, -77.4) rectangle (296.0400000000001,-82.80000000000001);
\draw(283.7300000000001, -78.4) node[anchor=north west,align=left] {Semialgebraic\\ sets\\ and related\\ spaces};
\draw (283.7300000000001, -78.4) rectangle (287.5800000000001,-80.5);
\draw(287.68000000000006, -78.4) node[anchor=north west,align=left] {Real-analytic\\ and\\ semi-analytic\\ sets};
\draw (287.68000000000006, -78.4) rectangle (291.5300000000001,-80.5);
\draw(291.63000000000005, -78.4) node[anchor=north west,align=left] {Nash \\ functions and\\ manifolds};
\draw (291.63000000000005, -78.4) rectangle (295.4800000000001,-80.0);
\draw(283.7300000000001, -80.60000000000001) node[anchor=north west,align=left] {Real \\ algebraic\\ sets};
\draw (283.7300000000001, -80.60000000000001) rectangle (286.5800000000001,-82.2);
\draw(286.68000000000006, -80.60000000000001) node[anchor=north west,align=left] {Topology\\ of real \\ algebraic\\ varieties};
\draw (286.68000000000006, -80.60000000000001) rectangle (289.5300000000001,-82.7);
\draw(296.1400000000001, -77.4) node[anchor=north west,align=left] {\large Local theory in algebraic geometry};
\draw (296.1400000000001, -77.4) rectangle (308.9900000000001,-85.5);
\draw(297.1400000000001, -78.4) node[anchor=north west,align=left] {Local deformation\\ theory,\\ Artin \\ approximation, etc.};
\draw (297.1400000000001, -78.4) rectangle (302.4900000000001,-80.5);
\draw(302.5900000000001, -78.4) node[anchor=north west,align=left] {Local structure\\ of morphisms\\ in algebraic\\ geometry:\\ étale, flat, etc.};
\draw (302.5900000000001, -78.4) rectangle (307.4400000000001,-81.0);
\draw(297.1400000000001, -81.10000000000001) node[anchor=north west,align=left] {Singularities\\ in\\ algebraic\\ geometry};
\draw (297.1400000000001, -81.10000000000001) rectangle (300.9900000000001,-83.2);
\draw(301.0900000000001, -81.10000000000001) node[anchor=north west,align=left] {Deformations\\ of \\ singularities};
\draw (301.0900000000001, -81.10000000000001) rectangle (304.9400000000001,-82.7);
\draw(305.0400000000001, -81.10000000000001) node[anchor=north west,align=left] {Infinitesimal\\ methods\\ in algebraic\\ geometry};
\draw (305.0400000000001, -81.10000000000001) rectangle (308.8900000000001,-83.2);
\draw(297.1400000000001, -83.30000000000001) node[anchor=north west,align=left] {Local \\ cohomology \\ and algebraic\\ geometry};
\draw (297.1400000000001, -83.30000000000001) rectangle (300.9900000000001,-85.4);
\draw(301.0900000000001, -83.30000000000001) node[anchor=north west,align=left] {Formal \\ neighborhoods\\ in algebraic\\ geometry};
\draw (301.0900000000001, -83.30000000000001) rectangle (304.9400000000001,-85.4);
\draw(309.0900000000001, -77.4) node[anchor=north west,align=left] {\large Foundations of algebraic geometry};
\draw (309.0900000000001, -77.4) rectangle (321.49000000000007,-88.7);
\draw(310.0900000000001, -78.4) node[anchor=north west,align=left] {Fundamental constructions\\ in algebraic\\ geometry involving\\ higher and derived\\ categories (homotopical\\ algebraic \\ geometry, derived \\ algebraic geometry, etc.)};
\draw (310.0900000000001, -78.4) rectangle (316.9400000000001,-82.5);
\draw(317.0400000000001, -78.4) node[anchor=north west,align=left] {Generalizations\\ (algebraic \\ spaces, stacks)};
\draw (317.0400000000001, -78.4) rectangle (321.3900000000001,-80.5);
\draw(310.0900000000001, -82.60000000000001) node[anchor=north west,align=left] {Noncommutative\\ algebraic\\ geometry};
\draw (310.0900000000001, -82.60000000000001) rectangle (314.1900000000001,-84.7);
\draw(314.2900000000001, -82.60000000000001) node[anchor=north west,align=left] {Elementary\\ questions\\ in algebraic\\ geometry};
\draw (314.2900000000001, -82.60000000000001) rectangle (317.8900000000001,-84.7);
\draw(317.99000000000007, -82.60000000000001) node[anchor=north west,align=left] {Relevant\\ commutative\\ algebra};
\draw (317.99000000000007, -82.60000000000001) rectangle (321.3400000000001,-84.2);
\draw(310.0900000000001, -84.80000000000001) node[anchor=north west,align=left] {Logarithmic\\ algebraic\\ geometry,\\ log schemes};
\draw (310.0900000000001, -84.80000000000001) rectangle (313.4400000000001,-86.9);
\draw(313.5400000000001, -84.80000000000001) node[anchor=north west,align=left] {Geometry \\ over the \\ field with \\ one element};
\draw (313.5400000000001, -84.80000000000001) rectangle (316.8900000000001,-86.9);
\draw(316.99000000000007, -84.80000000000001) node[anchor=north west,align=left] {Varieties\\ and\\ morphisms};
\draw (316.99000000000007, -84.80000000000001) rectangle (319.8400000000001,-86.4);
\draw(310.0900000000001, -87.0) node[anchor=north west,align=left] {Schemes\\ and \\ morphisms};
\draw (310.0900000000001, -87.0) rectangle (312.9400000000001,-88.6);
\draw(282.7300000000001, -88.80000000000001) node[anchor=north west,align=left] {\large Special varieties};
\draw (282.7300000000001, -88.80000000000001) rectangle (295.13000000000005,-101.30000000000001);
\draw(283.7300000000001, -89.80000000000001) node[anchor=north west,align=left] {Varieties defined\\ by ring \\ conditions (factorial,\\ Cohen-Macaulay,\\ seminormal)};
\draw (283.7300000000001, -89.80000000000001) rectangle (289.8300000000001,-92.4);
\draw(289.93000000000006, -89.80000000000001) node[anchor=north west,align=left] {Compactifications;\\ symmetric\\ and spherical\\ varieties};
\draw (289.93000000000006, -89.80000000000001) rectangle (295.0300000000001,-91.9);
\draw(283.7300000000001, -92.50000000000001) node[anchor=north west,align=left] {Determinantalvarieties};
\draw (283.7300000000001, -92.50000000000001) rectangle (289.8300000000001,-94.10000000000001);
\draw(289.93000000000006, -92.50000000000001) node[anchor=north west,align=left] {Toric varieties,\\ Newton\\ polyhedra, \\ Okounkov bodies};
\draw (289.93000000000006, -92.50000000000001) rectangle (294.5300000000001,-94.60000000000001);
\draw(283.7300000000001, -94.70000000000002) node[anchor=north west,align=left] {Low codimension\\ problems\\ in algebraic\\ geometry};
\draw (283.7300000000001, -94.70000000000002) rectangle (288.0800000000001,-96.80000000000001);
\draw(288.18000000000006, -94.70000000000002) node[anchor=north west,align=left] {Homogeneous\\ spaces\\ and \\ generalizations};
\draw (288.18000000000006, -94.70000000000002) rectangle (292.5300000000001,-96.80000000000001);
\draw(292.63000000000005, -94.70000000000002) node[anchor=north west,align=left] {Linkage};
\draw (292.63000000000005, -94.70000000000002) rectangle (294.9800000000001,-95.80000000000001);
\draw(283.7300000000001, -96.9) node[anchor=north west,align=left] {Grassmannians,\\ Schubert\\ varieties, \\ flag manifolds};
\draw (283.7300000000001, -96.9) rectangle (287.8300000000001,-99.0);
\draw(287.93000000000006, -96.9) node[anchor=north west,align=left] {Supervarieties};
\draw (287.93000000000006, -96.9) rectangle (292.0300000000001,-98.0);
\draw(292.13000000000005, -96.9) node[anchor=north west,align=left] {Character\\ varieties};
\draw (292.13000000000005, -96.9) rectangle (294.9800000000001,-98.5);
\draw(283.7300000000001, -99.10000000000001) node[anchor=north west,align=left] {Complete\\ intersections};
\draw (283.7300000000001, -99.10000000000001) rectangle (287.5800000000001,-100.7);
\draw(287.68000000000006, -99.10000000000001) node[anchor=north west,align=left] {Rational\\ and \\ unirational\\ varieties};
\draw (287.68000000000006, -99.10000000000001) rectangle (291.0300000000001,-101.2);
\draw(291.13000000000005, -99.10000000000001) node[anchor=north west,align=left] {Rationally\\ connected\\ varieties};
\draw (291.13000000000005, -99.10000000000001) rectangle (294.2300000000001,-100.7);
\draw(295.2300000000001, -88.80000000000001) node[anchor=north west,align=left] {\large Curves in algebraic geometry};
\draw (295.2300000000001, -88.80000000000001) rectangle (307.0800000000001,-106.9);
\draw(296.2300000000001, -89.80000000000001) node[anchor=north west,align=left] {Special \\ divisors on \\ curves (gonality,\\ Brill-Noether theory)};
\draw (296.2300000000001, -89.80000000000001) rectangle (302.0800000000001,-92.4);
\draw(302.18000000000006, -89.80000000000001) node[anchor=north west,align=left] {Theta functions\\ and \\ curves; Schottky\\ problem};
\draw (302.18000000000006, -89.80000000000001) rectangle (306.7800000000001,-91.9);
\draw(296.2300000000001, -92.50000000000001) node[anchor=north west,align=left] {Riemann \\ surfaces; Weierstrass\\ points;\\ gap sequences};
\draw (296.2300000000001, -92.50000000000001) rectangle (302.0800000000001,-94.60000000000001);
\draw(302.18000000000006, -92.50000000000001) node[anchor=north west,align=left] {Special \\ algebraic curves\\ and curves\\ of low genus};
\draw (302.18000000000006, -92.50000000000001) rectangle (306.7800000000001,-94.60000000000001);
\draw(296.2300000000001, -94.70000000000002) node[anchor=north west,align=left] {Algebraic \\ functions and\\ function \\ fields in algebraic\\ geometry};
\draw (296.2300000000001, -94.70000000000002) rectangle (301.5800000000001,-97.30000000000001);
\draw(301.68000000000006, -94.70000000000002) node[anchor=north west,align=left] {Relationships\\ between \\ algebraic curves\\ and \\ integrable systems};
\draw (301.68000000000006, -94.70000000000002) rectangle (306.7800000000001,-97.30000000000001);
\draw(296.2300000000001, -97.4) node[anchor=north west,align=left] {Relationships\\ between \\ algebraic curves\\ and physics};
\draw (296.2300000000001, -97.4) rectangle (300.8300000000001,-99.5);
\draw(300.93000000000006, -97.4) node[anchor=north west,align=left] {Automorphismsof\\ curves};
\draw (300.93000000000006, -97.4) rectangle (305.2800000000001,-99.0);
\draw(296.2300000000001, -99.60000000000001) node[anchor=north west,align=left] {Singularities\\ of \\ curves, \\ local rings};
\draw (296.2300000000001, -99.60000000000001) rectangle (300.0800000000001,-101.7);
\draw(300.18000000000006, -99.60000000000001) node[anchor=north west,align=left] {Vector \\ bundles on \\ curves and \\ their moduli};
\draw (300.18000000000006, -99.60000000000001) rectangle (303.7800000000001,-101.7);
\draw(303.88000000000005, -99.60000000000001) node[anchor=north west,align=left] {Families,\\ moduli \\ of curves\\ (analytic)};
\draw (303.88000000000005, -99.60000000000001) rectangle (306.9800000000001,-101.7);
\draw(296.2300000000001, -101.80000000000001) node[anchor=north west,align=left] {Families,\\ moduli \\ of curves\\ (algebraic)};
\draw (296.2300000000001, -101.80000000000001) rectangle (299.5800000000001,-103.9);
\draw(299.68000000000006, -101.80000000000001) node[anchor=north west,align=left] {Coverings\\ of curves,\\ fundamental\\ group};
\draw (299.68000000000006, -101.80000000000001) rectangle (303.0300000000001,-103.9);
\draw(303.13000000000005, -101.80000000000001) node[anchor=north west,align=left] {Arithmetic\\ ground\\ fields\\ for curves};
\draw (303.13000000000005, -101.80000000000001) rectangle (306.2300000000001,-103.9);
\draw(296.2300000000001, -104.00000000000001) node[anchor=north west,align=left] {Jacobians,\\ Prym\\ varieties};
\draw (296.2300000000001, -104.00000000000001) rectangle (299.3300000000001,-105.60000000000001);
\draw(299.43000000000006, -104.00000000000001) node[anchor=north west,align=left] {Plane \\ and space\\ curves};
\draw (299.43000000000006, -104.00000000000001) rectangle (302.2800000000001,-105.60000000000001);
\draw(302.38000000000005, -104.00000000000001) node[anchor=north west,align=left] {Dessins\\ d’enfants\\ theory};
\draw (302.38000000000005, -104.00000000000001) rectangle (305.2300000000001,-105.60000000000001);
\draw(296.2300000000001, -105.70000000000002) node[anchor=north west,align=left] {Elliptic\\ curves};
\draw (296.2300000000001, -105.70000000000002) rectangle (298.8300000000001,-106.80000000000001);
\draw(307.18000000000006, -88.80000000000001) node[anchor=north west,align=left] {\large Cycles and subschemes};
\draw (307.18000000000006, -88.80000000000001) rectangle (318.58000000000004,-101.30000000000001);
\draw(308.18000000000006, -89.80000000000001) node[anchor=north west,align=left] {Intersection \\ theory, characteristic\\ classes,\\ intersection \\ multiplicities in\\ algebraic geometry};
\draw (308.18000000000006, -89.80000000000001) rectangle (314.2800000000001,-92.9);
\draw(314.38000000000005, -89.80000000000001) node[anchor=north west,align=left] {(Equivariant)\\ Chow \\ groups and \\ rings; motives};
\draw (314.38000000000005, -89.80000000000001) rectangle (318.4800000000001,-91.9);
\draw(308.18000000000006, -93.00000000000001) node[anchor=north west,align=left] {Transcendental\\ methods,\\ Hodge theory\\ (algebro-geometric\\ aspects)};
\draw (308.18000000000006, -93.00000000000001) rectangle (313.2800000000001,-95.60000000000001);
\draw(313.38000000000005, -93.00000000000001) node[anchor=north west,align=left] {Applications \\ of methods of \\ algebraic \\ \(K\)-theory in \\ algebraic geometry};
\draw (313.38000000000005, -93.00000000000001) rectangle (318.4800000000001,-95.60000000000001);
\draw(308.18000000000006, -95.70000000000002) node[anchor=north west,align=left] {Parametrization\\ (Chow\\ and Hilbert\\ schemes)};
\draw (308.18000000000006, -95.70000000000002) rectangle (312.5300000000001,-97.80000000000001);
\draw(312.63000000000005, -95.70000000000002) node[anchor=north west,align=left] {Divisors, \\ linear systems,\\ invertible\\ sheaves};
\draw (312.63000000000005, -95.70000000000002) rectangle (316.9800000000001,-97.80000000000001);
\draw(308.18000000000006, -97.9) node[anchor=north west,align=left] {Pencils, \\ nets, webs\\ in algebraic\\ geometry};
\draw (308.18000000000006, -97.9) rectangle (311.7800000000001,-100.0);
\draw(311.88000000000005, -97.9) node[anchor=north west,align=left] {Riemann-Roch\\ theorems};
\draw (311.88000000000005, -97.9) rectangle (315.4800000000001,-99.5);
\draw(315.58000000000004, -97.9) node[anchor=north west,align=left] {Algebraic\\ cycles};
\draw (315.58000000000004, -97.9) rectangle (318.43000000000006,-99.0);
\draw(308.18000000000006, -100.10000000000001) node[anchor=north west,align=left] {Torelli\\ problem};
\draw (308.18000000000006, -100.10000000000001) rectangle (310.5300000000001,-101.2);
\draw(310.63000000000005, -100.10000000000001) node[anchor=north west,align=left] {Picard\\ groups};
\draw (310.63000000000005, -100.10000000000001) rectangle (312.7300000000001,-101.2);
\draw(282.7300000000001, -107.0) node[anchor=north west,align=left] {\large Abelian varieties and schemes};
\draw (282.7300000000001, -107.0) rectangle (293.88000000000005,-117.3);
\draw(283.7300000000001, -108.0) node[anchor=north west,align=left] {Analytic theory\\ of abelian \\ varieties; abelian\\ integrals\\ and differentials};
\draw (283.7300000000001, -108.0) rectangle (288.8300000000001,-110.6);
\draw(288.93000000000006, -108.0) node[anchor=north west,align=left] {Algebraic \\ moduli of abelian\\ varieties,\\ classification};
\draw (288.93000000000006, -108.0) rectangle (293.7800000000001,-110.1);
\draw(283.7300000000001, -110.7) node[anchor=north west,align=left] {Picard \\ schemes, higher\\ Jacobians};
\draw (283.7300000000001, -110.7) rectangle (288.0800000000001,-112.3);
\draw(288.18000000000006, -110.7) node[anchor=north west,align=left] {Complex \\ multiplication\\ and abelian\\ varieties};
\draw (288.18000000000006, -110.7) rectangle (292.2800000000001,-112.8);
\draw(283.7300000000001, -112.9) node[anchor=north west,align=left] {Arithmetic\\ ground fields\\ for abelian\\ varieties};
\draw (283.7300000000001, -112.9) rectangle (287.5800000000001,-115.0);
\draw(287.68000000000006, -112.9) node[anchor=north west,align=left] {Theta \\ functions and\\ abelian\\ varieties};
\draw (287.68000000000006, -112.9) rectangle (291.5300000000001,-115.0);
\draw(283.7300000000001, -115.1) node[anchor=north west,align=left] {Subvarieties\\ of\\ abelian\\ varieties};
\draw (283.7300000000001, -115.1) rectangle (287.3300000000001,-117.19999999999999);
\draw(287.43000000000006, -115.1) node[anchor=north west,align=left] {Algebraic\\ theory \\ of abelian\\ varieties};
\draw (287.43000000000006, -115.1) rectangle (290.5300000000001,-117.19999999999999);
\draw(290.63000000000005, -115.1) node[anchor=north west,align=left] {Isogeny};
\draw (290.63000000000005, -115.1) rectangle (292.9800000000001,-116.19999999999999);
\draw(293.9800000000001, -107.0) node[anchor=north west,align=left] {\large Affine geometry};
\draw (293.9800000000001, -107.0) rectangle (304.13000000000005,-114.6);
\draw(294.9800000000001, -108.0) node[anchor=north west,align=left] {Affine spaces\\ (automorphisms,\\ embeddings,\\ exotic \\ structures, \\ cancellation problem)};
\draw (294.9800000000001, -108.0) rectangle (300.8300000000001,-111.1);
\draw(300.93000000000006, -108.0) node[anchor=north west,align=left] {Group \\ actions on\\ affine\\ varieties};
\draw (300.93000000000006, -108.0) rectangle (304.0300000000001,-110.1);
\draw(294.9800000000001, -111.2) node[anchor=north west,align=left] {Classification\\ of affine\\ varieties};
\draw (294.9800000000001, -111.2) rectangle (299.0800000000001,-113.3);
\draw(299.18000000000006, -111.2) node[anchor=north west,align=left] {Affine\\ fibrations};
\draw (299.18000000000006, -111.2) rectangle (302.2800000000001,-112.8);
\draw(294.9800000000001, -113.4) node[anchor=north west,align=left] {Jacobian\\ problem};
\draw (294.9800000000001, -113.4) rectangle (297.5800000000001,-114.5);
\draw(304.2300000000001, -107.0) node[anchor=north west,align=left] {\large Birational geometry};
\draw (304.2300000000001, -107.0) rectangle (313.88000000000005,-119.5);
\draw(305.2300000000001, -108.0) node[anchor=north west,align=left] {Global theory\\ and resolution\\ of singularities\\ (algebro-geometric\\ aspects)};
\draw (305.2300000000001, -108.0) rectangle (310.3300000000001,-110.6);
\draw(310.43000000000006, -108.0) node[anchor=north west,align=left] {Arcs and\\ motivic \\ integration};
\draw (310.43000000000006, -108.0) rectangle (313.7800000000001,-109.6);
\draw(305.2300000000001, -110.7) node[anchor=north west,align=left] {Rational\\ and \\ birational maps};
\draw (305.2300000000001, -110.7) rectangle (309.5800000000001,-112.3);
\draw(309.68000000000006, -110.7) node[anchor=north west,align=left] {McKay\\ correspondence};
\draw (309.68000000000006, -110.7) rectangle (313.7800000000001,-112.3);
\draw(305.2300000000001, -112.4) node[anchor=north west,align=left] {Birational\\ automorphisms,\\ Cremona\\ group and\\ generalizations};
\draw (305.2300000000001, -112.4) rectangle (309.5800000000001,-115.0);
\draw(309.68000000000006, -112.4) node[anchor=north west,align=left] {Minimal model\\ program \\ (Mori theory,\\ extremal rays)};
\draw (309.68000000000006, -112.4) rectangle (313.7800000000001,-114.5);
\draw(305.2300000000001, -115.1) node[anchor=north west,align=left] {Rationality\\ questions\\ in algebraic\\ geometry};
\draw (305.2300000000001, -115.1) rectangle (308.8300000000001,-117.19999999999999);
\draw(308.93000000000006, -115.1) node[anchor=north west,align=left] {Coverings\\ in algebraic\\ geometry};
\draw (308.93000000000006, -115.1) rectangle (312.5300000000001,-116.69999999999999);
\draw(305.2300000000001, -117.3) node[anchor=north west,align=left] {Ramification\\ problems\\ in algebraic\\ geometry};
\draw (305.2300000000001, -117.3) rectangle (308.8300000000001,-119.39999999999999);
\draw(308.93000000000006, -117.3) node[anchor=north west,align=left] {Embeddings\\ in algebraic\\ geometry};
\draw (308.93000000000006, -117.3) rectangle (312.5300000000001,-118.89999999999999);
\draw(282.7300000000001, -119.6) node[anchor=north west,align=left] {\large Algebraic groups};
\draw (282.7300000000001, -119.6) rectangle (292.38000000000005,-129.4);
\draw(283.7300000000001, -120.6) node[anchor=north west,align=left] {Classical \\ groups \\ (algebro-geometric\\ aspects)};
\draw (283.7300000000001, -120.6) rectangle (288.8300000000001,-122.69999999999999);
\draw(288.93000000000006, -120.6) node[anchor=north west,align=left] {Group \\ varieties};
\draw (288.93000000000006, -120.6) rectangle (291.7800000000001,-121.69999999999999);
\draw(283.7300000000001, -122.8) node[anchor=north west,align=left] {Affine algebraic\\ groups,\\ hyperalgebra\\ constructions};
\draw (283.7300000000001, -122.8) rectangle (288.3300000000001,-124.89999999999999);
\draw(288.43000000000006, -122.8) node[anchor=north west,align=left] {Group actions\\ on varieties\\ or schemes\\ (quotients)};
\draw (288.43000000000006, -122.8) rectangle (292.2800000000001,-124.89999999999999);
\draw(283.7300000000001, -125.0) node[anchor=north west,align=left] {Other \\ algebraic groups\\ (geometric\\ aspects)};
\draw (283.7300000000001, -125.0) rectangle (288.3300000000001,-127.1);
\draw(288.43000000000006, -125.0) node[anchor=north west,align=left] {Geometric\\ invariant\\ theory};
\draw (288.43000000000006, -125.0) rectangle (291.2800000000001,-126.6);
\draw(283.7300000000001, -127.19999999999999) node[anchor=north west,align=left] {Formal \\ groups, \\ \(p\)-divisible\\ groups};
\draw (283.7300000000001, -127.19999999999999) rectangle (288.0800000000001,-129.29999999999998);
\draw(288.18000000000006, -127.19999999999999) node[anchor=north west,align=left] {Group\\ schemes};
\draw (288.18000000000006, -127.19999999999999) rectangle (290.5300000000001,-128.29999999999998);
\draw(323.3700000000001, -1) node[anchor=north west,align=left] {\LARGE Commutative algebra};
\draw (323.3700000000001, -1) rectangle (363.6800000000001,-49.6);
\draw(324.3700000000001, -2) node[anchor=north west,align=left] {\large Chain conditions, finiteness conditions in commutative ring theory};
\draw (324.3700000000001, -2) rectangle (345.4300000000001,-6.199999999999999);
\draw(325.3700000000001, -3) node[anchor=north west,align=left] {Commutative \\ rings and modules\\ of finite \\ generation or \\ presentation; \\ number of generators};
\draw (325.3700000000001, -3) rectangle (330.97000000000014,-6.1);
\draw(331.0700000000001, -3) node[anchor=north west,align=left] {Commutative \\ Artinian rings\\ and modules,\\ finite-dimensional\\ algebras};
\draw (331.0700000000001, -3) rectangle (336.17000000000013,-5.6);
\draw(336.2700000000001, -3) node[anchor=north west,align=left] {Commutative\\ Noetherian\\ rings \\ and modules};
\draw (336.2700000000001, -3) rectangle (339.6200000000001,-5.1);
\draw(345.53000000000014, -2) node[anchor=north west,align=left] {\large Theory of modules and ideals in commutative rings};
\draw (345.53000000000014, -2) rectangle (363.58000000000015,-13.3);
\draw(346.53000000000014, -3) node[anchor=north west,align=left] {Structure, \\ classification \\ theorems for modules\\ and ideals in\\ commutative rings};
\draw (346.53000000000014, -3) rectangle (352.13000000000017,-5.6);
\draw(352.23000000000013, -3) node[anchor=north west,align=left] {Dimension \\ theory, depth, \\ related commutative\\ rings \\ (catenary, etc.)};
\draw (352.23000000000013, -3) rectangle (357.58000000000015,-5.6);
\draw(357.6800000000001, -3) node[anchor=north west,align=left] {Projective\\ and free \\ modules and \\ ideals in \\ commutative rings};
\draw (357.6800000000001, -3) rectangle (362.53000000000014,-5.6);
\draw(346.53000000000014, -5.7) node[anchor=north west,align=left] {Injective \\ and flat \\ modules and \\ ideals in \\ commutative rings};
\draw (346.53000000000014, -5.7) rectangle (351.38000000000017,-8.3);
\draw(351.48000000000013, -5.7) node[anchor=north west,align=left] {Torsion \\ modules and \\ ideals in \\ commutative rings};
\draw (351.48000000000013, -5.7) rectangle (356.33000000000015,-7.800000000000001);
\draw(356.4300000000001, -5.7) node[anchor=north west,align=left] {Other special\\ types of \\ modules and \\ ideals in \\ commutative rings};
\draw (356.4300000000001, -5.7) rectangle (361.28000000000014,-8.3);
\draw(361.38000000000017, -5.7) node[anchor=north west,align=left] {Class\\ groups};
\draw (361.38000000000017, -5.7) rectangle (363.4800000000002,-6.800000000000001);
\draw(346.53000000000014, -8.4) node[anchor=north west,align=left] {Linkage, \\ complete \\ intersections \\ and determinantal\\ ideals};
\draw (346.53000000000014, -8.4) rectangle (351.38000000000017,-11.0);
\draw(351.48000000000013, -8.4) node[anchor=north west,align=left] {Module \\ categories\\ and \\ commutative rings};
\draw (351.48000000000013, -8.4) rectangle (356.33000000000015,-10.5);
\draw(356.4300000000001, -8.4) node[anchor=north west,align=left] {Theory of modules\\ and ideals\\ in commutative\\ rings described\\ by combinatorial\\ properties};
\draw (356.4300000000001, -8.4) rectangle (361.28000000000014,-11.5);
\draw(346.53000000000014, -11.600000000000001) node[anchor=north west,align=left] {Cohen-Macaulay\\ modules};
\draw (346.53000000000014, -11.600000000000001) rectangle (350.63000000000017,-13.200000000000001);
\draw(324.3700000000001, -6.299999999999999) node[anchor=north west,align=left] {\large Computational aspects and applications of commutative rings};
\draw (324.3700000000001, -6.299999999999999) rectangle (344.92000000000013,-12.7);
\draw(325.3700000000001, -7.299999999999999) node[anchor=north west,align=left] {Applications of\\ commutative \\ algebra (e.g., to\\ statistics, \\ control theory, \\ optimization, etc.)};
\draw (325.3700000000001, -7.299999999999999) rectangle (330.72000000000014,-10.399999999999999);
\draw(330.8200000000001, -7.299999999999999) node[anchor=north west,align=left] {Gröbner bases; \\ other bases for \\ ideals and modules\\ (e.g., Janet \\ and border bases)};
\draw (330.8200000000001, -7.299999999999999) rectangle (335.92000000000013,-9.899999999999999);
\draw(336.0200000000001, -7.299999999999999) node[anchor=north west,align=left] {Polynomials,\\ factorization\\ in \\ commutative rings};
\draw (336.0200000000001, -7.299999999999999) rectangle (340.8700000000001,-9.399999999999999);
\draw(340.97000000000014, -7.299999999999999) node[anchor=north west,align=left] {Computational\\ homological\\ algebra};
\draw (340.97000000000014, -7.299999999999999) rectangle (344.82000000000016,-9.399999999999999);
\draw(325.3700000000001, -10.5) node[anchor=north west,align=left] {Solving \\ polynomial\\ systems;\\ resultants};
\draw (325.3700000000001, -10.5) rectangle (328.47000000000014,-12.6);
\draw(324.3700000000001, -13.4) node[anchor=north west,align=left] {\large Arithmetic rings and other special commutative rings};
\draw (324.3700000000001, -13.4) rectangle (344.17000000000013,-23.700000000000003);
\draw(325.3700000000001, -14.4) node[anchor=north west,align=left] {Commutative rings\\ defined by \\ monomial ideals;\\ Stanley-Reisner\\ face rings; \\ simplicial complexes};
\draw (325.3700000000001, -14.4) rectangle (330.97000000000014,-17.5);
\draw(331.0700000000001, -14.4) node[anchor=north west,align=left] {Commutative rings\\ defined by \\ factorization \\ properties (e.g.,\\ atomic, factorial,\\ half-factorial)};
\draw (331.0700000000001, -14.4) rectangle (336.17000000000013,-17.5);
\draw(336.2700000000001, -14.4) node[anchor=north west,align=left] {Polynomial \\ rings and \\ ideals; rings \\ of integer-valued\\ polynomials};
\draw (336.2700000000001, -14.4) rectangle (341.1200000000001,-17.0);
\draw(341.22000000000014, -14.4) node[anchor=north west,align=left] {Principal\\ ideal\\ rings};
\draw (341.22000000000014, -14.4) rectangle (344.07000000000016,-16.0);
\draw(325.3700000000001, -17.6) node[anchor=north west,align=left] {Commutative\\ rings defined\\ by binomial\\ ideals, \\ toric rings, etc.};
\draw (325.3700000000001, -17.6) rectangle (330.22000000000014,-20.200000000000003);
\draw(330.3200000000001, -17.6) node[anchor=north west,align=left] {Other \\ commutative rings\\ defined \\ by combinatorial\\ properties};
\draw (330.3200000000001, -17.6) rectangle (335.17000000000013,-20.200000000000003);
\draw(335.2700000000001, -17.6) node[anchor=north west,align=left] {Dedekind, \\ Prüfer, Krull\\ and Mori rings\\ and their\\ generalizations};
\draw (335.2700000000001, -17.6) rectangle (339.6200000000001,-20.200000000000003);
\draw(339.72000000000014, -17.6) node[anchor=north west,align=left] {Euclidean\\ rings\\ and \\ generalizations};
\draw (339.72000000000014, -17.6) rectangle (344.07000000000016,-19.700000000000003);
\draw(325.3700000000001, -20.3) node[anchor=north west,align=left] {Rings with\\ straightening\\ laws, \\ Hodge algebras};
\draw (325.3700000000001, -20.3) rectangle (329.47000000000014,-22.400000000000002);
\draw(329.5700000000001, -20.3) node[anchor=north west,align=left] {Witt vectors\\ and \\ related rings};
\draw (329.5700000000001, -20.3) rectangle (333.42000000000013,-21.900000000000002);
\draw(333.5200000000001, -20.3) node[anchor=north west,align=left] {Formal \\ power \\ series rings};
\draw (333.5200000000001, -20.3) rectangle (337.1200000000001,-21.900000000000002);
\draw(337.22000000000014, -20.3) node[anchor=north west,align=left] {Seminormal\\ rings};
\draw (337.22000000000014, -20.3) rectangle (340.32000000000016,-21.400000000000002);
\draw(340.42000000000013, -20.3) node[anchor=north west,align=left] {Valuation\\ rings};
\draw (340.42000000000013, -20.3) rectangle (343.27000000000015,-21.400000000000002);
\draw(325.3700000000001, -22.5) node[anchor=north west,align=left] {Excellent\\ rings};
\draw (325.3700000000001, -22.5) rectangle (328.22000000000014,-23.6);
\draw(328.3200000000001, -22.5) node[anchor=north west,align=left] {Cluster\\ algebras};
\draw (328.3200000000001, -22.5) rectangle (330.92000000000013,-23.6);
\draw(344.2700000000001, -13.4) node[anchor=north west,align=left] {\large Commutative ring extensions and related topics};
\draw (344.2700000000001, -13.4) rectangle (361.8200000000001,-22.5);
\draw(345.2700000000001, -14.4) node[anchor=north west,align=left] {Integral \\ closure of \\ commutative rings\\ and ideals};
\draw (345.2700000000001, -14.4) rectangle (350.1200000000001,-16.5);
\draw(350.2200000000001, -14.4) node[anchor=north west,align=left] {Rings of \\ fractions and\\ localization\\ for \\ commutative rings};
\draw (350.2200000000001, -14.4) rectangle (355.0700000000001,-17.0);
\draw(355.1700000000001, -14.4) node[anchor=north west,align=left] {Galois theory\\ and \\ commutative ring\\ extensions};
\draw (355.1700000000001, -14.4) rectangle (359.7700000000001,-16.5);
\draw(345.2700000000001, -17.1) node[anchor=north west,align=left] {Étale and \\ flat extensions;\\ Henselization;\\ Artin\\ approximation};
\draw (345.2700000000001, -17.1) rectangle (349.8700000000001,-19.700000000000003);
\draw(349.9700000000001, -17.1) node[anchor=north west,align=left] {Extension\\ theory \\ of commutative\\ rings};
\draw (349.9700000000001, -17.1) rectangle (354.0700000000001,-19.200000000000003);
\draw(354.1700000000001, -17.1) node[anchor=north west,align=left] {Morphisms\\ of commutative\\ rings};
\draw (354.1700000000001, -17.1) rectangle (358.2700000000001,-18.700000000000003);
\draw(358.3700000000001, -17.1) node[anchor=north west,align=left] {Polynomials\\ over\\ commutative\\ rings};
\draw (358.3700000000001, -17.1) rectangle (361.72000000000014,-19.200000000000003);
\draw(345.2700000000001, -19.8) node[anchor=north west,align=left] {Integral \\ dependence in \\ commutative \\ rings; going\\ up, going down};
\draw (345.2700000000001, -19.8) rectangle (349.3700000000001,-22.400000000000002);
\draw(349.4700000000001, -19.8) node[anchor=north west,align=left] {Completion\\ of commutative\\ rings};
\draw (349.4700000000001, -19.8) rectangle (353.5700000000001,-21.400000000000002);
\draw(324.3700000000001, -23.8) node[anchor=north west,align=left] {\large Homological methods in commutative ring theory};
\draw (324.3700000000001, -23.8) rectangle (341.47000000000014,-36.1);
\draw(325.3700000000001, -24.8) node[anchor=north west,align=left] {Homological \\ conjectures \\ (intersection \\ theorems) in \\ commutative ring theory};
\draw (325.3700000000001, -24.8) rectangle (331.72000000000014,-27.400000000000002);
\draw(331.8200000000001, -24.8) node[anchor=north west,align=left] {(Co)homology of \\ commutative rings\\ and algebras (e.g.,\\ Hochschild, \\ André-Quillen, cyclic,\\ dihedral, etc.)};
\draw (331.8200000000001, -24.8) rectangle (337.92000000000013,-27.900000000000002);
\draw(338.0200000000001, -24.8) node[anchor=north west,align=left] {Torsion \\ theory for\\ commutative\\ rings};
\draw (338.0200000000001, -24.8) rectangle (341.3700000000001,-26.900000000000002);
\draw(325.3700000000001, -28.0) node[anchor=north west,align=left] {Syzygies,\\ resolutions,\\ complexes\\ and \\ commutative rings};
\draw (325.3700000000001, -28.0) rectangle (330.22000000000014,-30.6);
\draw(330.3200000000001, -28.0) node[anchor=north west,align=left] {Homological\\ dimension\\ and \\ commutative rings};
\draw (330.3200000000001, -28.0) rectangle (335.17000000000013,-30.1);
\draw(335.2700000000001, -28.0) node[anchor=north west,align=left] {Homological \\ functors on \\ modules of \\ commutative rings\\ (Tor, Ext, etc.)};
\draw (335.2700000000001, -28.0) rectangle (340.1200000000001,-30.6);
\draw(325.3700000000001, -30.700000000000003) node[anchor=north west,align=left] {Deformations\\ and infinitesimal\\ methods\\ in commutative\\ ring theory};
\draw (325.3700000000001, -30.700000000000003) rectangle (330.22000000000014,-33.300000000000004);
\draw(330.3200000000001, -30.700000000000003) node[anchor=north west,align=left] {Derived \\ categories \\ and commutative\\ rings};
\draw (330.3200000000001, -30.700000000000003) rectangle (334.67000000000013,-32.800000000000004);
\draw(334.7700000000001, -30.700000000000003) node[anchor=north west,align=left] {Grothendieck\\ groups, \\ \(K\)-theory\\ and commutative\\ rings};
\draw (334.7700000000001, -30.700000000000003) rectangle (339.1200000000001,-33.300000000000004);
\draw(325.3700000000001, -33.400000000000006) node[anchor=north west,align=left] {Hilbert-Samuel\\ and \\ Hilbert-Kunz \\ functions; \\ Poincaré series};
\draw (325.3700000000001, -33.400000000000006) rectangle (329.72000000000014,-36.00000000000001);
\draw(329.8200000000001, -33.400000000000006) node[anchor=north west,align=left] {Local \\ cohomology \\ and commutative\\ rings};
\draw (329.8200000000001, -33.400000000000006) rectangle (334.17000000000013,-35.50000000000001);
\draw(341.5700000000001, -23.8) node[anchor=north west,align=left] {\large Applications of logic to commutative algebra};
\draw (341.5700000000001, -23.8) rectangle (355.8100000000001,-27.0);
\draw(342.5700000000001, -24.8) node[anchor=north west,align=left] {Applications\\ of logic\\ to commutative\\ algebra};
\draw (342.5700000000001, -24.8) rectangle (346.67000000000013,-26.900000000000002);
\draw(341.5700000000001, -27.1) node[anchor=north west,align=left] {\large Finite commutative rings};
\draw (341.5700000000001, -27.1) rectangle (349.6100000000001,-30.3);
\draw(342.5700000000001, -28.1) node[anchor=north west,align=left] {Structure\\ of finite\\ commutative\\ rings};
\draw (342.5700000000001, -28.1) rectangle (345.92000000000013,-30.200000000000003);
\draw(346.0200000000001, -28.1) node[anchor=north west,align=left] {Polynomials\\ and finite\\ commutative\\ rings};
\draw (346.0200000000001, -28.1) rectangle (349.3700000000001,-30.200000000000003);
\draw(341.5700000000001, -30.400000000000002) node[anchor=north west,align=left] {\large History of \\ commutative algebra};
\draw (341.5700000000001, -30.400000000000002) rectangle (348.0600000000001,-31.500000000000004);
\draw(341.5700000000001, -31.6) node[anchor=north west,align=left] {\large Integral domains};
\draw (341.5700000000001, -31.6) rectangle (347.1300000000001,-33.800000000000004);
\draw(342.5700000000001, -32.6) node[anchor=north west,align=left] {Integral\\ domains};
\draw (342.5700000000001, -32.6) rectangle (345.17000000000013,-33.7);
\draw(355.91000000000014, -23.8) node[anchor=north west,align=left] {\large Differential algebra};
\draw (355.91000000000014, -23.8) rectangle (362.71000000000015,-31.4);
\draw(356.91000000000014, -24.8) node[anchor=north west,align=left] {Modules\\ of \\ differentials};
\draw (356.91000000000014, -24.8) rectangle (360.76000000000016,-26.400000000000002);
\draw(356.91000000000014, -26.5) node[anchor=north west,align=left] {Commutative\\ rings of \\ differential \\ operators and\\ their modules};
\draw (356.91000000000014, -26.5) rectangle (360.76000000000016,-29.1);
\draw(356.91000000000014, -29.200000000000003) node[anchor=north west,align=left] {Derivations\\ and\\ commutative\\ rings};
\draw (356.91000000000014, -29.200000000000003) rectangle (360.26000000000016,-31.300000000000004);
\draw(324.3700000000001, -36.2) node[anchor=north west,align=left] {\large General commutative ring theory};
\draw (324.3700000000001, -36.2) rectangle (337.0200000000001,-49.5);
\draw(325.3700000000001, -37.2) node[anchor=north west,align=left] {Characteristic \\ \(p\) methods \\ (Frobenius endomorphism)\\ and reduction\\ to characteristic\\ \(p\); tight closure};
\draw (325.3700000000001, -37.2) rectangle (331.97000000000014,-40.300000000000004);
\draw(332.0700000000001, -37.2) node[anchor=north west,align=left] {Ideals and\\ multiplicative\\ ideal \\ theory in \\ commutative rings};
\draw (332.0700000000001, -37.2) rectangle (336.92000000000013,-39.800000000000004);
\draw(325.3700000000001, -40.400000000000006) node[anchor=north west,align=left] {General commutative\\ ring theory and\\ combinatorics \\ (zero-divisor graphs,\\ annihilating-ideal\\ graphs, etc.)};
\draw (325.3700000000001, -40.400000000000006) rectangle (331.22000000000014,-43.50000000000001);
\draw(331.3200000000001, -40.400000000000006) node[anchor=north west,align=left] {Divisibility\\ and factorizations\\ in \\ commutative rings};
\draw (331.3200000000001, -40.400000000000006) rectangle (336.42000000000013,-42.50000000000001);
\draw(325.3700000000001, -43.6) node[anchor=north west,align=left] {Associated graded\\ rings of \\ ideals (Rees ring,\\ form ring), \\ analytic spread \\ and related topics};
\draw (325.3700000000001, -43.6) rectangle (330.47000000000014,-46.7);
\draw(330.5700000000001, -43.6) node[anchor=north west,align=left] {Valuations\\ and their \\ generalizations\\ for \\ commutative rings};
\draw (330.5700000000001, -43.6) rectangle (335.42000000000013,-46.2);
\draw(325.3700000000001, -46.800000000000004) node[anchor=north west,align=left] {Actions of\\ groups on \\ commutative\\ rings; \\ invariant theory};
\draw (325.3700000000001, -46.800000000000004) rectangle (329.97000000000014,-49.400000000000006);
\draw(330.0700000000001, -46.800000000000004) node[anchor=north west,align=left] {Graded\\ rings};
\draw (330.0700000000001, -46.800000000000004) rectangle (332.17000000000013,-47.900000000000006);
\draw(337.1200000000001, -36.2) node[anchor=north west,align=left] {\large Topological rings and modules};
\draw (337.1200000000001, -36.2) rectangle (348.17000000000013,-40.6);
\draw(338.1200000000001, -37.2) node[anchor=north west,align=left] {Global \\ topological\\ rings};
\draw (338.1200000000001, -37.2) rectangle (341.47000000000014,-38.800000000000004);
\draw(341.5700000000001, -37.2) node[anchor=north west,align=left] {Analytical\\ algebras\\ and rings};
\draw (341.5700000000001, -37.2) rectangle (344.67000000000013,-38.800000000000004);
\draw(344.7700000000001, -37.2) node[anchor=north west,align=left] {Complete\\ rings,\\ completion};
\draw (344.7700000000001, -37.2) rectangle (347.8700000000001,-38.800000000000004);
\draw(338.1200000000001, -38.900000000000006) node[anchor=north west,align=left] {Henselian\\ rings};
\draw (338.1200000000001, -38.900000000000006) rectangle (340.97000000000014,-40.00000000000001);
\draw(341.0700000000001, -38.900000000000006) node[anchor=north west,align=left] {Ordered\\ rings};
\draw (341.0700000000001, -38.900000000000006) rectangle (343.42000000000013,-40.00000000000001);
\draw(343.5200000000001, -38.900000000000006) node[anchor=north west,align=left] {Real \\ algebra};
\draw (343.5200000000001, -38.900000000000006) rectangle (345.8700000000001,-40.00000000000001);
\draw(345.97000000000014, -38.900000000000006) node[anchor=north west,align=left] {Power\\ series\\ rings};
\draw (345.97000000000014, -38.900000000000006) rectangle (348.07000000000016,-40.50000000000001);
\draw(337.1200000000001, -40.7) node[anchor=north west,align=left] {\large Local rings and semilocal rings};
\draw (337.1200000000001, -40.7) rectangle (347.3300000000001,-46.1);
\draw(338.1200000000001, -41.7) node[anchor=north west,align=left] {Special types\\ (Cohen-Macaulay, \\ Gorenstein, \\ Buchsbaum, etc.)};
\draw (338.1200000000001, -41.7) rectangle (342.97000000000014,-44.300000000000004);
\draw(343.0700000000001, -41.7) node[anchor=north west,align=left] {Multiplicity\\ theory\\ and \\ related topics};
\draw (343.0700000000001, -41.7) rectangle (347.17000000000013,-43.800000000000004);
\draw(338.1200000000001, -44.400000000000006) node[anchor=north west,align=left] {Regular\\ local\\ rings};
\draw (338.1200000000001, -44.400000000000006) rectangle (340.47000000000014,-46.00000000000001);
\draw(323.3700000000001, -49.7) node[anchor=north west,align=left] {\LARGE Real functions};
\draw (323.3700000000001, -49.7) rectangle (354.8200000000001,-88.0);
\draw(324.3700000000001, -50.7) node[anchor=north west,align=left] {\large Miscellaneous topics in real functions};
\draw (324.3700000000001, -50.7) rectangle (340.7700000000001,-60.5);
\draw(325.3700000000001, -51.7) node[anchor=north west,align=left] {\(C^\infty\)-functions,quasi-analytic\\ functions};
\draw (325.3700000000001, -51.7) rectangle (335.22000000000014,-53.800000000000004);
\draw(335.3200000000001, -51.7) node[anchor=north west,align=left] {Nonstandardanalysis};
\draw (335.3200000000001, -51.7) rectangle (340.67000000000013,-53.300000000000004);
\draw(325.3700000000001, -53.900000000000006) node[anchor=north west,align=left] {Non-Archimedeananalysis};
\draw (325.3700000000001, -53.900000000000006) rectangle (331.72000000000014,-55.50000000000001);
\draw(331.8200000000001, -53.900000000000006) node[anchor=north west,align=left] {Real-analyticfunctions};
\draw (331.8200000000001, -53.900000000000006) rectangle (337.92000000000013,-55.50000000000001);
\draw(338.0200000000001, -53.900000000000006) node[anchor=north west,align=left] {Fuzzy\\ real \\ analysis};
\draw (338.0200000000001, -53.900000000000006) rectangle (340.6200000000001,-55.50000000000001);
\draw(325.3700000000001, -55.6) node[anchor=north west,align=left] {Calculus of \\ functions on \\ infinite-dimensional\\ spaces};
\draw (325.3700000000001, -55.6) rectangle (330.97000000000014,-57.7);
\draw(331.0700000000001, -55.6) node[anchor=north west,align=left] {Calculus of \\ functions taking\\ values in\\ infinite-dimensional\\ spaces};
\draw (331.0700000000001, -55.6) rectangle (336.67000000000013,-58.2);
\draw(336.7700000000001, -55.6) node[anchor=north west,align=left] {Constructive\\ real\\ analysis};
\draw (336.7700000000001, -55.6) rectangle (340.3700000000001,-57.2);
\draw(325.3700000000001, -58.300000000000004) node[anchor=north west,align=left] {Real analysis\\ on time\\ scales or \\ measure chains};
\draw (325.3700000000001, -58.300000000000004) rectangle (329.47000000000014,-60.400000000000006);
\draw(329.5700000000001, -58.300000000000004) node[anchor=north west,align=left] {Set-valued\\ functions};
\draw (329.5700000000001, -58.300000000000004) rectangle (332.67000000000013,-59.900000000000006);
\draw(332.7700000000001, -58.300000000000004) node[anchor=north west,align=left] {Means};
\draw (332.7700000000001, -58.300000000000004) rectangle (334.6200000000001,-58.900000000000006);
\draw(340.8700000000001, -50.7) node[anchor=north west,align=left] {\large Functions of one variable};
\draw (340.8700000000001, -50.7) rectangle (354.72000000000014,-73.30000000000001);
\draw(341.8700000000001, -51.7) node[anchor=north west,align=left] {Continuity and \\ related questions \\ (modulus of continuity,\\ semicontinuity,\\ discontinuities,\\ etc.) for real \\ functions in one variable};
\draw (341.8700000000001, -51.7) rectangle (348.72000000000014,-55.300000000000004);
\draw(348.8200000000001, -51.7) node[anchor=north west,align=left] {Foundations: \\ limits and \\ generalizations, \\ elementary \\ topology of the line};
\draw (348.8200000000001, -51.7) rectangle (354.42000000000013,-54.300000000000004);
\draw(341.8700000000001, -55.400000000000006) node[anchor=north west,align=left] {Nondifferentiability\\ (nondifferentiable\\ functions, \\ points of \\ nondifferentiability), \\ discontinuous derivatives};
\draw (341.8700000000001, -55.400000000000006) rectangle (348.72000000000014,-58.50000000000001);
\draw(348.8200000000001, -55.400000000000006) node[anchor=north west,align=left] {Singular functions,\\ Cantor \\ functions, functions\\ with other \\ special properties};
\draw (348.8200000000001, -55.400000000000006) rectangle (354.42000000000013,-58.00000000000001);
\draw(341.8700000000001, -58.6) node[anchor=north west,align=left] {Differentiation \\ (real functions of \\ one variable): general\\ theory, generalized\\ derivatives,\\ mean value theorems};
\draw (341.8700000000001, -58.6) rectangle (347.97000000000014,-61.7);
\draw(348.0700000000001, -58.6) node[anchor=north west,align=left] {Antidifferentiation};
\draw (348.0700000000001, -58.6) rectangle (353.42000000000013,-60.2);
\draw(341.8700000000001, -61.800000000000004) node[anchor=north west,align=left] {Classification\\ of real functions;\\ Baire \\ classification of \\ sets and functions};
\draw (341.8700000000001, -61.800000000000004) rectangle (346.97000000000014,-64.4);
\draw(347.0700000000001, -61.800000000000004) node[anchor=north west,align=left] {Rate of growth\\ of functions,\\ orders of \\ infinity, slowly\\ varying functions};
\draw (347.0700000000001, -61.800000000000004) rectangle (351.92000000000013,-64.4);
\draw(341.8700000000001, -64.5) node[anchor=north west,align=left] {Denjoy and \\ Perron integrals,\\ other special\\ integrals};
\draw (341.8700000000001, -64.5) rectangle (346.72000000000014,-66.6);
\draw(346.8200000000001, -64.5) node[anchor=north west,align=left] {Functions \\ of bounded \\ variation, \\ generalizations};
\draw (346.8200000000001, -64.5) rectangle (351.17000000000013,-66.6);
\draw(351.2700000000001, -64.5) node[anchor=north west,align=left] {Fractional\\ derivatives\\ and\\ integrals};
\draw (351.2700000000001, -64.5) rectangle (354.6200000000001,-66.6);
\draw(341.8700000000001, -66.7) node[anchor=north west,align=left] {Absolutely \\ continuous real\\ functions in\\ one variable};
\draw (341.8700000000001, -66.7) rectangle (346.22000000000014,-68.8);
\draw(346.3200000000001, -66.7) node[anchor=north west,align=left] {Monotonic\\ functions,\\ generalizations};
\draw (346.3200000000001, -66.7) rectangle (350.67000000000013,-68.8);
\draw(350.7700000000001, -66.7) node[anchor=north west,align=left] {Integrals of\\ Riemann, \\ Stieltjes and\\ Lebesgue type};
\draw (350.7700000000001, -66.7) rectangle (354.6200000000001,-68.8);
\draw(341.8700000000001, -68.9) node[anchor=north west,align=left] {Convexity of\\ real functions\\ in one \\ variable, \\ generalizations};
\draw (341.8700000000001, -68.9) rectangle (346.22000000000014,-71.5);
\draw(346.3200000000001, -68.9) node[anchor=north west,align=left] {One-variable\\ calculus};
\draw (346.3200000000001, -68.9) rectangle (349.92000000000013,-70.5);
\draw(350.0200000000001, -68.9) node[anchor=north west,align=left] {Iteration\\ of real \\ functions in\\ one variable};
\draw (350.0200000000001, -68.9) rectangle (353.6200000000001,-71.0);
\draw(341.8700000000001, -71.6) node[anchor=north west,align=left] {Elementary\\ functions};
\draw (341.8700000000001, -71.6) rectangle (344.97000000000014,-73.19999999999999);
\draw(345.0700000000001, -71.6) node[anchor=north west,align=left] {Lipschitz\\ (Hölder)\\ classes};
\draw (345.0700000000001, -71.6) rectangle (347.92000000000013,-73.19999999999999);
\draw(324.3700000000001, -60.6) node[anchor=north west,align=left] {\large Polynomials, rational functions in real analysis};
\draw (324.3700000000001, -60.6) rectangle (339.85000000000014,-63.800000000000004);
\draw(325.3700000000001, -61.6) node[anchor=north west,align=left] {Real \\ polynomials: \\ analytic \\ properties, etc.};
\draw (325.3700000000001, -61.6) rectangle (329.97000000000014,-63.7);
\draw(330.0700000000001, -61.6) node[anchor=north west,align=left] {Real \\ polynomials:\\ location\\ of zeros};
\draw (330.0700000000001, -61.6) rectangle (333.67000000000013,-63.7);
\draw(333.7700000000001, -61.6) node[anchor=north west,align=left] {Real \\ rational\\ functions};
\draw (333.7700000000001, -61.6) rectangle (336.6200000000001,-63.2);
\draw(324.3700000000001, -63.900000000000006) node[anchor=north west,align=left] {\large Inequalities in real analysis};
\draw (324.3700000000001, -63.900000000000006) rectangle (334.7700000000001,-71.5);
\draw(325.3700000000001, -64.9) node[anchor=north west,align=left] {Inequalities \\ involving \\ derivatives and \\ differential and\\ integral operators};
\draw (325.3700000000001, -64.9) rectangle (330.47000000000014,-67.5);
\draw(330.5700000000001, -64.9) node[anchor=north west,align=left] {Inequalities\\ for \\ trigonometric \\ functions and\\ polynomials};
\draw (330.5700000000001, -64.9) rectangle (334.67000000000013,-67.5);
\draw(325.3700000000001, -67.60000000000001) node[anchor=north west,align=left] {Inequalities\\ for sums,\\ series\\ and integrals};
\draw (325.3700000000001, -67.60000000000001) rectangle (329.22000000000014,-69.7);
\draw(329.3200000000001, -67.60000000000001) node[anchor=north west,align=left] {Inequalities\\ involving\\ other types\\ of functions};
\draw (329.3200000000001, -67.60000000000001) rectangle (332.92000000000013,-69.7);
\draw(325.3700000000001, -69.80000000000001) node[anchor=north west,align=left] {Other \\ analytical \\ inequalities};
\draw (325.3700000000001, -69.80000000000001) rectangle (328.97000000000014,-71.4);
\draw(324.3700000000001, -71.60000000000001) node[anchor=north west,align=left] {\large Computational methods\\ for problems pertaining\\ to real functions};
\draw (324.3700000000001, -71.60000000000001) rectangle (332.10000000000014,-73.2);
\draw(324.3700000000001, -73.4) node[anchor=north west,align=left] {\large Functions of several variables};
\draw (324.3700000000001, -73.4) rectangle (336.0200000000001,-87.9);
\draw(325.3700000000001, -74.4) node[anchor=north west,align=left] {Implicit function\\ theorems, \\ Jacobians, \\ transformations with\\ several variables};
\draw (325.3700000000001, -74.4) rectangle (330.97000000000014,-77.0);
\draw(331.0700000000001, -74.4) node[anchor=north west,align=left] {Integral formulas\\ of real \\ functions of \\ several variables\\ (Stokes, Gauss,\\ Green, etc.)};
\draw (331.0700000000001, -74.4) rectangle (335.92000000000013,-77.5);
\draw(325.3700000000001, -77.60000000000001) node[anchor=north west,align=left] {Absolutely \\ continuous real \\ functions of \\ several variables,\\ functions \\ of bounded variation};
\draw (325.3700000000001, -77.60000000000001) rectangle (330.97000000000014,-80.7);
\draw(331.0700000000001, -77.60000000000001) node[anchor=north west,align=left] {Continuity\\ and \\ differentiation\\ questions};
\draw (331.0700000000001, -77.60000000000001) rectangle (335.42000000000013,-79.7);
\draw(325.3700000000001, -80.80000000000001) node[anchor=north west,align=left] {Integration of\\ real functions\\ of several \\ variables: length,\\ area, volume};
\draw (325.3700000000001, -80.80000000000001) rectangle (330.47000000000014,-83.4);
\draw(330.5700000000001, -80.80000000000001) node[anchor=north west,align=left] {Special properties\\ of functions\\ of several \\ variables, Hölder\\ conditions, etc.};
\draw (330.5700000000001, -80.80000000000001) rectangle (335.67000000000013,-83.4);
\draw(325.3700000000001, -83.5) node[anchor=north west,align=left] {Convexity of\\ real functions\\ of several\\ variables, \\ generalizations};
\draw (325.3700000000001, -83.5) rectangle (329.72000000000014,-86.1);
\draw(329.8200000000001, -83.5) node[anchor=north west,align=left] {Representation\\ and \\ superposition\\ of functions};
\draw (329.8200000000001, -83.5) rectangle (333.92000000000013,-85.6);
\draw(325.3700000000001, -86.2) node[anchor=north west,align=left] {Calculus\\ of vector\\ functions};
\draw (325.3700000000001, -86.2) rectangle (328.22000000000014,-87.8);
\draw(336.1200000000001, -73.4) node[anchor=north west,align=left] {\large History of \\ real functions};
\draw (336.1200000000001, -73.4) rectangle (341.0600000000001,-74.5);
\draw(323.3700000000001, -88.10000000000001) node[anchor=north west,align=left] {\LARGE Nonassociative rings and algebras};
\draw (323.3700000000001, -88.10000000000001) rectangle (354.72000000000014,-142.20000000000002);
\draw(324.3700000000001, -89.10000000000001) node[anchor=north west,align=left] {\large Jordan algebras (algebras, triples and pairs)};
\draw (324.3700000000001, -89.10000000000001) rectangle (341.42000000000013,-98.9);
\draw(325.3700000000001, -90.10000000000001) node[anchor=north west,align=left] {Jordan \\ structures associated\\ with \\ other structures};
\draw (325.3700000000001, -90.10000000000001) rectangle (331.22000000000014,-92.2);
\draw(331.3200000000001, -90.10000000000001) node[anchor=north west,align=left] {Finite-dimensional\\ structures of \\ Jordan algebras};
\draw (331.3200000000001, -90.10000000000001) rectangle (336.42000000000013,-92.2);
\draw(336.5200000000001, -90.10000000000001) node[anchor=north west,align=left] {Associated \\ groups, \\ automorphisms of\\ Jordan algebras};
\draw (336.5200000000001, -90.10000000000001) rectangle (341.1200000000001,-92.2);
\draw(325.3700000000001, -92.30000000000001) node[anchor=north west,align=left] {Associated\\ manifolds\\ of \\ Jordan algebras};
\draw (325.3700000000001, -92.30000000000001) rectangle (329.72000000000014,-94.4);
\draw(329.8200000000001, -92.30000000000001) node[anchor=north west,align=left] {Idempotents,\\ Peirce \\ decompositions};
\draw (329.8200000000001, -92.30000000000001) rectangle (333.92000000000013,-94.4);
\draw(334.0200000000001, -92.30000000000001) node[anchor=north west,align=left] {Jordan \\ structures on \\ Banach spaces\\ and algebras};
\draw (334.0200000000001, -92.30000000000001) rectangle (338.1200000000001,-94.4);
\draw(338.22000000000014, -92.30000000000001) node[anchor=north west,align=left] {Identities\\ and free\\ Jordan\\ structures};
\draw (338.22000000000014, -92.30000000000001) rectangle (341.32000000000016,-94.4);
\draw(325.3700000000001, -94.50000000000001) node[anchor=north west,align=left] {Applications\\ of Jordan \\ algebras to \\ physics, etc.};
\draw (325.3700000000001, -94.50000000000001) rectangle (329.22000000000014,-96.60000000000001);
\draw(329.3200000000001, -94.50000000000001) node[anchor=north west,align=left] {Exceptional\\ Jordan\\ structures};
\draw (329.3200000000001, -94.50000000000001) rectangle (332.67000000000013,-96.10000000000001);
\draw(332.7700000000001, -94.50000000000001) node[anchor=north west,align=left] {Structure\\ theory \\ for Jordan\\ algebras};
\draw (332.7700000000001, -94.50000000000001) rectangle (335.8700000000001,-96.60000000000001);
\draw(335.97000000000014, -94.50000000000001) node[anchor=north west,align=left] {Simple,\\ semisimple\\ Jordan\\ algebras};
\draw (335.97000000000014, -94.50000000000001) rectangle (339.07000000000016,-96.60000000000001);
\draw(325.3700000000001, -96.7) node[anchor=north west,align=left] {Associated\\ geometries\\ of Jordan\\ algebras};
\draw (325.3700000000001, -96.7) rectangle (328.47000000000014,-98.8);
\draw(328.5700000000001, -96.7) node[anchor=north west,align=left] {Division\\ algebras\\ and Jordan\\ algebras};
\draw (328.5700000000001, -96.7) rectangle (331.67000000000013,-98.8);
\draw(331.7700000000001, -96.7) node[anchor=north west,align=left] {Super \\ structures};
\draw (331.7700000000001, -96.7) rectangle (334.8700000000001,-97.8);
\draw(334.97000000000014, -96.7) node[anchor=north west,align=left] {Radicals\\ in Jordan\\ algebras};
\draw (334.97000000000014, -96.7) rectangle (337.82000000000016,-98.3);
\draw(341.5200000000001, -89.10000000000001) node[anchor=north west,align=left] {\large Lie algebras and Lie superalgebras};
\draw (341.5200000000001, -89.10000000000001) rectangle (354.6200000000001,-122.2);
\draw(342.5200000000001, -90.10000000000001) node[anchor=north west,align=left] {Lie (super)algebras\\ associated \\ with other structures\\ (associative,\\ Jordan, etc.)};
\draw (342.5200000000001, -90.10000000000001) rectangle (348.3700000000001,-92.7);
\draw(348.4700000000001, -90.10000000000001) node[anchor=north west,align=left] {Kac-Moody \\ (super)algebras; \\ extended affine Lie\\ algebras; \\ toroidal Lie algebras};
\draw (348.4700000000001, -90.10000000000001) rectangle (354.3200000000001,-92.7);
\draw(342.5200000000001, -92.80000000000001) node[anchor=north west,align=left] {Quantum groups\\ (quantized \\ enveloping algebras)\\ and related\\ deformations};
\draw (342.5200000000001, -92.80000000000001) rectangle (348.1200000000001,-95.4);
\draw(348.2200000000001, -92.80000000000001) node[anchor=north west,align=left] {Infinite-dimensional\\ Lie \\ (super)algebras};
\draw (348.2200000000001, -92.80000000000001) rectangle (353.8200000000001,-94.9);
\draw(342.5200000000001, -95.50000000000001) node[anchor=north west,align=left] {Automorphisms,\\ derivations, \\ other operators \\ for Lie algebras\\ and super algebras};
\draw (342.5200000000001, -95.50000000000001) rectangle (347.6200000000001,-98.10000000000001);
\draw(347.7200000000001, -95.50000000000001) node[anchor=north west,align=left] {Applications\\ of Lie algebras\\ and \\ superalgebras to \\ integrable systems};
\draw (347.7200000000001, -95.50000000000001) rectangle (352.8200000000001,-98.10000000000001);
\draw(342.5200000000001, -98.20000000000002) node[anchor=north west,align=left] {Applications\\ of Lie \\ (super)algebras to\\ physics, etc.};
\draw (342.5200000000001, -98.20000000000002) rectangle (347.6200000000001,-100.30000000000001);
\draw(347.7200000000001, -98.20000000000002) node[anchor=north west,align=left] {Vertex operators;\\ vertex \\ operator algebras\\ and related\\ structures};
\draw (347.7200000000001, -98.20000000000002) rectangle (352.5700000000001,-100.80000000000001);
\draw(342.5200000000001, -100.9) node[anchor=north west,align=left] {Representations\\ of Lie \\ algebras and Lie\\ superalgebras,\\ algebraic\\ theory (weights)};
\draw (342.5200000000001, -100.9) rectangle (347.1200000000001,-104.0);
\draw(347.2200000000001, -100.9) node[anchor=north west,align=left] {Lie algebras\\ of \\ linear \\ algebraic groups};
\draw (347.2200000000001, -100.9) rectangle (351.8200000000001,-103.0);
\draw(351.9200000000001, -100.9) node[anchor=north west,align=left] {Poisson\\ algebras};
\draw (351.9200000000001, -100.9) rectangle (354.5200000000001,-102.0);
\draw(342.5200000000001, -104.10000000000001) node[anchor=north west,align=left] {Lie algebras\\ of vector\\ fields and\\ related \\ (super) algebras};
\draw (342.5200000000001, -104.10000000000001) rectangle (347.1200000000001,-106.7);
\draw(347.2200000000001, -104.10000000000001) node[anchor=north west,align=left] {Identities,\\ free \\ Lie \\ (super)algebras};
\draw (347.2200000000001, -104.10000000000001) rectangle (351.5700000000001,-106.2);
\draw(351.6700000000001, -104.10000000000001) node[anchor=north west,align=left] {Coadjoint\\ orbits;\\ nilpotent\\ varieties};
\draw (351.6700000000001, -104.10000000000001) rectangle (354.5200000000001,-106.2);
\draw(342.5200000000001, -106.80000000000001) node[anchor=north west,align=left] {Representations\\ of Lie algebras\\ and Lie \\ superalgebras, \\ analytic theory};
\draw (342.5200000000001, -106.80000000000001) rectangle (346.8700000000001,-109.4);
\draw(346.9700000000001, -106.80000000000001) node[anchor=north west,align=left] {Simple, \\ semisimple, \\ reductive \\ (super)algebras};
\draw (346.9700000000001, -106.80000000000001) rectangle (351.3200000000001,-108.9);
\draw(351.4200000000001, -106.80000000000001) node[anchor=north west,align=left] {Root \\ systems};
\draw (351.4200000000001, -106.80000000000001) rectangle (353.7700000000001,-107.9);
\draw(342.5200000000001, -109.5) node[anchor=north west,align=left] {Exceptional\\ (super)algebras};
\draw (342.5200000000001, -109.5) rectangle (346.8700000000001,-111.1);
\draw(346.9700000000001, -109.5) node[anchor=north west,align=left] {Solvable,\\ nilpotent\\ (super)algebras};
\draw (346.9700000000001, -109.5) rectangle (351.3200000000001,-111.6);
\draw(342.5200000000001, -111.7) node[anchor=north west,align=left] {Universal\\ enveloping\\ (super)algebras};
\draw (342.5200000000001, -111.7) rectangle (346.8700000000001,-113.8);
\draw(346.9700000000001, -111.7) node[anchor=north west,align=left] {Yang-Baxter\\ equations\\ and Rota-Baxter\\ operators};
\draw (346.9700000000001, -111.7) rectangle (351.3200000000001,-113.8);
\draw(342.5200000000001, -113.9) node[anchor=north west,align=left] {Modular \\ Lie \\ (super)algebras};
\draw (342.5200000000001, -113.9) rectangle (346.8700000000001,-115.5);
\draw(346.9700000000001, -113.9) node[anchor=north west,align=left] {Homological\\ methods\\ in Lie \\ (super)algebras};
\draw (346.9700000000001, -113.9) rectangle (351.3200000000001,-116.0);
\draw(342.5200000000001, -116.10000000000001) node[anchor=north west,align=left] {Cohomology\\ of Lie\\ (super)algebras};
\draw (342.5200000000001, -116.10000000000001) rectangle (346.8700000000001,-118.2);
\draw(346.9700000000001, -116.10000000000001) node[anchor=north west,align=left] {Lie bialgebras;\\ Lie\\ coalgebras};
\draw (346.9700000000001, -116.10000000000001) rectangle (351.3200000000001,-117.7);
\draw(342.5200000000001, -118.30000000000001) node[anchor=north west,align=left] {Graded \\ Lie \\ (super)algebras};
\draw (342.5200000000001, -118.30000000000001) rectangle (346.8700000000001,-119.9);
\draw(346.9700000000001, -118.30000000000001) node[anchor=north west,align=left] {Color Lie\\ (super)algebras};
\draw (346.9700000000001, -118.30000000000001) rectangle (351.3200000000001,-119.9);
\draw(342.5200000000001, -120.0) node[anchor=north west,align=left] {Structure \\ theory for Lie\\ algebras and\\ superalgebras};
\draw (342.5200000000001, -120.0) rectangle (346.6200000000001,-122.1);
\draw(346.7200000000001, -120.0) node[anchor=north west,align=left] {Hom-Lie \\ and related\\ algebras};
\draw (346.7200000000001, -120.0) rectangle (350.0700000000001,-121.6);
\draw(350.1700000000001, -120.0) node[anchor=north west,align=left] {Virasoro\\ and related\\ algebras};
\draw (350.1700000000001, -120.0) rectangle (353.5200000000001,-121.6);
\draw(324.3700000000001, -99.00000000000001) node[anchor=north west,align=left] {\large Other nonassociative rings and algebras};
\draw (324.3700000000001, -99.00000000000001) rectangle (338.42000000000013,-106.10000000000001);
\draw(325.3700000000001, -100.00000000000001) node[anchor=north west,align=left] {\((\gamma,~\delta)\)-rings,\\ including \\ \((1,-1)\)-rings};
\draw (325.3700000000001, -100.00000000000001) rectangle (332.72000000000014,-102.60000000000001);
\draw(332.8200000000001, -100.00000000000001) node[anchor=north west,align=left] {Lie-admissible\\ algebras};
\draw (332.8200000000001, -100.00000000000001) rectangle (336.92000000000013,-101.60000000000001);
\draw(325.3700000000001, -102.70000000000002) node[anchor=north west,align=left] {Alternative\\ rings};
\draw (325.3700000000001, -102.70000000000002) rectangle (328.72000000000014,-104.30000000000001);
\draw(328.8200000000001, -102.70000000000002) node[anchor=north west,align=left] {Right \\ alternative\\ rings};
\draw (328.8200000000001, -102.70000000000002) rectangle (332.17000000000013,-104.30000000000001);
\draw(332.2700000000001, -102.70000000000002) node[anchor=north west,align=left] {(non-Lie)\\ Hom \\ algebras \\ and topics};
\draw (332.2700000000001, -102.70000000000002) rectangle (335.3700000000001,-104.80000000000001);
\draw(335.47000000000014, -102.70000000000002) node[anchor=north west,align=left] {Mal’tsev\\ rings and\\ algebras};
\draw (335.47000000000014, -102.70000000000002) rectangle (338.32000000000016,-104.30000000000001);
\draw(325.3700000000001, -104.90000000000002) node[anchor=north west,align=left] {Genetic\\ algebras};
\draw (325.3700000000001, -104.90000000000002) rectangle (327.97000000000014,-106.00000000000001);
\draw(324.3700000000001, -106.20000000000002) node[anchor=north west,align=left] {\large Computational methods\\ for problems \\ pertaining to nonassociative\\ rings and algebras};
\draw (324.3700000000001, -106.20000000000002) rectangle (333.6500000000001,-108.30000000000001);
\draw(324.3700000000001, -108.4) node[anchor=north west,align=left] {\large History of \\ nonassociative \\ rings and algebras};
\draw (324.3700000000001, -108.4) rectangle (330.5500000000001,-110.0);
\draw(324.3700000000001, -122.30000000000001) node[anchor=north west,align=left] {\large General nonassociative rings};
\draw (324.3700000000001, -122.30000000000001) rectangle (335.5200000000001,-142.10000000000002);
\draw(325.3700000000001, -123.30000000000001) node[anchor=north west,align=left] {General theory\\ of \\ nonassociative rings\\ and algebras};
\draw (325.3700000000001, -123.30000000000001) rectangle (330.97000000000014,-125.4);
\draw(331.0700000000001, -123.30000000000001) node[anchor=north west,align=left] {Radical theory\\ (nonassociative\\ rings\\ and algebras)};
\draw (331.0700000000001, -123.30000000000001) rectangle (335.42000000000013,-125.4);
\draw(325.3700000000001, -125.50000000000001) node[anchor=north west,align=left] {Automorphisms,\\ derivations, \\ other operators \\ (nonassociative\\ rings and algebras)};
\draw (325.3700000000001, -125.50000000000001) rectangle (330.72000000000014,-128.10000000000002);
\draw(330.8200000000001, -125.50000000000001) node[anchor=north west,align=left] {Nonassociative\\ algebras\\ satisfying \\ other identities};
\draw (330.8200000000001, -125.50000000000001) rectangle (335.42000000000013,-127.60000000000001);
\draw(325.3700000000001, -128.20000000000002) node[anchor=north west,align=left] {Compositionalgebras};
\draw (325.3700000000001, -128.20000000000002) rectangle (330.72000000000014,-129.8);
\draw(330.8200000000001, -128.20000000000002) node[anchor=north west,align=left] {Quadratic\\ algebras \\ (but not \\ quadratic \\ Jordan algebras)};
\draw (330.8200000000001, -128.20000000000002) rectangle (335.42000000000013,-130.8);
\draw(325.3700000000001, -130.9) node[anchor=north west,align=left] {Power-associative\\ rings};
\draw (325.3700000000001, -130.9) rectangle (330.22000000000014,-132.5);
\draw(330.3200000000001, -130.9) node[anchor=north west,align=left] {Gröbner-Shirshov\\ bases \\ in nonassociative\\ algebras};
\draw (330.3200000000001, -130.9) rectangle (335.17000000000013,-133.0);
\draw(325.3700000000001, -133.10000000000002) node[anchor=north west,align=left] {Noncommutative\\ Jordan\\ algebras};
\draw (325.3700000000001, -133.10000000000002) rectangle (329.47000000000014,-134.70000000000002);
\draw(329.5700000000001, -133.10000000000002) node[anchor=north west,align=left] {Nonassociative\\ division\\ algebras};
\draw (329.5700000000001, -133.10000000000002) rectangle (333.67000000000013,-134.70000000000002);
\draw(325.3700000000001, -134.8) node[anchor=north west,align=left] {Free \\ nonassociative\\ algebras};
\draw (325.3700000000001, -134.8) rectangle (329.47000000000014,-136.4);
\draw(329.5700000000001, -134.8) node[anchor=north west,align=left] {Structure \\ theory for \\ nonassociative\\ algebras};
\draw (329.5700000000001, -134.8) rectangle (333.67000000000013,-136.9);
\draw(325.3700000000001, -137.0) node[anchor=north west,align=left] {Other \\ \(n\)-ary \\ compositions \\ \((n~\ge~3)\)};
\draw (325.3700000000001, -137.0) rectangle (329.22000000000014,-139.1);
\draw(329.3200000000001, -137.0) node[anchor=north west,align=left] {Superalgebras};
\draw (329.3200000000001, -137.0) rectangle (333.17000000000013,-138.1);
\draw(325.3700000000001, -139.20000000000002) node[anchor=north west,align=left] {Ternary\\ compositions};
\draw (325.3700000000001, -139.20000000000002) rectangle (328.97000000000014,-140.8);
\draw(329.0700000000001, -139.20000000000002) node[anchor=north west,align=left] {Flexible\\ algebras};
\draw (329.0700000000001, -139.20000000000002) rectangle (331.67000000000013,-140.8);
\draw(331.7700000000001, -139.20000000000002) node[anchor=north west,align=left] {Leibniz\\ algebras};
\draw (331.7700000000001, -139.20000000000002) rectangle (334.3700000000001,-140.3);
\draw(325.3700000000001, -140.9) node[anchor=north west,align=left] {Valued\\ algebras};
\draw (325.3700000000001, -140.9) rectangle (327.97000000000014,-142.0);
\draw(363.78000000000014, -1) node[anchor=north west,align=left] {\LARGE Topological groups, Lie groups};
\draw (363.78000000000014, -1) rectangle (394.76000000000016,-37.6);
\draw(364.78000000000014, -2) node[anchor=north west,align=left] {\large Topological and differentiable algebraic systems};
\draw (364.78000000000014, -2) rectangle (380.63000000000017,-10.6);
\draw(365.78000000000014, -3) node[anchor=north west,align=left] {Topological \\ groupoids \\ (including \\ differentiable and\\ Lie groupoids)};
\draw (365.78000000000014, -3) rectangle (370.88000000000017,-5.6);
\draw(370.98000000000013, -3) node[anchor=north west,align=left] {Other topological\\ algebraic\\ systems\\ and their \\ representations};
\draw (370.98000000000013, -3) rectangle (375.83000000000015,-5.6);
\draw(375.9300000000001, -3) node[anchor=north west,align=left] {Topological\\ semilattices,\\ lattices \\ and applications};
\draw (375.9300000000001, -3) rectangle (380.53000000000014,-5.1);
\draw(365.78000000000014, -5.7) node[anchor=north west,align=left] {Representations\\ of general\\ topological\\ groups and\\ semigroups};
\draw (365.78000000000014, -5.7) rectangle (370.13000000000017,-8.3);
\draw(370.23000000000013, -5.7) node[anchor=north west,align=left] {Structure\\ of general\\ topological\\ groups};
\draw (370.23000000000013, -5.7) rectangle (373.58000000000015,-7.800000000000001);
\draw(373.6800000000001, -5.7) node[anchor=north west,align=left] {Analysis\\ on general\\ topological\\ groups};
\draw (373.6800000000001, -5.7) rectangle (377.03000000000014,-7.800000000000001);
\draw(377.13000000000017, -5.7) node[anchor=north west,align=left] {Structure\\ of \\ topological\\ semigroups};
\draw (377.13000000000017, -5.7) rectangle (380.4800000000002,-7.800000000000001);
\draw(365.78000000000014, -8.4) node[anchor=north west,align=left] {Analysis\\ on \\ topological\\ semigroups};
\draw (365.78000000000014, -8.4) rectangle (369.13000000000017,-10.5);
\draw(380.73000000000013, -2) node[anchor=north west,align=left] {\large Locally compact abelian groups (LCA groups)};
\draw (380.73000000000013, -2) rectangle (394.66000000000014,-5.2);
\draw(381.73000000000013, -3) node[anchor=north west,align=left] {General \\ properties and\\ structure\\ of LCA groups};
\draw (381.73000000000013, -3) rectangle (385.83000000000015,-5.1);
\draw(385.9300000000001, -3) node[anchor=north west,align=left] {Structure\\ of group \\ algebras of\\ LCA groups};
\draw (385.9300000000001, -3) rectangle (389.28000000000014,-5.1);
\draw(364.78000000000014, -10.7) node[anchor=north west,align=left] {\large Locally compact groups and their algebras};
\draw (364.78000000000014, -10.7) rectangle (379.63000000000017,-22.0);
\draw(365.78000000000014, -11.7) node[anchor=north west,align=left] {\(C^*\)-algebras\\ and \\ \(W^*\)-algebras in \\ relation to group\\ representations};
\draw (365.78000000000014, -11.7) rectangle (371.38000000000017,-14.299999999999999);
\draw(371.48000000000013, -11.7) node[anchor=north west,align=left] {Representationsof\\ group\\ algebras};
\draw (371.48000000000013, -11.7) rectangle (376.33000000000015,-13.799999999999999);
\draw(376.4300000000001, -11.7) node[anchor=north west,align=left] {Rigidity\\ in locally\\ compact\\ groups};
\draw (376.4300000000001, -11.7) rectangle (379.53000000000014,-13.799999999999999);
\draw(365.78000000000014, -14.399999999999999) node[anchor=north west,align=left] {Kazhdan’s \\ property (T), the\\ Haagerup \\ property, and \\ generalizations};
\draw (365.78000000000014, -14.399999999999999) rectangle (370.63000000000017,-17.0);
\draw(370.73000000000013, -14.399999999999999) node[anchor=north west,align=left] {Unitary \\ representations\\ of locally \\ compact groups};
\draw (370.73000000000013, -14.399999999999999) rectangle (375.08000000000015,-16.5);
\draw(375.1800000000001, -14.399999999999999) node[anchor=north west,align=left] {Other \\ representations\\ of locally\\ compact groups};
\draw (375.1800000000001, -14.399999999999999) rectangle (379.53000000000014,-16.5);
\draw(365.78000000000014, -17.1) node[anchor=north west,align=left] {Group \\ algebras of \\ locally compact\\ groups};
\draw (365.78000000000014, -17.1) rectangle (370.13000000000017,-19.200000000000003);
\draw(370.23000000000013, -17.1) node[anchor=north west,align=left] {Induced \\ representations\\ for locally\\ compact groups};
\draw (370.23000000000013, -17.1) rectangle (374.58000000000015,-19.200000000000003);
\draw(374.6800000000001, -17.1) node[anchor=north west,align=left] {General \\ properties and\\ structure\\ of locally\\ compact groups};
\draw (374.6800000000001, -17.1) rectangle (378.78000000000014,-19.700000000000003);
\draw(365.78000000000014, -19.8) node[anchor=north west,align=left] {Duality \\ theorems for\\ locally \\ compact groups};
\draw (365.78000000000014, -19.8) rectangle (369.88000000000017,-21.900000000000002);
\draw(369.98000000000013, -19.8) node[anchor=north west,align=left] {Automorphism\\ groups of\\ locally \\ compact groups};
\draw (369.98000000000013, -19.8) rectangle (374.08000000000015,-21.900000000000002);
\draw(374.1800000000001, -19.8) node[anchor=north west,align=left] {Ergodic\\ theory \\ on groups};
\draw (374.1800000000001, -19.8) rectangle (377.03000000000014,-21.400000000000002);
\draw(379.73000000000013, -10.7) node[anchor=north west,align=left] {\large Lie groups};
\draw (379.73000000000013, -10.7) rectangle (392.6300000000001,-37.5);
\draw(380.73000000000013, -11.7) node[anchor=north west,align=left] {Geometric \\ Langlands \\ program: \\ representation-theoretic\\ aspects};
\draw (380.73000000000013, -11.7) rectangle (387.33000000000015,-14.299999999999999);
\draw(387.4300000000001, -11.7) node[anchor=north west,align=left] {General \\ properties and \\ structure of other\\ Lie groups};
\draw (387.4300000000001, -11.7) rectangle (392.53000000000014,-13.799999999999999);
\draw(380.73000000000013, -14.399999999999999) node[anchor=north west,align=left] {Representations\\ of nilpotent and\\ solvable Lie \\ groups (special \\ orbital integrals,\\ non-type I \\ representations, etc.)};
\draw (380.73000000000013, -14.399999999999999) rectangle (386.83000000000015,-18.0);
\draw(386.9300000000001, -14.399999999999999) node[anchor=north west,align=left] {Infinite-dimensional\\ Lie \\ groups and their\\ Lie algebras:\\ general properties};
\draw (386.9300000000001, -14.399999999999999) rectangle (392.53000000000014,-17.0);
\draw(380.73000000000013, -18.1) node[anchor=north west,align=left] {Representations\\ of Lie and \\ real algebraic\\ groups: algebraic\\ methods \\ (Verma modules, etc.)};
\draw (380.73000000000013, -18.1) rectangle (386.58000000000015,-21.200000000000003);
\draw(386.6800000000001, -18.1) node[anchor=north west,align=left] {Analysis on \\ and representations\\ of \\ infinite-dimensional\\ Lie groups};
\draw (386.6800000000001, -18.1) rectangle (392.28000000000014,-20.700000000000003);
\draw(380.73000000000013, -21.3) node[anchor=north west,align=left] {Representations\\ of Lie and\\ linear algebraic\\ groups \\ over local fields};
\draw (380.73000000000013, -21.3) rectangle (385.58000000000015,-23.900000000000002);
\draw(385.6800000000001, -21.3) node[anchor=north west,align=left] {Representations\\ of Lie and \\ linear algebraic\\ groups over\\ real fields: \\ analytic methods};
\draw (385.6800000000001, -21.3) rectangle (390.28000000000014,-24.400000000000002);
\draw(380.73000000000013, -24.5) node[anchor=north west,align=left] {Representations\\ of Lie and \\ linear algebraic\\ groups over\\ global fields\\ and adèle rings};
\draw (380.73000000000013, -24.5) rectangle (385.33000000000015,-27.6);
\draw(385.4300000000001, -24.5) node[anchor=north west,align=left] {Applications\\ of Lie groups\\ to the sciences;\\ explicit\\ representations};
\draw (385.4300000000001, -24.5) rectangle (390.03000000000014,-27.1);
\draw(380.73000000000013, -27.7) node[anchor=north west,align=left] {General \\ properties and \\ structure of\\ real Lie groups};
\draw (380.73000000000013, -27.7) rectangle (385.08000000000015,-29.8);
\draw(385.1800000000001, -27.7) node[anchor=north west,align=left] {Semisimple\\ Lie groups\\ and their \\ representations};
\draw (385.1800000000001, -27.7) rectangle (389.53000000000014,-29.8);
\draw(389.6300000000001, -27.7) node[anchor=north west,align=left] {Local Lie\\ groups};
\draw (389.6300000000001, -27.7) rectangle (392.48000000000013,-28.8);
\draw(380.73000000000013, -29.9) node[anchor=north west,align=left] {Loop groups\\ and related\\ constructions,\\ group-theoretic\\ treatment};
\draw (380.73000000000013, -29.9) rectangle (385.08000000000015,-32.5);
\draw(385.1800000000001, -29.9) node[anchor=north west,align=left] {Structure and\\ representation\\ of the\\ Lorentz group};
\draw (385.1800000000001, -29.9) rectangle (389.28000000000014,-32.0);
\draw(389.3800000000001, -29.9) node[anchor=north west,align=left] {Nilpotent\\ and \\ solvable \\ Lie groups};
\draw (389.3800000000001, -29.9) rectangle (392.48000000000013,-32.0);
\draw(380.73000000000013, -32.599999999999994) node[anchor=north west,align=left] {General \\ properties \\ and structure\\ of complex\\ Lie groups};
\draw (380.73000000000013, -32.599999999999994) rectangle (384.58000000000015,-35.199999999999996);
\draw(384.6800000000001, -32.599999999999994) node[anchor=north west,align=left] {Discrete \\ subgroups \\ of Lie groups};
\draw (384.6800000000001, -32.599999999999994) rectangle (388.53000000000014,-34.199999999999996);
\draw(388.6300000000001, -32.599999999999994) node[anchor=north west,align=left] {Continuous\\ cohomologyof\\ Lie groups};
\draw (388.6300000000001, -32.599999999999994) rectangle (392.23000000000013,-34.699999999999996);
\draw(380.73000000000013, -35.3) node[anchor=north west,align=left] {Analysis\\ on real \\ and complex\\ Lie groups};
\draw (380.73000000000013, -35.3) rectangle (384.08000000000015,-37.4);
\draw(384.1800000000001, -35.3) node[anchor=north west,align=left] {Analysis\\ on \\ \(p\)-adic \\ Lie groups};
\draw (384.1800000000001, -35.3) rectangle (387.53000000000014,-37.4);
\draw(387.6300000000001, -35.3) node[anchor=north west,align=left] {Lie \\ algebras of\\ Lie groups};
\draw (387.6300000000001, -35.3) rectangle (390.98000000000013,-36.9);
\draw(364.78000000000014, -22.1) node[anchor=north west,align=left] {\large Noncompact transformation groups};
\draw (364.78000000000014, -22.1) rectangle (376.1800000000001,-27.5);
\draw(365.78000000000014, -23.1) node[anchor=north west,align=left] {General \\ theory of group\\ and \\ pseudogroup actions};
\draw (365.78000000000014, -23.1) rectangle (371.13000000000017,-25.200000000000003);
\draw(371.23000000000013, -23.1) node[anchor=north west,align=left] {Homogeneousspaces};
\draw (371.23000000000013, -23.1) rectangle (376.08000000000015,-24.700000000000003);
\draw(365.78000000000014, -25.3) node[anchor=north west,align=left] {Groups as\\ automorphisms\\ of other\\ structures};
\draw (365.78000000000014, -25.3) rectangle (369.63000000000017,-27.400000000000002);
\draw(369.73000000000013, -25.3) node[anchor=north west,align=left] {Measurable\\ group\\ actions};
\draw (369.73000000000013, -25.3) rectangle (372.83000000000015,-26.900000000000002);
\draw(364.78000000000014, -27.6) node[anchor=north west,align=left] {\large Computational methods\\ for problems pertaining\\ to topological groups};
\draw (364.78000000000014, -27.6) rectangle (372.51000000000016,-29.200000000000003);
\draw(364.78000000000014, -29.3) node[anchor=north west,align=left] {\large History of \\ topological groups};
\draw (364.78000000000014, -29.3) rectangle (370.96000000000015,-30.400000000000002);
\draw(364.78000000000014, -30.5) node[anchor=north west,align=left] {\large Compact groups};
\draw (364.78000000000014, -30.5) rectangle (369.72000000000014,-32.7);
\draw(365.78000000000014, -31.5) node[anchor=north west,align=left] {Compact\\ groups};
\draw (365.78000000000014, -31.5) rectangle (368.13000000000017,-32.6);
\draw(363.78000000000014, -37.7) node[anchor=north west,align=left] {\LARGE Mathematical logic and foundations};
\draw (363.78000000000014, -37.7) rectangle (393.1800000000001,-117.89999999999999);
\draw(364.78000000000014, -38.7) node[anchor=north west,align=left] {\large Proof theory and constructive mathematics};
\draw (364.78000000000014, -38.7) rectangle (380.38000000000017,-51.7);
\draw(365.78000000000014, -39.7) node[anchor=north west,align=left] {Provability \\ logics and related\\ algebras \\ (e.g., diagonalizable\\ algebras)};
\draw (365.78000000000014, -39.7) rectangle (371.63000000000017,-42.300000000000004);
\draw(371.73000000000013, -39.7) node[anchor=north west,align=left] {Proof-theoretic\\ aspects of \\ linear logic and\\ other \\ substructural logics};
\draw (371.73000000000013, -39.7) rectangle (377.33000000000015,-42.300000000000004);
\draw(377.4300000000001, -39.7) node[anchor=north west,align=left] {Structure\\ of\\ proofs};
\draw (377.4300000000001, -39.7) rectangle (380.28000000000014,-41.300000000000004);
\draw(365.78000000000014, -42.400000000000006) node[anchor=north west,align=left] {Proof theory,\\ general\\ (including\\ proof-theoretic\\ semantics)};
\draw (365.78000000000014, -42.400000000000006) rectangle (370.13000000000017,-45.00000000000001);
\draw(370.23000000000013, -42.400000000000006) node[anchor=north west,align=left] {Cut-elimination\\ and\\ normal-form\\ theorems};
\draw (370.23000000000013, -42.400000000000006) rectangle (374.58000000000015,-44.50000000000001);
\draw(374.6800000000001, -42.400000000000006) node[anchor=north west,align=left] {Relative \\ consistency\\ and \\ interpretations};
\draw (374.6800000000001, -42.400000000000006) rectangle (379.03000000000014,-44.50000000000001);
\draw(365.78000000000014, -45.1) node[anchor=north west,align=left] {Gödel \\ numberings and \\ issues of \\ incompleteness};
\draw (365.78000000000014, -45.1) rectangle (370.13000000000017,-47.2);
\draw(370.23000000000013, -45.1) node[anchor=north west,align=left] {Metamathematics\\ of\\ constructive\\ systems};
\draw (370.23000000000013, -45.1) rectangle (374.58000000000015,-47.2);
\draw(374.6800000000001, -45.1) node[anchor=north west,align=left] {Intuitionistic\\ mathematics};
\draw (374.6800000000001, -45.1) rectangle (378.78000000000014,-46.7);
\draw(365.78000000000014, -47.300000000000004) node[anchor=north west,align=left] {Second- and\\ higher-order\\ arithmetic\\ and fragments};
\draw (365.78000000000014, -47.300000000000004) rectangle (369.63000000000017,-49.400000000000006);
\draw(369.73000000000013, -47.300000000000004) node[anchor=north west,align=left] {Other \\ constructive \\ mathematics};
\draw (369.73000000000013, -47.300000000000004) rectangle (373.58000000000015,-48.900000000000006);
\draw(373.6800000000001, -47.300000000000004) node[anchor=north west,align=left] {Constructive\\ and\\ recursive\\ analysis};
\draw (373.6800000000001, -47.300000000000004) rectangle (377.28000000000014,-49.400000000000006);
\draw(365.78000000000014, -49.5) node[anchor=north west,align=left] {Functionals\\ in proof\\ theory};
\draw (365.78000000000014, -49.5) rectangle (369.13000000000017,-51.1);
\draw(369.23000000000013, -49.5) node[anchor=north west,align=left] {Recursive\\ ordinals \\ and ordinal\\ notations};
\draw (369.23000000000013, -49.5) rectangle (372.58000000000015,-51.6);
\draw(372.6800000000001, -49.5) node[anchor=north west,align=left] {First-order\\ arithmetic\\ and\\ fragments};
\draw (372.6800000000001, -49.5) rectangle (376.03000000000014,-51.6);
\draw(376.13000000000017, -49.5) node[anchor=north west,align=left] {Complexity\\ of proofs};
\draw (376.13000000000017, -49.5) rectangle (379.2300000000002,-51.1);
\draw(380.48000000000013, -38.7) node[anchor=north west,align=left] {\large Computability and recursion theory};
\draw (380.48000000000013, -38.7) rectangle (393.08000000000015,-63.0);
\draw(381.48000000000013, -39.7) node[anchor=north west,align=left] {Computability\\ and recursion\\ theory on \\ ordinals, \\ admissible sets, etc.};
\draw (381.48000000000013, -39.7) rectangle (387.33000000000015,-42.300000000000004);
\draw(387.4300000000001, -39.7) node[anchor=north west,align=left] {Complexity of\\ computation \\ (including implicit\\ computational\\ complexity)};
\draw (387.4300000000001, -39.7) rectangle (392.78000000000014,-42.300000000000004);
\draw(381.48000000000013, -42.400000000000006) node[anchor=north west,align=left] {Computation\\ over the \\ reals, \\ computable analysis};
\draw (381.48000000000013, -42.400000000000006) rectangle (386.83000000000015,-44.50000000000001);
\draw(386.9300000000001, -42.400000000000006) node[anchor=north west,align=left] {Other degrees\\ and reducibilities\\ in \\ computability and \\ recursion theory};
\draw (386.9300000000001, -42.400000000000006) rectangle (392.03000000000014,-45.00000000000001);
\draw(381.48000000000013, -45.1) node[anchor=north west,align=left] {Recursive \\ equivalence\\ types of \\ sets and \\ structures, isols};
\draw (381.48000000000013, -45.1) rectangle (386.33000000000015,-47.7);
\draw(386.4300000000001, -45.1) node[anchor=north west,align=left] {Word problems,\\ etc. in\\ computability\\ and \\ recursion theory};
\draw (386.4300000000001, -45.1) rectangle (391.03000000000014,-47.7);
\draw(381.48000000000013, -47.800000000000004) node[anchor=north west,align=left] {Hierarchies\\ of computability\\ and\\ definability};
\draw (381.48000000000013, -47.800000000000004) rectangle (386.08000000000015,-49.900000000000006);
\draw(386.1800000000001, -47.800000000000004) node[anchor=north west,align=left] {Abstract and\\ axiomatic\\ computability\\ and \\ recursion theory};
\draw (386.1800000000001, -47.800000000000004) rectangle (390.78000000000014,-50.400000000000006);
\draw(381.48000000000013, -50.5) node[anchor=north west,align=left] {Applications\\ of computability\\ and \\ recursion theory};
\draw (381.48000000000013, -50.5) rectangle (386.08000000000015,-52.6);
\draw(386.1800000000001, -50.5) node[anchor=north west,align=left] {Automata and\\ formal grammars\\ in connection\\ with logical\\ questions};
\draw (386.1800000000001, -50.5) rectangle (390.53000000000014,-53.1);
\draw(381.48000000000013, -53.2) node[anchor=north west,align=left] {Recursively \\ (computably) \\ enumerable sets\\ and degrees};
\draw (381.48000000000013, -53.2) rectangle (385.83000000000015,-55.300000000000004);
\draw(385.9300000000001, -53.2) node[anchor=north west,align=left] {Undecidability\\ and \\ degrees of sets\\ of sentences};
\draw (385.9300000000001, -53.2) rectangle (390.28000000000014,-55.300000000000004);
\draw(381.48000000000013, -55.400000000000006) node[anchor=north west,align=left] {Thue and\\ Post \\ systems, etc.};
\draw (381.48000000000013, -55.400000000000006) rectangle (385.33000000000015,-57.00000000000001);
\draw(385.4300000000001, -55.400000000000006) node[anchor=north west,align=left] {Recursive \\ functions and\\ relations,\\ subrecursive\\ hierarchies};
\draw (385.4300000000001, -55.400000000000006) rectangle (389.28000000000014,-58.00000000000001);
\draw(389.3800000000001, -55.400000000000006) node[anchor=north west,align=left] {Other Turing\\ degree\\ structures};
\draw (389.3800000000001, -55.400000000000006) rectangle (392.98000000000013,-57.00000000000001);
\draw(381.48000000000013, -58.1) node[anchor=north west,align=left] {Theory of \\ numerations,\\ effectively\\ presented\\ structures};
\draw (381.48000000000013, -58.1) rectangle (385.08000000000015,-60.7);
\draw(385.1800000000001, -58.1) node[anchor=north west,align=left] {Inductive\\ definability};
\draw (385.1800000000001, -58.1) rectangle (388.78000000000014,-59.7);
\draw(388.8800000000001, -58.1) node[anchor=north west,align=left] {Turing \\ machines\\ and related\\ notions};
\draw (388.8800000000001, -58.1) rectangle (392.23000000000013,-60.2);
\draw(381.48000000000013, -60.8) node[anchor=north west,align=left] {Algorithmic\\ randomness\\ and\\ dimension};
\draw (381.48000000000013, -60.8) rectangle (384.83000000000015,-62.9);
\draw(384.9300000000001, -60.8) node[anchor=north west,align=left] {Higher-type\\ and set\\ recursion\\ theory};
\draw (384.9300000000001, -60.8) rectangle (388.28000000000014,-62.9);
\draw(364.78000000000014, -51.800000000000004) node[anchor=north west,align=left] {\large Philosophical aspects of logic and foundations};
\draw (364.78000000000014, -51.800000000000004) rectangle (379.64000000000016,-55.50000000000001);
\draw(365.78000000000014, -52.800000000000004) node[anchor=north west,align=left] {Philosophical\\ and \\ critical aspects\\ of logic\\ and foundations};
\draw (365.78000000000014, -52.800000000000004) rectangle (370.38000000000017,-55.400000000000006);
\draw(370.48000000000013, -52.800000000000004) node[anchor=north west,align=left] {Logic in\\ the \\ philosophy\\ of science};
\draw (370.48000000000013, -52.800000000000004) rectangle (373.58000000000015,-54.900000000000006);
\draw(364.78000000000014, -55.60000000000001) node[anchor=north west,align=left] {\large Computational methods\\ for problems \\ pertaining to mathematical\\ logic and foundations};
\draw (364.78000000000014, -55.60000000000001) rectangle (373.44000000000017,-57.70000000000001);
\draw(364.78000000000014, -57.800000000000004) node[anchor=north west,align=left] {\large History of \\ mathematical logic\\ and foundations};
\draw (364.78000000000014, -57.800000000000004) rectangle (370.96000000000015,-59.400000000000006);
\draw(364.78000000000014, -63.10000000000001) node[anchor=north west,align=left] {\large General logic};
\draw (364.78000000000014, -63.10000000000001) rectangle (376.9300000000001,-84.4);
\draw(365.78000000000014, -64.10000000000001) node[anchor=north west,align=left] {Substructural \\ logics (including \\ relevance, entailment,\\ linear logic,\\ Lambek calculus,\\ BCK and BCI logics)};
\draw (365.78000000000014, -64.10000000000001) rectangle (371.88000000000017,-67.2);
\draw(371.98000000000013, -64.10000000000001) node[anchor=north west,align=left] {Probability\\ and \\ inductive logic};
\draw (371.98000000000013, -64.10000000000001) rectangle (376.33000000000015,-65.7);
\draw(365.78000000000014, -67.30000000000001) node[anchor=north west,align=left] {Subsystems \\ of classical\\ logic (including\\ intuitionistic logic)};
\draw (365.78000000000014, -67.30000000000001) rectangle (371.63000000000017,-69.9);
\draw(371.73000000000013, -67.30000000000001) node[anchor=north west,align=left] {Foundations \\ of classical \\ theories \\ (including reverse\\ mathematics)};
\draw (371.73000000000013, -67.30000000000001) rectangle (376.83000000000015,-69.9);
\draw(365.78000000000014, -70.00000000000001) node[anchor=north west,align=left] {Logics of \\ knowledge and\\ belief \\ (including \\ belief change)};
\draw (365.78000000000014, -70.00000000000001) rectangle (369.88000000000017,-72.60000000000001);
\draw(369.98000000000013, -70.00000000000001) node[anchor=north west,align=left] {Paraconsistent\\ logics};
\draw (369.98000000000013, -70.00000000000001) rectangle (374.08000000000015,-71.60000000000001);
\draw(374.1800000000001, -70.00000000000001) node[anchor=north west,align=left] {Temporal\\ logic};
\draw (374.1800000000001, -70.00000000000001) rectangle (376.78000000000014,-71.10000000000001);
\draw(365.78000000000014, -72.70000000000002) node[anchor=north west,align=left] {Classical\\ propositional\\ logic};
\draw (365.78000000000014, -72.70000000000002) rectangle (369.63000000000017,-74.30000000000001);
\draw(369.73000000000013, -72.70000000000002) node[anchor=north west,align=left] {Mechanization\\ of proofs\\ and logical\\ operations};
\draw (369.73000000000013, -72.70000000000002) rectangle (373.58000000000015,-74.80000000000001);
\draw(373.6800000000001, -72.70000000000002) node[anchor=north west,align=left] {Abstract\\ deductive\\ systems};
\draw (373.6800000000001, -72.70000000000002) rectangle (376.53000000000014,-74.30000000000001);
\draw(365.78000000000014, -74.9) node[anchor=north west,align=left] {Higher-order\\ logic};
\draw (365.78000000000014, -74.9) rectangle (369.38000000000017,-76.5);
\draw(369.48000000000013, -74.9) node[anchor=north west,align=left] {Decidability\\ of theories\\ and sets\\ of sentences};
\draw (369.48000000000013, -74.9) rectangle (373.08000000000015,-77.0);
\draw(373.1800000000001, -74.9) node[anchor=north west,align=left] {Fuzzy logic;\\ logic \\ of vagueness};
\draw (373.1800000000001, -74.9) rectangle (376.78000000000014,-76.5);
\draw(365.78000000000014, -77.10000000000001) node[anchor=north west,align=left] {Intermediate\\ logics};
\draw (365.78000000000014, -77.10000000000001) rectangle (369.38000000000017,-78.7);
\draw(369.48000000000013, -77.10000000000001) node[anchor=north west,align=left] {Other \\ nonclassical\\ logic};
\draw (369.48000000000013, -77.10000000000001) rectangle (373.08000000000015,-78.7);
\draw(373.1800000000001, -77.10000000000001) node[anchor=north west,align=left] {Other \\ applications\\ of logic};
\draw (373.1800000000001, -77.10000000000001) rectangle (376.78000000000014,-78.7);
\draw(365.78000000000014, -78.80000000000001) node[anchor=north west,align=left] {Classical\\ first-order\\ logic};
\draw (365.78000000000014, -78.80000000000001) rectangle (369.13000000000017,-80.4);
\draw(369.23000000000013, -78.80000000000001) node[anchor=north west,align=left] {Combinatory\\ logic\\ and lambda\\ calculus};
\draw (369.23000000000013, -78.80000000000001) rectangle (372.58000000000015,-80.9);
\draw(372.6800000000001, -78.80000000000001) node[anchor=north west,align=left] {Modal logic\\ (including\\ the logic\\ of norms)};
\draw (372.6800000000001, -78.80000000000001) rectangle (376.03000000000014,-80.9);
\draw(365.78000000000014, -81.0) node[anchor=north west,align=left] {Many-valued\\ logic};
\draw (365.78000000000014, -81.0) rectangle (369.13000000000017,-82.6);
\draw(369.23000000000013, -81.0) node[anchor=north west,align=left] {Logic of\\ natural\\ languages};
\draw (369.23000000000013, -81.0) rectangle (372.08000000000015,-82.6);
\draw(372.1800000000001, -81.0) node[anchor=north west,align=left] {Combined\\ logics};
\draw (372.1800000000001, -81.0) rectangle (374.78000000000014,-82.1);
\draw(365.78000000000014, -82.7) node[anchor=north west,align=left] {Logic in\\ computer\\ science};
\draw (365.78000000000014, -82.7) rectangle (368.38000000000017,-84.3);
\draw(368.48000000000013, -82.7) node[anchor=north west,align=left] {Type\\ theory};
\draw (368.48000000000013, -82.7) rectangle (370.58000000000015,-83.8);
\draw(377.03000000000014, -63.10000000000001) node[anchor=north west,align=left] {\large Set theory};
\draw (377.03000000000014, -63.10000000000001) rectangle (388.88000000000017,-82.70000000000002);
\draw(378.03000000000014, -64.10000000000001) node[anchor=north west,align=left] {Other classical\\ set theory \\ (including functions,\\ relations,\\ and set algebra)};
\draw (378.03000000000014, -64.10000000000001) rectangle (383.88000000000017,-66.7);
\draw(383.98000000000013, -64.10000000000001) node[anchor=north west,align=left] {Cardinal \\ characteristics\\ of the\\ continuum};
\draw (383.98000000000013, -64.10000000000001) rectangle (388.33000000000015,-66.2);
\draw(378.03000000000014, -66.80000000000001) node[anchor=north west,align=left] {Other aspects\\ of forcing\\ and \\ Boolean-valued models};
\draw (378.03000000000014, -66.80000000000001) rectangle (383.88000000000017,-68.9);
\draw(383.98000000000013, -66.80000000000001) node[anchor=north west,align=left] {Continuum\\ hypothesis\\ and \\ Martin’s axiom};
\draw (383.98000000000013, -66.80000000000001) rectangle (388.08000000000015,-68.9);
\draw(378.03000000000014, -69.00000000000001) node[anchor=north west,align=left] {Inner models, \\ including \\ constructibility, \\ ordinal definability,\\ and core models};
\draw (378.03000000000014, -69.00000000000001) rectangle (383.88000000000017,-71.60000000000001);
\draw(383.98000000000013, -69.00000000000001) node[anchor=north west,align=left] {Generic \\ absoluteness\\ and \\ forcing axioms};
\draw (383.98000000000013, -69.00000000000001) rectangle (388.08000000000015,-71.10000000000001);
\draw(378.03000000000014, -71.70000000000002) node[anchor=north west,align=left] {Ordered sets\\ and their\\ cofinalities;\\ pcf theory};
\draw (378.03000000000014, -71.70000000000002) rectangle (381.88000000000017,-73.80000000000001);
\draw(381.98000000000013, -71.70000000000002) node[anchor=north west,align=left] {Other \\ combinatorial\\ set theory};
\draw (381.98000000000013, -71.70000000000002) rectangle (385.83000000000015,-73.30000000000001);
\draw(385.9300000000001, -71.70000000000002) node[anchor=north west,align=left] {Partition\\ relations};
\draw (385.9300000000001, -71.70000000000002) rectangle (388.78000000000014,-73.30000000000001);
\draw(378.03000000000014, -73.9) node[anchor=north west,align=left] {Axiomatics of\\ classical set\\ theory and\\ its fragments};
\draw (378.03000000000014, -73.9) rectangle (381.88000000000017,-76.0);
\draw(381.98000000000013, -73.9) node[anchor=north west,align=left] {Other notions\\ of \\ set-theoretic\\ definability};
\draw (381.98000000000013, -73.9) rectangle (385.83000000000015,-76.0);
\draw(385.9300000000001, -73.9) node[anchor=north west,align=left] {Large \\ cardinals};
\draw (385.9300000000001, -73.9) rectangle (388.78000000000014,-75.0);
\draw(378.03000000000014, -76.10000000000001) node[anchor=north west,align=left] {Other \\ set-theoretic\\ hypotheses\\ and axioms};
\draw (378.03000000000014, -76.10000000000001) rectangle (381.88000000000017,-78.2);
\draw(381.98000000000013, -76.10000000000001) node[anchor=north west,align=left] {Ordinal \\ and cardinal\\ numbers};
\draw (381.98000000000013, -76.10000000000001) rectangle (385.58000000000015,-77.7);
\draw(385.6800000000001, -76.10000000000001) node[anchor=north west,align=left] {Theory \\ of fuzzy\\ sets, etc.};
\draw (385.6800000000001, -76.10000000000001) rectangle (388.78000000000014,-77.7);
\draw(378.03000000000014, -78.30000000000001) node[anchor=north west,align=left] {Axiom of \\ choice and\\ related \\ propositions};
\draw (378.03000000000014, -78.30000000000001) rectangle (381.63000000000017,-80.4);
\draw(381.73000000000013, -78.30000000000001) node[anchor=north west,align=left] {Consistency\\ and \\ independence\\ results};
\draw (381.73000000000013, -78.30000000000001) rectangle (385.33000000000015,-80.4);
\draw(385.4300000000001, -78.30000000000001) node[anchor=north west,align=left] {Descriptive\\ set\\ theory};
\draw (385.4300000000001, -78.30000000000001) rectangle (388.78000000000014,-79.9);
\draw(378.03000000000014, -80.5) node[anchor=north west,align=left] {Nonclassical\\ and \\ second-order\\ set theories};
\draw (378.03000000000014, -80.5) rectangle (381.63000000000017,-82.6);
\draw(381.73000000000013, -80.5) node[anchor=north west,align=left] {Applications\\ of\\ set theory};
\draw (381.73000000000013, -80.5) rectangle (385.33000000000015,-82.1);
\draw(385.4300000000001, -80.5) node[anchor=north west,align=left] {Determinacy\\ principles};
\draw (385.4300000000001, -80.5) rectangle (388.78000000000014,-82.1);
\draw(364.78000000000014, -84.5) node[anchor=north west,align=left] {\large Model theory};
\draw (364.78000000000014, -84.5) rectangle (374.9300000000001,-117.8);
\draw(365.78000000000014, -85.5) node[anchor=north west,align=left] {Equational \\ classes, universal\\ algebra \\ in model theory};
\draw (365.78000000000014, -85.5) rectangle (370.88000000000017,-87.6);
\draw(370.98000000000013, -85.5) node[anchor=north west,align=left] {Ultraproducts\\ and\\ related \\ constructions};
\draw (370.98000000000013, -85.5) rectangle (374.83000000000015,-87.6);
\draw(365.78000000000014, -87.7) node[anchor=north west,align=left] {Quantifier \\ elimination,\\ model \\ completeness and \\ related topics};
\draw (365.78000000000014, -87.7) rectangle (370.63000000000017,-90.3);
\draw(370.73000000000013, -87.7) node[anchor=north west,align=left] {Basic \\ properties of \\ first-order \\ languages and\\ structures};
\draw (370.73000000000013, -87.7) rectangle (374.83000000000015,-90.3);
\draw(365.78000000000014, -90.4) node[anchor=north west,align=left] {Computable \\ structure theory,\\ computable\\ model theory};
\draw (365.78000000000014, -90.4) rectangle (370.63000000000017,-92.5);
\draw(370.73000000000013, -90.4) node[anchor=north west,align=left] {Model theory\\ of denumerable\\ and separable\\ structures};
\draw (370.73000000000013, -90.4) rectangle (374.83000000000015,-92.5);
\draw(365.78000000000014, -92.6) node[anchor=north west,align=left] {Continuous\\ model theory,\\ model \\ theory of \\ metric structures};
\draw (365.78000000000014, -92.6) rectangle (370.63000000000017,-95.19999999999999);
\draw(370.73000000000013, -92.6) node[anchor=north west,align=left] {Interpolation,\\ preservation, \\ definability};
\draw (370.73000000000013, -92.6) rectangle (374.83000000000015,-94.69999999999999);
\draw(365.78000000000014, -95.3) node[anchor=north west,align=left] {Classification\\ theory, \\ stability and \\ related concepts\\ in model theory};
\draw (365.78000000000014, -95.3) rectangle (370.38000000000017,-97.89999999999999);
\draw(370.48000000000013, -95.3) node[anchor=north west,align=left] {Model-theoretic\\ forcing};
\draw (370.48000000000013, -95.3) rectangle (374.83000000000015,-96.89999999999999);
\draw(365.78000000000014, -98.0) node[anchor=north west,align=left] {Nonclassical\\ models \\ (Boolean-valued,\\ sheaf, etc.)};
\draw (365.78000000000014, -98.0) rectangle (370.38000000000017,-100.1);
\draw(370.48000000000013, -98.0) node[anchor=north west,align=left] {Model-theoretic\\ algebra};
\draw (370.48000000000013, -98.0) rectangle (374.83000000000015,-99.6);
\draw(365.78000000000014, -100.2) node[anchor=north west,align=left] {Abstract \\ elementary \\ classes and \\ related topics};
\draw (365.78000000000014, -100.2) rectangle (369.88000000000017,-102.3);
\draw(369.98000000000013, -100.2) node[anchor=north west,align=left] {Other \\ model \\ constructions};
\draw (369.98000000000013, -100.2) rectangle (373.83000000000015,-101.8);
\draw(365.78000000000014, -102.4) node[anchor=north west,align=left] {Set-theoretic\\ model theory};
\draw (365.78000000000014, -102.4) rectangle (369.63000000000017,-104.0);
\draw(369.73000000000013, -102.4) node[anchor=north west,align=left] {Models \\ of arithmetic\\ and\\ set theory};
\draw (369.73000000000013, -102.4) rectangle (373.58000000000015,-104.5);
\draw(365.78000000000014, -104.6) node[anchor=north west,align=left] {Logic with\\ extra \\ quantifiers \\ and operators};
\draw (365.78000000000014, -104.6) rectangle (369.63000000000017,-106.69999999999999);
\draw(369.73000000000013, -104.6) node[anchor=north west,align=left] {Second- \\ and \\ higher-order \\ model theory};
\draw (369.73000000000013, -104.6) rectangle (373.58000000000015,-106.69999999999999);
\draw(365.78000000000014, -106.8) node[anchor=north west,align=left] {Categoricity\\ and \\ completeness\\ of theories};
\draw (365.78000000000014, -106.8) rectangle (369.38000000000017,-108.89999999999999);
\draw(369.48000000000013, -106.8) node[anchor=north west,align=left] {Models with\\ special \\ properties \\ (saturated, \\ rigid, etc.)};
\draw (369.48000000000013, -106.8) rectangle (373.08000000000015,-109.39999999999999);
\draw(365.78000000000014, -109.5) node[anchor=north west,align=left] {Model theory\\ of ordered\\ structures;\\ o-minimality};
\draw (365.78000000000014, -109.5) rectangle (369.38000000000017,-111.6);
\draw(369.48000000000013, -109.5) node[anchor=north west,align=left] {Models of\\ other \\ mathematical\\ theories};
\draw (369.48000000000013, -109.5) rectangle (373.08000000000015,-111.6);
\draw(365.78000000000014, -111.69999999999999) node[anchor=north west,align=left] {Other \\ classical \\ first-order \\ model theory};
\draw (365.78000000000014, -111.69999999999999) rectangle (369.38000000000017,-113.79999999999998);
\draw(369.48000000000013, -111.69999999999999) node[anchor=north west,align=left] {Applications\\ of \\ model theory};
\draw (369.48000000000013, -111.69999999999999) rectangle (373.08000000000015,-113.29999999999998);
\draw(365.78000000000014, -113.89999999999999) node[anchor=north west,align=left] {Model \\ theory of\\ finite \\ structures};
\draw (365.78000000000014, -113.89999999999999) rectangle (368.88000000000017,-115.99999999999999);
\draw(368.98000000000013, -113.89999999999999) node[anchor=north west,align=left] {Properties\\ of classes\\ of models};
\draw (368.98000000000013, -113.89999999999999) rectangle (372.08000000000015,-115.49999999999999);
\draw(372.1800000000001, -113.89999999999999) node[anchor=north west,align=left] {Abstract\\ model\\ theory};
\draw (372.1800000000001, -113.89999999999999) rectangle (374.78000000000014,-115.49999999999999);
\draw(365.78000000000014, -116.1) node[anchor=north west,align=left] {Logic on\\ admissible\\ sets};
\draw (365.78000000000014, -116.1) rectangle (368.88000000000017,-117.69999999999999);
\draw(368.98000000000013, -116.1) node[anchor=north west,align=left] {Other \\ infinitary\\ logic};
\draw (368.98000000000013, -116.1) rectangle (372.08000000000015,-117.69999999999999);
\draw(375.03000000000014, -84.5) node[anchor=north west,align=left] {\large Algebraic logic};
\draw (375.03000000000014, -84.5) rectangle (384.4300000000001,-93.8);
\draw(376.03000000000014, -85.5) node[anchor=north west,align=left] {Cylindric and\\ polyadic \\ algebras; \\ relation algebras};
\draw (376.03000000000014, -85.5) rectangle (380.88000000000017,-87.6);
\draw(380.98000000000013, -85.5) node[anchor=north west,align=left] {Categorical\\ logic,\\ topoi};
\draw (380.98000000000013, -85.5) rectangle (384.33000000000015,-87.1);
\draw(376.03000000000014, -87.7) node[anchor=north west,align=left] {Logical aspects\\ of \\ Łukasiewicz and \\ Post algebras};
\draw (376.03000000000014, -87.7) rectangle (380.63000000000017,-89.8);
\draw(380.73000000000013, -87.7) node[anchor=north west,align=left] {Logical\\ aspects\\ of Boolean\\ algebras};
\draw (380.73000000000013, -87.7) rectangle (383.83000000000015,-89.8);
\draw(376.03000000000014, -89.9) node[anchor=north west,align=left] {Other \\ algebras related\\ to logic};
\draw (376.03000000000014, -89.9) rectangle (380.63000000000017,-91.5);
\draw(380.73000000000013, -89.9) node[anchor=north west,align=left] {Abstract\\ algebraic\\ logic};
\draw (380.73000000000013, -89.9) rectangle (383.58000000000015,-91.5);
\draw(376.03000000000014, -91.6) node[anchor=north west,align=left] {Logical aspects\\ of lattices\\ and related\\ structures};
\draw (376.03000000000014, -91.6) rectangle (380.38000000000017,-93.69999999999999);
\draw(380.48000000000013, -91.6) node[anchor=north west,align=left] {Quantum\\ logic};
\draw (380.48000000000013, -91.6) rectangle (382.83000000000015,-92.69999999999999);
\draw(375.03000000000014, -93.9) node[anchor=north west,align=left] {\large Nonstandard models};
\draw (375.03000000000014, -93.9) rectangle (383.4300000000001,-99.80000000000001);
\draw(376.03000000000014, -94.9) node[anchor=north west,align=left] {Other applications\\ of \\ nonstandard models\\ (economics,\\ physics, etc.)};
\draw (376.03000000000014, -94.9) rectangle (381.13000000000017,-97.5);
\draw(376.03000000000014, -97.60000000000001) node[anchor=north west,align=left] {Nonstandard\\ models \\ of arithmetic};
\draw (376.03000000000014, -97.60000000000001) rectangle (379.88000000000017,-99.2);
\draw(379.98000000000013, -97.60000000000001) node[anchor=north west,align=left] {Nonstandard\\ models\\ in \\ mathematics};
\draw (379.98000000000013, -97.60000000000001) rectangle (383.33000000000015,-99.7);
\draw(394.8600000000002, -1) node[anchor=north west,align=left] {\LARGE K-Theory};
\draw (394.8600000000002, -1) rectangle (423.3100000000002,-38.6);
\draw(395.8600000000002, -2) node[anchor=north west,align=left] {\large Higher algebraic \(K\)-theory};
\draw (395.8600000000002, -2) rectangle (411.26000000000016,-10.1);
\draw(396.8600000000002, -3) node[anchor=north west,align=left] {Karoubi-Villamayor-Gersten\(K\)-theory};
\draw (396.8600000000002, -3) rectangle (406.9600000000002,-5.1);
\draw(407.0600000000002, -3) node[anchor=north west,align=left] {\(K\)-theory\\ and homology;\\ cyclic\\ homology \\ and cohomology};
\draw (407.0600000000002, -3) rectangle (411.1600000000002,-5.6);
\draw(396.8600000000002, -5.7) node[anchor=north west,align=left] {\(Q\)- and\\ plus-constructions};
\draw (396.8600000000002, -5.7) rectangle (401.9600000000002,-7.300000000000001);
\draw(402.0600000000002, -5.7) node[anchor=north west,align=left] {Negative\\ \(K\)-theory,\\ NK and Nil};
\draw (402.0600000000002, -5.7) rectangle (405.9100000000002,-7.800000000000001);
\draw(406.01000000000016, -5.7) node[anchor=north west,align=left] {Algebraic\\ \(K\)-theory\\ of spaces};
\draw (406.01000000000016, -5.7) rectangle (409.6100000000002,-7.300000000000001);
\draw(396.8600000000002, -7.9) node[anchor=north west,align=left] {Higher \\ symbols,\\ Milnor \\ \(K\)-theory};
\draw (396.8600000000002, -7.9) rectangle (400.4600000000002,-10.0);
\draw(400.5600000000002, -7.9) node[anchor=north west,align=left] {Computations\\ of higher\\ \(K\)-theory\\ of rings};
\draw (400.5600000000002, -7.9) rectangle (404.1600000000002,-10.0);
\draw(404.26000000000016, -7.9) node[anchor=north west,align=left] {Symmetric\\ monoidal\\ categories};
\draw (404.26000000000016, -7.9) rectangle (407.3600000000002,-9.5);
\draw(411.3600000000002, -2) node[anchor=north west,align=left] {\large \(K\)-theory and operator algebras};
\draw (411.3600000000002, -2) rectangle (423.2100000000002,-6.9);
\draw(412.3600000000002, -3) node[anchor=north west,align=left] {Kasparov \\ theory \\ (\(KK\)-theory)};
\draw (412.3600000000002, -3) rectangle (416.7100000000002,-4.6);
\draw(416.8100000000002, -3) node[anchor=north west,align=left] {Ext and\\ \(K\)-homology};
\draw (416.8100000000002, -3) rectangle (420.9100000000002,-4.6);
\draw(421.01000000000016, -3) node[anchor=north west,align=left] {Index\\ theory};
\draw (421.01000000000016, -3) rectangle (423.1100000000002,-4.1);
\draw(412.3600000000002, -4.7) node[anchor=north west,align=left] {\(K_0\)\\ as an \\ ordered \\ group, traces};
\draw (412.3600000000002, -4.7) rectangle (416.2100000000002,-6.800000000000001);
\draw(395.8600000000002, -10.2) node[anchor=north west,align=left] {\large Miscellaneous applications of \(K\)-theory};
\draw (395.8600000000002, -10.2) rectangle (409.4800000000002,-13.399999999999999);
\draw(396.8600000000002, -11.2) node[anchor=north west,align=left] {Miscellaneous\\ applications of\\ \(K\)-theory};
\draw (396.8600000000002, -11.2) rectangle (401.2100000000002,-13.299999999999999);
\draw(409.58000000000015, -10.2) node[anchor=north west,align=left] {\large Grothendieck groups and \(K_0\)};
\draw (409.58000000000015, -10.2) rectangle (420.48000000000013,-16.1);
\draw(410.58000000000015, -11.2) node[anchor=north west,align=left] {Frobenius\\ induction,\\ Burnside\\ and \\ representation rings};
\draw (410.58000000000015, -11.2) rectangle (416.1800000000002,-13.799999999999999);
\draw(416.28000000000014, -11.2) node[anchor=north west,align=left] {Stability\\ for projective\\ modules};
\draw (416.28000000000014, -11.2) rectangle (420.38000000000017,-12.799999999999999);
\draw(410.58000000000015, -13.899999999999999) node[anchor=north west,align=left] {Efficient\\ generation\\ of modules};
\draw (410.58000000000015, -13.899999999999999) rectangle (413.6800000000002,-15.499999999999998);
\draw(413.78000000000014, -13.899999999999999) node[anchor=north west,align=left] {\(K_0\)\\ of group\\ rings \\ and orders};
\draw (413.78000000000014, -13.899999999999999) rectangle (416.88000000000017,-15.999999999999998);
\draw(416.98000000000013, -13.899999999999999) node[anchor=north west,align=left] {\(K_0\)\\ of other\\ rings};
\draw (416.98000000000013, -13.899999999999999) rectangle (419.58000000000015,-15.499999999999998);
\draw(395.8600000000002, -13.499999999999998) node[anchor=north west,align=left] {\large Computational methods\\ for problems \\ pertaining to \(K\)-theory};
\draw (395.8600000000002, -13.499999999999998) rectangle (404.5200000000002,-15.099999999999998);
\draw(395.8600000000002, -16.2) node[anchor=north west,align=left] {\large Whitehead groups and \(K_1\)};
\draw (395.8600000000002, -16.2) rectangle (406.4600000000002,-21.1);
\draw(396.8600000000002, -17.2) node[anchor=north west,align=left] {Stable\\ range \\ conditions};
\draw (396.8600000000002, -17.2) rectangle (399.9600000000002,-18.8);
\draw(400.0600000000002, -17.2) node[anchor=north west,align=left] {Stability\\ for linear\\ groups};
\draw (400.0600000000002, -17.2) rectangle (403.1600000000002,-18.8);
\draw(403.26000000000016, -17.2) node[anchor=north west,align=left] {\(K_1\)\\ of group\\ rings \\ and orders};
\draw (403.26000000000016, -17.2) rectangle (406.3600000000002,-19.3);
\draw(396.8600000000002, -19.4) node[anchor=north west,align=left] {Congruence\\ subgroup\\ problems};
\draw (396.8600000000002, -19.4) rectangle (399.9600000000002,-21.0);
\draw(406.5600000000002, -16.2) node[anchor=north west,align=left] {\large \(K\)-theory in number theory};
\draw (406.5600000000002, -16.2) rectangle (416.96000000000015,-23.1);
\draw(407.5600000000002, -17.2) node[anchor=north west,align=left] {Étale cohomology,\\ higher \\ regulators, zeta\\ and \(L\)-functions\\ (\(K\)-theoretic aspects)};
\draw (407.5600000000002, -17.2) rectangle (414.4100000000002,-20.3);
\draw(407.5600000000002, -20.4) node[anchor=north west,align=left] {Generalized\\ class field\\ theory \\ (\(K\)-theoretic\\ aspects)};
\draw (407.5600000000002, -20.4) rectangle (412.1600000000002,-23.0);
\draw(412.26000000000016, -20.4) node[anchor=north west,align=left] {Symbols and\\ arithmetic \\ (\(K\)-theoretic\\ aspects)};
\draw (412.26000000000016, -20.4) rectangle (416.8600000000002,-22.5);
\draw(395.8600000000002, -21.200000000000003) node[anchor=north west,align=left] {\large History of\\ \(K\)-theory};
\draw (395.8600000000002, -21.200000000000003) rectangle (400.1800000000002,-22.300000000000004);
\draw(395.8600000000002, -23.2) node[anchor=north west,align=left] {\large \(K\)-theory in geometry};
\draw (395.8600000000002, -23.2) rectangle (406.01000000000016,-29.1);
\draw(396.8600000000002, -24.2) node[anchor=north west,align=left] {Relations of\\ \(K\)-theory\\ with \\ cohomology theories};
\draw (396.8600000000002, -24.2) rectangle (402.2100000000002,-26.3);
\draw(402.3100000000002, -24.2) node[anchor=north west,align=left] {\(K\)-theory\\ of schemes};
\draw (402.3100000000002, -24.2) rectangle (405.9100000000002,-25.8);
\draw(396.8600000000002, -26.4) node[anchor=north west,align=left] {Algebraic \\ cycles and motivic\\ cohomology\\ (\(K\)-theoretic\\ aspects)};
\draw (396.8600000000002, -26.4) rectangle (401.9600000000002,-29.0);
\draw(406.1100000000002, -23.2) node[anchor=north west,align=left] {\large Obstructions from topology};
\draw (406.1100000000002, -23.2) rectangle (415.76000000000016,-29.1);
\draw(407.1100000000002, -24.2) node[anchor=north west,align=left] {Surgery \\ obstructions \\ (\(K\)-theoretic\\ aspects)};
\draw (407.1100000000002, -24.2) rectangle (411.7100000000002,-26.3);
\draw(411.8100000000002, -24.2) node[anchor=north west,align=left] {Whitehead\\ (and related)\\ torsion};
\draw (411.8100000000002, -24.2) rectangle (415.6600000000002,-25.8);
\draw(407.1100000000002, -26.4) node[anchor=north west,align=left] {Obstructions\\ to group\\ actions \\ (\(K\)-theoretic\\ aspects)};
\draw (407.1100000000002, -26.4) rectangle (411.7100000000002,-29.0);
\draw(411.8100000000002, -26.4) node[anchor=north west,align=left] {Finiteness\\ and other \\ obstructions\\ in \(K_0\)};
\draw (411.8100000000002, -26.4) rectangle (415.4100000000002,-28.5);
\draw(395.8600000000002, -29.2) node[anchor=north west,align=left] {\large Steinberg groups and \(K_2\)};
\draw (395.8600000000002, -29.2) rectangle (405.14000000000016,-34.6);
\draw(396.8600000000002, -30.2) node[anchor=north west,align=left] {Symbols, \\ presentations\\ and stability\\ of \(K_2\)};
\draw (396.8600000000002, -30.2) rectangle (400.7100000000002,-32.3);
\draw(400.8100000000002, -30.2) node[anchor=north west,align=left] {\(K_2\) \\ and the \\ Brauer group};
\draw (400.8100000000002, -30.2) rectangle (404.4100000000002,-31.8);
\draw(396.8600000000002, -32.4) node[anchor=north west,align=left] {Central \\ extensions \\ and Schur \\ multipliers};
\draw (396.8600000000002, -32.4) rectangle (400.2100000000002,-34.5);
\draw(400.3100000000002, -32.4) node[anchor=north west,align=left] {Excision\\ for\\ \(K_2\)};
\draw (400.3100000000002, -32.4) rectangle (402.9100000000002,-34.0);
\draw(405.2400000000002, -29.2) node[anchor=north west,align=left] {\large Topological \(K\)-theory};
\draw (405.2400000000002, -29.2) rectangle (414.39000000000016,-38.5);
\draw(406.2400000000002, -30.2) node[anchor=north west,align=left] {\(J\)-homomorphism,\\ Adams \\ operations};
\draw (406.2400000000002, -30.2) rectangle (411.5900000000002,-32.3);
\draw(406.2400000000002, -32.4) node[anchor=north west,align=left] {Geometric \\ applications \\ of topological\\ \(K\)-theory};
\draw (406.2400000000002, -32.4) rectangle (410.3400000000002,-34.5);
\draw(410.44000000000017, -32.4) node[anchor=north west,align=left] {Connective\\ \(K\)-theory,\\ cobordism};
\draw (410.44000000000017, -32.4) rectangle (414.2900000000002,-34.5);
\draw(406.2400000000002, -34.6) node[anchor=north west,align=left] {Twisted \\ \(K\)-theory;\\ differential\\ \(K\)-theory};
\draw (406.2400000000002, -34.6) rectangle (410.0900000000002,-36.7);
\draw(410.19000000000017, -34.6) node[anchor=north west,align=left] {Riemann-Roch\\ theorems,\\ Chern\\ characters};
\draw (410.19000000000017, -34.6) rectangle (413.7900000000002,-36.7);
\draw(406.2400000000002, -36.8) node[anchor=north west,align=left] {Equivariant\\ \(K\)-theory};
\draw (406.2400000000002, -36.8) rectangle (409.8400000000002,-38.4);
\draw(414.4900000000002, -29.2) node[anchor=north west,align=left] {\large \(K\)-theory of forms};
\draw (414.4900000000002, -29.2) rectangle (423.14000000000016,-34.6);
\draw(415.4900000000002, -30.2) node[anchor=north west,align=left] {Hermitian \\ \(K\)-theory,\\ relations \\ with \(K\)-theory\\ of rings};
\draw (415.4900000000002, -30.2) rectangle (420.3400000000002,-32.8);
\draw(420.44000000000017, -30.2) node[anchor=north west,align=left] {Witt \\ groups\\ of rings};
\draw (420.44000000000017, -30.2) rectangle (423.0400000000002,-31.8);
\draw(415.4900000000002, -32.9) node[anchor=north west,align=left] {Stability\\ for quadratic\\ modules};
\draw (415.4900000000002, -32.9) rectangle (419.3400000000002,-34.5);
\draw(419.44000000000017, -32.9) node[anchor=north west,align=left] {\(L\)-theory\\ of \\ group rings};
\draw (419.44000000000017, -32.9) rectangle (423.0400000000002,-34.5);
\draw(394.8600000000002, -38.7) node[anchor=north west,align=left] {\LARGE Field theory and polynomials};
\draw (394.8600000000002, -38.7) rectangle (422.50000000000017,-65.60000000000001);
\draw(395.8600000000002, -39.7) node[anchor=north west,align=left] {\large Connections between field theory and logic};
\draw (395.8600000000002, -39.7) rectangle (411.1600000000002,-42.900000000000006);
\draw(396.8600000000002, -40.7) node[anchor=north west,align=left] {Ultraproducts\\ and \\ field theory};
\draw (396.8600000000002, -40.7) rectangle (400.7100000000002,-42.300000000000004);
\draw(400.8100000000002, -40.7) node[anchor=north west,align=left] {Decidability\\ and \\ field theory};
\draw (400.8100000000002, -40.7) rectangle (404.4100000000002,-42.300000000000004);
\draw(404.51000000000016, -40.7) node[anchor=north west,align=left] {Nonstandard\\ arithmetic\\ and\\ field theory};
\draw (404.51000000000016, -40.7) rectangle (408.1100000000002,-42.800000000000004);
\draw(408.2100000000002, -40.7) node[anchor=north west,align=left] {Model \\ theory \\ of fields};
\draw (408.2100000000002, -40.7) rectangle (411.06000000000023,-42.300000000000004);
\draw(411.26000000000016, -39.7) node[anchor=north west,align=left] {\large Homological methods (field theory)};
\draw (411.26000000000016, -39.7) rectangle (422.40000000000015,-42.900000000000006);
\draw(412.26000000000016, -40.7) node[anchor=north west,align=left] {Cohomological\\ dimension\\ of fields};
\draw (412.26000000000016, -40.7) rectangle (416.1100000000002,-42.800000000000004);
\draw(416.21000000000015, -40.7) node[anchor=north west,align=left] {Galois\\ cohomology};
\draw (416.21000000000015, -40.7) rectangle (419.3100000000002,-42.300000000000004);
\draw(395.8600000000002, -43.0) node[anchor=north west,align=left] {\large General field theory};
\draw (395.8600000000002, -43.0) rectangle (408.01000000000016,-50.6);
\draw(396.8600000000002, -44.0) node[anchor=north west,align=left] {Polynomials \\ in general \\ fields (irreducibility,\\ etc.)};
\draw (396.8600000000002, -44.0) rectangle (403.2100000000002,-46.1);
\draw(403.3100000000002, -44.0) node[anchor=north west,align=left] {Finite \\ fields \\ (field-theoretic\\ aspects)};
\draw (403.3100000000002, -44.0) rectangle (407.9100000000002,-46.1);
\draw(396.8600000000002, -46.2) node[anchor=north west,align=left] {Hilbertian \\ fields; Hilbert’s\\ irreducibility theorem};
\draw (396.8600000000002, -46.2) rectangle (402.9600000000002,-48.300000000000004);
\draw(403.0600000000002, -46.2) node[anchor=north west,align=left] {Skew fields,\\ division\\ rings};
\draw (403.0600000000002, -46.2) rectangle (406.6600000000002,-47.800000000000004);
\draw(396.8600000000002, -48.4) node[anchor=north west,align=left] {Special \\ polynomials\\ in general\\ fields};
\draw (396.8600000000002, -48.4) rectangle (400.2100000000002,-50.5);
\draw(400.3100000000002, -48.4) node[anchor=north west,align=left] {Equations\\ in general\\ fields};
\draw (400.3100000000002, -48.4) rectangle (403.4100000000002,-50.0);
\draw(403.51000000000016, -48.4) node[anchor=north west,align=left] {Field \\ arithmetic};
\draw (403.51000000000016, -48.4) rectangle (406.6100000000002,-49.5);
\draw(408.1100000000002, -43.0) node[anchor=north west,align=left] {\large Differential and difference algebra};
\draw (408.1100000000002, -43.0) rectangle (420.2100000000002,-47.9);
\draw(409.1100000000002, -44.0) node[anchor=north west,align=left] {Differential\\ algebra};
\draw (409.1100000000002, -44.0) rectangle (412.7100000000002,-45.6);
\draw(412.8100000000002, -44.0) node[anchor=north west,align=left] {Abstract \\ differential\\ equations};
\draw (412.8100000000002, -44.0) rectangle (416.4100000000002,-45.6);
\draw(416.51000000000016, -44.0) node[anchor=north west,align=left] {\(p\)-adic\\ differential\\ equations};
\draw (416.51000000000016, -44.0) rectangle (420.1100000000002,-46.1);
\draw(409.1100000000002, -46.2) node[anchor=north west,align=left] {Difference\\ algebra};
\draw (409.1100000000002, -46.2) rectangle (412.2100000000002,-47.800000000000004);
\draw(395.8600000000002, -50.7) node[anchor=north west,align=left] {\large Real and complex fields};
\draw (395.8600000000002, -50.7) rectangle (406.26000000000016,-57.6);
\draw(396.8600000000002, -51.7) node[anchor=north west,align=left] {Polynomials in\\ real and complex\\ fields: location\\ of zeros \\ (algebraic theorems)};
\draw (396.8600000000002, -51.7) rectangle (402.4600000000002,-54.300000000000004);
\draw(396.8600000000002, -54.400000000000006) node[anchor=north west,align=left] {Fields related\\ with sums of\\ squares (formally\\ real fields,\\ Pythagorean\\ fields, etc.)};
\draw (396.8600000000002, -54.400000000000006) rectangle (401.7100000000002,-57.50000000000001);
\draw(401.8100000000002, -54.400000000000006) node[anchor=north west,align=left] {Polynomials \\ in real and \\ complex fields:\\ factorization};
\draw (401.8100000000002, -54.400000000000006) rectangle (406.1600000000002,-56.50000000000001);
\draw(406.3600000000002, -50.7) node[anchor=north west,align=left] {\large Computational methods\\ for problems \\ pertaining to field theory};
\draw (406.3600000000002, -50.7) rectangle (415.0200000000002,-52.300000000000004);
\draw(406.3600000000002, -52.400000000000006) node[anchor=north west,align=left] {\large History of\\ field theory};
\draw (406.3600000000002, -52.400000000000006) rectangle (410.6800000000002,-53.50000000000001);
\draw(395.8600000000002, -57.7) node[anchor=north west,align=left] {\large Generalizations of fields};
\draw (395.8600000000002, -57.7) rectangle (404.2100000000002,-59.900000000000006);
\draw(396.8600000000002, -58.7) node[anchor=north west,align=left] {Near-fields};
\draw (396.8600000000002, -58.7) rectangle (400.2100000000002,-59.800000000000004);
\draw(400.3100000000002, -58.7) node[anchor=north west,align=left] {Semifields};
\draw (400.3100000000002, -58.7) rectangle (403.4100000000002,-59.800000000000004);
\draw(404.3100000000002, -57.7) node[anchor=north west,align=left] {\large Field extensions};
\draw (404.3100000000002, -57.7) rectangle (412.46000000000015,-64.3);
\draw(405.3100000000002, -58.7) node[anchor=north west,align=left] {Transcendental\\ field\\ extensions};
\draw (405.3100000000002, -58.7) rectangle (409.4100000000002,-60.300000000000004);
\draw(409.51000000000016, -58.7) node[anchor=north west,align=left] {Inverse\\ Galois\\ theory};
\draw (409.51000000000016, -58.7) rectangle (411.8600000000002,-60.300000000000004);
\draw(405.3100000000002, -60.400000000000006) node[anchor=north west,align=left] {Separable\\ extensions,\\ Galois theory};
\draw (405.3100000000002, -60.400000000000006) rectangle (409.1600000000002,-62.50000000000001);
\draw(409.26000000000016, -60.400000000000006) node[anchor=north west,align=left] {Algebraic\\ field\\ extensions};
\draw (409.26000000000016, -60.400000000000006) rectangle (412.3600000000002,-62.00000000000001);
\draw(405.3100000000002, -62.6) node[anchor=north west,align=left] {Inseparable\\ field\\ extensions};
\draw (405.3100000000002, -62.6) rectangle (408.6600000000002,-64.2);
\draw(412.5600000000002, -57.7) node[anchor=north west,align=left] {\large Topological fields};
\draw (412.5600000000002, -57.7) rectangle (420.71000000000015,-65.5);
\draw(413.5600000000002, -58.7) node[anchor=north west,align=left] {Non-Archimedean\\ valued fields};
\draw (413.5600000000002, -58.7) rectangle (417.9100000000002,-60.300000000000004);
\draw(418.01000000000016, -58.7) node[anchor=north west,align=left] {Ordered\\ fields};
\draw (418.01000000000016, -58.7) rectangle (420.3600000000002,-59.800000000000004);
\draw(413.5600000000002, -60.400000000000006) node[anchor=north west,align=left] {Krasner-Tate\\ algebras};
\draw (413.5600000000002, -60.400000000000006) rectangle (417.1600000000002,-62.00000000000001);
\draw(417.26000000000016, -60.400000000000006) node[anchor=north west,align=left] {Topological\\ semifields};
\draw (417.26000000000016, -60.400000000000006) rectangle (420.6100000000002,-62.00000000000001);
\draw(413.5600000000002, -62.1) node[anchor=north west,align=left] {Formally\\ \(p\)-adic\\ fields};
\draw (413.5600000000002, -62.1) rectangle (416.6600000000002,-63.7);
\draw(416.76000000000016, -62.1) node[anchor=north west,align=left] {General\\ valuation\\ theory\\ for fields};
\draw (416.76000000000016, -62.1) rectangle (419.8600000000002,-64.2);
\draw(413.5600000000002, -64.3) node[anchor=north west,align=left] {Normed\\ fields};
\draw (413.5600000000002, -64.3) rectangle (415.6600000000002,-65.39999999999999);
\draw(415.76000000000016, -64.3) node[anchor=north west,align=left] {Valued\\ fields};
\draw (415.76000000000016, -64.3) rectangle (417.8600000000002,-65.39999999999999);
\draw(394.8600000000002, -65.7) node[anchor=north west,align=left] {\LARGE Special functions};
\draw (394.8600000000002, -65.7) rectangle (422.3600000000002,-108.60000000000002);
\draw(395.8600000000002, -66.7) node[anchor=north west,align=left] {\large Basic hypergeometric functions};
\draw (395.8600000000002, -66.7) rectangle (410.26000000000016,-83.2);
\draw(396.8600000000002, -67.7) node[anchor=north west,align=left] {Connections of basic\\ hypergeometric \\ functions with quantum\\ groups, Chevalley\\ groups, \(p\)-adic\\ groups, Hecke \\ algebras, and related topics};
\draw (396.8600000000002, -67.7) rectangle (404.4600000000002,-71.3);
\draw(404.5600000000002, -67.7) node[anchor=north west,align=left] {Basic orthogonal\\ polynomials and\\ functions \\ associated with root\\ systems (Macdonald\\ polynomials, etc.)};
\draw (404.5600000000002, -67.7) rectangle (410.1600000000002,-70.8);
\draw(396.8600000000002, -71.4) node[anchor=north west,align=left] {Orthogonal polynomials\\ and functions\\ in several variables\\ expressible in\\ terms of basic \\ hypergeometric functions\\ in one variable};
\draw (396.8600000000002, -71.4) rectangle (403.4600000000002,-75.0);
\draw(403.5600000000002, -71.4) node[anchor=north west,align=left] {Basic orthogonal\\ polynomials \\ and functions \\ (Askey-Wilson \\ polynomials, etc.)};
\draw (403.5600000000002, -71.4) rectangle (408.6600000000002,-74.0);
\draw(396.8600000000002, -75.10000000000001) node[anchor=north west,align=left] {Basic \\ hypergeometric \\ functions \\ associated \\ with root systems};
\draw (396.8600000000002, -75.10000000000001) rectangle (401.7100000000002,-77.7);
\draw(401.8100000000002, -75.10000000000001) node[anchor=north west,align=left] {Other basic \\ hypergeometric \\ functions and \\ integrals in \\ several variables};
\draw (401.8100000000002, -75.10000000000001) rectangle (406.6600000000002,-77.7);
\draw(396.8600000000002, -77.80000000000001) node[anchor=north west,align=left] {Basic \\ hypergeometric \\ functions in one\\ variable,\\ \({}_r\phi_s\)};
\draw (396.8600000000002, -77.80000000000001) rectangle (401.4600000000002,-80.4);
\draw(401.5600000000002, -77.80000000000001) node[anchor=north west,align=left] {Basic \\ hypergeometric \\ integrals and \\ functions \\ defined by them};
\draw (401.5600000000002, -77.80000000000001) rectangle (405.9100000000002,-80.4);
\draw(406.01000000000016, -77.80000000000001) node[anchor=north west,align=left] {Applications\\ of basic\\ hypergeometric\\ functions};
\draw (406.01000000000016, -77.80000000000001) rectangle (410.1100000000002,-79.9);
\draw(396.8600000000002, -80.5) node[anchor=north west,align=left] {\(q\)-gamma\\ functions,\\ \(q\)-beta\\ functions\\ and integrals};
\draw (396.8600000000002, -80.5) rectangle (400.7100000000002,-83.1);
\draw(400.8100000000002, -80.5) node[anchor=north west,align=left] {Bibasic \\ functions\\ and multiple\\ bases};
\draw (400.8100000000002, -80.5) rectangle (404.4100000000002,-82.6);
\draw(410.3600000000002, -66.7) node[anchor=north west,align=left] {\large Other special functions};
\draw (410.3600000000002, -66.7) rectangle (422.26000000000016,-76.5);
\draw(411.3600000000002, -67.7) node[anchor=north west,align=left] {Painlevé-typefunctions};
\draw (411.3600000000002, -67.7) rectangle (417.4600000000002,-69.3);
\draw(417.5600000000002, -67.7) node[anchor=north west,align=left] {Lamé, Mathieu,\\ and \\ spheroidal wave\\ functions};
\draw (417.5600000000002, -67.7) rectangle (421.9100000000002,-69.8);
\draw(411.3600000000002, -69.9) node[anchor=north west,align=left] {Special functions\\ in \\ characteristic \(p\)\\ (gamma \\ functions, etc.)};
\draw (411.3600000000002, -69.9) rectangle (416.9600000000002,-72.5);
\draw(417.0600000000002, -69.9) node[anchor=north west,align=left] {Other functions\\ coming from \\ differential, \\ difference and \\ integral equations};
\draw (417.0600000000002, -69.9) rectangle (422.1600000000002,-72.5);
\draw(411.3600000000002, -72.60000000000001) node[anchor=north west,align=left] {Mittag-Leffler\\ functions\\ and \\ generalizations};
\draw (411.3600000000002, -72.60000000000001) rectangle (415.7100000000002,-74.7);
\draw(415.8100000000002, -72.60000000000001) node[anchor=north west,align=left] {Other functions\\ defined\\ by series\\ and integrals};
\draw (415.8100000000002, -72.60000000000001) rectangle (420.1600000000002,-74.7);
\draw(411.3600000000002, -74.80000000000001) node[anchor=north west,align=left] {Elliptic \\ functions \\ and integrals};
\draw (411.3600000000002, -74.80000000000001) rectangle (415.2100000000002,-76.4);
\draw(415.3100000000002, -74.80000000000001) node[anchor=north west,align=left] {Other\\ wave \\ functions};
\draw (415.3100000000002, -74.80000000000001) rectangle (418.1600000000002,-76.4);
\draw(395.8600000000002, -83.30000000000001) node[anchor=north west,align=left] {\large Hypergeometric functions};
\draw (395.8600000000002, -83.30000000000001) rectangle (409.51000000000016,-104.20000000000002);
\draw(396.8600000000002, -84.30000000000001) node[anchor=north west,align=left] {Orthogonal \\ polynomials and functions\\ in several\\ variables expressible\\ in terms \\ of special functions\\ in one variable};
\draw (396.8600000000002, -84.30000000000001) rectangle (403.7100000000002,-87.9);
\draw(403.8100000000002, -84.30000000000001) node[anchor=north west,align=left] {Confluent \\ hypergeometric \\ functions, \\ Whittaker functions,\\ \({}_1F_1\)};
\draw (403.8100000000002, -84.30000000000001) rectangle (409.4100000000002,-86.9);
\draw(396.8600000000002, -88.00000000000001) node[anchor=north west,align=left] {Other \\ hypergeometric functions\\ and \\ integrals in several\\ variables};
\draw (396.8600000000002, -88.00000000000001) rectangle (403.4600000000002,-90.60000000000001);
\draw(403.5600000000002, -88.00000000000001) node[anchor=north west,align=left] {Hypergeometric \\ integrals and \\ functions defined \\ by them (\(E\), \\ \(G\), \(H\) and\\ \(I\) functions)};
\draw (403.5600000000002, -88.00000000000001) rectangle (408.6600000000002,-91.10000000000001);
\draw(396.8600000000002, -91.20000000000002) node[anchor=north west,align=left] {Orthogonal polynomials\\ and functions\\ of hypergeometric\\ type (Jacobi,\\ Laguerre, Hermite,\\ Askey scheme, etc.)};
\draw (396.8600000000002, -91.20000000000002) rectangle (402.9600000000002,-94.30000000000001);
\draw(403.0600000000002, -91.20000000000002) node[anchor=north west,align=left] {Bessel and\\ Airy functions,\\ cylinder\\ functions,\\ \({}_0F_1\)};
\draw (403.0600000000002, -91.20000000000002) rectangle (407.4100000000002,-93.80000000000001);
\draw(396.8600000000002, -94.4) node[anchor=north west,align=left] {Orthogonal \\ polynomials and\\ functions \\ associated with\\ root systems};
\draw (396.8600000000002, -94.4) rectangle (401.2100000000002,-97.0);
\draw(401.3100000000002, -94.4) node[anchor=north west,align=left] {Hypergeometric\\ functions \\ associated with\\ root systems};
\draw (401.3100000000002, -94.4) rectangle (405.6600000000002,-96.5);
\draw(405.76000000000016, -94.4) node[anchor=north west,align=left] {Appell, \\ Horn and \\ Lauricella\\ functions};
\draw (405.76000000000016, -94.4) rectangle (408.8600000000002,-96.5);
\draw(396.8600000000002, -97.10000000000001) node[anchor=north west,align=left] {Connections of\\ hypergeometric\\ functions \\ with groups and\\ algebras, and\\ related topics};
\draw (396.8600000000002, -97.10000000000001) rectangle (401.2100000000002,-100.2);
\draw(401.3100000000002, -97.10000000000001) node[anchor=north west,align=left] {Classical \\ hypergeometric\\ functions,\\ \({}_2F_1\)};
\draw (401.3100000000002, -97.10000000000001) rectangle (405.4100000000002,-99.2);
\draw(405.51000000000016, -97.10000000000001) node[anchor=north west,align=left] {Other special\\ orthogonal\\ polynomials\\ and functions};
\draw (405.51000000000016, -97.10000000000001) rectangle (409.3600000000002,-99.2);
\draw(396.8600000000002, -100.30000000000001) node[anchor=north west,align=left] {Generalized\\ hypergeometric\\ series,\\ \({}_pF_q\)};
\draw (396.8600000000002, -100.30000000000001) rectangle (400.9600000000002,-102.4);
\draw(401.0600000000002, -100.30000000000001) node[anchor=north west,align=left] {Elliptic \\ integrals as\\ hypergeometric\\ functions};
\draw (401.0600000000002, -100.30000000000001) rectangle (405.1600000000002,-102.4);
\draw(405.26000000000016, -100.30000000000001) node[anchor=north west,align=left] {Applications\\ of \\ hypergeometric\\ functions};
\draw (405.26000000000016, -100.30000000000001) rectangle (409.3600000000002,-102.4);
\draw(396.8600000000002, -102.50000000000001) node[anchor=north west,align=left] {Spherical\\ harmonics};
\draw (396.8600000000002, -102.50000000000001) rectangle (399.7100000000002,-104.10000000000001);
\draw(409.6100000000002, -83.30000000000001) node[anchor=north west,align=left] {\large Elementary classical functions};
\draw (409.6100000000002, -83.30000000000001) rectangle (420.76000000000016,-89.70000000000002);
\draw(410.6100000000002, -84.30000000000001) node[anchor=north west,align=left] {Incomplete beta\\ and gamma \\ functions (error \\ functions, probability\\ integral,\\ Fresnel integrals)};
\draw (410.6100000000002, -84.30000000000001) rectangle (416.7100000000002,-87.4);
\draw(416.8100000000002, -84.30000000000001) node[anchor=north west,align=left] {Exponential\\ and \\ trigonometric\\ functions};
\draw (416.8100000000002, -84.30000000000001) rectangle (420.6600000000002,-86.4);
\draw(410.6100000000002, -87.50000000000001) node[anchor=north west,align=left] {Gamma, \\ beta and\\ polygamma\\ functions};
\draw (410.6100000000002, -87.50000000000001) rectangle (413.4600000000002,-89.60000000000001);
\draw(413.5600000000002, -87.50000000000001) node[anchor=north west,align=left] {Higher \\ logarithm\\ functions};
\draw (413.5600000000002, -87.50000000000001) rectangle (416.4100000000002,-89.10000000000001);
\draw(395.8600000000002, -104.30000000000001) node[anchor=north west,align=left] {\large Computational aspects of special functions};
\draw (395.8600000000002, -104.30000000000001) rectangle (409.4800000000002,-108.50000000000001);
\draw(396.8600000000002, -105.30000000000001) node[anchor=north west,align=left] {Numerical \\ approximation\\ and \\ evaluation of \\ special functions};
\draw (396.8600000000002, -105.30000000000001) rectangle (401.7100000000002,-107.9);
\draw(401.8100000000002, -105.30000000000001) node[anchor=north west,align=left] {Symbolic \\ computation of \\ special functions\\ (Gosper and\\ Zeilberger\\ algorithms, etc.)};
\draw (401.8100000000002, -105.30000000000001) rectangle (406.6600000000002,-108.4);
\draw(409.58000000000015, -104.30000000000001) node[anchor=north west,align=left] {\large History of \\ special functions};
\draw (409.58000000000015, -104.30000000000001) rectangle (415.45000000000016,-105.4);
\draw(394.8600000000002, -108.70000000000002) node[anchor=north west,align=left] {\LARGE Potential theory};
\draw (394.8600000000002, -108.70000000000002) rectangle (421.5600000000002,-139.8);
\draw(395.8600000000002, -109.70000000000002) node[anchor=north west,align=left] {\large Potential theory on fractals and metric spaces};
\draw (395.8600000000002, -109.70000000000002) rectangle (410.7200000000002,-112.90000000000002);
\draw(396.8600000000002, -110.70000000000002) node[anchor=north west,align=left] {Potential \\ theory on \\ fractals and\\ metric spaces};
\draw (396.8600000000002, -110.70000000000002) rectangle (400.7100000000002,-112.80000000000001);
\draw(410.82000000000016, -109.70000000000002) node[anchor=north west,align=left] {\large Axiomatic potential theory};
\draw (410.82000000000016, -109.70000000000002) rectangle (419.4800000000002,-112.40000000000002);
\draw(411.82000000000016, -110.70000000000002) node[anchor=north west,align=left] {Axiomatic\\ potential\\ theory};
\draw (411.82000000000016, -110.70000000000002) rectangle (414.6700000000002,-112.30000000000001);
\draw(395.8600000000002, -113.00000000000001) node[anchor=north west,align=left] {\large Generalizations of potential theory};
\draw (395.8600000000002, -113.00000000000001) rectangle (408.9600000000002,-122.10000000000001);
\draw(396.8600000000002, -114.00000000000001) node[anchor=north west,align=left] {Pluriharmonic\\ and \\ plurisubharmonic\\ functions};
\draw (396.8600000000002, -114.00000000000001) rectangle (401.4600000000002,-116.10000000000001);
\draw(401.5600000000002, -114.00000000000001) node[anchor=north west,align=left] {Harmonic, \\ subharmonic, \\ superharmonic\\ functions \\ on other spaces};
\draw (401.5600000000002, -114.00000000000001) rectangle (405.9100000000002,-116.60000000000001);
\draw(406.01000000000016, -114.00000000000001) node[anchor=north west,align=left] {Discrete\\ potential\\ theory};
\draw (406.01000000000016, -114.00000000000001) rectangle (408.8600000000002,-115.60000000000001);
\draw(396.8600000000002, -116.70000000000002) node[anchor=north west,align=left] {Fine potential\\ theory; \\ fine properties\\ of sets \\ and functions};
\draw (396.8600000000002, -116.70000000000002) rectangle (401.2100000000002,-119.30000000000001);
\draw(401.3100000000002, -116.70000000000002) node[anchor=north west,align=left] {Other \\ generalizations\\ (nonlinear\\ potential \\ theory, etc.)};
\draw (401.3100000000002, -116.70000000000002) rectangle (405.6600000000002,-119.30000000000001);
\draw(405.76000000000016, -116.70000000000002) node[anchor=north west,align=left] {Dirichlet\\ forms};
\draw (405.76000000000016, -116.70000000000002) rectangle (408.6100000000002,-117.80000000000001);
\draw(396.8600000000002, -119.40000000000002) node[anchor=north west,align=left] {Potentials\\ and \\ capacities on \\ other spaces};
\draw (396.8600000000002, -119.40000000000002) rectangle (400.9600000000002,-121.50000000000001);
\draw(401.0600000000002, -119.40000000000002) node[anchor=north west,align=left] {Potential \\ theory on \\ Riemannian \\ manifolds and\\ other spaces};
\draw (401.0600000000002, -119.40000000000002) rectangle (404.9100000000002,-122.00000000000001);
\draw(405.01000000000016, -119.40000000000002) node[anchor=north west,align=left] {Martin\\ boundary\\ theory};
\draw (405.01000000000016, -119.40000000000002) rectangle (407.6100000000002,-121.00000000000001);
\draw(409.0600000000002, -113.00000000000001) node[anchor=north west,align=left] {\large Two-dimensional potential theory};
\draw (409.0600000000002, -113.00000000000001) rectangle (421.46000000000015,-126.30000000000001);
\draw(410.0600000000002, -114.00000000000001) node[anchor=north west,align=left] {Connections of\\ harmonic functions\\ with \\ differential equations\\ in two dimensions};
\draw (410.0600000000002, -114.00000000000001) rectangle (416.1600000000002,-116.60000000000001);
\draw(416.26000000000016, -114.00000000000001) node[anchor=north west,align=left] {Potentials and \\ capacity, harmonic\\ measure, extremal\\ length and \\ related notions \\ in two dimensions};
\draw (416.26000000000016, -114.00000000000001) rectangle (421.3600000000002,-117.10000000000001);
\draw(410.0600000000002, -117.20000000000002) node[anchor=north west,align=left] {Boundary value\\ and inverse \\ problems for harmonic\\ functions \\ in two dimensions};
\draw (410.0600000000002, -117.20000000000002) rectangle (415.9100000000002,-119.80000000000001);
\draw(416.01000000000016, -117.20000000000002) node[anchor=north west,align=left] {Integral \\ representations, \\ integral operators,\\ integral equations\\ methods in\\ two dimensions};
\draw (416.01000000000016, -117.20000000000002) rectangle (421.3600000000002,-120.30000000000001);
\draw(410.0600000000002, -120.40000000000002) node[anchor=north west,align=left] {Biharmonic, \\ polyharmonic \\ functions and \\ equations, Poisson’s\\ equation \\ in two dimensions};
\draw (410.0600000000002, -120.40000000000002) rectangle (415.6600000000002,-123.50000000000001);
\draw(415.76000000000016, -120.40000000000002) node[anchor=north west,align=left] {Boundary behavior\\ (theorems\\ of Fatou type,\\ etc.) of \\ harmonic functions\\ in two dimensions};
\draw (415.76000000000016, -120.40000000000002) rectangle (420.8600000000002,-123.50000000000001);
\draw(410.0600000000002, -123.60000000000002) node[anchor=north west,align=left] {Harmonic, \\ subharmonic, \\ superharmonic\\ functions \\ in two dimensions};
\draw (410.0600000000002, -123.60000000000002) rectangle (414.9100000000002,-126.20000000000002);
\draw(395.8600000000002, -122.20000000000002) node[anchor=north west,align=left] {\large Computational methods\\ for problems pertaining\\ to potential theory};
\draw (395.8600000000002, -122.20000000000002) rectangle (403.5900000000002,-123.80000000000001);
\draw(395.8600000000002, -123.90000000000002) node[anchor=north west,align=left] {\large History of \\ potential theory};
\draw (395.8600000000002, -123.90000000000002) rectangle (401.4200000000002,-125.00000000000001);
\draw(395.8600000000002, -126.40000000000002) node[anchor=north west,align=left] {\large Higher-dimensional potential theory};
\draw (395.8600000000002, -126.40000000000002) rectangle (408.26000000000016,-139.70000000000002);
\draw(396.8600000000002, -127.40000000000002) node[anchor=north west,align=left] {Boundary value\\ and inverse \\ problems for harmonic\\ functions \\ in higher dimensions};
\draw (396.8600000000002, -127.40000000000002) rectangle (402.7100000000002,-130.00000000000003);
\draw(402.8100000000002, -127.40000000000002) node[anchor=north west,align=left] {Integral \\ representations, \\ integral operators,\\ integral equations\\ methods in\\ higher dimensions};
\draw (402.8100000000002, -127.40000000000002) rectangle (408.1600000000002,-130.50000000000003);
\draw(396.8600000000002, -130.60000000000002) node[anchor=north west,align=left] {Potentials and\\ capacities,\\ extremal \\ length and related\\ notions in\\ higher dimensions};
\draw (396.8600000000002, -130.60000000000002) rectangle (401.9600000000002,-133.70000000000002);
\draw(402.0600000000002, -130.60000000000002) node[anchor=north west,align=left] {Boundary \\ behavior of \\ harmonic functions\\ in higher\\ dimensions};
\draw (402.0600000000002, -130.60000000000002) rectangle (407.1600000000002,-133.20000000000002);
\draw(396.8600000000002, -133.8) node[anchor=north west,align=left] {Harmonic, \\ subharmonic, \\ superharmonic \\ functions in \\ higher dimensions};
\draw (396.8600000000002, -133.8) rectangle (401.7100000000002,-136.4);
\draw(401.8100000000002, -133.8) node[anchor=north west,align=left] {Biharmonic and\\ polyharmonic\\ equations and\\ functions in \\ higher dimensions};
\draw (401.8100000000002, -133.8) rectangle (406.6600000000002,-136.4);
\draw(396.8600000000002, -136.50000000000003) node[anchor=north west,align=left] {Connections \\ of harmonic \\ functions with\\ differential\\ equations in\\ higher dimensions};
\draw (396.8600000000002, -136.50000000000003) rectangle (401.7100000000002,-139.60000000000002);
\draw(423.4100000000002, -1) node[anchor=north west,align=left] {\LARGE Order, lattices, ordered algebraic structures};
\draw (423.4100000000002, -1) rectangle (449.3600000000002,-36.1);
\draw(424.4100000000002, -2) node[anchor=north west,align=left] {\large Modular lattices, complemented lattices};
\draw (424.4100000000002, -2) rectangle (439.0100000000002,-7.9);
\draw(425.4100000000002, -3) node[anchor=north west,align=left] {Complemented\\ lattices, \\ orthocomplemented\\ lattices\\ and posets};
\draw (425.4100000000002, -3) rectangle (430.2600000000002,-5.6);
\draw(430.3600000000002, -3) node[anchor=north west,align=left] {Complemented\\ modular lattices,\\ continuous\\ geometries};
\draw (430.3600000000002, -3) rectangle (435.2100000000002,-5.1);
\draw(435.3100000000002, -3) node[anchor=north west,align=left] {Modular \\ lattices,\\ Desarguesian\\ lattices};
\draw (435.3100000000002, -3) rectangle (438.9100000000002,-5.1);
\draw(425.4100000000002, -5.7) node[anchor=north west,align=left] {Semimodular\\ lattices,\\ geometric\\ lattices};
\draw (425.4100000000002, -5.7) rectangle (428.7600000000002,-7.800000000000001);
\draw(439.1100000000002, -2) node[anchor=north west,align=left] {\large Ordered structures};
\draw (439.1100000000002, -2) rectangle (449.26000000000016,-12.3);
\draw(440.1100000000002, -3) node[anchor=north west,align=left] {Ordered \\ topological \\ structures (aspects\\ of ordered\\ structures)};
\draw (440.1100000000002, -3) rectangle (445.4600000000002,-5.6);
\draw(445.5600000000002, -3) node[anchor=north west,align=left] {Ordered \\ semigroups \\ and monoids};
\draw (445.5600000000002, -3) rectangle (448.9100000000002,-4.6);
\draw(440.1100000000002, -5.7) node[anchor=north west,align=left] {Ordered \\ rings, algebras,\\ modules};
\draw (440.1100000000002, -5.7) rectangle (444.7100000000002,-7.300000000000001);
\draw(444.8100000000002, -5.7) node[anchor=north west,align=left] {Ordered \\ abelian groups,\\ Riesz \\ groups, ordered\\ linear spaces};
\draw (444.8100000000002, -5.7) rectangle (449.1600000000002,-8.3);
\draw(440.1100000000002, -8.4) node[anchor=north west,align=left] {BCK-algebras,\\ BCI-algebras\\ (aspects\\ of ordered\\ structures)};
\draw (440.1100000000002, -8.4) rectangle (443.9600000000002,-11.0);
\draw(444.0600000000002, -8.4) node[anchor=north west,align=left] {Quantales};
\draw (444.0600000000002, -8.4) rectangle (446.9100000000002,-9.5);
\draw(440.1100000000002, -11.100000000000001) node[anchor=north west,align=left] {Noether\\ lattices};
\draw (440.1100000000002, -11.100000000000001) rectangle (442.7100000000002,-12.200000000000001);
\draw(442.8100000000002, -11.100000000000001) node[anchor=north west,align=left] {Ordered\\ groups};
\draw (442.8100000000002, -11.100000000000001) rectangle (445.1600000000002,-12.200000000000001);
\draw(424.4100000000002, -8.0) node[anchor=north west,align=left] {\large Computational methods\\ for problems pertaining\\ to ordered structures};
\draw (424.4100000000002, -8.0) rectangle (432.1400000000002,-9.6);
\draw(424.4100000000002, -9.700000000000001) node[anchor=north west,align=left] {\large History of \\ ordered structures};
\draw (424.4100000000002, -9.700000000000001) rectangle (430.5900000000002,-10.8);
\draw(424.4100000000002, -12.4) node[anchor=north west,align=left] {\large Distributive lattices};
\draw (424.4100000000002, -12.4) rectangle (435.5600000000002,-24.9);
\draw(425.4100000000002, -13.4) node[anchor=north west,align=left] {De Morgan \\ algebras, Łukasiewicz\\ algebras\\ (lattice-theoretic\\ aspects)};
\draw (425.4100000000002, -13.4) rectangle (431.2600000000002,-16.0);
\draw(431.3600000000002, -13.4) node[anchor=north west,align=left] {Complete\\ distributivity};
\draw (431.3600000000002, -13.4) rectangle (435.4600000000002,-15.0);
\draw(425.4100000000002, -16.1) node[anchor=north west,align=left] {Pseudocomplemented\\ lattices};
\draw (425.4100000000002, -16.1) rectangle (430.5100000000002,-17.700000000000003);
\draw(430.6100000000002, -16.1) node[anchor=north west,align=left] {Structure and\\ representation\\ theory\\ of distributive\\ lattices};
\draw (430.6100000000002, -16.1) rectangle (434.9600000000002,-18.700000000000003);
\draw(425.4100000000002, -18.8) node[anchor=north west,align=left] {Heyting \\ algebras \\ (lattice-theoretic\\ aspects)};
\draw (425.4100000000002, -18.8) rectangle (430.5100000000002,-20.900000000000002);
\draw(430.6100000000002, -18.8) node[anchor=north west,align=left] {Other \\ generalizations\\ of distributive\\ lattices};
\draw (430.6100000000002, -18.8) rectangle (434.9600000000002,-20.900000000000002);
\draw(425.4100000000002, -21.0) node[anchor=north west,align=left] {Post algebras\\ (lattice-theoretic\\ aspects)};
\draw (425.4100000000002, -21.0) rectangle (430.5100000000002,-23.1);
\draw(430.6100000000002, -21.0) node[anchor=north west,align=left] {Fuzzy lattices\\ (soft \\ algebras) and \\ related topics};
\draw (430.6100000000002, -21.0) rectangle (434.7100000000002,-23.1);
\draw(425.4100000000002, -23.200000000000003) node[anchor=north west,align=left] {MV-algebras};
\draw (425.4100000000002, -23.200000000000003) rectangle (428.7600000000002,-24.300000000000004);
\draw(428.8600000000002, -23.200000000000003) node[anchor=north west,align=left] {Lattices\\ and\\ duality};
\draw (428.8600000000002, -23.200000000000003) rectangle (431.4600000000002,-24.800000000000004);
\draw(431.5600000000002, -23.200000000000003) node[anchor=north west,align=left] {Frames,\\ locales};
\draw (431.5600000000002, -23.200000000000003) rectangle (433.9100000000002,-24.300000000000004);
\draw(435.6600000000002, -12.4) node[anchor=north west,align=left] {\large Ordered sets};
\draw (435.6600000000002, -12.4) rectangle (446.3100000000002,-20.0);
\draw(436.6600000000002, -13.4) node[anchor=north west,align=left] {Galois \\ correspondences, closure\\ operators\\ (in relation \\ to ordered sets)};
\draw (436.6600000000002, -13.4) rectangle (443.2600000000002,-16.0);
\draw(443.3600000000002, -13.4) node[anchor=north west,align=left] {Algebraic\\ aspects\\ of posets};
\draw (443.3600000000002, -13.4) rectangle (446.2100000000002,-15.0);
\draw(436.6600000000002, -16.1) node[anchor=north west,align=left] {Generalizations\\ of \\ ordered sets};
\draw (436.6600000000002, -16.1) rectangle (441.0100000000002,-17.700000000000003);
\draw(441.1100000000002, -16.1) node[anchor=north west,align=left] {Combinatorics\\ of \\ partially\\ ordered sets};
\draw (441.1100000000002, -16.1) rectangle (444.9600000000002,-18.200000000000003);
\draw(436.6600000000002, -18.3) node[anchor=north west,align=left] {Semilattices};
\draw (436.6600000000002, -18.3) rectangle (440.2600000000002,-19.400000000000002);
\draw(440.3600000000002, -18.3) node[anchor=north west,align=left] {Partial\\ orders,\\ general};
\draw (440.3600000000002, -18.3) rectangle (442.7100000000002,-19.900000000000002);
\draw(442.8100000000002, -18.3) node[anchor=north west,align=left] {Total\\ orders};
\draw (442.8100000000002, -18.3) rectangle (444.9100000000002,-19.400000000000002);
\draw(424.4100000000002, -25.0) node[anchor=north west,align=left] {\large Boolean algebras (Boolean rings)};
\draw (424.4100000000002, -25.0) rectangle (435.0600000000002,-33.1);
\draw(425.4100000000002, -26.0) node[anchor=north west,align=left] {Generalizationsof\\ Boolean\\ algebras};
\draw (425.4100000000002, -26.0) rectangle (430.2600000000002,-28.1);
\draw(430.3600000000002, -26.0) node[anchor=north west,align=left] {Stone spaces\\ (Boolean spaces)\\ and related\\ structures};
\draw (430.3600000000002, -26.0) rectangle (434.9600000000002,-28.1);
\draw(425.4100000000002, -28.2) node[anchor=north west,align=left] {Boolean algebras\\ with additional\\ operations \\ (diagonalizable\\ algebras, etc.)};
\draw (425.4100000000002, -28.2) rectangle (430.0100000000002,-30.8);
\draw(430.1100000000002, -28.2) node[anchor=north west,align=left] {Ring-theoretic\\ properties\\ of Boolean\\ algebras};
\draw (430.1100000000002, -28.2) rectangle (434.2100000000002,-30.3);
\draw(425.4100000000002, -30.9) node[anchor=north west,align=left] {Chain \\ conditions,\\ complete\\ algebras};
\draw (425.4100000000002, -30.9) rectangle (428.7600000000002,-33.0);
\draw(428.8600000000002, -30.9) node[anchor=north west,align=left] {Structure\\ theory \\ of Boolean\\ algebras};
\draw (428.8600000000002, -30.9) rectangle (431.9600000000002,-33.0);
\draw(432.0600000000002, -30.9) node[anchor=north west,align=left] {Boolean\\ functions};
\draw (432.0600000000002, -30.9) rectangle (434.9100000000002,-32.5);
\draw(435.1600000000002, -25.0) node[anchor=north west,align=left] {\large Lattices};
\draw (435.1600000000002, -25.0) rectangle (445.0600000000002,-36.0);
\draw(436.1600000000002, -26.0) node[anchor=north west,align=left] {Topologicallattices};
\draw (436.1600000000002, -26.0) rectangle (441.5100000000002,-27.6);
\draw(441.6100000000002, -26.0) node[anchor=north west,align=left] {Structure\\ theory \\ of lattices};
\draw (441.6100000000002, -26.0) rectangle (444.9600000000002,-27.6);
\draw(436.1600000000002, -27.7) node[anchor=north west,align=left] {Generalizations\\ of\\ lattices};
\draw (436.1600000000002, -27.7) rectangle (440.5100000000002,-29.3);
\draw(440.6100000000002, -27.7) node[anchor=north west,align=left] {Representation\\ theory of\\ lattices};
\draw (440.6100000000002, -27.7) rectangle (444.7100000000002,-29.8);
\draw(436.1600000000002, -29.9) node[anchor=north west,align=left] {Free lattices,\\ projective\\ lattices,\\ word problems};
\draw (436.1600000000002, -29.9) rectangle (440.2600000000002,-32.0);
\draw(440.3600000000002, -29.9) node[anchor=north west,align=left] {Continuous\\ lattices and\\ posets, \\ applications};
\draw (440.3600000000002, -29.9) rectangle (443.9600000000002,-32.0);
\draw(436.1600000000002, -32.1) node[anchor=north west,align=left] {Complete \\ lattices, \\ completions};
\draw (436.1600000000002, -32.1) rectangle (439.5100000000002,-33.7);
\draw(439.6100000000002, -32.1) node[anchor=north west,align=left] {Lattice \\ ideals, \\ congruence\\ relations};
\draw (439.6100000000002, -32.1) rectangle (442.7100000000002,-34.2);
\draw(436.1600000000002, -34.3) node[anchor=north west,align=left] {Varieties\\ of \\ lattices};
\draw (436.1600000000002, -34.3) rectangle (439.0100000000002,-35.9);
\draw(423.4100000000002, -36.2) node[anchor=north west,align=left] {\LARGE Combinatorics};
\draw (423.4100000000002, -36.2) rectangle (446.6100000000002,-102.19999999999999);
\draw(424.4100000000002, -37.2) node[anchor=north west,align=left] {\large Graph theory};
\draw (424.4100000000002, -37.2) rectangle (437.0100000000002,-73.5);
\draw(425.4100000000002, -38.2) node[anchor=north west,align=left] {Isomorphism \\ problems in graph \\ theory (reconstruction\\ conjecture,\\ etc.) and \\ homomorphisms (subgraph\\ embedding, etc.)};
\draw (425.4100000000002, -38.2) rectangle (431.7600000000002,-41.800000000000004);
\draw(431.8600000000002, -38.2) node[anchor=north west,align=left] {Graphs and\\ abstract \\ algebra (groups,\\ rings,\\ fields, etc.)};
\draw (431.8600000000002, -38.2) rectangle (436.4600000000002,-40.800000000000004);
\draw(425.4100000000002, -41.900000000000006) node[anchor=north west,align=left] {Graph representations\\ (geometric\\ and \\ intersection \\ representations, etc.)};
\draw (425.4100000000002, -41.900000000000006) rectangle (431.5100000000002,-44.50000000000001);
\draw(431.6100000000002, -41.900000000000006) node[anchor=north west,align=left] {Games on \\ graphs \\ (graph-theoretic\\ aspects)};
\draw (431.6100000000002, -41.900000000000006) rectangle (436.2100000000002,-44.00000000000001);
\draw(425.4100000000002, -44.6) node[anchor=north west,align=left] {Edge subsets with\\ special properties\\ (factorization,\\ matching, \\ partitioning, covering\\ and packing, etc.)};
\draw (425.4100000000002, -44.6) rectangle (431.5100000000002,-47.7);
\draw(431.6100000000002, -44.6) node[anchor=north west,align=left] {Structural \\ characterization\\ of families\\ of graphs};
\draw (431.6100000000002, -44.6) rectangle (436.2100000000002,-46.7);
\draw(425.4100000000002, -47.800000000000004) node[anchor=north west,align=left] {Vertex subsets\\ with special\\ properties \\ (dominating sets,\\ independent \\ sets, cliques, etc.)};
\draw (425.4100000000002, -47.800000000000004) rectangle (431.0100000000002,-50.900000000000006);
\draw(431.1100000000002, -47.800000000000004) node[anchor=north west,align=left] {Graph \\ operations (line\\ graphs, \\ products, etc.)};
\draw (431.1100000000002, -47.800000000000004) rectangle (435.7100000000002,-49.900000000000006);
\draw(425.4100000000002, -51.0) node[anchor=north west,align=left] {Graph labelling\\ (graceful\\ graphs, \\ bandwidth, etc.)};
\draw (425.4100000000002, -51.0) rectangle (430.0100000000002,-53.1);
\draw(430.1100000000002, -51.0) node[anchor=north west,align=left] {Random \\ graphs \\ (graph-theoretic\\ aspects)};
\draw (430.1100000000002, -51.0) rectangle (434.7100000000002,-53.1);
\draw(434.8100000000002, -51.0) node[anchor=north west,align=left] {Graph\\ minors};
\draw (434.8100000000002, -51.0) rectangle (436.9100000000002,-52.1);
\draw(425.4100000000002, -53.2) node[anchor=north west,align=left] {Small world\\ graphs, complex\\ networks\\ (graph-theoretic\\ aspects)};
\draw (425.4100000000002, -53.2) rectangle (430.0100000000002,-55.800000000000004);
\draw(430.1100000000002, -53.2) node[anchor=north west,align=left] {Graph \\ algorithms \\ (graph-theoretic\\ aspects)};
\draw (430.1100000000002, -53.2) rectangle (434.7100000000002,-55.300000000000004);
\draw(434.8100000000002, -53.2) node[anchor=north west,align=left] {Trees};
\draw (434.8100000000002, -53.2) rectangle (436.6600000000002,-53.800000000000004);
\draw(425.4100000000002, -55.900000000000006) node[anchor=north west,align=left] {Graphical \\ indices (Wiener\\ index, Zagreb\\ index, Randić\\ index, etc.)};
\draw (425.4100000000002, -55.900000000000006) rectangle (429.7600000000002,-58.50000000000001);
\draw(429.8600000000002, -55.900000000000006) node[anchor=north west,align=left] {Planar graphs;\\ geometric\\ and topological\\ aspects\\ of graph theory};
\draw (429.8600000000002, -55.900000000000006) rectangle (434.2100000000002,-58.50000000000001);
\draw(434.3100000000002, -55.900000000000006) node[anchor=north west,align=left] {Distance\\ in\\ graphs};
\draw (434.3100000000002, -55.900000000000006) rectangle (436.9100000000002,-57.50000000000001);
\draw(425.4100000000002, -58.6) node[anchor=north west,align=left] {Eulerian \\ and Hamiltonian\\ graphs};
\draw (425.4100000000002, -58.6) rectangle (429.7600000000002,-60.2);
\draw(429.8600000000002, -58.6) node[anchor=north west,align=left] {Graphs and\\ linear algebra\\ (matrices,\\ eigenvalues,\\ etc.)};
\draw (429.8600000000002, -58.6) rectangle (433.9600000000002,-61.2);
\draw(434.0600000000002, -58.6) node[anchor=north west,align=left] {Flows \\ in graphs};
\draw (434.0600000000002, -58.6) rectangle (436.9100000000002,-59.7);
\draw(425.4100000000002, -61.3) node[anchor=north west,align=left] {Graph designs\\ and \\ isomorphic \\ decomposition};
\draw (425.4100000000002, -61.3) rectangle (429.2600000000002,-63.4);
\draw(429.3600000000002, -61.3) node[anchor=north west,align=left] {Fractional\\ graph \\ theory, fuzzy\\ graph theory};
\draw (429.3600000000002, -61.3) rectangle (433.2100000000002,-63.4);
\draw(433.3100000000002, -61.3) node[anchor=north west,align=left] {Signed \\ and weighted\\ graphs};
\draw (433.3100000000002, -61.3) rectangle (436.9100000000002,-62.9);
\draw(425.4100000000002, -63.5) node[anchor=north west,align=left] {Connectivity};
\draw (425.4100000000002, -63.5) rectangle (429.0100000000002,-64.6);
\draw(429.1100000000002, -63.5) node[anchor=north west,align=left] {Applications\\ of \\ graph theory};
\draw (429.1100000000002, -63.5) rectangle (432.7100000000002,-65.1);
\draw(432.8100000000002, -63.5) node[anchor=north west,align=left] {Coloring\\ of graphs\\ and \\ hypergraphs};
\draw (432.8100000000002, -63.5) rectangle (436.1600000000002,-65.6);
\draw(425.4100000000002, -65.7) node[anchor=north west,align=left] {Directed\\ graphs \\ (digraphs),\\ tournaments};
\draw (425.4100000000002, -65.7) rectangle (428.7600000000002,-67.8);
\draw(428.8600000000002, -65.7) node[anchor=north west,align=left] {Enumeration\\ in graph\\ theory};
\draw (428.8600000000002, -65.7) rectangle (432.2100000000002,-67.3);
\draw(432.3100000000002, -65.7) node[anchor=north west,align=left] {Graph\\ polynomials};
\draw (432.3100000000002, -65.7) rectangle (435.6600000000002,-67.3);
\draw(425.4100000000002, -67.9) node[anchor=north west,align=left] {Density\\ (toughness,\\ etc.)};
\draw (425.4100000000002, -67.9) rectangle (428.7600000000002,-69.5);
\draw(428.8600000000002, -67.9) node[anchor=north west,align=left] {Generalized\\ Ramsey\\ theory};
\draw (428.8600000000002, -67.9) rectangle (432.2100000000002,-69.5);
\draw(432.3100000000002, -67.9) node[anchor=north west,align=left] {Hypergraphs};
\draw (432.3100000000002, -67.9) rectangle (435.6600000000002,-69.0);
\draw(425.4100000000002, -69.6) node[anchor=north west,align=left] {Paths and\\ cycles};
\draw (425.4100000000002, -69.6) rectangle (428.2600000000002,-70.69999999999999);
\draw(428.3600000000002, -69.6) node[anchor=north west,align=left] {Random\\ walks \\ on graphs};
\draw (428.3600000000002, -69.6) rectangle (431.2100000000002,-71.19999999999999);
\draw(431.3100000000002, -69.6) node[anchor=north west,align=left] {Extremal\\ problems\\ in graph\\ theory};
\draw (431.3100000000002, -69.6) rectangle (433.9100000000002,-71.69999999999999);
\draw(434.0100000000002, -69.6) node[anchor=north west,align=left] {Expander\\ graphs};
\draw (434.0100000000002, -69.6) rectangle (436.61000000000024,-70.69999999999999);
\draw(425.4100000000002, -71.80000000000001) node[anchor=north west,align=left] {Infinite\\ graphs};
\draw (425.4100000000002, -71.80000000000001) rectangle (428.0100000000002,-72.9);
\draw(428.1100000000002, -71.80000000000001) node[anchor=north west,align=left] {Chemical\\ graph\\ theory};
\draw (428.1100000000002, -71.80000000000001) rectangle (430.7100000000002,-73.4);
\draw(430.8100000000002, -71.80000000000001) node[anchor=north west,align=left] {Vertex\\ degrees};
\draw (430.8100000000002, -71.80000000000001) rectangle (433.1600000000002,-72.9);
\draw(433.2600000000002, -71.80000000000001) node[anchor=north west,align=left] {Perfect\\ graphs};
\draw (433.2600000000002, -71.80000000000001) rectangle (435.61000000000024,-72.9);
\draw(437.1100000000002, -37.2) node[anchor=north west,align=left] {\large Enumerative combinatorics};
\draw (437.1100000000002, -37.2) rectangle (446.51000000000016,-49.2);
\draw(438.1100000000002, -38.2) node[anchor=north west,align=left] {Exact enumeration\\ problems,\\ generating\\ functions};
\draw (438.1100000000002, -38.2) rectangle (442.9600000000002,-40.300000000000004);
\draw(443.0600000000002, -38.2) node[anchor=north west,align=left] {Asymptotic\\ enumeration};
\draw (443.0600000000002, -38.2) rectangle (446.4100000000002,-39.800000000000004);
\draw(438.1100000000002, -40.400000000000006) node[anchor=north west,align=left] {\(q\)-calculus\\ and related\\ topics};
\draw (438.1100000000002, -40.400000000000006) rectangle (442.2100000000002,-42.50000000000001);
\draw(442.3100000000002, -40.400000000000006) node[anchor=north west,align=left] {Permutations,\\ words,\\ matrices};
\draw (442.3100000000002, -40.400000000000006) rectangle (446.1600000000002,-42.00000000000001);
\draw(438.1100000000002, -42.6) node[anchor=north west,align=left] {Factorials,\\ binomial \\ coefficients,\\ combinatorial\\ functions};
\draw (438.1100000000002, -42.6) rectangle (441.9600000000002,-45.2);
\draw(442.0600000000002, -42.6) node[anchor=north west,align=left] {Combinatorial\\ aspects\\ of partitions\\ of integers};
\draw (442.0600000000002, -42.6) rectangle (445.9100000000002,-44.7);
\draw(438.1100000000002, -45.300000000000004) node[anchor=north west,align=left] {Combinatorial\\ identities,\\ bijective\\ combinatorics};
\draw (438.1100000000002, -45.300000000000004) rectangle (441.9600000000002,-47.400000000000006);
\draw(442.0600000000002, -45.300000000000004) node[anchor=north west,align=left] {Combinatorial\\ inequalities};
\draw (442.0600000000002, -45.300000000000004) rectangle (445.9100000000002,-46.900000000000006);
\draw(438.1100000000002, -47.5) node[anchor=north west,align=left] {Partitions\\ of sets};
\draw (438.1100000000002, -47.5) rectangle (441.2100000000002,-49.1);
\draw(441.3100000000002, -47.5) node[anchor=north west,align=left] {Umbral\\ calculus};
\draw (441.3100000000002, -47.5) rectangle (443.9100000000002,-48.6);
\draw(424.4100000000002, -73.60000000000001) node[anchor=north west,align=left] {\large Designs and configurations};
\draw (424.4100000000002, -73.60000000000001) rectangle (435.0600000000002,-88.80000000000001);
\draw(425.4100000000002, -74.60000000000001) node[anchor=north west,align=left] {Combinatorial\\ aspects of \\ difference sets\\ (number-theoretic,\\ group-theoretic, etc.)};
\draw (425.4100000000002, -74.60000000000001) rectangle (431.5100000000002,-77.7);
\draw(431.6100000000002, -74.60000000000001) node[anchor=north west,align=left] {Polyominoes};
\draw (431.6100000000002, -74.60000000000001) rectangle (434.9600000000002,-75.7);
\draw(425.4100000000002, -77.80000000000001) node[anchor=north west,align=left] {Combinatorial\\ aspects of\\ matrices \\ (incidence, \\ Hadamard, etc.)};
\draw (425.4100000000002, -77.80000000000001) rectangle (429.7600000000002,-80.4);
\draw(429.8600000000002, -77.80000000000001) node[anchor=north west,align=left] {Combinatorial\\ aspects\\ of tessellation\\ and \\ tiling problems};
\draw (429.8600000000002, -77.80000000000001) rectangle (434.2100000000002,-80.4);
\draw(425.4100000000002, -80.50000000000001) node[anchor=north west,align=left] {Other \\ designs, \\ configurations};
\draw (425.4100000000002, -80.50000000000001) rectangle (429.5100000000002,-82.10000000000001);
\draw(429.6100000000002, -80.50000000000001) node[anchor=north west,align=left] {Combinatorial\\ aspects\\ of \\ block designs};
\draw (429.6100000000002, -80.50000000000001) rectangle (433.4600000000002,-82.60000000000001);
\draw(425.4100000000002, -82.70000000000002) node[anchor=north west,align=left] {Orthogonal\\ arrays, Latin\\ squares,\\ Room squares};
\draw (425.4100000000002, -82.70000000000002) rectangle (429.2600000000002,-84.80000000000001);
\draw(429.3600000000002, -82.70000000000002) node[anchor=north west,align=left] {Combinatorial\\ aspects\\ of finite\\ geometries};
\draw (429.3600000000002, -82.70000000000002) rectangle (433.2100000000002,-84.80000000000001);
\draw(425.4100000000002, -84.9) node[anchor=north west,align=left] {Combinatorial\\ aspects\\ of matroids\\ and geometric\\ lattices};
\draw (425.4100000000002, -84.9) rectangle (429.2600000000002,-87.5);
\draw(429.3600000000002, -84.9) node[anchor=north west,align=left] {Combinatorial\\ aspects\\ of packing\\ and covering};
\draw (429.3600000000002, -84.9) rectangle (433.2100000000002,-87.0);
\draw(425.4100000000002, -87.60000000000001) node[anchor=north west,align=left] {Triple\\ systems};
\draw (425.4100000000002, -87.60000000000001) rectangle (427.7600000000002,-88.7);
\draw(435.1600000000002, -73.60000000000001) node[anchor=north west,align=left] {\large Extremal combinatorics};
\draw (435.1600000000002, -73.60000000000001) rectangle (444.5600000000002,-80.00000000000001);
\draw(436.1600000000002, -74.60000000000001) node[anchor=north west,align=left] {Probabilistic \\ methods in extremal\\ combinatorics,\\ including \\ polynomial methods \\ (combinatorial \\ Nullstellensatz, etc.)};
\draw (436.1600000000002, -74.60000000000001) rectangle (442.2600000000002,-78.2);
\draw(442.3600000000002, -74.60000000000001) node[anchor=north west,align=left] {Ramsey\\ theory};
\draw (442.3600000000002, -74.60000000000001) rectangle (444.4600000000002,-75.7);
\draw(436.1600000000002, -78.30000000000001) node[anchor=north west,align=left] {Transversal\\ (matching)\\ theory};
\draw (436.1600000000002, -78.30000000000001) rectangle (439.5100000000002,-79.9);
\draw(439.6100000000002, -78.30000000000001) node[anchor=north west,align=left] {Extremal\\ set\\ theory};
\draw (439.6100000000002, -78.30000000000001) rectangle (442.2100000000002,-79.9);
\draw(424.4100000000002, -88.9) node[anchor=north west,align=left] {\large Algebraic combinatorics};
\draw (424.4100000000002, -88.9) rectangle (433.8100000000002,-100.9);
\draw(425.4100000000002, -89.9) node[anchor=north west,align=left] {Combinatorial\\ aspects\\ of representation\\ theory};
\draw (425.4100000000002, -89.9) rectangle (430.2600000000002,-92.0);
\draw(425.4100000000002, -92.10000000000001) node[anchor=north west,align=left] {Symmetric\\ functions\\ and \\ generalizations};
\draw (425.4100000000002, -92.10000000000001) rectangle (429.7600000000002,-94.2);
\draw(429.8600000000002, -92.10000000000001) node[anchor=north west,align=left] {Combinatorial\\ aspects\\ of algebraic\\ geometry};
\draw (429.8600000000002, -92.10000000000001) rectangle (433.7100000000002,-94.2);
\draw(425.4100000000002, -94.30000000000001) node[anchor=north west,align=left] {Association\\ schemes, \\ strongly \\ regular graphs};
\draw (425.4100000000002, -94.30000000000001) rectangle (429.5100000000002,-96.4);
\draw(429.6100000000002, -94.30000000000001) node[anchor=north west,align=left] {Combinatorial\\ aspects\\ of commutative\\ algebra};
\draw (429.6100000000002, -94.30000000000001) rectangle (433.7100000000002,-96.4);
\draw(425.4100000000002, -96.5) node[anchor=north west,align=left] {Combinatorial\\ aspects\\ of groups \\ and algebras};
\draw (425.4100000000002, -96.5) rectangle (429.2600000000002,-98.6);
\draw(429.3600000000002, -96.5) node[anchor=north west,align=left] {Group actions\\ on \\ combinatorial\\ structures};
\draw (429.3600000000002, -96.5) rectangle (433.2100000000002,-98.6);
\draw(425.4100000000002, -98.7) node[anchor=north west,align=left] {Combinatorial\\ aspects\\ of simplicial\\ complexes};
\draw (425.4100000000002, -98.7) rectangle (429.2600000000002,-100.8);
\draw(433.9100000000002, -88.9) node[anchor=north west,align=left] {\large Computational methods\\ for problems \\ pertaining to combinatorics};
\draw (433.9100000000002, -88.9) rectangle (442.8800000000002,-90.5);
\draw(424.4100000000002, -101.0) node[anchor=north west,align=left] {\large History of \\ combinatorics};
\draw (424.4100000000002, -101.0) rectangle (429.0400000000002,-102.1);
\draw(423.4100000000002, -102.29999999999998) node[anchor=north west,align=left] {\LARGE Linear and multilinear algebra; matrix theory};
\draw (423.4100000000002, -102.29999999999998) rectangle (443.4500000000002,-148.79999999999998);
\draw(424.4100000000002, -103.29999999999998) node[anchor=north west,align=left] {\large Basic linear algebra};
\draw (424.4100000000002, -103.29999999999998) rectangle (435.5600000000002,-136.09999999999997);
\draw(425.4100000000002, -104.29999999999998) node[anchor=north west,align=left] {Theory of \\ matrix inversion\\ and \\ generalized inverses};
\draw (425.4100000000002, -104.29999999999998) rectangle (431.0100000000002,-106.39999999999998);
\draw(431.1100000000002, -104.29999999999998) node[anchor=north west,align=left] {Vector and\\ tensor \\ algebra, theory\\ of invariants};
\draw (431.1100000000002, -104.29999999999998) rectangle (435.4600000000002,-106.39999999999998);
\draw(425.4100000000002, -106.49999999999999) node[anchor=north west,align=left] {Norms of matrices,\\ numerical\\ range, \\ applications of \\ functional analysis\\ to matrix theory};
\draw (425.4100000000002, -106.49999999999999) rectangle (430.7600000000002,-109.59999999999998);
\draw(430.8600000000002, -106.49999999999999) node[anchor=north west,align=left] {Linear \\ transformations,\\ semilinear \\ transformations};
\draw (430.8600000000002, -106.49999999999999) rectangle (435.4600000000002,-108.59999999999998);
\draw(425.4100000000002, -109.69999999999999) node[anchor=north west,align=left] {Matrix exponential\\ and \\ similar functions\\ of matrices};
\draw (425.4100000000002, -109.69999999999999) rectangle (430.5100000000002,-111.79999999999998);
\draw(430.6100000000002, -109.69999999999999) node[anchor=north west,align=left] {Vector spaces,\\ linear \\ dependence, \\ rank, lineability};
\draw (430.6100000000002, -109.69999999999999) rectangle (435.4600000000002,-111.79999999999998);
\draw(425.4100000000002, -111.89999999999998) node[anchor=north west,align=left] {Linear \\ equations \\ (linear algebraic\\ aspects)};
\draw (425.4100000000002, -111.89999999999998) rectangle (430.2600000000002,-113.99999999999997);
\draw(430.3600000000002, -111.89999999999998) node[anchor=north west,align=left] {Determinants,\\ permanents,\\ traces, \\ other special\\ matrix functions};
\draw (430.3600000000002, -111.89999999999998) rectangle (434.9600000000002,-114.49999999999997);
\draw(425.4100000000002, -114.59999999999998) node[anchor=north west,align=left] {Diagonalization,\\ Jordan forms};
\draw (425.4100000000002, -114.59999999999998) rectangle (430.0100000000002,-116.19999999999997);
\draw(430.1100000000002, -114.59999999999998) node[anchor=north west,align=left] {Inequalities\\ involving \\ eigenvalues \\ and eigenvectors};
\draw (430.1100000000002, -114.59999999999998) rectangle (434.7100000000002,-116.69999999999997);
\draw(425.4100000000002, -116.79999999999998) node[anchor=north west,align=left] {Canonical\\ forms, \\ reductions, \\ classification};
\draw (425.4100000000002, -116.79999999999998) rectangle (429.5100000000002,-118.89999999999998);
\draw(429.6100000000002, -116.79999999999998) node[anchor=north west,align=left] {Matrices over\\ function rings\\ in one or\\ more variables};
\draw (429.6100000000002, -116.79999999999998) rectangle (433.7100000000002,-118.89999999999998);
\draw(425.4100000000002, -118.99999999999999) node[anchor=north west,align=left] {Factorization\\ of\\ matrices};
\draw (425.4100000000002, -118.99999999999999) rectangle (429.2600000000002,-120.59999999999998);
\draw(429.3600000000002, -118.99999999999999) node[anchor=north west,align=left] {Matrix \\ equations and\\ identities};
\draw (429.3600000000002, -118.99999999999999) rectangle (433.2100000000002,-120.59999999999998);
\draw(425.4100000000002, -120.69999999999999) node[anchor=north west,align=left] {Commutativity\\ of\\ matrices};
\draw (425.4100000000002, -120.69999999999999) rectangle (429.2600000000002,-122.29999999999998);
\draw(429.3600000000002, -120.69999999999999) node[anchor=north west,align=left] {Miscellaneous\\ inequalities\\ involving\\ matrices};
\draw (429.3600000000002, -120.69999999999999) rectangle (433.2100000000002,-122.79999999999998);
\draw(425.4100000000002, -122.89999999999998) node[anchor=north west,align=left] {Applications\\ of Clifford\\ algebras to\\ physics, etc.};
\draw (425.4100000000002, -122.89999999999998) rectangle (429.2600000000002,-124.99999999999997);
\draw(429.3600000000002, -122.89999999999998) node[anchor=north west,align=left] {Applications\\ of \\ generalized\\ inverses};
\draw (429.3600000000002, -122.89999999999998) rectangle (432.9600000000002,-124.99999999999997);
\draw(433.0600000000002, -122.89999999999998) node[anchor=north west,align=left] {Matrix\\ pencils};
\draw (433.0600000000002, -122.89999999999998) rectangle (435.4100000000002,-123.99999999999997);
\draw(425.4100000000002, -125.09999999999998) node[anchor=north west,align=left] {Conditioning\\ of\\ matrices};
\draw (425.4100000000002, -125.09999999999998) rectangle (429.0100000000002,-126.69999999999997);
\draw(429.1100000000002, -125.09999999999998) node[anchor=north west,align=left] {Eigenvalues,\\ singular\\ values, and\\ eigenvectors};
\draw (429.1100000000002, -125.09999999999998) rectangle (432.7100000000002,-127.19999999999997);
\draw(425.4100000000002, -127.29999999999998) node[anchor=north west,align=left] {Linear \\ inequalities\\ of matrices};
\draw (425.4100000000002, -127.29999999999998) rectangle (429.0100000000002,-128.89999999999998);
\draw(429.1100000000002, -127.29999999999998) node[anchor=north west,align=left] {Quadratic \\ and bilinear\\ forms, inner\\ products};
\draw (429.1100000000002, -127.29999999999998) rectangle (432.7100000000002,-129.39999999999998);
\draw(425.4100000000002, -129.49999999999997) node[anchor=north west,align=left] {Algebraic\\ systems\\ of matrices};
\draw (425.4100000000002, -129.49999999999997) rectangle (428.7600000000002,-131.09999999999997);
\draw(428.8600000000002, -129.49999999999997) node[anchor=north west,align=left] {Multilinear\\ algebra,\\ tensor\\ calculus};
\draw (428.8600000000002, -129.49999999999997) rectangle (432.2100000000002,-131.59999999999997);
\draw(432.3100000000002, -129.49999999999997) node[anchor=north west,align=left] {Other \\ algebras \\ built from\\ modules};
\draw (432.3100000000002, -129.49999999999997) rectangle (435.4100000000002,-131.59999999999997);
\draw(425.4100000000002, -131.7) node[anchor=north west,align=left] {Max-plus\\ and related\\ algebras};
\draw (425.4100000000002, -131.7) rectangle (428.7600000000002,-133.29999999999998);
\draw(428.8600000000002, -131.7) node[anchor=north west,align=left] {Matrix \\ completion\\ problems};
\draw (428.8600000000002, -131.7) rectangle (431.9600000000002,-133.29999999999998);
\draw(432.0600000000002, -131.7) node[anchor=north west,align=left] {Inverse\\ problems\\ in linear\\ algebra};
\draw (432.0600000000002, -131.7) rectangle (434.9100000000002,-133.79999999999998);
\draw(425.4100000000002, -133.89999999999998) node[anchor=north west,align=left] {Clifford\\ algebras,\\ spinors};
\draw (425.4100000000002, -133.89999999999998) rectangle (428.2600000000002,-135.49999999999997);
\draw(428.3600000000002, -133.89999999999998) node[anchor=north west,align=left] {Exterior\\ algebra,\\ Grassmann\\ algebras};
\draw (428.3600000000002, -133.89999999999998) rectangle (431.2100000000002,-135.99999999999997);
\draw(431.3100000000002, -133.89999999999998) node[anchor=north west,align=left] {Linear \\ preserver\\ problems};
\draw (431.3100000000002, -133.89999999999998) rectangle (434.1600000000002,-135.49999999999997);
\draw(435.6600000000002, -103.29999999999998) node[anchor=north west,align=left] {\large History of \\ linear algebra};
\draw (435.6600000000002, -103.29999999999998) rectangle (440.6000000000002,-104.39999999999998);
\draw(424.4100000000002, -136.2) node[anchor=north west,align=left] {\large Special matrices};
\draw (424.4100000000002, -136.2) rectangle (435.5600000000002,-148.7);
\draw(425.4100000000002, -137.2) node[anchor=north west,align=left] {Positive \\ matrices and \\ their \\ generalizations; cones\\ of matrices};
\draw (425.4100000000002, -137.2) rectangle (431.5100000000002,-139.79999999999998);
\draw(431.6100000000002, -137.2) node[anchor=north west,align=left] {Boolean \\ and Hadamard\\ matrices};
\draw (431.6100000000002, -137.2) rectangle (435.2100000000002,-138.79999999999998);
\draw(425.4100000000002, -139.89999999999998) node[anchor=north west,align=left] {Matrices over\\ special \\ rings (quaternions,\\ finite\\ fields, etc.)};
\draw (425.4100000000002, -139.89999999999998) rectangle (430.7600000000002,-142.49999999999997);
\draw(430.8600000000002, -139.89999999999998) node[anchor=north west,align=left] {Hermitian,\\ skew-Hermitian,\\ and \\ related matrices};
\draw (430.8600000000002, -139.89999999999998) rectangle (435.4600000000002,-141.99999999999997);
\draw(425.4100000000002, -142.6) node[anchor=north west,align=left] {Toeplitz,\\ Cauchy,\\ and related\\ matrices};
\draw (425.4100000000002, -142.6) rectangle (428.7600000000002,-144.7);
\draw(428.8600000000002, -142.6) node[anchor=north west,align=left] {Orthogonal\\ matrices};
\draw (428.8600000000002, -142.6) rectangle (431.9600000000002,-144.2);
\draw(432.0600000000002, -142.6) node[anchor=north west,align=left] {Stochastic\\ matrices};
\draw (432.0600000000002, -142.6) rectangle (435.1600000000002,-144.2);
\draw(425.4100000000002, -144.79999999999998) node[anchor=north west,align=left] {Random \\ matrices\\ (algebraic\\ aspects)};
\draw (425.4100000000002, -144.79999999999998) rectangle (428.5100000000002,-146.89999999999998);
\draw(428.6100000000002, -144.79999999999998) node[anchor=north west,align=left] {Fuzzy \\ matrices};
\draw (428.6100000000002, -144.79999999999998) rectangle (431.2100000000002,-145.89999999999998);
\draw(431.3100000000002, -144.79999999999998) node[anchor=north west,align=left] {Matrix\\ Lie \\ algebras};
\draw (431.3100000000002, -144.79999999999998) rectangle (433.9100000000002,-146.39999999999998);
\draw(425.4100000000002, -147.0) node[anchor=north west,align=left] {Sign \\ pattern \\ matrices};
\draw (425.4100000000002, -147.0) rectangle (428.0100000000002,-148.6);
\draw(428.1100000000002, -147.0) node[anchor=north west,align=left] {Matrices\\ of\\ integers};
\draw (428.1100000000002, -147.0) rectangle (430.7100000000002,-148.6);
\end{tikzpicture}

\end{document}