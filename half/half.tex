\documentclass[12pt]{article}
\usepackage[utf8]{inputenc}
\usepackage{pgf,tikz,pgfplots}
\pgfplotsset{compat=1.15}
\usepackage{mathrsfs}
\usetikzlibrary{arrows}
\usepackage{fontspec}
\setmainfont[Renderer=ICU,Mapping=tex-text]{Cousine}
\usepackage{amssymb}
\usepackage[paperwidth=395.0500000000001cm,paperheight=141.29999999999998cm,left=0.1cm,right=0.1cm,top=0.1cm,bottom=0.1cm]{geometry}
\begin{document}\begin{tikzpicture}[line cap=round,line join=round,>=triangle 45,x=1cm,y=1cm]
\clip(0, 0)rectangle(391.0500000000001, -137.29999999999998);

\draw(0, 0) node[anchor=north west,align=left] {\Huge half};
\draw (0, 0) rectangle (391.0500000000001,-137.29999999999998);
\draw(1, -1) node[anchor=north west,align=left] {\LARGE Partial differential equations};
\draw (1, -1) rectangle (79.79999999999998,-46.0);
\draw(2, -2) node[anchor=north west,align=left] {\large General higher-order partial differential equations and systems of higher-order partial differential equations};
\draw (2, -2) rectangle (43.300000000000004,-8.4);
\draw(3, -3) node[anchor=north west,align=left] {Systems \\ of nonlinear\\ higher-order PDEs};
\draw (3, -3) rectangle (8.1,-5.1);
\draw(8.2, -3) node[anchor=north west,align=left] {Nonlinear\\ higher-order PDEs};
\draw (8.2, -3) rectangle (13.049999999999999,-4.6);
\draw(13.149999999999999, -3) node[anchor=north west,align=left] {Systems\\ of linear\\ higher-order PDEs};
\draw (13.149999999999999, -3) rectangle (18.0,-5.1);
\draw(18.099999999999998, -3) node[anchor=north west,align=left] {Initial-boundary\\ value problems\\ for systems\\ of nonlinear\\ higher-order PDEs};
\draw (18.099999999999998, -3) rectangle (22.699999999999996,-5.6);
\draw(22.799999999999997, -3) node[anchor=north west,align=left] {Initial value\\ problems \\ for systems \\ of nonlinear\\ higher-order PDEs};
\draw (22.799999999999997, -3) rectangle (27.15,-5.6);
\draw(27.249999999999996, -3) node[anchor=north west,align=left] {Boundary value\\ problems\\ for linear \\ higher-order PDEs};
\draw (27.249999999999996, -3) rectangle (31.349999999999994,-5.1);
\draw(31.449999999999996, -3) node[anchor=north west,align=left] {Initial-boundary\\ value\\ problems \\ for linear \\ higher-order PDEs};
\draw (31.449999999999996, -3) rectangle (35.55,-5.6);
\draw(35.65, -3) node[anchor=north west,align=left] {Initial value\\ problems \\ for linear \\ higher-order PDEs};
\draw (35.65, -3) rectangle (39.5,-5.1);
\draw(39.6, -3) node[anchor=north west,align=left] {Initial value\\ problems \\ for systems \\ of linear \\ higher-order PDEs};
\draw (39.6, -3) rectangle (43.2,-5.6);
\draw(3, -5.7) node[anchor=north west,align=left] {Initial value\\ problems for\\ nonlinear \\ higher-order PDEs};
\draw (3, -5.7) rectangle (6.85,-7.800000000000001);
\draw(6.949999999999999, -5.7) node[anchor=north west,align=left] {Boundary \\ value problems\\ for \\ nonlinear \\ higher-order PDEs};
\draw (6.949999999999999, -5.7) rectangle (10.799999999999999,-8.3);
\draw(10.9, -5.7) node[anchor=north west,align=left] {Initial-boundary\\ value \\ problems for\\ nonlinear \\ higher-order PDEs};
\draw (10.9, -5.7) rectangle (14.75,-8.3);
\draw(14.85, -5.7) node[anchor=north west,align=left] {Boundary value\\ problems\\ for systems\\ of linear \\ higher-order PDEs};
\draw (14.85, -5.7) rectangle (18.7,-8.3);
\draw(18.8, -5.7) node[anchor=north west,align=left] {Initial-boundary\\ value problems\\ for systems\\ of linear \\ higher-order PDEs};
\draw (18.8, -5.7) rectangle (22.650000000000002,-8.3);
\draw(22.75, -5.7) node[anchor=north west,align=left] {Boundary value\\ problems for\\ systems of\\ nonlinear \\ higher-order PDEs};
\draw (22.75, -5.7) rectangle (26.35,-8.3);
\draw(26.45, -5.7) node[anchor=north west,align=left] {Linear\\ higher-order\\ PDEs};
\draw (26.45, -5.7) rectangle (29.05,-7.300000000000001);
\draw(43.400000000000006, -2) node[anchor=north west,align=left] {\large Partial differential equations and systems of partial differential equations with constant coefficients};
\draw (43.400000000000006, -2) rectangle (75.93,-6.199999999999999);
\draw(44.400000000000006, -3) node[anchor=north west,align=left] {Convexity \\ properties of \\ solutions to PDEs\\ and systems of\\ PDEs with \\ constant coefficients};
\draw (44.400000000000006, -3) rectangle (49.00000000000001,-6.1);
\draw(49.10000000000001, -3) node[anchor=north west,align=left] {Initial value \\ problems for PDEs\\ and systems \\ of PDEs with \\ constant coefficients};
\draw (49.10000000000001, -3) rectangle (53.70000000000001,-5.6);
\draw(53.800000000000004, -3) node[anchor=north west,align=left] {General theory\\ of PDEs and\\ systems of \\ PDEs with \\ constant coefficients};
\draw (53.800000000000004, -3) rectangle (58.150000000000006,-5.6);
\draw(58.25, -3) node[anchor=north west,align=left] {Fundamental \\ solutions to PDEs\\ and systems of\\ PDEs with constant\\ coefficients};
\draw (58.25, -3) rectangle (62.35,-5.6);
\draw(43.400000000000006, -6.299999999999999) node[anchor=north west,align=left] {\large History of partial \\ differential equations};
\draw (43.400000000000006, -6.299999999999999) rectangle (49.89000000000001,-7.399999999999999);
\draw(2, -8.5) node[anchor=north west,align=left] {\large General first-order partial differential equations and systems of first-order partial differential equations};
\draw (2, -8.5) rectangle (42.550000000000004,-14.9);
\draw(3, -9.5) node[anchor=north west,align=left] {Hamilton-Jacobiequations};
\draw (3, -9.5) rectangle (9.35,-11.1);
\draw(9.45, -9.5) node[anchor=north west,align=left] {Initial-boundary\\ value \\ problems for \\ systems of linear\\ first-order PDEs};
\draw (9.45, -9.5) rectangle (14.049999999999999,-12.1);
\draw(14.149999999999999, -9.5) node[anchor=north west,align=left] {Boundary value\\ problems\\ for systems\\ of nonlinear\\ first-order PDEs};
\draw (14.149999999999999, -9.5) rectangle (18.5,-12.1);
\draw(18.6, -9.5) node[anchor=north west,align=left] {Initial-boundary\\ value problems\\ for systems\\ of nonlinear\\ first-order PDEs};
\draw (18.6, -9.5) rectangle (22.950000000000003,-12.1);
\draw(23.05, -9.5) node[anchor=north west,align=left] {Initial value\\ problems \\ for systems \\ of nonlinear\\ first-order PDEs};
\draw (23.05, -9.5) rectangle (27.15,-12.1);
\draw(27.25, -9.5) node[anchor=north west,align=left] {Boundary value\\ problems\\ for linear \\ first-order PDEs};
\draw (27.25, -9.5) rectangle (31.1,-11.6);
\draw(31.2, -9.5) node[anchor=north west,align=left] {Initial-boundary\\ value\\ problems \\ for linear \\ first-order PDEs};
\draw (31.2, -9.5) rectangle (35.05,-12.1);
\draw(35.15, -9.5) node[anchor=north west,align=left] {Boundary value\\ problems for\\ nonlinear \\ first-order PDEs};
\draw (35.15, -9.5) rectangle (39.0,-11.6);
\draw(39.1, -9.5) node[anchor=north west,align=left] {Initial value\\ problems \\ for systems \\ of linear \\ first-order PDEs};
\draw (39.1, -9.5) rectangle (42.45,-12.1);
\draw(3, -12.2) node[anchor=north west,align=left] {Initial value\\ problems \\ for linear \\ first-order PDEs};
\draw (3, -12.2) rectangle (6.6,-14.299999999999999);
\draw(6.699999999999999, -12.2) node[anchor=north west,align=left] {Initial value\\ problems for\\ nonlinear \\ first-order PDEs};
\draw (6.699999999999999, -12.2) rectangle (10.299999999999999,-14.299999999999999);
\draw(10.4, -12.2) node[anchor=north west,align=left] {Initial-boundary\\ value \\ problems for\\ nonlinear \\ first-order PDEs};
\draw (10.4, -12.2) rectangle (14.0,-14.799999999999999);
\draw(14.1, -12.2) node[anchor=north west,align=left] {Boundary value\\ problems\\ for systems\\ of linear \\ first-order PDEs};
\draw (14.1, -12.2) rectangle (17.7,-14.799999999999999);
\draw(17.799999999999997, -12.2) node[anchor=north west,align=left] {Nonlinear\\ first-order\\ PDEs};
\draw (17.799999999999997, -12.2) rectangle (20.4,-13.799999999999999);
\draw(20.5, -12.2) node[anchor=north west,align=left] {Systems\\ of linear\\ first-order\\ PDEs};
\draw (20.5, -12.2) rectangle (23.1,-14.299999999999999);
\draw(23.2, -12.2) node[anchor=north west,align=left] {Systems \\ of nonlinear\\ first-order\\ PDEs};
\draw (23.2, -12.2) rectangle (25.8,-14.299999999999999);
\draw(25.9, -12.2) node[anchor=north west,align=left] {Linear\\ first-order\\ PDEs};
\draw (25.9, -12.2) rectangle (28.25,-13.799999999999999);
\draw(42.650000000000006, -8.5) node[anchor=north west,align=left] {\large Qualitative properties of solutions to partial differential equations};
\draw (42.650000000000006, -8.5) rectangle (68.7,-19.3);
\draw(43.650000000000006, -9.5) node[anchor=north west,align=left] {Oscillation, \\ zeros of solutions,\\ mean value \\ theorems, etc. \\ in context of PDEs};
\draw (43.650000000000006, -9.5) rectangle (48.00000000000001,-12.1);
\draw(48.10000000000001, -9.5) node[anchor=north west,align=left] {Dependence of \\ solutions to PDEs\\ on initial \\ and/or boundary \\ data and/or on \\ parameters of PDEs};
\draw (48.10000000000001, -9.5) rectangle (52.45000000000001,-12.6);
\draw(52.550000000000004, -9.5) node[anchor=north west,align=left] {Homogenization\\ in context of\\ PDEs; PDEs in\\ media with \\ periodic structure};
\draw (52.550000000000004, -9.5) rectangle (56.650000000000006,-12.1);
\draw(56.75, -9.5) node[anchor=north west,align=left] {Critical points\\ of functionals\\ in context of\\ PDEs (e.g., \\ energy functionals)};
\draw (56.75, -9.5) rectangle (60.85,-12.1);
\draw(60.95, -9.5) node[anchor=north west,align=left] {Liouville \\ theorems and \\ Phragmén-Lindelöf\\ theorems in\\ context of PDEs};
\draw (60.95, -9.5) rectangle (65.05,-12.1);
\draw(65.15, -9.5) node[anchor=north west,align=left] {Asymptotic\\ behavior\\ of \\ solutions to PDEs};
\draw (65.15, -9.5) rectangle (68.5,-11.6);
\draw(43.650000000000006, -12.7) node[anchor=north west,align=left] {Almost and \\ pseudo-almost\\ periodic \\ solutions to PDEs};
\draw (43.650000000000006, -12.7) rectangle (47.50000000000001,-14.799999999999999);
\draw(47.60000000000001, -12.7) node[anchor=north west,align=left] {Symmetries,\\ invariants,\\ etc. in \\ context of PDEs};
\draw (47.60000000000001, -12.7) rectangle (51.20000000000001,-14.799999999999999);
\draw(51.300000000000004, -12.7) node[anchor=north west,align=left] {Continuation\\ and prolongation\\ of \\ solutions to PDEs};
\draw (51.300000000000004, -12.7) rectangle (54.900000000000006,-14.799999999999999);
\draw(55.00000000000001, -12.7) node[anchor=north west,align=left] {Inertial\\ manifolds};
\draw (55.00000000000001, -12.7) rectangle (58.35000000000001,-14.299999999999999);
\draw(58.45, -12.7) node[anchor=north west,align=left] {Comparison\\ principles\\ in \\ context of PDEs};
\draw (58.45, -12.7) rectangle (61.800000000000004,-14.799999999999999);
\draw(61.900000000000006, -12.7) node[anchor=north west,align=left] {Smoothness\\ and regularity\\ of \\ solutions to PDEs};
\draw (61.900000000000006, -12.7) rectangle (65.25,-14.799999999999999);
\draw(65.35000000000001, -12.7) node[anchor=north west,align=left] {Perturbations\\ in \\ context of PDEs};
\draw (65.35000000000001, -12.7) rectangle (68.45,-14.299999999999999);
\draw(43.650000000000006, -14.9) node[anchor=north west,align=left] {Axially\\ symmetric\\ solutions\\ to PDEs};
\draw (43.650000000000006, -14.9) rectangle (46.50000000000001,-17.0);
\draw(46.60000000000001, -14.9) node[anchor=north west,align=left] {Singular \\ perturbations\\ in context\\ of PDEs};
\draw (46.60000000000001, -14.9) rectangle (49.45000000000001,-17.0);
\draw(49.550000000000004, -14.9) node[anchor=north west,align=left] {Bifurcations\\ in context\\ of PDEs};
\draw (49.550000000000004, -14.9) rectangle (52.400000000000006,-16.5);
\draw(52.50000000000001, -14.9) node[anchor=north west,align=left] {Entire\\ solutions\\ to PDEs};
\draw (52.50000000000001, -14.9) rectangle (55.10000000000001,-16.5);
\draw(55.2, -14.9) node[anchor=north west,align=left] {Positive\\ solutions\\ to PDEs};
\draw (55.2, -14.9) rectangle (57.800000000000004,-16.5);
\draw(57.900000000000006, -14.9) node[anchor=north west,align=left] {Periodic\\ solutions\\ to PDEs};
\draw (57.900000000000006, -14.9) rectangle (60.50000000000001,-16.5);
\draw(60.60000000000001, -14.9) node[anchor=north west,align=left] {Critical\\ exponents\\ in context\\ of PDEs};
\draw (60.60000000000001, -14.9) rectangle (63.20000000000001,-17.0);
\draw(63.300000000000004, -14.9) node[anchor=north west,align=left] {Resonance\\ in context\\ of PDEs};
\draw (63.300000000000004, -14.9) rectangle (65.9,-16.5);
\draw(66.0, -14.9) node[anchor=north west,align=left] {Stability\\ in context\\ of PDEs};
\draw (66.0, -14.9) rectangle (68.6,-16.5);
\draw(43.650000000000006, -17.1) node[anchor=north west,align=left] {Pattern \\ formations\\ in context\\ of PDEs};
\draw (43.650000000000006, -17.1) rectangle (46.25000000000001,-19.200000000000003);
\draw(46.35000000000001, -17.1) node[anchor=north west,align=left] {Attractors};
\draw (46.35000000000001, -17.1) rectangle (48.95000000000001,-18.200000000000003);
\draw(49.050000000000004, -17.1) node[anchor=north west,align=left] {Blow-up\\ in context\\ of PDEs};
\draw (49.050000000000004, -17.1) rectangle (51.650000000000006,-18.700000000000003);
\draw(51.75000000000001, -17.1) node[anchor=north west,align=left] {A priori\\ estimates\\ in context\\ of PDEs};
\draw (51.75000000000001, -17.1) rectangle (54.35000000000001,-19.200000000000003);
\draw(54.45, -17.1) node[anchor=north west,align=left] {Maximum \\ principles\\ in context\\ of PDEs};
\draw (54.45, -17.1) rectangle (57.050000000000004,-19.200000000000003);
\draw(2, -15.0) node[anchor=north west,align=left] {\large Generalized solutions to partial differential equations};
\draw (2, -15.0) rectangle (19.650000000000002,-17.7);
\draw(3, -16.0) node[anchor=north west,align=left] {Viscosity\\ solutions\\ to PDEs};
\draw (3, -16.0) rectangle (5.85,-17.6);
\draw(5.95, -16.0) node[anchor=north west,align=left] {Strong\\ solutions\\ to PDEs};
\draw (5.95, -16.0) rectangle (8.55,-17.6);
\draw(8.65, -16.0) node[anchor=north west,align=left] {Weak \\ solutions\\ to PDEs};
\draw (8.65, -16.0) rectangle (10.75,-17.6);
\draw(68.8, -8.5) node[anchor=north west,align=left] {\large Close-to-elliptic equations};
\draw (68.8, -8.5) rectangle (79.69999999999999,-12.9);
\draw(69.8, -9.5) node[anchor=north west,align=left] {Quasiellipticequations};
\draw (69.8, -9.5) rectangle (75.64999999999999,-11.1);
\draw(75.75, -9.5) node[anchor=north west,align=left] {Hypoelliptic\\ equations};
\draw (75.75, -9.5) rectangle (79.6,-11.1);
\draw(69.8, -11.2) node[anchor=north west,align=left] {Subellipticequations};
\draw (69.8, -11.2) rectangle (75.14999999999999,-12.799999999999999);
\draw(2, -19.4) node[anchor=north west,align=left] {\large Representations of solutions to partial differential equations};
\draw (2, -19.4) rectangle (24.150000000000002,-24.299999999999997);
\draw(3, -20.4) node[anchor=north west,align=left] {Solutions\\ to PDEs in\\ closed form};
\draw (3, -20.4) rectangle (6.1,-22.0);
\draw(6.199999999999999, -20.4) node[anchor=north west,align=left] {Self-similar\\ solutions\\ to PDEs};
\draw (6.199999999999999, -20.4) rectangle (9.299999999999999,-22.0);
\draw(9.399999999999999, -20.4) node[anchor=north west,align=left] {Integral \\ representations\\ of solutions\\ to PDEs};
\draw (9.399999999999999, -20.4) rectangle (12.499999999999998,-22.5);
\draw(12.6, -20.4) node[anchor=north west,align=left] {Asymptotic\\ expansions\\ of solutions\\ to PDEs};
\draw (12.6, -20.4) rectangle (15.7,-22.5);
\draw(15.799999999999999, -20.4) node[anchor=north west,align=left] {Trigonometric\\ solutions\\ to PDEs};
\draw (15.799999999999999, -20.4) rectangle (18.65,-22.0);
\draw(18.75, -20.4) node[anchor=north west,align=left] {Traveling\\ wave\\ solutions};
\draw (18.75, -20.4) rectangle (21.35,-22.0);
\draw(21.45, -20.4) node[anchor=north west,align=left] {Series\\ solutions\\ to PDEs};
\draw (21.45, -20.4) rectangle (24.05,-22.0);
\draw(3, -22.599999999999998) node[anchor=north west,align=left] {Polynomial\\ solutions\\ to PDEs};
\draw (3, -22.599999999999998) rectangle (5.6,-24.2);
\draw(5.7, -22.599999999999998) node[anchor=north west,align=left] {Soliton\\ solutions};
\draw (5.7, -22.599999999999998) rectangle (8.05,-23.7);
\draw(24.250000000000004, -19.4) node[anchor=north west,align=left] {\large General topics in partial differential equations};
\draw (24.250000000000004, -19.4) rectangle (42.55,-31.7);
\draw(25.250000000000004, -20.4) node[anchor=north west,align=left] {Inequalities \\ applied to PDEs \\ involving derivatives,\\ differential\\ and integral \\ operators, or integrals};
\draw (25.250000000000004, -20.4) rectangle (30.6,-23.5);
\draw(30.700000000000003, -20.4) node[anchor=north west,align=left] {Cauchy-Kovalevskaya\\ theorems};
\draw (30.700000000000003, -20.4) rectangle (35.550000000000004,-22.0);
\draw(35.650000000000006, -20.4) node[anchor=north west,align=left] {Existence problems\\ for PDEs: \\ global existence,\\ local existence,\\ non-existence};
\draw (35.650000000000006, -20.4) rectangle (40.00000000000001,-23.0);
\draw(25.250000000000004, -23.599999999999998) node[anchor=north west,align=left] {Uniqueness problems\\ for PDEs: \\ global uniqueness,\\ local uniqueness,\\ non-uniqueness};
\draw (25.250000000000004, -23.599999999999998) rectangle (29.6,-26.2);
\draw(29.700000000000003, -23.599999999999998) node[anchor=north west,align=left] {Geometric \\ theory, \\ characteristics, \\ transformations \\ in context of PDEs};
\draw (29.700000000000003, -23.599999999999998) rectangle (34.050000000000004,-26.2);
\draw(34.150000000000006, -23.599999999999998) node[anchor=north west,align=left] {Topological\\ and monotonicity\\ methods\\ applied to PDEs};
\draw (34.150000000000006, -23.599999999999998) rectangle (38.25000000000001,-25.7);
\draw(38.35, -23.599999999999998) node[anchor=north west,align=left] {Microlocal \\ methods and methods\\ of sheaf \\ theory and \\ homological algebra\\ applied to PDEs};
\draw (38.35, -23.599999999999998) rectangle (42.45,-26.7);
\draw(25.250000000000004, -26.799999999999997) node[anchor=north west,align=left] {Methods of\\ ordinary \\ differential\\ equations \\ applied to PDEs};
\draw (25.250000000000004, -26.799999999999997) rectangle (28.850000000000005,-29.4);
\draw(28.950000000000003, -26.799999999999997) node[anchor=north west,align=left] {Theoretical\\ approximation\\ in \\ context of PDEs};
\draw (28.950000000000003, -26.799999999999997) rectangle (32.550000000000004,-28.9);
\draw(32.650000000000006, -26.799999999999997) node[anchor=north west,align=left] {Transform \\ methods (e.g.,\\ integral \\ transforms) \\ applied to PDEs};
\draw (32.650000000000006, -26.799999999999997) rectangle (36.00000000000001,-29.4);
\draw(36.1, -26.799999999999997) node[anchor=north west,align=left] {Analyticity\\ in \\ context of PDEs};
\draw (36.1, -26.799999999999997) rectangle (39.2,-28.4);
\draw(39.300000000000004, -26.799999999999997) node[anchor=north west,align=left] {Singularity\\ in \\ context of PDEs};
\draw (39.300000000000004, -26.799999999999997) rectangle (42.400000000000006,-28.4);
\draw(25.250000000000004, -29.5) node[anchor=north west,align=left] {Fundamental\\ solutions\\ to PDEs};
\draw (25.250000000000004, -29.5) rectangle (28.100000000000005,-31.1);
\draw(28.200000000000003, -29.5) node[anchor=north west,align=left] {Classical\\ solutions\\ to PDEs};
\draw (28.200000000000003, -29.5) rectangle (31.050000000000004,-31.1);
\draw(31.150000000000006, -29.5) node[anchor=north west,align=left] {Variational\\ methods\\ applied\\ to PDEs};
\draw (31.150000000000006, -29.5) rectangle (34.00000000000001,-31.6);
\draw(34.1, -29.5) node[anchor=north west,align=left] {Parametrices\\ in context\\ of PDEs};
\draw (34.1, -29.5) rectangle (36.95,-31.1);
\draw(37.050000000000004, -29.5) node[anchor=north west,align=left] {Wave front\\ sets\\ in context\\ of PDEs};
\draw (37.050000000000004, -29.5) rectangle (39.650000000000006,-31.6);
\draw(39.75, -29.5) node[anchor=north west,align=left] {Other \\ special methods\\ applied\\ to PDEs};
\draw (39.75, -29.5) rectangle (42.35,-31.6);
\draw(2, -24.4) node[anchor=north west,align=left] {\large Hyperbolic equations and hyperbolic systems};
\draw (2, -24.4) rectangle (18.35,-30.799999999999997);
\draw(3, -25.4) node[anchor=north west,align=left] {Initial value\\ problems\\ for first-order\\ hyperbolic equations};
\draw (3, -25.4) rectangle (9.6,-28.0);
\draw(9.7, -25.4) node[anchor=north west,align=left] {Initial-boundary\\ value \\ problems for \\ second-order \\ hyperbolic equations};
\draw (9.7, -25.4) rectangle (14.049999999999999,-28.0);
\draw(14.149999999999999, -25.4) node[anchor=north west,align=left] {Initial-boundary\\ value \\ problems for \\ first-order \\ hyperbolic equations};
\draw (14.149999999999999, -25.4) rectangle (18.25,-28.0);
\draw(3, -28.099999999999998) node[anchor=north west,align=left] {Initial value\\ problems\\ for \\ second-order \\ hyperbolic equations};
\draw (3, -28.099999999999998) rectangle (7.1,-30.7);
\draw(7.199999999999999, -28.099999999999998) node[anchor=north west,align=left] {Second-order\\ hyperbolic\\ equations};
\draw (7.199999999999999, -28.099999999999998) rectangle (10.549999999999999,-29.7);
\draw(10.65, -28.099999999999998) node[anchor=north west,align=left] {First-order\\ hyperbolic\\ equations};
\draw (10.65, -28.099999999999998) rectangle (13.75,-29.7);
\draw(13.85, -28.099999999999998) node[anchor=north west,align=left] {Wave \\ equation};
\draw (13.85, -28.099999999999998) rectangle (15.7,-29.2);
\draw(42.65, -19.4) node[anchor=north west,align=left] {\large Parabolic equations and parabolic systems};
\draw (42.65, -19.4) rectangle (58.25,-43.7);
\draw(43.65, -20.4) node[anchor=north west,align=left] {Ultraparabolic\\ equations,\\ pseudoparabolic \\ equations, etc.};
\draw (43.65, -20.4) rectangle (49.75,-23.0);
\draw(49.849999999999994, -20.4) node[anchor=north west,align=left] {Unilateral problems\\ for linear parabolic\\ equations and\\ variational \\ inequalities with linear\\ parabolic operators};
\draw (49.849999999999994, -20.4) rectangle (55.199999999999996,-23.5);
\draw(55.3, -20.4) node[anchor=north west,align=left] {Nonlinear\\ parabolic\\ equations};
\draw (55.3, -20.4) rectangle (58.15,-22.0);
\draw(55.3, -22.099999999999998) node[anchor=north west,align=left] {Heat \\ equation};
\draw (55.3, -22.099999999999998) rectangle (57.15,-23.2);
\draw(43.65, -23.599999999999998) node[anchor=north west,align=left] {Unilateral problems\\ for nonlinear\\ parabolic \\ equations and variational\\ inequalities\\ with nonlinear\\ parabolic operators};
\draw (43.65, -23.599999999999998) rectangle (49.0,-27.2);
\draw(49.099999999999994, -23.599999999999998) node[anchor=north west,align=left] {Unilateral problems\\ for parabolic\\ systems and systems\\ of variational\\ inequalities with\\ parabolic operators};
\draw (49.099999999999994, -23.599999999999998) rectangle (54.449999999999996,-26.7);
\draw(54.55, -23.599999999999998) node[anchor=north west,align=left] {Initial-boundary\\ value \\ problems for \\ second-order \\ parabolic systems};
\draw (54.55, -23.599999999999998) rectangle (58.15,-26.2);
\draw(43.65, -27.299999999999997) node[anchor=north west,align=left] {Reaction-diffusion\\ equations};
\draw (43.65, -27.299999999999997) rectangle (48.5,-28.9);
\draw(48.599999999999994, -27.299999999999997) node[anchor=north west,align=left] {Nonlinear initial,\\ boundary and\\ initial-boundary\\ value problems\\ for linear\\ parabolic equations};
\draw (48.599999999999994, -27.299999999999997) rectangle (53.449999999999996,-30.4);
\draw(43.65, -29.0) node[anchor=north west,align=left] {Heat\\ kernel};
\draw (43.65, -29.0) rectangle (45.25,-30.1);
\draw(53.55, -27.299999999999997) node[anchor=north west,align=left] {Nonlinear initial,\\ boundary and\\ initial-boundary\\ value problems\\ for nonlinear \\ parabolic equations};
\draw (53.55, -27.299999999999997) rectangle (57.9,-30.4);
\draw(43.65, -30.5) node[anchor=north west,align=left] {Quasilinear\\ parabolic\\ equations \\ with \(p\)-Laplacian};
\draw (43.65, -30.5) rectangle (48.5,-32.6);
\draw(48.599999999999994, -30.5) node[anchor=north west,align=left] {Semilinear \\ parabolic equations\\ with Laplacian,\\ bi-Laplacian\\ or poly-Laplacian};
\draw (48.599999999999994, -30.5) rectangle (52.949999999999996,-33.1);
\draw(53.05, -30.5) node[anchor=north west,align=left] {Initial-boundary\\ value \\ problems for \\ second-order \\ parabolic equations};
\draw (53.05, -30.5) rectangle (57.15,-33.1);
\draw(43.65, -33.2) node[anchor=north west,align=left] {Initial-boundary\\ value \\ problems for \\ higher-order \\ parabolic equations};
\draw (43.65, -33.2) rectangle (47.75,-35.800000000000004);
\draw(47.849999999999994, -33.2) node[anchor=north west,align=left] {Quasilinear\\ parabolic\\ equations \\ with mean \\ curvature operator};
\draw (47.849999999999994, -33.2) rectangle (51.949999999999996,-35.800000000000004);
\draw(52.05, -33.2) node[anchor=north west,align=left] {Initial value\\ problems\\ for \\ second-order \\ parabolic equations};
\draw (52.05, -33.2) rectangle (55.9,-35.800000000000004);
\draw(43.65, -35.9) node[anchor=north west,align=left] {Initial value\\ problems\\ for \\ higher-order \\ parabolic equations};
\draw (43.65, -35.9) rectangle (47.5,-38.5);
\draw(47.599999999999994, -35.9) node[anchor=north west,align=left] {Initial-boundary\\ value \\ problems for \\ higher-order \\ parabolic systems};
\draw (47.599999999999994, -35.9) rectangle (51.199999999999996,-38.5);
\draw(51.3, -35.9) node[anchor=north west,align=left] {Initial value\\ problems\\ for \\ second-order \\ parabolic systems};
\draw (51.3, -35.9) rectangle (54.65,-38.5);
\draw(54.75, -35.9) node[anchor=north west,align=left] {Initial value\\ problems\\ for \\ higher-order \\ parabolic systems};
\draw (54.75, -35.9) rectangle (58.1,-38.5);
\draw(43.65, -38.599999999999994) node[anchor=north west,align=left] {Second-order\\ parabolic\\ equations};
\draw (43.65, -38.599999999999994) rectangle (46.75,-40.199999999999996);
\draw(46.849999999999994, -38.599999999999994) node[anchor=north west,align=left] {Higher-order\\ parabolic\\ equations};
\draw (46.849999999999994, -38.599999999999994) rectangle (49.949999999999996,-40.199999999999996);
\draw(50.05, -38.599999999999994) node[anchor=north west,align=left] {Second-order\\ parabolic\\ systems};
\draw (50.05, -38.599999999999994) rectangle (53.15,-40.199999999999996);
\draw(53.25, -38.599999999999994) node[anchor=north west,align=left] {Higher-order\\ parabolic\\ systems};
\draw (53.25, -38.599999999999994) rectangle (56.35,-40.199999999999996);
\draw(43.65, -40.3) node[anchor=north west,align=left] {Semilinear\\ parabolic\\ equations};
\draw (43.65, -40.3) rectangle (46.75,-41.9);
\draw(46.849999999999994, -40.3) node[anchor=north west,align=left] {Degenerate\\ parabolic\\ equations};
\draw (46.849999999999994, -40.3) rectangle (49.949999999999996,-41.9);
\draw(50.05, -40.3) node[anchor=north west,align=left] {Parabolic\\ Monge-Ampère\\ equations};
\draw (50.05, -40.3) rectangle (53.15,-41.9);
\draw(53.25, -40.3) node[anchor=north west,align=left] {Quasilinear\\ parabolic\\ equations};
\draw (53.25, -40.3) rectangle (56.1,-41.9);
\draw(43.65, -42.0) node[anchor=north west,align=left] {Singular\\ parabolic\\ equations};
\draw (43.65, -42.0) rectangle (46.25,-43.6);
\draw(46.35, -42.0) node[anchor=north west,align=left] {Abstract\\ parabolic\\ equations};
\draw (46.35, -42.0) rectangle (48.95,-43.6);
\draw(58.349999999999994, -19.4) node[anchor=north west,align=left] {\large Elliptic equations and elliptic systems};
\draw (58.349999999999994, -19.4) rectangle (73.19999999999999,-45.9);
\draw(59.349999999999994, -20.4) node[anchor=north west,align=left] {Unilateral problems\\ for linear elliptic\\ equations and\\ variational \\ inequalities with linear\\ elliptic operators};
\draw (59.349999999999994, -20.4) rectangle (64.19999999999999,-23.5);
\draw(64.3, -20.4) node[anchor=north west,align=left] {Unilateral problems\\ for nonlinear\\ elliptic equations\\ and variational\\ inequalities\\ with nonlinear\\ elliptic operators};
\draw (64.3, -20.4) rectangle (69.14999999999999,-24.0);
\draw(69.25, -20.4) node[anchor=north west,align=left] {Variational\\ methods\\ for \\ second-order \\ elliptic equations};
\draw (69.25, -20.4) rectangle (73.1,-23.0);
\draw(59.349999999999994, -24.099999999999998) node[anchor=north west,align=left] {Unilateral problems\\ for elliptic \\ systems and systems\\ of variational\\ inequalities with\\ elliptic operators};
\draw (59.349999999999994, -24.099999999999998) rectangle (64.19999999999999,-27.2);
\draw(64.3, -24.099999999999998) node[anchor=north west,align=left] {Laplace operator,\\ Helmholtz \\ equation (reduced\\ wave equation),\\ Poisson equation};
\draw (64.3, -24.099999999999998) rectangle (68.89999999999999,-26.7);
\draw(69.0, -24.099999999999998) node[anchor=north west,align=left] {Semilinear \\ elliptic equations\\ with Laplacian,\\ bi-Laplacian \\ or poly-Laplacian};
\draw (69.0, -24.099999999999998) rectangle (73.1,-26.7);
\draw(59.349999999999994, -27.299999999999997) node[anchor=north west,align=left] {Quasilinear\\ elliptic \\ equations \\ with \(p\)-Laplacian};
\draw (59.349999999999994, -27.299999999999997) rectangle (63.949999999999996,-29.4);
\draw(64.05, -27.299999999999997) node[anchor=north west,align=left] {Boundary \\ value problems\\ for \\ second-order \\ elliptic equations};
\draw (64.05, -27.299999999999997) rectangle (67.89999999999999,-29.9);
\draw(68.0, -27.299999999999997) node[anchor=north west,align=left] {Variational\\ methods\\ for \\ higher-order \\ elliptic equations};
\draw (68.0, -27.299999999999997) rectangle (71.85,-29.9);
\draw(59.349999999999994, -30.0) node[anchor=north west,align=left] {Boundary \\ value problems\\ for \\ higher-order \\ elliptic equations};
\draw (59.349999999999994, -30.0) rectangle (63.199999999999996,-32.6);
\draw(63.3, -30.0) node[anchor=north west,align=left] {Nonlinear \\ boundary value\\ problems for\\ linear \\ elliptic equations};
\draw (63.3, -30.0) rectangle (67.14999999999999,-32.6);
\draw(67.25, -30.0) node[anchor=north west,align=left] {Boundary values\\ of solutions\\ to elliptic \\ equations and \\ elliptic systems};
\draw (67.25, -30.0) rectangle (71.1,-32.6);
\draw(59.349999999999994, -32.7) node[anchor=north west,align=left] {Quasilinear\\ elliptic \\ equations \\ with mean \\ curvature operator};
\draw (59.349999999999994, -32.7) rectangle (63.199999999999996,-35.300000000000004);
\draw(63.3, -32.7) node[anchor=north west,align=left] {Monge-Ampère\\ equations};
\draw (63.3, -32.7) rectangle (67.14999999999999,-34.300000000000004);
\draw(67.25, -32.7) node[anchor=north west,align=left] {Nonlinear \\ boundary value \\ problems for \\ nonlinear \\ elliptic equations};
\draw (67.25, -32.7) rectangle (70.85,-35.300000000000004);
\draw(59.349999999999994, -35.4) node[anchor=north west,align=left] {Green’s \\ functions\\ for elliptic\\ equations};
\draw (59.349999999999994, -35.4) rectangle (62.699999999999996,-37.5);
\draw(62.8, -35.4) node[anchor=north west,align=left] {Variational\\ methods\\ for \\ elliptic systems};
\draw (62.8, -35.4) rectangle (66.14999999999999,-37.5);
\draw(66.25, -35.4) node[anchor=north west,align=left] {Boundary \\ value problems\\ for \\ second-order \\ elliptic systems};
\draw (66.25, -35.4) rectangle (69.6,-38.0);
\draw(69.69999999999999, -35.4) node[anchor=north west,align=left] {Boundary \\ value problems\\ for \\ higher-order \\ elliptic systems};
\draw (69.69999999999999, -35.4) rectangle (73.04999999999998,-38.0);
\draw(59.349999999999994, -38.099999999999994) node[anchor=north west,align=left] {Schrödinger\\ operator,\\ Schrödinger\\ equation};
\draw (59.349999999999994, -38.099999999999994) rectangle (62.449999999999996,-40.199999999999996);
\draw(62.55, -38.099999999999994) node[anchor=north west,align=left] {Boundary \\ value problems\\ for \\ first-order \\ elliptic systems};
\draw (62.55, -38.099999999999994) rectangle (65.64999999999999,-40.699999999999996);
\draw(65.75, -38.099999999999994) node[anchor=north west,align=left] {Quasilinear\\ elliptic\\ equations};
\draw (65.75, -38.099999999999994) rectangle (68.85,-39.699999999999996);
\draw(68.94999999999999, -38.099999999999994) node[anchor=north west,align=left] {Elliptic \\ equations \\ with \\ infinity-Laplacian};
\draw (68.94999999999999, -38.099999999999994) rectangle (72.04999999999998,-40.199999999999996);
\draw(59.349999999999994, -40.8) node[anchor=north west,align=left] {Second-order\\ elliptic\\ equations};
\draw (59.349999999999994, -40.8) rectangle (62.199999999999996,-42.4);
\draw(62.3, -40.8) node[anchor=north west,align=left] {Higher-order\\ elliptic\\ equations};
\draw (62.3, -40.8) rectangle (65.14999999999999,-42.4);
\draw(65.25, -40.8) node[anchor=north west,align=left] {Second-order\\ elliptic\\ systems};
\draw (65.25, -40.8) rectangle (68.1,-42.4);
\draw(68.19999999999999, -40.8) node[anchor=north west,align=left] {Higher-order\\ elliptic\\ systems};
\draw (68.19999999999999, -40.8) rectangle (71.04999999999998,-42.4);
\draw(59.349999999999994, -42.5) node[anchor=north west,align=left] {Semilinear\\ elliptic\\ equations};
\draw (59.349999999999994, -42.5) rectangle (62.199999999999996,-44.1);
\draw(62.3, -42.5) node[anchor=north west,align=left] {Degenerate\\ elliptic\\ equations};
\draw (62.3, -42.5) rectangle (65.14999999999999,-44.1);
\draw(65.25, -42.5) node[anchor=north west,align=left] {Singular\\ elliptic\\ equations};
\draw (65.25, -42.5) rectangle (68.1,-44.1);
\draw(68.19999999999999, -42.5) node[anchor=north west,align=left] {First-order\\ elliptic\\ systems};
\draw (68.19999999999999, -42.5) rectangle (70.79999999999998,-44.1);
\draw(59.349999999999994, -44.199999999999996) node[anchor=north west,align=left] {Nonlinear\\ elliptic\\ equations};
\draw (59.349999999999994, -44.199999999999996) rectangle (61.949999999999996,-45.8);
\draw(79.89999999999998, -1) node[anchor=north west,align=left] {\LARGE Ordinary differential equations};
\draw (79.89999999999998, -1) rectangle (142.07999999999998,-67.9);
\draw(80.89999999999998, -2) node[anchor=north west,align=left] {\large Functional-differential equations (including equations with delayed, advanced or state-dependent argument)};
\draw (80.89999999999998, -2) rectangle (120.59999999999997,-20.200000000000003);
\draw(81.89999999999998, -3) node[anchor=north west,align=left] {Functional-differentialequations\\ with state-dependent\\ arguments};
\draw (81.89999999999998, -3) rectangle (91.99999999999997,-5.6);
\draw(92.09999999999998, -3) node[anchor=north west,align=left] {Functional-differentialequations\\ with fractional\\ derivatives};
\draw (92.09999999999998, -3) rectangle (101.44999999999997,-5.6);
\draw(101.54999999999998, -3) node[anchor=north west,align=left] {Functional-differentialinequalities};
\draw (101.54999999999998, -3) rectangle (110.64999999999998,-5.1);
\draw(110.74999999999997, -3) node[anchor=north west,align=left] {Functional-differentialinclusions};
\draw (110.74999999999997, -3) rectangle (119.34999999999997,-5.1);
\draw(81.89999999999998, -5.7) node[anchor=north west,align=left] {Transformation\\ and reduction \\ of functional-differential\\ equations and systems,\\ normal forms};
\draw (81.89999999999998, -5.7) rectangle (89.74999999999997,-8.8);
\draw(89.84999999999998, -5.7) node[anchor=north west,align=left] {Functional-differentialequations\\ in\\ abstract spaces};
\draw (89.84999999999998, -5.7) rectangle (96.94999999999997,-7.800000000000001);
\draw(97.04999999999998, -5.7) node[anchor=north west,align=left] {Functional-differential\\ equations on \\ time scales or\\ measure chains};
\draw (97.04999999999998, -5.7) rectangle (103.89999999999998,-8.3);
\draw(103.99999999999997, -5.7) node[anchor=north west,align=left] {Discontinuous\\ functional-differential\\ equations};
\draw (103.99999999999997, -5.7) rectangle (110.34999999999997,-7.800000000000001);
\draw(110.44999999999997, -5.7) node[anchor=north west,align=left] {Functional-differential\\ equations\\ in the\\ complex domain};
\draw (110.44999999999997, -5.7) rectangle (116.79999999999997,-8.3);
\draw(116.89999999999998, -5.7) node[anchor=north west,align=left] {Stability \\ theory of \\ functional-differential\\ equations};
\draw (116.89999999999998, -5.7) rectangle (120.49999999999997,-7.800000000000001);
\draw(81.89999999999998, -8.9) node[anchor=north west,align=left] {Functional-differential\\ equations\\ with impulses};
\draw (81.89999999999998, -8.9) rectangle (88.24999999999997,-11.0);
\draw(88.34999999999998, -8.9) node[anchor=north west,align=left] {Stochastic\\ functional-differential\\ equations};
\draw (88.34999999999998, -8.9) rectangle (94.44999999999997,-11.0);
\draw(94.54999999999998, -8.9) node[anchor=north west,align=left] {Invariant \\ manifolds of\\ functional-differential\\ equations};
\draw (94.54999999999998, -8.9) rectangle (100.39999999999998,-11.5);
\draw(100.49999999999997, -8.9) node[anchor=north west,align=left] {Linear \\ functional-differential\\ equations};
\draw (100.49999999999997, -8.9) rectangle (106.09999999999997,-11.0);
\draw(106.19999999999997, -8.9) node[anchor=north west,align=left] {Implicit \\ functional-differential\\ equations};
\draw (106.19999999999997, -8.9) rectangle (111.79999999999997,-11.0);
\draw(111.89999999999998, -8.9) node[anchor=north west,align=left] {Lattice \\ functional-differential\\ equations};
\draw (111.89999999999998, -8.9) rectangle (117.24999999999997,-11.0);
\draw(81.89999999999998, -11.600000000000001) node[anchor=north west,align=left] {Fuzzy \\ functional-differential\\ equations};
\draw (81.89999999999998, -11.600000000000001) rectangle (87.24999999999997,-13.700000000000001);
\draw(87.34999999999998, -11.600000000000001) node[anchor=north west,align=left] {Neutral \\ functional-differential\\ equations};
\draw (87.34999999999998, -11.600000000000001) rectangle (92.69999999999997,-13.700000000000001);
\draw(92.79999999999998, -11.600000000000001) node[anchor=north west,align=left] {Almost and \\ pseudo-almost periodic\\ solutions \\ to functional-differential\\ equations};
\draw (92.79999999999998, -11.600000000000001) rectangle (97.89999999999998,-14.200000000000001);
\draw(97.99999999999997, -11.600000000000001) node[anchor=north west,align=left] {Theoretical \\ approximation of\\ solutions to\\ functional-differential\\ equations};
\draw (97.99999999999997, -11.600000000000001) rectangle (102.84999999999997,-14.200000000000001);
\draw(102.94999999999997, -11.600000000000001) node[anchor=north west,align=left] {Qualitative \\ investigation and\\ simulation of \\ models involving\\ functional-differential\\ equations};
\draw (102.94999999999997, -11.600000000000001) rectangle (107.79999999999997,-14.700000000000001);
\draw(107.89999999999998, -11.600000000000001) node[anchor=north west,align=left] {Growth, boundedness,\\ comparison\\ of solutions to\\ functional-differential\\ equations};
\draw (107.89999999999998, -11.600000000000001) rectangle (112.24999999999997,-14.200000000000001);
\draw(112.34999999999998, -11.600000000000001) node[anchor=north west,align=left] {Spectral \\ theory of \\ functional-differential\\ operators};
\draw (112.34999999999998, -11.600000000000001) rectangle (116.44999999999997,-13.700000000000001);
\draw(116.54999999999998, -11.600000000000001) node[anchor=north west,align=left] {Symmetries,\\ invariants\\ of \\ functional-differential\\ equations};
\draw (116.54999999999998, -11.600000000000001) rectangle (120.39999999999998,-14.200000000000001);
\draw(81.89999999999998, -14.8) node[anchor=north west,align=left] {Complex (chaotic)\\ behavior of\\ solutions to \\ functional-differential\\ equations};
\draw (81.89999999999998, -14.8) rectangle (85.99999999999997,-17.400000000000002);
\draw(86.09999999999998, -14.8) node[anchor=north west,align=left] {Synchronization\\ of \\ functional-differential\\ equations};
\draw (86.09999999999998, -14.8) rectangle (90.19999999999997,-16.900000000000002);
\draw(90.29999999999998, -14.8) node[anchor=north west,align=left] {General theory\\ of \\ functional-differential\\ equations};
\draw (90.29999999999998, -14.8) rectangle (94.14999999999998,-16.900000000000002);
\draw(94.24999999999997, -14.8) node[anchor=north west,align=left] {Oscillation\\ theory \\ of \\ functional-differential\\ equations};
\draw (94.24999999999997, -14.8) rectangle (98.09999999999997,-17.400000000000002);
\draw(98.19999999999997, -14.8) node[anchor=north west,align=left] {Periodic \\ solutions \\ to \\ functional-differential\\ equations};
\draw (98.19999999999997, -14.8) rectangle (102.04999999999997,-17.400000000000002);
\draw(102.14999999999998, -14.8) node[anchor=north west,align=left] {Heteroclinic \\ and homoclinic\\ orbits of \\ functional-differential\\ equations};
\draw (102.14999999999998, -14.8) rectangle (105.99999999999997,-17.400000000000002);
\draw(106.09999999999998, -14.8) node[anchor=north west,align=left] {Bifurcation\\ theory \\ of \\ functional-differential\\ equations};
\draw (106.09999999999998, -14.8) rectangle (109.94999999999997,-17.400000000000002);
\draw(110.04999999999998, -14.8) node[anchor=north west,align=left] {Asymptotic \\ theory of \\ functional-differential\\ equations};
\draw (110.04999999999998, -14.8) rectangle (113.89999999999998,-16.900000000000002);
\draw(113.99999999999997, -14.8) node[anchor=north west,align=left] {Singular \\ perturbations\\ of \\ functional-differential\\ equations};
\draw (113.99999999999997, -14.8) rectangle (117.84999999999997,-17.400000000000002);
\draw(81.89999999999998, -17.5) node[anchor=north west,align=left] {Inverse problems\\ for \\ functional-differential\\ equations};
\draw (81.89999999999998, -17.5) rectangle (85.74999999999997,-19.6);
\draw(85.84999999999998, -17.5) node[anchor=north west,align=left] {Hybrid systems\\ of \\ functional-differential\\ equations};
\draw (85.84999999999998, -17.5) rectangle (89.69999999999997,-19.6);
\draw(89.79999999999998, -17.5) node[anchor=north west,align=left] {Control problems\\ for \\ functional-differential\\ equations};
\draw (89.79999999999998, -17.5) rectangle (93.64999999999998,-19.6);
\draw(93.74999999999997, -17.5) node[anchor=north west,align=left] {Perturbations\\ of \\ functional-differential\\ equations};
\draw (93.74999999999997, -17.5) rectangle (97.34999999999997,-19.6);
\draw(97.44999999999997, -17.5) node[anchor=north west,align=left] {Averaging \\ for \\ functional-differential\\ equations};
\draw (97.44999999999997, -17.5) rectangle (101.04999999999997,-19.6);
\draw(101.14999999999998, -17.5) node[anchor=north west,align=left] {Boundary value\\ problems\\ for \\ functional-differential\\ equations};
\draw (101.14999999999998, -17.5) rectangle (104.49999999999997,-20.1);
\draw(104.59999999999998, -17.5) node[anchor=north west,align=left] {Stationary\\ solutions \\ of \\ functional-differential\\ equations};
\draw (104.59999999999998, -17.5) rectangle (107.94999999999997,-20.1);
\draw(120.69999999999997, -2) node[anchor=north west,align=left] {\large Qualitative theory for ordinary differential equations};
\draw (120.69999999999997, -2) rectangle (141.2,-21.7);
\draw(121.69999999999997, -3) node[anchor=north west,align=left] {Theory of limit cycles\\ of polynomial and \\ analytic vector fields\\ (existence, uniqueness,\\ bounds, Hilbert’s\\ 16th problem and \\ ramifications) for ordinary\\ differential equations};
\draw (121.69999999999997, -3) rectangle (128.29999999999998,-7.1);
\draw(128.39999999999998, -3) node[anchor=north west,align=left] {Ordinary differential\\ equations and \\ connections with real \\ algebraic geometry \\ (fewnomials, desingularization,\\ zeros of \\ abelian integrals, etc.)};
\draw (128.39999999999998, -3) rectangle (133.99999999999997,-6.6);
\draw(134.09999999999997, -3) node[anchor=north west,align=left] {Topological structure\\ of integral\\ curves, singular\\ points, limit \\ cycles of ordinary \\ differential equations};
\draw (134.09999999999997, -3) rectangle (139.44999999999996,-6.1);
\draw(121.69999999999997, -7.199999999999999) node[anchor=north west,align=left] {Transformation\\ and reduction\\ of ordinary \\ differential \\ equations and \\ systems, normal forms};
\draw (121.69999999999997, -7.199999999999999) rectangle (126.54999999999997,-10.299999999999999);
\draw(126.64999999999998, -7.199999999999999) node[anchor=north west,align=left] {Growth and \\ boundedness of\\ solutions to\\ ordinary \\ differential equations};
\draw (126.64999999999998, -7.199999999999999) rectangle (131.24999999999997,-9.799999999999999);
\draw(131.34999999999997, -7.199999999999999) node[anchor=north west,align=left] {Almost and \\ pseudo-almost periodic\\ solutions \\ to ordinary \\ differential equations};
\draw (131.34999999999997, -7.199999999999999) rectangle (135.94999999999996,-9.799999999999999);
\draw(136.04999999999998, -7.199999999999999) node[anchor=north west,align=left] {Oscillation theory,\\ zeros, \\ disconjugacy and \\ comparison theory for\\ ordinary \\ differential equations};
\draw (136.04999999999998, -7.199999999999999) rectangle (140.39999999999998,-10.299999999999999);
\draw(121.69999999999997, -10.399999999999999) node[anchor=north west,align=left] {Nonlinear \\ oscillations and\\ coupled \\ oscillators for \\ ordinary \\ differential equations};
\draw (121.69999999999997, -10.399999999999999) rectangle (126.04999999999997,-13.499999999999998);
\draw(126.14999999999998, -10.399999999999999) node[anchor=north west,align=left] {Equivalence and\\ asymptotic \\ equivalence of\\ ordinary \\ differential equations};
\draw (126.14999999999998, -10.399999999999999) rectangle (130.49999999999997,-12.999999999999998);
\draw(130.59999999999997, -10.399999999999999) node[anchor=north west,align=left] {Qualitative \\ investigation and \\ simulation of \\ ordinary differential\\ equation models};
\draw (130.59999999999997, -10.399999999999999) rectangle (134.94999999999996,-12.999999999999998);
\draw(135.04999999999998, -10.399999999999999) node[anchor=north west,align=left] {Averaging \\ method for \\ ordinary \\ differential equations};
\draw (135.04999999999998, -10.399999999999999) rectangle (139.14999999999998,-12.499999999999998);
\draw(121.69999999999997, -13.599999999999998) node[anchor=north west,align=left] {Homoclinic and\\ heteroclinic\\ solutions to \\ ordinary \\ differential equations};
\draw (121.69999999999997, -13.599999999999998) rectangle (125.79999999999997,-16.2);
\draw(125.89999999999998, -13.599999999999998) node[anchor=north west,align=left] {Bifurcation \\ theory for \\ ordinary differential\\ equations};
\draw (125.89999999999998, -13.599999999999998) rectangle (129.74999999999997,-15.699999999999998);
\draw(129.84999999999997, -13.599999999999998) node[anchor=north west,align=left] {Relaxation\\ oscillations\\ for ordinary\\ differential\\ equations};
\draw (129.84999999999997, -13.599999999999998) rectangle (133.69999999999996,-16.2);
\draw(133.79999999999998, -13.599999999999998) node[anchor=north west,align=left] {Complex behavior\\ and chaotic\\ systems of \\ ordinary differential\\ equations};
\draw (133.79999999999998, -13.599999999999998) rectangle (137.64999999999998,-16.2);
\draw(137.74999999999997, -13.599999999999998) node[anchor=north west,align=left] {Symmetries,\\ invariants\\ of ordinary\\ differential\\ equations};
\draw (137.74999999999997, -13.599999999999998) rectangle (141.09999999999997,-16.2);
\draw(121.69999999999997, -16.299999999999997) node[anchor=north west,align=left] {Invariant\\ manifolds\\ for ordinary\\ differential\\ equations};
\draw (121.69999999999997, -16.299999999999997) rectangle (125.54999999999997,-18.9);
\draw(125.64999999999998, -16.299999999999997) node[anchor=north west,align=left] {Monotone \\ systems involving\\ ordinary\\ differential\\ equations};
\draw (125.64999999999998, -16.299999999999997) rectangle (129.24999999999997,-18.9);
\draw(129.34999999999997, -16.299999999999997) node[anchor=north west,align=left] {Periodic \\ solutions to \\ ordinary differential\\ equations};
\draw (129.34999999999997, -16.299999999999997) rectangle (132.94999999999996,-18.4);
\draw(133.04999999999998, -16.299999999999997) node[anchor=north west,align=left] {Multifrequency\\ systems\\ of ordinary\\ differential\\ equations};
\draw (133.04999999999998, -16.299999999999997) rectangle (136.39999999999998,-18.9);
\draw(136.49999999999997, -16.299999999999997) node[anchor=north west,align=left] {Hysteresis\\ for ordinary\\ differential\\ equations};
\draw (136.49999999999997, -16.299999999999997) rectangle (139.84999999999997,-18.4);
\draw(121.69999999999997, -18.999999999999996) node[anchor=north west,align=left] {Ordinary \\ differential\\ equations \\ and systems\\ on manifolds};
\draw (121.69999999999997, -18.999999999999996) rectangle (124.79999999999997,-21.599999999999998);
\draw(80.89999999999998, -20.300000000000004) node[anchor=north west,align=left] {\large History of ordinary\\ differential equations};
\draw (80.89999999999998, -20.300000000000004) rectangle (87.69999999999997,-21.400000000000006);
\draw(80.89999999999998, -21.8) node[anchor=north west,align=left] {\large Boundary value problems for ordinary differential equations};
\draw (80.89999999999998, -21.8) rectangle (103.14999999999998,-35.1);
\draw(81.89999999999998, -22.8) node[anchor=north west,align=left] {Linear boundary\\ value problems \\ for ordinary \\ differential equations\\ with nonlinear\\ dependence on\\ the spectral parameter};
\draw (81.89999999999998, -22.8) rectangle (87.49999999999997,-26.400000000000002);
\draw(87.59999999999998, -22.8) node[anchor=north west,align=left] {Parameter dependent\\ boundary \\ value problems \\ for ordinary \\ differential equations};
\draw (87.59999999999998, -22.8) rectangle (92.44999999999997,-25.400000000000002);
\draw(92.54999999999998, -22.8) node[anchor=north west,align=left] {Nonlocal and\\ multipoint \\ boundary value\\ problems for\\ ordinary \\ differential equations};
\draw (92.54999999999998, -22.8) rectangle (97.14999999999998,-25.900000000000002);
\draw(97.24999999999997, -22.8) node[anchor=north west,align=left] {Singular nonlinear\\ boundary \\ value problems for\\ ordinary \\ differential equations};
\draw (97.24999999999997, -22.8) rectangle (101.84999999999997,-25.400000000000002);
\draw(81.89999999999998, -26.5) node[anchor=north west,align=left] {Nonlinear \\ boundary value \\ problems for \\ ordinary \\ differential equations};
\draw (81.89999999999998, -26.5) rectangle (86.24999999999997,-29.1);
\draw(86.34999999999998, -26.5) node[anchor=north west,align=left] {Special ordinary\\ differential\\ equations\\ (Mathieu, \\ Hill, Bessel, etc.)};
\draw (86.34999999999998, -26.5) rectangle (90.69999999999997,-29.1);
\draw(90.79999999999998, -26.5) node[anchor=north west,align=left] {Boundary value\\ problems with\\ impulses for\\ ordinary \\ differential equations};
\draw (90.79999999999998, -26.5) rectangle (95.14999999999998,-29.1);
\draw(95.24999999999997, -26.5) node[anchor=north west,align=left] {Boundary value \\ problems on infinite\\ intervals for\\ ordinary \\ differential equations};
\draw (95.24999999999997, -26.5) rectangle (99.59999999999997,-29.1);
\draw(81.89999999999998, -29.200000000000003) node[anchor=north west,align=left] {Boundary value\\ problems on\\ graphs and \\ networks for \\ ordinary \\ differential equations};
\draw (81.89999999999998, -29.200000000000003) rectangle (86.24999999999997,-32.300000000000004);
\draw(86.34999999999998, -29.200000000000003) node[anchor=north west,align=left] {Applications of\\ boundary value \\ problems involving\\ ordinary \\ differential equations};
\draw (86.34999999999998, -29.200000000000003) rectangle (90.69999999999997,-31.800000000000004);
\draw(90.79999999999998, -29.200000000000003) node[anchor=north west,align=left] {Boundary \\ eigenvalue \\ problems for \\ ordinary \\ differential equations};
\draw (90.79999999999998, -29.200000000000003) rectangle (94.89999999999998,-31.800000000000004);
\draw(94.99999999999997, -29.200000000000003) node[anchor=north west,align=left] {Positive solutions\\ to nonlinear\\ boundary value\\ problems for\\ ordinary \\ differential equations};
\draw (94.99999999999997, -29.200000000000003) rectangle (99.09999999999997,-32.300000000000004);
\draw(99.19999999999997, -29.200000000000003) node[anchor=north west,align=left] {Weyl theory and\\ its generalizations\\ for \\ ordinary differential\\ equations};
\draw (99.19999999999997, -29.200000000000003) rectangle (103.04999999999997,-31.800000000000004);
\draw(81.89999999999998, -32.400000000000006) node[anchor=north west,align=left] {Sturm-Liouville\\ theory};
\draw (81.89999999999998, -32.400000000000006) rectangle (85.74999999999997,-34.00000000000001);
\draw(85.84999999999998, -32.400000000000006) node[anchor=north west,align=left] {Linear boundary\\ value \\ problems for \\ ordinary differential\\ equations};
\draw (85.84999999999998, -32.400000000000006) rectangle (89.44999999999997,-35.00000000000001);
\draw(89.54999999999998, -32.400000000000006) node[anchor=north west,align=left] {Green’s functions\\ for \\ ordinary differential\\ equations};
\draw (89.54999999999998, -32.400000000000006) rectangle (93.14999999999998,-34.50000000000001);
\draw(103.24999999999997, -21.8) node[anchor=north west,align=left] {\large Ordinary differential equations in the complex domain};
\draw (103.24999999999997, -21.8) rectangle (123.29999999999997,-39.300000000000004);
\draw(104.24999999999997, -22.8) node[anchor=north west,align=left] {Algebraic aspects \\ (differential-algebraic,\\ hypertranscendence,\\ group-theoretical)\\ of ordinary \\ differential equations\\ in the complex domain};
\draw (104.24999999999997, -22.8) rectangle (110.34999999999997,-26.400000000000002);
\draw(110.44999999999997, -22.8) node[anchor=north west,align=left] {Singular perturbation\\ problems for \\ ordinary differential\\ equations in the \\ complex domain (complex\\ WKB, turning \\ points, steepest descent)};
\draw (110.44999999999997, -22.8) rectangle (116.04999999999997,-26.400000000000002);
\draw(116.14999999999998, -22.8) node[anchor=north west,align=left] {Nonlinear \\ ordinary differential\\ equations\\ and systems in\\ the complex domain};
\draw (116.14999999999998, -22.8) rectangle (120.99999999999997,-25.400000000000002);
\draw(104.24999999999997, -26.5) node[anchor=north west,align=left] {Entire and \\ meromorphic solutions\\ to ordinary\\ differential\\ equations in\\ the complex domain};
\draw (104.24999999999997, -26.5) rectangle (109.09999999999997,-29.6);
\draw(109.19999999999997, -26.5) node[anchor=north west,align=left] {Oscillation, \\ growth of solutions\\ to ordinary\\ differential\\ equations in\\ the complex domain};
\draw (109.19999999999997, -26.5) rectangle (114.04999999999997,-29.6);
\draw(114.14999999999998, -26.5) node[anchor=north west,align=left] {Formal solutions\\ and transform \\ techniques for \\ ordinary differential\\ equations in\\ the complex domain};
\draw (114.14999999999998, -26.5) rectangle (118.99999999999997,-29.6);
\draw(119.09999999999997, -26.5) node[anchor=north west,align=left] {Linear ordinary\\ differential\\ equations and\\ systems in the\\ complex domain};
\draw (119.09999999999997, -26.5) rectangle (123.19999999999996,-29.1);
\draw(104.24999999999997, -29.700000000000003) node[anchor=north west,align=left] {Singularities, \\ monodromy and local \\ behavior of solutions\\ to ordinary \\ differential equations\\ in the complex \\ domain, normal forms};
\draw (104.24999999999997, -29.700000000000003) rectangle (109.09999999999997,-33.300000000000004);
\draw(109.19999999999997, -29.700000000000003) node[anchor=north west,align=left] {Stokes phenomena\\ and connection \\ problems (linear and\\ nonlinear) for\\ ordinary differential\\ equations in\\ the complex domain};
\draw (109.19999999999997, -29.700000000000003) rectangle (114.04999999999997,-33.300000000000004);
\draw(114.14999999999998, -29.700000000000003) node[anchor=north west,align=left] {Inverse problems \\ (Riemann-Hilbert, \\ inverse differential\\ Galois, etc.) for\\ ordinary differential\\ equations in\\ the complex domain};
\draw (114.14999999999998, -29.700000000000003) rectangle (118.99999999999997,-33.300000000000004);
\draw(119.09999999999997, -29.700000000000003) node[anchor=north west,align=left] {Asymptotics and\\ summation methods\\ for ordinary\\ differential\\ equations in the\\ complex domain};
\draw (119.09999999999997, -29.700000000000003) rectangle (123.19999999999996,-32.800000000000004);
\draw(104.24999999999997, -33.400000000000006) node[anchor=north west,align=left] {Painlevé and other\\ special ordinary\\ differential equations\\ in the complex\\ domain; classification,\\ hierarchies};
\draw (104.24999999999997, -33.400000000000006) rectangle (109.09999999999997,-36.50000000000001);
\draw(109.19999999999997, -33.400000000000006) node[anchor=north west,align=left] {Isomonodromic\\ deformations\\ for ordinary\\ differential\\ equations in\\ the complex domain};
\draw (109.19999999999997, -33.400000000000006) rectangle (114.04999999999997,-36.50000000000001);
\draw(114.14999999999998, -33.400000000000006) node[anchor=north west,align=left] {Topological \\ structure of \\ trajectories of \\ ordinary differential\\ equations in\\ the complex domain};
\draw (114.14999999999998, -33.400000000000006) rectangle (118.99999999999997,-36.50000000000001);
\draw(119.09999999999997, -33.400000000000006) node[anchor=north west,align=left] {Ordinary \\ differential \\ equations on \\ complex manifolds};
\draw (119.09999999999997, -33.400000000000006) rectangle (123.19999999999996,-35.50000000000001);
\draw(104.24999999999997, -36.6) node[anchor=north west,align=left] {Spectral theory\\ for ordinary\\ differential\\ operators in \\ the complex domain};
\draw (104.24999999999997, -36.6) rectangle (108.59999999999997,-39.2);
\draw(80.89999999999998, -35.2) node[anchor=north west,align=left] {\large Ordinary differential equations and systems with randomness};
\draw (80.89999999999998, -35.2) rectangle (99.78999999999998,-38.900000000000006);
\draw(81.89999999999998, -36.2) node[anchor=north west,align=left] {Bifurcation of\\ solutions to \\ ordinary differential\\ equations\\ involving randomness};
\draw (81.89999999999998, -36.2) rectangle (86.99999999999997,-38.800000000000004);
\draw(87.09999999999998, -36.2) node[anchor=north west,align=left] {Resonance phenomena\\ for ordinary\\ differential\\ equations \\ involving randomness};
\draw (87.09999999999998, -36.2) rectangle (91.19999999999997,-38.800000000000004);
\draw(91.29999999999998, -36.2) node[anchor=north west,align=left] {Ordinary \\ differential\\ equations \\ and systems\\ with randomness};
\draw (91.29999999999998, -36.2) rectangle (95.14999999999998,-38.800000000000004);
\draw(123.39999999999998, -21.8) node[anchor=north west,align=left] {\large Control problems including ordinary differential equations};
\draw (123.39999999999998, -21.8) rectangle (141.98,-25.5);
\draw(124.39999999999998, -22.8) node[anchor=north west,align=left] {Chaos control\\ for problems\\ involving \\ ordinary \\ differential equations};
\draw (124.39999999999998, -22.8) rectangle (128.49999999999997,-25.400000000000002);
\draw(128.59999999999997, -22.8) node[anchor=north west,align=left] {Bifurcation \\ control of \\ ordinary differential\\ equations};
\draw (128.59999999999997, -22.8) rectangle (132.44999999999996,-24.900000000000002);
\draw(132.54999999999998, -22.8) node[anchor=north west,align=left] {Control \\ problems involving\\ ordinary\\ differential\\ equations};
\draw (132.54999999999998, -22.8) rectangle (136.14999999999998,-25.400000000000002);
\draw(136.24999999999997, -22.8) node[anchor=north west,align=left] {Stabilization\\ of solutions\\ to ordinary\\ differential\\ equations};
\draw (136.24999999999997, -22.8) rectangle (139.59999999999997,-25.400000000000002);
\draw(123.39999999999998, -25.6) node[anchor=north west,align=left] {\large Dynamic equations on time scales or measure chains};
\draw (123.39999999999998, -25.6) rectangle (139.49999999999997,-28.8);
\draw(124.39999999999998, -26.6) node[anchor=north west,align=left] {Dynamic \\ equations on time\\ scales or\\ measure chains};
\draw (124.39999999999998, -26.6) rectangle (127.99999999999997,-28.700000000000003);
\draw(123.39999999999998, -28.900000000000002) node[anchor=north west,align=left] {\large Differential equations in abstract spaces};
\draw (123.39999999999998, -28.900000000000002) rectangle (136.70999999999998,-32.1);
\draw(124.39999999999998, -29.900000000000002) node[anchor=north west,align=left] {Linear \\ differential \\ equations in\\ abstract spaces};
\draw (124.39999999999998, -29.900000000000002) rectangle (128.24999999999997,-32.0);
\draw(128.34999999999997, -29.900000000000002) node[anchor=north west,align=left] {Nonlinear \\ differential \\ equations in\\ abstract spaces};
\draw (128.34999999999997, -29.900000000000002) rectangle (132.19999999999996,-32.0);
\draw(132.29999999999998, -29.900000000000002) node[anchor=north west,align=left] {Evolution\\ inclusions};
\draw (132.29999999999998, -29.900000000000002) rectangle (135.89999999999998,-31.500000000000004);
\draw(80.89999999999998, -39.400000000000006) node[anchor=north west,align=left] {\large Asymptotic theory for ordinary differential equations};
\draw (80.89999999999998, -39.400000000000006) rectangle (100.44999999999997,-46.300000000000004);
\draw(81.89999999999998, -40.400000000000006) node[anchor=north west,align=left] {Perturbations,\\ asymptotics \\ of solutions to\\ ordinary \\ differential equations};
\draw (81.89999999999998, -40.400000000000006) rectangle (86.49999999999997,-43.00000000000001);
\draw(86.59999999999998, -40.400000000000006) node[anchor=north west,align=left] {Singular \\ perturbations, general\\ theory for\\ ordinary \\ differential equations};
\draw (86.59999999999998, -40.400000000000006) rectangle (91.19999999999997,-43.00000000000001);
\draw(91.29999999999998, -40.400000000000006) node[anchor=north west,align=left] {Singular \\ perturbations, turning\\ point theory,\\ WKB methods for\\ ordinary \\ differential equations};
\draw (91.29999999999998, -40.400000000000006) rectangle (95.89999999999998,-43.50000000000001);
\draw(95.99999999999997, -40.400000000000006) node[anchor=north west,align=left] {Asymptotic \\ expansions of \\ solutions to \\ ordinary \\ differential equations};
\draw (95.99999999999997, -40.400000000000006) rectangle (100.34999999999997,-43.00000000000001);
\draw(81.89999999999998, -43.60000000000001) node[anchor=north west,align=left] {Canard solutions\\ to \\ ordinary differential\\ equations};
\draw (81.89999999999998, -43.60000000000001) rectangle (85.74999999999997,-45.70000000000001);
\draw(85.84999999999998, -43.60000000000001) node[anchor=north west,align=left] {Methods of \\ nonstandard \\ analysis for \\ ordinary differential\\ equations};
\draw (85.84999999999998, -43.60000000000001) rectangle (89.69999999999997,-46.20000000000001);
\draw(89.79999999999998, -43.60000000000001) node[anchor=north west,align=left] {Multiple \\ scale methods\\ for ordinary\\ differential\\ equations};
\draw (89.79999999999998, -43.60000000000001) rectangle (93.39999999999998,-46.20000000000001);
\draw(100.54999999999998, -39.400000000000006) node[anchor=north west,align=left] {\large General theory for ordinary differential equations};
\draw (100.54999999999998, -39.400000000000006) rectangle (119.59999999999998,-54.400000000000006);
\draw(101.54999999999998, -40.400000000000006) node[anchor=north west,align=left] {Initial value \\ problems, existence, \\ uniqueness, continuous\\ dependence and\\ continuation of \\ solutions to ordinary\\ differential equations};
\draw (101.54999999999998, -40.400000000000006) rectangle (107.89999999999998,-44.00000000000001);
\draw(107.99999999999999, -40.400000000000006) node[anchor=north west,align=left] {Generalized \\ ordinary differential\\ equations \\ (measure-differential\\ equations, \\ set-valued differential\\ equations, etc.)};
\draw (107.99999999999999, -40.400000000000006) rectangle (113.34999999999998,-44.00000000000001);
\draw(113.44999999999999, -40.400000000000006) node[anchor=north west,align=left] {Analytical theory\\ of ordinary \\ differential \\ equations: series, \\ transformations, \\ transforms, \\ operational calculus, etc.};
\draw (113.44999999999999, -40.400000000000006) rectangle (118.79999999999998,-44.00000000000001);
\draw(101.54999999999998, -44.10000000000001) node[anchor=north west,align=left] {Explicit \\ solutions, first\\ integrals of\\ ordinary \\ differential equations};
\draw (101.54999999999998, -44.10000000000001) rectangle (106.14999999999998,-46.70000000000001);
\draw(106.24999999999999, -44.10000000000001) node[anchor=north west,align=left] {Implicit ordinary\\ differential\\ equations,\\ differential-algebraic\\ equations};
\draw (106.24999999999999, -44.10000000000001) rectangle (110.84999999999998,-46.70000000000001);
\draw(110.94999999999999, -44.10000000000001) node[anchor=north west,align=left] {Fractional ordinary\\ differential\\ equations and \\ fractional \\ differential inclusions};
\draw (110.94999999999999, -44.10000000000001) rectangle (115.29999999999998,-46.70000000000001);
\draw(115.39999999999998, -44.10000000000001) node[anchor=north west,align=left] {Geometric \\ methods in \\ ordinary \\ differential equations};
\draw (115.39999999999998, -44.10000000000001) rectangle (119.49999999999997,-46.20000000000001);
\draw(101.54999999999998, -46.800000000000004) node[anchor=north west,align=left] {Nonlinear \\ ordinary differential\\ equations\\ and systems,\\ general theory};
\draw (101.54999999999998, -46.800000000000004) rectangle (105.89999999999998,-49.400000000000006);
\draw(105.99999999999999, -46.800000000000004) node[anchor=north west,align=left] {Theoretical \\ approximation of\\ solutions to\\ ordinary \\ differential equations};
\draw (105.99999999999999, -46.800000000000004) rectangle (110.34999999999998,-49.400000000000006);
\draw(110.44999999999999, -46.800000000000004) node[anchor=north west,align=left] {Linear \\ ordinary \\ differential \\ equations and \\ systems, general};
\draw (110.44999999999999, -46.800000000000004) rectangle (114.29999999999998,-49.400000000000006);
\draw(114.39999999999998, -46.800000000000004) node[anchor=north west,align=left] {Differential \\ inequalities \\ involving functions\\ of a single\\ real variable};
\draw (114.39999999999998, -46.800000000000004) rectangle (118.24999999999997,-49.400000000000006);
\draw(101.54999999999998, -49.50000000000001) node[anchor=north west,align=left] {Inverse \\ problems involving\\ ordinary\\ differential\\ equations};
\draw (101.54999999999998, -49.50000000000001) rectangle (105.14999999999998,-52.10000000000001);
\draw(105.24999999999999, -49.50000000000001) node[anchor=north west,align=left] {Ordinary \\ differential \\ equations of \\ infinite order};
\draw (105.24999999999999, -49.50000000000001) rectangle (108.59999999999998,-51.60000000000001);
\draw(108.69999999999999, -49.50000000000001) node[anchor=north west,align=left] {Hybrid systems\\ of ordinary\\ differential\\ equations};
\draw (108.69999999999999, -49.50000000000001) rectangle (112.04999999999998,-51.60000000000001);
\draw(112.14999999999998, -49.50000000000001) node[anchor=north west,align=left] {Ordinary\\ lattice \\ differential\\ equations};
\draw (112.14999999999998, -49.50000000000001) rectangle (115.24999999999997,-51.60000000000001);
\draw(115.34999999999998, -49.50000000000001) node[anchor=north west,align=left] {Discontinuous\\ ordinary\\ differential\\ equations};
\draw (115.34999999999998, -49.50000000000001) rectangle (118.44999999999997,-51.60000000000001);
\draw(101.54999999999998, -52.2) node[anchor=north west,align=left] {Ordinary \\ differential\\ equations \\ with impulses};
\draw (101.54999999999998, -52.2) rectangle (104.64999999999998,-54.300000000000004);
\draw(104.74999999999999, -52.2) node[anchor=north west,align=left] {Ordinary \\ differential\\ inclusions};
\draw (104.74999999999999, -52.2) rectangle (107.84999999999998,-53.800000000000004);
\draw(107.94999999999999, -52.2) node[anchor=north west,align=left] {Fuzzy \\ ordinary \\ differential\\ equations};
\draw (107.94999999999999, -52.2) rectangle (110.54999999999998,-54.300000000000004);
\draw(119.69999999999997, -39.400000000000006) node[anchor=north west,align=left] {\large Stability theory for ordinary differential equations};
\draw (119.69999999999997, -39.400000000000006) rectangle (138.49999999999997,-49.00000000000001);
\draw(120.69999999999997, -40.400000000000006) node[anchor=north west,align=left] {Stability of\\ manifolds of\\ solutions to\\ ordinary \\ differential equations};
\draw (120.69999999999997, -40.400000000000006) rectangle (125.29999999999997,-43.00000000000001);
\draw(125.39999999999998, -40.400000000000006) node[anchor=north west,align=left] {Asymptotic \\ properties of \\ solutions to \\ ordinary \\ differential equations};
\draw (125.39999999999998, -40.400000000000006) rectangle (129.74999999999997,-43.00000000000001);
\draw(129.84999999999997, -40.400000000000006) node[anchor=north west,align=left] {Dichotomy, \\ trichotomy of \\ solutions to \\ ordinary \\ differential equations};
\draw (129.84999999999997, -40.400000000000006) rectangle (134.19999999999996,-43.00000000000001);
\draw(134.29999999999998, -40.400000000000006) node[anchor=north west,align=left] {Characteristic\\ and Lyapunov\\ exponents of \\ ordinary \\ differential equations};
\draw (134.29999999999998, -40.400000000000006) rectangle (138.39999999999998,-43.00000000000001);
\draw(120.69999999999997, -43.10000000000001) node[anchor=north west,align=left] {Global stability\\ of \\ solutions to \\ ordinary \\ differential equations};
\draw (120.69999999999997, -43.10000000000001) rectangle (124.79999999999997,-45.70000000000001);
\draw(124.89999999999998, -43.10000000000001) node[anchor=north west,align=left] {Structural \\ stability and analogous\\ concepts \\ of solutions to\\ ordinary \\ differential equations};
\draw (124.89999999999998, -43.10000000000001) rectangle (128.99999999999997,-46.20000000000001);
\draw(129.09999999999997, -43.10000000000001) node[anchor=north west,align=left] {Synchronization\\ of solutions\\ to \\ ordinary differential\\ equations};
\draw (129.09999999999997, -43.10000000000001) rectangle (132.94999999999996,-45.70000000000001);
\draw(133.04999999999998, -43.10000000000001) node[anchor=north west,align=left] {Perturbations\\ of ordinary\\ differential\\ equations};
\draw (133.04999999999998, -43.10000000000001) rectangle (136.89999999999998,-45.20000000000001);
\draw(120.69999999999997, -46.300000000000004) node[anchor=north west,align=left] {Attractors\\ of solutions\\ to ordinary\\ differential\\ equations};
\draw (120.69999999999997, -46.300000000000004) rectangle (124.29999999999997,-48.900000000000006);
\draw(124.39999999999998, -46.300000000000004) node[anchor=north west,align=left] {Singular \\ perturbations\\ of ordinary\\ differential\\ equations};
\draw (124.39999999999998, -46.300000000000004) rectangle (127.74999999999997,-48.900000000000006);
\draw(127.84999999999998, -46.300000000000004) node[anchor=north west,align=left] {Stability \\ of solutions\\ to ordinary\\ differential\\ equations};
\draw (127.84999999999998, -46.300000000000004) rectangle (131.2,-48.900000000000006);
\draw(80.89999999999998, -54.50000000000001) node[anchor=north west,align=left] {\large Ordinary differential operators};
\draw (80.89999999999998, -54.50000000000001) rectangle (92.79999999999998,-67.80000000000001);
\draw(81.89999999999998, -55.50000000000001) node[anchor=north west,align=left] {Numerical approximation\\ of eigenvalues\\ and of other\\ parts of the \\ spectrum of ordinary\\ differential operators};
\draw (81.89999999999998, -55.50000000000001) rectangle (87.74999999999997,-58.60000000000001);
\draw(87.84999999999998, -55.50000000000001) node[anchor=north west,align=left] {Scattering theory,\\ inverse \\ scattering involving\\ ordinary \\ differential operators};
\draw (87.84999999999998, -55.50000000000001) rectangle (92.69999999999997,-58.10000000000001);
\draw(81.89999999999998, -58.70000000000001) node[anchor=north west,align=left] {Eigenfunctions, \\ eigenfunction \\ expansions, completeness\\ of eigenfunctions\\ of ordinary\\ differential operators};
\draw (81.89999999999998, -58.70000000000001) rectangle (87.49999999999997,-61.80000000000001);
\draw(87.59999999999998, -58.70000000000001) node[anchor=north west,align=left] {Eigenvalues, \\ estimation of \\ eigenvalues, upper and\\ lower bounds \\ of ordinary \\ differential operators};
\draw (87.59999999999998, -58.70000000000001) rectangle (91.94999999999997,-61.80000000000001);
\draw(81.89999999999998, -61.900000000000006) node[anchor=north west,align=left] {Asymptotic distribution\\ of eigenvalues,\\ asymptotic \\ theory of eigenfunctions\\ for ordinary \\ differential operators};
\draw (81.89999999999998, -61.900000000000006) rectangle (87.24999999999997,-65.0);
\draw(87.34999999999998, -61.900000000000006) node[anchor=north west,align=left] {Particular \\ ordinary differential\\ operators\\ (Dirac, \\ one-dimensional \\ Schrödinger, etc.)};
\draw (87.34999999999998, -61.900000000000006) rectangle (91.19999999999997,-65.0);
\draw(81.89999999999998, -65.10000000000001) node[anchor=north west,align=left] {General \\ spectral theory\\ of ordinary\\ differential\\ operators};
\draw (81.89999999999998, -65.10000000000001) rectangle (85.49999999999997,-67.7);
\draw(85.59999999999998, -65.10000000000001) node[anchor=north west,align=left] {Nonlinear\\ ordinary \\ differential\\ operators};
\draw (85.59999999999998, -65.10000000000001) rectangle (88.44999999999997,-67.2);
\draw(1, -46.1) node[anchor=north west,align=left] {\LARGE General algebraic systems};
\draw (1, -46.1) rectangle (20.5,-67.1);
\draw(2, -47.1) node[anchor=north west,align=left] {\large Other classes of algebras};
\draw (2, -47.1) rectangle (11.85,-52.0);
\draw(3, -48.1) node[anchor=north west,align=left] {Quasivarieties};
\draw (3, -48.1) rectangle (6.6,-49.2);
\draw(6.699999999999999, -48.1) node[anchor=north west,align=left] {Natural \\ dualities for\\ classes\\ of algebras};
\draw (6.699999999999999, -48.1) rectangle (9.549999999999999,-50.2);
\draw(9.649999999999999, -48.1) node[anchor=north west,align=left] {Categories\\ of\\ algebras};
\draw (9.649999999999999, -48.1) rectangle (11.749999999999998,-49.7);
\draw(3, -50.300000000000004) node[anchor=north west,align=left] {Axiomatic\\ model\\ classes};
\draw (3, -50.300000000000004) rectangle (5.35,-51.900000000000006);
\draw(11.95, -47.1) node[anchor=north west,align=left] {\large Computational methods for\\ problems pertaining to\\ general algebraic systems};
\draw (11.95, -47.1) rectangle (19.99,-48.7);
\draw(11.95, -48.800000000000004) node[anchor=north west,align=left] {\large History of general\\ algebraic systems};
\draw (11.95, -48.800000000000004) rectangle (17.82,-49.900000000000006);
\draw(2, -52.1) node[anchor=north west,align=left] {\large Algebraic structures};
\draw (2, -52.1) rectangle (11.399999999999999,-67.0);
\draw(3, -53.1) node[anchor=north west,align=left] {Heterogeneousalgebras};
\draw (3, -53.1) rectangle (8.6,-54.7);
\draw(8.7, -53.1) node[anchor=north west,align=left] {Fuzzy \\ algebraic\\ structures};
\draw (8.7, -53.1) rectangle (11.299999999999999,-54.7);
\draw(3, -54.800000000000004) node[anchor=north west,align=left] {Automorphisms\\ and \\ endomorphisms\\ of algebraic\\ structures};
\draw (3, -54.800000000000004) rectangle (6.85,-57.400000000000006);
\draw(6.949999999999999, -54.800000000000004) node[anchor=north west,align=left] {Operations and\\ polynomials\\ in algebraic\\ structures,\\ primal algebras};
\draw (6.949999999999999, -54.800000000000004) rectangle (10.799999999999999,-57.400000000000006);
\draw(3, -57.5) node[anchor=north west,align=left] {Equational\\ compactness};
\draw (3, -57.5) rectangle (6.85,-59.1);
\draw(6.949999999999999, -57.5) node[anchor=north west,align=left] {Word problems\\ (aspects\\ of algebraic\\ structures)};
\draw (6.949999999999999, -57.5) rectangle (10.549999999999999,-59.6);
\draw(3, -59.7) node[anchor=north west,align=left] {Applications\\ of universal\\ algebra in \\ computer science};
\draw (3, -59.7) rectangle (6.6,-61.800000000000004);
\draw(6.699999999999999, -59.7) node[anchor=north west,align=left] {Structure\\ theory \\ of algebraic\\ structures};
\draw (6.699999999999999, -59.7) rectangle (10.049999999999999,-61.800000000000004);
\draw(3, -61.900000000000006) node[anchor=north west,align=left] {Subalgebras,\\ congruence\\ relations};
\draw (3, -61.900000000000006) rectangle (6.35,-63.50000000000001);
\draw(6.449999999999999, -61.900000000000006) node[anchor=north west,align=left] {Infinitary\\ algebras};
\draw (6.449999999999999, -61.900000000000006) rectangle (9.799999999999999,-63.50000000000001);
\draw(3, -63.6) node[anchor=north west,align=left] {Relational\\ systems,\\ laws of\\ composition};
\draw (3, -63.6) rectangle (6.1,-65.7);
\draw(6.199999999999999, -63.6) node[anchor=north west,align=left] {Partial\\ algebras};
\draw (6.199999999999999, -63.6) rectangle (8.549999999999999,-64.7);
\draw(8.649999999999999, -63.6) node[anchor=north west,align=left] {Finitary\\ algebras};
\draw (8.649999999999999, -63.6) rectangle (10.999999999999998,-64.7);
\draw(3, -65.8) node[anchor=north west,align=left] {Unary\\ algebras};
\draw (3, -65.8) rectangle (5.1,-66.89999999999999);
\draw(11.499999999999998, -52.1) node[anchor=north west,align=left] {\large Varieties};
\draw (11.499999999999998, -52.1) rectangle (20.4,-61.400000000000006);
\draw(12.499999999999998, -53.1) node[anchor=north west,align=left] {Products, \\ amalgamated \\ products, and other\\ kinds of \\ limits and colimits};
\draw (12.499999999999998, -53.1) rectangle (17.099999999999998,-55.7);
\draw(17.199999999999996, -53.1) node[anchor=north west,align=left] {Equational\\ logic,\\ Mal’tsev\\ conditions};
\draw (17.199999999999996, -53.1) rectangle (19.799999999999997,-55.2);
\draw(12.499999999999998, -55.800000000000004) node[anchor=north west,align=left] {Injectives,\\ projectives};
\draw (12.499999999999998, -55.800000000000004) rectangle (16.599999999999998,-57.400000000000006);
\draw(16.699999999999996, -55.800000000000004) node[anchor=north west,align=left] {Subdirect \\ products and\\ subdirect\\ irreducibility};
\draw (16.699999999999996, -55.800000000000004) rectangle (20.299999999999997,-57.900000000000006);
\draw(12.499999999999998, -58.0) node[anchor=north west,align=left] {Congruence\\ modularity,\\ congruence\\ distributivity};
\draw (12.499999999999998, -58.0) rectangle (16.349999999999998,-60.1);
\draw(16.449999999999996, -58.0) node[anchor=north west,align=left] {Lattices\\ of\\ varieties};
\draw (16.449999999999996, -58.0) rectangle (18.799999999999997,-59.6);
\draw(12.499999999999998, -60.2) node[anchor=north west,align=left] {Free \\ algebras};
\draw (12.499999999999998, -60.2) rectangle (14.349999999999998,-61.300000000000004);
\draw(1, -68.0) node[anchor=north west,align=left] {\LARGE Several complex variables and analytic spaces};
\draw (1, -68.0) rectangle (64.35,-127.9);
\draw(2, -69.0) node[anchor=north west,align=left] {\large Non-Archimedean analysis (should also be assigned at least one other classification number from Section 32-XX describing the type of problem)};
\draw (2, -69.0) rectangle (46.31,-73.7);
\draw(3, -70.0) node[anchor=north west,align=left] {Non-Archimedean \\ analysis (should also be\\ assigned at least \\ one other classification\\ number from Section\\ 32-XX describing\\ the type of problem)};
\draw (3, -70.0) rectangle (8.35,-73.6);
\draw(46.410000000000004, -69.0) node[anchor=north west,align=left] {\large Complex spaces with a group of automorphisms};
\draw (46.410000000000004, -69.0) rectangle (62.260000000000005,-75.4);
\draw(47.410000000000004, -70.0) node[anchor=north west,align=left] {Complex \\ vector fields,\\ holomorphic\\ foliations,\\ \(\mathbb{C}\)-actions};
\draw (47.410000000000004, -70.0) rectangle (53.760000000000005,-72.6);
\draw(53.86, -70.0) node[anchor=north west,align=left] {Hermitian symmetric\\ spaces, bounded\\ symmetric \\ domains, Jordan \\ algebras (complex-analytic\\ aspects)};
\draw (53.86, -70.0) rectangle (58.21,-73.1);
\draw(58.31, -70.0) node[anchor=north west,align=left] {Complex Lie\\ groups, group\\ actions on\\ complex spaces};
\draw (58.31, -70.0) rectangle (62.160000000000004,-72.1);
\draw(47.410000000000004, -73.2) node[anchor=north west,align=left] {Automorphism\\ groups\\ of other \\ complex spaces};
\draw (47.410000000000004, -73.2) rectangle (50.760000000000005,-75.3);
\draw(50.86, -73.2) node[anchor=north west,align=left] {Almost \\ homogeneous\\ manifolds\\ and spaces};
\draw (50.86, -73.2) rectangle (53.96,-75.3);
\draw(54.06, -73.2) node[anchor=north west,align=left] {Automorphism\\ groups of\\ \(\mathbb{C}^n\) and affine\\ manifolds};
\draw (54.06, -73.2) rectangle (57.160000000000004,-75.3);
\draw(57.260000000000005, -73.2) node[anchor=north west,align=left] {Homogeneous\\ complex\\ manifolds};
\draw (57.260000000000005, -73.2) rectangle (60.11000000000001,-74.8);
\draw(2, -73.8) node[anchor=north west,align=left] {\large History of several\\ complex variables\\ and analytic spaces};
\draw (2, -73.8) rectangle (8.18,-75.39999999999999);
\draw(2, -75.5) node[anchor=north west,align=left] {\large Holomorphic functions of several complex variables};
\draw (2, -75.5) rectangle (21.05,-94.7);
\draw(3, -76.5) node[anchor=north west,align=left] {Other generalizations\\ of function theory\\ of one complex \\ variable (should also be\\ assigned at least \\ one classification \\ number from Section 30-XX)};
\draw (3, -76.5) rectangle (9.1,-80.1);
\draw(9.2, -76.5) node[anchor=north west,align=left] {Other spaces of \\ holomorphic functions of\\ several complex \\ variables (e.g., bounded\\ mean oscillation \\ (BMOA), vanishing mean\\ oscillation (VMOA))};
\draw (9.2, -76.5) rectangle (14.549999999999999,-80.1);
\draw(14.649999999999999, -76.5) node[anchor=north west,align=left] {Functional \\ analysis techniques\\ applied to \\ functions of several\\ complex variables};
\draw (14.649999999999999, -76.5) rectangle (19.75,-79.1);
\draw(3, -80.2) node[anchor=north west,align=left] {Normal families \\ of holomorphic \\ functions, mappings\\ of several complex\\ variables, and\\ related topics \\ (taut manifolds etc.)};
\draw (3, -80.2) rectangle (7.85,-83.8);
\draw(7.949999999999999, -80.2) node[anchor=north west,align=left] {Nevanlinna \\ theory; growth\\ estimates; \\ other inequalities\\ of several\\ complex variables};
\draw (7.949999999999999, -80.2) rectangle (12.799999999999999,-83.3);
\draw(12.899999999999999, -80.2) node[anchor=north west,align=left] {Integral \\ representations,\\ constructed \\ kernels (e.g.,\\ Cauchy, \\ Fantappiè-type kernels)};
\draw (12.899999999999999, -80.2) rectangle (17.75,-83.3);
\draw(17.849999999999998, -80.2) node[anchor=north west,align=left] {Entire \\ functions of \\ several complex\\ variables};
\draw (17.849999999999998, -80.2) rectangle (20.95,-82.3);
\draw(3, -83.9) node[anchor=north west,align=left] {Banach algebra\\ techniques \\ applied to functions\\ of several\\ complex variables};
\draw (3, -83.9) rectangle (7.85,-86.5);
\draw(7.949999999999999, -83.9) node[anchor=north west,align=left] {\(H^p\)-spaces, \\ Nevanlinna spaces\\ of functions\\ in several\\ complex variables};
\draw (7.949999999999999, -83.9) rectangle (12.549999999999999,-86.5);
\draw(12.649999999999999, -83.9) node[anchor=north west,align=left] {Singular \\ integrals of\\ functions \\ in several \\ complex variables};
\draw (12.649999999999999, -83.9) rectangle (16.75,-86.5);
\draw(16.849999999999998, -83.9) node[anchor=north west,align=left] {Polynomials\\ and rational\\ functions \\ of several \\ complex variables};
\draw (16.849999999999998, -83.9) rectangle (20.7,-86.5);
\draw(3, -86.6) node[anchor=north west,align=left] {Boundary behavior\\ of holomorphic\\ functions\\ of several \\ complex variables};
\draw (3, -86.6) rectangle (6.85,-89.19999999999999);
\draw(6.949999999999999, -86.6) node[anchor=north west,align=left] {Zero sets of\\ holomorphic\\ functions \\ of several \\ complex variables};
\draw (6.949999999999999, -86.6) rectangle (10.799999999999999,-89.19999999999999);
\draw(10.9, -86.6) node[anchor=north west,align=left] {Power series,\\ series of\\ functions of\\ several \\ complex variables};
\draw (10.9, -86.6) rectangle (14.5,-89.19999999999999);
\draw(14.6, -86.6) node[anchor=north west,align=left] {Holomorphic\\ functions of\\ several \\ complex variables};
\draw (14.6, -86.6) rectangle (18.2,-88.69999999999999);
\draw(3, -89.3) node[anchor=north west,align=left] {Special \\ families of \\ functions of\\ several \\ complex variables};
\draw (3, -89.3) rectangle (6.6,-91.89999999999999);
\draw(6.699999999999999, -89.3) node[anchor=north west,align=left] {Bloch functions,\\ normal\\ functions of\\ several \\ complex variables};
\draw (6.699999999999999, -89.3) rectangle (10.299999999999999,-91.89999999999999);
\draw(10.4, -89.3) node[anchor=north west,align=left] {Meromorphic\\ functions of\\ several \\ complex variables};
\draw (10.4, -89.3) rectangle (14.0,-91.39999999999999);
\draw(14.1, -89.3) node[anchor=north west,align=left] {Integral \\ representations;\\ canonical \\ kernels (Szegő,\\ Bergman, etc.)};
\draw (14.1, -89.3) rectangle (17.7,-91.89999999999999);
\draw(17.799999999999997, -89.3) node[anchor=north west,align=left] {Bergman \\ spaces of \\ functions in \\ several complex\\ variables};
\draw (17.799999999999997, -89.3) rectangle (20.9,-91.89999999999999);
\draw(3, -92.0) node[anchor=north west,align=left] {Algebras of\\ holomorphic\\ functions of\\ several \\ complex variables};
\draw (3, -92.0) rectangle (6.6,-94.6);
\draw(6.699999999999999, -92.0) node[anchor=north west,align=left] {Hyperfunctions};
\draw (6.699999999999999, -92.0) rectangle (10.299999999999999,-93.1);
\draw(10.4, -92.0) node[anchor=north west,align=left] {Multifunctions\\ of \\ several complex\\ variables};
\draw (10.4, -92.0) rectangle (13.75,-94.1);
\draw(13.85, -92.0) node[anchor=north west,align=left] {Harmonic \\ analysis of \\ several complex\\ variables};
\draw (13.85, -92.0) rectangle (17.2,-94.1);
\draw(17.299999999999997, -92.0) node[anchor=north west,align=left] {Residues\\ for several\\ complex\\ variables};
\draw (17.299999999999997, -92.0) rectangle (20.15,-94.1);
\draw(21.150000000000002, -75.5) node[anchor=north west,align=left] {\large Geometric convexity in several complex variables};
\draw (21.150000000000002, -75.5) rectangle (36.63,-81.9);
\draw(22.150000000000002, -76.5) node[anchor=north west,align=left] {\(q\)-convexity,\\ \(q\)-concavity};
\draw (22.150000000000002, -76.5) rectangle (27.5,-78.1);
\draw(27.6, -76.5) node[anchor=north west,align=left] {Analytical \\ consequences of \\ geometric convexity\\ (vanishing\\ theorems, etc.)};
\draw (27.6, -76.5) rectangle (31.950000000000003,-79.1);
\draw(32.05, -76.5) node[anchor=north west,align=left] {Other notions\\ of convexity\\ in relation\\ to several \\ complex variables};
\draw (32.05, -76.5) rectangle (35.9,-79.1);
\draw(22.150000000000002, -79.2) node[anchor=north west,align=left] {Finite-type\\ conditions \\ for the boundary\\ of a domain};
\draw (22.150000000000002, -79.2) rectangle (26.000000000000004,-81.3);
\draw(26.1, -79.2) node[anchor=north west,align=left] {Topological\\ consequences\\ of geometric\\ convexity};
\draw (26.1, -79.2) rectangle (29.700000000000003,-81.3);
\draw(29.800000000000004, -79.2) node[anchor=north west,align=left] {Invariant \\ metrics and \\ pseudodistances \\ in several \\ complex variables};
\draw (29.800000000000004, -79.2) rectangle (33.400000000000006,-81.8);
\draw(21.150000000000002, -82.0) node[anchor=north west,align=left] {\large Holomorphic mappings and correspondences};
\draw (21.150000000000002, -82.0) rectangle (36.0,-91.6);
\draw(22.150000000000002, -83.0) node[anchor=north west,align=left] {Picard-type \\ theorems and \\ generalizations\\ for several\\ complex variables};
\draw (22.150000000000002, -83.0) rectangle (26.75,-85.6);
\draw(26.85, -83.0) node[anchor=north west,align=left] {Iteration of \\ holomorphic maps, \\ fixed points of \\ holomorphic maps\\ and related \\ problems for several\\ complex variables};
\draw (26.85, -83.0) rectangle (31.450000000000003,-86.6);
\draw(31.55, -83.0) node[anchor=north west,align=left] {Holomorphic \\ mappings, (holomorphic)\\ embeddings \\ and related questions\\ in several\\ complex variables};
\draw (31.55, -83.0) rectangle (35.9,-86.1);
\draw(22.150000000000002, -86.7) node[anchor=north west,align=left] {Boundary \\ uniqueness of\\ mappings \\ in several \\ complex variables};
\draw (22.150000000000002, -86.7) rectangle (26.25,-89.3);
\draw(26.35, -86.7) node[anchor=north west,align=left] {Boundary \\ regularity of\\ mappings \\ in several \\ complex variables};
\draw (26.35, -86.7) rectangle (30.450000000000003,-89.3);
\draw(30.55, -86.7) node[anchor=north west,align=left] {Proper \\ holomorphic \\ mappings, \\ finiteness theorems};
\draw (30.55, -86.7) rectangle (34.4,-88.8);
\draw(22.150000000000002, -89.4) node[anchor=north west,align=left] {Meromorphic\\ mappings in\\ several complex\\ variables};
\draw (22.150000000000002, -89.4) rectangle (25.500000000000004,-91.5);
\draw(25.6, -89.4) node[anchor=north west,align=left] {Value \\ distribution \\ theory in higher\\ dimensions};
\draw (25.6, -89.4) rectangle (28.950000000000003,-91.5);
\draw(21.150000000000002, -91.7) node[anchor=north west,align=left] {\large Computational methods \\ for problems pertaining\\ to several complex \\ variables and analytic spaces};
\draw (21.150000000000002, -91.7) rectangle (28.880000000000003,-93.8);
\draw(36.73, -75.5) node[anchor=north west,align=left] {\large Differential operators in several variables};
\draw (36.73, -75.5) rectangle (51.58,-81.4);
\draw(37.73, -76.5) node[anchor=north west,align=left] {Pseudodifferential\\ operators in \\ several complex\\ variables};
\draw (37.73, -76.5) rectangle (43.83,-79.1);
\draw(43.92999999999999, -76.5) node[anchor=north west,align=left] {Other partial \\ differential \\ equations of complex\\ analysis in\\ several variables};
\draw (43.92999999999999, -76.5) rectangle (48.279999999999994,-79.1);
\draw(48.379999999999995, -76.5) node[anchor=north west,align=left] {Complex \\ Monge-Ampère\\ operators};
\draw (48.379999999999995, -76.5) rectangle (51.48,-78.1);
\draw(37.73, -79.2) node[anchor=north west,align=left] {Heat kernels\\ in several\\ complex\\ variables};
\draw (37.73, -79.2) rectangle (40.58,-81.3);
\draw(40.68, -79.2) node[anchor=north west,align=left] {\(\overline\partial\) and \\ \(\overline\partial\)-Neumann\\ operators};
\draw (40.68, -79.2) rectangle (43.28,-80.8);
\draw(43.379999999999995, -79.2) node[anchor=north west,align=left] {\(\overline\partial_b\) and \\ \(\overline\partial_b\)-Neumann\\ operators};
\draw (43.379999999999995, -79.2) rectangle (45.98,-80.8);
\draw(51.67999999999999, -75.5) node[anchor=north west,align=left] {\large Generalizations of analytic spaces};
\draw (51.67999999999999, -75.5) rectangle (63.279999999999994,-80.9);
\draw(52.67999999999999, -76.5) node[anchor=north west,align=left] {Holomorphic\\ maps with \\ infinite-dimensional\\ arguments\\ or values};
\draw (52.67999999999999, -76.5) rectangle (56.279999999999994,-79.1);
\draw(56.379999999999995, -76.5) node[anchor=north west,align=left] {Differentiable\\ functions \\ on analytic \\ spaces, \\ differentiable spaces};
\draw (56.379999999999995, -76.5) rectangle (59.98,-79.1);
\draw(60.07999999999999, -76.5) node[anchor=north west,align=left] {Banach \\ analytic\\ manifolds\\ and spaces};
\draw (60.07999999999999, -76.5) rectangle (63.17999999999999,-78.6);
\draw(52.67999999999999, -79.2) node[anchor=north west,align=left] {Formal and\\ graded \\ complex spaces};
\draw (52.67999999999999, -79.2) rectangle (55.779999999999994,-80.8);
\draw(2, -94.80000000000001) node[anchor=north west,align=left] {\large Complex singularities};
\draw (2, -94.80000000000001) rectangle (16.799999999999997,-110.30000000000001);
\draw(3, -95.80000000000001) node[anchor=north west,align=left] {Monodromy; \\ relations with \\ differential \\ equations and \(D\)-modules\\ (complex-analytic aspects)};
\draw (3, -95.80000000000001) rectangle (11.35,-98.9);
\draw(11.45, -95.80000000000001) node[anchor=north west,align=left] {Stratifications;\\ constructible\\ sheaves; \\ intersection \\ cohomology \\ (complex-analytic aspects)};
\draw (11.45, -95.80000000000001) rectangle (16.549999999999997,-98.9);
\draw(3, -99.00000000000001) node[anchor=north west,align=left] {Deformations\\ of complex\\ singularities; \\ vanishing cycles};
\draw (3, -99.00000000000001) rectangle (9.35,-101.60000000000001);
\draw(9.45, -99.00000000000001) node[anchor=north west,align=left] {Equisingularity\\ (topological \\ and analytic)};
\draw (9.45, -99.00000000000001) rectangle (15.049999999999999,-101.10000000000001);
\draw(3, -101.70000000000002) node[anchor=north west,align=left] {Topological aspects\\ of complex \\ singularities: Lefschetz\\ theorems, \\ topological \\ classification, invariants};
\draw (3, -101.70000000000002) rectangle (7.85,-104.80000000000001);
\draw(7.949999999999999, -101.70000000000002) node[anchor=north west,align=left] {Modifications;\\ resolution\\ of singularities\\ (complex-analytic\\ aspects)};
\draw (7.949999999999999, -101.70000000000002) rectangle (12.299999999999999,-104.30000000000001);
\draw(12.399999999999999, -101.70000000000002) node[anchor=north west,align=left] {Global theory\\ of complex\\ singularities;\\ cohomological\\ properties};
\draw (12.399999999999999, -101.70000000000002) rectangle (16.5,-104.30000000000001);
\draw(3, -104.9) node[anchor=north west,align=left] {Mixed Hodge \\ theory of \\ singular varieties\\ (complex-analytic\\ aspects)};
\draw (3, -104.9) rectangle (6.85,-107.5);
\draw(6.949999999999999, -104.9) node[anchor=north west,align=left] {Milnor \\ fibration; \\ relations \\ with knot theory};
\draw (6.949999999999999, -104.9) rectangle (10.549999999999999,-107.0);
\draw(10.65, -104.9) node[anchor=north west,align=left] {Complex \\ surface and \\ hypersurface\\ singularities};
\draw (10.65, -104.9) rectangle (14.0,-107.0);
\draw(14.1, -104.9) node[anchor=north west,align=left] {Invariants\\ of \\ analytic \\ local rings};
\draw (14.1, -104.9) rectangle (16.7,-107.0);
\draw(3, -107.60000000000001) node[anchor=north west,align=left] {Singularities\\ of \\ holomorphic vector\\ fields \\ and foliations};
\draw (3, -107.60000000000001) rectangle (6.35,-110.2);
\draw(6.449999999999999, -107.60000000000001) node[anchor=north west,align=left] {Relations\\ with \\ arrangements of\\ hyperplanes};
\draw (6.449999999999999, -107.60000000000001) rectangle (9.549999999999999,-109.7);
\draw(9.649999999999999, -107.60000000000001) node[anchor=north west,align=left] {Local \\ complex \\ singularities};
\draw (9.649999999999999, -107.60000000000001) rectangle (12.499999999999998,-109.2);
\draw(12.6, -107.60000000000001) node[anchor=north west,align=left] {Other \\ operations on\\ complex \\ singularities};
\draw (12.6, -107.60000000000001) rectangle (15.45,-109.7);
\draw(16.9, -94.80000000000001) node[anchor=north west,align=left] {\large Deformations of analytic structures};
\draw (16.9, -94.80000000000001) rectangle (30.249999999999996,-107.10000000000001);
\draw(17.9, -95.80000000000001) node[anchor=north west,align=left] {Complex-analytic\\ moduli problems};
\draw (17.9, -95.80000000000001) rectangle (23.5,-97.4);
\draw(23.599999999999998, -95.80000000000001) node[anchor=north west,align=left] {Moduli of Riemann\\ surfaces, \\ Teichmüller theory\\ (complex-analytic\\ aspects in\\ several variables)};
\draw (23.599999999999998, -95.80000000000001) rectangle (28.449999999999996,-98.9);
\draw(17.9, -99.00000000000001) node[anchor=north west,align=left] {Moduli and \\ deformations for \\ ordinary differential\\ equations (e.g.,\\ Knizhnik-Zamolodchikov\\ equation)};
\draw (17.9, -99.00000000000001) rectangle (22.75,-102.10000000000001);
\draw(22.849999999999998, -99.00000000000001) node[anchor=north west,align=left] {Period \\ matrices, variation\\ of Hodge\\ structure;\\ degenerations};
\draw (22.849999999999998, -99.00000000000001) rectangle (26.949999999999996,-101.60000000000001);
\draw(27.049999999999997, -99.00000000000001) node[anchor=north west,align=left] {Deformations\\ of special\\ (e.g., CR)\\ structures};
\draw (27.049999999999997, -99.00000000000001) rectangle (30.15,-101.10000000000001);
\draw(17.9, -102.20000000000002) node[anchor=north west,align=left] {Applications\\ of deformations\\ of analytic\\ structures\\ to the sciences};
\draw (17.9, -102.20000000000002) rectangle (21.75,-104.80000000000001);
\draw(21.849999999999998, -102.20000000000002) node[anchor=north west,align=left] {Deformations\\ of \\ submanifolds \\ and subspaces};
\draw (21.849999999999998, -102.20000000000002) rectangle (24.95,-104.30000000000001);
\draw(25.049999999999997, -102.20000000000002) node[anchor=north west,align=left] {Deformations\\ of \\ fiber bundles};
\draw (25.049999999999997, -102.20000000000002) rectangle (27.9,-103.80000000000001);
\draw(17.9, -104.9) node[anchor=north west,align=left] {Deformations\\ of\\ complex\\ structures};
\draw (17.9, -104.9) rectangle (20.5,-107.0);
\draw(30.349999999999998, -94.80000000000001) node[anchor=north west,align=left] {\large Complex manifolds};
\draw (30.349999999999998, -94.80000000000001) rectangle (41.75,-110.00000000000001);
\draw(31.349999999999998, -95.80000000000001) node[anchor=north west,align=left] {Kähler-Einsteinmanifolds};
\draw (31.349999999999998, -95.80000000000001) rectangle (37.699999999999996,-97.4);
\draw(37.8, -95.80000000000001) node[anchor=north west,align=left] {Special domains\\ (Reinhardt,\\ Hartogs, circular,\\ tube, etc.)\\ in \(\mathbb{C}^n\) and \\ complex manifolds};
\draw (37.8, -95.80000000000001) rectangle (41.65,-98.9);
\draw(31.349999999999998, -97.50000000000001) node[anchor=north west,align=left] {Kähler\\ manifolds};
\draw (31.349999999999998, -97.50000000000001) rectangle (33.699999999999996,-98.60000000000001);
\draw(31.349999999999998, -99.00000000000001) node[anchor=north west,align=left] {Uniformization\\ of complex\\ manifolds};
\draw (31.349999999999998, -99.00000000000001) rectangle (36.199999999999996,-101.10000000000001);
\draw(36.3, -99.00000000000001) node[anchor=north west,align=left] {Pseudoholomorphic\\ curves};
\draw (36.3, -99.00000000000001) rectangle (40.65,-100.60000000000001);
\draw(31.349999999999998, -101.20000000000002) node[anchor=north west,align=left] {Calabi-Yau\\ theory \\ (complex-analytic\\ aspects)};
\draw (31.349999999999998, -101.20000000000002) rectangle (34.949999999999996,-103.30000000000001);
\draw(35.05, -101.20000000000002) node[anchor=north west,align=left] {Complex \\ manifolds as \\ subdomains of \\ Euclidean space};
\draw (35.05, -101.20000000000002) rectangle (38.65,-103.30000000000001);
\draw(38.75, -101.20000000000002) node[anchor=north west,align=left] {Oka principle\\ and Oka\\ manifolds};
\draw (38.75, -101.20000000000002) rectangle (41.6,-102.80000000000001);
\draw(31.349999999999998, -103.4) node[anchor=north west,align=left] {Embedding\\ theorems\\ for complex\\ manifolds};
\draw (31.349999999999998, -103.4) rectangle (34.699999999999996,-105.5);
\draw(34.8, -103.4) node[anchor=north west,align=left] {Topological\\ aspects\\ of complex\\ manifolds};
\draw (34.8, -103.4) rectangle (38.15,-105.5);
\draw(38.25, -103.4) node[anchor=north west,align=left] {Negative\\ curvature\\ complex\\ manifolds};
\draw (38.25, -103.4) rectangle (41.35,-105.5);
\draw(31.349999999999998, -105.60000000000001) node[anchor=north west,align=left] {Positive\\ curvature\\ complex\\ manifolds};
\draw (31.349999999999998, -105.60000000000001) rectangle (34.449999999999996,-107.7);
\draw(34.55, -105.60000000000001) node[anchor=north west,align=left] {Notions of\\ stability\\ for complex\\ manifolds};
\draw (34.55, -105.60000000000001) rectangle (37.65,-107.7);
\draw(37.75, -105.60000000000001) node[anchor=north west,align=left] {Hyperbolic\\ and Kobayashi\\ hyperbolic\\ manifolds};
\draw (37.75, -105.60000000000001) rectangle (40.85,-107.7);
\draw(31.349999999999998, -107.80000000000001) node[anchor=north west,align=left] {Classification\\ theorems\\ for complex\\ manifolds};
\draw (31.349999999999998, -107.80000000000001) rectangle (34.449999999999996,-109.9);
\draw(34.55, -107.80000000000001) node[anchor=north west,align=left] {Almost\\ complex\\ manifolds};
\draw (34.55, -107.80000000000001) rectangle (37.15,-109.4);
\draw(37.25, -107.80000000000001) node[anchor=north west,align=left] {Stein \\ manifolds};
\draw (37.25, -107.80000000000001) rectangle (39.35,-108.9);
\draw(41.849999999999994, -94.80000000000001) node[anchor=north west,align=left] {\large Local analytic geometry};
\draw (41.849999999999994, -94.80000000000001) rectangle (52.99999999999999,-103.4);
\draw(42.849999999999994, -95.80000000000001) node[anchor=north west,align=left] {Analytic \\ algebras and\\ generalizations,\\ preparation theorems};
\draw (42.849999999999994, -95.80000000000001) rectangle (49.449999999999996,-98.4);
\draw(49.55, -95.80000000000001) node[anchor=north west,align=left] {Semi-analytic\\ sets, \\ subanalytic \\ sets, and \\ generalizations};
\draw (49.55, -95.80000000000001) rectangle (52.9,-98.4);
\draw(42.849999999999994, -98.50000000000001) node[anchor=north west,align=left] {Triangulation and\\ topological \\ properties of \\ semi-analytic \\ andsubanalytic sets, \\ and related questions};
\draw (42.849999999999994, -98.50000000000001) rectangle (47.949999999999996,-101.60000000000001);
\draw(48.05, -98.50000000000001) node[anchor=north west,align=left] {Germs of \\ analytic sets,\\ local \\ parametrization};
\draw (48.05, -98.50000000000001) rectangle (51.15,-100.60000000000001);
\draw(42.849999999999994, -101.70000000000002) node[anchor=north west,align=left] {Analytic \\ subsets of\\ affine space};
\draw (42.849999999999994, -101.70000000000002) rectangle (45.949999999999996,-103.30000000000001);
\draw(41.849999999999994, -103.5) node[anchor=north west,align=left] {\large Automorphic functions};
\draw (41.849999999999994, -103.5) rectangle (49.49999999999999,-109.4);
\draw(42.849999999999994, -104.5) node[anchor=north west,align=left] {General theory\\ of automorphic\\ functions\\ of several \\ complex variables};
\draw (42.849999999999994, -104.5) rectangle (46.949999999999996,-107.1);
\draw(42.849999999999994, -107.2) node[anchor=north west,align=left] {Automorphic\\ forms in \\ several complex\\ variables};
\draw (42.849999999999994, -107.2) rectangle (46.199999999999996,-109.3);
\draw(46.3, -107.2) node[anchor=north west,align=left] {Automorphic\\ functions\\ in symmetric\\ domains};
\draw (46.3, -107.2) rectangle (49.4,-109.3);
\draw(53.099999999999994, -94.80000000000001) node[anchor=north west,align=left] {\large Compact analytic spaces};
\draw (53.099999999999994, -94.80000000000001) rectangle (64.25,-104.60000000000001);
\draw(54.099999999999994, -95.80000000000001) node[anchor=north west,align=left] {Compact \\ Kähler manifolds:\\ generalizations, \\ classification};
\draw (54.099999999999994, -95.80000000000001) rectangle (60.199999999999996,-98.4);
\draw(60.3, -95.80000000000001) node[anchor=north west,align=left] {Transcendental\\ methods of \\ algebraic geometry\\ (complex-analytic\\ aspects)};
\draw (60.3, -95.80000000000001) rectangle (64.14999999999999,-98.4);
\draw(54.099999999999994, -98.50000000000001) node[anchor=north west,align=left] {Compactification\\ of \\ analytic spaces};
\draw (54.099999999999994, -98.50000000000001) rectangle (58.949999999999996,-100.60000000000001);
\draw(59.05, -98.50000000000001) node[anchor=north west,align=left] {Applications\\ of compact\\ analytic\\ spaces\\ to the sciences};
\draw (59.05, -98.50000000000001) rectangle (62.9,-101.10000000000001);
\draw(54.099999999999994, -101.20000000000002) node[anchor=north west,align=left] {Compact\\ complex\\ \(3\)-folds};
\draw (54.099999999999994, -101.20000000000002) rectangle (57.449999999999996,-102.80000000000001);
\draw(57.55, -101.20000000000002) node[anchor=north west,align=left] {Compact\\ complex\\ \(n\)-folds};
\draw (57.55, -101.20000000000002) rectangle (60.9,-102.80000000000001);
\draw(60.99999999999999, -101.20000000000002) node[anchor=north west,align=left] {Algebraic\\ dependence\\ theorems};
\draw (60.99999999999999, -101.20000000000002) rectangle (63.849999999999994,-102.80000000000001);
\draw(54.099999999999994, -102.9) node[anchor=north west,align=left] {Compact\\ complex\\ surfaces};
\draw (54.099999999999994, -102.9) rectangle (56.699999999999996,-104.5);
\draw(2, -110.4) node[anchor=north west,align=left] {\large Holomorphic convexity};
\draw (2, -110.4) rectangle (12.149999999999999,-120.5);
\draw(3, -111.4) node[anchor=north west,align=left] {Holomorphic, \\ polynomial and rational\\ approximation, and\\ interpolation in\\ several complex \\ variables; Runge pairs};
\draw (3, -111.4) rectangle (8.35,-114.5);
\draw(8.45, -111.4) node[anchor=north west,align=left] {Stein \\ spaces, Stein\\ manifolds};
\draw (8.45, -111.4) rectangle (11.299999999999999,-113.0);
\draw(8.45, -113.10000000000001) node[anchor=north west,align=left] {The Levi\\ problem};
\draw (8.45, -113.10000000000001) rectangle (10.799999999999999,-114.2);
\draw(3, -114.60000000000001) node[anchor=north west,align=left] {Global boundary\\ behavior of \\ holomorphic functions\\ of several\\ complex variables};
\draw (3, -114.60000000000001) rectangle (7.6,-117.2);
\draw(7.699999999999999, -114.60000000000001) node[anchor=north west,align=left] {Polynomial \\ convexity, rational\\ convexity, \\ meromorphic convexity\\ in several\\ complex variables};
\draw (7.699999999999999, -114.60000000000001) rectangle (12.049999999999999,-117.7);
\draw(3, -117.80000000000001) node[anchor=north west,align=left] {Holomorphically\\ convex\\ complex\\ spaces, \\ reduction theory};
\draw (3, -117.80000000000001) rectangle (6.85,-120.4);
\draw(12.249999999999998, -110.4) node[anchor=north west,align=left] {\large Pluripotential theory};
\draw (12.249999999999998, -110.4) rectangle (21.65,-119.7);
\draw(13.249999999999998, -111.4) node[anchor=north west,align=left] {Plurisubharmonic\\ exhaustion\\ functions};
\draw (13.249999999999998, -111.4) rectangle (18.099999999999998,-113.5);
\draw(18.199999999999996, -111.4) node[anchor=north west,align=left] {Plurisubharmonic\\ functions\\ and \\ generalizations};
\draw (18.199999999999996, -111.4) rectangle (21.549999999999997,-113.5);
\draw(13.249999999999998, -113.60000000000001) node[anchor=north west,align=left] {Plurisubharmonic\\ extremal\\ functions,\\ pluricomplex\\ Green functions};
\draw (13.249999999999998, -113.60000000000001) rectangle (17.599999999999998,-116.2);
\draw(17.699999999999996, -113.60000000000001) node[anchor=north west,align=left] {Removable\\ sets in\\ pluripotential\\ theory};
\draw (17.699999999999996, -113.60000000000001) rectangle (20.799999999999997,-115.7);
\draw(13.249999999999998, -116.30000000000001) node[anchor=north west,align=left] {General \\ pluripotential\\ theory};
\draw (13.249999999999998, -116.30000000000001) rectangle (16.099999999999998,-117.9);
\draw(16.2, -116.30000000000001) node[anchor=north west,align=left] {Capacity\\ theory\\ and \\ generalizations};
\draw (16.2, -116.30000000000001) rectangle (19.05,-118.4);
\draw(19.15, -116.30000000000001) node[anchor=north west,align=left] {Lelong\\ numbers};
\draw (19.15, -116.30000000000001) rectangle (21.25,-117.4);
\draw(13.249999999999998, -118.5) node[anchor=north west,align=left] {Currents};
\draw (13.249999999999998, -118.5) rectangle (15.349999999999998,-119.6);
\draw(21.75, -110.4) node[anchor=north west,align=left] {\large Analytic spaces};
\draw (21.75, -110.4) rectangle (30.9,-127.80000000000001);
\draw(22.75, -111.4) node[anchor=north west,align=left] {Sheaves of \\ differential\\ operators and\\ their \\ modules, \(D\)-modules};
\draw (22.75, -111.4) rectangle (27.35,-114.0);
\draw(27.45, -111.4) node[anchor=north west,align=left] {Analytic \\ subsets and\\ submanifolds};
\draw (27.45, -111.4) rectangle (30.8,-113.0);
\draw(22.75, -114.10000000000001) node[anchor=north west,align=left] {Applications\\ of analytic \\ spaces to physics\\ and other\\ areas of science};
\draw (22.75, -114.10000000000001) rectangle (27.1,-116.7);
\draw(27.2, -114.10000000000001) node[anchor=north west,align=left] {Integration\\ on analytic\\ sets and \\ spaces, currents};
\draw (27.2, -114.10000000000001) rectangle (30.55,-116.2);
\draw(22.75, -116.80000000000001) node[anchor=north west,align=left] {Complex\\ supergeometry};
\draw (22.75, -116.80000000000001) rectangle (26.6,-118.4);
\draw(26.7, -116.80000000000001) node[anchor=north west,align=left] {The Levi \\ problem in complex\\ spaces;\\ generalizations};
\draw (26.7, -116.80000000000001) rectangle (30.55,-118.9);
\draw(22.75, -119.0) node[anchor=north west,align=left] {Real-analytic\\ manifolds,\\ real-analytic\\ spaces};
\draw (22.75, -119.0) rectangle (25.85,-121.1);
\draw(25.95, -119.0) node[anchor=north west,align=left] {Real-analytic\\ sets,\\ complex \\ Nash functions};
\draw (25.95, -119.0) rectangle (29.05,-121.1);
\draw(22.75, -121.2) node[anchor=north west,align=left] {Embedding\\ of \\ real-analytic\\ manifolds};
\draw (22.75, -121.2) rectangle (25.85,-123.3);
\draw(25.95, -121.2) node[anchor=north west,align=left] {Duality\\ theorems\\ for \\ analytic spaces};
\draw (25.95, -121.2) rectangle (29.05,-123.3);
\draw(22.75, -123.4) node[anchor=north west,align=left] {Topology\\ of \\ analytic spaces};
\draw (22.75, -123.4) rectangle (25.6,-125.0);
\draw(25.7, -123.4) node[anchor=north west,align=left] {Analytic\\ sheaves \\ and cohomology\\ groups};
\draw (25.7, -123.4) rectangle (28.55,-125.5);
\draw(28.65, -123.4) node[anchor=north west,align=left] {Complex\\ spaces};
\draw (28.65, -123.4) rectangle (30.75,-124.5);
\draw(22.75, -125.60000000000001) node[anchor=north west,align=left] {Local \\ cohomology\\ of \\ analytic spaces};
\draw (22.75, -125.60000000000001) rectangle (25.6,-127.7);
\draw(25.7, -125.60000000000001) node[anchor=north west,align=left] {Embedding\\ of analytic\\ spaces};
\draw (25.7, -125.60000000000001) rectangle (28.3,-127.2);
\draw(28.4, -125.60000000000001) node[anchor=north west,align=left] {Normal\\ analytic\\ spaces};
\draw (28.4, -125.60000000000001) rectangle (30.5,-127.2);
\draw(12.249999999999998, -119.80000000000001) node[anchor=north west,align=left] {\large Pseudoconvex domains};
\draw (12.249999999999998, -119.80000000000001) rectangle (21.15,-126.9);
\draw(13.249999999999998, -120.80000000000001) node[anchor=north west,align=left] {Finite-typedomains};
\draw (13.249999999999998, -120.80000000000001) rectangle (18.099999999999998,-122.4);
\draw(18.199999999999996, -120.80000000000001) node[anchor=north west,align=left] {Strongly\\ pseudoconvex\\ domains};
\draw (18.199999999999996, -120.80000000000001) rectangle (21.049999999999997,-122.4);
\draw(13.249999999999998, -122.50000000000001) node[anchor=north west,align=left] {Geometric and\\ analytic \\ invariants on \\ weakly pseudoconvex\\ boundaries};
\draw (13.249999999999998, -122.50000000000001) rectangle (17.099999999999998,-125.10000000000001);
\draw(17.199999999999996, -122.50000000000001) node[anchor=north west,align=left] {Exhaustion\\ functions};
\draw (17.199999999999996, -122.50000000000001) rectangle (20.799999999999997,-124.10000000000001);
\draw(13.249999999999998, -125.20000000000002) node[anchor=north west,align=left] {Domains\\ of \\ holomorphy};
\draw (13.249999999999998, -125.20000000000002) rectangle (15.599999999999998,-126.80000000000001);
\draw(15.7, -125.20000000000002) node[anchor=north west,align=left] {Peak \\ functions};
\draw (15.7, -125.20000000000002) rectangle (17.8,-126.30000000000001);
\draw(17.9, -125.20000000000002) node[anchor=north west,align=left] {Worm\\ domains};
\draw (17.9, -125.20000000000002) rectangle (19.75,-126.30000000000001);
\draw(31.0, -110.4) node[anchor=north west,align=left] {\large Holomorphic fiber spaces};
\draw (31.0, -110.4) rectangle (39.9,-119.0);
\draw(32.0, -111.4) node[anchor=north west,align=left] {Vanishingtheorems};
\draw (32.0, -111.4) rectangle (36.6,-113.0);
\draw(36.7, -111.4) node[anchor=north west,align=left] {Holomorphic\\ bundles\\ and \\ generalizations};
\draw (36.7, -111.4) rectangle (39.800000000000004,-113.5);
\draw(32.0, -113.60000000000001) node[anchor=north west,align=left] {Sheaves and \\ cohomology of sections\\ of holomorphic\\ vector bundles,\\ general results};
\draw (32.0, -113.60000000000001) rectangle (36.35,-116.2);
\draw(36.45, -113.60000000000001) node[anchor=north west,align=left] {Applications\\ of holomorphic\\ fiber\\ spaces to\\ the sciences};
\draw (36.45, -113.60000000000001) rectangle (39.550000000000004,-116.2);
\draw(32.0, -116.30000000000001) node[anchor=north west,align=left] {Twistor \\ theory, double\\ fibrations\\ (complex-analytic\\ aspects)};
\draw (32.0, -116.30000000000001) rectangle (36.1,-118.9);
\draw(36.2, -116.30000000000001) node[anchor=north west,align=left] {Bundle\\ convexity};
\draw (36.2, -116.30000000000001) rectangle (38.550000000000004,-117.4);
\draw(31.0, -119.10000000000001) node[anchor=north west,align=left] {\large Analytic continuation};
\draw (31.0, -119.10000000000001) rectangle (38.65,-126.2);
\draw(32.0, -120.10000000000001) node[anchor=north west,align=left] {Removable \\ singularities\\ in several \\ complex variables};
\draw (32.0, -120.10000000000001) rectangle (36.1,-122.2);
\draw(36.2, -120.10000000000001) node[anchor=north west,align=left] {Domains\\ of \\ holomorphy};
\draw (36.2, -120.10000000000001) rectangle (38.550000000000004,-121.7);
\draw(32.0, -122.30000000000001) node[anchor=north west,align=left] {Continuation\\ of analytic\\ objects in\\ several complex\\ variables};
\draw (32.0, -122.30000000000001) rectangle (35.35,-124.9);
\draw(35.45, -122.30000000000001) node[anchor=north west,align=left] {Envelopes\\ of \\ holomorphy};
\draw (35.45, -122.30000000000001) rectangle (37.800000000000004,-123.9);
\draw(32.0, -125.00000000000001) node[anchor=north west,align=left] {Riemann\\ domains};
\draw (32.0, -125.00000000000001) rectangle (34.1,-126.10000000000001);
\draw(40.0, -110.4) node[anchor=north west,align=left] {\large CR manifolds};
\draw (40.0, -110.4) rectangle (47.9,-120.2);
\draw(41.0, -111.4) node[anchor=north west,align=left] {CR structures,\\ CR \\ operators, and\\ generalizations};
\draw (41.0, -111.4) rectangle (44.85,-113.5);
\draw(44.95, -111.4) node[anchor=north west,align=left] {Finite-type\\ conditions\\ on \\ CR manifolds};
\draw (44.95, -111.4) rectangle (47.800000000000004,-113.5);
\draw(41.0, -113.60000000000001) node[anchor=north west,align=left] {Extension of\\ functions and\\ other analytic\\ objects \\ from CR manifolds};
\draw (41.0, -113.60000000000001) rectangle (44.85,-116.2);
\draw(44.95, -113.60000000000001) node[anchor=north west,align=left] {Real \\ submanifolds\\ in complex\\ manifolds};
\draw (44.95, -113.60000000000001) rectangle (47.800000000000004,-115.7);
\draw(41.0, -116.30000000000001) node[anchor=north west,align=left] {CR manifolds\\ as \\ boundaries\\ of domains};
\draw (41.0, -116.30000000000001) rectangle (43.6,-118.4);
\draw(43.7, -116.30000000000001) node[anchor=north west,align=left] {Analysis\\ on CR\\ manifolds};
\draw (43.7, -116.30000000000001) rectangle (46.300000000000004,-117.9);
\draw(41.0, -118.5) node[anchor=north west,align=left] {Embeddings\\ of CR\\ manifolds};
\draw (41.0, -118.5) rectangle (43.6,-120.1);
\draw(43.7, -118.5) node[anchor=north west,align=left] {CR \\ functions};
\draw (43.7, -118.5) rectangle (45.550000000000004,-119.6);
\draw(64.44999999999999, -68.0) node[anchor=north west,align=left] {\LARGE Associative rings and algebras};
\draw (64.44999999999999, -68.0) rectangle (121.17999999999998,-112.3);
\draw(65.44999999999999, -69.0) node[anchor=north west,align=left] {\large Chain conditions, growth conditions, and other forms of finiteness for associative rings and algebras};
\draw (65.44999999999999, -69.0) rectangle (97.35999999999999,-73.7);
\draw(66.44999999999999, -70.0) node[anchor=north west,align=left] {Chain conditions\\ on other classes\\ of submodules,\\ ideals, subrings,\\ etc.; coherence\\ (associative\\ rings and algebras)};
\draw (66.44999999999999, -70.0) rectangle (71.29999999999998,-73.6);
\draw(71.39999999999999, -70.0) node[anchor=north west,align=left] {Chain conditions\\ on \\ annihilators and \\ summands: \\ Goldie-type conditions};
\draw (71.39999999999999, -70.0) rectangle (75.74999999999999,-72.6);
\draw(75.85, -70.0) node[anchor=north west,align=left] {Artinian \\ rings and \\ modules \\ (associative rings\\ and algebras)};
\draw (75.85, -70.0) rectangle (79.69999999999999,-72.6);
\draw(79.79999999999998, -70.0) node[anchor=north west,align=left] {Noetherian \\ rings and \\ modules (associative\\ rings\\ and algebras)};
\draw (79.79999999999998, -70.0) rectangle (83.14999999999998,-72.6);
\draw(83.24999999999999, -70.0) node[anchor=north west,align=left] {Finite rings\\ and \\ finite-dimensional\\ associative\\ algebras};
\draw (83.24999999999999, -70.0) rectangle (86.34999999999998,-72.6);
\draw(86.44999999999999, -70.0) node[anchor=north west,align=left] {Localization\\ and \\ associative \\ Noetherian rings};
\draw (86.44999999999999, -70.0) rectangle (89.54999999999998,-72.1);
\draw(89.64999999999999, -70.0) node[anchor=north west,align=left] {Growth \\ rate, \\ Gelfand-Kirillov\\ dimension};
\draw (89.64999999999999, -70.0) rectangle (92.74999999999999,-72.1);
\draw(97.45999999999998, -69.0) node[anchor=north west,align=left] {\large Associative rings and algebras with additional structure};
\draw (97.45999999999998, -69.0) rectangle (117.75999999999998,-75.9);
\draw(98.45999999999998, -70.0) node[anchor=north west,align=left] {Derivations,\\ actions of\\ Lie algebras};
\draw (98.45999999999998, -70.0) rectangle (103.55999999999997,-72.1);
\draw(103.65999999999998, -70.0) node[anchor=north west,align=left] {Rings with \\ involution; Lie,\\ Jordan and\\ other \\ nonassociative structures};
\draw (103.65999999999998, -70.0) rectangle (108.25999999999998,-72.6);
\draw(108.35999999999999, -70.0) node[anchor=north west,align=left] {Actions of \\ groups and \\ semigroups; invariant\\ theory \\ (associative \\ rings and algebras)};
\draw (108.35999999999999, -70.0) rectangle (112.95999999999998,-73.1);
\draw(113.05999999999997, -70.0) node[anchor=north west,align=left] {Valuations, \\ completions, formal \\ power series and \\ related constructions\\ (associative \\ rings and algebras)};
\draw (113.05999999999997, -70.0) rectangle (117.65999999999997,-73.1);
\draw(98.45999999999998, -73.2) node[anchor=north west,align=left] {Filtered \\ associative \\ rings; filtrational\\ and \\ graded techniques};
\draw (98.45999999999998, -73.2) rectangle (102.05999999999997,-75.8);
\draw(102.15999999999998, -73.2) node[anchor=north west,align=left] {Graded rings\\ and modules\\ (associative\\ rings\\ and algebras)};
\draw (102.15999999999998, -73.2) rectangle (105.50999999999998,-75.8);
\draw(105.60999999999999, -73.2) node[anchor=north west,align=left] {Automorphisms\\ and \\ endomorphisms};
\draw (105.60999999999999, -73.2) rectangle (108.45999999999998,-74.8);
\draw(108.55999999999997, -73.2) node[anchor=north west,align=left] {“Super” \\ (or “skew”)\\ structure};
\draw (108.55999999999997, -73.2) rectangle (111.40999999999997,-74.8);
\draw(111.50999999999998, -73.2) node[anchor=north west,align=left] {Topological\\ and ordered\\ rings\\ and modules};
\draw (111.50999999999998, -73.2) rectangle (114.35999999999997,-75.3);
\draw(65.44999999999999, -73.8) node[anchor=north west,align=left] {\large History of associative\\ rings and algebras};
\draw (65.44999999999999, -73.8) rectangle (71.93999999999998,-74.89999999999999);
\draw(65.44999999999999, -76.0) node[anchor=north west,align=left] {\large Associative rings and algebras arising under various constructions};
\draw (65.44999999999999, -76.0) rectangle (90.1,-85.1);
\draw(66.44999999999999, -77.0) node[anchor=north west,align=left] {Associative rings\\ determined by \\ universal properties \\ (free algebras, \\ coproducts, adjunction\\ of inverses, etc.)};
\draw (66.44999999999999, -77.0) rectangle (71.54999999999998,-80.1);
\draw(71.64999999999999, -77.0) node[anchor=north west,align=left] {Torsion theories;\\ radicals on\\ module categories\\ (associative\\ algebraic aspects)};
\draw (71.64999999999999, -77.0) rectangle (76.49999999999999,-79.6);
\draw(76.6, -77.0) node[anchor=north west,align=left] {Finite generation,\\ finite \\ presentability,\\ normal forms \\ (diamond lemma,\\ term-rewriting)};
\draw (76.6, -77.0) rectangle (80.69999999999999,-80.1);
\draw(80.79999999999998, -77.0) node[anchor=north west,align=left] {Rings arising\\ from \\ noncommutative \\ algebraic geometry};
\draw (80.79999999999998, -77.0) rectangle (84.89999999999998,-79.1);
\draw(84.99999999999999, -77.0) node[anchor=north west,align=left] {Associative \\ rings of \\ functions, subdirect\\ products,\\ sheaves of rings};
\draw (84.99999999999999, -77.0) rectangle (89.09999999999998,-79.6);
\draw(66.44999999999999, -80.2) node[anchor=north west,align=left] {Rings of \\ differential \\ operators \\ (associative \\ algebraic aspects)};
\draw (66.44999999999999, -80.2) rectangle (70.29999999999998,-82.8);
\draw(70.39999999999999, -80.2) node[anchor=north west,align=left] {Ordinary and\\ skew polynomial\\ rings and\\ semigroup rings};
\draw (70.39999999999999, -80.2) rectangle (74.24999999999999,-82.3);
\draw(74.35, -80.2) node[anchor=north west,align=left] {Associative\\ rings of \\ fractions and\\ localizations};
\draw (74.35, -80.2) rectangle (78.19999999999999,-82.3);
\draw(78.29999999999998, -80.2) node[anchor=north west,align=left] {Twisted and\\ skew group\\ rings, \\ crossed products};
\draw (78.29999999999998, -80.2) rectangle (81.64999999999998,-82.3);
\draw(81.74999999999999, -80.2) node[anchor=north west,align=left] {Deformations\\ of \\ associative rings};
\draw (81.74999999999999, -80.2) rectangle (85.09999999999998,-81.8);
\draw(85.19999999999999, -80.2) node[anchor=north west,align=left] {Centralizing\\ and \\ normalizing\\ extensions};
\draw (85.19999999999999, -80.2) rectangle (88.29999999999998,-82.3);
\draw(88.39999999999999, -80.2) node[anchor=north west,align=left] {Group\\ rings};
\draw (88.39999999999999, -80.2) rectangle (89.99999999999999,-81.3);
\draw(66.44999999999999, -82.9) node[anchor=north west,align=left] {Universal \\ enveloping \\ algebras of\\ Lie algebras};
\draw (66.44999999999999, -82.9) rectangle (69.54999999999998,-85.0);
\draw(69.64999999999999, -82.9) node[anchor=north west,align=left] {Endomorphism\\ rings;\\ matrix rings};
\draw (69.64999999999999, -82.9) rectangle (72.74999999999999,-84.5);
\draw(72.85, -82.9) node[anchor=north west,align=left] {Quadratic\\ and Koszul\\ algebras};
\draw (72.85, -82.9) rectangle (75.69999999999999,-84.5);
\draw(75.79999999999998, -82.9) node[anchor=north west,align=left] {Smash \\ products of\\ general \\ Hopf actions};
\draw (75.79999999999998, -82.9) rectangle (78.64999999999998,-85.0);
\draw(78.74999999999999, -82.9) node[anchor=north west,align=left] {Extensions\\ of associative\\ rings\\ by ideals};
\draw (78.74999999999999, -82.9) rectangle (81.59999999999998,-85.0);
\draw(81.69999999999999, -82.9) node[anchor=north west,align=left] {Leavitt\\ path\\ algebras};
\draw (81.69999999999999, -82.9) rectangle (84.04999999999998,-84.5);
\draw(90.19999999999999, -76.0) node[anchor=north west,align=left] {\large Modules, bimodules and ideals in associative algebras};
\draw (90.19999999999999, -76.0) rectangle (110.0,-86.1);
\draw(91.19999999999999, -77.0) node[anchor=north west,align=left] {Structure and \\ classification for \\ modules, bimodules and\\ ideals (except \\ as in 16Gxx), direct\\ sum decomposition\\ and cancellation\\ in associative algebras)};
\draw (91.19999999999999, -77.0) rectangle (97.54999999999998,-81.1);
\draw(97.64999999999999, -77.0) node[anchor=north west,align=left] {Free, projective,\\ and flat\\ modules and\\ ideals in \\ associative algebras};
\draw (97.64999999999999, -77.0) rectangle (101.99999999999999,-79.6);
\draw(102.1, -77.0) node[anchor=north west,align=left] {Simple and \\ semisimple modules, \\ primitive rings \\ and ideals in \\ associative algebras};
\draw (102.1, -77.0) rectangle (106.44999999999999,-79.6);
\draw(106.54999999999998, -77.0) node[anchor=north west,align=left] {Bimodules\\ in associative\\ algebras};
\draw (106.54999999999998, -77.0) rectangle (109.89999999999998,-78.6);
\draw(91.19999999999999, -81.2) node[anchor=north west,align=left] {Infinite-dimensional\\ simple\\ rings (except\\ as in 16Kxx)};
\draw (91.19999999999999, -81.2) rectangle (95.04999999999998,-83.3);
\draw(95.14999999999999, -81.2) node[anchor=north west,align=left] {Injective \\ modules, \\ self-injective \\ associative rings};
\draw (95.14999999999999, -81.2) rectangle (98.99999999999999,-83.3);
\draw(99.1, -81.2) node[anchor=north west,align=left] {Other classes\\ of modules\\ and ideals\\ in \\ associative algebras};
\draw (99.1, -81.2) rectangle (102.94999999999999,-83.8);
\draw(103.04999999999998, -81.2) node[anchor=north west,align=left] {General \\ module theory\\ in associative\\ algebras};
\draw (103.04999999999998, -81.2) rectangle (106.14999999999998,-83.3);
\draw(106.24999999999999, -81.2) node[anchor=north west,align=left] {Ideals in\\ associative\\ algebras};
\draw (106.24999999999999, -81.2) rectangle (109.34999999999998,-82.8);
\draw(91.19999999999999, -83.9) node[anchor=north west,align=left] {Module \\ categories in\\ associative\\ algebras};
\draw (91.19999999999999, -83.9) rectangle (94.04999999999998,-86.0);
\draw(110.1, -76.0) node[anchor=north west,align=left] {\large Conditions on elements};
\draw (110.1, -76.0) rectangle (120.5,-87.3);
\draw(111.1, -77.0) node[anchor=north west,align=left] {Divisibility,\\ noncommutative UFDs};
\draw (111.1, -77.0) rectangle (116.94999999999999,-78.6);
\draw(117.05, -77.0) node[anchor=north west,align=left] {Integral \\ domains (associative\\ rings\\ and algebras)};
\draw (117.05, -77.0) rectangle (120.39999999999999,-79.1);
\draw(111.1, -79.2) node[anchor=north west,align=left] {Generalizations\\ of \\ commutativity \\ (associative \\ rings and algebras)};
\draw (111.1, -79.2) rectangle (115.44999999999999,-81.8);
\draw(115.55, -79.2) node[anchor=north west,align=left] {Idempotent \\ elements \\ (associative rings\\ and algebras)};
\draw (115.55, -79.2) rectangle (119.64999999999999,-81.3);
\draw(111.1, -81.9) node[anchor=north west,align=left] {Center, normalizer\\ (invariant\\ elements) \\ (associative rings\\ and algebras)};
\draw (111.1, -81.9) rectangle (115.19999999999999,-84.5);
\draw(115.3, -81.9) node[anchor=north west,align=left] {Ore rings,\\ multiplicative\\ sets, \\ Ore localization};
\draw (115.3, -81.9) rectangle (119.14999999999999,-84.0);
\draw(111.1, -84.6) node[anchor=north west,align=left] {Units, \\ groups of units\\ (associative\\ rings\\ and algebras)};
\draw (111.1, -84.6) rectangle (114.94999999999999,-87.19999999999999);
\draw(115.05, -84.6) node[anchor=north west,align=left] {Generalized \\ inverses \\ (associative rings\\ and algebras)};
\draw (115.05, -84.6) rectangle (118.64999999999999,-86.69999999999999);
\draw(65.44999999999999, -87.4) node[anchor=north west,align=left] {\large Representation theory of associative rings and algebras};
\draw (65.44999999999999, -87.4) rectangle (84.24999999999999,-93.30000000000001);
\draw(66.44999999999999, -88.4) node[anchor=north west,align=left] {Auslander-Reiten\\ sequences \\ (almost split \\ sequences) and \\ Auslander-Reiten quivers};
\draw (66.44999999999999, -88.4) rectangle (71.54999999999998,-91.0);
\draw(71.64999999999999, -88.4) node[anchor=north west,align=left] {Representation\\ type (finite,\\ tame, wild,\\ etc.) of \\ associative algebras};
\draw (71.64999999999999, -88.4) rectangle (76.24999999999999,-91.0);
\draw(76.35, -88.4) node[anchor=north west,align=left] {Representations\\ of orders,\\ lattices, \\ algebras over \\ commutative rings};
\draw (76.35, -88.4) rectangle (80.44999999999999,-91.0);
\draw(80.54999999999998, -88.4) node[anchor=north west,align=left] {Representations\\ of quivers\\ and partially\\ ordered sets};
\draw (80.54999999999998, -88.4) rectangle (84.14999999999998,-90.5);
\draw(66.44999999999999, -91.10000000000001) node[anchor=north west,align=left] {Cohen-Macaulay\\ modules\\ in associative\\ algebras};
\draw (66.44999999999999, -91.10000000000001) rectangle (69.79999999999998,-93.2);
\draw(69.89999999999999, -91.10000000000001) node[anchor=north west,align=left] {Representations\\ of \\ associative \\ Artinian rings};
\draw (69.89999999999999, -91.10000000000001) rectangle (72.99999999999999,-93.2);
\draw(84.35, -87.4) node[anchor=north west,align=left] {\large Radicals and radical properties of associative rings};
\draw (84.35, -87.4) rectangle (101.07,-91.10000000000001);
\draw(85.35, -88.4) node[anchor=north west,align=left] {Jacobson\\ radical,\\ quasimultiplication};
\draw (85.35, -88.4) rectangle (90.19999999999999,-90.5);
\draw(90.3, -88.4) node[anchor=north west,align=left] {Nil and \\ nilpotent \\ radicals, sets,\\ ideals, \\ associative rings};
\draw (90.3, -88.4) rectangle (94.14999999999999,-91.0);
\draw(94.25, -88.4) node[anchor=north west,align=left] {General \\ radicals \\ and associative\\ rings};
\draw (94.25, -88.4) rectangle (97.1,-90.5);
\draw(97.19999999999999, -88.4) node[anchor=north west,align=left] {Prime and\\ semiprime\\ associative\\ rings};
\draw (97.19999999999999, -88.4) rectangle (99.79999999999998,-90.5);
\draw(101.16999999999999, -87.4) node[anchor=north west,align=left] {\large Homological methods in associative algebras};
\draw (101.16999999999999, -87.4) rectangle (117.26999999999998,-97.5);
\draw(102.16999999999999, -88.4) node[anchor=north west,align=left] {Grothendieck\\ groups, \(K\)-theory,\\ etc.};
\draw (102.16999999999999, -88.4) rectangle (108.01999999999998,-90.5);
\draw(108.11999999999999, -88.4) node[anchor=north west,align=left] {Homological conditions\\ on associative\\ rings (generalizations\\ of regular,\\ Gorenstein, \\ Cohen-Macaulay rings, etc.)};
\draw (108.11999999999999, -88.4) rectangle (113.71999999999998,-91.5);
\draw(113.82, -88.4) node[anchor=north west,align=left] {Derived \\ categories \\ and associative\\ algebras};
\draw (113.82, -88.4) rectangle (117.16999999999999,-90.5);
\draw(102.16999999999999, -91.60000000000001) node[anchor=north west,align=left] {(Co)homology \\ of rings and \\ associative \\ algebras (e.g., \\ Hochschild, cyclic,\\ dihedral, etc.)};
\draw (102.16999999999999, -91.60000000000001) rectangle (107.01999999999998,-94.7);
\draw(107.11999999999999, -91.60000000000001) node[anchor=north west,align=left] {Differential \\ graded algebras \\ and applications\\ (associative \\ algebraic aspects)};
\draw (107.11999999999999, -91.60000000000001) rectangle (111.46999999999998,-94.2);
\draw(111.57, -91.60000000000001) node[anchor=north west,align=left] {von Neumann \\ regular rings and\\ generalizations\\ (associative \\ algebraic aspects)};
\draw (111.57, -91.60000000000001) rectangle (115.91999999999999,-94.2);
\draw(102.16999999999999, -94.80000000000001) node[anchor=north west,align=left] {Homological \\ functors on modules\\ (Tor, Ext,\\ etc.) in \\ associative algebras};
\draw (102.16999999999999, -94.80000000000001) rectangle (106.26999999999998,-97.4);
\draw(106.36999999999999, -94.80000000000001) node[anchor=north west,align=left] {Semihereditary\\ and hereditary\\ rings, free ideal\\ rings, Sylvester\\ rings, etc.};
\draw (106.36999999999999, -94.80000000000001) rectangle (110.46999999999998,-97.4);
\draw(110.57, -94.80000000000001) node[anchor=north west,align=left] {Syzygies, \\ resolutions,\\ complexes\\ in associative\\ algebras};
\draw (110.57, -94.80000000000001) rectangle (113.66999999999999,-97.4);
\draw(113.76999999999998, -94.80000000000001) node[anchor=north west,align=left] {Homological\\ dimension\\ in associative\\ algebras};
\draw (113.76999999999998, -94.80000000000001) rectangle (116.86999999999998,-96.9);
\draw(84.35, -91.2) node[anchor=north west,align=left] {\large Hopf algebras, quantum groups and related topics};
\draw (84.35, -91.2) rectangle (100.19999999999999,-96.60000000000001);
\draw(85.35, -92.2) node[anchor=north west,align=left] {Ring-theoretic\\ aspects of \\ quantum groups};
\draw (85.35, -92.2) rectangle (90.94999999999999,-94.3);
\draw(91.05, -92.2) node[anchor=north west,align=left] {Yang-Baxterequations};
\draw (91.05, -92.2) rectangle (96.39999999999999,-93.8);
\draw(96.5, -92.2) node[anchor=north west,align=left] {Connections\\ of Hopf \\ algebras with\\ combinatorics};
\draw (96.5, -92.2) rectangle (100.1,-94.3);
\draw(85.35, -94.4) node[anchor=north west,align=left] {Coalgebras\\ and \\ comodules; corings};
\draw (85.35, -94.4) rectangle (88.69999999999999,-96.0);
\draw(88.8, -94.4) node[anchor=north west,align=left] {Hopf \\ algebras and\\ their\\ applications};
\draw (88.8, -94.4) rectangle (91.89999999999999,-96.5);
\draw(92.0, -94.4) node[anchor=north west,align=left] {Bialgebras};
\draw (92.0, -94.4) rectangle (94.6,-95.5);
\draw(65.44999999999999, -97.6) node[anchor=north west,align=left] {\large Computational aspects of associative rings};
\draw (65.44999999999999, -97.6) rectangle (79.07,-101.3);
\draw(66.44999999999999, -98.6) node[anchor=north west,align=left] {Computational\\ aspects\\ of associative\\ rings \\ (general theory)};
\draw (66.44999999999999, -98.6) rectangle (70.29999999999998,-101.19999999999999);
\draw(70.39999999999999, -98.6) node[anchor=north west,align=left] {Gröbner-Shirshov\\ bases};
\draw (70.39999999999999, -98.6) rectangle (74.24999999999999,-100.19999999999999);
\draw(79.16999999999999, -97.6) node[anchor=north west,align=left] {\large Division rings and semisimple Artin rings};
\draw (79.16999999999999, -97.6) rectangle (92.57,-102.5);
\draw(80.16999999999999, -98.6) node[anchor=north west,align=left] {Infinite-dimensional\\ and general\\ division rings};
\draw (80.16999999999999, -98.6) rectangle (86.51999999999998,-100.69999999999999);
\draw(86.61999999999999, -98.6) node[anchor=north west,align=left] {Finite-dimensional\\ division rings};
\draw (86.61999999999999, -98.6) rectangle (92.46999999999998,-100.19999999999999);
\draw(80.16999999999999, -100.8) node[anchor=north west,align=left] {Brauer \\ groups (algebraic\\ aspects)};
\draw (80.16999999999999, -100.8) rectangle (83.51999999999998,-102.39999999999999);
\draw(92.66999999999999, -97.6) node[anchor=north west,align=left] {\large Associative algebras and orders};
\draw (92.66999999999999, -97.6) rectangle (104.51999999999998,-103.0);
\draw(93.66999999999999, -98.6) node[anchor=north west,align=left] {Separable \\ algebras (e.g.,\\ quaternion \\ algebras, Azumaya\\ algebras, etc.)};
\draw (93.66999999999999, -98.6) rectangle (98.26999999999998,-101.19999999999999);
\draw(98.36999999999999, -98.6) node[anchor=north west,align=left] {Commutative\\ orders};
\draw (98.36999999999999, -98.6) rectangle (101.71999999999998,-100.19999999999999);
\draw(101.82, -98.6) node[anchor=north west,align=left] {Orders in\\ separable\\ algebras};
\draw (101.82, -98.6) rectangle (104.41999999999999,-100.19999999999999);
\draw(93.66999999999999, -101.3) node[anchor=north west,align=left] {Lattices\\ over\\ orders};
\draw (93.66999999999999, -101.3) rectangle (95.76999999999998,-102.89999999999999);
\draw(104.61999999999998, -97.6) node[anchor=north west,align=left] {\large Local rings and generalizations};
\draw (104.61999999999998, -97.6) rectangle (114.82999999999997,-101.3);
\draw(105.61999999999998, -98.6) node[anchor=north west,align=left] {Noncommutative\\ local\\ and semilocal\\ rings,\\ perfect rings};
\draw (105.61999999999998, -98.6) rectangle (109.46999999999997,-101.19999999999999);
\draw(109.56999999999998, -98.6) node[anchor=north west,align=left] {Quasi-Frobenius\\ rings};
\draw (109.56999999999998, -98.6) rectangle (113.41999999999997,-100.19999999999999);
\draw(114.92999999999998, -97.6) node[anchor=north west,align=left] {\large Generalizations};
\draw (114.92999999999998, -97.6) rectangle (121.07999999999998,-101.5);
\draw(115.92999999999998, -98.6) node[anchor=north west,align=left] {Hyperrings};
\draw (115.92999999999998, -98.6) rectangle (118.52999999999997,-99.69999999999999);
\draw(118.62999999999998, -98.6) node[anchor=north west,align=left] {Semirings};
\draw (118.62999999999998, -98.6) rectangle (120.97999999999998,-99.69999999999999);
\draw(115.92999999999998, -99.8) node[anchor=north west,align=left] {Near-rings};
\draw (115.92999999999998, -99.8) rectangle (118.52999999999997,-100.89999999999999);
\draw(118.62999999999998, -99.8) node[anchor=north west,align=left] {\(\Gamma\) and\\ fuzzy \\ structures};
\draw (118.62999999999998, -99.8) rectangle (120.97999999999998,-101.39999999999999);
\draw(65.44999999999999, -103.1) node[anchor=north west,align=left] {\large Rings with polynomial identity};
\draw (65.44999999999999, -103.1) rectangle (75.35,-112.19999999999999);
\draw(66.44999999999999, -104.1) node[anchor=north west,align=left] {Semiprime p.i.\\ rings, rings\\ embeddable in\\ matrices over\\ commutative rings};
\draw (66.44999999999999, -104.1) rectangle (70.79999999999998,-106.69999999999999);
\draw(70.89999999999999, -104.1) node[anchor=north west,align=left] {Trace rings\\ and invariant\\ theory \\ (associative \\ rings and algebras)};
\draw (70.89999999999999, -104.1) rectangle (75.24999999999999,-106.69999999999999);
\draw(66.44999999999999, -106.8) node[anchor=north west,align=left] {Other kinds of\\ identities \\ (generalized \\ polynomial, rational,\\ involution)};
\draw (66.44999999999999, -106.8) rectangle (70.54999999999998,-109.39999999999999);
\draw(70.64999999999999, -106.8) node[anchor=north west,align=left] {Functional \\ identities \\ (associative rings\\ and algebras)};
\draw (70.64999999999999, -106.8) rectangle (74.49999999999999,-108.89999999999999);
\draw(66.44999999999999, -109.5) node[anchor=north west,align=left] {\(T\)-ideals, \\ identities, \\ varieties of \\ associative rings\\ and algebras};
\draw (66.44999999999999, -109.5) rectangle (70.04999999999998,-112.1);
\draw(70.14999999999999, -109.5) node[anchor=north west,align=left] {Identities \\ other than \\ those of matrices\\ over \\ commutative rings};
\draw (70.14999999999999, -109.5) rectangle (73.74999999999999,-112.1);
\draw(75.44999999999999, -103.1) node[anchor=north west,align=left] {\large General and miscellaneous};
\draw (75.44999999999999, -103.1) rectangle (84.35,-107.3);
\draw(76.44999999999999, -104.1) node[anchor=north west,align=left] {Category-theoretic\\ methods\\ and results\\ in associative\\ algebras \\ (except as in 16D90)};
\draw (76.44999999999999, -104.1) rectangle (81.04999999999998,-107.19999999999999);
\draw(81.14999999999999, -104.1) node[anchor=north west,align=left] {Applications\\ of logic\\ in associative\\ algebras};
\draw (81.14999999999999, -104.1) rectangle (84.24999999999999,-106.19999999999999);
\draw(121.27999999999999, -68.0) node[anchor=north west,align=left] {\LARGE Combinatorics};
\draw (121.27999999999999, -68.0) rectangle (142.23,-118.5);
\draw(122.27999999999999, -69.0) node[anchor=north west,align=left] {\large Designs and configurations};
\draw (122.27999999999999, -69.0) rectangle (133.38,-80.8);
\draw(123.27999999999999, -70.0) node[anchor=north west,align=left] {Combinatorial\\ aspects of \\ difference sets\\ (number-theoretic,\\ group-theoretic, etc.)};
\draw (123.27999999999999, -70.0) rectangle (130.88,-73.1);
\draw(130.98, -70.0) node[anchor=north west,align=left] {Triple\\ systems};
\draw (130.98, -70.0) rectangle (133.07999999999998,-71.1);
\draw(123.27999999999999, -73.2) node[anchor=north west,align=left] {Orthogonal\\ arrays, Latin\\ squares,\\ Room squares};
\draw (123.27999999999999, -73.2) rectangle (126.62999999999998,-75.3);
\draw(126.72999999999999, -73.2) node[anchor=north west,align=left] {Combinatorial\\ aspects of\\ matrices \\ (incidence, \\ Hadamard, etc.)};
\draw (126.72999999999999, -73.2) rectangle (130.07999999999998,-75.8);
\draw(130.17999999999998, -73.2) node[anchor=north west,align=left] {Combinatorial\\ aspects\\ of finite\\ geometries};
\draw (130.17999999999998, -73.2) rectangle (133.27999999999997,-75.3);
\draw(123.27999999999999, -75.9) node[anchor=north west,align=left] {Combinatorial\\ aspects\\ of tessellation\\ and \\ tiling problems};
\draw (123.27999999999999, -75.9) rectangle (126.62999999999998,-78.5);
\draw(126.72999999999999, -75.9) node[anchor=north west,align=left] {Combinatorial\\ aspects\\ of matroids\\ and geometric\\ lattices};
\draw (126.72999999999999, -75.9) rectangle (129.82999999999998,-78.5);
\draw(129.92999999999998, -75.9) node[anchor=north west,align=left] {Combinatorial\\ aspects\\ of packing\\ and covering};
\draw (129.92999999999998, -75.9) rectangle (133.02999999999997,-78.0);
\draw(123.27999999999999, -78.6) node[anchor=north west,align=left] {Combinatorial\\ aspects\\ of \\ block designs};
\draw (123.27999999999999, -78.6) rectangle (126.12999999999998,-80.69999999999999);
\draw(126.22999999999999, -78.6) node[anchor=north west,align=left] {Other \\ designs, \\ configurations};
\draw (126.22999999999999, -78.6) rectangle (129.07999999999998,-80.19999999999999);
\draw(129.17999999999998, -78.6) node[anchor=north west,align=left] {Polyominoes};
\draw (129.17999999999998, -78.6) rectangle (132.02999999999997,-79.69999999999999);
\draw(133.48, -69.0) node[anchor=north west,align=left] {\large Algebraic combinatorics};
\draw (133.48, -69.0) rectangle (141.88,-78.8);
\draw(134.48, -70.0) node[anchor=north west,align=left] {Combinatorial\\ aspects\\ of groups\\ and algebras};
\draw (134.48, -70.0) rectangle (138.07999999999998,-72.1);
\draw(138.17999999999998, -70.0) node[anchor=north west,align=left] {Association\\ schemes,\\ strongly\\ regular graphs};
\draw (138.17999999999998, -70.0) rectangle (141.77999999999997,-72.1);
\draw(134.48, -72.2) node[anchor=north west,align=left] {Combinatorial\\ aspects\\ of \\ commutative algebra};
\draw (134.48, -72.2) rectangle (138.07999999999998,-74.3);
\draw(138.17999999999998, -72.2) node[anchor=north west,align=left] {Combinatorial\\ aspects\\ of representation\\ theory};
\draw (138.17999999999998, -72.2) rectangle (141.52999999999997,-74.3);
\draw(134.48, -74.4) node[anchor=north west,align=left] {Combinatorial\\ aspects\\ of algebraic\\ geometry};
\draw (134.48, -74.4) rectangle (137.82999999999998,-76.5);
\draw(137.92999999999998, -74.4) node[anchor=north west,align=left] {Symmetric\\ functions\\ and \\ generalizations};
\draw (137.92999999999998, -74.4) rectangle (141.02999999999997,-76.5);
\draw(134.48, -76.6) node[anchor=north west,align=left] {Combinatorial\\ aspects\\ of simplicial\\ complexes};
\draw (134.48, -76.6) rectangle (137.57999999999998,-78.69999999999999);
\draw(137.67999999999998, -76.6) node[anchor=north west,align=left] {Group actions\\ on \\ combinatorial\\ structures};
\draw (137.67999999999998, -76.6) rectangle (140.52999999999997,-78.69999999999999);
\draw(133.48, -78.9) node[anchor=north west,align=left] {\large Computational methods\\ for problems \\ pertaining to combinatorics};
\draw (133.48, -78.9) rectangle (140.28,-80.5);
\draw(122.27999999999999, -80.9) node[anchor=north west,align=left] {\large Graph theory};
\draw (122.27999999999999, -80.9) rectangle (132.17999999999998,-118.4);
\draw(123.27999999999999, -81.9) node[anchor=north west,align=left] {Isomorphism \\ problems in graph \\ theory (reconstruction\\ conjecture,\\ etc.) and \\ homomorphisms (subgraph\\ embedding, etc.)};
\draw (123.27999999999999, -81.9) rectangle (128.13,-85.5);
\draw(128.23, -81.9) node[anchor=north west,align=left] {Graphs and\\ linear \\ algebra \\ (matrices, \\ eigenvalues, etc.)};
\draw (128.23, -81.9) rectangle (132.07999999999998,-84.5);
\draw(128.23, -84.60000000000001) node[anchor=north west,align=left] {Trees};
\draw (128.23, -84.60000000000001) rectangle (129.57999999999998,-85.2);
\draw(123.27999999999999, -85.60000000000001) node[anchor=north west,align=left] {Vertex subsets\\ with special\\ properties \\ (dominating sets,\\ independent \\ sets, cliques, etc.)};
\draw (123.27999999999999, -85.60000000000001) rectangle (128.13,-88.7);
\draw(128.23, -85.60000000000001) node[anchor=north west,align=left] {Graphical \\ indices (Wiener\\ index, Zagreb\\ index, Randić\\ index, etc.)};
\draw (128.23, -85.60000000000001) rectangle (131.82999999999998,-88.2);
\draw(123.27999999999999, -88.80000000000001) node[anchor=north west,align=left] {Edge subsets with\\ special properties\\ (factorization,\\ matching, \\ partitioning, covering\\ and packing, etc.)};
\draw (123.27999999999999, -88.80000000000001) rectangle (128.13,-91.9);
\draw(128.23, -88.80000000000001) node[anchor=north west,align=left] {Graph labelling\\ (graceful\\ graphs, \\ bandwidth, etc.)};
\draw (128.23, -88.80000000000001) rectangle (131.82999999999998,-90.9);
\draw(123.27999999999999, -92.0) node[anchor=north west,align=left] {Planar graphs;\\ geometric\\ and topological\\ aspects\\ of graph theory};
\draw (123.27999999999999, -92.0) rectangle (127.37999999999998,-94.6);
\draw(127.47999999999999, -92.0) node[anchor=north west,align=left] {Graph representations\\ (geometric\\ and \\ intersection \\ representations, etc.)};
\draw (127.47999999999999, -92.0) rectangle (131.57999999999998,-94.6);
\draw(123.27999999999999, -94.7) node[anchor=north west,align=left] {Small world\\ graphs, complex\\ networks\\ (graph-theoretic\\ aspects)};
\draw (123.27999999999999, -94.7) rectangle (126.87999999999998,-97.3);
\draw(126.97999999999999, -94.7) node[anchor=north west,align=left] {Distance\\ in graphs};
\draw (126.97999999999999, -94.7) rectangle (130.32999999999998,-96.3);
\draw(123.27999999999999, -97.4) node[anchor=north west,align=left] {Graphs and\\ abstract \\ algebra (groups,\\ rings,\\ fields, etc.)};
\draw (123.27999999999999, -97.4) rectangle (126.62999999999998,-100.0);
\draw(126.72999999999999, -97.4) node[anchor=north west,align=left] {Extremal \\ problems in\\ graph theory};
\draw (126.72999999999999, -97.4) rectangle (130.07999999999998,-99.0);
\draw(130.17999999999998, -97.4) node[anchor=north west,align=left] {Graph\\ minors};
\draw (130.17999999999998, -97.4) rectangle (132.02999999999997,-98.5);
\draw(123.27999999999999, -100.10000000000001) node[anchor=north west,align=left] {Fractional\\ graph \\ theory, fuzzy\\ graph theory};
\draw (123.27999999999999, -100.10000000000001) rectangle (126.62999999999998,-102.2);
\draw(126.72999999999999, -100.10000000000001) node[anchor=north west,align=left] {Structural \\ characterization\\ of families\\ of graphs};
\draw (126.72999999999999, -100.10000000000001) rectangle (130.07999999999998,-102.2);
\draw(123.27999999999999, -102.30000000000001) node[anchor=north west,align=left] {Graph \\ operations (line\\ graphs, \\ products, etc.)};
\draw (123.27999999999999, -102.30000000000001) rectangle (126.62999999999998,-104.4);
\draw(126.72999999999999, -102.30000000000001) node[anchor=north west,align=left] {Directed\\ graphs \\ (digraphs),\\ tournaments};
\draw (126.72999999999999, -102.30000000000001) rectangle (129.82999999999998,-104.4);
\draw(129.92999999999998, -102.30000000000001) node[anchor=north west,align=left] {Vertex\\ degrees};
\draw (129.92999999999998, -102.30000000000001) rectangle (132.02999999999997,-103.4);
\draw(123.27999999999999, -104.5) node[anchor=north west,align=left] {Connectivity};
\draw (123.27999999999999, -104.5) rectangle (126.37999999999998,-105.6);
\draw(126.47999999999999, -104.5) node[anchor=north west,align=left] {Graph designs\\ and \\ isomorphic \\ decomposition};
\draw (126.47999999999999, -104.5) rectangle (129.57999999999998,-106.6);
\draw(129.67999999999998, -104.5) node[anchor=north west,align=left] {Graph \\ polynomials};
\draw (129.67999999999998, -104.5) rectangle (132.02999999999997,-105.6);
\draw(123.27999999999999, -106.7) node[anchor=north west,align=left] {Random \\ graphs \\ (graph-theoretic\\ aspects)};
\draw (123.27999999999999, -106.7) rectangle (126.37999999999998,-108.8);
\draw(126.47999999999999, -106.7) node[anchor=north west,align=left] {Graph \\ algorithms \\ (graph-theoretic\\ aspects)};
\draw (126.47999999999999, -106.7) rectangle (129.57999999999998,-108.8);
\draw(129.67999999999998, -106.7) node[anchor=north west,align=left] {Paths \\ and cycles};
\draw (129.67999999999998, -106.7) rectangle (132.02999999999997,-107.8);
\draw(123.27999999999999, -108.9) node[anchor=north west,align=left] {Enumeration\\ in \\ graph theory};
\draw (123.27999999999999, -108.9) rectangle (126.12999999999998,-110.5);
\draw(126.22999999999999, -108.9) node[anchor=north west,align=left] {Eulerian \\ and Hamiltonian\\ graphs};
\draw (126.22999999999999, -108.9) rectangle (129.07999999999998,-110.5);
\draw(129.17999999999998, -108.9) node[anchor=north west,align=left] {Games on \\ graphs \\ (graph-theoretic\\ aspects)};
\draw (129.17999999999998, -108.9) rectangle (132.02999999999997,-111.0);
\draw(123.27999999999999, -111.1) node[anchor=north west,align=left] {Hypergraphs};
\draw (123.27999999999999, -111.1) rectangle (126.12999999999998,-112.19999999999999);
\draw(126.22999999999999, -111.1) node[anchor=north west,align=left] {Coloring\\ of graphs\\ and \\ hypergraphs};
\draw (126.22999999999999, -111.1) rectangle (128.82999999999998,-113.19999999999999);
\draw(128.92999999999998, -111.1) node[anchor=north west,align=left] {Signed \\ and weighted\\ graphs};
\draw (128.92999999999998, -111.1) rectangle (131.52999999999997,-112.69999999999999);
\draw(123.27999999999999, -113.30000000000001) node[anchor=north west,align=left] {Applications\\ of \\ graph theory};
\draw (123.27999999999999, -113.30000000000001) rectangle (125.87999999999998,-114.9);
\draw(125.97999999999999, -113.30000000000001) node[anchor=north west,align=left] {Density\\ (toughness,\\ etc.)};
\draw (125.97999999999999, -113.30000000000001) rectangle (128.32999999999998,-114.9);
\draw(128.42999999999998, -113.30000000000001) node[anchor=north west,align=left] {Generalized\\ Ramsey\\ theory};
\draw (128.42999999999998, -113.30000000000001) rectangle (130.77999999999997,-114.9);
\draw(123.27999999999999, -115.0) node[anchor=north west,align=left] {Chemical\\ graph\\ theory};
\draw (123.27999999999999, -115.0) rectangle (125.62999999999998,-116.6);
\draw(125.72999999999999, -115.0) node[anchor=north west,align=left] {Perfect\\ graphs};
\draw (125.72999999999999, -115.0) rectangle (127.82999999999998,-116.1);
\draw(127.92999999999999, -115.0) node[anchor=north west,align=left] {Flows \\ in graphs};
\draw (127.92999999999999, -115.0) rectangle (130.03,-116.1);
\draw(123.27999999999999, -116.70000000000002) node[anchor=north west,align=left] {Expander\\ graphs};
\draw (123.27999999999999, -116.70000000000002) rectangle (125.37999999999998,-117.80000000000001);
\draw(125.47999999999999, -116.70000000000002) node[anchor=north west,align=left] {Infinite\\ graphs};
\draw (125.47999999999999, -116.70000000000002) rectangle (127.57999999999998,-117.80000000000001);
\draw(127.67999999999999, -116.70000000000002) node[anchor=north west,align=left] {Random\\ walks \\ on graphs};
\draw (127.67999999999999, -116.70000000000002) rectangle (129.78,-118.30000000000001);
\draw(132.27999999999997, -80.9) node[anchor=north west,align=left] {\large Enumerative combinatorics};
\draw (132.27999999999997, -80.9) rectangle (142.12999999999997,-92.9);
\draw(133.27999999999997, -81.9) node[anchor=north west,align=left] {Combinatorial\\ inequalities};
\draw (133.27999999999997, -81.9) rectangle (137.87999999999997,-83.5);
\draw(137.97999999999996, -81.9) node[anchor=north west,align=left] {Exact \\ enumeration \\ problems, \\ generating functions};
\draw (137.97999999999996, -81.9) rectangle (141.82999999999996,-84.0);
\draw(133.27999999999997, -84.10000000000001) node[anchor=north west,align=left] {Asymptotic\\ enumeration};
\draw (133.27999999999997, -84.10000000000001) rectangle (137.12999999999997,-85.7);
\draw(137.22999999999996, -84.10000000000001) node[anchor=north west,align=left] {Factorials,\\ binomial \\ coefficients,\\ combinatorial\\ functions};
\draw (137.22999999999996, -84.10000000000001) rectangle (140.82999999999996,-86.7);
\draw(133.27999999999997, -86.80000000000001) node[anchor=north west,align=left] {Combinatorial\\ aspects\\ of partitions\\ of integers};
\draw (133.27999999999997, -86.80000000000001) rectangle (136.87999999999997,-88.9);
\draw(136.97999999999996, -86.80000000000001) node[anchor=north west,align=left] {Partitions\\ of sets};
\draw (136.97999999999996, -86.80000000000001) rectangle (140.32999999999996,-88.4);
\draw(133.27999999999997, -89.0) node[anchor=north west,align=left] {Combinatorial\\ identities,\\ bijective\\ combinatorics};
\draw (133.27999999999997, -89.0) rectangle (136.62999999999997,-91.1);
\draw(136.72999999999996, -89.0) node[anchor=north west,align=left] {\(q\)-calculus\\ and \\ related topics};
\draw (136.72999999999996, -89.0) rectangle (139.82999999999996,-90.6);
\draw(139.92999999999998, -89.0) node[anchor=north west,align=left] {Umbral\\ calculus};
\draw (139.92999999999998, -89.0) rectangle (142.02999999999997,-90.1);
\draw(133.27999999999997, -91.2) node[anchor=north west,align=left] {Permutations,\\ words,\\ matrices};
\draw (133.27999999999997, -91.2) rectangle (136.12999999999997,-92.8);
\draw(132.27999999999997, -93.0) node[anchor=north west,align=left] {\large Extremal combinatorics};
\draw (132.27999999999997, -93.0) rectangle (140.17999999999998,-99.4);
\draw(133.27999999999997, -94.0) node[anchor=north west,align=left] {Probabilistic \\ methods in extremal\\ combinatorics,\\ including \\ polynomial methods \\ (combinatorial \\ Nullstellensatz, etc.)};
\draw (133.27999999999997, -94.0) rectangle (138.12999999999997,-97.6);
\draw(138.22999999999996, -94.0) node[anchor=north west,align=left] {Extremal\\ set\\ theory};
\draw (138.22999999999996, -94.0) rectangle (140.07999999999996,-95.6);
\draw(138.22999999999996, -95.7) node[anchor=north west,align=left] {Ramsey\\ theory};
\draw (138.22999999999996, -95.7) rectangle (140.07999999999996,-96.8);
\draw(133.27999999999997, -97.7) node[anchor=north west,align=left] {Transversal\\ (matching)\\ theory};
\draw (133.27999999999997, -97.7) rectangle (136.12999999999997,-99.3);
\draw(132.27999999999997, -99.5) node[anchor=north west,align=left] {\large History of\\ combinatorics};
\draw (132.27999999999997, -99.5) rectangle (136.28999999999996,-100.6);
\draw(142.32999999999998, -1) node[anchor=north west,align=left] {\LARGE Functions of a complex variable};
\draw (142.32999999999998, -1) rectangle (194.98,-45.1);
\draw(143.32999999999998, -2) node[anchor=north west,align=left] {\large Entire and meromorphic functions of one complex variable, and related topics};
\draw (143.32999999999998, -2) rectangle (171.02999999999997,-9.4);
\draw(144.32999999999998, -3) node[anchor=north west,align=left] {Special classes\\ of entire \\ functions of one \\ complex variable\\ and growth estimates};
\draw (144.32999999999998, -3) rectangle (149.42999999999998,-5.6);
\draw(149.52999999999997, -3) node[anchor=north west,align=left] {Functional equations\\ in the complex\\ plane, iteration\\ and composition\\ of analytic\\ functions of one\\ complex variable};
\draw (149.52999999999997, -3) rectangle (154.12999999999997,-6.6);
\draw(154.23, -3) node[anchor=north west,align=left] {Representations\\ of entire\\ functions of\\ one complex \\ variable by \\ series and integrals};
\draw (154.23, -3) rectangle (158.82999999999998,-6.1);
\draw(158.92999999999998, -3) node[anchor=north west,align=left] {Quasi-analytic\\ and other \\ classes of \\ functions of one\\ complex variable};
\draw (158.92999999999998, -3) rectangle (163.27999999999997,-5.6);
\draw(163.38, -3) node[anchor=north west,align=left] {Value distribution\\ of meromorphic\\ functions\\ of one complex\\ variable, \\ Nevanlinna theory};
\draw (163.38, -3) rectangle (167.48,-6.1);
\draw(167.57999999999998, -3) node[anchor=north west,align=left] {Meromorphic\\ functions of\\ one complex\\ variable, \\ general theory};
\draw (167.57999999999998, -3) rectangle (170.92999999999998,-5.6);
\draw(144.32999999999998, -6.7) node[anchor=north west,align=left] {Cluster \\ sets, prime\\ ends, \\ boundary behavior};
\draw (144.32999999999998, -6.7) rectangle (147.67999999999998,-8.8);
\draw(147.77999999999997, -6.7) node[anchor=north west,align=left] {Normal \\ functions of one\\ complex \\ variable, \\ normal families};
\draw (147.77999999999997, -6.7) rectangle (151.12999999999997,-9.3);
\draw(151.23, -6.7) node[anchor=north west,align=left] {Entire \\ functions of one\\ complex \\ variable, \\ general theory};
\draw (151.23, -6.7) rectangle (154.32999999999998,-9.3);
\draw(171.13, -2) node[anchor=north west,align=left] {\large Series expansions of functions of one complex variable};
\draw (171.13, -2) rectangle (190.68,-8.4);
\draw(172.13, -3) node[anchor=north west,align=left] {Boundary behavior\\ of power \\ series in one \\ complex variable;\\ over-convergence};
\draw (172.13, -3) rectangle (176.98,-5.6);
\draw(177.07999999999998, -3) node[anchor=north west,align=left] {Completeness \\ problems, closure\\ of a system of\\ functions of \\ one complex variable};
\draw (177.07999999999998, -3) rectangle (181.92999999999998,-5.6);
\draw(182.03, -3) node[anchor=north west,align=left] {Dirichlet series,\\ exponential\\ series and other\\ series in one\\ complex variable};
\draw (182.03, -3) rectangle (186.38,-5.6);
\draw(186.48, -3) node[anchor=north west,align=left] {Power series\\ (including\\ lacunary \\ series) in one\\ complex variable};
\draw (186.48, -3) rectangle (190.57999999999998,-5.6);
\draw(172.13, -5.7) node[anchor=north west,align=left] {Analytic \\ continuation\\ of functions\\ of one \\ complex variable};
\draw (172.13, -5.7) rectangle (175.73,-8.3);
\draw(175.82999999999998, -5.7) node[anchor=north west,align=left] {Random power\\ series\\ in one \\ complex variable};
\draw (175.82999999999998, -5.7) rectangle (179.17999999999998,-7.800000000000001);
\draw(179.28, -5.7) node[anchor=north west,align=left] {Continued \\ fractions; \\ complex-analytic\\ aspects};
\draw (179.28, -5.7) rectangle (182.38,-7.800000000000001);
\draw(143.32999999999998, -9.5) node[anchor=north west,align=left] {\large Spaces and algebras of analytic functions of one complex variable};
\draw (143.32999999999998, -9.5) rectangle (167.98,-14.4);
\draw(144.32999999999998, -10.5) node[anchor=north west,align=left] {Spaces of \\ bounded \\ analytic functions\\ of one\\ complex variable};
\draw (144.32999999999998, -10.5) rectangle (148.42999999999998,-13.1);
\draw(148.52999999999997, -10.5) node[anchor=north west,align=left] {Besov \\ spaces and\\ \(Q_p\)-spaces};
\draw (148.52999999999997, -10.5) rectangle (152.37999999999997,-12.1);
\draw(152.48, -10.5) node[anchor=north west,align=left] {Algebras of \\ analytic functions\\ of one \\ complex variable};
\draw (152.48, -10.5) rectangle (156.32999999999998,-12.6);
\draw(156.42999999999998, -10.5) node[anchor=north west,align=left] {Nevanlinna\\ spaces\\ and \\ Smirnov spaces};
\draw (156.42999999999998, -10.5) rectangle (159.52999999999997,-12.6);
\draw(159.63, -10.5) node[anchor=north west,align=left] {Bergman \\ spaces and\\ Fock spaces};
\draw (159.63, -10.5) rectangle (162.73,-12.1);
\draw(162.82999999999998, -10.5) node[anchor=north west,align=left] {BMO-spaces};
\draw (162.82999999999998, -10.5) rectangle (165.42999999999998,-11.6);
\draw(165.52999999999997, -10.5) node[anchor=north west,align=left] {de \\ Branges-Rovnyak\\ spaces};
\draw (165.52999999999997, -10.5) rectangle (167.87999999999997,-12.1);
\draw(144.32999999999998, -13.2) node[anchor=north west,align=left] {Zygmund\\ spaces};
\draw (144.32999999999998, -13.2) rectangle (146.42999999999998,-14.299999999999999);
\draw(146.52999999999997, -13.2) node[anchor=north west,align=left] {Corona\\ theorems};
\draw (146.52999999999997, -13.2) rectangle (148.62999999999997,-14.299999999999999);
\draw(148.73, -13.2) node[anchor=north west,align=left] {Hardy\\ spaces};
\draw (148.73, -13.2) rectangle (150.57999999999998,-14.299999999999999);
\draw(150.67999999999998, -13.2) node[anchor=north west,align=left] {Bloch\\ spaces};
\draw (150.67999999999998, -13.2) rectangle (152.52999999999997,-14.299999999999999);
\draw(168.07999999999998, -9.5) node[anchor=north west,align=left] {\large Miscellaneous topics of analysis in the complex plane};
\draw (168.07999999999998, -9.5) rectangle (185.88,-15.9);
\draw(169.07999999999998, -10.5) node[anchor=north west,align=left] {Integration, \\ integrals of Cauchy \\ type, integral \\ representations of \\ analytic functions\\ in the complex plane};
\draw (169.07999999999998, -10.5) rectangle (174.42999999999998,-13.6);
\draw(174.52999999999997, -10.5) node[anchor=north west,align=left] {Moment problems\\ and \\ interpolation \\ problems in the\\ complex plane};
\draw (174.52999999999997, -10.5) rectangle (178.37999999999997,-13.1);
\draw(178.48, -10.5) node[anchor=north west,align=left] {Asymptotic\\ representations\\ in the\\ complex plane};
\draw (178.48, -10.5) rectangle (182.07999999999998,-12.6);
\draw(182.17999999999998, -10.5) node[anchor=north west,align=left] {Boundary \\ value problems\\ in the\\ complex plane};
\draw (182.17999999999998, -10.5) rectangle (185.77999999999997,-12.6);
\draw(169.07999999999998, -13.7) node[anchor=north west,align=left] {Approximation\\ in\\ the \\ complex plane};
\draw (169.07999999999998, -13.7) rectangle (171.67999999999998,-15.799999999999999);
\draw(143.32999999999998, -14.5) node[anchor=north west,align=left] {\large History of functions\\ of a complex variable};
\draw (143.32999999999998, -14.5) rectangle (150.13,-15.6);
\draw(185.98, -9.5) node[anchor=north west,align=left] {\large Riemann surfaces};
\draw (185.98, -9.5) rectangle (194.88,-23.5);
\draw(186.98, -10.5) node[anchor=north west,align=left] {Fuchsian groups\\ and automorphic\\ functions (aspects\\ of compact \\ Riemann surfaces \\ and uniformization)};
\draw (186.98, -10.5) rectangle (191.57999999999998,-13.6);
\draw(191.67999999999998, -10.5) node[anchor=north west,align=left] {Ideal \\ boundary theory\\ for Riemann\\ surfaces};
\draw (191.67999999999998, -10.5) rectangle (194.77999999999997,-12.6);
\draw(186.98, -13.7) node[anchor=north west,align=left] {Kleinian groups\\ (aspects of\\ compact Riemann\\ surfaces and\\ uniformization)};
\draw (186.98, -13.7) rectangle (191.07999999999998,-16.3);
\draw(191.17999999999998, -13.7) node[anchor=north west,align=left] {Compact \\ Riemann \\ surfaces and \\ uniformization};
\draw (191.17999999999998, -13.7) rectangle (194.52999999999997,-15.799999999999999);
\draw(186.98, -16.4) node[anchor=north west,align=left] {Conformal \\ metrics \\ (hyperbolic, \\ Poincaré, \\ distance functions)};
\draw (186.98, -16.4) rectangle (191.07999999999998,-19.0);
\draw(191.17999999999998, -16.4) node[anchor=north west,align=left] {Differentials\\ on \\ Riemann surfaces};
\draw (191.17999999999998, -16.4) rectangle (194.52999999999997,-18.0);
\draw(186.98, -19.1) node[anchor=north west,align=left] {Teichmüller\\ theory\\ for Riemann\\ surfaces};
\draw (186.98, -19.1) rectangle (190.07999999999998,-21.200000000000003);
\draw(190.17999999999998, -19.1) node[anchor=north west,align=left] {Harmonic\\ functions\\ on Riemann\\ surfaces};
\draw (190.17999999999998, -19.1) rectangle (193.02999999999997,-21.200000000000003);
\draw(186.98, -21.3) node[anchor=north west,align=left] {Classification\\ theory\\ of Riemann\\ surfaces};
\draw (186.98, -21.3) rectangle (189.82999999999998,-23.400000000000002);
\draw(189.92999999999998, -21.3) node[anchor=north west,align=left] {Klein\\ surfaces};
\draw (189.92999999999998, -21.3) rectangle (192.02999999999997,-22.400000000000002);
\draw(168.07999999999998, -16.0) node[anchor=north west,align=left] {\large General properties of functions of one complex variable};
\draw (168.07999999999998, -16.0) rectangle (185.73,-19.7);
\draw(169.07999999999998, -17.0) node[anchor=north west,align=left] {Monogenic \\ and polygenic\\ functions\\ of one \\ complex variable};
\draw (169.07999999999998, -17.0) rectangle (172.42999999999998,-19.6);
\draw(172.52999999999997, -17.0) node[anchor=north west,align=left] {Inequalities\\ in the\\ complex plane};
\draw (172.52999999999997, -17.0) rectangle (175.87999999999997,-18.6);
\draw(168.07999999999998, -19.800000000000004) node[anchor=north west,align=left] {\large Universal holomorphic functions of one complex variable};
\draw (168.07999999999998, -19.800000000000004) rectangle (185.73,-23.000000000000004);
\draw(169.07999999999998, -20.800000000000004) node[anchor=north west,align=left] {Compositional\\ universality};
\draw (169.07999999999998, -20.800000000000004) rectangle (173.67999999999998,-22.400000000000006);
\draw(173.77999999999997, -20.800000000000004) node[anchor=north west,align=left] {Universal \\ Taylor series\\ in one \\ complex variable};
\draw (173.77999999999997, -20.800000000000004) rectangle (177.37999999999997,-22.900000000000006);
\draw(177.48, -20.800000000000004) node[anchor=north west,align=left] {Universal \\ Dirichlet series\\ in one \\ complex variable};
\draw (177.48, -20.800000000000004) rectangle (181.07999999999998,-22.900000000000006);
\draw(181.17999999999998, -20.800000000000004) node[anchor=north west,align=left] {Universal\\ functions\\ of one \\ complex variable};
\draw (181.17999999999998, -20.800000000000004) rectangle (184.52999999999997,-22.900000000000006);
\draw(143.32999999999998, -23.6) node[anchor=north west,align=left] {\large Geometric function theory};
\draw (143.32999999999998, -23.6) rectangle (154.48,-45.0);
\draw(144.32999999999998, -24.6) node[anchor=north west,align=left] {Schwarz-Christoffel-type\\ mappings};
\draw (144.32999999999998, -24.6) rectangle (150.17999999999998,-26.200000000000003);
\draw(150.27999999999997, -24.6) node[anchor=north west,align=left] {Quasiconformal\\ mappings\\ in \(\mathbb{R}^n\), other\\ generalizations};
\draw (150.27999999999997, -24.6) rectangle (154.12999999999997,-26.700000000000003);
\draw(144.32999999999998, -26.8) node[anchor=north west,align=left] {Zeros of polynomials,\\ rational functions,\\ and other analytic\\ functions of one\\ complex variable\\ (e.g., zeros of \\ functions with bounded\\ Dirichlet integral)};
\draw (144.32999999999998, -26.8) rectangle (149.42999999999998,-30.9);
\draw(149.52999999999997, -26.8) node[anchor=north west,align=left] {Special classes \\ of univalent and \\ multivalent functions\\ of one complex\\ variable (starlike,\\ convex, bounded\\ rotation, etc.)};
\draw (149.52999999999997, -26.8) rectangle (154.37999999999997,-30.400000000000002);
\draw(144.32999999999998, -31.0) node[anchor=north west,align=left] {Extremal problems\\ for conformal\\ and quasiconformal\\ mappings,\\ variational methods};
\draw (144.32999999999998, -31.0) rectangle (149.42999999999998,-33.6);
\draw(149.52999999999997, -31.0) node[anchor=north west,align=left] {Coefficient \\ problems for \\ univalent and\\ multivalent \\ functions of \\ one complex variable};
\draw (149.52999999999997, -31.0) rectangle (154.37999999999997,-34.1);
\draw(144.32999999999998, -34.2) node[anchor=north west,align=left] {Maximum principle,\\ Schwarz’s \\ lemma, Lindelöf \\ principle, analogues\\ and generalizations;\\ subordination};
\draw (144.32999999999998, -34.2) rectangle (149.42999999999998,-37.300000000000004);
\draw(149.52999999999997, -34.2) node[anchor=north west,align=left] {Extremal problems\\ for conformal\\ and \\ quasiconformal mappings,\\ other methods};
\draw (149.52999999999997, -34.2) rectangle (154.12999999999997,-36.800000000000004);
\draw(144.32999999999998, -37.400000000000006) node[anchor=north west,align=left] {General theory\\ of univalent and\\ multivalent \\ functions of one\\ complex variable};
\draw (144.32999999999998, -37.400000000000006) rectangle (148.67999999999998,-40.00000000000001);
\draw(148.77999999999997, -37.400000000000006) node[anchor=north west,align=left] {Polynomials\\ and rational\\ functions\\ of one \\ complex variable};
\draw (148.77999999999997, -37.400000000000006) rectangle (152.37999999999997,-40.00000000000001);
\draw(144.32999999999998, -40.1) node[anchor=north west,align=left] {Quasiconformal\\ mappings\\ in the\\ complex plane};
\draw (144.32999999999998, -40.1) rectangle (147.92999999999998,-42.2);
\draw(148.02999999999997, -40.1) node[anchor=north west,align=left] {Capacity and\\ harmonic \\ measure in the\\ complex plane};
\draw (148.02999999999997, -40.1) rectangle (151.62999999999997,-42.2);
\draw(151.73, -40.1) node[anchor=north west,align=left] {Conformal\\ mappings\\ of special\\ domains};
\draw (151.73, -40.1) rectangle (154.32999999999998,-42.2);
\draw(144.32999999999998, -42.3) node[anchor=north west,align=left] {Kernel \\ functions in one\\ complex \\ variable and\\ applications};
\draw (144.32999999999998, -42.3) rectangle (147.67999999999998,-44.9);
\draw(147.77999999999997, -42.3) node[anchor=north west,align=left] {Covering \\ theorems in \\ conformal \\ mapping theory};
\draw (147.77999999999997, -42.3) rectangle (150.87999999999997,-44.4);
\draw(150.98, -42.3) node[anchor=north west,align=left] {General\\ theory \\ of conformal\\ mappings};
\draw (150.98, -42.3) rectangle (154.07999999999998,-44.4);
\draw(154.57999999999998, -23.6) node[anchor=north west,align=left] {\large Generalized function theory};
\draw (154.57999999999998, -23.6) rectangle (164.98,-32.7);
\draw(155.57999999999998, -24.6) node[anchor=north west,align=left] {Non-Archimedean\\ function theory};
\draw (155.57999999999998, -24.6) rectangle (160.92999999999998,-26.200000000000003);
\draw(161.02999999999997, -24.6) node[anchor=north west,align=left] {Functions of\\ hypercomplex\\ variables\\ and generalized\\ variables};
\draw (161.02999999999997, -24.6) rectangle (164.37999999999997,-27.200000000000003);
\draw(155.57999999999998, -27.3) node[anchor=north west,align=left] {Generalizations\\ of Bers and\\ Vekua type \\ (pseudoanalytic, \\ \(p\)-analytic, etc.)};
\draw (155.57999999999998, -27.3) rectangle (160.42999999999998,-29.900000000000002);
\draw(160.52999999999997, -27.3) node[anchor=north west,align=left] {Other generalizations\\ of analytic\\ functions \\ (including abstract-valued\\ functions)};
\draw (160.52999999999997, -27.3) rectangle (164.87999999999997,-29.900000000000002);
\draw(155.57999999999998, -30.0) node[anchor=north west,align=left] {Finely \\ holomorphic \\ functions and\\ topological\\ function theory};
\draw (155.57999999999998, -30.0) rectangle (159.67999999999998,-32.6);
\draw(159.77999999999997, -30.0) node[anchor=north west,align=left] {Discrete\\ analytic\\ functions};
\draw (159.77999999999997, -30.0) rectangle (162.62999999999997,-31.6);
\draw(154.57999999999998, -32.800000000000004) node[anchor=north west,align=left] {\large Analysis on metric spaces};
\draw (154.57999999999998, -32.800000000000004) rectangle (164.23,-38.2);
\draw(155.57999999999998, -33.800000000000004) node[anchor=north west,align=left] {Quasiconformal\\ mappings in\\ metric spaces};
\draw (155.57999999999998, -33.800000000000004) rectangle (161.17999999999998,-35.900000000000006);
\draw(161.27999999999997, -33.800000000000004) node[anchor=north west,align=left] {Inequalities\\ in \\ metric spaces};
\draw (161.27999999999997, -33.800000000000004) rectangle (164.12999999999997,-35.400000000000006);
\draw(155.57999999999998, -36.00000000000001) node[anchor=north west,align=left] {Geometric\\ embeddings\\ of \\ metric spaces};
\draw (155.57999999999998, -36.00000000000001) rectangle (158.17999999999998,-38.10000000000001);
\draw(154.57999999999998, -38.300000000000004) node[anchor=north west,align=left] {\large Function theory on the disc};
\draw (154.57999999999998, -38.300000000000004) rectangle (163.54999999999998,-42.7);
\draw(155.57999999999998, -39.300000000000004) node[anchor=north west,align=left] {Singular \\ inner functions\\ of one \\ complex variable};
\draw (155.57999999999998, -39.300000000000004) rectangle (159.42999999999998,-41.400000000000006);
\draw(159.52999999999997, -39.300000000000004) node[anchor=north west,align=left] {Inner \\ functions of\\ one complex\\ variable};
\draw (159.52999999999997, -39.300000000000004) rectangle (162.62999999999997,-41.400000000000006);
\draw(155.57999999999998, -41.50000000000001) node[anchor=north west,align=left] {Blaschke\\ products};
\draw (155.57999999999998, -41.50000000000001) rectangle (157.92999999999998,-42.60000000000001);
\draw(154.57999999999998, -42.800000000000004) node[anchor=north west,align=left] {\large Computational methods for\\ problems pertaining to \\ functions of a complex variable};
\draw (154.57999999999998, -42.800000000000004) rectangle (163.23999999999998,-44.400000000000006);
\draw(142.32999999999998, -45.2) node[anchor=north west,align=left] {\LARGE Number theory};
\draw (142.32999999999998, -45.2) rectangle (192.78,-130.60000000000002);
\draw(143.32999999999998, -46.2) node[anchor=north west,align=left] {\large Probabilistic theory: distribution modulo \(1\); metric theory of algorithms};
\draw (143.32999999999998, -46.2) rectangle (171.73,-53.1);
\draw(144.32999999999998, -47.2) node[anchor=north west,align=left] {Normal numbers,\\ radix expansions,\\ Pisot \\ numbers, Salem \\ numbers, good \\ lattice points, etc.};
\draw (144.32999999999998, -47.2) rectangle (149.17999999999998,-50.300000000000004);
\draw(149.27999999999997, -47.2) node[anchor=north west,align=left] {Metric theory of\\ other algorithms\\ and expansions;\\ measure and \\ Hausdorff dimension};
\draw (149.27999999999997, -47.2) rectangle (153.62999999999997,-49.800000000000004);
\draw(153.73, -47.2) node[anchor=north west,align=left] {General \\ theory of\\ distribution\\ modulo \(1\)};
\draw (153.73, -47.2) rectangle (157.57999999999998,-49.300000000000004);
\draw(157.67999999999998, -47.2) node[anchor=north west,align=left] {Arithmetic \\ functions in\\ probabilistic\\ number theory};
\draw (157.67999999999998, -47.2) rectangle (161.52999999999997,-49.300000000000004);
\draw(161.63, -47.2) node[anchor=north west,align=left] {Harmonic analysis\\ and almost\\ periodicity \\ in probabilistic\\ number theory};
\draw (161.63, -47.2) rectangle (165.48,-49.800000000000004);
\draw(165.57999999999998, -47.2) node[anchor=north west,align=left] {Well-distributed\\ sequences\\ and \\ other variations};
\draw (165.57999999999998, -47.2) rectangle (169.17999999999998,-49.300000000000004);
\draw(169.27999999999997, -47.2) node[anchor=north west,align=left] {Special\\ sequences};
\draw (169.27999999999997, -47.2) rectangle (171.62999999999997,-48.300000000000004);
\draw(144.32999999999998, -50.400000000000006) node[anchor=north west,align=left] {Irregularities\\ of \\ distribution,\\ discrepancy};
\draw (144.32999999999998, -50.400000000000006) rectangle (147.67999999999998,-52.50000000000001);
\draw(147.77999999999997, -50.400000000000006) node[anchor=north west,align=left] {Pseudo-random\\ numbers;\\ Monte \\ Carlo methods};
\draw (147.77999999999997, -50.400000000000006) rectangle (150.87999999999997,-52.50000000000001);
\draw(150.98, -50.400000000000006) node[anchor=north west,align=left] {Metric \\ theory of\\ continued\\ fractions};
\draw (150.98, -50.400000000000006) rectangle (154.07999999999998,-52.50000000000001);
\draw(154.17999999999998, -50.400000000000006) node[anchor=north west,align=left] {Diophantine\\ approximation\\ in \\ probabilistic \\ number theory};
\draw (154.17999999999998, -50.400000000000006) rectangle (157.27999999999997,-53.00000000000001);
\draw(157.38, -50.400000000000006) node[anchor=north west,align=left] {Continuous,\\ \(p\)-adic \\ and abstract\\ analogues};
\draw (157.38, -50.400000000000006) rectangle (160.23,-52.50000000000001);
\draw(171.82999999999998, -46.2) node[anchor=north west,align=left] {\large Diophantine approximation, transcendental number theory};
\draw (171.82999999999998, -46.2) rectangle (192.63,-59.2);
\draw(172.82999999999998, -47.2) node[anchor=north west,align=left] {Homogeneous\\ approximation \\ to one number};
\draw (172.82999999999998, -47.2) rectangle (178.17999999999998,-49.300000000000004);
\draw(178.27999999999997, -47.2) node[anchor=north west,align=left] {Transcendence\\ (general theory)};
\draw (178.27999999999997, -47.2) rectangle (183.62999999999997,-48.800000000000004);
\draw(183.73, -47.2) node[anchor=north west,align=left] {Inhomogeneous\\ linear forms};
\draw (183.73, -47.2) rectangle (188.32999999999998,-48.800000000000004);
\draw(188.42999999999998, -47.2) node[anchor=north west,align=left] {Simultaneous\\ homogeneous\\ approximation,\\ linear forms};
\draw (188.42999999999998, -47.2) rectangle (192.52999999999997,-49.300000000000004);
\draw(172.82999999999998, -49.400000000000006) node[anchor=north west,align=left] {Transcendence\\ theory \\ of Drinfel’d\\ and \(t\)-modules};
\draw (172.82999999999998, -49.400000000000006) rectangle (177.42999999999998,-51.50000000000001);
\draw(177.52999999999997, -49.400000000000006) node[anchor=north west,align=left] {Diophantine\\ inequalities};
\draw (177.52999999999997, -49.400000000000006) rectangle (181.87999999999997,-51.00000000000001);
\draw(181.98, -49.400000000000006) node[anchor=north west,align=left] {Number-theoretic\\ analogues of \\ methods in Nevanlinna\\ theory (work\\ of Vojta et al.)};
\draw (181.98, -49.400000000000006) rectangle (186.32999999999998,-52.00000000000001);
\draw(186.42999999999998, -49.400000000000006) node[anchor=north west,align=left] {Distribution\\ modulo one};
\draw (186.42999999999998, -49.400000000000006) rectangle (190.52999999999997,-51.00000000000001);
\draw(190.63, -49.400000000000006) node[anchor=north west,align=left] {Metric\\ theory};
\draw (190.63, -49.400000000000006) rectangle (192.48,-50.50000000000001);
\draw(172.82999999999998, -52.1) node[anchor=north west,align=left] {Transcendence\\ theory of \\ elliptic and \\ abelian functions};
\draw (172.82999999999998, -52.1) rectangle (176.67999999999998,-54.2);
\draw(176.77999999999997, -52.1) node[anchor=north west,align=left] {Markov and\\ Lagrange \\ spectra and \\ generalizations};
\draw (176.77999999999997, -52.1) rectangle (180.37999999999997,-54.2);
\draw(180.48, -52.1) node[anchor=north west,align=left] {Irrationality;\\ linear\\ independence\\ over a field};
\draw (180.48, -52.1) rectangle (184.07999999999998,-54.2);
\draw(184.17999999999998, -52.1) node[anchor=north west,align=left] {Transcendence\\ theory \\ of other \\ special functions};
\draw (184.17999999999998, -52.1) rectangle (187.77999999999997,-54.2);
\draw(187.88, -52.1) node[anchor=north west,align=left] {Approximation\\ by \\ numbers from\\ a fixed field};
\draw (187.88, -52.1) rectangle (191.23,-54.2);
\draw(172.82999999999998, -54.300000000000004) node[anchor=north west,align=left] {Small \\ fractional parts\\ of polynomials\\ and \\ generalizations};
\draw (172.82999999999998, -54.300000000000004) rectangle (176.17999999999998,-56.900000000000006);
\draw(176.27999999999997, -54.300000000000004) node[anchor=north west,align=left] {Approximation\\ in \\ non-Archimedean\\ valuations};
\draw (176.27999999999997, -54.300000000000004) rectangle (179.62999999999997,-56.400000000000006);
\draw(179.73, -54.300000000000004) node[anchor=north west,align=left] {Measures of\\ irrationality\\ and of\\ transcendence};
\draw (179.73, -54.300000000000004) rectangle (183.07999999999998,-56.400000000000006);
\draw(183.17999999999998, -54.300000000000004) node[anchor=north west,align=left] {Continued\\ fractions\\ and \\ generalizations};
\draw (183.17999999999998, -54.300000000000004) rectangle (186.27999999999997,-56.400000000000006);
\draw(186.38, -54.300000000000004) node[anchor=north west,align=left] {Linear forms\\ in \\ logarithms; \\ Baker’s method};
\draw (186.38, -54.300000000000004) rectangle (189.48,-56.400000000000006);
\draw(189.57999999999998, -54.300000000000004) node[anchor=north west,align=left] {Algebraic\\ independence;\\ Gel’fond’s\\ method};
\draw (189.57999999999998, -54.300000000000004) rectangle (192.42999999999998,-56.400000000000006);
\draw(172.82999999999998, -57.0) node[anchor=north west,align=left] {Schmidt \\ Subspace \\ Theorem and \\ applications};
\draw (172.82999999999998, -57.0) rectangle (175.67999999999998,-59.1);
\draw(175.77999999999997, -57.0) node[anchor=north west,align=left] {Results\\ involving\\ abelian\\ varieties};
\draw (175.77999999999997, -57.0) rectangle (178.62999999999997,-59.1);
\draw(178.73, -57.0) node[anchor=north west,align=left] {Approximation\\ to\\ algebraic\\ numbers};
\draw (178.73, -57.0) rectangle (181.32999999999998,-59.1);
\draw(143.32999999999998, -53.2) node[anchor=north west,align=left] {\large Finite fields and commutative rings (number-theoretic aspects)};
\draw (143.32999999999998, -53.2) rectangle (165.27999999999997,-58.6);
\draw(144.32999999999998, -54.2) node[anchor=north west,align=left] {Structure \\ theory for finite\\ fields and\\ commutative \\ rings \\ (number-theoretic aspects)};
\draw (144.32999999999998, -54.2) rectangle (148.92999999999998,-57.300000000000004);
\draw(149.02999999999997, -54.2) node[anchor=north west,align=left] {Algebraic \\ coding theory;\\ cryptography\\ (number-theoretic\\ aspects)};
\draw (149.02999999999997, -54.2) rectangle (153.12999999999997,-56.800000000000004);
\draw(153.23, -54.2) node[anchor=north west,align=left] {Polynomials\\ over \\ finite fields};
\draw (153.23, -54.2) rectangle (156.32999999999998,-55.800000000000004);
\draw(156.42999999999998, -54.2) node[anchor=north west,align=left] {Arithmetic\\ theory of \\ polynomial \\ rings over \\ finite fields};
\draw (156.42999999999998, -54.2) rectangle (159.52999999999997,-56.800000000000004);
\draw(159.63, -54.2) node[anchor=north west,align=left] {Other \\ character\\ sums and\\ Gauss sums};
\draw (159.63, -54.2) rectangle (162.48,-56.300000000000004);
\draw(162.57999999999998, -54.2) node[anchor=north west,align=left] {Finite\\ upper \\ half-planes};
\draw (162.57999999999998, -54.2) rectangle (165.17999999999998,-55.800000000000004);
\draw(144.32999999999998, -57.400000000000006) node[anchor=north west,align=left] {Cyclotomy};
\draw (144.32999999999998, -57.400000000000006) rectangle (146.67999999999998,-58.50000000000001);
\draw(146.77999999999997, -57.400000000000006) node[anchor=north west,align=left] {Exponential\\ sums};
\draw (146.77999999999997, -57.400000000000006) rectangle (149.12999999999997,-58.50000000000001);
\draw(143.32999999999998, -59.300000000000004) node[anchor=north west,align=left] {\large Arithmetic algebraic geometry (Diophantine geometry)};
\draw (143.32999999999998, -59.300000000000004) rectangle (163.07999999999998,-69.60000000000001);
\draw(144.32999999999998, -60.300000000000004) node[anchor=north west,align=left] {\(L\)-functions of\\ varieties over\\ global fields;\\ Birch-Swinnerton-Dyer\\ conjecture};
\draw (144.32999999999998, -60.300000000000004) rectangle (148.92999999999998,-62.900000000000006);
\draw(149.02999999999997, -60.300000000000004) node[anchor=north west,align=left] {Polylogarithms\\ and \\ relations \\ with \(K\)-theory};
\draw (149.02999999999997, -60.300000000000004) rectangle (153.12999999999997,-62.400000000000006);
\draw(153.23, -60.300000000000004) node[anchor=north west,align=left] {Drinfel’d \\ modules; \\ higher-dimensional\\ motives, etc.};
\draw (153.23, -60.300000000000004) rectangle (157.07999999999998,-62.400000000000006);
\draw(157.17999999999998, -60.300000000000004) node[anchor=north west,align=left] {Complex \\ multiplication and\\ moduli of \\ abelian varieties};
\draw (157.17999999999998, -60.300000000000004) rectangle (161.02999999999997,-62.400000000000006);
\draw(161.13, -60.300000000000004) node[anchor=north west,align=left] {Heights};
\draw (161.13, -60.300000000000004) rectangle (162.98,-61.400000000000006);
\draw(144.32999999999998, -63.00000000000001) node[anchor=north west,align=left] {Arithmetic \\ aspects of \\ modular and \\ Shimura varieties};
\draw (144.32999999999998, -63.00000000000001) rectangle (148.17999999999998,-65.10000000000001);
\draw(148.27999999999997, -63.00000000000001) node[anchor=north west,align=left] {Arithmetic \\ aspects of dessins\\ d’enfants,\\ Belyĭ theory};
\draw (148.27999999999997, -63.00000000000001) rectangle (151.87999999999997,-65.10000000000001);
\draw(151.98, -63.00000000000001) node[anchor=north west,align=left] {Elliptic \\ curves over\\ local fields};
\draw (151.98, -63.00000000000001) rectangle (155.32999999999998,-64.60000000000001);
\draw(155.42999999999998, -63.00000000000001) node[anchor=north west,align=left] {Abelian\\ varieties\\ of \\ dimension \(>~1\)};
\draw (155.42999999999998, -63.00000000000001) rectangle (158.77999999999997,-65.10000000000001);
\draw(158.88, -63.00000000000001) node[anchor=north west,align=left] {Curves of \\ arbitrary \\ genus or genus\\ \(\ne~1\) over \\ global fields};
\draw (158.88, -63.00000000000001) rectangle (161.98,-65.60000000000001);
\draw(144.32999999999998, -65.7) node[anchor=north west,align=left] {Curves \\ over finite\\ and \\ local fields};
\draw (144.32999999999998, -65.7) rectangle (147.17999999999998,-67.8);
\draw(147.27999999999997, -65.7) node[anchor=north west,align=left] {Varieties\\ over \\ finite and \\ local fields};
\draw (147.27999999999997, -65.7) rectangle (150.12999999999997,-67.8);
\draw(150.23, -65.7) node[anchor=north west,align=left] {Elliptic\\ curves\\ over \\ global fields};
\draw (150.23, -65.7) rectangle (152.82999999999998,-67.8);
\draw(152.92999999999998, -65.7) node[anchor=north west,align=left] {Elliptic\\ and \\ modular units};
\draw (152.92999999999998, -65.7) rectangle (155.52999999999997,-67.3);
\draw(155.63, -65.7) node[anchor=north west,align=left] {Varieties\\ over \\ global fields};
\draw (155.63, -65.7) rectangle (158.23,-67.3);
\draw(158.32999999999998, -65.7) node[anchor=north west,align=left] {Arithmetic\\ mirror\\ symmetry};
\draw (158.32999999999998, -65.7) rectangle (160.92999999999998,-67.3);
\draw(144.32999999999998, -67.9) node[anchor=north west,align=left] {Geometric\\ class \\ field theory};
\draw (144.32999999999998, -67.9) rectangle (146.92999999999998,-69.5);
\draw(163.17999999999998, -59.300000000000004) node[anchor=north west,align=left] {\large Algebraic number theory: local and \(p\)-adic fields};
\draw (163.17999999999998, -59.300000000000004) rectangle (182.67999999999998,-68.4);
\draw(164.17999999999998, -60.300000000000004) node[anchor=north west,align=left] {Langlands-Weil\\ conjectures,\\ nonabelian class\\ field theory};
\draw (164.17999999999998, -60.300000000000004) rectangle (170.27999999999997,-62.900000000000006);
\draw(170.37999999999997, -60.300000000000004) node[anchor=north west,align=left] {Non-Archimedean\\ dynamical systems};
\draw (170.37999999999997, -60.300000000000004) rectangle (176.22999999999996,-61.900000000000006);
\draw(176.32999999999998, -60.300000000000004) node[anchor=north west,align=left] {Prehomogeneous\\ vector spaces};
\draw (176.32999999999998, -60.300000000000004) rectangle (181.17999999999998,-61.900000000000006);
\draw(164.17999999999998, -63.00000000000001) node[anchor=north west,align=left] {Integral\\ representations};
\draw (164.17999999999998, -63.00000000000001) rectangle (168.52999999999997,-64.60000000000001);
\draw(168.62999999999997, -63.00000000000001) node[anchor=north west,align=left] {Other analytic\\ theory (analogues\\ of beta \\ and gamma \\ functions, \(p\)-adic \\ integration, etc.)};
\draw (168.62999999999997, -63.00000000000001) rectangle (172.97999999999996,-66.10000000000001);
\draw(164.17999999999998, -64.7) node[anchor=north west,align=left] {Polynomials};
\draw (164.17999999999998, -64.7) rectangle (167.02999999999997,-65.8);
\draw(173.07999999999998, -63.00000000000001) node[anchor=north west,align=left] {Zeta \\ functions and\\ \(L\)-functions};
\draw (173.07999999999998, -63.00000000000001) rectangle (177.17999999999998,-64.60000000000001);
\draw(173.07999999999998, -64.7) node[anchor=north west,align=left] {Galois \\ cohomology};
\draw (173.07999999999998, -64.7) rectangle (175.42999999999998,-65.8);
\draw(177.27999999999997, -63.00000000000001) node[anchor=north west,align=left] {Ramification\\ and \\ extension theory};
\draw (177.27999999999997, -63.00000000000001) rectangle (180.62999999999997,-64.60000000000001);
\draw(180.73, -63.00000000000001) node[anchor=north west,align=left] {Galois\\ theory};
\draw (180.73, -63.00000000000001) rectangle (182.57999999999998,-64.10000000000001);
\draw(164.17999999999998, -66.2) node[anchor=north west,align=left] {Class field\\ theory;\\ \(p\)-adic\\ formal groups};
\draw (164.17999999999998, -66.2) rectangle (167.52999999999997,-68.3);
\draw(167.62999999999997, -66.2) node[anchor=north west,align=left] {Algebras and\\ orders, \\ and their \\ zeta functions};
\draw (167.62999999999997, -66.2) rectangle (170.72999999999996,-68.3);
\draw(170.82999999999998, -66.2) node[anchor=north west,align=left] {Other \\ nonanalytic\\ theory};
\draw (170.82999999999998, -66.2) rectangle (173.42999999999998,-67.8);
\draw(173.52999999999997, -66.2) node[anchor=north west,align=left] {\(K\)-theory\\ of local\\ fields};
\draw (173.52999999999997, -66.2) rectangle (175.87999999999997,-67.8);
\draw(163.17999999999998, -68.5) node[anchor=north west,align=left] {\large History of\\ number theory};
\draw (163.17999999999998, -68.5) rectangle (167.18999999999997,-69.6);
\draw(182.78, -59.300000000000004) node[anchor=north west,align=left] {\large Diophantine equations};
\draw (182.78, -59.300000000000004) rectangle (192.68,-73.0);
\draw(183.78, -60.300000000000004) node[anchor=north west,align=left] {Thue-Mahlerequations};
\draw (183.78, -60.300000000000004) rectangle (189.13,-61.900000000000006);
\draw(189.23, -60.300000000000004) node[anchor=north west,align=left] {Counting \\ solutions \\ of Diophantine\\ equations};
\draw (189.23, -60.300000000000004) rectangle (192.57999999999998,-62.400000000000006);
\draw(183.78, -62.50000000000001) node[anchor=north west,align=left] {Diophantine\\ inequalities};
\draw (183.78, -62.50000000000001) rectangle (188.13,-64.10000000000001);
\draw(188.23, -62.50000000000001) node[anchor=north west,align=left] {Representation\\ problems};
\draw (188.23, -62.50000000000001) rectangle (192.32999999999998,-64.10000000000001);
\draw(183.78, -64.2) node[anchor=north west,align=left] {Quadratic\\ and bilinear\\ Diophantine\\ equations};
\draw (183.78, -64.2) rectangle (187.38,-66.3);
\draw(187.48, -64.2) node[anchor=north west,align=left] {Exponential\\ Diophantine\\ equations};
\draw (187.48, -64.2) rectangle (190.82999999999998,-65.8);
\draw(183.78, -66.4) node[anchor=north west,align=left] {Cubic and\\ quartic\\ Diophantine\\ equations};
\draw (183.78, -66.4) rectangle (186.88,-68.5);
\draw(186.98, -66.4) node[anchor=north west,align=left] {Higher \\ degree equations;\\ Fermat’s\\ equation};
\draw (186.98, -66.4) rectangle (190.07999999999998,-68.5);
\draw(190.18, -66.4) node[anchor=north west,align=left] {The \\ Frobenius\\ problem};
\draw (190.18, -66.4) rectangle (192.53,-68.0);
\draw(183.78, -68.60000000000001) node[anchor=north west,align=left] {Diophantine\\ equations\\ in \\ many variables};
\draw (183.78, -68.60000000000001) rectangle (186.88,-70.7);
\draw(186.98, -68.60000000000001) node[anchor=north west,align=left] {Multiplicative\\ and\\ norm form\\ equations};
\draw (186.98, -68.60000000000001) rectangle (189.82999999999998,-70.7);
\draw(189.93, -68.60000000000001) node[anchor=north west,align=left] {Linear \\ Diophantine\\ equations};
\draw (189.93, -68.60000000000001) rectangle (192.53,-70.2);
\draw(183.78, -70.80000000000001) node[anchor=north west,align=left] {Congruences\\ in many\\ variables};
\draw (183.78, -70.80000000000001) rectangle (186.63,-72.4);
\draw(186.73, -70.80000000000001) node[anchor=north west,align=left] {\(p\)-adic and\\ power \\ series fields};
\draw (186.73, -70.80000000000001) rectangle (189.57999999999998,-72.4);
\draw(189.68, -70.80000000000001) node[anchor=north west,align=left] {Rational\\ numbers \\ as sums of\\ fractions};
\draw (189.68, -70.80000000000001) rectangle (192.28,-72.9);
\draw(163.17999999999998, -69.7) node[anchor=north west,align=left] {\large Miscellaneous applications of number theory};
\draw (163.17999999999998, -69.7) rectangle (177.10999999999999,-72.9);
\draw(164.17999999999998, -70.7) node[anchor=north west,align=left] {Miscellaneous\\ applications of\\ number theory};
\draw (164.17999999999998, -70.7) rectangle (170.02999999999997,-72.8);
\draw(143.32999999999998, -73.1) node[anchor=north west,align=left] {\large Discontinuous groups and automorphic forms};
\draw (143.32999999999998, -73.1) rectangle (159.38,-97.69999999999999);
\draw(144.32999999999998, -74.1) node[anchor=north west,align=left] {Representation-theoretic\\ methods; automorphic\\ representations\\ over local\\ and global fields};
\draw (144.32999999999998, -74.1) rectangle (152.92999999999998,-77.19999999999999);
\draw(153.02999999999997, -74.1) node[anchor=north west,align=left] {Automorphic forms \\ on \(\mbox{GL}(2)\); Hilbert and\\ Hilbert-Siegel \\ modular groups and their\\ modular and \\ automorphic forms; Hilbert\\ modular surfaces};
\draw (153.02999999999997, -74.1) rectangle (158.12999999999997,-77.69999999999999);
\draw(144.32999999999998, -77.8) node[anchor=north west,align=left] {Dirichlet series\\ in several complex\\ variables \\ associated to \\ automorphic forms; \\ Weyl group multiple\\ Dirichlet series};
\draw (144.32999999999998, -77.8) rectangle (149.17999999999998,-81.39999999999999);
\draw(149.27999999999997, -77.8) node[anchor=north west,align=left] {Siegel modular\\ groups; Siegel\\ and Hilbert-Siegel\\ modular and\\ automorphic forms};
\draw (149.27999999999997, -77.8) rectangle (153.87999999999997,-80.39999999999999);
\draw(153.98, -77.8) node[anchor=north west,align=left] {Hecke-Petersson\\ operators,\\ differential\\ operators \\ (several variables)};
\draw (153.98, -77.8) rectangle (158.32999999999998,-80.39999999999999);
\draw(144.32999999999998, -81.5) node[anchor=north west,align=left] {Langlands \(L\)-functions;\\ one \\ variable Dirichlet\\ series and \\ functional equations};
\draw (144.32999999999998, -81.5) rectangle (148.67999999999998,-84.1);
\draw(148.77999999999997, -81.5) node[anchor=north west,align=left] {Special values \\ of automorphic \\ \(L\)-series, periods\\ of automorphic \\ forms, cohomology,\\ modular symbols};
\draw (148.77999999999997, -81.5) rectangle (153.12999999999997,-84.6);
\draw(153.23, -81.5) node[anchor=north west,align=left] {Hecke-Petersson\\ operators,\\ differential\\ operators\\ (one variable)};
\draw (153.23, -81.5) rectangle (157.32999999999998,-84.1);
\draw(157.42999999999998, -81.5) node[anchor=north west,align=left] {Jacobi\\ forms};
\draw (157.42999999999998, -81.5) rectangle (159.27999999999997,-82.6);
\draw(144.32999999999998, -84.69999999999999) node[anchor=north west,align=left] {Other groups \\ and their modular\\ and automorphic\\ forms \\ (several variables)};
\draw (144.32999999999998, -84.69999999999999) rectangle (148.42999999999998,-87.29999999999998);
\draw(148.52999999999997, -84.69999999999999) node[anchor=north west,align=left] {Automorphic\\ forms and\\ their relations\\ with \\ perfectoid spaces};
\draw (148.52999999999997, -84.69999999999999) rectangle (152.62999999999997,-87.29999999999998);
\draw(152.73, -84.69999999999999) node[anchor=north west,align=left] {Congruences\\ for modular\\ and \(p\)-adic\\ modular forms};
\draw (152.73, -84.69999999999999) rectangle (156.57999999999998,-86.79999999999998);
\draw(156.67999999999998, -84.69999999999999) node[anchor=north west,align=left] {\(p\)-adic \\ theory, \\ local fields};
\draw (156.67999999999998, -84.69999999999999) rectangle (159.27999999999997,-86.29999999999998);
\draw(144.32999999999998, -87.39999999999999) node[anchor=north west,align=left] {Spectral \\ theory; trace\\ formulas\\ (e.g., \\ that of Selberg)};
\draw (144.32999999999998, -87.39999999999999) rectangle (148.17999999999998,-89.99999999999999);
\draw(148.27999999999997, -87.39999999999999) node[anchor=north west,align=left] {Galois\\ representations};
\draw (148.27999999999997, -87.39999999999999) rectangle (152.12999999999997,-88.99999999999999);
\draw(152.23, -87.39999999999999) node[anchor=north west,align=left] {Structure of\\ modular groups\\ and \\ generalizations; \\ arithmetic groups};
\draw (152.23, -87.39999999999999) rectangle (155.82999999999998,-89.99999999999999);
\draw(155.92999999999998, -87.39999999999999) node[anchor=north west,align=left] {Modular and\\ automorphic\\ functions};
\draw (155.92999999999998, -87.39999999999999) rectangle (159.27999999999997,-88.99999999999999);
\draw(144.32999999999998, -90.1) node[anchor=north west,align=left] {Holomorphic\\ modular\\ forms of \\ integral weight};
\draw (144.32999999999998, -90.1) rectangle (147.92999999999998,-92.19999999999999);
\draw(148.02999999999997, -90.1) node[anchor=north west,align=left] {Relationship\\ to Lie algebras\\ and finite\\ simple groups};
\draw (148.02999999999997, -90.1) rectangle (151.62999999999997,-92.19999999999999);
\draw(151.73, -90.1) node[anchor=north west,align=left] {Forms of \\ half-integer\\ weight; \\ nonholomorphic\\ modular forms};
\draw (151.73, -90.1) rectangle (155.32999999999998,-92.69999999999999);
\draw(155.42999999999998, -90.1) node[anchor=north west,align=left] {Relations \\ with algebraic\\ geometry\\ and topology};
\draw (155.42999999999998, -90.1) rectangle (158.77999999999997,-92.19999999999999);
\draw(144.32999999999998, -92.8) node[anchor=north west,align=left] {Dedekind\\ eta \\ function, \\ Dedekind sums};
\draw (144.32999999999998, -92.8) rectangle (147.42999999999998,-94.89999999999999);
\draw(147.52999999999997, -92.8) node[anchor=north west,align=left] {Theta series;\\ Weil \\ representation;\\ theta \\ correspondences};
\draw (147.52999999999997, -92.8) rectangle (150.62999999999997,-95.39999999999999);
\draw(150.73, -92.8) node[anchor=north west,align=left] {Modular forms\\ associated\\ to Drinfel’d\\ modules};
\draw (150.73, -92.8) rectangle (153.82999999999998,-94.89999999999999);
\draw(153.92999999999998, -92.8) node[anchor=north west,align=left] {Automorphic\\ forms, \\ one variable};
\draw (153.92999999999998, -92.8) rectangle (156.77999999999997,-94.39999999999999);
\draw(144.32999999999998, -95.5) node[anchor=north west,align=left] {Fourier \\ coefficients\\ of automorphic\\ forms};
\draw (144.32999999999998, -95.5) rectangle (147.17999999999998,-97.6);
\draw(147.27999999999997, -95.5) node[anchor=north west,align=left] {Modular \\ correspondences,\\ etc.};
\draw (147.27999999999997, -95.5) rectangle (150.12999999999997,-97.1);
\draw(150.23, -95.5) node[anchor=north west,align=left] {Cohomology\\ of arithmetic\\ groups};
\draw (150.23, -95.5) rectangle (153.07999999999998,-97.1);
\draw(159.48, -73.1) node[anchor=north west,align=left] {\large Zeta and \(L\)-functions: analytic theory};
\draw (159.48, -73.1) rectangle (175.07999999999998,-82.69999999999999);
\draw(160.48, -74.1) node[anchor=north west,align=left] {Selberg zeta functions\\ and regularized \\ determinants; applications\\ to spectral \\ theory, Dirichlet series,\\ Eisenstein series,\\ etc. (explicit formulas)};
\draw (160.48, -74.1) rectangle (167.32999999999998,-77.69999999999999);
\draw(167.42999999999998, -74.1) node[anchor=north west,align=left] {Real zeros\\ of \(L(s,~\chi)\);\\ results\\ on \(L(1,~\chi)\)};
\draw (167.42999999999998, -74.1) rectangle (172.02999999999997,-76.19999999999999);
\draw(167.42999999999998, -76.3) node[anchor=north west,align=left] {\(\zeta~(s)\) \\ and \(L(s,~\chi)\)};
\draw (167.42999999999998, -76.3) rectangle (171.52999999999997,-77.39999999999999);
\draw(172.13, -74.1) node[anchor=north west,align=left] {Other \\ Dirichlet series\\ and zeta\\ functions};
\draw (172.13, -74.1) rectangle (174.98,-76.19999999999999);
\draw(160.48, -77.8) node[anchor=north west,align=left] {Tauberiantheorems};
\draw (160.48, -77.8) rectangle (165.07999999999998,-79.39999999999999);
\draw(165.17999999999998, -77.8) node[anchor=north west,align=left] {Multiple \\ Dirichlet series\\ and zeta \\ functions and\\ multizeta values};
\draw (165.17999999999998, -77.8) rectangle (169.52999999999997,-80.39999999999999);
\draw(169.63, -77.8) node[anchor=north west,align=left] {Nonreal zeros\\ of \(\zeta~(s)\) and\\ \(L(s,~\chi)\); \\ Riemann and other\\ hypotheses};
\draw (169.63, -77.8) rectangle (172.98,-80.39999999999999);
\draw(160.48, -80.5) node[anchor=north west,align=left] {Zeta and \\ \(L\)-functions\\ in \\ characteristic \(p\)};
\draw (160.48, -80.5) rectangle (163.82999999999998,-82.6);
\draw(163.92999999999998, -80.5) node[anchor=north west,align=left] {Hurwitz and\\ Lerch \\ zeta functions};
\draw (163.92999999999998, -80.5) rectangle (167.02999999999997,-82.1);
\draw(167.13, -80.5) node[anchor=north west,align=left] {Relations\\ with \\ random matrices};
\draw (167.13, -80.5) rectangle (170.23,-82.1);
\draw(170.32999999999998, -80.5) node[anchor=north west,align=left] {Relations\\ with \\ noncommutative\\ geometry};
\draw (170.32999999999998, -80.5) rectangle (173.17999999999998,-82.6);
\draw(159.48, -82.80000000000001) node[anchor=north west,align=left] {\large Exponential sums and character sums};
\draw (159.48, -82.80000000000001) rectangle (172.28,-88.20000000000002);
\draw(160.48, -83.80000000000001) node[anchor=north west,align=left] {Trigonometric\\ and \\ exponential\\ sums, general};
\draw (160.48, -83.80000000000001) rectangle (163.82999999999998,-85.9);
\draw(163.92999999999998, -83.80000000000001) node[anchor=north west,align=left] {Gauss and\\ Kloosterman\\ sums; \\ generalizations};
\draw (163.92999999999998, -83.80000000000001) rectangle (167.27999999999997,-85.9);
\draw(167.38, -83.80000000000001) node[anchor=north west,align=left] {Jacobsthal\\ and Brewer\\ sums; other\\ complete \\ character sums};
\draw (167.38, -83.80000000000001) rectangle (170.73,-86.4);
\draw(170.82999999999998, -83.80000000000001) node[anchor=north west,align=left] {Weyl\\ sums};
\draw (170.82999999999998, -83.80000000000001) rectangle (172.17999999999998,-84.9);
\draw(160.48, -86.50000000000001) node[anchor=north west,align=left] {Estimates\\ on \\ exponential sums};
\draw (160.48, -86.50000000000001) rectangle (163.32999999999998,-88.10000000000001);
\draw(163.42999999999998, -86.50000000000001) node[anchor=north west,align=left] {Sums over\\ arbitrary\\ intervals};
\draw (163.42999999999998, -86.50000000000001) rectangle (166.27999999999997,-88.10000000000001);
\draw(166.38, -86.50000000000001) node[anchor=north west,align=left] {Estimates\\ on \\ character sums};
\draw (166.38, -86.50000000000001) rectangle (169.23,-88.10000000000001);
\draw(169.32999999999998, -86.50000000000001) node[anchor=north west,align=left] {Sums \\ over primes};
\draw (169.32999999999998, -86.50000000000001) rectangle (171.67999999999998,-87.60000000000001);
\draw(159.48, -88.30000000000001) node[anchor=north west,align=left] {\large Additive number theory; partitions};
\draw (159.48, -88.30000000000001) rectangle (172.07999999999998,-96.9);
\draw(160.48, -89.30000000000001) node[anchor=north west,align=left] {Goldbach-type\\ theorems;\\ other additive\\ questions\\ involving primes};
\draw (160.48, -89.30000000000001) rectangle (164.82999999999998,-91.9);
\draw(164.92999999999998, -89.30000000000001) node[anchor=north west,align=left] {Partitions; \\ congruences and\\ congruential\\ restrictions};
\draw (164.92999999999998, -89.30000000000001) rectangle (168.52999999999997,-91.4);
\draw(168.63, -89.30000000000001) node[anchor=north west,align=left] {Applications\\ of the\\ Hardy-Littlewood\\ method};
\draw (168.63, -89.30000000000001) rectangle (171.98,-91.4);
\draw(160.48, -92.00000000000001) node[anchor=north west,align=left] {Inverse \\ problems of \\ additive number\\ theory, \\ including sumsets};
\draw (160.48, -92.00000000000001) rectangle (163.82999999999998,-94.60000000000001);
\draw(163.92999999999998, -92.00000000000001) node[anchor=north west,align=left] {Partition \\ identities;\\ identities\\ of \\ Rogers-Ramanujan type};
\draw (163.92999999999998, -92.00000000000001) rectangle (167.27999999999997,-94.60000000000001);
\draw(167.38, -92.00000000000001) node[anchor=north west,align=left] {Waring’s\\ problem \\ and variants};
\draw (167.38, -92.00000000000001) rectangle (170.23,-93.60000000000001);
\draw(160.48, -94.70000000000002) node[anchor=north west,align=left] {Lattice\\ points \\ in specified\\ regions};
\draw (160.48, -94.70000000000002) rectangle (163.32999999999998,-96.80000000000001);
\draw(163.42999999999998, -94.70000000000002) node[anchor=north west,align=left] {Elementary\\ theory of\\ partitions};
\draw (163.42999999999998, -94.70000000000002) rectangle (166.27999999999997,-96.30000000000001);
\draw(166.38, -94.70000000000002) node[anchor=north west,align=left] {Analytic\\ theory of\\ partitions};
\draw (166.38, -94.70000000000002) rectangle (169.23,-96.30000000000001);
\draw(175.17999999999998, -73.1) node[anchor=north west,align=left] {\large Algebraic number theory: global fields};
\draw (175.17999999999998, -73.1) rectangle (189.77999999999997,-93.19999999999999);
\draw(176.17999999999998, -74.1) node[anchor=north west,align=left] {Langlands-Weil\\ conjectures,\\ nonabelian class\\ field theory};
\draw (176.17999999999998, -74.1) rectangle (182.27999999999997,-76.69999999999999);
\draw(182.37999999999997, -74.1) node[anchor=north west,align=left] {PV-numbers and\\ generalizations;\\ other special\\ algebraic \\ numbers; Mahler measure};
\draw (182.37999999999997, -74.1) rectangle (187.22999999999996,-76.69999999999999);
\draw(187.32999999999998, -74.1) node[anchor=north west,align=left] {Units \\ and \\ factorization};
\draw (187.32999999999998, -74.1) rectangle (189.67999999999998,-75.69999999999999);
\draw(176.17999999999998, -76.8) node[anchor=north west,align=left] {Integral \\ representations related\\ to algebraic \\ numbers; Galois \\ module structure \\ of rings of integers};
\draw (176.17999999999998, -76.8) rectangle (181.02999999999997,-79.89999999999999);
\draw(181.12999999999997, -76.8) node[anchor=north west,align=left] {Polynomials\\ (irreducibility,\\ etc.)};
\draw (181.12999999999997, -76.8) rectangle (185.72999999999996,-78.89999999999999);
\draw(185.82999999999998, -76.8) node[anchor=north west,align=left] {Cyclotomic\\ extensions};
\draw (185.82999999999998, -76.8) rectangle (189.67999999999998,-78.39999999999999);
\draw(176.17999999999998, -80.0) node[anchor=north west,align=left] {Other algebras\\ and \\ orders, and \\ their zeta and\\ \(L\)-functions};
\draw (176.17999999999998, -80.0) rectangle (180.52999999999997,-82.6);
\draw(180.62999999999997, -80.0) node[anchor=north west,align=left] {Quadratic\\ extensions};
\draw (180.62999999999997, -80.0) rectangle (184.22999999999996,-81.6);
\draw(184.32999999999998, -80.0) node[anchor=north west,align=left] {Zeta functions\\ and \\ \(L\)-functions of\\ number fields};
\draw (184.32999999999998, -80.0) rectangle (187.92999999999998,-82.1);
\draw(176.17999999999998, -82.69999999999999) node[anchor=north west,align=left] {Quaternion and\\ other division\\ algebras:\\ arithmetic,\\ zeta functions};
\draw (176.17999999999998, -82.69999999999999) rectangle (179.77999999999997,-85.29999999999998);
\draw(179.87999999999997, -82.69999999999999) node[anchor=north west,align=left] {Cyclotomic \\ function fields\\ (class groups,\\ Bernoulli\\ objects, etc.)};
\draw (179.87999999999997, -82.69999999999999) rectangle (183.47999999999996,-85.29999999999998);
\draw(183.57999999999998, -82.69999999999999) node[anchor=north west,align=left] {Algebraic \\ numbers; rings\\ of algebraic\\ integers};
\draw (183.57999999999998, -82.69999999999999) rectangle (186.92999999999998,-84.79999999999998);
\draw(187.02999999999997, -82.69999999999999) node[anchor=north west,align=left] {Cubic and\\ quartic\\ extensions};
\draw (187.02999999999997, -82.69999999999999) rectangle (189.62999999999997,-84.29999999999998);
\draw(176.17999999999998, -85.39999999999999) node[anchor=north west,align=left] {Arithmetic\\ theory of \\ algebraic \\ function fields};
\draw (176.17999999999998, -85.39999999999999) rectangle (179.52999999999997,-87.49999999999999);
\draw(179.62999999999997, -85.39999999999999) node[anchor=north west,align=left] {Zeta functions\\ and \(L\)-functions\\ of \\ function fields};
\draw (179.62999999999997, -85.39999999999999) rectangle (182.97999999999996,-87.49999999999999);
\draw(183.07999999999998, -85.39999999999999) node[anchor=north west,align=left] {Class \\ groups and \\ Picard \\ groups of orders};
\draw (183.07999999999998, -85.39999999999999) rectangle (186.42999999999998,-87.49999999999999);
\draw(186.52999999999997, -85.39999999999999) node[anchor=north west,align=left] {Other \\ abelian and\\ metabelian\\ extensions};
\draw (186.52999999999997, -85.39999999999999) rectangle (189.62999999999997,-87.49999999999999);
\draw(176.17999999999998, -87.6) node[anchor=north west,align=left] {Class \\ numbers, class\\ groups, \\ discriminants};
\draw (176.17999999999998, -87.6) rectangle (179.27999999999997,-89.69999999999999);
\draw(179.37999999999997, -87.6) node[anchor=north west,align=left] {Distribution\\ of \\ prime ideals};
\draw (179.37999999999997, -87.6) rectangle (181.97999999999996,-89.19999999999999);
\draw(182.07999999999998, -87.6) node[anchor=north west,align=left] {\(K\)-theory\\ of \\ global fields};
\draw (182.07999999999998, -87.6) rectangle (184.67999999999998,-89.19999999999999);
\draw(184.77999999999997, -87.6) node[anchor=north west,align=left] {Galois \\ cohomology};
\draw (184.77999999999997, -87.6) rectangle (187.12999999999997,-88.69999999999999);
\draw(187.23, -87.6) node[anchor=north west,align=left] {Density\\ theorems};
\draw (187.23, -87.6) rectangle (189.57999999999998,-88.69999999999999);
\draw(176.17999999999998, -89.8) node[anchor=north west,align=left] {Other\\ analytic\\ theory};
\draw (176.17999999999998, -89.8) rectangle (178.52999999999997,-91.39999999999999);
\draw(178.62999999999997, -89.8) node[anchor=north west,align=left] {Iwasawa\\ theory};
\draw (178.62999999999997, -89.8) rectangle (180.72999999999996,-90.89999999999999);
\draw(180.82999999999998, -89.8) node[anchor=north west,align=left] {Class\\ field\\ theory};
\draw (180.82999999999998, -89.8) rectangle (182.92999999999998,-91.39999999999999);
\draw(183.02999999999997, -89.8) node[anchor=north west,align=left] {Adèle \\ rings and\\ groups};
\draw (183.02999999999997, -89.8) rectangle (185.12999999999997,-91.39999999999999);
\draw(185.23, -89.8) node[anchor=north west,align=left] {Other\\ number\\ fields};
\draw (185.23, -89.8) rectangle (187.07999999999998,-91.39999999999999);
\draw(187.17999999999998, -89.8) node[anchor=north west,align=left] {Galois\\ theory};
\draw (187.17999999999998, -89.8) rectangle (189.02999999999997,-90.89999999999999);
\draw(176.17999999999998, -91.5) node[anchor=north west,align=left] {Totally\\ real\\ fields};
\draw (176.17999999999998, -91.5) rectangle (178.02999999999997,-93.1);
\draw(143.32999999999998, -97.80000000000001) node[anchor=north west,align=left] {\large Forms and linear algebraic groups};
\draw (143.32999999999998, -97.80000000000001) rectangle (156.13,-113.00000000000001);
\draw(144.32999999999998, -98.80000000000001) node[anchor=north west,align=left] {General ternary\\ and quaternary\\ quadratic forms;\\ forms of more\\ than two variables};
\draw (144.32999999999998, -98.80000000000001) rectangle (149.42999999999998,-101.4);
\draw(149.52999999999997, -98.80000000000001) node[anchor=north west,align=left] {Analytic theory\\ (Epstein zeta\\ functions; \\ relations with\\ automorphic \\ forms and functions)};
\draw (149.52999999999997, -98.80000000000001) rectangle (154.12999999999997,-101.9);
\draw(144.32999999999998, -102.00000000000001) node[anchor=north west,align=left] {Sums of squares\\ and representations\\ by other\\ particular\\ quadratic forms};
\draw (144.32999999999998, -102.00000000000001) rectangle (148.17999999999998,-104.60000000000001);
\draw(148.27999999999997, -102.00000000000001) node[anchor=north west,align=left] {Quadratic\\ forms over\\ global \\ rings and fields};
\draw (148.27999999999997, -102.00000000000001) rectangle (151.87999999999997,-104.10000000000001);
\draw(151.98, -102.00000000000001) node[anchor=north west,align=left] {Galois \\ cohomology of\\ linear \\ algebraic groups};
\draw (151.98, -102.00000000000001) rectangle (155.57999999999998,-104.10000000000001);
\draw(144.32999999999998, -104.70000000000002) node[anchor=north west,align=left] {Algebraic \\ theory of \\ quadratic \\ forms; Witt \\ groups and rings};
\draw (144.32999999999998, -104.70000000000002) rectangle (147.92999999999998,-107.30000000000001);
\draw(148.02999999999997, -104.70000000000002) node[anchor=north west,align=left] {Quadratic\\ forms over\\ local rings\\ and fields};
\draw (148.02999999999997, -104.70000000000002) rectangle (151.37999999999997,-106.80000000000001);
\draw(151.48, -104.70000000000002) node[anchor=north west,align=left] {Class numbers\\ of quadratic\\ and \\ Hermitian forms};
\draw (151.48, -104.70000000000002) rectangle (154.82999999999998,-106.80000000000001);
\draw(144.32999999999998, -107.4) node[anchor=north west,align=left] {General \\ binary \\ quadratic forms};
\draw (144.32999999999998, -107.4) rectangle (147.42999999999998,-109.0);
\draw(147.52999999999997, -107.4) node[anchor=north west,align=left] {\(K\)-theory\\ of quadratic\\ and \\ Hermitian forms};
\draw (147.52999999999997, -107.4) rectangle (150.62999999999997,-109.5);
\draw(150.73, -107.4) node[anchor=north west,align=left] {Quadratic\\ forms\\ over \\ general fields};
\draw (150.73, -107.4) rectangle (153.57999999999998,-109.5);
\draw(153.67999999999998, -107.4) node[anchor=north west,align=left] {Classical\\ groups};
\draw (153.67999999999998, -107.4) rectangle (156.02999999999997,-108.5);
\draw(144.32999999999998, -109.60000000000001) node[anchor=north west,align=left] {Forms of \\ degree higher\\ than two};
\draw (144.32999999999998, -109.60000000000001) rectangle (147.17999999999998,-111.2);
\draw(147.27999999999997, -109.60000000000001) node[anchor=north west,align=left] {Quadratic\\ spaces;\\ Clifford\\ algebras};
\draw (147.27999999999997, -109.60000000000001) rectangle (150.12999999999997,-111.7);
\draw(150.23, -109.60000000000001) node[anchor=north west,align=left] {Bilinear\\ and Hermitian\\ forms};
\draw (150.23, -109.60000000000001) rectangle (152.82999999999998,-111.2);
\draw(152.92999999999998, -109.60000000000001) node[anchor=north west,align=left] {Forms \\ over real\\ fields};
\draw (152.92999999999998, -109.60000000000001) rectangle (155.02999999999997,-111.2);
\draw(144.32999999999998, -111.80000000000001) node[anchor=north west,align=left] {\(p\)-adic\\ theory};
\draw (144.32999999999998, -111.80000000000001) rectangle (146.42999999999998,-112.9);
\draw(156.23, -97.80000000000001) node[anchor=north west,align=left] {\large Connections of number theory and logic};
\draw (156.23, -97.80000000000001) rectangle (168.60999999999999,-103.20000000000002);
\draw(157.23, -98.80000000000001) node[anchor=north west,align=left] {Ultraproducts\\ (number-theoretic\\ aspects)};
\draw (157.23, -98.80000000000001) rectangle (162.82999999999998,-100.9);
\draw(162.92999999999998, -98.80000000000001) node[anchor=north west,align=left] {Decidability\\ (number-theoretic\\ aspects)};
\draw (162.92999999999998, -98.80000000000001) rectangle (168.27999999999997,-100.9);
\draw(157.23, -101.00000000000001) node[anchor=north west,align=left] {Nonstandard\\ arithmetic \\ (number-theoretic\\ aspects)};
\draw (157.23, -101.00000000000001) rectangle (160.57999999999998,-103.10000000000001);
\draw(160.67999999999998, -101.00000000000001) node[anchor=north west,align=left] {Model \\ theory \\ (number-theoretic\\ aspects)};
\draw (160.67999999999998, -101.00000000000001) rectangle (163.77999999999997,-103.10000000000001);
\draw(156.23, -103.30000000000001) node[anchor=north west,align=left] {\large Polynomials and matrices};
\draw (156.23, -103.30000000000001) rectangle (164.26999999999998,-106.50000000000001);
\draw(157.23, -104.30000000000001) node[anchor=north west,align=left] {Matrices,\\ determinants\\ in \\ number theory};
\draw (157.23, -104.30000000000001) rectangle (160.32999999999998,-106.4);
\draw(160.42999999999998, -104.30000000000001) node[anchor=north west,align=left] {Polynomials\\ in \\ number theory};
\draw (160.42999999999998, -104.30000000000001) rectangle (163.02999999999997,-105.9);
\draw(168.70999999999998, -97.80000000000001) node[anchor=north west,align=left] {\large Multiplicative number theory};
\draw (168.70999999999998, -97.80000000000001) rectangle (179.30999999999997,-115.00000000000001);
\draw(169.70999999999998, -98.80000000000001) node[anchor=north west,align=left] {Applications \\ of automorphic\\ functions and\\ forms to \\ multiplicative problems};
\draw (169.70999999999998, -98.80000000000001) rectangle (174.30999999999997,-101.4);
\draw(174.40999999999997, -98.80000000000001) node[anchor=north west,align=left] {Primes represented\\ by \\ polynomials; other \\ multiplicative\\ structures of\\ polynomial values};
\draw (174.40999999999997, -98.80000000000001) rectangle (178.75999999999996,-101.9);
\draw(169.70999999999998, -102.00000000000001) node[anchor=north west,align=left] {Asymptotic results\\ on counting \\ functions for \\ algebraic and \\ topological structures};
\draw (169.70999999999998, -102.00000000000001) rectangle (174.05999999999997,-104.60000000000001);
\draw(174.15999999999997, -102.00000000000001) node[anchor=north west,align=left] {Other results \\ on the distribution\\ of values \\ or the characterization\\ of \\ arithmetic functions};
\draw (174.15999999999997, -102.00000000000001) rectangle (178.25999999999996,-105.10000000000001);
\draw(169.70999999999998, -105.20000000000002) node[anchor=north west,align=left] {Distribution\\ of primes};
\draw (169.70999999999998, -105.20000000000002) rectangle (173.55999999999997,-106.80000000000001);
\draw(173.65999999999997, -105.20000000000002) node[anchor=north west,align=left] {Distribution \\ functions \\ associated with \\ additive and \\ positive multiplicative\\ functions};
\draw (173.65999999999997, -105.20000000000002) rectangle (177.50999999999996,-108.30000000000001);
\draw(169.70999999999998, -106.9) node[anchor=north west,align=left] {Turán\\ theory};
\draw (169.70999999999998, -106.9) rectangle (171.55999999999997,-108.0);
\draw(177.60999999999999, -105.20000000000002) node[anchor=north west,align=left] {Sieves};
\draw (177.60999999999999, -105.20000000000002) rectangle (179.20999999999998,-105.80000000000001);
\draw(169.70999999999998, -108.4) node[anchor=north west,align=left] {Distribution\\ of integers\\ in special\\ residue classes};
\draw (169.70999999999998, -108.4) rectangle (173.55999999999997,-110.5);
\draw(173.65999999999997, -108.4) node[anchor=north west,align=left] {Distribution\\ of integers \\ with specified\\ multiplicative\\ constraints};
\draw (173.65999999999997, -108.4) rectangle (177.25999999999996,-111.0);
\draw(169.70999999999998, -111.10000000000001) node[anchor=north west,align=left] {Asymptotic\\ results \\ on arithmetic\\ functions};
\draw (169.70999999999998, -111.10000000000001) rectangle (172.80999999999997,-113.2);
\draw(172.90999999999997, -111.10000000000001) node[anchor=north west,align=left] {Applications\\ of \\ sieve methods};
\draw (172.90999999999997, -111.10000000000001) rectangle (175.75999999999996,-112.7);
\draw(175.85999999999999, -111.10000000000001) node[anchor=north west,align=left] {Rate of \\ growth of\\ arithmetic\\ functions};
\draw (175.85999999999999, -111.10000000000001) rectangle (178.70999999999998,-113.2);
\draw(169.70999999999998, -113.30000000000001) node[anchor=north west,align=left] {Generalized\\ primes \\ and integers};
\draw (169.70999999999998, -113.30000000000001) rectangle (172.55999999999997,-114.9);
\draw(172.65999999999997, -113.30000000000001) node[anchor=north west,align=left] {Primes in\\ congruence\\ classes};
\draw (172.65999999999997, -113.30000000000001) rectangle (175.25999999999996,-114.9);
\draw(179.40999999999997, -97.80000000000001) node[anchor=north west,align=left] {\large Computational number theory};
\draw (179.40999999999997, -97.80000000000001) rectangle (189.80999999999997,-108.10000000000001);
\draw(180.40999999999997, -98.80000000000001) node[anchor=north west,align=left] {Number-theoretic\\ algorithms;\\ complexity};
\draw (180.40999999999997, -98.80000000000001) rectangle (185.75999999999996,-100.9);
\draw(185.85999999999996, -98.80000000000001) node[anchor=north west,align=left] {Analytic\\ computations};
\draw (185.85999999999996, -98.80000000000001) rectangle (189.70999999999995,-100.4);
\draw(180.40999999999997, -101.00000000000001) node[anchor=north west,align=left] {Continued \\ fraction \\ calculations \\ (number-theoretic\\ aspects)};
\draw (180.40999999999997, -101.00000000000001) rectangle (184.00999999999996,-103.60000000000001);
\draw(184.10999999999996, -101.00000000000001) node[anchor=north west,align=left] {Factorization};
\draw (184.10999999999996, -101.00000000000001) rectangle (187.45999999999995,-102.10000000000001);
\draw(184.10999999999996, -102.20000000000002) node[anchor=north west,align=left] {Primality};
\draw (184.10999999999996, -102.20000000000002) rectangle (186.45999999999995,-103.30000000000001);
\draw(180.40999999999997, -103.70000000000002) node[anchor=north west,align=left] {Evaluation\\ of \\ number-theoretic\\ constants};
\draw (180.40999999999997, -103.70000000000002) rectangle (183.75999999999996,-105.80000000000001);
\draw(183.85999999999996, -103.70000000000002) node[anchor=north west,align=left] {Algebraic\\ number\\ theory\\ computations};
\draw (183.85999999999996, -103.70000000000002) rectangle (186.95999999999995,-105.80000000000001);
\draw(180.40999999999997, -105.9) node[anchor=north west,align=left] {Computer \\ solution of\\ Diophantine\\ equations};
\draw (180.40999999999997, -105.9) rectangle (183.50999999999996,-108.0);
\draw(183.60999999999996, -105.9) node[anchor=north west,align=left] {Calculation\\ of integer\\ sequences};
\draw (183.60999999999996, -105.9) rectangle (186.70999999999995,-107.5);
\draw(186.80999999999997, -105.9) node[anchor=north west,align=left] {Values of\\ arithmetic\\ functions;\\ tables};
\draw (186.80999999999997, -105.9) rectangle (189.65999999999997,-108.0);
\draw(143.32999999999998, -115.10000000000001) node[anchor=north west,align=left] {\large Sequences and sets};
\draw (143.32999999999998, -115.10000000000001) rectangle (153.48,-130.5);
\draw(144.32999999999998, -116.10000000000001) node[anchor=north west,align=left] {Farey \\ sequences;\\ the \\ sequences \(1^k,~2^k,~\dots\)};
\draw (144.32999999999998, -116.10000000000001) rectangle (149.67999999999998,-118.2);
\draw(149.77999999999997, -116.10000000000001) node[anchor=north west,align=left] {Arithmetic \\ combinatorics;\\ higher \\ degree uniformity};
\draw (149.77999999999997, -116.10000000000001) rectangle (153.37999999999997,-118.2);
\draw(144.32999999999998, -118.30000000000001) node[anchor=north west,align=left] {Binomial \\ coefficients;\\ factorials;\\ \(q\)-identities};
\draw (144.32999999999998, -118.30000000000001) rectangle (148.92999999999998,-120.4);
\draw(149.02999999999997, -118.30000000000001) node[anchor=north west,align=left] {Representation\\ functions};
\draw (149.02999999999997, -118.30000000000001) rectangle (153.37999999999997,-119.9);
\draw(144.32999999999998, -120.50000000000001) node[anchor=north west,align=left] {Arithmetic\\ progressions};
\draw (144.32999999999998, -120.50000000000001) rectangle (148.42999999999998,-122.10000000000001);
\draw(148.52999999999997, -120.50000000000001) node[anchor=north west,align=left] {Fibonacci and\\ Lucas numbers\\ and \\ polynomials and\\ generalizations};
\draw (148.52999999999997, -120.50000000000001) rectangle (152.37999999999997,-123.10000000000001);
\draw(144.32999999999998, -123.20000000000002) node[anchor=north west,align=left] {Additive \\ bases, \\ including sumsets};
\draw (144.32999999999998, -123.20000000000002) rectangle (147.67999999999998,-124.80000000000001);
\draw(147.77999999999997, -123.20000000000002) node[anchor=north west,align=left] {Bernoulli\\ and Euler\\ numbers and\\ polynomials};
\draw (147.77999999999997, -123.20000000000002) rectangle (151.12999999999997,-125.30000000000001);
\draw(144.32999999999998, -125.4) node[anchor=north west,align=left] {Other \\ combinatorial\\ number theory};
\draw (144.32999999999998, -125.4) rectangle (147.67999999999998,-127.0);
\draw(147.77999999999997, -125.4) node[anchor=north west,align=left] {Special \\ sequences and\\ polynomials};
\draw (147.77999999999997, -125.4) rectangle (151.12999999999997,-127.0);
\draw(144.32999999999998, -127.10000000000001) node[anchor=north west,align=left] {Automata\\ sequences};
\draw (144.32999999999998, -127.10000000000001) rectangle (147.67999999999998,-128.70000000000002);
\draw(147.77999999999997, -127.10000000000001) node[anchor=north west,align=left] {Recurrences};
\draw (147.77999999999997, -127.10000000000001) rectangle (150.62999999999997,-128.20000000000002);
\draw(150.73, -127.10000000000001) node[anchor=north west,align=left] {Density,\\ gaps,\\ topology};
\draw (150.73, -127.10000000000001) rectangle (153.07999999999998,-128.70000000000002);
\draw(144.32999999999998, -128.8) node[anchor=north west,align=left] {Sequences\\ (mod \(m\))};
\draw (144.32999999999998, -128.8) rectangle (147.17999999999998,-129.9);
\draw(147.27999999999997, -128.8) node[anchor=north west,align=left] {Bell and\\ Stirling\\ numbers};
\draw (147.27999999999997, -128.8) rectangle (149.62999999999997,-130.4);
\draw(153.57999999999998, -115.10000000000001) node[anchor=north west,align=left] {\large Elementary number theory};
\draw (153.57999999999998, -115.10000000000001) rectangle (162.98,-125.9);
\draw(154.57999999999998, -116.10000000000001) node[anchor=north west,align=left] {Factorization;\\ primality};
\draw (154.57999999999998, -116.10000000000001) rectangle (158.92999999999998,-117.7);
\draw(159.02999999999997, -116.10000000000001) node[anchor=north west,align=left] {Multiplicative\\ structure; \\ Euclidean algorithm;\\ greatest\\ common divisors};
\draw (159.02999999999997, -116.10000000000001) rectangle (162.87999999999997,-118.7);
\draw(154.57999999999998, -117.80000000000001) node[anchor=north west,align=left] {Primes};
\draw (154.57999999999998, -117.80000000000001) rectangle (156.17999999999998,-118.4);
\draw(154.57999999999998, -118.80000000000001) node[anchor=north west,align=left] {Arithmetic\\ functions;\\ related \\ numbers; inversion\\ formulas};
\draw (154.57999999999998, -118.80000000000001) rectangle (157.92999999999998,-121.4);
\draw(158.02999999999997, -118.80000000000001) node[anchor=north west,align=left] {Continued\\ fractions};
\draw (158.02999999999997, -118.80000000000001) rectangle (161.37999999999997,-120.4);
\draw(154.57999999999998, -121.50000000000001) node[anchor=north west,align=left] {Congruences;\\ primitive\\ roots; \\ residue systems};
\draw (154.57999999999998, -121.50000000000001) rectangle (157.67999999999998,-123.60000000000001);
\draw(157.77999999999997, -121.50000000000001) node[anchor=north west,align=left] {Power \\ residues,\\ reciprocity};
\draw (157.77999999999997, -121.50000000000001) rectangle (160.62999999999997,-123.10000000000001);
\draw(154.57999999999998, -123.70000000000002) node[anchor=north west,align=left] {Radix \\ representation;\\ digital\\ problems};
\draw (154.57999999999998, -123.70000000000002) rectangle (157.42999999999998,-125.80000000000001);
\draw(157.52999999999997, -123.70000000000002) node[anchor=north west,align=left] {Other \\ number \\ representations};
\draw (157.52999999999997, -123.70000000000002) rectangle (160.12999999999997,-125.30000000000001);
\draw(163.07999999999998, -115.10000000000001) node[anchor=north west,align=left] {\large Geometry of numbers};
\draw (163.07999999999998, -115.10000000000001) rectangle (170.98,-126.60000000000001);
\draw(164.07999999999998, -116.10000000000001) node[anchor=north west,align=left] {Lattice \\ packing and\\ covering \\ (number-theoretic\\ aspects)};
\draw (164.07999999999998, -116.10000000000001) rectangle (167.92999999999998,-118.7);
\draw(168.02999999999997, -116.10000000000001) node[anchor=north west,align=left] {Automorphism\\ groups\\ of lattices};
\draw (168.02999999999997, -116.10000000000001) rectangle (170.87999999999997,-117.7);
\draw(164.07999999999998, -118.80000000000001) node[anchor=north west,align=left] {Quadratic\\ forms \\ (reduction \\ theory, extreme\\ forms, etc.)};
\draw (164.07999999999998, -118.80000000000001) rectangle (167.92999999999998,-121.4);
\draw(168.02999999999997, -118.80000000000001) node[anchor=north west,align=left] {Products\\ of \\ linear forms};
\draw (168.02999999999997, -118.80000000000001) rectangle (170.62999999999997,-120.4);
\draw(164.07999999999998, -121.50000000000001) node[anchor=north west,align=left] {Lattices and\\ convex bodies\\ (number-theoretic\\ aspects)};
\draw (164.07999999999998, -121.50000000000001) rectangle (167.67999999999998,-123.60000000000001);
\draw(167.77999999999997, -121.50000000000001) node[anchor=north west,align=left] {Mean value\\ and transfer\\ theorems};
\draw (167.77999999999997, -121.50000000000001) rectangle (170.87999999999997,-123.10000000000001);
\draw(164.07999999999998, -123.70000000000002) node[anchor=north west,align=left] {Relations\\ with \\ coding theory};
\draw (164.07999999999998, -123.70000000000002) rectangle (166.67999999999998,-125.30000000000001);
\draw(166.77999999999997, -123.70000000000002) node[anchor=north west,align=left] {Nonconvex\\ bodies};
\draw (166.77999999999997, -123.70000000000002) rectangle (169.12999999999997,-124.80000000000001);
\draw(164.07999999999998, -125.4) node[anchor=north west,align=left] {Minima\\ of forms};
\draw (164.07999999999998, -125.4) rectangle (166.17999999999998,-126.5);
\draw(195.07999999999998, -1) node[anchor=north west,align=left] {\LARGE Group theory and generalizations};
\draw (195.07999999999998, -1) rectangle (241.07999999999998,-56.10000000000001);
\draw(196.07999999999998, -2) node[anchor=north west,align=left] {\large Groupoids (i.e. small categories in which all morphisms are isomorphisms)};
\draw (196.07999999999998, -2) rectangle (219.30999999999997,-5.7);
\draw(197.07999999999998, -3) node[anchor=north west,align=left] {Groupoids (i.e.\\ small categories\\ in which\\ all morphisms\\ are isomorphisms)};
\draw (197.07999999999998, -3) rectangle (201.42999999999998,-5.6);
\draw(219.41, -2) node[anchor=north west,align=left] {\large Structure and classification of infinite or finite groups};
\draw (219.41, -2) rectangle (240.91,-10.1);
\draw(220.41, -3) node[anchor=north west,align=left] {Free products of \\ groups, free products\\ with amalgamation,\\ Higman-Neumann-Neumann\\ extensions,\\ and generalizations};
\draw (220.41, -3) rectangle (226.01,-6.1);
\draw(226.10999999999999, -3) node[anchor=north west,align=left] {Residual \\ properties and\\ generalizations;\\ residually\\ finite groups};
\draw (226.10999999999999, -3) rectangle (230.45999999999998,-5.6);
\draw(230.56, -3) node[anchor=north west,align=left] {Chains and \\ lattices of \\ subgroups, subnormal\\ subgroups};
\draw (230.56, -3) rectangle (234.16,-5.1);
\draw(234.26, -3) node[anchor=north west,align=left] {Extensions,\\ wreath products,\\ and other\\ compositions\\ of groups};
\draw (234.26, -3) rectangle (237.60999999999999,-5.6);
\draw(237.71, -3) node[anchor=north west,align=left] {Automorphisms\\ of \\ infinite groups};
\draw (237.71, -3) rectangle (240.81,-4.6);
\draw(220.41, -6.2) node[anchor=north west,align=left] {Groups with\\ a \(BN\)-pair;\\ buildings};
\draw (220.41, -6.2) rectangle (223.51,-7.800000000000001);
\draw(223.60999999999999, -6.2) node[anchor=north west,align=left] {Quasivarieties\\ and\\ varieties\\ of groups};
\draw (223.60999999999999, -6.2) rectangle (226.45999999999998,-8.3);
\draw(226.56, -6.2) node[anchor=north west,align=left] {Subgroup\\ theorems;\\ subgroup\\ growth};
\draw (226.56, -6.2) rectangle (229.16,-8.3);
\draw(229.26, -6.2) node[anchor=north west,align=left] {Limits,\\ profinite\\ groups};
\draw (229.26, -6.2) rectangle (231.85999999999999,-7.800000000000001);
\draw(231.96, -6.2) node[anchor=north west,align=left] {Local \\ properties\\ of groups};
\draw (231.96, -6.2) rectangle (234.56,-7.800000000000001);
\draw(234.66, -6.2) node[anchor=north west,align=left] {General \\ structure\\ theorems\\ for groups};
\draw (234.66, -6.2) rectangle (237.26,-8.3);
\draw(237.35999999999999, -6.2) node[anchor=north west,align=left] {Conjugacy\\ classes\\ for groups};
\draw (237.35999999999999, -6.2) rectangle (239.95999999999998,-7.800000000000001);
\draw(220.41, -8.4) node[anchor=north west,align=left] {Maximal\\ subgroups};
\draw (220.41, -8.4) rectangle (222.76,-9.5);
\draw(222.85999999999999, -8.4) node[anchor=north west,align=left] {Free \\ nonabelian\\ groups};
\draw (222.85999999999999, -8.4) rectangle (224.95999999999998,-10.0);
\draw(225.06, -8.4) node[anchor=north west,align=left] {Groups\\ acting\\ on trees};
\draw (225.06, -8.4) rectangle (227.16,-10.0);
\draw(227.26, -8.4) node[anchor=north west,align=left] {Simple\\ groups};
\draw (227.26, -8.4) rectangle (229.10999999999999,-9.5);
\draw(196.07999999999998, -5.8) node[anchor=north west,align=left] {\large Connections of group theory with homological algebra and category theory};
\draw (196.07999999999998, -5.8) rectangle (219.0,-9.0);
\draw(197.07999999999998, -6.8) node[anchor=north west,align=left] {Cohomology\\ of groups};
\draw (197.07999999999998, -6.8) rectangle (200.67999999999998,-8.4);
\draw(200.77999999999997, -6.8) node[anchor=north west,align=left] {Category\\ of groups};
\draw (200.77999999999997, -6.8) rectangle (204.12999999999997,-8.4);
\draw(204.23, -6.8) node[anchor=north west,align=left] {Homological\\ methods\\ in \\ group theory};
\draw (204.23, -6.8) rectangle (207.07999999999998,-8.9);
\draw(196.07999999999998, -10.2) node[anchor=north west,align=left] {\large Special aspects of infinite or finite groups};
\draw (196.07999999999998, -10.2) rectangle (212.88,-26.400000000000002);
\draw(197.07999999999998, -11.2) node[anchor=north west,align=left] {Word problems, \\ other decision \\ problems, connections\\ with logic \\ and automata \\ (group-theoretic aspects)};
\draw (197.07999999999998, -11.2) rectangle (202.42999999999998,-14.299999999999999);
\draw(202.52999999999997, -11.2) node[anchor=north west,align=left] {Cancellation\\ theory of \\ groups; \\ application of van\\ Kampen diagrams};
\draw (202.52999999999997, -11.2) rectangle (206.87999999999997,-13.799999999999999);
\draw(206.98, -11.2) node[anchor=north west,align=left] {Representations\\ of groups\\ as automorphism\\ groups of\\ algebraic systems};
\draw (206.98, -11.2) rectangle (211.32999999999998,-13.799999999999999);
\draw(197.07999999999998, -14.399999999999999) node[anchor=north west,align=left] {Fundamental \\ groups and their\\ automorphisms\\ (group-theoretic\\ aspects)};
\draw (197.07999999999998, -14.399999999999999) rectangle (201.17999999999998,-17.0);
\draw(201.27999999999997, -14.399999999999999) node[anchor=north west,align=left] {Generalizations\\ of \\ solvable and \\ nilpotent groups};
\draw (201.27999999999997, -14.399999999999999) rectangle (205.12999999999997,-16.5);
\draw(205.23, -14.399999999999999) node[anchor=north west,align=left] {Reflection\\ and Coxeter\\ groups\\ (group-theoretic\\ aspects)};
\draw (205.23, -14.399999999999999) rectangle (209.07999999999998,-17.0);
\draw(209.17999999999998, -14.399999999999999) node[anchor=north west,align=left] {Generators,\\ relations, \\ and presentations\\ of groups};
\draw (209.17999999999998, -14.399999999999999) rectangle (212.77999999999997,-16.5);
\draw(197.07999999999998, -17.1) node[anchor=north west,align=left] {Derived series,\\ central\\ series, and\\ generalizations\\ for groups};
\draw (197.07999999999998, -17.1) rectangle (200.67999999999998,-19.700000000000003);
\draw(200.77999999999997, -17.1) node[anchor=north west,align=left] {Other classes\\ of groups\\ defined by \\ subgroup chains};
\draw (200.77999999999997, -17.1) rectangle (204.37999999999997,-19.200000000000003);
\draw(204.48, -17.1) node[anchor=north west,align=left] {Other groups\\ related\\ to topology\\ or analysis};
\draw (204.48, -17.1) rectangle (208.07999999999998,-19.200000000000003);
\draw(208.17999999999998, -17.1) node[anchor=north west,align=left] {Algebraic \\ geometry over \\ groups; equations\\ over groups};
\draw (208.17999999999998, -17.1) rectangle (211.77999999999997,-19.200000000000003);
\draw(197.07999999999998, -19.8) node[anchor=north west,align=left] {Commutator\\ calculus};
\draw (197.07999999999998, -19.8) rectangle (200.42999999999998,-21.400000000000002);
\draw(200.52999999999997, -19.8) node[anchor=north west,align=left] {Ordered \\ groups \\ (group-theoretic\\ aspects)};
\draw (200.52999999999997, -19.8) rectangle (203.87999999999997,-21.900000000000002);
\draw(203.98, -19.8) node[anchor=north west,align=left] {Hyperbolic \\ groups and \\ nonpositively\\ curved groups};
\draw (203.98, -19.8) rectangle (207.32999999999998,-21.900000000000002);
\draw(207.42999999999998, -19.8) node[anchor=north west,align=left] {Associated\\ Lie \\ structures\\ for groups};
\draw (207.42999999999998, -19.8) rectangle (210.52999999999997,-21.900000000000002);
\draw(210.63, -19.8) node[anchor=north west,align=left] {Geometric\\ group\\ theory};
\draw (210.63, -19.8) rectangle (212.73,-21.400000000000002);
\draw(197.07999999999998, -22.0) node[anchor=north west,align=left] {FC-groups\\ and \\ their \\ generalizations};
\draw (197.07999999999998, -22.0) rectangle (199.92999999999998,-24.1);
\draw(200.02999999999997, -22.0) node[anchor=north west,align=left] {Automorphism\\ groups\\ of groups};
\draw (200.02999999999997, -22.0) rectangle (202.87999999999997,-23.6);
\draw(202.98, -22.0) node[anchor=north west,align=left] {Periodic\\ groups; \\ locally \\ finite groups};
\draw (202.98, -22.0) rectangle (205.82999999999998,-24.1);
\draw(205.92999999999998, -22.0) node[anchor=north west,align=left] {Asymptotic\\ properties\\ of groups};
\draw (205.92999999999998, -22.0) rectangle (208.77999999999997,-23.6);
\draw(208.88, -22.0) node[anchor=north west,align=left] {Groups of\\ finite \\ Morley rank};
\draw (208.88, -22.0) rectangle (211.48,-23.6);
\draw(197.07999999999998, -24.2) node[anchor=north west,align=left] {Solvable\\ groups, \\ supersolvable\\ groups};
\draw (197.07999999999998, -24.2) rectangle (199.67999999999998,-26.3);
\draw(199.77999999999997, -24.2) node[anchor=north west,align=left] {Formations\\ of groups,\\ Fitting\\ classes};
\draw (199.77999999999997, -24.2) rectangle (202.37999999999997,-26.3);
\draw(202.48, -24.2) node[anchor=north west,align=left] {Braid \\ groups; \\ Artin groups};
\draw (202.48, -24.2) rectangle (205.07999999999998,-25.8);
\draw(205.17999999999998, -24.2) node[anchor=north west,align=left] {Nilpotent\\ groups};
\draw (205.17999999999998, -24.2) rectangle (207.52999999999997,-25.3);
\draw(207.63, -24.2) node[anchor=north west,align=left] {Engel \\ conditions};
\draw (207.63, -24.2) rectangle (209.98,-25.3);
\draw(212.98, -10.2) node[anchor=north west,align=left] {\large Linear algebraic groups and related topics};
\draw (212.98, -10.2) rectangle (229.03,-21.5);
\draw(213.98, -11.2) node[anchor=north west,align=left] {Linear algebraic\\ groups \\ over the reals,\\ the complexes,\\ the quaternions};
\draw (213.98, -11.2) rectangle (218.57999999999998,-13.799999999999999);
\draw(218.67999999999998, -11.2) node[anchor=north west,align=left] {Quantum groups\\ (quantized\\ function \\ algebras) and their\\ representations};
\draw (218.67999999999998, -11.2) rectangle (223.27999999999997,-13.799999999999999);
\draw(223.38, -11.2) node[anchor=north west,align=left] {Linear \\ algebraic groups\\ over local\\ fields and\\ their integers};
\draw (223.38, -11.2) rectangle (227.48,-13.799999999999999);
\draw(213.98, -13.899999999999999) node[anchor=north west,align=left] {Linear algebraic\\ groups\\ over adèles\\ and other \\ rings and schemes};
\draw (213.98, -13.899999999999999) rectangle (217.82999999999998,-16.5);
\draw(217.92999999999998, -13.899999999999999) node[anchor=north west,align=left] {Applications\\ of linear\\ algebraic\\ groups\\ to the sciences};
\draw (217.92999999999998, -13.899999999999999) rectangle (221.77999999999997,-16.5);
\draw(221.88, -13.899999999999999) node[anchor=north west,align=left] {Structure\\ theory for\\ linear \\ algebraic groups};
\draw (221.88, -13.899999999999999) rectangle (225.48,-15.999999999999998);
\draw(225.57999999999998, -13.899999999999999) node[anchor=north west,align=left] {Representation\\ theory for\\ linear \\ algebraic groups};
\draw (225.57999999999998, -13.899999999999999) rectangle (228.92999999999998,-15.999999999999998);
\draw(213.98, -16.6) node[anchor=north west,align=left] {Linear algebraic\\ groups\\ over global\\ fields and \\ their integers};
\draw (213.98, -16.6) rectangle (217.32999999999998,-19.200000000000003);
\draw(217.42999999999998, -16.6) node[anchor=north west,align=left] {Exceptional\\ groups};
\draw (217.42999999999998, -16.6) rectangle (220.77999999999997,-18.200000000000003);
\draw(220.88, -16.6) node[anchor=north west,align=left] {Cohomology\\ theory for\\ linear \\ algebraic groups};
\draw (220.88, -16.6) rectangle (223.98,-18.700000000000003);
\draw(224.07999999999998, -16.6) node[anchor=north west,align=left] {Linear \\ algebraic groups\\ over \\ arbitrary fields};
\draw (224.07999999999998, -16.6) rectangle (227.17999999999998,-18.700000000000003);
\draw(213.98, -19.3) node[anchor=north west,align=left] {Linear \\ algebraic \\ groups over \\ finite fields};
\draw (213.98, -19.3) rectangle (217.07999999999998,-21.400000000000002);
\draw(217.17999999999998, -19.3) node[anchor=north west,align=left] {Schur and\\ \(q\)-Schur\\ algebras};
\draw (217.17999999999998, -19.3) rectangle (219.77999999999997,-20.900000000000002);
\draw(219.88, -19.3) node[anchor=north west,align=left] {Kac-Moody\\ groups};
\draw (219.88, -19.3) rectangle (222.23,-20.400000000000002);
\draw(212.98, -21.6) node[anchor=north west,align=left] {\large Probabilistic methods in group theory};
\draw (212.98, -21.6) rectangle (225.04999999999998,-24.8);
\draw(213.98, -22.6) node[anchor=north west,align=left] {Probabilistic\\ methods\\ in \\ group theory};
\draw (213.98, -22.6) rectangle (216.57999999999998,-24.700000000000003);
\draw(212.98, -24.900000000000002) node[anchor=north west,align=left] {\large History of\\ group theory};
\draw (212.98, -24.900000000000002) rectangle (216.67999999999998,-26.000000000000004);
\draw(229.13, -10.2) node[anchor=north west,align=left] {\large Representation theory of groups};
\draw (229.13, -10.2) rectangle (240.98,-25.4);
\draw(230.13, -11.2) node[anchor=north west,align=left] {Applications \\ of group \\ representations to \\ physics and other\\ areas of science};
\draw (230.13, -11.2) rectangle (234.98,-13.799999999999999);
\draw(235.07999999999998, -11.2) node[anchor=north west,align=left] {Group rings of\\ infinite groups\\ and their \\ modules \\ (group-theoretic aspects)};
\draw (235.07999999999998, -11.2) rectangle (239.67999999999998,-13.799999999999999);
\draw(230.13, -13.899999999999999) node[anchor=north west,align=left] {Representationsof\\ sporadic\\ groups};
\draw (230.13, -13.899999999999999) rectangle (234.73,-15.999999999999998);
\draw(234.82999999999998, -13.899999999999999) node[anchor=north west,align=left] {Group rings of\\ finite groups\\ and their \\ modules (group-theoretic\\ aspects)};
\draw (234.82999999999998, -13.899999999999999) rectangle (238.92999999999998,-16.5);
\draw(230.13, -16.6) node[anchor=north west,align=left] {Representations\\ of \\ infinite \\ symmetric groups};
\draw (230.13, -16.6) rectangle (233.73,-18.700000000000003);
\draw(233.82999999999998, -16.6) node[anchor=north west,align=left] {Representations\\ of \\ finite groups\\ of Lie type};
\draw (233.82999999999998, -16.6) rectangle (237.42999999999998,-18.700000000000003);
\draw(237.53, -16.6) node[anchor=north west,align=left] {Hecke \\ algebras and\\ their \\ representations};
\draw (237.53, -16.6) rectangle (240.88,-18.700000000000003);
\draw(230.13, -18.8) node[anchor=north west,align=left] {\(p\)-adic \\ representations\\ of\\ finite groups};
\draw (230.13, -18.8) rectangle (233.48,-20.900000000000002);
\draw(233.57999999999998, -18.8) node[anchor=north west,align=left] {Integral \\ representations\\ of \\ infinite groups};
\draw (233.57999999999998, -18.8) rectangle (236.92999999999998,-20.900000000000002);
\draw(237.03, -18.8) node[anchor=north west,align=left] {Ordinary\\ representations\\ and\\ characters};
\draw (237.03, -18.8) rectangle (240.13,-20.900000000000002);
\draw(230.13, -21.0) node[anchor=north west,align=left] {Projective\\ representations\\ and\\ multipliers};
\draw (230.13, -21.0) rectangle (233.23,-23.1);
\draw(233.32999999999998, -21.0) node[anchor=north west,align=left] {Representations\\ of \\ finite \\ symmetric groups};
\draw (233.32999999999998, -21.0) rectangle (236.42999999999998,-23.1);
\draw(236.53, -21.0) node[anchor=north west,align=left] {Integral \\ representations\\ of \\ finite groups};
\draw (236.53, -21.0) rectangle (239.38,-23.1);
\draw(230.13, -23.2) node[anchor=north west,align=left] {Modular \\ representations\\ and\\ characters};
\draw (230.13, -23.2) rectangle (232.98,-25.3);
\draw(196.07999999999998, -26.500000000000004) node[anchor=north west,align=left] {\large Other generalizations of groups};
\draw (196.07999999999998, -26.500000000000004) rectangle (207.92999999999998,-31.400000000000006);
\draw(197.07999999999998, -27.500000000000004) node[anchor=north west,align=left] {Sets with a\\ single \\ binary operation\\ (groupoids)};
\draw (197.07999999999998, -27.500000000000004) rectangle (200.67999999999998,-29.600000000000005);
\draw(200.77999999999997, -27.500000000000004) node[anchor=north west,align=left] {Ternary systems\\ (heaps,\\ semiheaps, \\ heapoids, etc.)};
\draw (200.77999999999997, -27.500000000000004) rectangle (204.37999999999997,-29.600000000000005);
\draw(204.48, -27.500000000000004) node[anchor=north west,align=left] {Loops,\\ quasigroups};
\draw (204.48, -27.500000000000004) rectangle (207.82999999999998,-29.100000000000005);
\draw(197.07999999999998, -29.700000000000003) node[anchor=north west,align=left] {\(n\)-ary\\ systems\\ \((n\ge~3)\)};
\draw (197.07999999999998, -29.700000000000003) rectangle (200.67999999999998,-31.300000000000004);
\draw(200.77999999999997, -29.700000000000003) node[anchor=north west,align=left] {Hypergroups};
\draw (200.77999999999997, -29.700000000000003) rectangle (203.62999999999997,-30.800000000000004);
\draw(203.73, -29.700000000000003) node[anchor=north west,align=left] {Fuzzy\\ groups};
\draw (203.73, -29.700000000000003) rectangle (205.57999999999998,-30.800000000000004);
\draw(208.02999999999997, -26.500000000000004) node[anchor=north west,align=left] {\large Abstract finite groups};
\draw (208.02999999999997, -26.500000000000004) rectangle (219.42999999999998,-40.5);
\draw(209.02999999999997, -27.500000000000004) node[anchor=north west,align=left] {Finite solvable\\ groups, theory\\ of formations,\\ Schunck classes,\\ Fitting \\ classes, \(\pi\)-length, ranks};
\draw (209.02999999999997, -27.500000000000004) rectangle (215.62999999999997,-30.600000000000005);
\draw(215.72999999999996, -27.500000000000004) node[anchor=north west,align=left] {Products of\\ subgroups\\ of abstract\\ finite groups};
\draw (215.72999999999996, -27.500000000000004) rectangle (219.32999999999996,-29.600000000000005);
\draw(209.02999999999997, -30.700000000000003) node[anchor=north west,align=left] {Sylow \\ subgroups, Sylow\\ properties,\\ \(\pi\)-groups,\\ \(\pi\)-structure};
\draw (209.02999999999997, -30.700000000000003) rectangle (213.62999999999997,-33.300000000000004);
\draw(213.72999999999996, -30.700000000000003) node[anchor=north west,align=left] {Arithmetic and\\ combinatorial\\ problems \\ involving \\ abstract finite groups};
\draw (213.72999999999996, -30.700000000000003) rectangle (218.32999999999996,-33.300000000000004);
\draw(209.02999999999997, -33.400000000000006) node[anchor=north west,align=left] {Finite \\ nilpotent \\ groups, \(p\)-groups};
\draw (209.02999999999997, -33.400000000000006) rectangle (213.37999999999997,-35.00000000000001);
\draw(213.47999999999996, -33.400000000000006) node[anchor=north west,align=left] {Special \\ subgroups \\ (Frattini, \\ Fitting, etc.)};
\draw (213.47999999999996, -33.400000000000006) rectangle (216.82999999999996,-35.50000000000001);
\draw(209.02999999999997, -35.60000000000001) node[anchor=north west,align=left] {Finite simple\\ groups \\ and their \\ classification};
\draw (209.02999999999997, -35.60000000000001) rectangle (212.12999999999997,-37.70000000000001);
\draw(212.22999999999996, -35.60000000000001) node[anchor=north west,align=left] {Simple groups:\\ alternating\\ groups\\ and groups\\ of Lie type};
\draw (212.22999999999996, -35.60000000000001) rectangle (215.32999999999996,-38.20000000000001);
\draw(215.42999999999998, -35.60000000000001) node[anchor=north west,align=left] {Simple \\ groups: \\ sporadic groups};
\draw (215.42999999999998, -35.60000000000001) rectangle (218.52999999999997,-37.20000000000001);
\draw(209.02999999999997, -38.300000000000004) node[anchor=north west,align=left] {Series and\\ lattices\\ of subgroups};
\draw (209.02999999999997, -38.300000000000004) rectangle (212.12999999999997,-39.900000000000006);
\draw(212.22999999999996, -38.300000000000004) node[anchor=north west,align=left] {Subnormal \\ subgroups of\\ abstract \\ finite groups};
\draw (212.22999999999996, -38.300000000000004) rectangle (215.32999999999996,-40.400000000000006);
\draw(215.42999999999998, -38.300000000000004) node[anchor=north west,align=left] {Automorphisms\\ of \\ abstract \\ finite groups};
\draw (215.42999999999998, -38.300000000000004) rectangle (218.52999999999997,-40.400000000000006);
\draw(196.07999999999998, -31.500000000000007) node[anchor=north west,align=left] {\large Other groups of matrices};
\draw (196.07999999999998, -31.500000000000007) rectangle (207.48,-39.60000000000001);
\draw(197.07999999999998, -32.50000000000001) node[anchor=north west,align=left] {Unimodular \\ groups, \\ congruence subgroups\\ (group-theoretic aspects)};
\draw (197.07999999999998, -32.50000000000001) rectangle (204.42999999999998,-35.10000000000001);
\draw(204.52999999999997, -32.50000000000001) node[anchor=north west,align=left] {Other matrix\\ groups\\ over fields};
\draw (204.52999999999997, -32.50000000000001) rectangle (207.37999999999997,-34.10000000000001);
\draw(197.07999999999998, -35.20000000000001) node[anchor=north west,align=left] {Fuchsian groups\\ and their\\ generalizations\\ (group-theoretic\\ aspects)};
\draw (197.07999999999998, -35.20000000000001) rectangle (200.92999999999998,-37.80000000000001);
\draw(201.02999999999997, -35.20000000000001) node[anchor=north west,align=left] {Other geometric\\ groups,\\ including\\ crystallographic\\ groups};
\draw (201.02999999999997, -35.20000000000001) rectangle (204.37999999999997,-37.80000000000001);
\draw(204.48, -35.20000000000001) node[anchor=north west,align=left] {Other \\ matrix groups\\ over \\ finite fields};
\draw (204.48, -35.20000000000001) rectangle (207.32999999999998,-37.30000000000001);
\draw(197.07999999999998, -37.900000000000006) node[anchor=north west,align=left] {Other \\ matrix groups\\ over rings};
\draw (197.07999999999998, -37.900000000000006) rectangle (200.17999999999998,-39.50000000000001);
\draw(219.52999999999997, -26.500000000000004) node[anchor=north west,align=left] {\large Semigroups};
\draw (219.52999999999997, -26.500000000000004) rectangle (230.17999999999998,-43.400000000000006);
\draw(220.52999999999997, -27.500000000000004) node[anchor=north west,align=left] {Free semigroups,\\ generators and \\ relations, \\ word problems};
\draw (220.52999999999997, -27.500000000000004) rectangle (226.37999999999997,-30.100000000000005);
\draw(226.47999999999996, -27.500000000000004) node[anchor=north west,align=left] {Semigroups\\ of \\ transformations, \\ relations, \\ partitions, etc.};
\draw (226.47999999999996, -27.500000000000004) rectangle (230.07999999999996,-30.100000000000005);
\draw(220.52999999999997, -30.200000000000003) node[anchor=north west,align=left] {Connections \\ of semigroups\\ with homological\\ algebra \\ and category theory};
\draw (220.52999999999997, -30.200000000000003) rectangle (225.12999999999997,-32.800000000000004);
\draw(225.22999999999996, -30.200000000000003) node[anchor=north west,align=left] {Commutative\\ semigroups};
\draw (225.22999999999996, -30.200000000000003) rectangle (229.07999999999996,-31.800000000000004);
\draw(220.52999999999997, -32.900000000000006) node[anchor=north west,align=left] {Semigroup \\ rings, \\ multiplicative \\ semigroups of rings};
\draw (220.52999999999997, -32.900000000000006) rectangle (224.37999999999997,-35.00000000000001);
\draw(224.47999999999996, -32.900000000000006) node[anchor=north west,align=left] {Representation\\ of \\ semigroups; \\ actions of \\ semigroups on sets};
\draw (224.47999999999996, -32.900000000000006) rectangle (228.32999999999996,-35.50000000000001);
\draw(220.52999999999997, -35.60000000000001) node[anchor=north west,align=left] {Semigroups\\ in automata\\ theory, \\ linguistics, etc.};
\draw (220.52999999999997, -35.60000000000001) rectangle (224.37999999999997,-37.70000000000001);
\draw(224.47999999999996, -35.60000000000001) node[anchor=north west,align=left] {Varieties\\ and \\ pseudovarieties\\ of semigroups};
\draw (224.47999999999996, -35.60000000000001) rectangle (227.82999999999996,-37.70000000000001);
\draw(227.92999999999998, -35.60000000000001) node[anchor=north west,align=left] {Mappings\\ of \\ semigroups};
\draw (227.92999999999998, -35.60000000000001) rectangle (230.02999999999997,-37.20000000000001);
\draw(220.52999999999997, -37.800000000000004) node[anchor=north west,align=left] {Regular\\ semigroups};
\draw (220.52999999999997, -37.800000000000004) rectangle (223.87999999999997,-39.400000000000006);
\draw(223.97999999999996, -37.800000000000004) node[anchor=north west,align=left] {Inverse\\ semigroups};
\draw (223.97999999999996, -37.800000000000004) rectangle (227.32999999999996,-39.400000000000006);
\draw(227.42999999999998, -37.800000000000004) node[anchor=north west,align=left] {Algebraic\\ monoids};
\draw (227.42999999999998, -37.800000000000004) rectangle (229.77999999999997,-38.900000000000006);
\draw(220.52999999999997, -39.5) node[anchor=north west,align=left] {Orthodox\\ semigroups};
\draw (220.52999999999997, -39.5) rectangle (223.87999999999997,-41.1);
\draw(223.97999999999996, -39.5) node[anchor=north west,align=left] {General \\ structure\\ theory for\\ semigroups};
\draw (223.97999999999996, -39.5) rectangle (227.07999999999996,-41.6);
\draw(227.17999999999998, -39.5) node[anchor=north west,align=left] {Radical \\ theory for\\ semigroups};
\draw (227.17999999999998, -39.5) rectangle (230.02999999999997,-41.1);
\draw(220.52999999999997, -41.7) node[anchor=north west,align=left] {Ideal \\ theory for\\ semigroups};
\draw (220.52999999999997, -41.7) rectangle (223.37999999999997,-43.300000000000004);
\draw(223.47999999999996, -41.7) node[anchor=north west,align=left] {Arithmetic\\ theory of\\ semigroups};
\draw (223.47999999999996, -41.7) rectangle (226.32999999999996,-43.300000000000004);
\draw(226.42999999999998, -41.7) node[anchor=north west,align=left] {Generalizations\\ of\\ semigroups};
\draw (226.42999999999998, -41.7) rectangle (229.27999999999997,-43.300000000000004);
\draw(208.02999999999997, -40.6) node[anchor=north west,align=left] {\large Computational methods\\ for problems \\ pertaining to group theory};
\draw (208.02999999999997, -40.6) rectangle (214.51999999999998,-42.2);
\draw(230.27999999999997, -26.500000000000004) node[anchor=north west,align=left] {\large Permutation groups};
\draw (230.27999999999997, -26.500000000000004) rectangle (240.17999999999998,-37.5);
\draw(231.27999999999997, -27.500000000000004) node[anchor=north west,align=left] {Multiply\\ transitive\\ infinite groups};
\draw (231.27999999999997, -27.500000000000004) rectangle (236.12999999999997,-29.600000000000005);
\draw(236.22999999999996, -27.500000000000004) node[anchor=north west,align=left] {Characterization\\ theorems\\ for \\ permutation groups};
\draw (236.22999999999996, -27.500000000000004) rectangle (240.07999999999996,-29.600000000000005);
\draw(231.27999999999997, -29.700000000000003) node[anchor=north west,align=left] {Finite automorphism\\ groups of\\ algebraic, \\ geometric, or \\ combinatorial structures};
\draw (231.27999999999997, -29.700000000000003) rectangle (236.12999999999997,-32.300000000000004);
\draw(236.22999999999996, -29.700000000000003) node[anchor=north west,align=left] {General \\ theory for \\ finite \\ permutation groups};
\draw (236.22999999999996, -29.700000000000003) rectangle (239.82999999999996,-31.800000000000004);
\draw(231.27999999999997, -32.400000000000006) node[anchor=north west,align=left] {General \\ theory for \\ infinite \\ permutation groups};
\draw (231.27999999999997, -32.400000000000006) rectangle (234.62999999999997,-34.50000000000001);
\draw(234.72999999999996, -32.400000000000006) node[anchor=north west,align=left] {Multiply \\ transitive\\ finite groups};
\draw (234.72999999999996, -32.400000000000006) rectangle (238.07999999999996,-34.00000000000001);
\draw(231.27999999999997, -34.60000000000001) node[anchor=north west,align=left] {Subgroups\\ of symmetric\\ groups};
\draw (231.27999999999997, -34.60000000000001) rectangle (234.12999999999997,-36.20000000000001);
\draw(234.22999999999996, -34.60000000000001) node[anchor=north west,align=left] {Infinite\\ automorphism\\ groups};
\draw (234.22999999999996, -34.60000000000001) rectangle (236.82999999999996,-36.20000000000001);
\draw(236.92999999999998, -34.60000000000001) node[anchor=north west,align=left] {Primitive\\ groups};
\draw (236.92999999999998, -34.60000000000001) rectangle (239.27999999999997,-35.70000000000001);
\draw(231.27999999999997, -36.300000000000004) node[anchor=north west,align=left] {Symmetric\\ groups};
\draw (231.27999999999997, -36.300000000000004) rectangle (233.62999999999997,-37.400000000000006);
\draw(230.27999999999997, -37.6) node[anchor=north west,align=left] {\large Foundations};
\draw (230.27999999999997, -37.6) rectangle (237.72999999999996,-43.0);
\draw(231.27999999999997, -38.6) node[anchor=north west,align=left] {Metamathematical\\ considerations \\ in group theory};
\draw (231.27999999999997, -38.6) rectangle (237.62999999999997,-40.7);
\draw(231.27999999999997, -40.800000000000004) node[anchor=north west,align=left] {Axiomatics\\ and elementary\\ properties\\ of groups};
\draw (231.27999999999997, -40.800000000000004) rectangle (234.62999999999997,-42.900000000000006);
\draw(234.72999999999996, -40.800000000000004) node[anchor=north west,align=left] {Applications\\ of \\ logic to \\ group theory};
\draw (234.72999999999996, -40.800000000000004) rectangle (237.32999999999996,-42.900000000000006);
\draw(196.07999999999998, -43.50000000000001) node[anchor=north west,align=left] {\large Abelian groups};
\draw (196.07999999999998, -43.50000000000001) rectangle (204.98,-56.00000000000001);
\draw(197.07999999999998, -44.50000000000001) node[anchor=north west,align=left] {Torsion-free\\ groups, \\ infinite rank};
\draw (197.07999999999998, -44.50000000000001) rectangle (201.67999999999998,-46.60000000000001);
\draw(201.77999999999997, -44.50000000000001) node[anchor=north west,align=left] {Torsion-free\\ groups,\\ finite rank};
\draw (201.77999999999997, -44.50000000000001) rectangle (204.87999999999997,-46.10000000000001);
\draw(197.07999999999998, -46.70000000000001) node[anchor=north west,align=left] {Automorphisms,\\ homomorphisms,\\ endomorphisms,\\ etc. for\\ abelian groups};
\draw (197.07999999999998, -46.70000000000001) rectangle (201.17999999999998,-49.30000000000001);
\draw(201.27999999999997, -46.70000000000001) node[anchor=north west,align=left] {Direct sums,\\ direct products,\\ etc. for\\ abelian groups};
\draw (201.27999999999997, -46.70000000000001) rectangle (204.87999999999997,-48.80000000000001);
\draw(197.07999999999998, -49.400000000000006) node[anchor=north west,align=left] {Homological \\ and categorical\\ methods for\\ abelian groups};
\draw (197.07999999999998, -49.400000000000006) rectangle (200.92999999999998,-51.50000000000001);
\draw(201.02999999999997, -49.400000000000006) node[anchor=north west,align=left] {Torsion groups,\\ primary\\ groups and \\ generalized \\ primary groups};
\draw (201.02999999999997, -49.400000000000006) rectangle (204.37999999999997,-52.00000000000001);
\draw(197.07999999999998, -52.10000000000001) node[anchor=north west,align=left] {Subgroups\\ of \\ abelian groups};
\draw (197.07999999999998, -52.10000000000001) rectangle (199.92999999999998,-53.70000000000001);
\draw(200.02999999999997, -52.10000000000001) node[anchor=north west,align=left] {Topological\\ methods\\ for \\ abelian groups};
\draw (200.02999999999997, -52.10000000000001) rectangle (202.87999999999997,-54.20000000000001);
\draw(202.98, -52.10000000000001) node[anchor=north west,align=left] {Mixed\\ groups};
\draw (202.98, -52.10000000000001) rectangle (204.82999999999998,-53.20000000000001);
\draw(197.07999999999998, -54.30000000000001) node[anchor=north west,align=left] {Extensions\\ of abelian\\ groups};
\draw (197.07999999999998, -54.30000000000001) rectangle (199.67999999999998,-55.90000000000001);
\draw(199.77999999999997, -54.30000000000001) node[anchor=north west,align=left] {Finite\\ abelian\\ groups};
\draw (199.77999999999997, -54.30000000000001) rectangle (202.12999999999997,-55.90000000000001);
\draw(195.07999999999998, -56.20000000000001) node[anchor=north west,align=left] {\LARGE Measure and integration};
\draw (195.07999999999998, -56.20000000000001) rectangle (238.6,-84.30000000000001);
\draw(196.07999999999998, -57.20000000000001) node[anchor=north west,align=left] {\large Set functions and measures on spaces with additional structure};
\draw (196.07999999999998, -57.20000000000001) rectangle (217.88,-61.900000000000006);
\draw(197.07999999999998, -58.20000000000001) node[anchor=north west,align=left] {Integration theory\\ via linear \\ functionals (Radon \\ measures, Daniell \\ integrals, etc.),\\ representing set\\ functions and measures};
\draw (197.07999999999998, -58.20000000000001) rectangle (202.67999999999998,-61.80000000000001);
\draw(202.77999999999997, -58.20000000000001) node[anchor=north west,align=left] {Set functions and\\ measures and integrals\\ in \\ infinite-dimensional spaces\\ (Wiener measure, \\ Gaussian measure, etc.)};
\draw (202.77999999999997, -58.20000000000001) rectangle (208.37999999999997,-61.30000000000001);
\draw(208.48, -58.20000000000001) node[anchor=north west,align=left] {Set functions \\ and measures on\\ topological \\ spaces (regularity\\ of measures, etc.)};
\draw (208.48, -58.20000000000001) rectangle (213.32999999999998,-60.80000000000001);
\draw(213.42999999999998, -58.20000000000001) node[anchor=north west,align=left] {Set functions and\\ measures on \\ topological groups\\ or semigroups,\\ Haar measures, \\ invariant measures};
\draw (213.42999999999998, -58.20000000000001) rectangle (217.77999999999997,-61.30000000000001);
\draw(217.98, -57.20000000000001) node[anchor=north west,align=left] {\large Miscellaneous topics in measure theory};
\draw (217.98, -57.20000000000001) rectangle (230.35999999999999,-60.40000000000001);
\draw(218.98, -58.20000000000001) node[anchor=north west,align=left] {Other \\ connections \\ with logic \\ and set theory};
\draw (218.98, -58.20000000000001) rectangle (222.32999999999998,-60.30000000000001);
\draw(222.42999999999998, -58.20000000000001) node[anchor=north west,align=left] {Nonstandard\\ measure\\ theory};
\draw (222.42999999999998, -58.20000000000001) rectangle (225.02999999999997,-59.80000000000001);
\draw(225.13, -58.20000000000001) node[anchor=north west,align=left] {Fuzzy\\ measure\\ theory};
\draw (225.13, -58.20000000000001) rectangle (227.23,-59.80000000000001);
\draw(217.98, -60.50000000000001) node[anchor=north west,align=left] {\large History of measure\\ and integration};
\draw (217.98, -60.50000000000001) rectangle (223.54,-61.60000000000001);
\draw(230.45999999999998, -57.20000000000001) node[anchor=north west,align=left] {\large Computational methods \\ for problems pertaining\\ to measure and integration};
\draw (230.45999999999998, -57.20000000000001) rectangle (238.49999999999997,-58.80000000000001);
\draw(196.07999999999998, -62.00000000000001) node[anchor=north west,align=left] {\large Set functions, measures and integrals with values in abstract spaces};
\draw (196.07999999999998, -62.00000000000001) rectangle (217.76,-66.2);
\draw(197.07999999999998, -63.00000000000001) node[anchor=north west,align=left] {Set-valued set \\ functions and \\ measures; integration\\ of set-valued\\ functions; \\ measurable selections};
\draw (197.07999999999998, -63.00000000000001) rectangle (201.42999999999998,-66.10000000000001);
\draw(201.52999999999997, -63.00000000000001) node[anchor=north west,align=left] {Group- or \\ semigroup-valued\\ set \\ functions, measures\\ and integrals};
\draw (201.52999999999997, -63.00000000000001) rectangle (205.62999999999997,-65.60000000000001);
\draw(205.73, -63.00000000000001) node[anchor=north west,align=left] {Vector-valued\\ set functions,\\ measures\\ and integrals};
\draw (205.73, -63.00000000000001) rectangle (209.57999999999998,-65.10000000000001);
\draw(209.67999999999998, -63.00000000000001) node[anchor=north west,align=left] {Set functions,\\ measures and\\ integrals \\ with values in\\ ordered spaces};
\draw (209.67999999999998, -63.00000000000001) rectangle (213.52999999999997,-65.60000000000001);
\draw(217.85999999999999, -62.00000000000001) node[anchor=north west,align=left] {\large Measure-theoretic ergodic theory};
\draw (217.85999999999999, -62.00000000000001) rectangle (229.76,-67.9);
\draw(218.85999999999999, -63.00000000000001) node[anchor=north west,align=left] {Measure-preserving\\ transformations};
\draw (218.85999999999999, -63.00000000000001) rectangle (225.45999999999998,-65.10000000000001);
\draw(225.55999999999997, -63.00000000000001) node[anchor=north west,align=left] {General groups\\ of \\ measure-preserving\\ transformations};
\draw (225.55999999999997, -63.00000000000001) rectangle (229.65999999999997,-65.10000000000001);
\draw(218.85999999999999, -65.2) node[anchor=north west,align=left] {One-parameter\\ continuous \\ families of \\ measure-preserving\\ transformations};
\draw (218.85999999999999, -65.2) rectangle (222.70999999999998,-67.8);
\draw(222.80999999999997, -65.2) node[anchor=north west,align=left] {Entropy \\ and other\\ invariants};
\draw (222.80999999999997, -65.2) rectangle (225.40999999999997,-66.8);
\draw(196.07999999999998, -68.00000000000001) node[anchor=north west,align=left] {\large Classical measure theory};
\draw (196.07999999999998, -68.00000000000001) rectangle (206.48,-84.20000000000002);
\draw(197.07999999999998, -69.00000000000001) node[anchor=north west,align=left] {Classes of sets\\ (Borel fields,\\ \(\sigma\)-rings, \\ etc.), measurable\\ sets, Suslin\\ sets, analytic sets};
\draw (197.07999999999998, -69.00000000000001) rectangle (202.17999999999998,-72.10000000000001);
\draw(202.27999999999997, -69.00000000000001) node[anchor=north west,align=left] {Integration\\ with respect\\ to measures\\ and other\\ set functions};
\draw (202.27999999999997, -69.00000000000001) rectangle (206.37999999999997,-71.60000000000001);
\draw(197.07999999999998, -72.20000000000002) node[anchor=north west,align=left] {Measurable and\\ nonmeasurable \\ functions, sequences\\ of measurable\\ functions,\\ modes of convergence};
\draw (197.07999999999998, -72.20000000000002) rectangle (202.17999999999998,-75.30000000000001);
\draw(202.27999999999997, -72.20000000000002) node[anchor=north west,align=left] {Length, area,\\ volume, \\ other geometric\\ measure theory};
\draw (202.27999999999997, -72.20000000000002) rectangle (206.37999999999997,-74.30000000000001);
\draw(197.07999999999998, -75.40000000000002) node[anchor=north west,align=left] {Abstract \\ differentiation\\ theory, \\ differentiation\\ of set functions};
\draw (197.07999999999998, -75.40000000000002) rectangle (201.42999999999998,-78.00000000000001);
\draw(201.52999999999997, -75.40000000000002) node[anchor=north west,align=left] {Contents, \\ measures, \\ outer measures,\\ capacities};
\draw (201.52999999999997, -75.40000000000002) rectangle (204.87999999999997,-77.50000000000001);
\draw(197.07999999999998, -78.10000000000002) node[anchor=north west,align=left] {Spaces of\\ measures,\\ convergence\\ of measures};
\draw (197.07999999999998, -78.10000000000002) rectangle (200.42999999999998,-80.20000000000002);
\draw(200.52999999999997, -78.10000000000002) node[anchor=north west,align=left] {Measures\\ and integrals\\ in \\ product spaces};
\draw (200.52999999999997, -78.10000000000002) rectangle (203.87999999999997,-80.20000000000002);
\draw(203.98, -78.10000000000002) node[anchor=north west,align=left] {Lifting\\ theory};
\draw (203.98, -78.10000000000002) rectangle (206.07999999999998,-79.20000000000002);
\draw(197.07999999999998, -80.30000000000001) node[anchor=north west,align=left] {Measures \\ on Boolean\\ rings, \\ measure algebras};
\draw (197.07999999999998, -80.30000000000001) rectangle (200.42999999999998,-82.4);
\draw(200.52999999999997, -80.30000000000001) node[anchor=north west,align=left] {Integration\\ and \\ disintegration\\ of measures};
\draw (200.52999999999997, -80.30000000000001) rectangle (203.62999999999997,-82.4);
\draw(203.73, -80.30000000000001) node[anchor=north west,align=left] {Real- or\\ complex-valued\\ set\\ functions};
\draw (203.73, -80.30000000000001) rectangle (206.32999999999998,-82.4);
\draw(197.07999999999998, -82.50000000000001) node[anchor=north west,align=left] {Hausdorff\\ and packing\\ measures};
\draw (197.07999999999998, -82.50000000000001) rectangle (200.17999999999998,-84.10000000000001);
\draw(200.27999999999997, -82.50000000000001) node[anchor=north west,align=left] {Fractals};
\draw (200.27999999999997, -82.50000000000001) rectangle (202.37999999999997,-83.60000000000001);
\draw(195.07999999999998, -84.4) node[anchor=north west,align=left] {\LARGE Commutative algebra};
\draw (195.07999999999998, -84.4) rectangle (237.04,-120.5);
\draw(196.07999999999998, -85.4) node[anchor=north west,align=left] {\large Chain conditions, finiteness conditions in commutative ring theory};
\draw (196.07999999999998, -85.4) rectangle (217.14,-89.60000000000001);
\draw(197.07999999999998, -86.4) node[anchor=north west,align=left] {Commutative \\ rings and modules\\ of finite \\ generation or \\ presentation; \\ number of generators};
\draw (197.07999999999998, -86.4) rectangle (201.67999999999998,-89.5);
\draw(201.77999999999997, -86.4) node[anchor=north west,align=left] {Commutative \\ Artinian rings\\ and modules,\\ finite-dimensional\\ algebras};
\draw (201.77999999999997, -86.4) rectangle (205.62999999999997,-89.0);
\draw(205.73, -86.4) node[anchor=north west,align=left] {Commutative\\ Noetherian\\ rings\\ and modules};
\draw (205.73, -86.4) rectangle (209.07999999999998,-88.5);
\draw(217.23999999999998, -85.4) node[anchor=north west,align=left] {\large Arithmetic rings and other special commutative rings};
\draw (217.23999999999998, -85.4) rectangle (236.78999999999996,-94.5);
\draw(218.23999999999998, -86.4) node[anchor=north west,align=left] {Commutative rings\\ defined by \\ factorization \\ properties (e.g.,\\ atomic, factorial,\\ half-factorial)};
\draw (218.23999999999998, -86.4) rectangle (223.33999999999997,-89.5);
\draw(223.43999999999997, -86.4) node[anchor=north west,align=left] {Commutative rings\\ defined by \\ monomial ideals;\\ Stanley-Reisner\\ face rings; \\ simplicial complexes};
\draw (223.43999999999997, -86.4) rectangle (228.28999999999996,-89.5);
\draw(228.39, -86.4) node[anchor=north west,align=left] {Dedekind, \\ Prüfer, Krull\\ and Mori rings\\ and their\\ generalizations};
\draw (228.39, -86.4) rectangle (232.48999999999998,-89.0);
\draw(232.58999999999997, -86.4) node[anchor=north west,align=left] {Polynomial \\ rings and \\ ideals; rings \\ of integer-valued\\ polynomials};
\draw (232.58999999999997, -86.4) rectangle (236.68999999999997,-89.0);
\draw(218.23999999999998, -89.60000000000001) node[anchor=north west,align=left] {Principal\\ ideal rings};
\draw (218.23999999999998, -89.60000000000001) rectangle (222.08999999999997,-91.2);
\draw(222.18999999999997, -89.60000000000001) node[anchor=north west,align=left] {Commutative\\ rings defined\\ by binomial\\ ideals, \\ toric rings, etc.};
\draw (222.18999999999997, -89.60000000000001) rectangle (226.03999999999996,-92.2);
\draw(226.14, -89.60000000000001) node[anchor=north west,align=left] {Other \\ commutative rings\\ defined \\ by combinatorial\\ properties};
\draw (226.14, -89.60000000000001) rectangle (229.98999999999998,-92.2);
\draw(230.08999999999997, -89.60000000000001) node[anchor=north west,align=left] {Witt vectors\\ and \\ related rings};
\draw (230.08999999999997, -89.60000000000001) rectangle (233.18999999999997,-91.2);
\draw(233.29, -89.60000000000001) node[anchor=north west,align=left] {Rings with\\ straightening\\ laws, \\ Hodge algebras};
\draw (233.29, -89.60000000000001) rectangle (236.39,-91.7);
\draw(218.23999999999998, -92.30000000000001) node[anchor=north west,align=left] {Euclidean\\ rings\\ and \\ generalizations};
\draw (218.23999999999998, -92.30000000000001) rectangle (221.08999999999997,-94.4);
\draw(221.18999999999997, -92.30000000000001) node[anchor=north west,align=left] {Formal \\ power \\ series rings};
\draw (221.18999999999997, -92.30000000000001) rectangle (223.53999999999996,-93.9);
\draw(223.64, -92.30000000000001) node[anchor=north west,align=left] {Seminormal\\ rings};
\draw (223.64, -92.30000000000001) rectangle (225.98999999999998,-93.4);
\draw(226.08999999999997, -92.30000000000001) node[anchor=north west,align=left] {Cluster\\ algebras};
\draw (226.08999999999997, -92.30000000000001) rectangle (228.43999999999997,-93.4);
\draw(228.54, -92.30000000000001) node[anchor=north west,align=left] {Valuation\\ rings};
\draw (228.54, -92.30000000000001) rectangle (230.64,-93.4);
\draw(230.73999999999998, -92.30000000000001) node[anchor=north west,align=left] {Excellent\\ rings};
\draw (230.73999999999998, -92.30000000000001) rectangle (232.83999999999997,-93.4);
\draw(196.07999999999998, -89.7) node[anchor=north west,align=left] {\large Computational aspects and applications of commutative rings};
\draw (196.07999999999998, -89.7) rectangle (217.07999999999998,-93.9);
\draw(197.07999999999998, -90.7) node[anchor=north west,align=left] {Gröbner bases;\\ other bases for\\ ideals and \\ modules (e.g., Janet\\ and border bases)};
\draw (197.07999999999998, -90.7) rectangle (202.17999999999998,-93.3);
\draw(202.27999999999997, -90.7) node[anchor=north west,align=left] {Applications of\\ commutative \\ algebra (e.g., to\\ statistics, \\ control theory, \\ optimization, etc.)};
\draw (202.27999999999997, -90.7) rectangle (206.87999999999997,-93.8);
\draw(206.98, -90.7) node[anchor=north west,align=left] {Polynomials,\\ factorization\\ in \\ commutative rings};
\draw (206.98, -90.7) rectangle (210.57999999999998,-92.8);
\draw(210.67999999999998, -90.7) node[anchor=north west,align=left] {Computational\\ homological\\ algebra};
\draw (210.67999999999998, -90.7) rectangle (214.02999999999997,-92.3);
\draw(214.13, -90.7) node[anchor=north west,align=left] {Solving \\ polynomial\\ systems;\\ resultants};
\draw (214.13, -90.7) rectangle (216.98,-92.8);
\draw(196.07999999999998, -94.60000000000001) node[anchor=north west,align=left] {\large Theory of modules and ideals in commutative rings};
\draw (196.07999999999998, -94.60000000000001) rectangle (214.13,-104.2);
\draw(197.07999999999998, -95.60000000000001) node[anchor=north west,align=left] {Structure, \\ classification \\ theorems for modules\\ and ideals in\\ commutative rings};
\draw (197.07999999999998, -95.60000000000001) rectangle (201.67999999999998,-98.2);
\draw(201.77999999999997, -95.60000000000001) node[anchor=north west,align=left] {Theory of modules\\ and ideals\\ in commutative\\ rings described\\ by combinatorial\\ properties};
\draw (201.77999999999997, -95.60000000000001) rectangle (206.12999999999997,-98.7);
\draw(206.23, -95.60000000000001) node[anchor=north west,align=left] {Projective\\ and free \\ modules and \\ ideals in \\ commutative rings};
\draw (206.23, -95.60000000000001) rectangle (210.07999999999998,-98.2);
\draw(210.17999999999998, -95.60000000000001) node[anchor=north west,align=left] {Cohen-Macaulay\\ modules};
\draw (210.17999999999998, -95.60000000000001) rectangle (214.02999999999997,-97.2);
\draw(197.07999999999998, -98.80000000000001) node[anchor=north west,align=left] {Dimension \\ theory, depth, \\ related commutative\\ rings \\ (catenary, etc.)};
\draw (197.07999999999998, -98.80000000000001) rectangle (200.92999999999998,-101.4);
\draw(201.02999999999997, -98.80000000000001) node[anchor=north west,align=left] {Injective \\ and flat \\ modules and \\ ideals in \\ commutative rings};
\draw (201.02999999999997, -98.80000000000001) rectangle (204.62999999999997,-101.4);
\draw(204.73, -98.80000000000001) node[anchor=north west,align=left] {Torsion \\ modules and \\ ideals in \\ commutative rings};
\draw (204.73, -98.80000000000001) rectangle (208.32999999999998,-100.9);
\draw(208.42999999999998, -98.80000000000001) node[anchor=north west,align=left] {Other special\\ types of \\ modules and \\ ideals in \\ commutative rings};
\draw (208.42999999999998, -98.80000000000001) rectangle (212.02999999999997,-101.4);
\draw(212.13, -98.80000000000001) node[anchor=north west,align=left] {Class\\ groups};
\draw (212.13, -98.80000000000001) rectangle (213.98,-99.9);
\draw(197.07999999999998, -101.50000000000001) node[anchor=north west,align=left] {Linkage, \\ complete \\ intersections \\ and determinantal\\ ideals};
\draw (197.07999999999998, -101.50000000000001) rectangle (200.17999999999998,-104.10000000000001);
\draw(200.27999999999997, -101.50000000000001) node[anchor=north west,align=left] {Module \\ categories\\ and \\ commutative rings};
\draw (200.27999999999997, -101.50000000000001) rectangle (203.37999999999997,-103.60000000000001);
\draw(214.23, -94.60000000000001) node[anchor=north west,align=left] {\large Homological methods in commutative ring theory};
\draw (214.23, -94.60000000000001) rectangle (231.28,-103.7);
\draw(215.23, -95.60000000000001) node[anchor=north west,align=left] {(Co)homology of \\ commutative rings\\ and algebras (e.g.,\\ Hochschild, \\ André-Quillen, cyclic,\\ dihedral, etc.)};
\draw (215.23, -95.60000000000001) rectangle (220.07999999999998,-98.7);
\draw(220.17999999999998, -95.60000000000001) node[anchor=north west,align=left] {Homological \\ conjectures \\ (intersection \\ theorems) in \\ commutative ring theory};
\draw (220.17999999999998, -95.60000000000001) rectangle (224.77999999999997,-98.2);
\draw(224.88, -95.60000000000001) node[anchor=north west,align=left] {Homological \\ functors on \\ modules of \\ commutative rings\\ (Tor, Ext, etc.)};
\draw (224.88, -95.60000000000001) rectangle (229.23,-98.2);
\draw(215.23, -98.80000000000001) node[anchor=north west,align=left] {Deformations\\ and infinitesimal\\ methods\\ in commutative\\ ring theory};
\draw (215.23, -98.80000000000001) rectangle (219.57999999999998,-101.4);
\draw(219.67999999999998, -98.80000000000001) node[anchor=north west,align=left] {Syzygies, \\ resolutions, \\ complexes and \\ commutative rings};
\draw (219.67999999999998, -98.80000000000001) rectangle (223.52999999999997,-100.9);
\draw(223.63, -98.80000000000001) node[anchor=north west,align=left] {Grothendieck\\ groups, \(K\)-theory\\ and \\ commutative rings};
\draw (223.63, -98.80000000000001) rectangle (227.48,-100.9);
\draw(227.57999999999998, -98.80000000000001) node[anchor=north west,align=left] {Hilbert-Samuel\\ and \\ Hilbert-Kunz \\ functions; \\ Poincaré series};
\draw (227.57999999999998, -98.80000000000001) rectangle (231.17999999999998,-101.4);
\draw(215.23, -101.50000000000001) node[anchor=north west,align=left] {Homological\\ dimension\\ and \\ commutative rings};
\draw (215.23, -101.50000000000001) rectangle (218.57999999999998,-103.60000000000001);
\draw(218.67999999999998, -101.50000000000001) node[anchor=north west,align=left] {Derived \\ categories\\ and \\ commutative rings};
\draw (218.67999999999998, -101.50000000000001) rectangle (222.02999999999997,-103.60000000000001);
\draw(222.13, -101.50000000000001) node[anchor=north west,align=left] {Torsion\\ theory \\ for \\ commutative rings};
\draw (222.13, -101.50000000000001) rectangle (225.23,-103.60000000000001);
\draw(225.32999999999998, -101.50000000000001) node[anchor=north west,align=left] {Local \\ cohomology \\ and commutative\\ rings};
\draw (225.32999999999998, -101.50000000000001) rectangle (228.17999999999998,-103.60000000000001);
\draw(231.38, -94.60000000000001) node[anchor=north west,align=left] {\large Integral domains};
\draw (231.38, -94.60000000000001) rectangle (236.94,-96.80000000000001);
\draw(232.38, -95.60000000000001) node[anchor=north west,align=left] {Integral\\ domains};
\draw (232.38, -95.60000000000001) rectangle (234.73,-96.7);
\draw(231.38, -96.9) node[anchor=north west,align=left] {\large History of \\ commutative algebra};
\draw (231.38, -96.9) rectangle (236.32,-98.0);
\draw(196.07999999999998, -104.30000000000001) node[anchor=north west,align=left] {\large Commutative ring extensions and related topics};
\draw (196.07999999999998, -104.30000000000001) rectangle (211.82999999999998,-110.20000000000002);
\draw(197.07999999999998, -105.30000000000001) node[anchor=north west,align=left] {Integral \\ closure of \\ commutative \\ rings and ideals};
\draw (197.07999999999998, -105.30000000000001) rectangle (200.92999999999998,-107.4);
\draw(201.02999999999997, -105.30000000000001) node[anchor=north west,align=left] {Integral \\ dependence in \\ commutative \\ rings; going\\ up, going down};
\draw (201.02999999999997, -105.30000000000001) rectangle (204.62999999999997,-107.9);
\draw(204.73, -105.30000000000001) node[anchor=north west,align=left] {Rings of \\ fractions and\\ localization\\ for \\ commutative rings};
\draw (204.73, -105.30000000000001) rectangle (208.07999999999998,-107.9);
\draw(208.17999999999998, -105.30000000000001) node[anchor=north west,align=left] {Étale and \\ flat extensions;\\ Henselization;\\ Artin\\ approximation};
\draw (208.17999999999998, -105.30000000000001) rectangle (211.52999999999997,-107.9);
\draw(197.07999999999998, -108.00000000000001) node[anchor=north west,align=left] {Galois theory\\ and \\ commutative ring\\ extensions};
\draw (197.07999999999998, -108.00000000000001) rectangle (200.17999999999998,-110.10000000000001);
\draw(200.27999999999997, -108.00000000000001) node[anchor=north west,align=left] {Morphisms\\ of \\ commutative rings};
\draw (200.27999999999997, -108.00000000000001) rectangle (203.37999999999997,-109.60000000000001);
\draw(203.48, -108.00000000000001) node[anchor=north west,align=left] {Completion\\ of commutative\\ rings};
\draw (203.48, -108.00000000000001) rectangle (206.32999999999998,-109.60000000000001);
\draw(206.42999999999998, -108.00000000000001) node[anchor=north west,align=left] {Extension\\ theory \\ of commutative\\ rings};
\draw (206.42999999999998, -108.00000000000001) rectangle (209.02999999999997,-110.10000000000001);
\draw(209.13, -108.00000000000001) node[anchor=north west,align=left] {Polynomials\\ over\\ commutative\\ rings};
\draw (209.13, -108.00000000000001) rectangle (211.73,-110.10000000000001);
\draw(211.92999999999998, -104.30000000000001) node[anchor=north west,align=left] {\large Applications of logic to commutative algebra};
\draw (211.92999999999998, -104.30000000000001) rectangle (226.17,-107.50000000000001);
\draw(212.92999999999998, -105.30000000000001) node[anchor=north west,align=left] {Applications\\ of logic\\ to \\ commutative algebra};
\draw (212.92999999999998, -105.30000000000001) rectangle (216.52999999999997,-107.4);
\draw(226.26999999999998, -104.30000000000001) node[anchor=north west,align=left] {\large Local rings and semilocal rings};
\draw (226.26999999999998, -104.30000000000001) rectangle (236.92,-109.70000000000002);
\draw(227.26999999999998, -105.30000000000001) node[anchor=north west,align=left] {Special types\\ (Cohen-Macaulay, \\ Gorenstein, \\ Buchsbaum, etc.)};
\draw (227.26999999999998, -105.30000000000001) rectangle (233.86999999999998,-107.9);
\draw(233.96999999999997, -105.30000000000001) node[anchor=north west,align=left] {Multiplicity\\ theory\\ and \\ related topics};
\draw (233.96999999999997, -105.30000000000001) rectangle (236.81999999999996,-107.4);
\draw(227.26999999999998, -108.00000000000001) node[anchor=north west,align=left] {Regular\\ local\\ rings};
\draw (227.26999999999998, -108.00000000000001) rectangle (229.11999999999998,-109.60000000000001);
\draw(196.07999999999998, -110.30000000000001) node[anchor=north west,align=left] {\large General commutative ring theory};
\draw (196.07999999999998, -110.30000000000001) rectangle (207.92999999999998,-120.4);
\draw(197.07999999999998, -111.30000000000001) node[anchor=north west,align=left] {Characteristic \(p\)\\ methods (Frobenius\\ endomorphism)\\ and reduction\\ to characteristic\\ \(p\); tight closure};
\draw (197.07999999999998, -111.30000000000001) rectangle (202.67999999999998,-114.4);
\draw(202.77999999999997, -111.30000000000001) node[anchor=north west,align=left] {General commutative\\ ring theory and\\ combinatorics \\ (zero-divisor graphs,\\ annihilating-ideal\\ graphs, etc.)};
\draw (202.77999999999997, -111.30000000000001) rectangle (207.62999999999997,-114.4);
\draw(197.07999999999998, -114.50000000000001) node[anchor=north west,align=left] {Associated graded\\ rings of \\ ideals (Rees ring,\\ form ring), \\ analytic spread \\ and related topics};
\draw (197.07999999999998, -114.50000000000001) rectangle (201.42999999999998,-117.60000000000001);
\draw(201.52999999999997, -114.50000000000001) node[anchor=north west,align=left] {Divisibility\\ and \\ factorizations in \\ commutative rings};
\draw (201.52999999999997, -114.50000000000001) rectangle (205.62999999999997,-116.60000000000001);
\draw(205.73, -114.50000000000001) node[anchor=north west,align=left] {Graded\\ rings};
\draw (205.73, -114.50000000000001) rectangle (207.57999999999998,-115.60000000000001);
\draw(197.07999999999998, -117.70000000000002) node[anchor=north west,align=left] {Ideals and\\ multiplicative\\ ideal \\ theory in \\ commutative rings};
\draw (197.07999999999998, -117.70000000000002) rectangle (200.92999999999998,-120.30000000000001);
\draw(201.02999999999997, -117.70000000000002) node[anchor=north west,align=left] {Valuations\\ and their \\ generalizations\\ for \\ commutative rings};
\draw (201.02999999999997, -117.70000000000002) rectangle (204.62999999999997,-120.30000000000001);
\draw(204.73, -117.70000000000002) node[anchor=north west,align=left] {Actions of\\ groups on \\ commutative\\ rings; \\ invariant theory};
\draw (204.73, -117.70000000000002) rectangle (207.82999999999998,-120.30000000000001);
\draw(208.02999999999997, -110.30000000000001) node[anchor=north west,align=left] {\large Topological rings and modules};
\draw (208.02999999999997, -110.30000000000001) rectangle (219.32999999999998,-114.70000000000002);
\draw(209.02999999999997, -111.30000000000001) node[anchor=north west,align=left] {Analytical\\ algebras\\ and rings};
\draw (209.02999999999997, -111.30000000000001) rectangle (211.87999999999997,-112.9);
\draw(211.97999999999996, -111.30000000000001) node[anchor=north west,align=left] {Complete\\ rings,\\ completion};
\draw (211.97999999999996, -111.30000000000001) rectangle (214.57999999999996,-112.9);
\draw(214.67999999999998, -111.30000000000001) node[anchor=north west,align=left] {Global\\ topological\\ rings};
\draw (214.67999999999998, -111.30000000000001) rectangle (217.27999999999997,-112.9);
\draw(217.37999999999997, -111.30000000000001) node[anchor=north west,align=left] {Ordered\\ rings};
\draw (217.37999999999997, -111.30000000000001) rectangle (219.22999999999996,-112.4);
\draw(209.02999999999997, -113.00000000000001) node[anchor=north west,align=left] {Power\\ series\\ rings};
\draw (209.02999999999997, -113.00000000000001) rectangle (211.12999999999997,-114.60000000000001);
\draw(211.22999999999996, -113.00000000000001) node[anchor=north west,align=left] {Henselian\\ rings};
\draw (211.22999999999996, -113.00000000000001) rectangle (213.32999999999996,-114.10000000000001);
\draw(213.42999999999998, -113.00000000000001) node[anchor=north west,align=left] {Real\\ algebra};
\draw (213.42999999999998, -113.00000000000001) rectangle (215.27999999999997,-114.10000000000001);
\draw(208.02999999999997, -114.80000000000001) node[anchor=north west,align=left] {\large Finite commutative rings};
\draw (208.02999999999997, -114.80000000000001) rectangle (216.06999999999996,-118.00000000000001);
\draw(209.02999999999997, -115.80000000000001) node[anchor=north west,align=left] {Polynomials\\ and \\ finite \\ commutative rings};
\draw (209.02999999999997, -115.80000000000001) rectangle (212.37999999999997,-117.9);
\draw(212.47999999999996, -115.80000000000001) node[anchor=north west,align=left] {Structure\\ of finite\\ commutative\\ rings};
\draw (212.47999999999996, -115.80000000000001) rectangle (215.07999999999996,-117.9);
\draw(219.42999999999998, -110.30000000000001) node[anchor=north west,align=left] {\large Differential algebra};
\draw (219.42999999999998, -110.30000000000001) rectangle (227.32999999999998,-115.70000000000002);
\draw(220.42999999999998, -111.30000000000001) node[anchor=north west,align=left] {Commutative\\ rings of \\ differential \\ operators and\\ their modules};
\draw (220.42999999999998, -111.30000000000001) rectangle (223.77999999999997,-113.9);
\draw(223.87999999999997, -111.30000000000001) node[anchor=north west,align=left] {Derivations\\ and \\ commutative rings};
\draw (223.87999999999997, -111.30000000000001) rectangle (227.22999999999996,-112.9);
\draw(220.42999999999998, -114.00000000000001) node[anchor=north west,align=left] {Modules\\ of \\ differentials};
\draw (220.42999999999998, -114.00000000000001) rectangle (223.02999999999997,-115.60000000000001);
\draw(241.17999999999998, -1) node[anchor=north west,align=left] {\LARGE Category theory; homological algebra};
\draw (241.17999999999998, -1) rectangle (282.43,-42.6);
\draw(242.17999999999998, -2) node[anchor=north west,align=left] {\large Homological algebra in category theory, derived categories and functors};
\draw (242.17999999999998, -2) rectangle (266.63,-11.1);
\draw(243.17999999999998, -3) node[anchor=north west,align=left] {\(A_{\infty}\)-categories,\\ relations with\\ homological\\ mirror symmetry};
\draw (243.17999999999998, -3) rectangle (250.02999999999997,-5.6);
\draw(250.12999999999997, -3) node[anchor=north west,align=left] {Chain complexes\\ (category-theoretic\\ aspects),\\ dg categories};
\draw (250.12999999999997, -3) rectangle (256.72999999999996,-5.6);
\draw(256.83, -3) node[anchor=north west,align=left] {Spectral\\ sequences,\\ hypercohomology};
\draw (256.83, -3) rectangle (261.68,-5.1);
\draw(261.78, -3) node[anchor=north west,align=left] {Ext and Tor, \\ generalizations,\\ Künneth formula\\ (category-theoretic\\ aspects)};
\draw (261.78, -3) rectangle (266.38,-5.6);
\draw(243.17999999999998, -5.7) node[anchor=north west,align=left] {Relative homological\\ algebra,\\ projective classes\\ (category-theoretic\\ aspects)};
\draw (243.17999999999998, -5.7) rectangle (247.52999999999997,-8.3);
\draw(247.62999999999997, -5.7) node[anchor=north west,align=left] {Projectives\\ and \\ injectives \\ (category-theoretic\\ aspects)};
\draw (247.62999999999997, -5.7) rectangle (251.47999999999996,-8.3);
\draw(251.57999999999998, -5.7) node[anchor=north west,align=left] {Nonabelian\\ homological\\ algebra \\ (category-theoretic\\ aspects)};
\draw (251.57999999999998, -5.7) rectangle (255.42999999999998,-8.3);
\draw(255.52999999999997, -5.7) node[anchor=north west,align=left] {Resolutions;\\ derived \\ functors \\ (category-theoretic\\ aspects)};
\draw (255.52999999999997, -5.7) rectangle (259.13,-8.3);
\draw(259.22999999999996, -5.7) node[anchor=north west,align=left] {Homological\\ dimension \\ (category-theoretic\\ aspects)};
\draw (259.22999999999996, -5.7) rectangle (262.83,-7.800000000000001);
\draw(262.92999999999995, -5.7) node[anchor=north west,align=left] {2-groups,\\ crossed \\ modules, \\ crossed complexes};
\draw (262.92999999999995, -5.7) rectangle (266.53,-7.800000000000001);
\draw(243.17999999999998, -8.4) node[anchor=north west,align=left] {Simplicial\\ modules and\\ Dold-Kan \\ correspondence};
\draw (243.17999999999998, -8.4) rectangle (246.52999999999997,-10.5);
\draw(246.62999999999997, -8.4) node[anchor=north west,align=left] {Derived \\ categories,\\ triangulated\\ categories};
\draw (246.62999999999997, -8.4) rectangle (249.97999999999996,-10.5);
\draw(250.07999999999998, -8.4) node[anchor=north west,align=left] {Other \\ (co)homology \\ theories \\ (category-theoretic\\ aspects)};
\draw (250.07999999999998, -8.4) rectangle (253.17999999999998,-11.0);
\draw(253.27999999999997, -8.4) node[anchor=north west,align=left] {Stable\\ module\\ categories};
\draw (253.27999999999997, -8.4) rectangle (255.87999999999997,-10.0);
\draw(255.98, -8.4) node[anchor=north west,align=left] {Graph \\ complexes\\ and graph\\ homology};
\draw (255.98, -8.4) rectangle (258.58,-10.5);
\draw(266.72999999999996, -2) node[anchor=north west,align=left] {\large General theory of categories and functors};
\draw (266.72999999999996, -2) rectangle (282.33,-14.8);
\draw(267.72999999999996, -3) node[anchor=north west,align=left] {Limits and colimits\\ (products, sums,\\ directed limits,\\ pushouts, fiber\\ products, \\ equalizers, kernels, \\ ends and coends, etc.)};
\draw (267.72999999999996, -3) rectangle (273.08,-6.6);
\draw(273.17999999999995, -3) node[anchor=north west,align=left] {Factorization \\ systems, \\ substructures, quotient\\ structures, \\ congruences, amalgams};
\draw (273.17999999999995, -3) rectangle (278.28,-5.6);
\draw(278.37999999999994, -3) node[anchor=north west,align=left] {Definitions\\ and \\ generalizations\\ in theory\\ of categories};
\draw (278.37999999999994, -3) rectangle (282.22999999999996,-5.6);
\draw(267.72999999999996, -6.7) node[anchor=north west,align=left] {Categories \\ admitting limits\\ (complete \\ categories), functors\\ preserving\\ limits, completions};
\draw (267.72999999999996, -6.7) rectangle (272.83,-9.8);
\draw(272.92999999999995, -6.7) node[anchor=north west,align=left] {Adjoint functors\\ (universal\\ constructions,\\ reflective \\ subcategories, \\ Kan extensions, etc.)};
\draw (272.92999999999995, -6.7) rectangle (278.03,-9.8);
\draw(278.12999999999994, -6.7) node[anchor=north west,align=left] {Foundations,\\ relations to\\ logic and \\ deductive systems};
\draw (278.12999999999994, -6.7) rectangle (281.72999999999996,-8.8);
\draw(267.72999999999996, -9.9) node[anchor=north west,align=left] {Epimorphisms,\\ monomorphisms,\\ special classes\\ of morphisms,\\ null morphisms};
\draw (267.72999999999996, -9.9) rectangle (272.08,-12.5);
\draw(272.17999999999995, -9.9) node[anchor=north west,align=left] {Special \\ properties of \\ functors (faithful,\\ full, etc.)};
\draw (272.17999999999995, -9.9) rectangle (275.78,-12.0);
\draw(275.87999999999994, -9.9) node[anchor=north west,align=left] {Functor\\ categories,\\ comma\\ categories};
\draw (275.87999999999994, -9.9) rectangle (278.97999999999996,-12.0);
\draw(279.08, -9.9) node[anchor=north west,align=left] {Graphs, \\ diagram \\ schemes, \\ precategories};
\draw (279.08, -9.9) rectangle (281.93,-12.0);
\draw(267.72999999999996, -12.600000000000001) node[anchor=north west,align=left] {Natural \\ morphisms,\\ dinatural\\ morphisms};
\draw (267.72999999999996, -12.600000000000001) rectangle (270.58,-14.700000000000001);
\draw(270.67999999999995, -12.600000000000001) node[anchor=north west,align=left] {Graded \\ categories\\ (general)};
\draw (270.67999999999995, -12.600000000000001) rectangle (273.53,-14.200000000000001);
\draw(242.17999999999998, -11.2) node[anchor=north west,align=left] {\large Computational methods\\ for problems pertaining\\ to category theory};
\draw (242.17999999999998, -11.2) rectangle (248.98,-12.799999999999999);
\draw(242.17999999999998, -12.9) node[anchor=north west,align=left] {\large History of \\ category theory};
\draw (242.17999999999998, -12.9) rectangle (246.49999999999997,-14.0);
\draw(242.17999999999998, -14.9) node[anchor=north west,align=left] {\large Higher categories and homotopical algebra};
\draw (242.17999999999998, -14.9) rectangle (258.83,-26.200000000000003);
\draw(243.17999999999998, -15.9) node[anchor=north west,align=left] {\((\infty,~n)\)-categories\\ and \((\infty,\infty)\)-categories};
\draw (243.17999999999998, -15.9) rectangle (254.02999999999997,-18.0);
\draw(254.12999999999997, -15.9) node[anchor=north west,align=left] {2-dimensional\\ monad theory};
\draw (254.12999999999997, -15.9) rectangle (258.72999999999996,-17.5);
\draw(243.17999999999998, -18.1) node[anchor=north west,align=left] {\((\infty,1)\)-categories\\ (quasi-categories,\\ Segal\\ spaces, etc.);\\ \(\infty\)-topoi, stable\\ \(\infty\)-categories};
\draw (243.17999999999998, -18.1) rectangle (249.27999999999997,-21.200000000000003);
\draw(249.37999999999997, -18.1) node[anchor=north west,align=left] {Tricategories,\\ weak \\ \(n\)-categories, \\ coherence, \\ semi-strictification};
\draw (249.37999999999997, -18.1) rectangle (253.72999999999996,-20.700000000000003);
\draw(253.82999999999998, -18.1) node[anchor=north west,align=left] {Categories of\\ fibrations, \\ relations to \(K\)-theory,\\ relations\\ to type theory};
\draw (253.82999999999998, -18.1) rectangle (258.18,-20.700000000000003);
\draw(243.17999999999998, -21.3) node[anchor=north west,align=left] {Localizations\\ (e.g., \\ simplicial localization,\\ Bousfield\\ localization)};
\draw (243.17999999999998, -21.3) rectangle (247.52999999999997,-23.900000000000002);
\draw(247.62999999999997, -21.3) node[anchor=north west,align=left] {Categorification};
\draw (247.62999999999997, -21.3) rectangle (251.72999999999996,-22.400000000000002);
\draw(251.82999999999998, -21.3) node[anchor=north west,align=left] {Strict \\ omega-categories,\\ computads,\\ polygraphs};
\draw (251.82999999999998, -21.3) rectangle (255.17999999999998,-23.400000000000002);
\draw(255.27999999999997, -21.3) node[anchor=north west,align=left] {Homotopical\\ algebra, \\ Quillen model\\ categories,\\ derivators};
\draw (255.27999999999997, -21.3) rectangle (258.63,-23.900000000000002);
\draw(243.17999999999998, -24.0) node[anchor=north west,align=left] {2-categories,\\ bicategories,\\ double\\ categories};
\draw (243.17999999999998, -24.0) rectangle (246.27999999999997,-26.1);
\draw(246.37999999999997, -24.0) node[anchor=north west,align=left] {Simplicial\\ sets,\\ simplicial\\ objects};
\draw (246.37999999999997, -24.0) rectangle (249.22999999999996,-26.1);
\draw(249.32999999999998, -24.0) node[anchor=north west,align=left] {\(\infty\)-operads\\ and \\ higher algebra};
\draw (249.32999999999998, -24.0) rectangle (252.17999999999998,-25.6);
\draw(258.92999999999995, -14.9) node[anchor=north west,align=left] {\large Categories in geometry and topology};
\draw (258.92999999999995, -14.9) rectangle (272.28,-26.200000000000003);
\draw(259.92999999999995, -15.9) node[anchor=north west,align=left] {Presheaves and\\ sheaves, stacks,\\ descent \\ conditions \\ (category-theoretic aspects)};
\draw (259.92999999999995, -15.9) rectangle (264.78,-18.5);
\draw(264.87999999999994, -15.9) node[anchor=north west,align=left] {Abstract \\ manifolds and \\ fiber bundles\\ (category-theoretic\\ aspects)};
\draw (264.87999999999994, -15.9) rectangle (269.22999999999996,-18.5);
\draw(269.3299999999999, -15.9) node[anchor=north west,align=left] {Local \\ categories \\ and functors};
\draw (269.3299999999999, -15.9) rectangle (272.17999999999995,-17.5);
\draw(259.92999999999995, -18.6) node[anchor=north west,align=left] {Algebraic \\ \(K\)-theory and\\ \(L\)-theory \\ (category-theoretic\\ aspects)};
\draw (259.92999999999995, -18.6) rectangle (264.03,-21.200000000000003);
\draw(264.12999999999994, -18.6) node[anchor=north west,align=left] {Synthetic \\ differential geometry,\\ tangent \\ categories, differential\\ categories};
\draw (264.12999999999994, -18.6) rectangle (268.22999999999996,-21.200000000000003);
\draw(268.3299999999999, -18.6) node[anchor=north west,align=left] {Grothendieck\\ groups \\ (category-theoretic\\ aspects)};
\draw (268.3299999999999, -18.6) rectangle (272.17999999999995,-20.700000000000003);
\draw(259.92999999999995, -21.3) node[anchor=north west,align=left] {Frames and \\ locales, pointfree\\ topology,\\ Stone duality};
\draw (259.92999999999995, -21.3) rectangle (263.53,-23.400000000000002);
\draw(263.62999999999994, -21.3) node[anchor=north west,align=left] {Grothendieck\\ topologies\\ and \\ Grothendieck topoi};
\draw (263.62999999999994, -21.3) rectangle (266.97999999999996,-23.400000000000002);
\draw(267.0799999999999, -21.3) node[anchor=north west,align=left] {Goodwillie\\ calculus\\ and \\ functor calculus};
\draw (267.0799999999999, -21.3) rectangle (270.42999999999995,-23.400000000000002);
\draw(259.92999999999995, -23.5) node[anchor=north west,align=left] {Categories\\ of topological\\ spaces\\ and continuous\\ mappings};
\draw (259.92999999999995, -23.5) rectangle (263.03,-26.1);
\draw(263.12999999999994, -23.5) node[anchor=north west,align=left] {Quantales};
\draw (263.12999999999994, -23.5) rectangle (265.47999999999996,-24.6);
\draw(272.38, -14.9) node[anchor=north west,align=left] {\large Categories and theories};
\draw (272.38, -14.9) rectangle (282.28,-26.200000000000003);
\draw(273.38, -15.9) node[anchor=north west,align=left] {Monads (= standard\\ construction, \\ triple or triad), \\ algebras for monads,\\ homology and derived\\ functors for monads};
\draw (273.38, -15.9) rectangle (278.98,-19.0);
\draw(279.08, -15.9) node[anchor=north west,align=left] {Accessible\\ and locally\\ presentable\\ categories};
\draw (279.08, -15.9) rectangle (282.18,-18.0);
\draw(273.38, -19.1) node[anchor=north west,align=left] {Eilenberg-Moore\\ and Kleisli\\ constructions\\ for monads};
\draw (273.38, -19.1) rectangle (277.48,-21.200000000000003);
\draw(277.58, -19.1) node[anchor=north west,align=left] {Equational\\ categories};
\draw (277.58, -19.1) rectangle (281.43,-20.700000000000003);
\draw(273.38, -21.3) node[anchor=north west,align=left] {Structured \\ objects in a \\ category (group\\ objects, etc.)};
\draw (273.38, -21.3) rectangle (277.23,-23.400000000000002);
\draw(277.33, -21.3) node[anchor=north west,align=left] {Theories \\ (e.g., algebraic\\ theories),\\ structure,\\ and semantics};
\draw (277.33, -21.3) rectangle (280.68,-23.900000000000002);
\draw(273.38, -24.0) node[anchor=north west,align=left] {Categorical\\ semantics\\ of formal\\ languages};
\draw (273.38, -24.0) rectangle (276.23,-26.1);
\draw(276.33, -24.0) node[anchor=north west,align=left] {Sketches\\ and \\ generalizations};
\draw (276.33, -24.0) rectangle (278.93,-25.6);
\draw(242.17999999999998, -26.3) node[anchor=north west,align=left] {\large Monoidal categories and operads};
\draw (242.17999999999998, -26.3) rectangle (255.07999999999998,-42.5);
\draw(243.17999999999998, -27.3) node[anchor=north west,align=left] {Polycategories/dioperads,\\ properads, PROPs,\\ cyclic operads,\\ modular operads};
\draw (243.17999999999998, -27.3) rectangle (251.52999999999997,-29.900000000000002);
\draw(251.62999999999997, -27.3) node[anchor=north west,align=left] {Braided \\ monoidal categories\\ and ribbon\\ categories};
\draw (251.62999999999997, -27.3) rectangle (254.97999999999996,-29.400000000000002);
\draw(243.17999999999998, -30.0) node[anchor=north west,align=left] {Traced monoidal\\ categories, compact\\ closed \\ categories, star-autonomous\\ categories};
\draw (243.17999999999998, -30.0) rectangle (247.52999999999997,-32.6);
\draw(247.62999999999997, -30.0) node[anchor=north west,align=left] {Categories of\\ networks and\\ processes, \\ compositionality};
\draw (247.62999999999997, -30.0) rectangle (251.47999999999996,-32.1);
\draw(251.57999999999998, -30.0) node[anchor=north west,align=left] {Bimonoidal,\\ skew-monoidal,\\ duoidal\\ categories};
\draw (251.57999999999998, -30.0) rectangle (254.67999999999998,-32.1);
\draw(243.17999999999998, -32.7) node[anchor=north west,align=left] {Dagger \\ categories, \\ categorical \\ quantum mechanics};
\draw (243.17999999999998, -32.7) rectangle (247.02999999999997,-34.800000000000004);
\draw(247.12999999999997, -32.7) node[anchor=north west,align=left] {Non-symmetric\\ operads, \\ multicategories,\\ generalized\\ multicategories};
\draw (247.12999999999997, -32.7) rectangle (250.97999999999996,-35.300000000000004);
\draw(251.07999999999998, -32.7) node[anchor=north west,align=left] {Algebraic \\ operads, \\ cooperads, and\\ Koszul duality};
\draw (251.07999999999998, -32.7) rectangle (254.92999999999998,-34.800000000000004);
\draw(243.17999999999998, -35.400000000000006) node[anchor=north west,align=left] {Monoidal \\ categories, \\ symmetric monoidal\\ categories};
\draw (243.17999999999998, -35.400000000000006) rectangle (246.77999999999997,-37.50000000000001);
\draw(246.87999999999997, -35.400000000000006) node[anchor=north west,align=left] {Fusion \\ categories, modular\\ tensor \\ categories, \\ modular functors};
\draw (246.87999999999997, -35.400000000000006) rectangle (250.47999999999996,-38.00000000000001);
\draw(250.57999999999998, -35.400000000000006) node[anchor=north west,align=left] {Tannakian\\ categories};
\draw (250.57999999999998, -35.400000000000006) rectangle (254.17999999999998,-37.00000000000001);
\draw(243.17999999999998, -38.1) node[anchor=north west,align=left] {Species, \\ Hopf monoids,\\ operads in\\ combinatorics};
\draw (243.17999999999998, -38.1) rectangle (246.77999999999997,-40.2);
\draw(246.87999999999997, -38.1) node[anchor=north west,align=left] {Categorical\\ aspects\\ of \\ linear logic};
\draw (246.87999999999997, -38.1) rectangle (249.72999999999996,-40.2);
\draw(249.82999999999998, -38.1) node[anchor=north west,align=left] {String \\ diagrams and\\ graphical\\ calculi};
\draw (249.82999999999998, -38.1) rectangle (252.42999999999998,-40.2);
\draw(252.52999999999997, -38.1) node[anchor=north west,align=left] {Operads\\ (general)};
\draw (252.52999999999997, -38.1) rectangle (254.87999999999997,-39.2);
\draw(243.17999999999998, -40.3) node[anchor=north west,align=left] {Topological\\ and\\ simplicial\\ operads};
\draw (243.17999999999998, -40.3) rectangle (245.77999999999997,-42.4);
\draw(245.87999999999997, -40.3) node[anchor=north west,align=left] {Globular\\ operads};
\draw (245.87999999999997, -40.3) rectangle (248.22999999999996,-41.4);
\draw(255.17999999999998, -26.3) node[anchor=north west,align=left] {\large Special categories};
\draw (255.17999999999998, -26.3) rectangle (264.08,-37.1);
\draw(256.17999999999995, -27.3) node[anchor=north west,align=left] {Categories\\ of sets,\\ characterizations};
\draw (256.17999999999995, -27.3) rectangle (261.03,-29.400000000000002);
\draw(261.13, -27.3) node[anchor=north west,align=left] {Topoi};
\draw (261.13, -27.3) rectangle (262.48,-27.900000000000002);
\draw(256.17999999999995, -29.5) node[anchor=north west,align=left] {Categories\\ of \\ spans/cospans, \\ relations,\\ or partial maps};
\draw (256.17999999999995, -29.5) rectangle (260.03,-32.1);
\draw(260.13, -29.5) node[anchor=north west,align=left] {Groupoids, \\ semigroupoids,\\ semigroups, \\ groups (viewed\\ as categories)};
\draw (260.13, -29.5) rectangle (263.98,-32.1);
\draw(256.17999999999995, -32.2) node[anchor=north west,align=left] {Extensive,\\ distributive,\\ and \\ adhesive categories};
\draw (256.17999999999995, -32.2) rectangle (260.03,-34.300000000000004);
\draw(260.13, -32.2) node[anchor=north west,align=left] {Preorders, \\ orders, domains\\ and lattices\\ (viewed\\ as categories)};
\draw (260.13, -32.2) rectangle (263.73,-34.800000000000004);
\draw(256.17999999999995, -34.900000000000006) node[anchor=north west,align=left] {Embedding\\ theorems,\\ universal\\ categories};
\draw (256.17999999999995, -34.900000000000006) rectangle (259.53,-37.00000000000001);
\draw(259.63, -34.900000000000006) node[anchor=north west,align=left] {Categories\\ of machines,\\ automata};
\draw (259.63, -34.900000000000006) rectangle (262.73,-36.50000000000001);
\draw(264.17999999999995, -26.3) node[anchor=north west,align=left] {\large Categorical structures};
\draw (264.17999999999995, -26.3) rectangle (272.8299999999999,-36.6);
\draw(265.17999999999995, -27.3) node[anchor=north west,align=left] {Closed categories\\ (closed \\ monoidal and \\ Cartesian closed\\ categories, etc.)};
\draw (265.17999999999995, -27.3) rectangle (269.53,-29.900000000000002);
\draw(269.62999999999994, -27.3) node[anchor=north west,align=left] {Enriched \\ categories \\ (over closed\\ or monoidal\\ categories)};
\draw (269.62999999999994, -27.3) rectangle (272.72999999999996,-29.900000000000002);
\draw(265.17999999999995, -30.0) node[anchor=north west,align=left] {Proarrow equipments,\\ Yoneda \\ structures, KZ \\ doctrines (lax \\ idempotent monads)};
\draw (265.17999999999995, -30.0) rectangle (269.28,-32.6);
\draw(269.37999999999994, -30.0) node[anchor=north west,align=left] {Fibered\\ categories};
\draw (269.37999999999994, -30.0) rectangle (272.72999999999996,-31.6);
\draw(265.17999999999995, -32.7) node[anchor=north west,align=left] {Profunctors \\ (= correspondences,\\ distributors,\\ modules)};
\draw (265.17999999999995, -32.7) rectangle (269.03,-34.800000000000004);
\draw(269.12999999999994, -32.7) node[anchor=north west,align=left] {Actions of a\\ monoidal \\ category, \\ tensorial strength};
\draw (269.12999999999994, -32.7) rectangle (272.72999999999996,-34.800000000000004);
\draw(265.17999999999995, -34.900000000000006) node[anchor=north west,align=left] {Internal \\ categories\\ and groupoids};
\draw (265.17999999999995, -34.900000000000006) rectangle (268.53,-36.50000000000001);
\draw(268.62999999999994, -34.900000000000006) node[anchor=north west,align=left] {Formal\\ category\\ theory};
\draw (268.62999999999994, -34.900000000000006) rectangle (270.72999999999996,-36.50000000000001);
\draw(272.92999999999995, -26.3) node[anchor=north west,align=left] {\large Categorical algebra};
\draw (272.92999999999995, -26.3) rectangle (280.8299999999999,-38.8);
\draw(273.92999999999995, -27.3) node[anchor=north west,align=left] {Protomodular\\ categories, \\ semi-abelian \\ categories, \\ Mal’tsev categories};
\draw (273.92999999999995, -27.3) rectangle (278.28,-29.900000000000002);
\draw(278.37999999999994, -27.3) node[anchor=north west,align=left] {Categorical\\ Galois\\ theory};
\draw (278.37999999999994, -27.3) rectangle (280.72999999999996,-28.900000000000002);
\draw(273.92999999999995, -30.0) node[anchor=north west,align=left] {Abelian \\ categories,\\ Grothendieck\\ categories};
\draw (273.92999999999995, -30.0) rectangle (277.28,-32.1);
\draw(277.37999999999994, -30.0) node[anchor=north west,align=left] {Localization\\ of categories,\\ calculus\\ of fractions};
\draw (277.37999999999994, -30.0) rectangle (280.72999999999996,-32.1);
\draw(273.92999999999995, -32.2) node[anchor=north west,align=left] {Definable \\ subcategories\\ and \\ connections with\\ model theory};
\draw (273.92999999999995, -32.2) rectangle (277.28,-34.800000000000004);
\draw(277.37999999999994, -32.2) node[anchor=north west,align=left] {Preadditive,\\ additive\\ categories};
\draw (277.37999999999994, -32.2) rectangle (280.47999999999996,-33.800000000000004);
\draw(273.92999999999995, -34.900000000000006) node[anchor=north west,align=left] {Categorical\\ embedding\\ theorems};
\draw (273.92999999999995, -34.900000000000006) rectangle (277.03,-36.50000000000001);
\draw(277.12999999999994, -34.900000000000006) node[anchor=north west,align=left] {Regular \\ categories,\\ Barr-exact\\ categories};
\draw (277.12999999999994, -34.900000000000006) rectangle (279.97999999999996,-37.00000000000001);
\draw(273.92999999999995, -37.1) node[anchor=north west,align=left] {Torsion\\ theories,\\ radicals};
\draw (273.92999999999995, -37.1) rectangle (276.53,-38.7);
\draw(241.17999999999998, -42.7) node[anchor=north west,align=left] {\LARGE Algebraic geometry};
\draw (241.17999999999998, -42.7) rectangle (282.03,-118.1);
\draw(242.17999999999998, -43.7) node[anchor=north west,align=left] {\large Arithmetic problems in algebraic geometry; Diophantine geometry};
\draw (242.17999999999998, -43.7) rectangle (265.92999999999995,-52.300000000000004);
\draw(243.17999999999998, -44.7) node[anchor=north west,align=left] {Zeta functions \\ and related questions\\ in algebraic\\ geometry (e.g.,\\ Birch-Swinnerton-Dyer\\ conjecture)};
\draw (243.17999999999998, -44.7) rectangle (248.02999999999997,-47.800000000000004);
\draw(248.12999999999997, -44.7) node[anchor=north west,align=left] {Universal profinite\\ groups (relationship\\ to moduli\\ spaces, projective\\ and moduli towers,\\ Galois theory)};
\draw (248.12999999999997, -44.7) rectangle (252.72999999999996,-47.800000000000004);
\draw(252.82999999999998, -44.7) node[anchor=north west,align=left] {Other \\ nonalgebraically \\ closed ground\\ fields in \\ algebraic geometry};
\draw (252.82999999999998, -44.7) rectangle (257.18,-47.300000000000004);
\draw(257.28, -44.7) node[anchor=north west,align=left] {Applications\\ to coding \\ theory and \\ cryptography of \\ arithmetic geometry};
\draw (257.28, -44.7) rectangle (261.63,-47.300000000000004);
\draw(261.72999999999996, -44.7) node[anchor=north west,align=left] {Hasse principle,\\ weak and \\ strong approximation,\\ Brauer-Manin\\ obstruction};
\draw (261.72999999999996, -44.7) rectangle (265.83,-47.300000000000004);
\draw(243.17999999999998, -47.900000000000006) node[anchor=north west,align=left] {Positive \\ characteristic\\ ground \\ fields in \\ algebraic geometry};
\draw (243.17999999999998, -47.900000000000006) rectangle (247.02999999999997,-50.50000000000001);
\draw(247.12999999999997, -47.900000000000006) node[anchor=north west,align=left] {Arithmetic\\ varieties \\ and schemes;\\ Arakelov \\ theory; heights};
\draw (247.12999999999997, -47.900000000000006) rectangle (250.72999999999996,-50.50000000000001);
\draw(250.82999999999998, -47.900000000000006) node[anchor=north west,align=left] {Finite \\ ground fields\\ in algebraic\\ geometry};
\draw (250.82999999999998, -47.900000000000006) rectangle (253.92999999999998,-50.00000000000001);
\draw(254.02999999999997, -47.900000000000006) node[anchor=north west,align=left] {Global \\ ground fields\\ in algebraic\\ geometry};
\draw (254.02999999999997, -47.900000000000006) rectangle (257.13,-50.00000000000001);
\draw(257.22999999999996, -47.900000000000006) node[anchor=north west,align=left] {Perfectoid\\ spaces and\\ mixed \\ characteristic};
\draw (257.22999999999996, -47.900000000000006) rectangle (260.33,-50.00000000000001);
\draw(260.42999999999995, -47.900000000000006) node[anchor=north west,align=left] {Local ground\\ fields\\ in algebraic\\ geometry};
\draw (260.42999999999995, -47.900000000000006) rectangle (263.28,-50.00000000000001);
\draw(263.38, -47.900000000000006) node[anchor=north west,align=left] {Rigid \\ analytic\\ geometry};
\draw (263.38, -47.900000000000006) rectangle (265.73,-49.50000000000001);
\draw(243.17999999999998, -50.6) node[anchor=north west,align=left] {Modular \\ and Shimura\\ varieties};
\draw (243.17999999999998, -50.6) rectangle (246.02999999999997,-52.2);
\draw(246.12999999999997, -50.6) node[anchor=north west,align=left] {Rational\\ points};
\draw (246.12999999999997, -50.6) rectangle (248.22999999999996,-51.7);
\draw(266.03, -43.7) node[anchor=north west,align=left] {\large Computational aspects in algebraic geometry};
\draw (266.03, -43.7) rectangle (281.83,-49.6);
\draw(267.03, -44.7) node[anchor=north west,align=left] {Effectivity, \\ complexity and\\ computational\\ aspects of \\ algebraic geometry};
\draw (267.03, -44.7) rectangle (270.88,-47.300000000000004);
\draw(270.97999999999996, -44.7) node[anchor=north west,align=left] {Computational\\ aspects of \\ higher-dimensional\\ varieties};
\draw (270.97999999999996, -44.7) rectangle (274.58,-46.800000000000004);
\draw(274.67999999999995, -44.7) node[anchor=north west,align=left] {Computational\\ algebraic\\ geometry over\\ arithmetic\\ ground fields};
\draw (274.67999999999995, -44.7) rectangle (278.28,-47.300000000000004);
\draw(278.38, -44.7) node[anchor=north west,align=left] {Computational\\ aspects\\ of algebraic\\ surfaces};
\draw (278.38, -44.7) rectangle (281.73,-46.800000000000004);
\draw(267.03, -47.400000000000006) node[anchor=north west,align=left] {Geometric \\ aspects of \\ numerical algebraic\\ geometry};
\draw (267.03, -47.400000000000006) rectangle (270.38,-49.50000000000001);
\draw(270.47999999999996, -47.400000000000006) node[anchor=north west,align=left] {Computational\\ aspects\\ of algebraic\\ curves};
\draw (270.47999999999996, -47.400000000000006) rectangle (273.33,-49.50000000000001);
\draw(273.42999999999995, -47.400000000000006) node[anchor=north west,align=left] {Computational\\ real\\ algebraic\\ geometry};
\draw (273.42999999999995, -47.400000000000006) rectangle (276.03,-49.50000000000001);
\draw(266.03, -49.7) node[anchor=north west,align=left] {\large History of \\ algebraic geometry};
\draw (266.03, -49.7) rectangle (270.65999999999997,-50.800000000000004);
\draw(242.17999999999998, -52.400000000000006) node[anchor=north west,align=left] {\large Projective and enumerative algebraic geometry};
\draw (242.17999999999998, -52.400000000000006) rectangle (258.97999999999996,-59.800000000000004);
\draw(243.17999999999998, -53.400000000000006) node[anchor=north west,align=left] {Gromov-Witten \\ invariants, quantum \\ cohomology, Gopakumar-Vafa\\ invariants,\\ Donaldson-Thomas\\ invariants \\ (algebro-geometric aspects)};
\draw (243.17999999999998, -53.400000000000006) rectangle (248.27999999999997,-57.00000000000001);
\draw(248.37999999999997, -53.400000000000006) node[anchor=north west,align=left] {Enumerative \\ problems \\ (combinatorial \\ problems) in \\ algebraic geometry};
\draw (248.37999999999997, -53.400000000000006) rectangle (252.22999999999996,-56.00000000000001);
\draw(252.32999999999998, -53.400000000000006) node[anchor=north west,align=left] {Configurations\\ and \\ arrangements of \\ linear subspaces};
\draw (252.32999999999998, -53.400000000000006) rectangle (256.18,-55.50000000000001);
\draw(256.28, -53.400000000000006) node[anchor=north west,align=left] {Classical\\ problems,\\ Schubert\\ calculus};
\draw (256.28, -53.400000000000006) rectangle (258.88,-55.50000000000001);
\draw(243.17999999999998, -57.10000000000001) node[anchor=north west,align=left] {Secant \\ varieties, tensor\\ rank, \\ varieties of\\ sums of powers};
\draw (243.17999999999998, -57.10000000000001) rectangle (246.77999999999997,-59.70000000000001);
\draw(246.87999999999997, -57.10000000000001) node[anchor=north west,align=left] {Projective\\ techniques\\ in algebraic\\ geometry};
\draw (246.87999999999997, -57.10000000000001) rectangle (250.22999999999996,-59.20000000000001);
\draw(250.32999999999998, -57.10000000000001) node[anchor=north west,align=left] {Adjunction\\ problems};
\draw (250.32999999999998, -57.10000000000001) rectangle (253.67999999999998,-58.70000000000001);
\draw(253.77999999999997, -57.10000000000001) node[anchor=north west,align=left] {Varieties\\ of \\ low degree};
\draw (253.77999999999997, -57.10000000000001) rectangle (256.13,-58.70000000000001);
\draw(259.08, -52.400000000000006) node[anchor=north west,align=left] {\large Families, fibrations in algebraic geometry};
\draw (259.08, -52.400000000000006) rectangle (275.18,-62.00000000000001);
\draw(260.08, -53.400000000000006) node[anchor=north west,align=left] {Structure \\ of families\\ (Picard-Lefschetz,\\ monodromy, etc.)};
\draw (260.08, -53.400000000000006) rectangle (266.18,-56.00000000000001);
\draw(266.28, -53.400000000000006) node[anchor=north west,align=left] {Applications of \\ vector bundles and\\ moduli spaces in\\ mathematical \\ physics (twistor \\ theory, instantons,\\ quantum field theory)};
\draw (266.28, -53.400000000000006) rectangle (271.63,-57.00000000000001);
\draw(271.72999999999996, -53.400000000000006) node[anchor=north west,align=left] {Variation \\ of Hodge \\ structures \\ (algebro-geometric\\ aspects)};
\draw (271.72999999999996, -53.400000000000006) rectangle (275.08,-56.00000000000001);
\draw(260.08, -57.10000000000001) node[anchor=north west,align=left] {Arithmetic \\ ground fields \\ (finite, local, \\ global) and \\ families or fibrations};
\draw (260.08, -57.10000000000001) rectangle (264.93,-59.70000000000001);
\draw(265.03, -57.10000000000001) node[anchor=north west,align=left] {Algebraic \\ moduli problems,\\ moduli of\\ vector bundles};
\draw (265.03, -57.10000000000001) rectangle (268.88,-59.20000000000001);
\draw(268.97999999999996, -57.10000000000001) node[anchor=north west,align=left] {Geometric\\ Langlands\\ program \\ (algebro-geometric\\ aspects)};
\draw (268.97999999999996, -57.10000000000001) rectangle (272.83,-59.70000000000001);
\draw(260.08, -59.800000000000004) node[anchor=north west,align=left] {Fibrations,\\ degenerations\\ in \\ algebraic geometry};
\draw (260.08, -59.800000000000004) rectangle (263.68,-61.900000000000006);
\draw(263.78, -59.800000000000004) node[anchor=north west,align=left] {Formal methods\\ and deformations\\ in \\ algebraic geometry};
\draw (263.78, -59.800000000000004) rectangle (267.38,-61.900000000000006);
\draw(267.47999999999996, -59.800000000000004) node[anchor=north west,align=left] {Fine and\\ coarse \\ moduli spaces};
\draw (267.47999999999996, -59.800000000000004) rectangle (270.33,-61.400000000000006);
\draw(270.43, -59.800000000000004) node[anchor=north west,align=left] {Stacks \\ and moduli\\ problems};
\draw (270.43, -59.800000000000004) rectangle (273.03000000000003,-61.400000000000006);
\draw(275.28, -52.400000000000006) node[anchor=north west,align=left] {\large Tropical geometry};
\draw (275.28, -52.400000000000006) rectangle (281.15,-64.4);
\draw(276.28, -53.400000000000006) node[anchor=north west,align=left] {Foundations\\ of tropical\\ geometry\\ and relations\\ with algebra};
\draw (276.28, -53.400000000000006) rectangle (280.13,-56.00000000000001);
\draw(276.28, -56.10000000000001) node[anchor=north west,align=left] {Combinatorial\\ aspects\\ of tropical\\ varieties};
\draw (276.28, -56.10000000000001) rectangle (279.63,-58.20000000000001);
\draw(276.28, -58.300000000000004) node[anchor=north west,align=left] {Arithmetic\\ aspects\\ of tropical\\ varieties};
\draw (276.28, -58.300000000000004) rectangle (279.63,-60.400000000000006);
\draw(276.28, -60.50000000000001) node[anchor=north west,align=left] {Applications\\ of \\ tropical geometry};
\draw (276.28, -60.50000000000001) rectangle (279.63,-62.10000000000001);
\draw(276.28, -62.2) node[anchor=north west,align=left] {Geometric\\ aspects\\ of tropical\\ varieties};
\draw (276.28, -62.2) rectangle (279.38,-64.3);
\draw(242.17999999999998, -64.5) node[anchor=north west,align=left] {\large (Co)homology theory in algebraic geometry};
\draw (242.17999999999998, -64.5) rectangle (257.78,-76.3);
\draw(243.17999999999998, -65.5) node[anchor=north west,align=left] {Differentials\\ and other special\\ sheaves; \\ D-modules; \\ Bernstein-Sato \\ ideals and polynomials};
\draw (243.17999999999998, -65.5) rectangle (248.02999999999997,-68.6);
\draw(248.12999999999997, -65.5) node[anchor=north west,align=left] {Other algebro-geometric\\ (co)homologies\\ (e.g., \\ intersection, equivariant,\\ Lawson, Deligne\\ (co)homologies)};
\draw (248.12999999999997, -65.5) rectangle (252.97999999999996,-68.6);
\draw(253.07999999999998, -65.5) node[anchor=north west,align=left] {Derived categories\\ of sheaves,\\ dg categories,\\ and related \\ constructions in\\ algebraic geometry};
\draw (253.07999999999998, -65.5) rectangle (257.68,-68.6);
\draw(243.17999999999998, -68.7) node[anchor=north west,align=left] {Homotopy \\ theory and \\ fundamental \\ groups in \\ algebraic geometry};
\draw (243.17999999999998, -68.7) rectangle (247.27999999999997,-71.3);
\draw(247.37999999999997, -68.7) node[anchor=north west,align=left] {Étale and \\ other \\ Grothendieck \\ topologies and\\ (co)homologies};
\draw (247.37999999999997, -68.7) rectangle (250.97999999999996,-71.3);
\draw(251.07999999999998, -68.7) node[anchor=north west,align=left] {Topological\\ properties\\ in \\ algebraic geometry};
\draw (251.07999999999998, -68.7) rectangle (254.67999999999998,-70.8);
\draw(254.77999999999997, -68.7) node[anchor=north west,align=left] {Sheaves \\ in algebraic\\ geometry};
\draw (254.77999999999997, -68.7) rectangle (257.63,-70.3);
\draw(243.17999999999998, -71.4) node[anchor=north west,align=left] {Vanishing\\ theorems\\ in algebraic\\ geometry};
\draw (243.17999999999998, -71.4) rectangle (246.52999999999997,-73.5);
\draw(246.62999999999997, -71.4) node[anchor=north west,align=left] {Classical \\ real and complex\\ (co)homology\\ in algebraic\\ geometry};
\draw (246.62999999999997, -71.4) rectangle (249.97999999999996,-74.0);
\draw(250.07999999999998, -71.4) node[anchor=north west,align=left] {Motivic \\ cohomology;\\ motivic \\ homotopy theory};
\draw (250.07999999999998, -71.4) rectangle (253.42999999999998,-73.5);
\draw(253.52999999999997, -71.4) node[anchor=north west,align=left] {\(p\)-adic \\ cohomology,\\ crystalline\\ cohomology};
\draw (253.52999999999997, -71.4) rectangle (256.63,-73.5);
\draw(243.17999999999998, -74.1) node[anchor=north west,align=left] {de Rham \\ cohomology \\ and algebraic\\ geometry};
\draw (243.17999999999998, -74.1) rectangle (246.02999999999997,-76.19999999999999);
\draw(246.12999999999997, -74.1) node[anchor=north west,align=left] {Brauer\\ groups\\ of schemes};
\draw (246.12999999999997, -74.1) rectangle (248.72999999999996,-75.69999999999999);
\draw(248.82999999999998, -74.1) node[anchor=north west,align=left] {Multiplier\\ ideals};
\draw (248.82999999999998, -74.1) rectangle (251.17999999999998,-75.19999999999999);
\draw(257.88, -64.5) node[anchor=north west,align=left] {\large Surfaces and higher-dimensional varieties};
\draw (257.88, -64.5) rectangle (273.43,-79.7);
\draw(258.88, -65.5) node[anchor=north west,align=left] {Relationships\\ with physics};
\draw (258.88, -65.5) rectangle (263.48,-67.1);
\draw(263.58, -65.5) node[anchor=north west,align=left] {Moduli, \\ classification: \\ analytic theory;\\ relations \\ with modular forms};
\draw (263.58, -65.5) rectangle (267.93,-68.1);
\draw(268.03, -65.5) node[anchor=north west,align=left] {Arithmetic \\ ground fields \\ for surfaces \\ or higher-dimensional\\ varieties};
\draw (268.03, -65.5) rectangle (272.38,-68.1);
\draw(258.88, -68.2) node[anchor=north west,align=left] {Vector bundles\\ on surfaces and\\ higher-dimensional\\ varieties,\\ and their moduli};
\draw (258.88, -68.2) rectangle (263.23,-70.8);
\draw(263.33, -68.2) node[anchor=north west,align=left] {Elliptic \\ surfaces, elliptic\\ or \\ Calabi-Yau fibrations};
\draw (263.33, -68.2) rectangle (267.43,-70.3);
\draw(267.53, -68.2) node[anchor=north west,align=left] {Calabi-Yau\\ manifolds \\ (algebro-geometric\\ aspects)};
\draw (267.53, -68.2) rectangle (271.38,-70.3);
\draw(258.88, -70.9) node[anchor=north west,align=left] {Topology of \\ surfaces (Donaldson\\ polynomials,\\ Seiberg-Witten\\ invariants)};
\draw (258.88, -70.9) rectangle (262.73,-73.5);
\draw(262.83, -70.9) node[anchor=north west,align=left] {Families, \\ moduli, \\ classification: \\ algebraic theory};
\draw (262.83, -70.9) rectangle (266.43,-73.0);
\draw(266.53, -70.9) node[anchor=north west,align=left] {Automorphisms\\ of surfaces\\ and \\ higher-dimensional\\ varieties};
\draw (266.53, -70.9) rectangle (270.13,-73.5);
\draw(270.23, -70.9) node[anchor=north west,align=left] {Holomorphic\\ symplectic\\ varieties,\\ hyper-Kähler\\ varieties};
\draw (270.23, -70.9) rectangle (273.33000000000004,-73.5);
\draw(258.88, -73.6) node[anchor=north west,align=left] {Singularities\\ of surfaces\\ or \\ higher-dimensional\\ varieties};
\draw (258.88, -73.6) rectangle (262.23,-76.19999999999999);
\draw(262.33, -73.6) node[anchor=north west,align=left] {Mirror \\ symmetry \\ (algebro-geometric\\ aspects)};
\draw (262.33, -73.6) rectangle (265.68,-75.69999999999999);
\draw(265.78, -73.6) node[anchor=north west,align=left] {Hypersurfaces\\ and\\ algebraic\\ geometry};
\draw (265.78, -73.6) rectangle (268.88,-75.69999999999999);
\draw(268.98, -73.6) node[anchor=north west,align=left] {Rational\\ and ruled\\ surfaces};
\draw (268.98, -73.6) rectangle (271.83000000000004,-75.19999999999999);
\draw(258.88, -76.3) node[anchor=north west,align=left] {\(3\)-folds};
\draw (258.88, -76.3) rectangle (261.73,-77.39999999999999);
\draw(261.83, -76.3) node[anchor=north west,align=left] {\(4\)-folds};
\draw (261.83, -76.3) rectangle (264.68,-77.39999999999999);
\draw(264.78, -76.3) node[anchor=north west,align=left] {\(K3\) \\ surfaces and\\ Enriques\\ surfaces};
\draw (264.78, -76.3) rectangle (267.38,-78.39999999999999);
\draw(267.48, -76.3) node[anchor=north west,align=left] {Surfaces\\ of \\ general type};
\draw (267.48, -76.3) rectangle (270.08000000000004,-77.89999999999999);
\draw(270.18, -76.3) node[anchor=north west,align=left] {\(n\)-folds\\ (\(n>4\))};
\draw (270.18, -76.3) rectangle (272.78000000000003,-77.39999999999999);
\draw(258.88, -78.5) node[anchor=north west,align=left] {Special\\ surfaces};
\draw (258.88, -78.5) rectangle (261.23,-79.6);
\draw(261.33, -78.5) node[anchor=north west,align=left] {Fano \\ varieties};
\draw (261.33, -78.5) rectangle (263.43,-79.6);
\draw(273.53, -64.5) node[anchor=north west,align=left] {\large Birational geometry};
\draw (273.53, -64.5) rectangle (281.92999999999995,-77.0);
\draw(274.53, -65.5) node[anchor=north west,align=left] {Global theory\\ and resolution\\ of singularities\\ (algebro-geometric\\ aspects)};
\draw (274.53, -65.5) rectangle (278.88,-68.1);
\draw(278.97999999999996, -65.5) node[anchor=north west,align=left] {Coverings\\ in algebraic\\ geometry};
\draw (278.97999999999996, -65.5) rectangle (281.83,-67.1);
\draw(274.53, -68.2) node[anchor=north west,align=left] {Birational\\ automorphisms,\\ Cremona\\ group and\\ generalizations};
\draw (274.53, -68.2) rectangle (278.38,-70.8);
\draw(278.47999999999996, -68.2) node[anchor=north west,align=left] {Rationality\\ questions\\ in algebraic\\ geometry};
\draw (278.47999999999996, -68.2) rectangle (281.83,-70.3);
\draw(274.53, -70.9) node[anchor=north west,align=left] {McKay\\ correspondence};
\draw (274.53, -70.9) rectangle (278.13,-72.5);
\draw(278.22999999999996, -70.9) node[anchor=north west,align=left] {Minimal model\\ program \\ (Mori theory,\\ extremal rays)};
\draw (278.22999999999996, -70.9) rectangle (281.83,-73.0);
\draw(274.53, -73.1) node[anchor=north west,align=left] {Ramification\\ problems\\ in algebraic\\ geometry};
\draw (274.53, -73.1) rectangle (277.88,-75.19999999999999);
\draw(277.97999999999996, -73.1) node[anchor=north west,align=left] {Embeddings\\ in algebraic\\ geometry};
\draw (277.97999999999996, -73.1) rectangle (281.08,-74.69999999999999);
\draw(274.53, -75.3) node[anchor=north west,align=left] {Rational\\ and \\ birational maps};
\draw (274.53, -75.3) rectangle (277.13,-76.89999999999999);
\draw(277.22999999999996, -75.3) node[anchor=north west,align=left] {Arcs and\\ motivic \\ integration};
\draw (277.22999999999996, -75.3) rectangle (279.83,-76.89999999999999);
\draw(242.17999999999998, -79.8) node[anchor=north west,align=left] {\large Real algebraic and real-analytic geometry};
\draw (242.17999999999998, -79.8) rectangle (256.97999999999996,-84.7);
\draw(243.17999999999998, -80.8) node[anchor=north west,align=left] {Real-analytic\\ and\\ semi-analytic sets};
\draw (243.17999999999998, -80.8) rectangle (248.27999999999997,-82.89999999999999);
\draw(248.37999999999997, -80.8) node[anchor=north west,align=left] {Topology\\ of real\\ algebraic\\ varieties};
\draw (248.37999999999997, -80.8) rectangle (251.47999999999996,-82.89999999999999);
\draw(251.57999999999998, -80.8) node[anchor=north west,align=left] {Semialgebraic\\ sets\\ and related\\ spaces};
\draw (251.57999999999998, -80.8) rectangle (254.17999999999998,-82.89999999999999);
\draw(254.27999999999997, -80.8) node[anchor=north west,align=left] {Nash \\ functions and\\ manifolds};
\draw (254.27999999999997, -80.8) rectangle (256.88,-82.39999999999999);
\draw(243.17999999999998, -83.0) node[anchor=north west,align=left] {Real \\ algebraic\\ sets};
\draw (243.17999999999998, -83.0) rectangle (245.02999999999997,-84.6);
\draw(257.08, -79.8) node[anchor=north west,align=left] {\large Local theory in algebraic geometry};
\draw (257.08, -79.8) rectangle (269.93,-87.39999999999999);
\draw(258.08, -80.8) node[anchor=north west,align=left] {Local structure\\ of morphisms\\ in algebraic\\ geometry:\\ étale, flat, etc.};
\draw (258.08, -80.8) rectangle (262.43,-83.39999999999999);
\draw(262.53, -80.8) node[anchor=north west,align=left] {Local deformation\\ theory,\\ Artin \\ approximation, etc.};
\draw (262.53, -80.8) rectangle (266.38,-82.89999999999999);
\draw(266.47999999999996, -80.8) node[anchor=north west,align=left] {Infinitesimal\\ methods\\ in algebraic\\ geometry};
\draw (266.47999999999996, -80.8) rectangle (269.83,-82.89999999999999);
\draw(258.08, -83.5) node[anchor=north west,align=left] {Local \\ cohomology \\ and algebraic\\ geometry};
\draw (258.08, -83.5) rectangle (261.18,-85.6);
\draw(261.28, -83.5) node[anchor=north west,align=left] {Formal \\ neighborhoods\\ in algebraic\\ geometry};
\draw (261.28, -83.5) rectangle (264.38,-85.6);
\draw(264.47999999999996, -83.5) node[anchor=north west,align=left] {Singularities\\ in\\ algebraic\\ geometry};
\draw (264.47999999999996, -83.5) rectangle (267.33,-85.6);
\draw(258.08, -85.7) node[anchor=north west,align=left] {Deformations\\ of \\ singularities};
\draw (258.08, -85.7) rectangle (260.93,-87.3);
\draw(270.03, -79.8) node[anchor=north west,align=left] {\large Abelian varieties and schemes};
\draw (270.03, -79.8) rectangle (281.13,-87.89999999999999);
\draw(271.03, -80.8) node[anchor=north west,align=left] {Analytic theory\\ of abelian \\ varieties; abelian\\ integrals\\ and differentials};
\draw (271.03, -80.8) rectangle (275.38,-83.39999999999999);
\draw(275.47999999999996, -80.8) node[anchor=north west,align=left] {Algebraic \\ moduli of abelian\\ varieties,\\ classification};
\draw (275.47999999999996, -80.8) rectangle (279.08,-82.89999999999999);
\draw(279.17999999999995, -80.8) node[anchor=north west,align=left] {Isogeny};
\draw (279.17999999999995, -80.8) rectangle (281.03,-81.89999999999999);
\draw(271.03, -83.5) node[anchor=north west,align=left] {Complex \\ multiplication\\ and \\ abelian varieties};
\draw (271.03, -83.5) rectangle (274.63,-85.6);
\draw(274.72999999999996, -83.5) node[anchor=north west,align=left] {Subvarieties\\ of abelian\\ varieties};
\draw (274.72999999999996, -83.5) rectangle (278.08,-85.1);
\draw(278.17999999999995, -83.5) node[anchor=north west,align=left] {Algebraic\\ theory \\ of abelian\\ varieties};
\draw (278.17999999999995, -83.5) rectangle (280.78,-85.6);
\draw(271.03, -85.7) node[anchor=north west,align=left] {Arithmetic\\ ground fields\\ for abelian\\ varieties};
\draw (271.03, -85.7) rectangle (274.38,-87.8);
\draw(274.47999999999996, -85.7) node[anchor=north west,align=left] {Picard \\ schemes, higher\\ Jacobians};
\draw (274.47999999999996, -85.7) rectangle (277.58,-87.3);
\draw(277.67999999999995, -85.7) node[anchor=north west,align=left] {Theta \\ functions and\\ abelian\\ varieties};
\draw (277.67999999999995, -85.7) rectangle (280.28,-87.8);
\draw(242.17999999999998, -88.0) node[anchor=north west,align=left] {\large Foundations of algebraic geometry};
\draw (242.17999999999998, -88.0) rectangle (254.32999999999998,-97.1);
\draw(243.17999999999998, -89.0) node[anchor=north west,align=left] {Generalizations\\ (algebraic \\ spaces, stacks)};
\draw (243.17999999999998, -89.0) rectangle (248.77999999999997,-91.1);
\draw(248.87999999999997, -89.0) node[anchor=north west,align=left] {Fundamental constructions\\ in algebraic\\ geometry involving\\ higher and derived\\ categories (homotopical\\ algebraic \\ geometry, derived \\ algebraic geometry, etc.)};
\draw (248.87999999999997, -89.0) rectangle (254.22999999999996,-93.1);
\draw(243.17999999999998, -91.2) node[anchor=north west,align=left] {Noncommutative\\ algebraic\\ geometry};
\draw (243.17999999999998, -91.2) rectangle (246.52999999999997,-92.8);
\draw(243.17999999999998, -93.2) node[anchor=north west,align=left] {Logarithmic\\ algebraic\\ geometry,\\ log schemes};
\draw (243.17999999999998, -93.2) rectangle (246.52999999999997,-95.3);
\draw(246.62999999999997, -93.2) node[anchor=north west,align=left] {Geometry\\ over the\\ field with\\ one element};
\draw (246.62999999999997, -93.2) rectangle (249.97999999999996,-95.3);
\draw(250.07999999999998, -93.2) node[anchor=north west,align=left] {Elementary\\ questions\\ in algebraic\\ geometry};
\draw (250.07999999999998, -93.2) rectangle (253.17999999999998,-95.3);
\draw(243.17999999999998, -95.4) node[anchor=north west,align=left] {Relevant\\ commutative\\ algebra};
\draw (243.17999999999998, -95.4) rectangle (245.77999999999997,-97.0);
\draw(245.87999999999997, -95.4) node[anchor=north west,align=left] {Varieties\\ and\\ morphisms};
\draw (245.87999999999997, -95.4) rectangle (248.22999999999996,-97.0);
\draw(248.32999999999998, -95.4) node[anchor=north west,align=left] {Schemes\\ and\\ morphisms};
\draw (248.32999999999998, -95.4) rectangle (250.67999999999998,-97.0);
\draw(254.42999999999998, -88.0) node[anchor=north west,align=left] {\large Curves in algebraic geometry};
\draw (254.42999999999998, -88.0) rectangle (265.33,-105.4);
\draw(255.42999999999998, -89.0) node[anchor=north west,align=left] {Special \\ divisors on \\ curves (gonality,\\ Brill-Noether theory)};
\draw (255.42999999999998, -89.0) rectangle (262.03,-91.6);
\draw(262.13, -89.0) node[anchor=north west,align=left] {Singularities\\ of\\ curves,\\ local rings};
\draw (262.13, -89.0) rectangle (265.23,-91.1);
\draw(255.42999999999998, -91.7) node[anchor=north west,align=left] {Automorphismsof\\ curves};
\draw (255.42999999999998, -91.7) rectangle (259.53,-93.3);
\draw(259.63, -91.7) node[anchor=north west,align=left] {Special \\ algebraic curves\\ and curves\\ of low genus};
\draw (259.63, -91.7) rectangle (263.48,-93.8);
\draw(255.42999999999998, -93.9) node[anchor=north west,align=left] {Riemann \\ surfaces; Weierstrass\\ points;\\ gap sequences};
\draw (255.42999999999998, -93.9) rectangle (259.28,-96.0);
\draw(259.38, -93.9) node[anchor=north west,align=left] {Theta \\ functions and\\ curves; \\ Schottky problem};
\draw (259.38, -93.9) rectangle (262.98,-96.0);
\draw(263.08, -93.9) node[anchor=north west,align=left] {Plane \\ and space\\ curves};
\draw (263.08, -93.9) rectangle (265.18,-95.5);
\draw(255.42999999999998, -96.1) node[anchor=north west,align=left] {Relationships\\ between \\ algebraic curves\\ and physics};
\draw (255.42999999999998, -96.1) rectangle (259.03,-98.19999999999999);
\draw(259.13, -96.1) node[anchor=north west,align=left] {Algebraic \\ functions and\\ function \\ fields in algebraic\\ geometry};
\draw (259.13, -96.1) rectangle (262.48,-98.69999999999999);
\draw(262.58, -96.1) node[anchor=north west,align=left] {Families,\\ moduli \\ of curves\\ (analytic)};
\draw (262.58, -96.1) rectangle (265.18,-98.19999999999999);
\draw(255.42999999999998, -98.8) node[anchor=north west,align=left] {Relationships\\ between \\ algebraic curves\\ and \\ integrable systems};
\draw (255.42999999999998, -98.8) rectangle (258.78,-101.39999999999999);
\draw(258.88, -98.8) node[anchor=north west,align=left] {Families,\\ moduli \\ of curves\\ (algebraic)};
\draw (258.88, -98.8) rectangle (261.73,-100.89999999999999);
\draw(261.83, -98.8) node[anchor=north west,align=left] {Arithmetic\\ ground\\ fields\\ for curves};
\draw (261.83, -98.8) rectangle (264.68,-100.89999999999999);
\draw(255.42999999999998, -101.5) node[anchor=north west,align=left] {Coverings\\ of curves,\\ fundamental\\ group};
\draw (255.42999999999998, -101.5) rectangle (258.28,-103.6);
\draw(258.38, -101.5) node[anchor=north west,align=left] {Vector \\ bundles on \\ curves and \\ their moduli};
\draw (258.38, -101.5) rectangle (261.23,-103.6);
\draw(261.33, -101.5) node[anchor=north west,align=left] {Dessins\\ d’enfants\\ theory};
\draw (261.33, -101.5) rectangle (263.93,-103.1);
\draw(255.42999999999998, -103.7) node[anchor=north west,align=left] {Jacobians,\\ Prym\\ varieties};
\draw (255.42999999999998, -103.7) rectangle (257.78,-105.3);
\draw(257.88, -103.7) node[anchor=north west,align=left] {Elliptic\\ curves};
\draw (257.88, -103.7) rectangle (259.98,-104.8);
\draw(242.17999999999998, -97.2) node[anchor=north west,align=left] {\large Affine geometry};
\draw (242.17999999999998, -97.2) rectangle (251.32999999999998,-103.60000000000001);
\draw(243.17999999999998, -98.2) node[anchor=north west,align=left] {Classification\\ of affine\\ varieties};
\draw (243.17999999999998, -98.2) rectangle (247.77999999999997,-100.3);
\draw(247.87999999999997, -98.2) node[anchor=north west,align=left] {Group actions\\ on \\ affine varieties};
\draw (247.87999999999997, -98.2) rectangle (251.22999999999996,-99.8);
\draw(243.17999999999998, -100.4) node[anchor=north west,align=left] {Affine spaces\\ (automorphisms,\\ embeddings,\\ exotic \\ structures, \\ cancellation problem)};
\draw (243.17999999999998, -100.4) rectangle (247.52999999999997,-103.5);
\draw(247.62999999999997, -100.4) node[anchor=north west,align=left] {Jacobian\\ problem};
\draw (247.62999999999997, -100.4) rectangle (249.97999999999996,-101.5);
\draw(247.62999999999997, -101.60000000000001) node[anchor=north west,align=left] {Affine \\ fibrations};
\draw (247.62999999999997, -101.60000000000001) rectangle (249.97999999999996,-102.7);
\draw(265.42999999999995, -88.0) node[anchor=north west,align=left] {\large Special varieties};
\draw (265.42999999999995, -88.0) rectangle (275.5799999999999,-102.2);
\draw(266.42999999999995, -89.0) node[anchor=north west,align=left] {Determinantalvarieties};
\draw (266.42999999999995, -89.0) rectangle (272.28,-90.6);
\draw(272.37999999999994, -89.0) node[anchor=north west,align=left] {Rationally\\ connected\\ varieties};
\draw (272.37999999999994, -89.0) rectangle (275.47999999999996,-90.6);
\draw(266.42999999999995, -90.7) node[anchor=north west,align=left] {Varieties defined\\ by ring \\ conditions (factorial,\\ Cohen-Macaulay,\\ seminormal)};
\draw (266.42999999999995, -90.7) rectangle (270.53,-93.3);
\draw(270.62999999999994, -90.7) node[anchor=north west,align=left] {Compactifications;\\ symmetric\\ and \\ spherical varieties};
\draw (270.62999999999994, -90.7) rectangle (274.72999999999996,-92.8);
\draw(266.42999999999995, -93.4) node[anchor=north west,align=left] {Complete\\ intersections};
\draw (266.42999999999995, -93.4) rectangle (270.28,-95.0);
\draw(270.37999999999994, -93.4) node[anchor=north west,align=left] {Toric varieties,\\ Newton\\ polyhedra, \\ Okounkov bodies};
\draw (270.37999999999994, -93.4) rectangle (273.97999999999996,-95.5);
\draw(266.42999999999995, -95.6) node[anchor=north west,align=left] {Supervarieties};
\draw (266.42999999999995, -95.6) rectangle (270.03,-96.69999999999999);
\draw(270.12999999999994, -95.6) node[anchor=north west,align=left] {Low codimension\\ problems\\ in algebraic\\ geometry};
\draw (270.12999999999994, -95.6) rectangle (273.47999999999996,-97.69999999999999);
\draw(273.5799999999999, -95.6) node[anchor=north west,align=left] {Linkage};
\draw (273.5799999999999, -95.6) rectangle (275.42999999999995,-96.69999999999999);
\draw(266.42999999999995, -97.8) node[anchor=north west,align=left] {Grassmannians,\\ Schubert\\ varieties, \\ flag manifolds};
\draw (266.42999999999995, -97.8) rectangle (269.78,-99.89999999999999);
\draw(269.87999999999994, -97.8) node[anchor=north west,align=left] {Character\\ varieties};
\draw (269.87999999999994, -97.8) rectangle (273.22999999999996,-99.39999999999999);
\draw(266.42999999999995, -100.0) node[anchor=north west,align=left] {Homogeneous\\ spaces\\ and \\ generalizations};
\draw (266.42999999999995, -100.0) rectangle (269.28,-102.1);
\draw(269.37999999999994, -100.0) node[anchor=north west,align=left] {Rational\\ and \\ unirational\\ varieties};
\draw (269.37999999999994, -100.0) rectangle (271.97999999999996,-102.1);
\draw(242.17999999999998, -105.5) node[anchor=north west,align=left] {\large Cycles and subschemes};
\draw (242.17999999999998, -105.5) rectangle (251.32999999999998,-118.0);
\draw(243.17999999999998, -106.5) node[anchor=north west,align=left] {Intersection \\ theory, characteristic\\ classes,\\ intersection \\ multiplicities in\\ algebraic geometry};
\draw (243.17999999999998, -106.5) rectangle (248.02999999999997,-109.6);
\draw(248.12999999999997, -106.5) node[anchor=north west,align=left] {Parametrization\\ (Chow\\ and Hilbert\\ schemes)};
\draw (248.12999999999997, -106.5) rectangle (251.22999999999996,-108.6);
\draw(243.17999999999998, -109.7) node[anchor=north west,align=left] {Transcendental\\ methods,\\ Hodge theory\\ (algebro-geometric\\ aspects)};
\draw (243.17999999999998, -109.7) rectangle (247.27999999999997,-112.3);
\draw(247.37999999999997, -109.7) node[anchor=north west,align=left] {Applications\\ of methods of\\ algebraic \\ \(K\)-theory in \\ algebraic geometry};
\draw (247.37999999999997, -109.7) rectangle (251.22999999999996,-112.3);
\draw(243.17999999999998, -112.4) node[anchor=north west,align=left] {Riemann-Roch\\ theorems};
\draw (243.17999999999998, -112.4) rectangle (247.02999999999997,-114.0);
\draw(247.12999999999997, -112.4) node[anchor=north west,align=left] {(Equivariant)\\ Chow \\ groups and\\ rings; motives};
\draw (247.12999999999997, -112.4) rectangle (250.72999999999996,-114.5);
\draw(243.17999999999998, -114.6) node[anchor=north west,align=left] {Divisors,\\ linear \\ systems, \\ invertible sheaves};
\draw (243.17999999999998, -114.6) rectangle (246.77999999999997,-116.69999999999999);
\draw(246.87999999999997, -114.6) node[anchor=north west,align=left] {Pencils, \\ nets, webs\\ in algebraic\\ geometry};
\draw (246.87999999999997, -114.6) rectangle (249.72999999999996,-116.69999999999999);
\draw(243.17999999999998, -116.8) node[anchor=north west,align=left] {Algebraic\\ cycles};
\draw (243.17999999999998, -116.8) rectangle (245.52999999999997,-117.89999999999999);
\draw(245.62999999999997, -116.8) node[anchor=north west,align=left] {Torelli\\ problem};
\draw (245.62999999999997, -116.8) rectangle (247.72999999999996,-117.89999999999999);
\draw(247.82999999999998, -116.8) node[anchor=north west,align=left] {Picard\\ groups};
\draw (247.82999999999998, -116.8) rectangle (249.67999999999998,-117.89999999999999);
\draw(251.42999999999998, -105.5) node[anchor=north west,align=left] {\large Algebraic groups};
\draw (251.42999999999998, -105.5) rectangle (259.08,-115.3);
\draw(252.42999999999998, -106.5) node[anchor=north west,align=left] {Affine algebraic\\ groups,\\ hyperalgebra\\ constructions};
\draw (252.42999999999998, -106.5) rectangle (256.28,-108.6);
\draw(256.38, -106.5) node[anchor=north west,align=left] {Geometric\\ invariant\\ theory};
\draw (256.38, -106.5) rectangle (258.98,-108.1);
\draw(252.42999999999998, -108.7) node[anchor=north west,align=left] {Classical\\ groups \\ (algebro-geometric\\ aspects)};
\draw (252.42999999999998, -108.7) rectangle (256.03,-110.8);
\draw(256.13, -108.7) node[anchor=north west,align=left] {Group \\ varieties};
\draw (256.13, -108.7) rectangle (258.23,-109.8);
\draw(252.42999999999998, -110.9) node[anchor=north west,align=left] {Formal \\ groups, \(p\)-divisible\\ groups};
\draw (252.42999999999998, -110.9) rectangle (255.77999999999997,-112.5);
\draw(255.87999999999997, -110.9) node[anchor=north west,align=left] {Other \\ algebraic groups\\ (geometric\\ aspects)};
\draw (255.87999999999997, -110.9) rectangle (258.97999999999996,-113.0);
\draw(252.42999999999998, -113.1) node[anchor=north west,align=left] {Group actions\\ on varieties\\ or schemes\\ (quotients)};
\draw (252.42999999999998, -113.1) rectangle (255.77999999999997,-115.19999999999999);
\draw(255.87999999999997, -113.1) node[anchor=north west,align=left] {Group\\ schemes};
\draw (255.87999999999997, -113.1) rectangle (257.72999999999996,-114.19999999999999);
\draw(282.53000000000003, -1) node[anchor=north west,align=left] {\LARGE Topological groups, Lie groups};
\draw (282.53000000000003, -1) rectangle (313.96000000000004,-47.0);
\draw(283.53000000000003, -2) node[anchor=north west,align=left] {\large Topological and differentiable algebraic systems};
\draw (283.53000000000003, -2) rectangle (299.83000000000004,-7.9);
\draw(284.53000000000003, -3) node[anchor=north west,align=left] {Representations\\ of \\ general topological\\ groups\\ and semigroups};
\draw (284.53000000000003, -3) rectangle (288.63000000000005,-5.6);
\draw(288.73, -3) node[anchor=north west,align=left] {Topological\\ semilattices,\\ lattices \\ and applications};
\draw (288.73, -3) rectangle (292.58000000000004,-5.1);
\draw(292.68, -3) node[anchor=north west,align=left] {Topological \\ groupoids \\ (including \\ differentiable and\\ Lie groupoids)};
\draw (292.68, -3) rectangle (296.28000000000003,-5.6);
\draw(296.38000000000005, -3) node[anchor=north west,align=left] {Other topological\\ algebraic\\ systems\\ and their \\ representations};
\draw (296.38000000000005, -3) rectangle (299.7300000000001,-5.6);
\draw(284.53000000000003, -5.7) node[anchor=north west,align=left] {Structure\\ of general\\ topological\\ groups};
\draw (284.53000000000003, -5.7) rectangle (287.63000000000005,-7.800000000000001);
\draw(287.73, -5.7) node[anchor=north west,align=left] {Analysis\\ on general\\ topological\\ groups};
\draw (287.73, -5.7) rectangle (290.58000000000004,-7.800000000000001);
\draw(290.68, -5.7) node[anchor=north west,align=left] {Structure\\ of \\ topological\\ semigroups};
\draw (290.68, -5.7) rectangle (293.53000000000003,-7.800000000000001);
\draw(293.63000000000005, -5.7) node[anchor=north west,align=left] {Analysis\\ on \\ topological\\ semigroups};
\draw (293.63000000000005, -5.7) rectangle (296.2300000000001,-7.800000000000001);
\draw(299.93, -2) node[anchor=north west,align=left] {\large Locally compact abelian groups (LCA groups)};
\draw (299.93, -2) rectangle (313.86,-5.2);
\draw(300.93, -3) node[anchor=north west,align=left] {General \\ properties and\\ structure\\ of LCA groups};
\draw (300.93, -3) rectangle (304.28000000000003,-5.1);
\draw(304.38, -3) node[anchor=north west,align=left] {Structure\\ of group \\ algebras of\\ LCA groups};
\draw (304.38, -3) rectangle (307.23,-5.1);
\draw(299.93, -5.3) node[anchor=north west,align=left] {\large Computational methods\\ for problems pertaining\\ to topological groups};
\draw (299.93, -5.3) rectangle (307.04,-6.9);
\draw(283.53000000000003, -8.0) node[anchor=north west,align=left] {\large Locally compact groups and their algebras};
\draw (283.53000000000003, -8.0) rectangle (299.08000000000004,-16.6);
\draw(284.53000000000003, -9.0) node[anchor=north west,align=left] {General \\ properties and\\ structure\\ of locally\\ compact groups};
\draw (284.53000000000003, -9.0) rectangle (288.13000000000005,-11.6);
\draw(288.23, -9.0) node[anchor=north west,align=left] {Other \\ representations\\ of locally\\ compact groups};
\draw (288.23, -9.0) rectangle (291.83000000000004,-11.1);
\draw(291.93, -9.0) node[anchor=north west,align=left] {\(C^*\)-algebras\\ and \(W^*\)-algebras\\ in relation\\ to group \\ representations};
\draw (291.93, -9.0) rectangle (295.53000000000003,-11.6);
\draw(295.63000000000005, -9.0) node[anchor=north west,align=left] {Unitary \\ representations\\ of locally \\ compact groups};
\draw (295.63000000000005, -9.0) rectangle (298.9800000000001,-11.1);
\draw(284.53000000000003, -11.7) node[anchor=north west,align=left] {Induced \\ representations\\ for locally\\ compact groups};
\draw (284.53000000000003, -11.7) rectangle (288.13000000000005,-13.799999999999999);
\draw(288.23, -11.7) node[anchor=north west,align=left] {Kazhdan’s \\ property (T), the\\ Haagerup \\ property, and \\ generalizations};
\draw (288.23, -11.7) rectangle (291.83000000000004,-14.299999999999999);
\draw(291.93, -11.7) node[anchor=north west,align=left] {Group \\ algebras of\\ locally \\ compact groups};
\draw (291.93, -11.7) rectangle (295.28000000000003,-13.799999999999999);
\draw(295.38000000000005, -11.7) node[anchor=north west,align=left] {Representations\\ of \\ group algebras};
\draw (295.38000000000005, -11.7) rectangle (298.7300000000001,-13.299999999999999);
\draw(284.53000000000003, -14.4) node[anchor=north west,align=left] {Duality \\ theorems for\\ locally \\ compact groups};
\draw (284.53000000000003, -14.4) rectangle (287.88000000000005,-16.5);
\draw(287.98, -14.4) node[anchor=north west,align=left] {Automorphism\\ groups of\\ locally \\ compact groups};
\draw (287.98, -14.4) rectangle (291.08000000000004,-16.5);
\draw(291.18, -14.4) node[anchor=north west,align=left] {Ergodic\\ theory\\ on groups};
\draw (291.18, -14.4) rectangle (293.78000000000003,-16.0);
\draw(293.88000000000005, -14.4) node[anchor=north west,align=left] {Rigidity\\ in locally\\ compact\\ groups};
\draw (293.88000000000005, -14.4) rectangle (296.4800000000001,-16.5);
\draw(299.18, -8.0) node[anchor=north west,align=left] {\large Noncompact transformation groups};
\draw (299.18, -8.0) rectangle (310.78000000000003,-12.9);
\draw(300.18, -9.0) node[anchor=north west,align=left] {General \\ theory of group\\ and \\ pseudogroup actions};
\draw (300.18, -9.0) rectangle (303.78000000000003,-11.1);
\draw(303.88, -9.0) node[anchor=north west,align=left] {Homogeneous\\ spaces};
\draw (303.88, -9.0) rectangle (307.23,-10.6);
\draw(307.33, -9.0) node[anchor=north west,align=left] {Groups as\\ automorphisms\\ of other\\ structures};
\draw (307.33, -9.0) rectangle (310.68,-11.1);
\draw(300.18, -11.2) node[anchor=north west,align=left] {Measurable\\ group\\ actions};
\draw (300.18, -11.2) rectangle (302.78000000000003,-12.799999999999999);
\draw(299.18, -13.0) node[anchor=north west,align=left] {\large Compact groups};
\draw (299.18, -13.0) rectangle (304.12,-15.2);
\draw(300.18, -14.0) node[anchor=north west,align=left] {Compact\\ groups};
\draw (300.18, -14.0) rectangle (302.28000000000003,-15.1);
\draw(299.18, -15.3) node[anchor=north west,align=left] {\large History of \\ topological groups};
\draw (299.18, -15.3) rectangle (303.81,-16.400000000000002);
\draw(283.53000000000003, -16.7) node[anchor=north west,align=left] {\large Lie groups};
\draw (283.53000000000003, -16.7) rectangle (293.18,-46.9);
\draw(284.53000000000003, -17.7) node[anchor=north west,align=left] {Representations\\ of Lie and \\ real algebraic\\ groups: algebraic\\ methods \\ (Verma modules, etc.)};
\draw (284.53000000000003, -17.7) rectangle (289.38000000000005,-20.8);
\draw(289.48, -17.7) node[anchor=north west,align=left] {Semisimple\\ Lie groups\\ and their \\ representations};
\draw (289.48, -17.7) rectangle (293.08000000000004,-19.8);
\draw(284.53000000000003, -20.9) node[anchor=north west,align=left] {Representations\\ of nilpotent and\\ solvable Lie \\ groups (special \\ orbital integrals,\\ non-type I \\ representations, etc.)};
\draw (284.53000000000003, -20.9) rectangle (289.13000000000005,-24.5);
\draw(289.23, -20.9) node[anchor=north west,align=left] {General \\ properties and \\ structure of \\ complex Lie groups};
\draw (289.23, -20.9) rectangle (293.08000000000004,-23.0);
\draw(289.23, -23.1) node[anchor=north west,align=left] {Local \\ Lie groups};
\draw (289.23, -23.1) rectangle (291.58000000000004,-24.200000000000003);
\draw(284.53000000000003, -24.6) node[anchor=north west,align=left] {Continuous\\ cohomologyof\\ Lie groups};
\draw (284.53000000000003, -24.6) rectangle (289.13000000000005,-26.700000000000003);
\draw(289.23, -24.6) node[anchor=north west,align=left] {General \\ properties and \\ structure of\\ real Lie groups};
\draw (289.23, -24.6) rectangle (293.08000000000004,-26.700000000000003);
\draw(284.53000000000003, -26.8) node[anchor=north west,align=left] {Infinite-dimensional\\ Lie \\ groups and their\\ Lie algebras:\\ general properties};
\draw (284.53000000000003, -26.8) rectangle (289.13000000000005,-29.400000000000002);
\draw(289.23, -26.8) node[anchor=north west,align=left] {Representations\\ of Lie and \\ linear algebraic\\ groups over\\ real fields: \\ analytic methods};
\draw (289.23, -26.8) rectangle (293.08000000000004,-29.900000000000002);
\draw(284.53000000000003, -30.0) node[anchor=north west,align=left] {General \\ properties and \\ structure of\\ other Lie groups};
\draw (284.53000000000003, -30.0) rectangle (288.63000000000005,-32.1);
\draw(288.73, -30.0) node[anchor=north west,align=left] {Representations\\ of Lie and\\ linear algebraic\\ groups \\ over local fields};
\draw (288.73, -30.0) rectangle (292.83000000000004,-32.6);
\draw(284.53000000000003, -32.7) node[anchor=north west,align=left] {Loop groups\\ and related\\ constructions,\\ group-theoretic\\ treatment};
\draw (284.53000000000003, -32.7) rectangle (288.63000000000005,-35.300000000000004);
\draw(288.73, -32.7) node[anchor=north west,align=left] {Applications\\ of Lie groups\\ to the sciences;\\ explicit\\ representations};
\draw (288.73, -32.7) rectangle (292.83000000000004,-35.300000000000004);
\draw(284.53000000000003, -35.4) node[anchor=north west,align=left] {Representations\\ of Lie and \\ linear algebraic\\ groups over\\ global fields\\ and adèle rings};
\draw (284.53000000000003, -35.4) rectangle (288.38000000000005,-38.5);
\draw(288.48, -35.4) node[anchor=north west,align=left] {Analysis on \\ and representations\\ of \\ infinite-dimensional\\ Lie groups};
\draw (288.48, -35.4) rectangle (292.08000000000004,-38.0);
\draw(284.53000000000003, -38.599999999999994) node[anchor=north west,align=left] {Nilpotent\\ and solvable\\ Lie groups};
\draw (284.53000000000003, -38.599999999999994) rectangle (287.88000000000005,-40.199999999999996);
\draw(287.98, -38.599999999999994) node[anchor=north west,align=left] {Structure and\\ representation\\ of the\\ Lorentz group};
\draw (287.98, -38.599999999999994) rectangle (291.33000000000004,-40.699999999999996);
\draw(284.53000000000003, -40.8) node[anchor=north west,align=left] {Geometric \\ Langlands \\ program: \\ representation-theoretic\\ aspects};
\draw (284.53000000000003, -40.8) rectangle (287.88000000000005,-43.4);
\draw(287.98, -40.8) node[anchor=north west,align=left] {Analysis\\ on real \\ and complex\\ Lie groups};
\draw (287.98, -40.8) rectangle (291.08000000000004,-42.9);
\draw(284.53000000000003, -43.5) node[anchor=north west,align=left] {Discrete \\ subgroups \\ of Lie groups};
\draw (284.53000000000003, -43.5) rectangle (287.63000000000005,-45.1);
\draw(287.73, -43.5) node[anchor=north west,align=left] {Analysis\\ on \(p\)-adic\\ Lie groups};
\draw (287.73, -43.5) rectangle (290.58000000000004,-45.1);
\draw(284.53000000000003, -45.199999999999996) node[anchor=north west,align=left] {Lie \\ algebras of\\ Lie groups};
\draw (284.53000000000003, -45.199999999999996) rectangle (287.13000000000005,-46.8);
\draw(282.53000000000003, -47.1) node[anchor=north west,align=left] {\LARGE Nonassociative rings and algebras};
\draw (282.53000000000003, -47.1) rectangle (313.18000000000006,-90.9);
\draw(283.53000000000003, -48.1) node[anchor=north west,align=left] {\large Jordan algebras (algebras, triples and pairs)};
\draw (283.53000000000003, -48.1) rectangle (299.88000000000005,-57.900000000000006);
\draw(284.53000000000003, -49.1) node[anchor=north west,align=left] {Finite-dimensional\\ structures of\\ Jordan algebras};
\draw (284.53000000000003, -49.1) rectangle (291.13000000000005,-51.2);
\draw(291.23, -49.1) node[anchor=north west,align=left] {Idempotents,\\ Peirce \\ decompositions};
\draw (291.23, -49.1) rectangle (295.83000000000004,-51.2);
\draw(295.93, -49.1) node[anchor=north west,align=left] {Associated \\ groups, \\ automorphisms of\\ Jordan algebras};
\draw (295.93, -49.1) rectangle (299.78000000000003,-51.2);
\draw(284.53000000000003, -51.300000000000004) node[anchor=north west,align=left] {Applications\\ of Jordan\\ algebras \\ to physics, etc.};
\draw (284.53000000000003, -51.300000000000004) rectangle (288.38000000000005,-53.400000000000006);
\draw(288.48, -51.300000000000004) node[anchor=north west,align=left] {Jordan \\ structures associated\\ with \\ other structures};
\draw (288.48, -51.300000000000004) rectangle (292.08000000000004,-53.400000000000006);
\draw(292.18, -51.300000000000004) node[anchor=north west,align=left] {Jordan \\ structures on \\ Banach spaces\\ and algebras};
\draw (292.18, -51.300000000000004) rectangle (295.78000000000003,-53.400000000000006);
\draw(295.88000000000005, -51.300000000000004) node[anchor=north west,align=left] {Associated\\ geometries\\ of \\ Jordan algebras};
\draw (295.88000000000005, -51.300000000000004) rectangle (299.2300000000001,-53.400000000000006);
\draw(284.53000000000003, -53.5) node[anchor=north west,align=left] {Structure\\ theory\\ for \\ Jordan algebras};
\draw (284.53000000000003, -53.5) rectangle (287.63000000000005,-55.6);
\draw(287.73, -53.5) node[anchor=north west,align=left] {Associated\\ manifolds\\ of \\ Jordan algebras};
\draw (287.73, -53.5) rectangle (290.83000000000004,-55.6);
\draw(290.93, -53.5) node[anchor=north west,align=left] {Radicals\\ in Jordan\\ algebras};
\draw (290.93, -53.5) rectangle (293.78000000000003,-55.1);
\draw(293.88000000000005, -53.5) node[anchor=north west,align=left] {Exceptional\\ Jordan\\ structures};
\draw (293.88000000000005, -53.5) rectangle (296.7300000000001,-55.1);
\draw(296.83000000000004, -53.5) node[anchor=north west,align=left] {Identities\\ and free\\ Jordan\\ structures};
\draw (296.83000000000004, -53.5) rectangle (299.43000000000006,-55.6);
\draw(284.53000000000003, -55.7) node[anchor=north west,align=left] {Simple,\\ semisimple\\ Jordan\\ algebras};
\draw (284.53000000000003, -55.7) rectangle (287.13000000000005,-57.800000000000004);
\draw(287.23, -55.7) node[anchor=north west,align=left] {Division\\ algebras\\ and Jordan\\ algebras};
\draw (287.23, -55.7) rectangle (289.83000000000004,-57.800000000000004);
\draw(289.93, -55.7) node[anchor=north west,align=left] {Super \\ structures};
\draw (289.93, -55.7) rectangle (292.28000000000003,-56.800000000000004);
\draw(299.98, -48.1) node[anchor=north west,align=left] {\large Lie algebras and Lie superalgebras};
\draw (299.98, -48.1) rectangle (313.08000000000004,-74.1);
\draw(300.98, -49.1) node[anchor=north west,align=left] {Infinite-dimensional\\ Lie \\ (super)algebras};
\draw (300.98, -49.1) rectangle (306.58000000000004,-51.2);
\draw(306.68, -49.1) node[anchor=north west,align=left] {Universal\\ enveloping\\ (super)algebras};
\draw (306.68, -49.1) rectangle (311.78000000000003,-51.2);
\draw(300.98, -51.300000000000004) node[anchor=north west,align=left] {Lie (super)algebras\\ associated\\ with other \\ structures (associative,\\ Jordan, etc.)};
\draw (300.98, -51.300000000000004) rectangle (306.08000000000004,-53.900000000000006);
\draw(306.18, -51.300000000000004) node[anchor=north west,align=left] {Exceptional\\ (super)algebras};
\draw (306.18, -51.300000000000004) rectangle (311.03000000000003,-52.900000000000006);
\draw(311.13, -51.300000000000004) node[anchor=north west,align=left] {Root\\ systems};
\draw (311.13, -51.300000000000004) rectangle (312.98,-52.400000000000006);
\draw(300.98, -54.0) node[anchor=north west,align=left] {Solvable,\\ nilpotent\\ (super)algebras};
\draw (300.98, -54.0) rectangle (305.83000000000004,-56.1);
\draw(305.93, -54.0) node[anchor=north west,align=left] {Automorphisms,\\ derivations, \\ other operators \\ for Lie algebras\\ and super algebras};
\draw (305.93, -54.0) rectangle (310.78000000000003,-56.6);
\draw(300.98, -56.7) node[anchor=north west,align=left] {Representations\\ of Lie \\ algebras and Lie\\ superalgebras,\\ analytic theory};
\draw (300.98, -56.7) rectangle (305.58000000000004,-59.300000000000004);
\draw(305.68, -56.7) node[anchor=north west,align=left] {Kac-Moody \\ (super)algebras; \\ extended affine Lie\\ algebras; \\ toroidal Lie algebras};
\draw (305.68, -56.7) rectangle (310.28000000000003,-59.300000000000004);
\draw(310.38, -56.7) node[anchor=north west,align=left] {Coadjoint\\ orbits;\\ nilpotent\\ varieties};
\draw (310.38, -56.7) rectangle (312.98,-58.800000000000004);
\draw(300.98, -59.400000000000006) node[anchor=north west,align=left] {Representations\\ of Lie algebras\\ and Lie superalgebras,\\ algebraic\\ theory (weights)};
\draw (300.98, -59.400000000000006) rectangle (305.33000000000004,-62.00000000000001);
\draw(305.43, -59.400000000000006) node[anchor=north west,align=left] {Vertex \\ operators; vertex\\ operator \\ algebras and \\ related structures};
\draw (305.43, -59.400000000000006) rectangle (309.78000000000003,-62.00000000000001);
\draw(309.88, -59.400000000000006) node[anchor=north west,align=left] {Identities,\\ free\\ Lie \\ (super)algebras};
\draw (309.88, -59.400000000000006) rectangle (312.98,-61.50000000000001);
\draw(300.98, -62.1) node[anchor=north west,align=left] {Color Lie\\ (super)algebras};
\draw (300.98, -62.1) rectangle (305.33000000000004,-63.7);
\draw(305.43, -62.1) node[anchor=north west,align=left] {Applications\\ of Lie algebras\\ and \\ superalgebras to \\ integrable systems};
\draw (305.43, -62.1) rectangle (309.53000000000003,-64.7);
\draw(309.63, -62.1) node[anchor=north west,align=left] {Cohomology\\ of Lie \\ (super)algebras};
\draw (309.63, -62.1) rectangle (312.98,-63.7);
\draw(300.98, -64.8) node[anchor=north west,align=left] {Applications\\ of Lie \\ (super)algebras\\ to physics, etc.};
\draw (300.98, -64.8) rectangle (305.08000000000004,-66.89999999999999);
\draw(305.18, -64.8) node[anchor=north west,align=left] {Structure \\ theory for Lie\\ algebras and\\ superalgebras};
\draw (305.18, -64.8) rectangle (309.03000000000003,-66.89999999999999);
\draw(309.13, -64.8) node[anchor=north west,align=left] {Quantum groups\\ (quantized \\ enveloping algebras)\\ and related\\ deformations};
\draw (309.13, -64.8) rectangle (312.98,-67.39999999999999);
\draw(300.98, -67.5) node[anchor=north west,align=left] {Yang-Baxter\\ equations\\ and Rota-Baxter\\ operators};
\draw (300.98, -67.5) rectangle (304.58000000000004,-69.6);
\draw(304.68, -67.5) node[anchor=north west,align=left] {Lie algebras\\ of vector\\ fields and\\ related \\ (super) algebras};
\draw (304.68, -67.5) rectangle (308.28000000000003,-70.1);
\draw(308.38, -67.5) node[anchor=north west,align=left] {Simple, \\ semisimple, \\ reductive \\ (super)algebras};
\draw (308.38, -67.5) rectangle (311.48,-69.6);
\draw(300.98, -70.2) node[anchor=north west,align=left] {Lie algebras\\ of \\ linear \\ algebraic groups};
\draw (300.98, -70.2) rectangle (304.08000000000004,-72.3);
\draw(304.18, -70.2) node[anchor=north west,align=left] {Homological\\ methods\\ in Lie \\ (super)algebras};
\draw (304.18, -70.2) rectangle (307.28000000000003,-72.3);
\draw(307.38, -70.2) node[anchor=north west,align=left] {Lie \\ bialgebras; Lie\\ coalgebras};
\draw (307.38, -70.2) rectangle (310.48,-71.8);
\draw(310.58000000000004, -70.2) node[anchor=north west,align=left] {Poisson\\ algebras};
\draw (310.58000000000004, -70.2) rectangle (312.93000000000006,-71.3);
\draw(300.98, -72.4) node[anchor=north west,align=left] {Modular\\ Lie \\ (super)algebras};
\draw (300.98, -72.4) rectangle (303.83000000000004,-74.0);
\draw(303.93, -72.4) node[anchor=north west,align=left] {Virasoro\\ and related\\ algebras};
\draw (303.93, -72.4) rectangle (306.78000000000003,-74.0);
\draw(306.88, -72.4) node[anchor=north west,align=left] {Hom-Lie \\ and related\\ algebras};
\draw (306.88, -72.4) rectangle (309.48,-74.0);
\draw(309.58000000000004, -72.4) node[anchor=north west,align=left] {Graded \\ Lie \\ (super)algebras};
\draw (309.58000000000004, -72.4) rectangle (312.18000000000006,-74.0);
\draw(283.53000000000003, -58.0) node[anchor=north west,align=left] {\large Other nonassociative rings and algebras};
\draw (283.53000000000003, -58.0) rectangle (298.13000000000005,-62.9);
\draw(284.53000000000003, -59.0) node[anchor=north west,align=left] {\((\gamma,~\delta)\)-rings,\\ including \\ \((1,-1)\)-rings};
\draw (284.53000000000003, -59.0) rectangle (290.38000000000005,-61.1);
\draw(290.48, -59.0) node[anchor=north west,align=left] {Lie-admissible\\ algebras};
\draw (290.48, -59.0) rectangle (294.58000000000004,-60.6);
\draw(294.68, -59.0) node[anchor=north west,align=left] {(non-Lie)\\ Hom algebras\\ and topics};
\draw (294.68, -59.0) rectangle (298.03000000000003,-60.6);
\draw(284.53000000000003, -61.2) node[anchor=north west,align=left] {Mal’tsev\\ rings \\ and algebras};
\draw (284.53000000000003, -61.2) rectangle (287.38000000000005,-62.800000000000004);
\draw(287.48, -61.2) node[anchor=north west,align=left] {Alternative\\ rings};
\draw (287.48, -61.2) rectangle (289.83000000000004,-62.300000000000004);
\draw(289.93, -61.2) node[anchor=north west,align=left] {Right \\ alternative\\ rings};
\draw (289.93, -61.2) rectangle (292.28000000000003,-62.800000000000004);
\draw(292.38000000000005, -61.2) node[anchor=north west,align=left] {Genetic\\ algebras};
\draw (292.38000000000005, -61.2) rectangle (294.7300000000001,-62.300000000000004);
\draw(283.53000000000003, -63.0) node[anchor=north west,align=left] {\large Computational methods for \\ problems pertaining to \\ nonassociative rings and algebras};
\draw (283.53000000000003, -63.0) rectangle (292.19000000000005,-64.6);
\draw(283.53000000000003, -64.7) node[anchor=north west,align=left] {\large History of \\ nonassociative \\ rings and algebras};
\draw (283.53000000000003, -64.7) rectangle (288.47,-66.3);
\draw(283.53000000000003, -74.2) node[anchor=north west,align=left] {\large General nonassociative rings};
\draw (283.53000000000003, -74.2) rectangle (294.43,-90.80000000000001);
\draw(284.53000000000003, -75.2) node[anchor=north west,align=left] {Noncommutative\\ Jordan algebras};
\draw (284.53000000000003, -75.2) rectangle (289.88000000000005,-76.8);
\draw(289.98, -75.2) node[anchor=north west,align=left] {Other \(n\)-ary\\ compositions\\ \((n~\ge~3)\)};
\draw (289.98, -75.2) rectangle (294.33000000000004,-76.8);
\draw(284.53000000000003, -76.9) node[anchor=north west,align=left] {Compositionalgebras};
\draw (284.53000000000003, -76.9) rectangle (289.63000000000005,-78.5);
\draw(289.73, -76.9) node[anchor=north west,align=left] {Power-associative\\ rings};
\draw (289.73, -76.9) rectangle (293.83000000000004,-78.5);
\draw(284.53000000000003, -78.60000000000001) node[anchor=north west,align=left] {Automorphisms,\\ derivations, \\ other operators \\ (nonassociative\\ rings and algebras)};
\draw (284.53000000000003, -78.60000000000001) rectangle (289.38000000000005,-81.2);
\draw(289.48, -78.60000000000001) node[anchor=north west,align=left] {General theory\\ of \\ nonassociative rings\\ and algebras};
\draw (289.48, -78.60000000000001) rectangle (293.33000000000004,-80.7);
\draw(284.53000000000003, -81.3) node[anchor=north west,align=left] {Nonassociative\\ algebras\\ satisfying \\ other identities};
\draw (284.53000000000003, -81.3) rectangle (288.38000000000005,-83.39999999999999);
\draw(288.48, -81.3) node[anchor=north west,align=left] {Quadratic \\ algebras (but not\\ quadratic \\ Jordan algebras)};
\draw (288.48, -81.3) rectangle (292.33000000000004,-83.39999999999999);
\draw(284.53000000000003, -83.5) node[anchor=north west,align=left] {Ternary\\ compositions};
\draw (284.53000000000003, -83.5) rectangle (288.13000000000005,-85.1);
\draw(288.23, -83.5) node[anchor=north west,align=left] {Structure\\ theory for\\ nonassociative\\ algebras};
\draw (288.23, -83.5) rectangle (291.83000000000004,-85.6);
\draw(291.93, -83.5) node[anchor=north west,align=left] {Flexible\\ algebras};
\draw (291.93, -83.5) rectangle (294.28000000000003,-84.6);
\draw(284.53000000000003, -85.7) node[anchor=north west,align=left] {Radical theory\\ (nonassociative\\ rings\\ and algebras)};
\draw (284.53000000000003, -85.7) rectangle (288.13000000000005,-87.8);
\draw(288.23, -85.7) node[anchor=north west,align=left] {Gröbner-Shirshov\\ bases \\ in nonassociative\\ algebras};
\draw (288.23, -85.7) rectangle (291.58000000000004,-87.8);
\draw(291.68, -85.7) node[anchor=north west,align=left] {Free \\ nonassociative\\ algebras};
\draw (291.68, -85.7) rectangle (294.28000000000003,-87.3);
\draw(284.53000000000003, -87.9) node[anchor=north west,align=left] {Superalgebras};
\draw (284.53000000000003, -87.9) rectangle (287.88000000000005,-89.0);
\draw(287.98, -87.9) node[anchor=north west,align=left] {Nonassociative\\ division\\ algebras};
\draw (287.98, -87.9) rectangle (291.08000000000004,-89.5);
\draw(291.18, -87.9) node[anchor=north west,align=left] {Leibniz\\ algebras};
\draw (291.18, -87.9) rectangle (293.53000000000003,-89.0);
\draw(284.53000000000003, -89.6) node[anchor=north west,align=left] {Valued\\ algebras};
\draw (284.53000000000003, -89.6) rectangle (286.63000000000005,-90.69999999999999);
\draw(314.06000000000006, -1) node[anchor=north west,align=left] {\LARGE Mathematical logic and foundations};
\draw (314.06000000000006, -1) rectangle (343.96000000000004,-73.1);
\draw(315.06000000000006, -2) node[anchor=north west,align=left] {\large Proof theory and constructive mathematics};
\draw (315.06000000000006, -2) rectangle (330.6600000000001,-12.8);
\draw(316.06000000000006, -3) node[anchor=north west,align=left] {First-order\\ arithmetic \\ and fragments};
\draw (316.06000000000006, -3) rectangle (321.1600000000001,-5.1);
\draw(321.26000000000005, -3) node[anchor=north west,align=left] {Proof-theoretic\\ aspects of\\ linear logic\\ and other \\ substructural logics};
\draw (321.26000000000005, -3) rectangle (325.86000000000007,-5.6);
\draw(325.96000000000004, -3) node[anchor=north west,align=left] {Intuitionistic\\ mathematics};
\draw (325.96000000000004, -3) rectangle (330.56000000000006,-4.6);
\draw(316.06000000000006, -5.7) node[anchor=north west,align=left] {Proof theory,\\ general\\ (including\\ proof-theoretic\\ semantics)};
\draw (316.06000000000006, -5.7) rectangle (319.9100000000001,-8.3);
\draw(320.01000000000005, -5.7) node[anchor=north west,align=left] {Provability \\ logics and related\\ algebras \\ (e.g., diagonalizable\\ algebras)};
\draw (320.01000000000005, -5.7) rectangle (323.86000000000007,-8.3);
\draw(323.96000000000004, -5.7) node[anchor=north west,align=left] {Complexity\\ of proofs};
\draw (323.96000000000004, -5.7) rectangle (327.56000000000006,-7.300000000000001);
\draw(327.6600000000001, -5.7) node[anchor=north west,align=left] {Functionals\\ in \\ proof theory};
\draw (327.6600000000001, -5.7) rectangle (330.5100000000001,-7.300000000000001);
\draw(316.06000000000006, -8.4) node[anchor=north west,align=left] {Cut-elimination\\ and\\ normal-form\\ theorems};
\draw (316.06000000000006, -8.4) rectangle (319.4100000000001,-10.5);
\draw(319.51000000000005, -8.4) node[anchor=north west,align=left] {Recursive\\ ordinals\\ and ordinal\\ notations};
\draw (319.51000000000005, -8.4) rectangle (322.86000000000007,-10.5);
\draw(322.96000000000004, -8.4) node[anchor=north west,align=left] {Relative\\ consistency\\ and \\ interpretations};
\draw (322.96000000000004, -8.4) rectangle (326.31000000000006,-10.5);
\draw(326.4100000000001, -8.4) node[anchor=north west,align=left] {Second- and\\ higher-order\\ arithmetic\\ and fragments};
\draw (326.4100000000001, -8.4) rectangle (329.7600000000001,-10.5);
\draw(316.06000000000006, -10.600000000000001) node[anchor=north west,align=left] {Gödel \\ numberings and \\ issues of \\ incompleteness};
\draw (316.06000000000006, -10.600000000000001) rectangle (319.1600000000001,-12.700000000000001);
\draw(319.26000000000005, -10.600000000000001) node[anchor=north west,align=left] {Metamathematics\\ of\\ constructive\\ systems};
\draw (319.26000000000005, -10.600000000000001) rectangle (322.36000000000007,-12.700000000000001);
\draw(322.46000000000004, -10.600000000000001) node[anchor=north west,align=left] {Other \\ constructive\\ mathematics};
\draw (322.46000000000004, -10.600000000000001) rectangle (325.56000000000006,-12.200000000000001);
\draw(325.6600000000001, -10.600000000000001) node[anchor=north west,align=left] {Constructive\\ and\\ recursive\\ analysis};
\draw (325.6600000000001, -10.600000000000001) rectangle (328.5100000000001,-12.700000000000001);
\draw(328.61000000000007, -10.600000000000001) node[anchor=north west,align=left] {Structure\\ of\\ proofs};
\draw (328.61000000000007, -10.600000000000001) rectangle (330.4600000000001,-12.200000000000001);
\draw(330.76000000000005, -2) node[anchor=north west,align=left] {\large Computability and recursion theory};
\draw (330.76000000000005, -2) rectangle (343.86000000000007,-21.400000000000002);
\draw(331.76000000000005, -3) node[anchor=north west,align=left] {Algorithmic\\ randomness \\ and dimension};
\draw (331.76000000000005, -3) rectangle (336.86000000000007,-5.1);
\draw(336.96000000000004, -3) node[anchor=north west,align=left] {Other degrees\\ and reducibilities\\ in \\ computability and\\ recursion theory};
\draw (336.96000000000004, -3) rectangle (341.56000000000006,-5.6);
\draw(331.76000000000005, -5.7) node[anchor=north west,align=left] {Automata and\\ formal \\ grammars in \\ connection with\\ logical questions};
\draw (331.76000000000005, -5.7) rectangle (336.11000000000007,-8.3);
\draw(336.21000000000004, -5.7) node[anchor=north west,align=left] {Computability\\ and recursion\\ theory on \\ ordinals, \\ admissible sets, etc.};
\draw (336.21000000000004, -5.7) rectangle (340.56000000000006,-8.3);
\draw(340.66, -5.7) node[anchor=north west,align=left] {Other \\ Turing degree\\ structures};
\draw (340.66, -5.7) rectangle (343.76000000000005,-7.300000000000001);
\draw(331.76000000000005, -8.4) node[anchor=north west,align=left] {Recursively\\ (computably)\\ enumerable\\ sets and degrees};
\draw (331.76000000000005, -8.4) rectangle (335.86000000000007,-10.5);
\draw(335.96000000000004, -8.4) node[anchor=north west,align=left] {Complexity of\\ computation \\ (including implicit\\ computational\\ complexity)};
\draw (335.96000000000004, -8.4) rectangle (339.81000000000006,-11.0);
\draw(339.91, -8.4) node[anchor=north west,align=left] {Recursive \\ functions and\\ relations,\\ subrecursive\\ hierarchies};
\draw (339.91, -8.4) rectangle (343.76000000000005,-11.0);
\draw(331.76000000000005, -11.100000000000001) node[anchor=north west,align=left] {Theory of\\ numerations,\\ effectively\\ presented\\ structures};
\draw (331.76000000000005, -11.100000000000001) rectangle (335.61000000000007,-13.700000000000001);
\draw(335.71000000000004, -11.100000000000001) node[anchor=north west,align=left] {Inductive\\ definability};
\draw (335.71000000000004, -11.100000000000001) rectangle (339.56000000000006,-12.700000000000001);
\draw(339.66, -11.100000000000001) node[anchor=north west,align=left] {Undecidability\\ and \\ degrees of sets\\ of sentences};
\draw (339.66, -11.100000000000001) rectangle (343.26000000000005,-13.200000000000001);
\draw(331.76000000000005, -13.8) node[anchor=north west,align=left] {Computation\\ over the \\ reals, \\ computable analysis};
\draw (331.76000000000005, -13.8) rectangle (335.36000000000007,-15.9);
\draw(335.46000000000004, -13.8) node[anchor=north west,align=left] {Applications\\ of computability\\ and \\ recursion theory};
\draw (335.46000000000004, -13.8) rectangle (339.06000000000006,-15.9);
\draw(339.16, -13.8) node[anchor=north west,align=left] {Word problems,\\ etc. in\\ computability\\ and \\ recursion theory};
\draw (339.16, -13.8) rectangle (342.51000000000005,-16.400000000000002);
\draw(331.76000000000005, -16.5) node[anchor=north west,align=left] {Recursive \\ equivalence\\ types of \\ sets and \\ structures, isols};
\draw (331.76000000000005, -16.5) rectangle (335.11000000000007,-19.1);
\draw(335.21000000000004, -16.5) node[anchor=north west,align=left] {Abstract and\\ axiomatic\\ computability\\ and \\ recursion theory};
\draw (335.21000000000004, -16.5) rectangle (338.56000000000006,-19.1);
\draw(338.66, -16.5) node[anchor=north west,align=left] {Hierarchies\\ of computability\\ and\\ definability};
\draw (338.66, -16.5) rectangle (341.76000000000005,-18.6);
\draw(331.76000000000005, -19.2) node[anchor=north west,align=left] {Higher-type\\ and\\ set \\ recursion theory};
\draw (331.76000000000005, -19.2) rectangle (334.86000000000007,-21.3);
\draw(334.96000000000004, -19.2) node[anchor=north west,align=left] {Thue and\\ Post \\ systems, etc.};
\draw (334.96000000000004, -19.2) rectangle (337.81000000000006,-20.8);
\draw(337.91, -19.2) node[anchor=north west,align=left] {Turing \\ machines\\ and related\\ notions};
\draw (337.91, -19.2) rectangle (340.76000000000005,-21.3);
\draw(315.06000000000006, -12.9) node[anchor=north west,align=left] {\large Philosophical aspects of logic and foundations};
\draw (315.06000000000006, -12.9) rectangle (329.9200000000001,-16.6);
\draw(316.06000000000006, -13.9) node[anchor=north west,align=left] {Philosophical\\ and \\ critical aspects\\ of logic\\ and foundations};
\draw (316.06000000000006, -13.9) rectangle (319.9100000000001,-16.5);
\draw(320.01000000000005, -13.9) node[anchor=north west,align=left] {Logic in\\ the \\ philosophy\\ of science};
\draw (320.01000000000005, -13.9) rectangle (322.61000000000007,-16.0);
\draw(315.06000000000006, -16.700000000000003) node[anchor=north west,align=left] {\large Computational methods\\ for problems \\ pertaining to mathematical\\ logic and foundations};
\draw (315.06000000000006, -16.700000000000003) rectangle (322.1700000000001,-18.800000000000004);
\draw(315.06000000000006, -18.900000000000002) node[anchor=north west,align=left] {\large History of \\ mathematical logic\\ and foundations};
\draw (315.06000000000006, -18.900000000000002) rectangle (320.31000000000006,-20.500000000000004);
\draw(315.06000000000006, -21.500000000000004) node[anchor=north west,align=left] {\large General logic};
\draw (315.06000000000006, -21.500000000000004) rectangle (326.86000000000007,-40.60000000000001);
\draw(316.06000000000006, -22.500000000000004) node[anchor=north west,align=left] {Subsystems \\ of classical\\ logic (including\\ intuitionistic logic)};
\draw (316.06000000000006, -22.500000000000004) rectangle (322.6600000000001,-25.100000000000005);
\draw(322.76000000000005, -22.500000000000004) node[anchor=north west,align=left] {Paraconsistent\\ logics};
\draw (322.76000000000005, -22.500000000000004) rectangle (326.61000000000007,-24.100000000000005);
\draw(316.06000000000006, -25.200000000000003) node[anchor=north west,align=left] {Substructural \\ logics (including \\ relevance, entailment,\\ linear logic,\\ Lambek calculus,\\ BCK and BCI logics)};
\draw (316.06000000000006, -25.200000000000003) rectangle (321.1600000000001,-28.300000000000004);
\draw(321.26000000000005, -25.200000000000003) node[anchor=north west,align=left] {Foundations\\ of classical\\ theories \\ (including \\ reverse mathematics)};
\draw (321.26000000000005, -25.200000000000003) rectangle (325.61000000000007,-27.800000000000004);
\draw(316.06000000000006, -28.400000000000006) node[anchor=north west,align=left] {Higher-order\\ logic};
\draw (316.06000000000006, -28.400000000000006) rectangle (319.4100000000001,-30.000000000000007);
\draw(319.51000000000005, -28.400000000000006) node[anchor=north west,align=left] {Decidability\\ of theories\\ and sets\\ of sentences};
\draw (319.51000000000005, -28.400000000000006) rectangle (322.86000000000007,-30.500000000000007);
\draw(322.96000000000004, -28.400000000000006) node[anchor=north west,align=left] {Mechanization\\ of proofs\\ and logical\\ operations};
\draw (322.96000000000004, -28.400000000000006) rectangle (326.31000000000006,-30.500000000000007);
\draw(316.06000000000006, -30.600000000000005) node[anchor=north west,align=left] {Intermediate\\ logics};
\draw (316.06000000000006, -30.600000000000005) rectangle (319.4100000000001,-32.2);
\draw(319.51000000000005, -30.600000000000005) node[anchor=north west,align=left] {Logics of \\ knowledge and\\ belief \\ (including \\ belief change)};
\draw (319.51000000000005, -30.600000000000005) rectangle (322.61000000000007,-33.2);
\draw(322.71000000000004, -30.600000000000005) node[anchor=north west,align=left] {Modal logic\\ (including\\ the logic\\ of norms)};
\draw (322.71000000000004, -30.600000000000005) rectangle (325.81000000000006,-32.7);
\draw(316.06000000000006, -33.300000000000004) node[anchor=north west,align=left] {Classical\\ propositional\\ logic};
\draw (316.06000000000006, -33.300000000000004) rectangle (318.9100000000001,-34.900000000000006);
\draw(319.01000000000005, -33.300000000000004) node[anchor=north west,align=left] {Classical\\ first-order\\ logic};
\draw (319.01000000000005, -33.300000000000004) rectangle (321.86000000000007,-34.900000000000006);
\draw(321.96000000000004, -33.300000000000004) node[anchor=north west,align=left] {Probability\\ and \\ inductive logic};
\draw (321.96000000000004, -33.300000000000004) rectangle (324.81000000000006,-34.900000000000006);
\draw(324.9100000000001, -33.300000000000004) node[anchor=north west,align=left] {Type\\ theory};
\draw (324.9100000000001, -33.300000000000004) rectangle (326.5100000000001,-34.400000000000006);
\draw(316.06000000000006, -35.0) node[anchor=north west,align=left] {Fuzzy logic;\\ logic \\ of vagueness};
\draw (316.06000000000006, -35.0) rectangle (318.9100000000001,-36.6);
\draw(319.01000000000005, -35.0) node[anchor=north west,align=left] {Other \\ applications\\ of logic};
\draw (319.01000000000005, -35.0) rectangle (321.86000000000007,-36.6);
\draw(321.96000000000004, -35.0) node[anchor=north west,align=left] {Abstract\\ deductive\\ systems};
\draw (321.96000000000004, -35.0) rectangle (324.56000000000006,-36.6);
\draw(324.6600000000001, -35.0) node[anchor=north west,align=left] {Temporal\\ logic};
\draw (324.6600000000001, -35.0) rectangle (326.7600000000001,-36.1);
\draw(316.06000000000006, -36.7) node[anchor=north west,align=left] {Combinatory\\ logic\\ and lambda\\ calculus};
\draw (316.06000000000006, -36.7) rectangle (318.6600000000001,-38.800000000000004);
\draw(318.76000000000005, -36.7) node[anchor=north west,align=left] {Other \\ nonclassical\\ logic};
\draw (318.76000000000005, -36.7) rectangle (321.36000000000007,-38.300000000000004);
\draw(321.46000000000004, -36.7) node[anchor=north west,align=left] {Logic of\\ natural\\ languages};
\draw (321.46000000000004, -36.7) rectangle (324.06000000000006,-38.300000000000004);
\draw(324.1600000000001, -36.7) node[anchor=north west,align=left] {Many-valued\\ logic};
\draw (324.1600000000001, -36.7) rectangle (326.5100000000001,-37.800000000000004);
\draw(316.06000000000006, -38.900000000000006) node[anchor=north west,align=left] {Logic in\\ computer\\ science};
\draw (316.06000000000006, -38.900000000000006) rectangle (318.4100000000001,-40.50000000000001);
\draw(318.51000000000005, -38.900000000000006) node[anchor=north west,align=left] {Combined\\ logics};
\draw (318.51000000000005, -38.900000000000006) rectangle (320.61000000000007,-40.00000000000001);
\draw(326.96000000000004, -21.500000000000004) node[anchor=north west,align=left] {\large Model theory};
\draw (326.96000000000004, -21.500000000000004) rectangle (337.11,-50.400000000000006);
\draw(327.96000000000004, -22.500000000000004) node[anchor=north west,align=left] {Interpolation,\\ preservation, \\ definability};
\draw (327.96000000000004, -22.500000000000004) rectangle (333.31000000000006,-24.600000000000005);
\draw(333.41, -22.500000000000004) node[anchor=north west,align=left] {Equational \\ classes, universal\\ algebra \\ in model theory};
\draw (333.41, -22.500000000000004) rectangle (337.01000000000005,-24.600000000000005);
\draw(327.96000000000004, -24.700000000000003) node[anchor=north west,align=left] {Set-theoretic\\ model theory};
\draw (327.96000000000004, -24.700000000000003) rectangle (332.56000000000006,-26.300000000000004);
\draw(332.66, -24.700000000000003) node[anchor=north west,align=left] {Classification\\ theory, \\ stability and \\ related concepts\\ in model theory};
\draw (332.66, -24.700000000000003) rectangle (337.01000000000005,-27.300000000000004);
\draw(327.96000000000004, -27.400000000000006) node[anchor=north west,align=left] {Model theory\\ of denumerable\\ and \\ separable structures};
\draw (327.96000000000004, -27.400000000000006) rectangle (332.06000000000006,-29.500000000000007);
\draw(332.16, -27.400000000000006) node[anchor=north west,align=left] {Model-theoretic\\ forcing};
\draw (332.16, -27.400000000000006) rectangle (336.26000000000005,-29.000000000000007);
\draw(327.96000000000004, -29.600000000000005) node[anchor=north west,align=left] {Computable \\ structure \\ theory, computable\\ model theory};
\draw (327.96000000000004, -29.600000000000005) rectangle (332.06000000000006,-31.700000000000006);
\draw(332.16, -29.600000000000005) node[anchor=north west,align=left] {Model-theoretic\\ algebra};
\draw (332.16, -29.600000000000005) rectangle (336.26000000000005,-31.200000000000006);
\draw(327.96000000000004, -31.800000000000004) node[anchor=north west,align=left] {Models with\\ special\\ properties\\ (saturated,\\ rigid, etc.)};
\draw (327.96000000000004, -31.800000000000004) rectangle (331.81000000000006,-34.400000000000006);
\draw(331.91, -31.800000000000004) node[anchor=north west,align=left] {Model theory\\ of ordered\\ structures;\\ o-minimality};
\draw (331.91, -31.800000000000004) rectangle (335.76000000000005,-33.900000000000006);
\draw(327.96000000000004, -34.5) node[anchor=north west,align=left] {Continuous\\ model theory,\\ model \\ theory of \\ metric structures};
\draw (327.96000000000004, -34.5) rectangle (331.56000000000006,-37.1);
\draw(331.66, -34.5) node[anchor=north west,align=left] {Quantifier \\ elimination,\\ model \\ completeness and \\ related topics};
\draw (331.66, -34.5) rectangle (335.01000000000005,-37.1);
\draw(327.96000000000004, -37.2) node[anchor=north west,align=left] {Model theory\\ of finite\\ structures};
\draw (327.96000000000004, -37.2) rectangle (331.31000000000006,-38.800000000000004);
\draw(331.41, -37.2) node[anchor=north west,align=left] {Abstract \\ elementary \\ classes and \\ related topics};
\draw (331.41, -37.2) rectangle (334.76000000000005,-39.300000000000004);
\draw(334.86, -37.2) node[anchor=north west,align=left] {Other \\ infinitary\\ logic};
\draw (334.86, -37.2) rectangle (336.96000000000004,-38.800000000000004);
\draw(327.96000000000004, -39.400000000000006) node[anchor=north west,align=left] {Other \\ classical \\ first-order\\ model theory};
\draw (327.96000000000004, -39.400000000000006) rectangle (331.31000000000006,-41.50000000000001);
\draw(331.41, -39.400000000000006) node[anchor=north west,align=left] {Nonclassical\\ models \\ (Boolean-valued,\\ sheaf, etc.)};
\draw (331.41, -39.400000000000006) rectangle (334.76000000000005,-41.50000000000001);
\draw(327.96000000000004, -41.6) node[anchor=north west,align=left] {Basic \\ properties of \\ first-order \\ languages and\\ structures};
\draw (327.96000000000004, -41.6) rectangle (331.06000000000006,-44.2);
\draw(331.16, -41.6) node[anchor=north west,align=left] {Ultraproducts\\ and\\ related \\ constructions};
\draw (331.16, -41.6) rectangle (334.26000000000005,-43.7);
\draw(334.36, -41.6) node[anchor=north west,align=left] {Models of\\ other \\ mathematical\\ theories};
\draw (334.36, -41.6) rectangle (336.96000000000004,-43.7);
\draw(327.96000000000004, -44.3) node[anchor=north west,align=left] {Logic with\\ extra \\ quantifiers \\ and operators};
\draw (327.96000000000004, -44.3) rectangle (331.06000000000006,-46.4);
\draw(331.16, -44.3) node[anchor=north west,align=left] {Categoricity\\ and \\ completeness\\ of theories};
\draw (331.16, -44.3) rectangle (334.01000000000005,-46.4);
\draw(334.11, -44.3) node[anchor=north west,align=left] {Properties\\ of classes\\ of models};
\draw (334.11, -44.3) rectangle (336.96000000000004,-45.9);
\draw(327.96000000000004, -46.5) node[anchor=north west,align=left] {Models \\ of arithmetic\\ and\\ set theory};
\draw (327.96000000000004, -46.5) rectangle (330.81000000000006,-48.6);
\draw(330.91, -46.5) node[anchor=north west,align=left] {Logic \\ on \\ admissible sets};
\draw (330.91, -46.5) rectangle (333.51000000000005,-48.1);
\draw(333.61, -46.5) node[anchor=north west,align=left] {Second- \\ and \\ higher-order \\ model theory};
\draw (333.61, -46.5) rectangle (336.21000000000004,-48.6);
\draw(327.96000000000004, -48.7) node[anchor=north west,align=left] {Applications\\ of \\ model theory};
\draw (327.96000000000004, -48.7) rectangle (330.56000000000006,-50.300000000000004);
\draw(330.66, -48.7) node[anchor=north west,align=left] {Other \\ model \\ constructions};
\draw (330.66, -48.7) rectangle (333.01000000000005,-50.300000000000004);
\draw(333.11, -48.7) node[anchor=north west,align=left] {Abstract\\ model\\ theory};
\draw (333.11, -48.7) rectangle (335.46000000000004,-50.300000000000004);
\draw(315.06000000000006, -40.70000000000001) node[anchor=north west,align=left] {\large Algebraic logic};
\draw (315.06000000000006, -40.70000000000001) rectangle (323.46000000000004,-50.000000000000014);
\draw(316.06000000000006, -41.70000000000001) node[anchor=north west,align=left] {Categorical\\ logic, topoi};
\draw (316.06000000000006, -41.70000000000001) rectangle (320.4100000000001,-43.30000000000001);
\draw(320.51000000000005, -41.70000000000001) node[anchor=north west,align=left] {Logical\\ aspects\\ of Boolean\\ algebras};
\draw (320.51000000000005, -41.70000000000001) rectangle (323.36000000000007,-43.80000000000001);
\draw(316.06000000000006, -43.90000000000001) node[anchor=north west,align=left] {Logical \\ aspects of \\ Łukasiewicz and\\ Post algebras};
\draw (316.06000000000006, -43.90000000000001) rectangle (319.9100000000001,-46.000000000000014);
\draw(320.01000000000005, -43.90000000000001) node[anchor=north west,align=left] {Other \\ algebras related\\ to logic};
\draw (320.01000000000005, -43.90000000000001) rectangle (322.86000000000007,-45.500000000000014);
\draw(316.06000000000006, -46.10000000000001) node[anchor=north west,align=left] {Logical aspects\\ of lattices\\ and related\\ structures};
\draw (316.06000000000006, -46.10000000000001) rectangle (319.6600000000001,-48.20000000000001);
\draw(319.76000000000005, -46.10000000000001) node[anchor=north west,align=left] {Cylindric and\\ polyadic \\ algebras; \\ relation algebras};
\draw (319.76000000000005, -46.10000000000001) rectangle (323.36000000000007,-48.20000000000001);
\draw(316.06000000000006, -48.30000000000001) node[anchor=north west,align=left] {Abstract\\ algebraic\\ logic};
\draw (316.06000000000006, -48.30000000000001) rectangle (318.6600000000001,-49.90000000000001);
\draw(318.76000000000005, -48.30000000000001) node[anchor=north west,align=left] {Quantum\\ logic};
\draw (318.76000000000005, -48.30000000000001) rectangle (320.61000000000007,-49.40000000000001);
\draw(337.21000000000004, -21.500000000000004) node[anchor=north west,align=left] {\large Nonstandard models};
\draw (337.21000000000004, -21.500000000000004) rectangle (343.39000000000004,-28.6);
\draw(338.21000000000004, -22.500000000000004) node[anchor=north west,align=left] {Other applications\\ of \\ nonstandard models\\ (economics,\\ physics, etc.)};
\draw (338.21000000000004, -22.500000000000004) rectangle (342.06000000000006,-25.100000000000005);
\draw(338.21000000000004, -25.200000000000003) node[anchor=north west,align=left] {Nonstandard\\ models \\ in mathematics};
\draw (338.21000000000004, -25.200000000000003) rectangle (341.56000000000006,-26.800000000000004);
\draw(338.21000000000004, -26.900000000000006) node[anchor=north west,align=left] {Nonstandard\\ models \\ of arithmetic};
\draw (338.21000000000004, -26.900000000000006) rectangle (341.31000000000006,-28.500000000000007);
\draw(315.06000000000006, -50.5) node[anchor=north west,align=left] {\large Set theory};
\draw (315.06000000000006, -50.5) rectangle (324.21000000000004,-73.0);
\draw(316.06000000000006, -51.5) node[anchor=north west,align=left] {Inner models, \\ including \\ constructibility, \\ ordinal definability,\\ and core models};
\draw (316.06000000000006, -51.5) rectangle (320.9100000000001,-54.1);
\draw(321.01000000000005, -51.5) node[anchor=north west,align=left] {Other \\ combinatorial\\ set theory};
\draw (321.01000000000005, -51.5) rectangle (324.11000000000007,-53.1);
\draw(316.06000000000006, -54.2) node[anchor=north west,align=left] {Other classical\\ set theory \\ (including functions,\\ relations,\\ and set algebra)};
\draw (316.06000000000006, -54.2) rectangle (320.1600000000001,-56.800000000000004);
\draw(320.26000000000005, -54.2) node[anchor=north west,align=left] {Determinacy\\ principles};
\draw (320.26000000000005, -54.2) rectangle (324.11000000000007,-55.800000000000004);
\draw(316.06000000000006, -56.9) node[anchor=north west,align=left] {Axiomatics \\ of classical\\ set theory \\ and its fragments};
\draw (316.06000000000006, -56.9) rectangle (320.1600000000001,-59.0);
\draw(320.26000000000005, -56.9) node[anchor=north west,align=left] {Ordered sets\\ and their\\ cofinalities;\\ pcf theory};
\draw (320.26000000000005, -56.9) rectangle (323.86000000000007,-59.0);
\draw(316.06000000000006, -59.1) node[anchor=north west,align=left] {Other aspects\\ of forcing\\ and \\ Boolean-valued models};
\draw (316.06000000000006, -59.1) rectangle (319.6600000000001,-61.2);
\draw(319.76000000000005, -59.1) node[anchor=north west,align=left] {Partition\\ relations};
\draw (319.76000000000005, -59.1) rectangle (323.11000000000007,-60.7);
\draw(316.06000000000006, -61.3) node[anchor=north west,align=left] {Axiom of\\ choice and\\ related\\ propositions};
\draw (316.06000000000006, -61.3) rectangle (319.4100000000001,-63.4);
\draw(319.51000000000005, -61.3) node[anchor=north west,align=left] {Other notions\\ of \\ set-theoretic\\ definability};
\draw (319.51000000000005, -61.3) rectangle (322.86000000000007,-63.4);
\draw(316.06000000000006, -63.5) node[anchor=north west,align=left] {Consistency\\ and\\ independence\\ results};
\draw (316.06000000000006, -63.5) rectangle (319.1600000000001,-65.6);
\draw(319.26000000000005, -63.5) node[anchor=north west,align=left] {Continuum\\ hypothesis\\ and \\ Martin’s axiom};
\draw (319.26000000000005, -63.5) rectangle (322.36000000000007,-65.6);
\draw(316.06000000000006, -65.7) node[anchor=north west,align=left] {Generic \\ absoluteness\\ and \\ forcing axioms};
\draw (316.06000000000006, -65.7) rectangle (319.1600000000001,-67.8);
\draw(319.26000000000005, -65.7) node[anchor=north west,align=left] {Nonclassical\\ and \\ second-order\\ set theories};
\draw (319.26000000000005, -65.7) rectangle (322.36000000000007,-67.8);
\draw(316.06000000000006, -67.9) node[anchor=north west,align=left] {Cardinal \\ characteristics\\ of the\\ continuum};
\draw (316.06000000000006, -67.9) rectangle (318.9100000000001,-70.0);
\draw(319.01000000000005, -67.9) node[anchor=north west,align=left] {Other \\ set-theoretic\\ hypotheses\\ and axioms};
\draw (319.01000000000005, -67.9) rectangle (321.86000000000007,-70.0);
\draw(321.96000000000004, -67.9) node[anchor=north west,align=left] {Descriptive\\ set\\ theory};
\draw (321.96000000000004, -67.9) rectangle (324.06000000000006,-69.5);
\draw(316.06000000000006, -70.1) node[anchor=north west,align=left] {Ordinal \\ and cardinal\\ numbers};
\draw (316.06000000000006, -70.1) rectangle (318.6600000000001,-71.69999999999999);
\draw(318.76000000000005, -70.1) node[anchor=north west,align=left] {Theory \\ of fuzzy\\ sets, etc.};
\draw (318.76000000000005, -70.1) rectangle (321.36000000000007,-71.69999999999999);
\draw(321.46000000000004, -70.1) node[anchor=north west,align=left] {Applications\\ of\\ set theory};
\draw (321.46000000000004, -70.1) rectangle (324.06000000000006,-71.69999999999999);
\draw(316.06000000000006, -71.8) node[anchor=north west,align=left] {Large \\ cardinals};
\draw (316.06000000000006, -71.8) rectangle (318.1600000000001,-72.89999999999999);
\draw(314.06000000000006, -73.19999999999999) node[anchor=north west,align=left] {\LARGE Potential theory};
\draw (314.06000000000006, -73.19999999999999) rectangle (341.77000000000004,-97.79999999999998);
\draw(315.06000000000006, -74.19999999999999) node[anchor=north west,align=left] {\large Potential theory on fractals and metric spaces};
\draw (315.06000000000006, -74.19999999999999) rectangle (329.9200000000001,-77.39999999999999);
\draw(316.06000000000006, -75.19999999999999) node[anchor=north west,align=left] {Potential \\ theory on \\ fractals and\\ metric spaces};
\draw (316.06000000000006, -75.19999999999999) rectangle (319.4100000000001,-77.29999999999998);
\draw(330.02000000000004, -74.19999999999999) node[anchor=north west,align=left] {\large Two-dimensional potential theory};
\draw (330.02000000000004, -74.19999999999999) rectangle (341.67,-87.49999999999999);
\draw(331.02000000000004, -75.19999999999999) node[anchor=north west,align=left] {Potentials and\\ capacity, harmonic\\ measure, \\ extremal length \\ and related notions\\ in two dimensions};
\draw (331.02000000000004, -75.19999999999999) rectangle (336.37000000000006,-78.29999999999998);
\draw(336.47, -75.19999999999999) node[anchor=north west,align=left] {Integral \\ representations, \\ integral operators,\\ integral \\ equations methods\\ in two dimensions};
\draw (336.47, -75.19999999999999) rectangle (341.57000000000005,-78.29999999999998);
\draw(331.02000000000004, -78.39999999999999) node[anchor=north west,align=left] {Connections of\\ harmonic functions\\ with \\ differential equations\\ in two dimensions};
\draw (331.02000000000004, -78.39999999999999) rectangle (335.87000000000006,-80.99999999999999);
\draw(335.97, -78.39999999999999) node[anchor=north west,align=left] {Boundary behavior\\ (theorems\\ of Fatou type,\\ etc.) of \\ harmonic functions\\ in two dimensions};
\draw (335.97, -78.39999999999999) rectangle (340.57000000000005,-81.49999999999999);
\draw(331.02000000000004, -81.6) node[anchor=north west,align=left] {Harmonic, \\ subharmonic, \\ superharmonic\\ functions \\ in two dimensions};
\draw (331.02000000000004, -81.6) rectangle (335.12000000000006,-84.19999999999999);
\draw(335.22, -81.6) node[anchor=north west,align=left] {Boundary value\\ and inverse \\ problems for harmonic\\ functions \\ in two dimensions};
\draw (335.22, -81.6) rectangle (339.32000000000005,-84.19999999999999);
\draw(331.02000000000004, -84.29999999999998) node[anchor=north west,align=left] {Biharmonic, \\ polyharmonic \\ functions and \\ equations, Poisson’s\\ equation \\ in two dimensions};
\draw (331.02000000000004, -84.29999999999998) rectangle (335.12000000000006,-87.39999999999998);
\draw(315.06000000000006, -77.49999999999999) node[anchor=north west,align=left] {\large Generalizations of potential theory};
\draw (315.06000000000006, -77.49999999999999) rectangle (328.11000000000007,-85.59999999999998);
\draw(316.06000000000006, -78.49999999999999) node[anchor=north west,align=left] {Fine potential\\ theory;\\ fine \\ properties of \\ sets and functions};
\draw (316.06000000000006, -78.49999999999999) rectangle (320.1600000000001,-81.09999999999998);
\draw(320.26000000000005, -78.49999999999999) node[anchor=north west,align=left] {Harmonic, \\ subharmonic, \\ superharmonic\\ functions \\ on other spaces};
\draw (320.26000000000005, -78.49999999999999) rectangle (323.86000000000007,-81.09999999999998);
\draw(323.96000000000004, -78.49999999999999) node[anchor=north west,align=left] {Pluriharmonic\\ and \\ plurisubharmonic\\ functions};
\draw (323.96000000000004, -78.49999999999999) rectangle (327.56000000000006,-80.59999999999998);
\draw(316.06000000000006, -81.19999999999999) node[anchor=north west,align=left] {Potential \\ theory on \\ Riemannian \\ manifolds and\\ other spaces};
\draw (316.06000000000006, -81.19999999999999) rectangle (319.4100000000001,-83.79999999999998);
\draw(319.51000000000005, -81.19999999999999) node[anchor=north west,align=left] {Other \\ generalizations\\ (nonlinear\\ potential \\ theory, etc.)};
\draw (319.51000000000005, -81.19999999999999) rectangle (322.61000000000007,-83.79999999999998);
\draw(322.71000000000004, -81.19999999999999) node[anchor=north west,align=left] {Potentials\\ and \\ capacities on \\ other spaces};
\draw (322.71000000000004, -81.19999999999999) rectangle (325.56000000000006,-83.29999999999998);
\draw(325.6600000000001, -81.19999999999999) node[anchor=north west,align=left] {Discrete\\ potential\\ theory};
\draw (325.6600000000001, -81.19999999999999) rectangle (328.0100000000001,-82.79999999999998);
\draw(316.06000000000006, -83.89999999999999) node[anchor=north west,align=left] {Dirichlet\\ forms};
\draw (316.06000000000006, -83.89999999999999) rectangle (318.1600000000001,-84.99999999999999);
\draw(318.26000000000005, -83.89999999999999) node[anchor=north west,align=left] {Martin\\ boundary\\ theory};
\draw (318.26000000000005, -83.89999999999999) rectangle (320.36000000000007,-85.49999999999999);
\draw(315.06000000000006, -85.69999999999999) node[anchor=north west,align=left] {\large Computational methods\\ for problems pertaining\\ to potential theory};
\draw (315.06000000000006, -85.69999999999999) rectangle (322.1700000000001,-87.29999999999998);
\draw(315.06000000000006, -87.6) node[anchor=north west,align=left] {\large Higher-dimensional potential theory};
\draw (315.06000000000006, -87.6) rectangle (328.4100000000001,-97.69999999999999);
\draw(316.06000000000006, -88.6) node[anchor=north west,align=left] {Boundary value\\ and inverse \\ problems for harmonic\\ functions \\ in higher dimensions};
\draw (316.06000000000006, -88.6) rectangle (320.9100000000001,-91.19999999999999);
\draw(321.01000000000005, -88.6) node[anchor=north west,align=left] {Integral \\ representations, \\ integral operators,\\ integral equations\\ methods in\\ higher dimensions};
\draw (321.01000000000005, -88.6) rectangle (325.61000000000007,-91.69999999999999);
\draw(316.06000000000006, -91.8) node[anchor=north west,align=left] {Potentials and\\ capacities,\\ extremal \\ length and related\\ notions in\\ higher dimensions};
\draw (316.06000000000006, -91.8) rectangle (320.6600000000001,-94.89999999999999);
\draw(320.76000000000005, -91.8) node[anchor=north west,align=left] {Connections \\ of harmonic \\ functions with\\ differential\\ equations in\\ higher dimensions};
\draw (320.76000000000005, -91.8) rectangle (325.11000000000007,-94.89999999999999);
\draw(316.06000000000006, -95.0) node[anchor=north west,align=left] {Boundary \\ behavior of \\ harmonic \\ functions in \\ higher dimensions};
\draw (316.06000000000006, -95.0) rectangle (320.1600000000001,-97.6);
\draw(320.26000000000005, -95.0) node[anchor=north west,align=left] {Biharmonic and\\ polyharmonic\\ equations and\\ functions in \\ higher dimensions};
\draw (320.26000000000005, -95.0) rectangle (324.36000000000007,-97.6);
\draw(324.46000000000004, -95.0) node[anchor=north west,align=left] {Harmonic, \\ subharmonic, \\ superharmonic \\ functions in \\ higher dimensions};
\draw (324.46000000000004, -95.0) rectangle (328.31000000000006,-97.6);
\draw(328.51000000000005, -87.6) node[anchor=north west,align=left] {\large Axiomatic potential theory};
\draw (328.51000000000005, -87.6) rectangle (337.1700000000001,-90.3);
\draw(329.51000000000005, -88.6) node[anchor=north west,align=left] {Axiomatic\\ potential\\ theory};
\draw (329.51000000000005, -88.6) rectangle (332.11000000000007,-90.19999999999999);
\draw(328.51000000000005, -90.39999999999999) node[anchor=north west,align=left] {\large History of \\ potential theory};
\draw (328.51000000000005, -90.39999999999999) rectangle (332.83000000000004,-91.49999999999999);
\draw(314.06000000000006, -97.89999999999999) node[anchor=north west,align=left] {\LARGE Real functions};
\draw (314.06000000000006, -97.89999999999999) rectangle (341.64000000000004,-137.2);
\draw(315.06000000000006, -98.89999999999999) node[anchor=north west,align=left] {\large Polynomials, rational functions in real analysis};
\draw (315.06000000000006, -98.89999999999999) rectangle (330.5400000000001,-102.1);
\draw(316.06000000000006, -99.89999999999999) node[anchor=north west,align=left] {Real \\ polynomials: \\ analytic \\ properties, etc.};
\draw (316.06000000000006, -99.89999999999999) rectangle (319.4100000000001,-101.99999999999999);
\draw(319.51000000000005, -99.89999999999999) node[anchor=north west,align=left] {Real \\ polynomials:\\ location\\ of zeros};
\draw (319.51000000000005, -99.89999999999999) rectangle (322.36000000000007,-101.99999999999999);
\draw(322.46000000000004, -99.89999999999999) node[anchor=north west,align=left] {Real \\ rational\\ functions};
\draw (322.46000000000004, -99.89999999999999) rectangle (324.81000000000006,-101.49999999999999);
\draw(330.64000000000004, -98.89999999999999) node[anchor=north west,align=left] {\large Functions of one variable};
\draw (330.64000000000004, -98.89999999999999) rectangle (341.54,-124.69999999999999);
\draw(331.64000000000004, -99.89999999999999) node[anchor=north west,align=left] {Continuity and \\ related questions \\ (modulus of continuity,\\ semicontinuity,\\ discontinuities,\\ etc.) for real \\ functions in one variable};
\draw (331.64000000000004, -99.89999999999999) rectangle (337.24000000000007,-103.49999999999999);
\draw(337.34000000000003, -99.89999999999999) node[anchor=north west,align=left] {Foundations: \\ limits and \\ generalizations, \\ elementary \\ topology of the line};
\draw (337.34000000000003, -99.89999999999999) rectangle (341.44000000000005,-102.49999999999999);
\draw(331.64000000000004, -103.6) node[anchor=north west,align=left] {Differentiation \\ (real functions of \\ one variable): general\\ theory, generalized\\ derivatives,\\ mean value theorems};
\draw (331.64000000000004, -103.6) rectangle (336.74000000000007,-106.69999999999999);
\draw(336.84000000000003, -103.6) node[anchor=north west,align=left] {Rate of growth\\ of functions,\\ orders of \\ infinity, slowly\\ varying functions};
\draw (336.84000000000003, -103.6) rectangle (341.19000000000005,-106.19999999999999);
\draw(331.64000000000004, -106.8) node[anchor=north west,align=left] {Fractional\\ derivatives \\ and integrals};
\draw (331.64000000000004, -106.8) rectangle (336.74000000000007,-108.89999999999999);
\draw(336.84000000000003, -106.8) node[anchor=north west,align=left] {Singular functions,\\ Cantor \\ functions, functions\\ with other \\ special properties};
\draw (336.84000000000003, -106.8) rectangle (341.19000000000005,-109.39999999999999);
\draw(331.64000000000004, -109.5) node[anchor=north west,align=left] {Monotonic\\ functions,\\ generalizations};
\draw (331.64000000000004, -109.5) rectangle (336.74000000000007,-111.6);
\draw(336.84000000000003, -109.5) node[anchor=north west,align=left] {Classification\\ of real functions;\\ Baire \\ classification of \\ sets and functions};
\draw (336.84000000000003, -109.5) rectangle (340.94000000000005,-112.1);
\draw(331.64000000000004, -112.19999999999999) node[anchor=north west,align=left] {Nondifferentiability\\ (nondifferentiable\\ functions, \\ points of \\ nondifferentiability), \\ discontinuous derivatives};
\draw (331.64000000000004, -112.19999999999999) rectangle (336.49000000000007,-115.29999999999998);
\draw(336.59000000000003, -112.19999999999999) node[anchor=north west,align=left] {Antidifferentiation};
\draw (336.59000000000003, -112.19999999999999) rectangle (341.44000000000005,-113.79999999999998);
\draw(331.64000000000004, -115.39999999999999) node[anchor=north west,align=left] {Denjoy and \\ Perron integrals,\\ other \\ special integrals};
\draw (331.64000000000004, -115.39999999999999) rectangle (335.74000000000007,-117.49999999999999);
\draw(335.84000000000003, -115.39999999999999) node[anchor=north west,align=left] {Absolutely \\ continuous \\ real functions\\ in one variable};
\draw (335.84000000000003, -115.39999999999999) rectangle (339.94000000000005,-117.49999999999999);
\draw(331.64000000000004, -117.6) node[anchor=north west,align=left] {Convexity \\ of real \\ functions in one\\ variable,\\ generalizations};
\draw (331.64000000000004, -117.6) rectangle (335.74000000000007,-120.19999999999999);
\draw(335.84000000000003, -117.6) node[anchor=north west,align=left] {One-variable\\ calculus};
\draw (335.84000000000003, -117.6) rectangle (339.69000000000005,-119.19999999999999);
\draw(331.64000000000004, -120.29999999999998) node[anchor=north west,align=left] {Elementary\\ functions};
\draw (331.64000000000004, -120.29999999999998) rectangle (335.24000000000007,-121.89999999999998);
\draw(335.34000000000003, -120.29999999999998) node[anchor=north west,align=left] {Functions \\ of bounded \\ variation, \\ generalizations};
\draw (335.34000000000003, -120.29999999999998) rectangle (338.94000000000005,-122.39999999999998);
\draw(331.64000000000004, -122.49999999999999) node[anchor=north west,align=left] {Iteration\\ of real \\ functions in\\ one variable};
\draw (331.64000000000004, -122.49999999999999) rectangle (334.99000000000007,-124.59999999999998);
\draw(335.09000000000003, -122.49999999999999) node[anchor=north west,align=left] {Integrals of\\ Riemann, \\ Stieltjes and\\ Lebesgue type};
\draw (335.09000000000003, -122.49999999999999) rectangle (338.44000000000005,-124.59999999999998);
\draw(338.54, -122.49999999999999) node[anchor=north west,align=left] {Lipschitz\\ (Hölder)\\ classes};
\draw (338.54, -122.49999999999999) rectangle (341.14000000000004,-124.09999999999998);
\draw(315.06000000000006, -102.19999999999999) node[anchor=north west,align=left] {\large Miscellaneous topics in real functions};
\draw (315.06000000000006, -102.19999999999999) rectangle (329.4100000000001,-111.99999999999999);
\draw(316.06000000000006, -103.19999999999999) node[anchor=north west,align=left] {Real-analyticfunctions};
\draw (316.06000000000006, -103.19999999999999) rectangle (321.9100000000001,-104.79999999999998);
\draw(322.01000000000005, -103.19999999999999) node[anchor=north west,align=left] {\(C^\infty\)-functions,\\ quasi-analytic\\ functions};
\draw (322.01000000000005, -103.19999999999999) rectangle (327.36000000000007,-105.29999999999998);
\draw(327.46000000000004, -103.19999999999999) node[anchor=north west,align=left] {Fuzzy\\ real \\ analysis};
\draw (327.46000000000004, -103.19999999999999) rectangle (329.31000000000006,-104.79999999999998);
\draw(316.06000000000006, -105.39999999999999) node[anchor=north west,align=left] {Nonstandardanalysis};
\draw (316.06000000000006, -105.39999999999999) rectangle (321.1600000000001,-106.99999999999999);
\draw(321.26000000000005, -105.39999999999999) node[anchor=north west,align=left] {Non-Archimedean\\ analysis};
\draw (321.26000000000005, -105.39999999999999) rectangle (325.61000000000007,-106.99999999999999);
\draw(325.71000000000004, -105.39999999999999) node[anchor=north west,align=left] {Calculus of \\ functions taking\\ values in\\ infinite-dimensional\\ spaces};
\draw (325.71000000000004, -105.39999999999999) rectangle (329.31000000000006,-107.99999999999999);
\draw(321.26000000000005, -107.1) node[anchor=north west,align=left] {Means};
\draw (321.26000000000005, -107.1) rectangle (322.61000000000007,-107.69999999999999);
\draw(316.06000000000006, -108.1) node[anchor=north west,align=left] {Calculus of\\ functions on\\ infinite-dimensional\\ spaces};
\draw (316.06000000000006, -108.1) rectangle (320.1600000000001,-110.19999999999999);
\draw(320.26000000000005, -108.1) node[anchor=north west,align=left] {Set-valued\\ functions};
\draw (320.26000000000005, -108.1) rectangle (323.86000000000007,-109.69999999999999);
\draw(323.96000000000004, -108.1) node[anchor=north west,align=left] {Real analysis\\ on time\\ scales or \\ measure chains};
\draw (323.96000000000004, -108.1) rectangle (327.31000000000006,-110.19999999999999);
\draw(316.06000000000006, -110.29999999999998) node[anchor=north west,align=left] {Constructive\\ real\\ analysis};
\draw (316.06000000000006, -110.29999999999998) rectangle (318.6600000000001,-111.89999999999998);
\draw(315.06000000000006, -112.1) node[anchor=north west,align=left] {\large Inequalities in real analysis};
\draw (315.06000000000006, -112.1) rectangle (325.6600000000001,-118.5);
\draw(316.06000000000006, -113.1) node[anchor=north west,align=left] {Inequalities \\ involving \\ derivatives and \\ differential and\\ integral operators};
\draw (316.06000000000006, -113.1) rectangle (320.6600000000001,-115.69999999999999);
\draw(320.76000000000005, -113.1) node[anchor=north west,align=left] {Inequalities\\ involving\\ other types\\ of functions};
\draw (320.76000000000005, -113.1) rectangle (324.36000000000007,-115.19999999999999);
\draw(316.06000000000006, -115.8) node[anchor=north west,align=left] {Inequalities\\ for sums,\\ series\\ and integrals};
\draw (316.06000000000006, -115.8) rectangle (319.4100000000001,-117.89999999999999);
\draw(319.51000000000005, -115.8) node[anchor=north west,align=left] {Inequalities\\ for \\ trigonometric \\ functions and\\ polynomials};
\draw (319.51000000000005, -115.8) rectangle (322.61000000000007,-118.39999999999999);
\draw(322.71000000000004, -115.8) node[anchor=north west,align=left] {Other \\ analytical \\ inequalities};
\draw (322.71000000000004, -115.8) rectangle (325.56000000000006,-117.39999999999999);
\draw(315.06000000000006, -118.6) node[anchor=north west,align=left] {\large Computational methods\\ for problems \\ pertaining to real functions};
\draw (315.06000000000006, -118.6) rectangle (322.1700000000001,-120.19999999999999);
\draw(315.06000000000006, -120.3) node[anchor=north west,align=left] {\large History of \\ real functions};
\draw (315.06000000000006, -120.3) rectangle (319.07000000000005,-121.39999999999999);
\draw(315.06000000000006, -124.8) node[anchor=north west,align=left] {\large Functions of several variables};
\draw (315.06000000000006, -124.8) rectangle (325.46000000000004,-137.1);
\draw(316.06000000000006, -125.8) node[anchor=north west,align=left] {Absolutely \\ continuous real \\ functions of \\ several variables,\\ functions \\ of bounded variation};
\draw (316.06000000000006, -125.8) rectangle (320.9100000000001,-128.9);
\draw(321.01000000000005, -125.8) node[anchor=north west,align=left] {Implicit function\\ theorems, \\ Jacobians, \\ transformations with\\ several variables};
\draw (321.01000000000005, -125.8) rectangle (325.36000000000007,-128.4);
\draw(316.06000000000006, -129.0) node[anchor=north west,align=left] {Convexity of\\ real functions\\ of several\\ variables,\\ generalizations};
\draw (316.06000000000006, -129.0) rectangle (320.4100000000001,-131.6);
\draw(320.51000000000005, -129.0) node[anchor=north west,align=left] {Special properties\\ of functions\\ of several \\ variables, Hölder\\ conditions, etc.};
\draw (320.51000000000005, -129.0) rectangle (324.86000000000007,-131.6);
\draw(316.06000000000006, -131.7) node[anchor=north west,align=left] {Integration of\\ real functions\\ of several \\ variables: length,\\ area, volume};
\draw (316.06000000000006, -131.7) rectangle (320.1600000000001,-134.29999999999998);
\draw(320.26000000000005, -131.7) node[anchor=north west,align=left] {Integral formulas\\ of real \\ functions of \\ several variables\\ (Stokes, Gauss,\\ Green, etc.)};
\draw (320.26000000000005, -131.7) rectangle (324.11000000000007,-134.79999999999998);
\draw(316.06000000000006, -134.9) node[anchor=north west,align=left] {Continuity\\ and \\ differentiation\\ questions};
\draw (316.06000000000006, -134.9) rectangle (319.4100000000001,-137.0);
\draw(319.51000000000005, -134.9) node[anchor=north west,align=left] {Representation\\ and \\ superposition\\ of functions};
\draw (319.51000000000005, -134.9) rectangle (322.61000000000007,-137.0);
\draw(322.71000000000004, -134.9) node[anchor=north west,align=left] {Calculus\\ of vector\\ functions};
\draw (322.71000000000004, -134.9) rectangle (325.31000000000006,-136.5);
\draw(344.06000000000006, -1) node[anchor=north west,align=left] {\LARGE Special functions};
\draw (344.06000000000006, -1) rectangle (370.81000000000006,-39.900000000000006);
\draw(345.06000000000006, -2) node[anchor=north west,align=left] {\large Hypergeometric functions};
\draw (345.06000000000006, -2) rectangle (358.96000000000004,-22.900000000000002);
\draw(346.06000000000006, -3) node[anchor=north west,align=left] {Hypergeometric\\ integrals and \\ functions defined\\ by them (\(E\), \(G\),\\ \(H\) and \(I\) functions)};
\draw (346.06000000000006, -3) rectangle (353.1600000000001,-5.6);
\draw(353.26000000000005, -3) node[anchor=north west,align=left] {Orthogonal polynomials\\ and functions\\ of hypergeometric\\ type (Jacobi,\\ Laguerre, Hermite,\\ Askey scheme, etc.)};
\draw (353.26000000000005, -3) rectangle (358.86000000000007,-6.1);
\draw(346.06000000000006, -6.2) node[anchor=north west,align=left] {Classical\\ hypergeometric\\ functions, \({}_2F_1\)};
\draw (346.06000000000006, -6.2) rectangle (352.6600000000001,-8.3);
\draw(352.76000000000005, -6.2) node[anchor=north west,align=left] {Confluent \\ hypergeometric\\ functions,\\ Whittaker \\ functions, \({}_1F_1\)};
\draw (352.76000000000005, -6.2) rectangle (357.61000000000007,-8.8);
\draw(346.06000000000006, -8.9) node[anchor=north west,align=left] {Generalized\\ hypergeometric\\ series, \({}_pF_q\)};
\draw (346.06000000000006, -8.9) rectangle (352.4100000000001,-11.0);
\draw(352.51000000000005, -8.9) node[anchor=north west,align=left] {Orthogonal \\ polynomials and functions\\ in several\\ variables expressible\\ in terms \\ of special functions\\ in one variable};
\draw (352.51000000000005, -8.9) rectangle (357.36000000000007,-12.5);
\draw(346.06000000000006, -12.600000000000001) node[anchor=north west,align=left] {Connections \\ of hypergeometric\\ functions\\ with groups\\ and algebras,\\ and related topics};
\draw (346.06000000000006, -12.600000000000001) rectangle (350.9100000000001,-15.700000000000001);
\draw(351.01000000000005, -12.600000000000001) node[anchor=north west,align=left] {Bessel and\\ Airy \\ functions, \\ cylinder \\ functions, \({}_0F_1\)};
\draw (351.01000000000005, -12.600000000000001) rectangle (355.36000000000007,-15.200000000000001);
\draw(355.46000000000004, -12.600000000000001) node[anchor=north west,align=left] {Spherical\\ harmonics};
\draw (355.46000000000004, -12.600000000000001) rectangle (358.81000000000006,-14.200000000000001);
\draw(346.06000000000006, -15.8) node[anchor=north west,align=left] {Orthogonal \\ polynomials \\ and functions\\ associated\\ with root systems};
\draw (346.06000000000006, -15.8) rectangle (350.4100000000001,-18.400000000000002);
\draw(350.51000000000005, -15.8) node[anchor=north west,align=left] {Other \\ hypergeometric \\ functions and \\ integrals in\\ several variables};
\draw (350.51000000000005, -15.8) rectangle (354.86000000000007,-18.400000000000002);
\draw(354.96000000000004, -15.8) node[anchor=north west,align=left] {Other special\\ orthogonal\\ polynomials\\ and functions};
\draw (354.96000000000004, -15.8) rectangle (358.56000000000006,-17.900000000000002);
\draw(346.06000000000006, -18.5) node[anchor=north west,align=left] {Hypergeometric\\ functions \\ associated with\\ root systems};
\draw (346.06000000000006, -18.5) rectangle (349.6600000000001,-20.6);
\draw(349.76000000000005, -18.5) node[anchor=north west,align=left] {Elliptic \\ integrals as\\ hypergeometric\\ functions};
\draw (349.76000000000005, -18.5) rectangle (353.11000000000007,-20.6);
\draw(353.21000000000004, -18.5) node[anchor=north west,align=left] {Applications\\ of \\ hypergeometric\\ functions};
\draw (353.21000000000004, -18.5) rectangle (356.56000000000006,-20.6);
\draw(346.06000000000006, -20.7) node[anchor=north west,align=left] {Appell, \\ Horn and \\ Lauricella\\ functions};
\draw (346.06000000000006, -20.7) rectangle (348.6600000000001,-22.8);
\draw(359.06000000000006, -2) node[anchor=north west,align=left] {\large Basic hypergeometric functions};
\draw (359.06000000000006, -2) rectangle (370.71000000000004,-18.0);
\draw(360.06000000000006, -3) node[anchor=north west,align=left] {Basic orthogonal\\ polynomials and\\ functions \\ associated with root\\ systems (Macdonald\\ polynomials, etc.)};
\draw (360.06000000000006, -3) rectangle (365.4100000000001,-6.1);
\draw(365.51000000000005, -3) node[anchor=north west,align=left] {Orthogonal polynomials\\ and functions\\ in several variables\\ expressible in\\ terms of basic \\ hypergeometric functions\\ in one variable};
\draw (365.51000000000005, -3) rectangle (370.61000000000007,-6.6);
\draw(360.06000000000006, -6.7) node[anchor=north west,align=left] {Connections of basic\\ hypergeometric \\ functions with quantum\\ groups, Chevalley\\ groups, \(p\)-adic \\ groups, Hecke algebras,\\ and related topics};
\draw (360.06000000000006, -6.7) rectangle (365.1600000000001,-10.3);
\draw(365.26000000000005, -6.7) node[anchor=north west,align=left] {Basic orthogonal\\ polynomials\\ and functions\\ (Askey-Wilson\\ polynomials, etc.)};
\draw (365.26000000000005, -6.7) rectangle (370.11000000000007,-9.3);
\draw(360.06000000000006, -10.4) node[anchor=north west,align=left] {Basic \\ hypergeometric functions\\ in one \\ variable, \({}_r\phi_s\)};
\draw (360.06000000000006, -10.4) rectangle (364.4100000000001,-12.5);
\draw(364.51000000000005, -10.4) node[anchor=north west,align=left] {Basic \\ hypergeometric \\ integrals and\\ functions\\ defined by them};
\draw (364.51000000000005, -10.4) rectangle (368.61000000000007,-13.0);
\draw(360.06000000000006, -13.100000000000001) node[anchor=north west,align=left] {Basic \\ hypergeometric \\ functions \\ associated \\ with root systems};
\draw (360.06000000000006, -13.100000000000001) rectangle (363.9100000000001,-15.700000000000001);
\draw(364.01000000000005, -13.100000000000001) node[anchor=north west,align=left] {Other basic \\ hypergeometric \\ functions and \\ integrals in \\ several variables};
\draw (364.01000000000005, -13.100000000000001) rectangle (367.86000000000007,-15.700000000000001);
\draw(360.06000000000006, -15.8) node[anchor=north west,align=left] {\(q\)-gamma \\ functions, \(q\)-beta\\ functions\\ and integrals};
\draw (360.06000000000006, -15.8) rectangle (363.6600000000001,-17.900000000000002);
\draw(363.76000000000005, -15.8) node[anchor=north west,align=left] {Applications\\ of basic\\ hypergeometric\\ functions};
\draw (363.76000000000005, -15.8) rectangle (367.11000000000007,-17.900000000000002);
\draw(367.21000000000004, -15.8) node[anchor=north west,align=left] {Bibasic\\ functions\\ and \\ multiple bases};
\draw (367.21000000000004, -15.8) rectangle (370.31000000000006,-17.900000000000002);
\draw(359.06000000000006, -18.1) node[anchor=north west,align=left] {\large History of \\ special functions};
\draw (359.06000000000006, -18.1) rectangle (363.69000000000005,-19.200000000000003);
\draw(345.06000000000006, -23.000000000000004) node[anchor=north west,align=left] {\large Computational aspects of special functions};
\draw (345.06000000000006, -23.000000000000004) rectangle (358.68000000000006,-27.200000000000003);
\draw(346.06000000000006, -24.000000000000004) node[anchor=north west,align=left] {Symbolic \\ computation of \\ special functions\\ (Gosper and\\ Zeilberger\\ algorithms, etc.)};
\draw (346.06000000000006, -24.000000000000004) rectangle (350.4100000000001,-27.100000000000005);
\draw(350.51000000000005, -24.000000000000004) node[anchor=north west,align=left] {Numerical \\ approximation\\ and \\ evaluation of \\ special functions};
\draw (350.51000000000005, -24.000000000000004) rectangle (354.36000000000007,-26.600000000000005);
\draw(358.7800000000001, -23.000000000000004) node[anchor=north west,align=left] {\large Other special functions};
\draw (358.7800000000001, -23.000000000000004) rectangle (369.43000000000006,-33.300000000000004);
\draw(359.7800000000001, -24.000000000000004) node[anchor=north west,align=left] {Painlevé-typefunctions};
\draw (359.7800000000001, -24.000000000000004) rectangle (365.6300000000001,-25.600000000000005);
\draw(365.7300000000001, -24.000000000000004) node[anchor=north west,align=left] {Lamé, \\ Mathieu, and \\ spheroidal\\ wave functions};
\draw (365.7300000000001, -24.000000000000004) rectangle (369.3300000000001,-26.100000000000005);
\draw(359.7800000000001, -26.200000000000003) node[anchor=north west,align=left] {Other functions\\ coming from\\ differential,\\ difference and\\ integral equations};
\draw (359.7800000000001, -26.200000000000003) rectangle (364.6300000000001,-28.800000000000004);
\draw(364.7300000000001, -26.200000000000003) node[anchor=north west,align=left] {Mittag-Leffler\\ functions\\ and \\ generalizations};
\draw (364.7300000000001, -26.200000000000003) rectangle (368.3300000000001,-28.300000000000004);
\draw(359.7800000000001, -28.900000000000006) node[anchor=north west,align=left] {Other functions\\ defined\\ by series\\ and integrals};
\draw (359.7800000000001, -28.900000000000006) rectangle (363.3800000000001,-31.000000000000007);
\draw(363.4800000000001, -28.900000000000006) node[anchor=north west,align=left] {Special \\ functions in \\ characteristic \\ \(p\) (gamma \\ functions, etc.)};
\draw (363.4800000000001, -28.900000000000006) rectangle (366.8300000000001,-31.500000000000007);
\draw(366.93000000000006, -28.900000000000006) node[anchor=north west,align=left] {Other\\ wave \\ functions};
\draw (366.93000000000006, -28.900000000000006) rectangle (369.0300000000001,-30.500000000000007);
\draw(359.7800000000001, -31.600000000000005) node[anchor=north west,align=left] {Elliptic \\ functions \\ and integrals};
\draw (359.7800000000001, -31.600000000000005) rectangle (362.8800000000001,-33.2);
\draw(345.06000000000006, -33.400000000000006) node[anchor=north west,align=left] {\large Elementary classical functions};
\draw (345.06000000000006, -33.400000000000006) rectangle (354.96000000000004,-39.800000000000004);
\draw(346.06000000000006, -34.400000000000006) node[anchor=north west,align=left] {Incomplete beta\\ and gamma \\ functions (error \\ functions, probability\\ integral,\\ Fresnel integrals)};
\draw (346.06000000000006, -34.400000000000006) rectangle (350.9100000000001,-37.50000000000001);
\draw(351.01000000000005, -34.400000000000006) node[anchor=north west,align=left] {Exponential\\ and \\ trigonometric\\ functions};
\draw (351.01000000000005, -34.400000000000006) rectangle (354.11000000000007,-36.50000000000001);
\draw(346.06000000000006, -37.60000000000001) node[anchor=north west,align=left] {Gamma, \\ beta and\\ polygamma\\ functions};
\draw (346.06000000000006, -37.60000000000001) rectangle (348.9100000000001,-39.70000000000001);
\draw(349.01000000000005, -37.60000000000001) node[anchor=north west,align=left] {Higher \\ logarithm\\ functions};
\draw (349.01000000000005, -37.60000000000001) rectangle (351.61000000000007,-39.20000000000001);
\draw(344.06000000000006, -40.00000000000001) node[anchor=north west,align=left] {\LARGE K-Theory};
\draw (344.06000000000006, -40.00000000000001) rectangle (370.2300000000001,-71.9);
\draw(345.06000000000006, -41.00000000000001) node[anchor=north west,align=left] {\large Miscellaneous applications of \(K\)-theory};
\draw (345.06000000000006, -41.00000000000001) rectangle (358.68000000000006,-44.20000000000001);
\draw(346.06000000000006, -42.00000000000001) node[anchor=north west,align=left] {Miscellaneous\\ applications\\ of \(K\)-theory};
\draw (346.06000000000006, -42.00000000000001) rectangle (352.1600000000001,-44.10000000000001);
\draw(358.7800000000001, -41.00000000000001) node[anchor=north west,align=left] {\large Grothendieck groups and \(K_0\)};
\draw (358.7800000000001, -41.00000000000001) rectangle (370.1300000000001,-45.900000000000006);
\draw(359.7800000000001, -42.00000000000001) node[anchor=north west,align=left] {Frobenius \\ induction, Burnside\\ and \\ representation rings};
\draw (359.7800000000001, -42.00000000000001) rectangle (363.6300000000001,-44.10000000000001);
\draw(363.7300000000001, -42.00000000000001) node[anchor=north west,align=left] {Stability\\ for projective\\ modules};
\draw (363.7300000000001, -42.00000000000001) rectangle (366.8300000000001,-43.60000000000001);
\draw(366.93000000000006, -42.00000000000001) node[anchor=north west,align=left] {\(K_0\) of \\ group rings\\ and orders};
\draw (366.93000000000006, -42.00000000000001) rectangle (370.0300000000001,-43.60000000000001);
\draw(359.7800000000001, -44.20000000000001) node[anchor=north west,align=left] {Efficient\\ generation\\ of modules};
\draw (359.7800000000001, -44.20000000000001) rectangle (362.6300000000001,-45.80000000000001);
\draw(362.7300000000001, -44.20000000000001) node[anchor=north west,align=left] {\(K_0\) \\ of other\\ rings};
\draw (362.7300000000001, -44.20000000000001) rectangle (364.5800000000001,-45.80000000000001);
\draw(345.06000000000006, -44.300000000000004) node[anchor=north west,align=left] {\large History \\ of \(K\)-theory};
\draw (345.06000000000006, -44.300000000000004) rectangle (349.38000000000005,-45.400000000000006);
\draw(345.06000000000006, -46.00000000000001) node[anchor=north west,align=left] {\large Higher algebraic \(K\)-theory};
\draw (345.06000000000006, -46.00000000000001) rectangle (357.6600000000001,-53.10000000000001);
\draw(346.06000000000006, -47.00000000000001) node[anchor=north west,align=left] {Karoubi-Villamayor-Gersten\\ \(K\)-theory};
\draw (346.06000000000006, -47.00000000000001) rectangle (353.9100000000001,-49.10000000000001);
\draw(354.01000000000005, -47.00000000000001) node[anchor=north west,align=left] {Computations\\ of \\ higher \(K\)-theory\\ of rings};
\draw (354.01000000000005, -47.00000000000001) rectangle (357.36000000000007,-49.10000000000001);
\draw(346.06000000000006, -49.20000000000001) node[anchor=north west,align=left] {\(Q\)- and\\ plus-constructions};
\draw (346.06000000000006, -49.20000000000001) rectangle (350.6600000000001,-50.80000000000001);
\draw(350.76000000000005, -49.20000000000001) node[anchor=north west,align=left] {Higher \\ symbols, Milnor\\ \(K\)-theory};
\draw (350.76000000000005, -49.20000000000001) rectangle (354.61000000000007,-50.80000000000001);
\draw(354.71000000000004, -49.20000000000001) node[anchor=north west,align=left] {Symmetric\\ monoidal\\ categories};
\draw (354.71000000000004, -49.20000000000001) rectangle (357.56000000000006,-50.80000000000001);
\draw(346.06000000000006, -50.900000000000006) node[anchor=north west,align=left] {\(K\)-theory and\\ homology; \\ cyclic homology\\ and cohomology};
\draw (346.06000000000006, -50.900000000000006) rectangle (349.9100000000001,-53.00000000000001);
\draw(350.01000000000005, -50.900000000000006) node[anchor=north west,align=left] {Negative \\ \(K\)-theory,\\ NK and Nil};
\draw (350.01000000000005, -50.900000000000006) rectangle (352.86000000000007,-52.50000000000001);
\draw(352.96000000000004, -50.900000000000006) node[anchor=north west,align=left] {Algebraic\\ \(K\)-theory\\ of spaces};
\draw (352.96000000000004, -50.900000000000006) rectangle (355.56000000000006,-52.50000000000001);
\draw(357.76000000000005, -46.00000000000001) node[anchor=north west,align=left] {\large \(K\)-theory and operator algebras};
\draw (357.76000000000005, -46.00000000000001) rectangle (370.06000000000006,-48.70000000000001);
\draw(358.76000000000005, -47.00000000000001) node[anchor=north west,align=left] {Kasparov\\ theory\\ (\(KK\)-theory)};
\draw (358.76000000000005, -47.00000000000001) rectangle (362.61000000000007,-48.60000000000001);
\draw(362.71000000000004, -47.00000000000001) node[anchor=north west,align=left] {\(K_0\) as an\\ ordered \\ group, traces};
\draw (362.71000000000004, -47.00000000000001) rectangle (365.81000000000006,-48.60000000000001);
\draw(365.91, -47.00000000000001) node[anchor=north west,align=left] {Ext \\ and \\ \(K\)-homology};
\draw (365.91, -47.00000000000001) rectangle (368.01000000000005,-48.60000000000001);
\draw(368.11000000000007, -47.00000000000001) node[anchor=north west,align=left] {Index\\ theory};
\draw (368.11000000000007, -47.00000000000001) rectangle (369.9600000000001,-48.10000000000001);
\draw(357.76000000000005, -48.800000000000004) node[anchor=north west,align=left] {\large Computational methods\\ for problems \\ pertaining to \(K\)-theory};
\draw (357.76000000000005, -48.800000000000004) rectangle (365.49000000000007,-50.400000000000006);
\draw(345.06000000000006, -53.2) node[anchor=north west,align=left] {\large \(K\)-theory in number theory};
\draw (345.06000000000006, -53.2) rectangle (355.21000000000004,-59.1);
\draw(346.06000000000006, -54.2) node[anchor=north west,align=left] {Étale cohomology,\\ higher \\ regulators, zeta and\\ \(L\)-functions \\ (\(K\)-theoretic aspects)};
\draw (346.06000000000006, -54.2) rectangle (350.9100000000001,-56.800000000000004);
\draw(351.01000000000005, -54.2) node[anchor=north west,align=left] {Generalized\\ class field\\ theory (\(K\)-theoretic\\ aspects)};
\draw (351.01000000000005, -54.2) rectangle (355.11000000000007,-56.300000000000004);
\draw(346.06000000000006, -56.900000000000006) node[anchor=north west,align=left] {Symbols and\\ arithmetic\\ (\(K\)-theoretic\\ aspects)};
\draw (346.06000000000006, -56.900000000000006) rectangle (349.6600000000001,-59.00000000000001);
\draw(355.31000000000006, -53.2) node[anchor=north west,align=left] {\large Whitehead groups and \(K_1\)};
\draw (355.31000000000006, -53.2) rectangle (365.1600000000001,-57.6);
\draw(356.31000000000006, -54.2) node[anchor=north west,align=left] {\(K_1\) of \\ group rings\\ and orders};
\draw (356.31000000000006, -54.2) rectangle (359.4100000000001,-55.800000000000004);
\draw(359.51000000000005, -54.2) node[anchor=north west,align=left] {Stability\\ for \\ linear groups};
\draw (359.51000000000005, -54.2) rectangle (362.36000000000007,-55.800000000000004);
\draw(362.46000000000004, -54.2) node[anchor=north west,align=left] {Congruence\\ subgroup\\ problems};
\draw (362.46000000000004, -54.2) rectangle (365.06000000000006,-55.800000000000004);
\draw(356.31000000000006, -55.900000000000006) node[anchor=north west,align=left] {Stable\\ range \\ conditions};
\draw (356.31000000000006, -55.900000000000006) rectangle (358.6600000000001,-57.50000000000001);
\draw(345.06000000000006, -59.2) node[anchor=north west,align=left] {\large Steinberg groups and \(K_2\)};
\draw (345.06000000000006, -59.2) rectangle (354.34000000000003,-64.10000000000001);
\draw(346.06000000000006, -60.2) node[anchor=north west,align=left] {Symbols, \\ presentations\\ and \\ stability of \(K_2\)};
\draw (346.06000000000006, -60.2) rectangle (350.1600000000001,-62.300000000000004);
\draw(350.26000000000005, -60.2) node[anchor=north west,align=left] {Central \\ extensions\\ and Schur\\ multipliers};
\draw (350.26000000000005, -60.2) rectangle (353.61000000000007,-62.300000000000004);
\draw(346.06000000000006, -62.400000000000006) node[anchor=north west,align=left] {Excision\\ for \(K_2\)};
\draw (346.06000000000006, -62.400000000000006) rectangle (348.9100000000001,-63.50000000000001);
\draw(349.01000000000005, -62.400000000000006) node[anchor=north west,align=left] {\(K_2\) and\\ the \\ Brauer group};
\draw (349.01000000000005, -62.400000000000006) rectangle (351.36000000000007,-64.0);
\draw(354.44000000000005, -59.2) node[anchor=north west,align=left] {\large Topological \(K\)-theory};
\draw (354.44000000000005, -59.2) rectangle (363.59000000000003,-66.3);
\draw(355.44000000000005, -60.2) node[anchor=north west,align=left] {Equivariant\\ \(K\)-theory};
\draw (355.44000000000005, -60.2) rectangle (360.0400000000001,-61.800000000000004);
\draw(360.14000000000004, -60.2) node[anchor=north west,align=left] {\(J\)-homomorphism,\\ Adams\\ operations};
\draw (360.14000000000004, -60.2) rectangle (363.49000000000007,-61.800000000000004);
\draw(355.44000000000005, -61.900000000000006) node[anchor=north west,align=left] {Geometric \\ applications\\ of topological\\ \(K\)-theory};
\draw (355.44000000000005, -61.900000000000006) rectangle (359.7900000000001,-64.0);
\draw(359.89000000000004, -61.900000000000006) node[anchor=north west,align=left] {Riemann-Roch\\ theorems,\\ Chern\\ characters};
\draw (359.89000000000004, -61.900000000000006) rectangle (362.99000000000007,-64.0);
\draw(355.44000000000005, -64.10000000000001) node[anchor=north west,align=left] {Twisted \\ \(K\)-theory;\\ differential\\ \(K\)-theory};
\draw (355.44000000000005, -64.10000000000001) rectangle (359.5400000000001,-66.2);
\draw(359.64000000000004, -64.10000000000001) node[anchor=north west,align=left] {Connective\\ \(K\)-theory,\\ cobordism};
\draw (359.64000000000004, -64.10000000000001) rectangle (362.74000000000007,-65.7);
\draw(345.06000000000006, -66.4) node[anchor=north west,align=left] {\large Obstructions from topology};
\draw (345.06000000000006, -66.4) rectangle (353.7200000000001,-71.80000000000001);
\draw(346.06000000000006, -67.4) node[anchor=north west,align=left] {Obstructions\\ to group \\ actions (\(K\)-theoretic\\ aspects)};
\draw (346.06000000000006, -67.4) rectangle (349.9100000000001,-69.5);
\draw(350.01000000000005, -67.4) node[anchor=north west,align=left] {Finiteness\\ and other\\ obstructions\\ in \(K_0\)};
\draw (350.01000000000005, -67.4) rectangle (353.36000000000007,-69.5);
\draw(346.06000000000006, -69.60000000000001) node[anchor=north west,align=left] {Surgery \\ obstructions\\ (\(K\)-theoretic\\ aspects)};
\draw (346.06000000000006, -69.60000000000001) rectangle (349.1600000000001,-71.7);
\draw(349.26000000000005, -69.60000000000001) node[anchor=north west,align=left] {Whitehead\\ (and related)\\ torsion};
\draw (349.26000000000005, -69.60000000000001) rectangle (352.11000000000007,-71.2);
\draw(353.82000000000005, -66.4) node[anchor=north west,align=left] {\large \(K\)-theory of forms};
\draw (353.82000000000005, -66.4) rectangle (361.97,-71.30000000000001);
\draw(354.82000000000005, -67.4) node[anchor=north west,align=left] {Hermitian \\ \(K\)-theory, \\ relations with \\ \(K\)-theory of rings};
\draw (354.82000000000005, -67.4) rectangle (358.9200000000001,-69.5);
\draw(359.02000000000004, -67.4) node[anchor=north west,align=left] {Stability\\ for quadratic\\ modules};
\draw (359.02000000000004, -67.4) rectangle (361.87000000000006,-69.0);
\draw(354.82000000000005, -69.60000000000001) node[anchor=north west,align=left] {\(L\)-theory\\ of \\ group rings};
\draw (354.82000000000005, -69.60000000000001) rectangle (357.4200000000001,-71.2);
\draw(357.52000000000004, -69.60000000000001) node[anchor=north west,align=left] {Witt \\ groups\\ of rings};
\draw (357.52000000000004, -69.60000000000001) rectangle (359.62000000000006,-71.2);
\draw(362.07000000000005, -66.4) node[anchor=north west,align=left] {\large \(K\)-theory in geometry};
\draw (362.07000000000005, -66.4) rectangle (370.11000000000007,-71.80000000000001);
\draw(363.07000000000005, -67.4) node[anchor=north west,align=left] {Algebraic \\ cycles and \\ motivic cohomology\\ (\(K\)-theoretic\\ aspects)};
\draw (363.07000000000005, -67.4) rectangle (366.4200000000001,-70.0);
\draw(366.52000000000004, -67.4) node[anchor=north west,align=left] {Relations \\ of \(K\)-theory\\ with cohomology\\ theories};
\draw (366.52000000000004, -67.4) rectangle (369.87000000000006,-69.5);
\draw(363.07000000000005, -70.10000000000001) node[anchor=north west,align=left] {\(K\)-theory\\ of\\ schemes};
\draw (363.07000000000005, -70.10000000000001) rectangle (365.1700000000001,-71.7);
\draw(344.06000000000006, -72.0) node[anchor=north west,align=left] {\LARGE Field theory and polynomials};
\draw (344.06000000000006, -72.0) rectangle (368.5300000000001,-95.6);
\draw(345.06000000000006, -73.0) node[anchor=north west,align=left] {\large Connections between field theory and logic};
\draw (345.06000000000006, -73.0) rectangle (358.68000000000006,-76.2);
\draw(346.06000000000006, -74.0) node[anchor=north west,align=left] {Ultraproducts\\ and\\ field theory};
\draw (346.06000000000006, -74.0) rectangle (349.1600000000001,-75.6);
\draw(349.26000000000005, -74.0) node[anchor=north west,align=left] {Nonstandard\\ arithmetic\\ and\\ field theory};
\draw (349.26000000000005, -74.0) rectangle (352.36000000000007,-76.1);
\draw(352.46000000000004, -74.0) node[anchor=north west,align=left] {Decidability\\ and \\ field theory};
\draw (352.46000000000004, -74.0) rectangle (355.31000000000006,-75.6);
\draw(355.4100000000001, -74.0) node[anchor=north west,align=left] {Model \\ theory \\ of fields};
\draw (355.4100000000001, -74.0) rectangle (357.5100000000001,-75.6);
\draw(358.7800000000001, -73.0) node[anchor=north west,align=left] {\large Real and complex fields};
\draw (358.7800000000001, -73.0) rectangle (368.43000000000006,-79.9);
\draw(359.7800000000001, -74.0) node[anchor=north west,align=left] {Fields related\\ with sums \\ of squares \\ (formally real \\ fields, Pythagorean\\ fields, etc.)};
\draw (359.7800000000001, -74.0) rectangle (364.6300000000001,-77.1);
\draw(364.7300000000001, -74.0) node[anchor=north west,align=left] {Polynomials \\ in real and \\ complex fields:\\ factorization};
\draw (364.7300000000001, -74.0) rectangle (368.3300000000001,-76.1);
\draw(359.7800000000001, -77.2) node[anchor=north west,align=left] {Polynomials in\\ real and complex\\ fields: location\\ of zeros \\ (algebraic theorems)};
\draw (359.7800000000001, -77.2) rectangle (364.3800000000001,-79.8);
\draw(345.06000000000006, -76.3) node[anchor=north west,align=left] {\large Homological methods (field theory)};
\draw (345.06000000000006, -76.3) rectangle (356.20000000000005,-79.0);
\draw(346.06000000000006, -77.3) node[anchor=north west,align=left] {Cohomological\\ dimension\\ of fields};
\draw (346.06000000000006, -77.3) rectangle (349.4100000000001,-78.89999999999999);
\draw(349.51000000000005, -77.3) node[anchor=north west,align=left] {Galois \\ cohomology};
\draw (349.51000000000005, -77.3) rectangle (351.86000000000007,-78.39999999999999);
\draw(345.06000000000006, -80.0) node[anchor=north west,align=left] {\large Differential and difference algebra};
\draw (345.06000000000006, -80.0) rectangle (356.51000000000005,-84.4);
\draw(346.06000000000006, -81.0) node[anchor=north west,align=left] {Differential\\ algebra};
\draw (346.06000000000006, -81.0) rectangle (349.6600000000001,-82.6);
\draw(349.76000000000005, -81.0) node[anchor=north west,align=left] {Difference\\ algebra};
\draw (349.76000000000005, -81.0) rectangle (353.11000000000007,-82.6);
\draw(353.21000000000004, -81.0) node[anchor=north west,align=left] {\(p\)-adic \\ differential\\ equations};
\draw (353.21000000000004, -81.0) rectangle (356.31000000000006,-82.6);
\draw(346.06000000000006, -82.7) node[anchor=north west,align=left] {Abstract \\ differential\\ equations};
\draw (346.06000000000006, -82.7) rectangle (348.9100000000001,-84.3);
\draw(356.61000000000007, -80.0) node[anchor=north west,align=left] {\large General field theory};
\draw (356.61000000000007, -80.0) rectangle (367.76000000000005,-87.6);
\draw(357.61000000000007, -81.0) node[anchor=north west,align=left] {Hilbertian \\ fields; Hilbert’s\\ irreducibility theorem};
\draw (357.61000000000007, -81.0) rectangle (364.4600000000001,-83.1);
\draw(364.56000000000006, -81.0) node[anchor=north west,align=left] {Finite \\ fields \\ (field-theoretic\\ aspects)};
\draw (364.56000000000006, -81.0) rectangle (367.6600000000001,-83.1);
\draw(357.61000000000007, -83.2) node[anchor=north west,align=left] {Polynomials\\ in general \\ fields \\ (irreducibility, etc.)};
\draw (357.61000000000007, -83.2) rectangle (361.7100000000001,-85.3);
\draw(361.81000000000006, -83.2) node[anchor=north west,align=left] {Equations\\ in \\ general fields};
\draw (361.81000000000006, -83.2) rectangle (364.6600000000001,-84.8);
\draw(364.76000000000005, -83.2) node[anchor=north west,align=left] {Skew \\ fields, \\ division rings};
\draw (364.76000000000005, -83.2) rectangle (367.61000000000007,-84.8);
\draw(357.61000000000007, -85.4) node[anchor=north west,align=left] {Special \\ polynomials\\ in general\\ fields};
\draw (357.61000000000007, -85.4) rectangle (360.2100000000001,-87.5);
\draw(360.31000000000006, -85.4) node[anchor=north west,align=left] {Field \\ arithmetic};
\draw (360.31000000000006, -85.4) rectangle (362.6600000000001,-86.5);
\draw(345.06000000000006, -84.5) node[anchor=north west,align=left] {\large Generalizations of fields};
\draw (345.06000000000006, -84.5) rectangle (353.4100000000001,-86.7);
\draw(346.06000000000006, -85.5) node[anchor=north west,align=left] {Near-fields};
\draw (346.06000000000006, -85.5) rectangle (348.9100000000001,-86.6);
\draw(349.01000000000005, -85.5) node[anchor=north west,align=left] {Semifields};
\draw (349.01000000000005, -85.5) rectangle (351.61000000000007,-86.6);
\draw(345.06000000000006, -87.7) node[anchor=north west,align=left] {\large Topological fields};
\draw (345.06000000000006, -87.7) rectangle (354.21000000000004,-95.5);
\draw(346.06000000000006, -88.7) node[anchor=north west,align=left] {Non-Archimedean\\ valued fields};
\draw (346.06000000000006, -88.7) rectangle (351.1600000000001,-90.3);
\draw(351.26000000000005, -88.7) node[anchor=north west,align=left] {General\\ valuation\\ theory\\ for fields};
\draw (351.26000000000005, -88.7) rectangle (354.11000000000007,-90.8);
\draw(346.06000000000006, -90.9) node[anchor=north west,align=left] {Topological\\ semifields};
\draw (346.06000000000006, -90.9) rectangle (349.9100000000001,-92.5);
\draw(350.01000000000005, -90.9) node[anchor=north west,align=left] {Krasner-Tate\\ algebras};
\draw (350.01000000000005, -90.9) rectangle (353.86000000000007,-92.5);
\draw(346.06000000000006, -92.60000000000001) node[anchor=north west,align=left] {Formally\\ \(p\)-adic\\ fields};
\draw (346.06000000000006, -92.60000000000001) rectangle (348.1600000000001,-94.2);
\draw(348.26000000000005, -92.60000000000001) node[anchor=north west,align=left] {Ordered\\ fields};
\draw (348.26000000000005, -92.60000000000001) rectangle (350.36000000000007,-93.7);
\draw(350.46000000000004, -92.60000000000001) node[anchor=north west,align=left] {Normed\\ fields};
\draw (350.46000000000004, -92.60000000000001) rectangle (352.31000000000006,-93.7);
\draw(346.06000000000006, -94.3) node[anchor=north west,align=left] {Valued\\ fields};
\draw (346.06000000000006, -94.3) rectangle (347.9100000000001,-95.39999999999999);
\draw(354.31000000000006, -87.7) node[anchor=north west,align=left] {\large Field extensions};
\draw (354.31000000000006, -87.7) rectangle (360.96000000000004,-94.3);
\draw(355.31000000000006, -88.7) node[anchor=north west,align=left] {Separable\\ extensions,\\ Galois theory};
\draw (355.31000000000006, -88.7) rectangle (360.1600000000001,-90.8);
\draw(355.31000000000006, -90.9) node[anchor=north west,align=left] {Transcendental\\ field\\ extensions};
\draw (355.31000000000006, -90.9) rectangle (358.1600000000001,-92.5);
\draw(358.26000000000005, -90.9) node[anchor=north west,align=left] {Algebraic\\ field\\ extensions};
\draw (358.26000000000005, -90.9) rectangle (360.86000000000007,-92.5);
\draw(355.31000000000006, -92.60000000000001) node[anchor=north west,align=left] {Inseparable\\ field\\ extensions};
\draw (355.31000000000006, -92.60000000000001) rectangle (357.9100000000001,-94.2);
\draw(358.01000000000005, -92.60000000000001) node[anchor=north west,align=left] {Inverse\\ Galois\\ theory};
\draw (358.01000000000005, -92.60000000000001) rectangle (360.36000000000007,-94.2);
\draw(354.31000000000006, -94.4) node[anchor=north west,align=left] {\large History of\\ field theory};
\draw (354.31000000000006, -94.4) rectangle (358.01000000000005,-95.5);
\draw(361.06000000000006, -87.7) node[anchor=north west,align=left] {\large Computational methods\\ for problems \\ pertaining to field theory};
\draw (361.06000000000006, -87.7) rectangle (367.55000000000007,-89.3);
\draw(344.06000000000006, -95.7) node[anchor=north west,align=left] {\LARGE Order, lattices, ordered algebraic structures};
\draw (344.06000000000006, -95.7) rectangle (368.1000000000001,-129.1);
\draw(345.06000000000006, -96.7) node[anchor=north west,align=left] {\large Modular lattices, complemented lattices};
\draw (345.06000000000006, -96.7) rectangle (357.75000000000006,-102.60000000000001);
\draw(346.06000000000006, -97.7) node[anchor=north west,align=left] {Complemented\\ lattices,\\ orthocomplemented\\ lattices and posets};
\draw (346.06000000000006, -97.7) rectangle (352.9100000000001,-100.3);
\draw(353.01000000000005, -97.7) node[anchor=north west,align=left] {Complemented\\ modular \\ lattices, \\ continuous geometries};
\draw (353.01000000000005, -97.7) rectangle (357.11000000000007,-99.8);
\draw(346.06000000000006, -100.4) node[anchor=north west,align=left] {Semimodular\\ lattices,\\ geometric\\ lattices};
\draw (346.06000000000006, -100.4) rectangle (349.4100000000001,-102.5);
\draw(349.51000000000005, -100.4) node[anchor=north west,align=left] {Modular \\ lattices,\\ Desarguesian\\ lattices};
\draw (349.51000000000005, -100.4) rectangle (352.61000000000007,-102.5);
\draw(357.8500000000001, -96.7) node[anchor=north west,align=left] {\large Distributive lattices};
\draw (357.8500000000001, -96.7) rectangle (368.00000000000006,-109.2);
\draw(358.8500000000001, -97.7) node[anchor=north west,align=left] {Post algebras\\ (lattice-theoretic\\ aspects)};
\draw (358.8500000000001, -97.7) rectangle (364.2000000000001,-99.8);
\draw(364.30000000000007, -97.7) node[anchor=north west,align=left] {Heyting \\ algebras \\ (lattice-theoretic\\ aspects)};
\draw (364.30000000000007, -97.7) rectangle (367.9000000000001,-99.8);
\draw(358.8500000000001, -99.9) node[anchor=north west,align=left] {Pseudocomplemented\\ lattices};
\draw (358.8500000000001, -99.9) rectangle (363.7000000000001,-101.5);
\draw(363.80000000000007, -99.9) node[anchor=north west,align=left] {Structure \\ and representation\\ theory\\ of \\ distributive lattices};
\draw (363.80000000000007, -99.9) rectangle (367.9000000000001,-102.5);
\draw(358.8500000000001, -102.60000000000001) node[anchor=north west,align=left] {Complete\\ distributivity};
\draw (358.8500000000001, -102.60000000000001) rectangle (362.9500000000001,-104.2);
\draw(363.05000000000007, -102.60000000000001) node[anchor=north west,align=left] {De Morgan \\ algebras, Łukasiewicz\\ algebras\\ (lattice-theoretic\\ aspects)};
\draw (363.05000000000007, -102.60000000000001) rectangle (367.1500000000001,-105.2);
\draw(358.8500000000001, -105.30000000000001) node[anchor=north west,align=left] {Fuzzy lattices\\ (soft \\ algebras) and \\ related topics};
\draw (358.8500000000001, -105.30000000000001) rectangle (362.2000000000001,-107.4);
\draw(362.30000000000007, -105.30000000000001) node[anchor=north west,align=left] {Other \\ generalizations\\ of distributive\\ lattices};
\draw (362.30000000000007, -105.30000000000001) rectangle (365.6500000000001,-107.4);
\draw(365.75000000000006, -105.30000000000001) node[anchor=north west,align=left] {Frames,\\ locales};
\draw (365.75000000000006, -105.30000000000001) rectangle (367.8500000000001,-106.4);
\draw(358.8500000000001, -107.5) node[anchor=north west,align=left] {MV-algebras};
\draw (358.8500000000001, -107.5) rectangle (361.7000000000001,-108.6);
\draw(361.80000000000007, -107.5) node[anchor=north west,align=left] {Lattices\\ and\\ duality};
\draw (361.80000000000007, -107.5) rectangle (363.9000000000001,-109.1);
\draw(345.06000000000006, -102.7) node[anchor=north west,align=left] {\large Computational methods\\ for problems pertaining\\ to ordered structures};
\draw (345.06000000000006, -102.7) rectangle (352.1700000000001,-104.3);
\draw(345.06000000000006, -104.4) node[anchor=north west,align=left] {\large History of \\ ordered structures};
\draw (345.06000000000006, -104.4) rectangle (349.69000000000005,-105.5);
\draw(345.06000000000006, -109.3) node[anchor=north west,align=left] {\large Boolean algebras (Boolean rings)};
\draw (345.06000000000006, -109.3) rectangle (356.6600000000001,-117.39999999999999);
\draw(346.06000000000006, -110.3) node[anchor=north west,align=left] {Boolean algebras\\ with additional\\ operations\\ (diagonalizable\\ algebras, etc.)};
\draw (346.06000000000006, -110.3) rectangle (350.9100000000001,-112.89999999999999);
\draw(351.01000000000005, -110.3) node[anchor=north west,align=left] {Generalizationsof\\ Boolean\\ algebras};
\draw (351.01000000000005, -110.3) rectangle (355.86000000000007,-112.39999999999999);
\draw(346.06000000000006, -113.0) node[anchor=north west,align=left] {Stone spaces\\ (Boolean \\ spaces) and \\ related structures};
\draw (346.06000000000006, -113.0) rectangle (350.1600000000001,-115.1);
\draw(350.26000000000005, -113.0) node[anchor=north west,align=left] {Structure\\ theory\\ of Boolean\\ algebras};
\draw (350.26000000000005, -113.0) rectangle (353.36000000000007,-115.1);
\draw(353.46000000000004, -113.0) node[anchor=north west,align=left] {Ring-theoretic\\ properties\\ of Boolean\\ algebras};
\draw (353.46000000000004, -113.0) rectangle (356.56000000000006,-115.1);
\draw(346.06000000000006, -115.2) node[anchor=north west,align=left] {Chain \\ conditions,\\ complete\\ algebras};
\draw (346.06000000000006, -115.2) rectangle (348.9100000000001,-117.3);
\draw(349.01000000000005, -115.2) node[anchor=north west,align=left] {Boolean\\ functions};
\draw (349.01000000000005, -115.2) rectangle (351.36000000000007,-116.3);
\draw(356.76000000000005, -109.3) node[anchor=north west,align=left] {\large Lattices};
\draw (356.76000000000005, -109.3) rectangle (366.86000000000007,-118.1);
\draw(357.76000000000005, -110.3) node[anchor=north west,align=left] {Topologicallattices};
\draw (357.76000000000005, -110.3) rectangle (362.86000000000007,-111.89999999999999);
\draw(362.96000000000004, -110.3) node[anchor=north west,align=left] {Continuous\\ lattices\\ and posets,\\ applications};
\draw (362.96000000000004, -110.3) rectangle (366.56000000000006,-112.39999999999999);
\draw(357.76000000000005, -112.5) node[anchor=north west,align=left] {Generalizationsof\\ lattices};
\draw (357.76000000000005, -112.5) rectangle (362.61000000000007,-114.1);
\draw(362.71000000000004, -112.5) node[anchor=north west,align=left] {Representation\\ theory\\ of lattices};
\draw (362.71000000000004, -112.5) rectangle (366.06000000000006,-114.1);
\draw(357.76000000000005, -114.2) node[anchor=north west,align=left] {Free lattices,\\ projective\\ lattices,\\ word problems};
\draw (357.76000000000005, -114.2) rectangle (361.11000000000007,-116.3);
\draw(361.21000000000004, -114.2) node[anchor=north west,align=left] {Lattice\\ ideals,\\ congruence\\ relations};
\draw (361.21000000000004, -114.2) rectangle (364.31000000000006,-116.3);
\draw(364.41, -114.2) node[anchor=north west,align=left] {Varieties\\ of\\ lattices};
\draw (364.41, -114.2) rectangle (366.76000000000005,-115.8);
\draw(357.76000000000005, -116.39999999999999) node[anchor=north west,align=left] {Complete\\ lattices,\\ completions};
\draw (357.76000000000005, -116.39999999999999) rectangle (360.86000000000007,-117.99999999999999);
\draw(360.96000000000004, -116.39999999999999) node[anchor=north west,align=left] {Structure\\ theory \\ of lattices};
\draw (360.96000000000004, -116.39999999999999) rectangle (363.56000000000006,-117.99999999999999);
\draw(345.06000000000006, -118.2) node[anchor=north west,align=left] {\large Ordered structures};
\draw (345.06000000000006, -118.2) rectangle (352.46000000000004,-129.0);
\draw(346.06000000000006, -119.2) node[anchor=north west,align=left] {Ordered \\ abelian groups,\\ Riesz \\ groups, ordered\\ linear spaces};
\draw (346.06000000000006, -119.2) rectangle (349.9100000000001,-121.8);
\draw(350.01000000000005, -119.2) node[anchor=north west,align=left] {Quantales};
\draw (350.01000000000005, -119.2) rectangle (352.36000000000007,-120.3);
\draw(350.01000000000005, -120.4) node[anchor=north west,align=left] {Noether\\ lattices};
\draw (350.01000000000005, -120.4) rectangle (352.36000000000007,-121.5);
\draw(346.06000000000006, -121.9) node[anchor=north west,align=left] {Ordered \\ topological \\ structures (aspects\\ of ordered\\ structures)};
\draw (346.06000000000006, -121.9) rectangle (349.6600000000001,-124.5);
\draw(349.76000000000005, -121.9) node[anchor=north west,align=left] {Ordered\\ groups};
\draw (349.76000000000005, -121.9) rectangle (351.86000000000007,-123.0);
\draw(346.06000000000006, -124.60000000000001) node[anchor=north west,align=left] {BCK-algebras,\\ BCI-algebras\\ (aspects\\ of ordered\\ structures)};
\draw (346.06000000000006, -124.60000000000001) rectangle (349.6600000000001,-127.2);
\draw(346.06000000000006, -127.30000000000001) node[anchor=north west,align=left] {Ordered \\ semigroups\\ and monoids};
\draw (346.06000000000006, -127.30000000000001) rectangle (349.1600000000001,-128.9);
\draw(349.26000000000005, -127.30000000000001) node[anchor=north west,align=left] {Ordered \\ rings, algebras,\\ modules};
\draw (349.26000000000005, -127.30000000000001) rectangle (352.36000000000007,-128.9);
\draw(352.56000000000006, -118.2) node[anchor=north west,align=left] {\large Ordered sets};
\draw (352.56000000000006, -118.2) rectangle (359.71000000000004,-127.5);
\draw(353.56000000000006, -119.2) node[anchor=north west,align=left] {Galois \\ correspondences, closure\\ operators\\ (in relation \\ to ordered sets)};
\draw (353.56000000000006, -119.2) rectangle (357.4100000000001,-121.8);
\draw(357.51000000000005, -119.2) node[anchor=north west,align=left] {Total\\ orders};
\draw (357.51000000000005, -119.2) rectangle (359.36000000000007,-120.3);
\draw(353.56000000000006, -121.9) node[anchor=north west,align=left] {Combinatorics\\ of \\ partially\\ ordered sets};
\draw (353.56000000000006, -121.9) rectangle (356.6600000000001,-124.0);
\draw(356.76000000000005, -121.9) node[anchor=north west,align=left] {Algebraic\\ aspects\\ of posets};
\draw (356.76000000000005, -121.9) rectangle (359.61000000000007,-123.5);
\draw(353.56000000000006, -124.10000000000001) node[anchor=north west,align=left] {Semilattices};
\draw (353.56000000000006, -124.10000000000001) rectangle (356.6600000000001,-125.2);
\draw(356.76000000000005, -124.10000000000001) node[anchor=north west,align=left] {Generalizations\\ of \\ ordered sets};
\draw (356.76000000000005, -124.10000000000001) rectangle (359.61000000000007,-125.7);
\draw(353.56000000000006, -125.8) node[anchor=north west,align=left] {Partial\\ orders,\\ general};
\draw (353.56000000000006, -125.8) rectangle (355.9100000000001,-127.39999999999999);
\draw(370.9100000000001, -1) node[anchor=north west,align=left] {\LARGE Linear and multilinear algebra; matrix theory};
\draw (370.9100000000001, -1) rectangle (390.9500000000001,-32.7);
\draw(371.9100000000001, -2) node[anchor=north west,align=left] {\large Basic linear algebra};
\draw (371.9100000000001, -2) rectangle (382.31000000000006,-32.599999999999994);
\draw(372.9100000000001, -3) node[anchor=north west,align=left] {Diagonalization,\\ Jordan forms};
\draw (372.9100000000001, -3) rectangle (378.0100000000001,-4.6);
\draw(378.11000000000007, -3) node[anchor=north west,align=left] {Determinants,\\ permanents,\\ traces, \\ other special\\ matrix functions};
\draw (378.11000000000007, -3) rectangle (382.2100000000001,-5.6);
\draw(372.9100000000001, -5.7) node[anchor=north west,align=left] {Multilinear\\ algebra, \\ tensor calculus};
\draw (372.9100000000001, -5.7) rectangle (378.0100000000001,-7.800000000000001);
\draw(378.11000000000007, -5.7) node[anchor=north west,align=left] {Matrix \\ exponential and \\ similar functions\\ of matrices};
\draw (378.11000000000007, -5.7) rectangle (382.2100000000001,-7.800000000000001);
\draw(372.9100000000001, -7.9) node[anchor=north west,align=left] {Norms of matrices,\\ numerical\\ range, \\ applications of \\ functional analysis\\ to matrix theory};
\draw (372.9100000000001, -7.9) rectangle (377.5100000000001,-11.0);
\draw(377.61000000000007, -7.9) node[anchor=north west,align=left] {Conditioning\\ of matrices};
\draw (377.61000000000007, -7.9) rectangle (381.9600000000001,-9.5);
\draw(377.61000000000007, -9.6) node[anchor=north west,align=left] {Matrix\\ pencils};
\draw (377.61000000000007, -9.6) rectangle (379.7100000000001,-10.7);
\draw(372.9100000000001, -11.100000000000001) node[anchor=north west,align=left] {Vector spaces,\\ linear \\ dependence, \\ rank, lineability};
\draw (372.9100000000001, -11.100000000000001) rectangle (376.7600000000001,-13.200000000000001);
\draw(376.86000000000007, -11.100000000000001) node[anchor=north west,align=left] {Theory of \\ matrix inversion\\ and \\ generalized inverses};
\draw (376.86000000000007, -11.100000000000001) rectangle (380.7100000000001,-13.200000000000001);
\draw(372.9100000000001, -13.3) node[anchor=north west,align=left] {Inequalities\\ involving \\ eigenvalues \\ and eigenvectors};
\draw (372.9100000000001, -13.3) rectangle (376.7600000000001,-15.4);
\draw(376.86000000000007, -13.3) node[anchor=north west,align=left] {Linear \\ transformations,\\ semilinear \\ transformations};
\draw (376.86000000000007, -13.3) rectangle (380.4600000000001,-15.4);
\draw(372.9100000000001, -15.5) node[anchor=north west,align=left] {Matrices over\\ function rings\\ in one or\\ more variables};
\draw (372.9100000000001, -15.5) rectangle (376.5100000000001,-17.6);
\draw(376.61000000000007, -15.5) node[anchor=north west,align=left] {Quadratic\\ and bilinear\\ forms,\\ inner products};
\draw (376.61000000000007, -15.5) rectangle (380.2100000000001,-17.6);
\draw(372.9100000000001, -17.7) node[anchor=north west,align=left] {Applications\\ of Clifford\\ algebras to\\ physics, etc.};
\draw (372.9100000000001, -17.7) rectangle (376.5100000000001,-19.8);
\draw(376.61000000000007, -17.7) node[anchor=north west,align=left] {Vector and\\ tensor \\ algebra, theory\\ of invariants};
\draw (376.61000000000007, -17.7) rectangle (380.2100000000001,-19.8);
\draw(372.9100000000001, -19.9) node[anchor=north west,align=left] {Linear \\ equations \\ (linear algebraic\\ aspects)};
\draw (372.9100000000001, -19.9) rectangle (376.2600000000001,-22.0);
\draw(376.36000000000007, -19.9) node[anchor=north west,align=left] {Eigenvalues,\\ singular\\ values, and\\ eigenvectors};
\draw (376.36000000000007, -19.9) rectangle (379.7100000000001,-22.0);
\draw(379.81000000000006, -19.9) node[anchor=north west,align=left] {Factorization\\ of\\ matrices};
\draw (379.81000000000006, -19.9) rectangle (382.1600000000001,-21.5);
\draw(372.9100000000001, -22.099999999999998) node[anchor=north west,align=left] {Canonical\\ forms, \\ reductions, \\ classification};
\draw (372.9100000000001, -22.099999999999998) rectangle (376.2600000000001,-24.2);
\draw(376.36000000000007, -22.099999999999998) node[anchor=north west,align=left] {Other \\ algebras built\\ from modules};
\draw (376.36000000000007, -22.099999999999998) rectangle (379.7100000000001,-23.7);
\draw(379.81000000000006, -22.099999999999998) node[anchor=north west,align=left] {Commutativity\\ of\\ matrices};
\draw (379.81000000000006, -22.099999999999998) rectangle (382.1600000000001,-23.7);
\draw(372.9100000000001, -24.299999999999997) node[anchor=north west,align=left] {Applications\\ of\\ generalized\\ inverses};
\draw (372.9100000000001, -24.299999999999997) rectangle (376.0100000000001,-26.4);
\draw(376.11000000000007, -24.299999999999997) node[anchor=north west,align=left] {Miscellaneous\\ inequalities\\ involving\\ matrices};
\draw (376.11000000000007, -24.299999999999997) rectangle (379.2100000000001,-26.4);
\draw(379.31000000000006, -24.299999999999997) node[anchor=north west,align=left] {Matrix \\ equations and\\ identities};
\draw (379.31000000000006, -24.299999999999997) rectangle (382.1600000000001,-25.9);
\draw(372.9100000000001, -26.499999999999996) node[anchor=north west,align=left] {Exterior\\ algebra,\\ Grassmann\\ algebras};
\draw (372.9100000000001, -26.499999999999996) rectangle (376.0100000000001,-28.599999999999998);
\draw(376.11000000000007, -26.499999999999996) node[anchor=north west,align=left] {Algebraic\\ systems\\ of matrices};
\draw (376.11000000000007, -26.499999999999996) rectangle (378.9600000000001,-28.099999999999998);
\draw(379.06000000000006, -26.499999999999996) node[anchor=north west,align=left] {Linear \\ inequalities\\ of matrices};
\draw (379.06000000000006, -26.499999999999996) rectangle (381.9100000000001,-28.099999999999998);
\draw(372.9100000000001, -28.699999999999996) node[anchor=north west,align=left] {Max-plus\\ and related\\ algebras};
\draw (372.9100000000001, -28.699999999999996) rectangle (375.7600000000001,-30.299999999999997);
\draw(375.86000000000007, -28.699999999999996) node[anchor=north west,align=left] {Inverse\\ problems\\ in linear\\ algebra};
\draw (375.86000000000007, -28.699999999999996) rectangle (378.4600000000001,-30.799999999999997);
\draw(378.56000000000006, -28.699999999999996) node[anchor=north west,align=left] {Clifford\\ algebras,\\ spinors};
\draw (378.56000000000006, -28.699999999999996) rectangle (381.1600000000001,-30.299999999999997);
\draw(372.9100000000001, -30.899999999999995) node[anchor=north west,align=left] {Matrix \\ completion\\ problems};
\draw (372.9100000000001, -30.899999999999995) rectangle (375.5100000000001,-32.49999999999999);
\draw(375.61000000000007, -30.899999999999995) node[anchor=north west,align=left] {Linear \\ preserver\\ problems};
\draw (375.61000000000007, -30.899999999999995) rectangle (377.9600000000001,-32.49999999999999);
\draw(382.4100000000001, -2) node[anchor=north west,align=left] {\large Special matrices};
\draw (382.4100000000001, -2) rectangle (389.81000000000006,-16.7);
\draw(383.4100000000001, -3) node[anchor=north west,align=left] {Matrices over\\ special \\ rings (quaternions,\\ finite\\ fields, etc.)};
\draw (383.4100000000001, -3) rectangle (387.0100000000001,-5.6);
\draw(387.11000000000007, -3) node[anchor=north west,align=left] {Sign \\ pattern\\ matrices};
\draw (387.11000000000007, -3) rectangle (389.4600000000001,-4.6);
\draw(383.4100000000001, -5.7) node[anchor=north west,align=left] {Positive \\ matrices and \\ their \\ generalizations; cones\\ of matrices};
\draw (383.4100000000001, -5.7) rectangle (387.0100000000001,-8.3);
\draw(387.11000000000007, -5.7) node[anchor=north west,align=left] {Fuzzy\\ matrices};
\draw (387.11000000000007, -5.7) rectangle (389.2100000000001,-6.800000000000001);
\draw(383.4100000000001, -8.4) node[anchor=north west,align=left] {Hermitian,\\ skew-Hermitian,\\ and \\ related matrices};
\draw (383.4100000000001, -8.4) rectangle (387.0100000000001,-10.5);
\draw(387.11000000000007, -8.4) node[anchor=north west,align=left] {Matrices\\ of\\ integers};
\draw (387.11000000000007, -8.4) rectangle (389.2100000000001,-10.0);
\draw(383.4100000000001, -10.600000000000001) node[anchor=north west,align=left] {Orthogonal\\ matrices};
\draw (383.4100000000001, -10.600000000000001) rectangle (386.7600000000001,-12.200000000000001);
\draw(386.86000000000007, -10.600000000000001) node[anchor=north west,align=left] {Toeplitz,\\ Cauchy,\\ and related\\ matrices};
\draw (386.86000000000007, -10.600000000000001) rectangle (389.7100000000001,-12.700000000000001);
\draw(383.4100000000001, -12.8) node[anchor=north west,align=left] {Stochastic\\ matrices};
\draw (383.4100000000001, -12.8) rectangle (386.7600000000001,-14.4);
\draw(386.86000000000007, -12.8) node[anchor=north west,align=left] {Boolean \\ and Hadamard\\ matrices};
\draw (386.86000000000007, -12.8) rectangle (389.7100000000001,-14.4);
\draw(383.4100000000001, -14.5) node[anchor=north west,align=left] {Random \\ matrices\\ (algebraic\\ aspects)};
\draw (383.4100000000001, -14.5) rectangle (386.2600000000001,-16.6);
\draw(386.36000000000007, -14.5) node[anchor=north west,align=left] {Matrix\\ Lie \\ algebras};
\draw (386.36000000000007, -14.5) rectangle (388.2100000000001,-16.1);
\draw(382.4100000000001, -16.799999999999997) node[anchor=north west,align=left] {\large History of \\ linear algebra};
\draw (382.4100000000001, -16.799999999999997) rectangle (386.4200000000001,-17.9);
\draw(370.9100000000001, -32.8) node[anchor=north west,align=left] {\LARGE History and biography};
\draw (370.9100000000001, -32.8) rectangle (387.56000000000006,-54.5);
\draw(371.9100000000001, -33.8) node[anchor=north west,align=left] {\large History of mathematics and mathematicians};
\draw (371.9100000000001, -33.8) rectangle (387.4600000000001,-54.4);
\draw(372.9100000000001, -34.8) node[anchor=north west,align=left] {History of \\ mathematics of the\\ indigenous \\ cultures of Africa,\\ Asia, and Oceania};
\draw (372.9100000000001, -34.8) rectangle (377.5100000000001,-37.4);
\draw(377.61000000000007, -34.8) node[anchor=north west,align=left] {Ethnomathematics,\\ general};
\draw (377.61000000000007, -34.8) rectangle (381.9600000000001,-36.4);
\draw(382.06000000000006, -34.8) node[anchor=north west,align=left] {History of \\ mathematics \\ of the indigenous\\ cultures\\ of the Americas};
\draw (382.06000000000006, -34.8) rectangle (386.4100000000001,-37.4);
\draw(372.9100000000001, -37.5) node[anchor=north west,align=left] {History of \\ mathematics of the\\ indigenous \\ cultures of Europe\\ (pre-Greek, etc.)};
\draw (372.9100000000001, -37.5) rectangle (377.2600000000001,-40.1);
\draw(377.36000000000007, -37.5) node[anchor=north west,align=left] {History of \\ mathematics at\\ institutions\\ and academies\\ (non-university)};
\draw (377.36000000000007, -37.5) rectangle (381.7100000000001,-40.1);
\draw(381.81000000000006, -37.5) node[anchor=north west,align=left] {History of\\ mathematics\\ in late \\ antiquity and\\ medieval Europe};
\draw (381.81000000000006, -37.5) rectangle (385.9100000000001,-40.1);
\draw(372.9100000000001, -40.199999999999996) node[anchor=north west,align=left] {Collected or\\ selected works;\\ reprintings\\ or translations\\ of classics};
\draw (372.9100000000001, -40.199999999999996) rectangle (377.0100000000001,-42.8);
\draw(377.11000000000007, -40.199999999999996) node[anchor=north west,align=left] {Biographies,\\ obituaries,\\ personalia,\\ bibliographies};
\draw (377.11000000000007, -40.199999999999996) rectangle (380.9600000000001,-42.3);
\draw(381.06000000000006, -40.199999999999996) node[anchor=north west,align=left] {Bibliographic\\ studies};
\draw (381.06000000000006, -40.199999999999996) rectangle (384.9100000000001,-41.8);
\draw(385.0100000000001, -40.199999999999996) node[anchor=north west,align=left] {Schools\\ of \\ mathematics};
\draw (385.0100000000001, -40.199999999999996) rectangle (387.1100000000001,-41.8);
\draw(372.9100000000001, -42.9) node[anchor=north west,align=left] {History of \\ mathematics in\\ the 15th and\\ 16th centuries,\\ Renaissance};
\draw (372.9100000000001, -42.9) rectangle (376.5100000000001,-45.5);
\draw(376.61000000000007, -42.9) node[anchor=north west,align=left] {History of\\ mathematics\\ at specific\\ universities};
\draw (376.61000000000007, -42.9) rectangle (380.2100000000001,-45.0);
\draw(380.31000000000006, -42.9) node[anchor=north west,align=left] {Historiography};
\draw (380.31000000000006, -42.9) rectangle (383.9100000000001,-44.0);
\draw(384.0100000000001, -42.9) node[anchor=north west,align=left] {History of\\ mathematics\\ in Paleolithic\\ and \\ Neolithic times};
\draw (384.0100000000001, -42.9) rectangle (387.3600000000001,-45.5);
\draw(372.9100000000001, -45.599999999999994) node[anchor=north west,align=left] {History of \\ mathematics \\ in Ancient \\ Greece and Rome};
\draw (372.9100000000001, -45.599999999999994) rectangle (376.2600000000001,-47.699999999999996);
\draw(376.36000000000007, -45.599999999999994) node[anchor=north west,align=left] {History \\ of mathematics\\ in \\ Southeast Asia};
\draw (376.36000000000007, -45.599999999999994) rectangle (379.7100000000001,-47.699999999999996);
\draw(379.81000000000006, -45.599999999999994) node[anchor=north west,align=left] {History of \\ mathematics \\ in the Golden\\ Age of Islam};
\draw (379.81000000000006, -45.599999999999994) rectangle (383.1600000000001,-47.699999999999996);
\draw(383.2600000000001, -45.599999999999994) node[anchor=north west,align=left] {History \\ of mathematics\\ in \\ Ancient Egypt};
\draw (383.2600000000001, -45.599999999999994) rectangle (386.3600000000001,-47.699999999999996);
\draw(372.9100000000001, -47.8) node[anchor=north west,align=left] {History of\\ mathematics\\ in the\\ 17th century};
\draw (372.9100000000001, -47.8) rectangle (376.0100000000001,-49.9);
\draw(376.11000000000007, -47.8) node[anchor=north west,align=left] {History of\\ mathematics\\ in the\\ 18th century};
\draw (376.11000000000007, -47.8) rectangle (379.2100000000001,-49.9);
\draw(379.31000000000006, -47.8) node[anchor=north west,align=left] {History of\\ mathematics\\ in the\\ 19th century};
\draw (379.31000000000006, -47.8) rectangle (382.4100000000001,-49.9);
\draw(382.5100000000001, -47.8) node[anchor=north west,align=left] {History of\\ mathematics\\ in the\\ 20th century};
\draw (382.5100000000001, -47.8) rectangle (385.6100000000001,-49.9);
\draw(372.9100000000001, -50.0) node[anchor=north west,align=left] {History of\\ mathematics\\ in the\\ 21st century};
\draw (372.9100000000001, -50.0) rectangle (376.0100000000001,-52.1);
\draw(376.11000000000007, -50.0) node[anchor=north west,align=left] {Development\\ of \\ contemporary\\ mathematics};
\draw (376.11000000000007, -50.0) rectangle (379.2100000000001,-52.1);
\draw(379.31000000000006, -50.0) node[anchor=north west,align=left] {General \\ histories, \\ source books};
\draw (379.31000000000006, -50.0) rectangle (382.1600000000001,-51.6);
\draw(382.2600000000001, -50.0) node[anchor=north west,align=left] {History of\\ mathematics\\ in Ancient\\ Babylon};
\draw (382.2600000000001, -50.0) rectangle (385.1100000000001,-52.1);
\draw(372.9100000000001, -52.199999999999996) node[anchor=north west,align=left] {History of\\ mathematics\\ in China};
\draw (372.9100000000001, -52.199999999999996) rectangle (375.7600000000001,-53.8);
\draw(375.86000000000007, -52.199999999999996) node[anchor=north west,align=left] {History of\\ mathematics\\ in Japan};
\draw (375.86000000000007, -52.199999999999996) rectangle (378.7100000000001,-53.8);
\draw(378.81000000000006, -52.199999999999996) node[anchor=north west,align=left] {History of\\ mathematics\\ in India};
\draw (378.81000000000006, -52.199999999999996) rectangle (381.6600000000001,-53.8);
\draw(381.7600000000001, -52.199999999999996) node[anchor=north west,align=left] {Sociology\\ (and \\ profession) of\\ mathematics};
\draw (381.7600000000001, -52.199999999999996) rectangle (384.6100000000001,-54.3);
\draw(384.7100000000001, -52.199999999999996) node[anchor=north west,align=left] {Future \\ perspectives\\ in \\ mathematics};
\draw (384.7100000000001, -52.199999999999996) rectangle (387.3100000000001,-54.3);
\end{tikzpicture}

\end{document}