\documentclass[12pt]{article}
\usepackage[utf8]{inputenc}
\usepackage{pgf,tikz,pgfplots}
\pgfplotsset{compat=1.15}
\usepackage{mathrsfs}
\usetikzlibrary{arrows}
\usepackage{fontspec}
\setmainfont[Renderer=ICU,Mapping=tex-text]{Cousine}
\usepackage{amssymb}
\usepackage[paperwidth=401.8500000000001cm,paperheight=152.5cm,left=0.1cm,right=0.1cm,top=0.1cm,bottom=0.1cm]{geometry}
\begin{document}\begin{tikzpicture}[line cap=round,line join=round,>=triangle 45,x=1cm,y=1cm]
\clip(0, 0)rectangle(397.8500000000001, -148.5);

\draw(0, 0) node[anchor=north west,align=left] {\Huge half};
\draw (0, 0) rectangle (397.8500000000001,-148.5);
\draw(1, -1) node[anchor=north west,align=left] {\LARGE Partial differential equations};
\draw (1, -1) rectangle (76.86999999999999,-52.4);
\draw(2, -2) node[anchor=north west,align=left] {\large General higher-order partial differential equations and systems of higher-order partial differential equations};
\draw (2, -2) rectangle (42.6,-8.4);
\draw(3, -3) node[anchor=north west,align=left] {Initial value\\ problems \\ for linear \\ higher-order PDEs};
\draw (3, -3) rectangle (7.85,-5.1);
\draw(7.949999999999999, -3) node[anchor=north west,align=left] {Boundary value\\ problems for\\ linear \\ higher-order PDEs};
\draw (7.949999999999999, -3) rectangle (12.799999999999999,-5.1);
\draw(12.899999999999999, -3) node[anchor=north west,align=left] {Initial-boundary\\ value \\ problems for\\ linear \\ higher-order PDEs};
\draw (12.899999999999999, -3) rectangle (17.75,-5.6);
\draw(17.849999999999998, -3) node[anchor=north west,align=left] {Initial \\ value problems\\ for \\ nonlinear \\ higher-order PDEs};
\draw (17.849999999999998, -3) rectangle (22.699999999999996,-5.6);
\draw(22.799999999999997, -3) node[anchor=north west,align=left] {Boundary \\ value problems\\ for nonlinear\\ higher-order PDEs};
\draw (22.799999999999997, -3) rectangle (27.65,-5.6);
\draw(27.749999999999996, -3) node[anchor=north west,align=left] {Initial-boundary\\ value \\ problems for\\ nonlinear \\ higher-order PDEs};
\draw (27.749999999999996, -3) rectangle (32.599999999999994,-5.6);
\draw(32.699999999999996, -3) node[anchor=north west,align=left] {Systems\\ of linear\\ higher-order PDEs};
\draw (32.699999999999996, -3) rectangle (37.55,-5.1);
\draw(37.65, -3) node[anchor=north west,align=left] {Initial value\\ problems \\ for systems \\ of linear \\ higher-order PDEs};
\draw (37.65, -3) rectangle (42.5,-5.6);
\draw(3, -5.7) node[anchor=north west,align=left] {Boundary value\\ problems\\ for systems\\ of linear \\ higher-order PDEs};
\draw (3, -5.7) rectangle (7.85,-8.3);
\draw(7.949999999999999, -5.7) node[anchor=north west,align=left] {Initial-boundary\\ value problems\\ for systems\\ of linear \\ higher-order PDEs};
\draw (7.949999999999999, -5.7) rectangle (12.799999999999999,-8.3);
\draw(12.899999999999999, -5.7) node[anchor=north west,align=left] {Systems \\ of nonlinear\\ higher-order PDEs};
\draw (12.899999999999999, -5.7) rectangle (17.75,-7.800000000000001);
\draw(17.849999999999998, -5.7) node[anchor=north west,align=left] {Initial value\\ problems for\\ systems of \\ nonlinear \\ higher-order PDEs};
\draw (17.849999999999998, -5.7) rectangle (22.699999999999996,-8.3);
\draw(22.799999999999997, -5.7) node[anchor=north west,align=left] {Boundary value\\ problems for\\ systems of\\ nonlinear \\ higher-order PDEs};
\draw (22.799999999999997, -5.7) rectangle (27.65,-8.3);
\draw(27.749999999999996, -5.7) node[anchor=north west,align=left] {Initial-boundary\\ value problems\\ for systems\\ of nonlinear\\ higher-order PDEs};
\draw (27.749999999999996, -5.7) rectangle (32.599999999999994,-8.3);
\draw(32.699999999999996, -5.7) node[anchor=north west,align=left] {Linear \\ higher-order\\ PDEs};
\draw (32.699999999999996, -5.7) rectangle (36.3,-7.300000000000001);
\draw(36.4, -5.7) node[anchor=north west,align=left] {Nonlinear\\ higher-order\\ PDEs};
\draw (36.4, -5.7) rectangle (40.0,-7.300000000000001);
\draw(42.7, -2) node[anchor=north west,align=left] {\large Partial differential equations and systems of partial differential equations with constant coefficients};
\draw (42.7, -2) rectangle (75.23,-6.199999999999999);
\draw(43.7, -3) node[anchor=north west,align=left] {Convexity \\ properties of \\ solutions to PDEs\\ and systems of\\ PDEs with \\ constant coefficients};
\draw (43.7, -3) rectangle (49.550000000000004,-6.1);
\draw(49.650000000000006, -3) node[anchor=north west,align=left] {Initial value \\ problems for PDEs\\ and systems \\ of PDEs with \\ constant coefficients};
\draw (49.650000000000006, -3) rectangle (55.50000000000001,-5.6);
\draw(55.6, -3) node[anchor=north west,align=left] {General theory\\ of PDEs and\\ systems of \\ PDEs with \\ constant coefficients};
\draw (55.6, -3) rectangle (61.45,-5.6);
\draw(61.550000000000004, -3) node[anchor=north west,align=left] {Fundamental \\ solutions to PDEs\\ and systems of\\ PDEs with constant\\ coefficients};
\draw (61.550000000000004, -3) rectangle (66.65,-5.6);
\draw(42.7, -6.299999999999999) node[anchor=north west,align=left] {\large History of partial \\ differential equations};
\draw (42.7, -6.299999999999999) rectangle (50.120000000000005,-7.399999999999999);
\draw(2, -8.5) node[anchor=north west,align=left] {\large General first-order partial differential equations and systems of first-order partial differential equations};
\draw (2, -8.5) rectangle (41.6,-14.9);
\draw(3, -9.5) node[anchor=north west,align=left] {Hamilton-Jacobiequations};
\draw (3, -9.5) rectangle (9.6,-11.1);
\draw(9.7, -9.5) node[anchor=north west,align=left] {Initial value\\ problems \\ for linear \\ first-order PDEs};
\draw (9.7, -9.5) rectangle (14.299999999999999,-11.6);
\draw(14.399999999999999, -9.5) node[anchor=north west,align=left] {Boundary value\\ problems\\ for linear \\ first-order PDEs};
\draw (14.399999999999999, -9.5) rectangle (19.0,-11.6);
\draw(19.1, -9.5) node[anchor=north west,align=left] {Initial-boundary\\ value\\ problems \\ for linear \\ first-order PDEs};
\draw (19.1, -9.5) rectangle (23.700000000000003,-12.1);
\draw(23.8, -9.5) node[anchor=north west,align=left] {Initial value\\ problems for\\ nonlinear \\ first-order PDEs};
\draw (23.8, -9.5) rectangle (28.4,-11.6);
\draw(28.5, -9.5) node[anchor=north west,align=left] {Boundary \\ value problems\\ for \\ nonlinear \\ first-order PDEs};
\draw (28.5, -9.5) rectangle (33.1,-12.1);
\draw(33.2, -9.5) node[anchor=north west,align=left] {Initial-boundary\\ value \\ problems for\\ nonlinear \\ first-order PDEs};
\draw (33.2, -9.5) rectangle (37.800000000000004,-12.1);
\draw(37.9, -9.5) node[anchor=north west,align=left] {Systems \\ of nonlinear\\ first-order\\ PDEs};
\draw (37.9, -9.5) rectangle (41.5,-11.6);
\draw(3, -12.2) node[anchor=north west,align=left] {Initial value\\ problems \\ for systems \\ of linear \\ first-order PDEs};
\draw (3, -12.2) rectangle (7.6,-14.799999999999999);
\draw(7.699999999999999, -12.2) node[anchor=north west,align=left] {Boundary value\\ problems\\ for systems\\ of linear \\ first-order PDEs};
\draw (7.699999999999999, -12.2) rectangle (12.299999999999999,-14.799999999999999);
\draw(12.399999999999999, -12.2) node[anchor=north west,align=left] {Initial-boundary\\ value problems\\ for systems\\ of linear \\ first-order PDEs};
\draw (12.399999999999999, -12.2) rectangle (17.0,-14.799999999999999);
\draw(17.099999999999998, -12.2) node[anchor=north west,align=left] {Initial value\\ problems \\ for systems \\ of nonlinear\\ first-order PDEs};
\draw (17.099999999999998, -12.2) rectangle (21.699999999999996,-14.799999999999999);
\draw(21.799999999999997, -12.2) node[anchor=north west,align=left] {Boundary value\\ problems for\\ systems of\\ nonlinear \\ first-order PDEs};
\draw (21.799999999999997, -12.2) rectangle (26.4,-14.799999999999999);
\draw(26.499999999999996, -12.2) node[anchor=north west,align=left] {Initial-boundary\\ value problems\\ for systems\\ of nonlinear\\ first-order PDEs};
\draw (26.499999999999996, -12.2) rectangle (31.099999999999994,-14.799999999999999);
\draw(31.199999999999996, -12.2) node[anchor=north west,align=left] {Linear\\ first-order\\ PDEs};
\draw (31.199999999999996, -12.2) rectangle (34.55,-13.799999999999999);
\draw(34.65, -12.2) node[anchor=north west,align=left] {Nonlinear\\ first-order\\ PDEs};
\draw (34.65, -12.2) rectangle (38.0,-13.799999999999999);
\draw(38.1, -12.2) node[anchor=north west,align=left] {Systems\\ of linear\\ first-order\\ PDEs};
\draw (38.1, -12.2) rectangle (41.45,-14.299999999999999);
\draw(41.7, -8.5) node[anchor=north west,align=left] {\large Qualitative properties of solutions to partial differential equations};
\draw (41.7, -8.5) rectangle (67.7,-21.5);
\draw(42.7, -9.5) node[anchor=north west,align=left] {Oscillation, \\ zeros of solutions,\\ mean value \\ theorems, etc. \\ in context of PDEs};
\draw (42.7, -9.5) rectangle (48.050000000000004,-12.1);
\draw(48.150000000000006, -9.5) node[anchor=north west,align=left] {Critical points\\ of functionals\\ in context of\\ PDEs (e.g., \\ energy functionals)};
\draw (48.150000000000006, -9.5) rectangle (53.50000000000001,-12.1);
\draw(53.6, -9.5) node[anchor=north west,align=left] {Homogenization\\ in context of\\ PDEs; PDEs in\\ media with \\ periodic structure};
\draw (53.6, -9.5) rectangle (58.7,-12.1);
\draw(58.800000000000004, -9.5) node[anchor=north west,align=left] {Dependence of \\ solutions to PDEs\\ on initial \\ and/or boundary \\ data and/or on \\ parameters of PDEs};
\draw (58.800000000000004, -9.5) rectangle (63.900000000000006,-12.6);
\draw(64.0, -9.5) node[anchor=north west,align=left] {Bifurcations\\ in context\\ of PDEs};
\draw (64.0, -9.5) rectangle (67.6,-11.1);
\draw(64.0, -11.2) node[anchor=north west,align=left] {Attractors};
\draw (64.0, -11.2) rectangle (67.1,-12.299999999999999);
\draw(42.7, -12.7) node[anchor=north west,align=left] {Almost and \\ pseudo-almost\\ periodic \\ solutions to PDEs};
\draw (42.7, -12.7) rectangle (47.550000000000004,-14.799999999999999);
\draw(47.650000000000006, -12.7) node[anchor=north west,align=left] {Liouville \\ theorems and \\ Phragmén-Lindelöf\\ theorems in\\ context of PDEs};
\draw (47.650000000000006, -12.7) rectangle (52.50000000000001,-15.299999999999999);
\draw(52.6, -12.7) node[anchor=north west,align=left] {Continuation\\ and prolongation\\ of \\ solutions to PDEs};
\draw (52.6, -12.7) rectangle (57.45,-14.799999999999999);
\draw(57.55, -12.7) node[anchor=north west,align=left] {Smoothness\\ and regularity\\ of \\ solutions to PDEs};
\draw (57.55, -12.7) rectangle (62.4,-14.799999999999999);
\draw(62.5, -12.7) node[anchor=north west,align=left] {Symmetries,\\ invariants,\\ etc. in \\ context of PDEs};
\draw (62.5, -12.7) rectangle (66.85,-14.799999999999999);
\draw(42.7, -15.4) node[anchor=north west,align=left] {Perturbations\\ in \\ context of PDEs};
\draw (42.7, -15.4) rectangle (47.050000000000004,-17.0);
\draw(47.150000000000006, -15.4) node[anchor=north west,align=left] {Singular \\ perturbations\\ in context\\ of PDEs};
\draw (47.150000000000006, -15.4) rectangle (51.00000000000001,-17.5);
\draw(51.1, -15.4) node[anchor=north west,align=left] {Asymptotic\\ behavior\\ of solutions\\ to PDEs};
\draw (51.1, -15.4) rectangle (54.7,-17.5);
\draw(54.800000000000004, -15.4) node[anchor=north west,align=left] {Critical\\ exponents\\ in context\\ of PDEs};
\draw (54.800000000000004, -15.4) rectangle (57.900000000000006,-17.5);
\draw(58.0, -15.4) node[anchor=north west,align=left] {Resonance\\ in context\\ of PDEs};
\draw (58.0, -15.4) rectangle (61.1,-17.0);
\draw(61.2, -15.4) node[anchor=north west,align=left] {Stability\\ in context\\ of PDEs};
\draw (61.2, -15.4) rectangle (64.3,-17.0);
\draw(64.4, -15.4) node[anchor=north west,align=left] {Pattern \\ formations\\ in context\\ of PDEs};
\draw (64.4, -15.4) rectangle (67.5,-17.5);
\draw(42.7, -17.6) node[anchor=north west,align=left] {Blow-up\\ in context\\ of PDEs};
\draw (42.7, -17.6) rectangle (45.800000000000004,-19.200000000000003);
\draw(45.900000000000006, -17.6) node[anchor=north west,align=left] {A priori\\ estimates\\ in context\\ of PDEs};
\draw (45.900000000000006, -17.6) rectangle (49.00000000000001,-19.700000000000003);
\draw(49.1, -17.6) node[anchor=north west,align=left] {Maximum \\ principles\\ in context\\ of PDEs};
\draw (49.1, -17.6) rectangle (52.2,-19.700000000000003);
\draw(52.300000000000004, -17.6) node[anchor=north west,align=left] {Comparison\\ principles\\ in context\\ of PDEs};
\draw (52.300000000000004, -17.6) rectangle (55.400000000000006,-19.700000000000003);
\draw(55.5, -17.6) node[anchor=north west,align=left] {Axially\\ symmetric\\ solutions\\ to PDEs};
\draw (55.5, -17.6) rectangle (58.35,-19.700000000000003);
\draw(58.45, -17.6) node[anchor=north west,align=left] {Entire \\ solutions\\ to PDEs};
\draw (58.45, -17.6) rectangle (61.300000000000004,-19.200000000000003);
\draw(61.400000000000006, -17.6) node[anchor=north west,align=left] {Positive\\ solutions\\ to PDEs};
\draw (61.400000000000006, -17.6) rectangle (64.25,-19.200000000000003);
\draw(64.35, -17.6) node[anchor=north west,align=left] {Periodic\\ solutions\\ to PDEs};
\draw (64.35, -17.6) rectangle (67.19999999999999,-19.200000000000003);
\draw(42.7, -19.8) node[anchor=north west,align=left] {Inertial\\ manifolds};
\draw (42.7, -19.8) rectangle (45.550000000000004,-21.400000000000002);
\draw(2, -15.0) node[anchor=north west,align=left] {\large Representations of solutions to partial differential equations};
\draw (2, -15.0) rectangle (25.450000000000003,-19.9);
\draw(3, -16.0) node[anchor=north west,align=left] {Integral \\ representations\\ of solutions\\ to PDEs};
\draw (3, -16.0) rectangle (7.35,-18.1);
\draw(7.449999999999999, -16.0) node[anchor=north west,align=left] {Trigonometric\\ solutions\\ to PDEs};
\draw (7.449999999999999, -16.0) rectangle (11.299999999999999,-17.6);
\draw(11.4, -16.0) node[anchor=north west,align=left] {Self-similar\\ solutions\\ to PDEs};
\draw (11.4, -16.0) rectangle (15.0,-17.6);
\draw(15.1, -16.0) node[anchor=north west,align=left] {Asymptotic\\ expansions\\ of solutions\\ to PDEs};
\draw (15.1, -16.0) rectangle (18.7,-18.1);
\draw(18.8, -16.0) node[anchor=north west,align=left] {Solutions\\ to PDEs in\\ closed form};
\draw (18.8, -16.0) rectangle (22.150000000000002,-17.6);
\draw(22.25, -16.0) node[anchor=north west,align=left] {Polynomial\\ solutions\\ to PDEs};
\draw (22.25, -16.0) rectangle (25.35,-17.6);
\draw(3, -18.2) node[anchor=north west,align=left] {Traveling\\ wave \\ solutions};
\draw (3, -18.2) rectangle (5.85,-19.8);
\draw(5.95, -18.2) node[anchor=north west,align=left] {Soliton\\ solutions};
\draw (5.95, -18.2) rectangle (8.8,-19.8);
\draw(8.9, -18.2) node[anchor=north west,align=left] {Series \\ solutions\\ to PDEs};
\draw (8.9, -18.2) rectangle (11.75,-19.8);
\draw(67.8, -8.5) node[anchor=north west,align=left] {\large Close-to-elliptic equations};
\draw (67.8, -8.5) rectangle (76.77,-12.9);
\draw(68.8, -9.5) node[anchor=north west,align=left] {Quasiellipticequations};
\draw (68.8, -9.5) rectangle (74.89999999999999,-11.1);
\draw(68.8, -11.2) node[anchor=north west,align=left] {Hypoelliptic\\ equations};
\draw (68.8, -11.2) rectangle (72.39999999999999,-12.799999999999999);
\draw(72.5, -11.2) node[anchor=north west,align=left] {Subelliptic\\ equations};
\draw (72.5, -11.2) rectangle (75.85,-12.799999999999999);
\draw(2, -21.6) node[anchor=north west,align=left] {\large General topics in partial differential equations};
\draw (2, -21.6) rectangle (20.35,-36.1);
\draw(3, -22.6) node[anchor=north west,align=left] {Inequalities \\ applied to PDEs \\ involving derivatives,\\ differential\\ and integral \\ operators, or integrals};
\draw (3, -22.6) rectangle (9.35,-25.700000000000003);
\draw(9.45, -22.6) node[anchor=north west,align=left] {Cauchy-Kovalevskaya\\ theorems};
\draw (9.45, -22.6) rectangle (14.799999999999999,-24.200000000000003);
\draw(14.899999999999999, -22.6) node[anchor=north west,align=left] {Microlocal \\ methods and methods\\ of sheaf \\ theory and \\ homological algebra\\ applied to PDEs};
\draw (14.899999999999999, -22.6) rectangle (20.25,-25.700000000000003);
\draw(3, -25.8) node[anchor=north west,align=left] {Existence problems\\ for PDEs: \\ global existence,\\ local existence,\\ non-existence};
\draw (3, -25.8) rectangle (8.1,-28.400000000000002);
\draw(8.2, -25.8) node[anchor=north west,align=left] {Geometric \\ theory, \\ characteristics, \\ transformations \\ in context of PDEs};
\draw (8.2, -25.8) rectangle (13.299999999999999,-28.400000000000002);
\draw(13.399999999999999, -25.8) node[anchor=north west,align=left] {Uniqueness \\ problems for \\ PDEs: global \\ uniqueness, local\\ uniqueness,\\ non-uniqueness};
\draw (13.399999999999999, -25.8) rectangle (18.25,-28.900000000000002);
\draw(3, -29.0) node[anchor=north west,align=left] {Topological \\ and monotonicity\\ methods \\ applied to PDEs};
\draw (3, -29.0) rectangle (7.6,-31.1);
\draw(7.699999999999999, -29.0) node[anchor=north west,align=left] {Transform \\ methods (e.g.,\\ integral \\ transforms) \\ applied to PDEs};
\draw (7.699999999999999, -29.0) rectangle (12.049999999999999,-31.6);
\draw(12.149999999999999, -29.0) node[anchor=north west,align=left] {Methods of\\ ordinary \\ differential\\ equations \\ applied to PDEs};
\draw (12.149999999999999, -29.0) rectangle (16.5,-31.6);
\draw(16.599999999999998, -29.0) node[anchor=north west,align=left] {Parametrices\\ in context\\ of PDEs};
\draw (16.599999999999998, -29.0) rectangle (20.2,-30.6);
\draw(3, -31.700000000000003) node[anchor=north west,align=left] {Other \\ special methods\\ applied\\ to PDEs};
\draw (3, -31.700000000000003) rectangle (7.35,-33.800000000000004);
\draw(7.449999999999999, -31.700000000000003) node[anchor=north west,align=left] {Theoretical\\ approximation\\ in context\\ of PDEs};
\draw (7.449999999999999, -31.700000000000003) rectangle (11.299999999999999,-33.800000000000004);
\draw(11.4, -31.700000000000003) node[anchor=north west,align=left] {Fundamental\\ solutions\\ to PDEs};
\draw (11.4, -31.700000000000003) rectangle (14.75,-33.300000000000004);
\draw(14.85, -31.700000000000003) node[anchor=north west,align=left] {Variational\\ methods\\ applied\\ to PDEs};
\draw (14.85, -31.700000000000003) rectangle (18.2,-33.800000000000004);
\draw(3, -33.900000000000006) node[anchor=north west,align=left] {Analyticity\\ in context\\ of PDEs};
\draw (3, -33.900000000000006) rectangle (6.35,-35.50000000000001);
\draw(6.449999999999999, -33.900000000000006) node[anchor=north west,align=left] {Singularity\\ in context\\ of PDEs};
\draw (6.449999999999999, -33.900000000000006) rectangle (9.799999999999999,-35.50000000000001);
\draw(9.899999999999999, -33.900000000000006) node[anchor=north west,align=left] {Wave front\\ sets\\ in context\\ of PDEs};
\draw (9.899999999999999, -33.900000000000006) rectangle (12.999999999999998,-36.00000000000001);
\draw(13.1, -33.900000000000006) node[anchor=north west,align=left] {Classical\\ solutions\\ to PDEs};
\draw (13.1, -33.900000000000006) rectangle (15.95,-35.50000000000001);
\draw(20.450000000000003, -21.6) node[anchor=north west,align=left] {\large Generalized solutions to partial differential equations};
\draw (20.450000000000003, -21.6) rectangle (38.10000000000001,-24.3);
\draw(21.450000000000003, -22.6) node[anchor=north west,align=left] {Weak \\ solutions\\ to PDEs};
\draw (21.450000000000003, -22.6) rectangle (24.300000000000004,-24.200000000000003);
\draw(24.400000000000002, -22.6) node[anchor=north west,align=left] {Strong \\ solutions\\ to PDEs};
\draw (24.400000000000002, -22.6) rectangle (27.250000000000004,-24.200000000000003);
\draw(27.35, -22.6) node[anchor=north west,align=left] {Viscosity\\ solutions\\ to PDEs};
\draw (27.35, -22.6) rectangle (30.200000000000003,-24.200000000000003);
\draw(20.450000000000003, -24.400000000000002) node[anchor=north west,align=left] {\large Hyperbolic equations and hyperbolic systems};
\draw (20.450000000000003, -24.400000000000002) rectangle (36.55,-32.0);
\draw(21.450000000000003, -25.400000000000002) node[anchor=north west,align=left] {Initial value\\ problems\\ for first-order\\ hyperbolic equations};
\draw (21.450000000000003, -25.400000000000002) rectangle (27.050000000000004,-28.000000000000004);
\draw(27.150000000000002, -25.400000000000002) node[anchor=north west,align=left] {Initial-boundary\\ value \\ problems for \\ first-order \\ hyperbolic equations};
\draw (27.150000000000002, -25.400000000000002) rectangle (32.75,-28.000000000000004);
\draw(32.85, -25.400000000000002) node[anchor=north west,align=left] {Second-order\\ hyperbolic\\ equations};
\draw (32.85, -25.400000000000002) rectangle (36.45,-27.500000000000004);
\draw(21.450000000000003, -28.1) node[anchor=north west,align=left] {Initial-boundary\\ value \\ problems for \\ second-order \\ hyperbolic equations};
\draw (21.450000000000003, -28.1) rectangle (27.050000000000004,-30.700000000000003);
\draw(27.150000000000002, -28.1) node[anchor=north west,align=left] {Initial value\\ problems \\ for second-order\\ hyperbolic\\ equations};
\draw (27.150000000000002, -28.1) rectangle (31.75,-30.700000000000003);
\draw(31.85, -28.1) node[anchor=north west,align=left] {First-order\\ hyperbolic\\ equations};
\draw (31.85, -28.1) rectangle (35.2,-29.700000000000003);
\draw(21.450000000000003, -30.800000000000004) node[anchor=north west,align=left] {Wave \\ equation};
\draw (21.450000000000003, -30.800000000000004) rectangle (24.050000000000004,-31.900000000000006);
\draw(38.20000000000001, -21.6) node[anchor=north west,align=left] {\large Parabolic equations and parabolic systems};
\draw (38.20000000000001, -21.6) rectangle (53.80000000000001,-49.6);
\draw(39.20000000000001, -22.6) node[anchor=north west,align=left] {Unilateral problems\\ for nonlinear\\ parabolic \\ equations and variational\\ inequalities\\ with nonlinear\\ parabolic operators};
\draw (39.20000000000001, -22.6) rectangle (46.05000000000001,-26.200000000000003);
\draw(46.150000000000006, -22.6) node[anchor=north west,align=left] {Unilateral problems\\ for linear parabolic\\ equations and\\ variational \\ inequalities with linear\\ parabolic operators};
\draw (46.150000000000006, -22.6) rectangle (52.75000000000001,-25.700000000000003);
\draw(39.20000000000001, -26.3) node[anchor=north west,align=left] {Initial value\\ problems\\ for \\ second-order \\ parabolic equations};
\draw (39.20000000000001, -26.3) rectangle (44.55000000000001,-28.900000000000002);
\draw(44.650000000000006, -26.3) node[anchor=north west,align=left] {Initial-boundary\\ value \\ problems for \\ second-order \\ parabolic equations};
\draw (44.650000000000006, -26.3) rectangle (50.00000000000001,-28.900000000000002);
\draw(50.10000000000001, -26.3) node[anchor=north west,align=left] {Second-order\\ parabolic\\ equations};
\draw (50.10000000000001, -26.3) rectangle (53.70000000000001,-27.900000000000002);
\draw(39.20000000000001, -29.0) node[anchor=north west,align=left] {Initial value\\ problems\\ for \\ higher-order \\ parabolic equations};
\draw (39.20000000000001, -29.0) rectangle (44.55000000000001,-31.6);
\draw(44.650000000000006, -29.0) node[anchor=north west,align=left] {Initial-boundary\\ value \\ problems for \\ higher-order \\ parabolic equations};
\draw (44.650000000000006, -29.0) rectangle (50.00000000000001,-31.6);
\draw(50.10000000000001, -29.0) node[anchor=north west,align=left] {Higher-order\\ parabolic\\ equations};
\draw (50.10000000000001, -29.0) rectangle (53.70000000000001,-30.6);
\draw(39.20000000000001, -31.700000000000003) node[anchor=north west,align=left] {Nonlinear initial,\\ boundary and\\ initial-boundary\\ value problems\\ for linear\\ parabolic equations};
\draw (39.20000000000001, -31.700000000000003) rectangle (44.55000000000001,-34.800000000000004);
\draw(44.650000000000006, -31.700000000000003) node[anchor=north west,align=left] {Nonlinear initial,\\ boundary and\\ initial-boundary\\ value problems\\ for nonlinear \\ parabolic equations};
\draw (44.650000000000006, -31.700000000000003) rectangle (50.00000000000001,-34.800000000000004);
\draw(50.10000000000001, -31.700000000000003) node[anchor=north west,align=left] {Second-order\\ parabolic\\ systems};
\draw (50.10000000000001, -31.700000000000003) rectangle (53.70000000000001,-33.300000000000004);
\draw(50.10000000000001, -33.400000000000006) node[anchor=north west,align=left] {Heat \\ equation};
\draw (50.10000000000001, -33.400000000000006) rectangle (52.70000000000001,-34.50000000000001);
\draw(39.20000000000001, -34.900000000000006) node[anchor=north west,align=left] {Unilateral problems\\ for parabolic\\ systems and systems\\ of variational\\ inequalities with\\ parabolic operators};
\draw (39.20000000000001, -34.900000000000006) rectangle (44.55000000000001,-38.00000000000001);
\draw(44.650000000000006, -34.900000000000006) node[anchor=north west,align=left] {Semilinear \\ parabolic equations\\ with Laplacian,\\ bi-Laplacian\\ or poly-Laplacian};
\draw (44.650000000000006, -34.900000000000006) rectangle (50.00000000000001,-37.50000000000001);
\draw(50.10000000000001, -34.900000000000006) node[anchor=north west,align=left] {Higher-order\\ parabolic\\ systems};
\draw (50.10000000000001, -34.900000000000006) rectangle (53.70000000000001,-36.50000000000001);
\draw(39.20000000000001, -38.1) node[anchor=north west,align=left] {Reaction-diffusion\\ equations};
\draw (39.20000000000001, -38.1) rectangle (44.30000000000001,-39.7);
\draw(44.400000000000006, -38.1) node[anchor=north west,align=left] {Initial value\\ problems\\ for \\ second-order \\ parabolic systems};
\draw (44.400000000000006, -38.1) rectangle (49.25000000000001,-40.7);
\draw(49.35000000000001, -38.1) node[anchor=north west,align=left] {Quasilinear \\ parabolic \\ equations with \\ \(p\)-Laplacian};
\draw (49.35000000000001, -38.1) rectangle (53.70000000000001,-40.2);
\draw(39.20000000000001, -40.8) node[anchor=north west,align=left] {Initial value\\ problems\\ for \\ higher-order \\ parabolic systems};
\draw (39.20000000000001, -40.8) rectangle (44.05000000000001,-43.4);
\draw(44.150000000000006, -40.8) node[anchor=north west,align=left] {Initial-boundary\\ value \\ problems for \\ second-order \\ parabolic systems};
\draw (44.150000000000006, -40.8) rectangle (49.00000000000001,-43.4);
\draw(49.10000000000001, -40.8) node[anchor=north west,align=left] {Ultraparabolic\\ equations,\\ pseudoparabolic \\ equations, etc.};
\draw (49.10000000000001, -40.8) rectangle (53.70000000000001,-43.4);
\draw(39.20000000000001, -43.5) node[anchor=north west,align=left] {Initial-boundary\\ value \\ problems for \\ higher-order \\ parabolic systems};
\draw (39.20000000000001, -43.5) rectangle (44.05000000000001,-46.1);
\draw(44.150000000000006, -43.5) node[anchor=north west,align=left] {Quasilinear\\ parabolic \\ equations with\\ mean curvature\\ operator};
\draw (44.150000000000006, -43.5) rectangle (48.25000000000001,-46.1);
\draw(48.35000000000001, -43.5) node[anchor=north west,align=left] {Parabolic\\ Monge-Ampère\\ equations};
\draw (48.35000000000001, -43.5) rectangle (51.95000000000001,-45.1);
\draw(39.20000000000001, -46.2) node[anchor=north west,align=left] {Quasilinear\\ parabolic\\ equations};
\draw (39.20000000000001, -46.2) rectangle (42.55000000000001,-47.800000000000004);
\draw(42.650000000000006, -46.2) node[anchor=north west,align=left] {Semilinear\\ parabolic\\ equations};
\draw (42.650000000000006, -46.2) rectangle (45.75000000000001,-47.800000000000004);
\draw(45.85000000000001, -46.2) node[anchor=north west,align=left] {Degenerate\\ parabolic\\ equations};
\draw (45.85000000000001, -46.2) rectangle (48.95000000000001,-47.800000000000004);
\draw(49.05000000000001, -46.2) node[anchor=north west,align=left] {Nonlinear\\ parabolic\\ equations};
\draw (49.05000000000001, -46.2) rectangle (51.90000000000001,-47.800000000000004);
\draw(39.20000000000001, -47.9) node[anchor=north west,align=left] {Singular\\ parabolic\\ equations};
\draw (39.20000000000001, -47.9) rectangle (42.05000000000001,-49.5);
\draw(42.15000000000001, -47.9) node[anchor=north west,align=left] {Abstract\\ parabolic\\ equations};
\draw (42.15000000000001, -47.9) rectangle (45.000000000000014,-49.5);
\draw(45.10000000000001, -47.9) node[anchor=north west,align=left] {Heat\\ kernel};
\draw (45.10000000000001, -47.9) rectangle (47.20000000000001,-49.0);
\draw(53.900000000000006, -21.6) node[anchor=north west,align=left] {\large Elliptic equations and elliptic systems};
\draw (53.900000000000006, -21.6) rectangle (68.75,-52.3);
\draw(54.900000000000006, -22.6) node[anchor=north west,align=left] {Unilateral problems\\ for linear elliptic\\ equations and\\ variational \\ inequalities with linear\\ elliptic operators};
\draw (54.900000000000006, -22.6) rectangle (61.50000000000001,-25.700000000000003);
\draw(61.60000000000001, -22.6) node[anchor=north west,align=left] {Unilateral problems\\ for nonlinear\\ elliptic equations\\ and variational\\ inequalities\\ with nonlinear\\ elliptic operators};
\draw (61.60000000000001, -22.6) rectangle (66.95,-26.200000000000003);
\draw(54.900000000000006, -26.3) node[anchor=north west,align=left] {Unilateral problems\\ for elliptic \\ systems and systems\\ of variational\\ inequalities with\\ elliptic operators};
\draw (54.900000000000006, -26.3) rectangle (60.25000000000001,-29.400000000000002);
\draw(60.35000000000001, -26.3) node[anchor=north west,align=left] {Boundary \\ value problems\\ for \\ second-order \\ elliptic equations};
\draw (60.35000000000001, -26.3) rectangle (65.45,-28.900000000000002);
\draw(65.55000000000001, -26.3) node[anchor=north west,align=left] {Semilinear\\ elliptic\\ equations};
\draw (65.55000000000001, -26.3) rectangle (68.65,-27.900000000000002);
\draw(54.900000000000006, -29.5) node[anchor=north west,align=left] {Boundary \\ value problems\\ for \\ higher-order \\ elliptic equations};
\draw (54.900000000000006, -29.5) rectangle (60.00000000000001,-32.1);
\draw(60.10000000000001, -29.5) node[anchor=north west,align=left] {Nonlinear \\ boundary value\\ problems for\\ linear \\ elliptic equations};
\draw (60.10000000000001, -29.5) rectangle (65.2,-32.1);
\draw(65.30000000000001, -29.5) node[anchor=north west,align=left] {Schrödinger\\ operator,\\ Schrödinger\\ equation};
\draw (65.30000000000001, -29.5) rectangle (68.65,-31.6);
\draw(54.900000000000006, -32.2) node[anchor=north west,align=left] {Nonlinear \\ boundary value \\ problems for \\ nonlinear \\ elliptic equations};
\draw (54.900000000000006, -32.2) rectangle (60.00000000000001,-34.800000000000004);
\draw(60.10000000000001, -32.2) node[anchor=north west,align=left] {Semilinear \\ elliptic equations\\ with Laplacian,\\ bi-Laplacian \\ or poly-Laplacian};
\draw (60.10000000000001, -32.2) rectangle (65.2,-34.800000000000004);
\draw(65.30000000000001, -32.2) node[anchor=north west,align=left] {First-order\\ elliptic\\ systems};
\draw (65.30000000000001, -32.2) rectangle (68.65,-33.800000000000004);
\draw(54.900000000000006, -34.900000000000006) node[anchor=north west,align=left] {Quasilinear\\ elliptic \\ equations \\ with mean \\ curvature operator};
\draw (54.900000000000006, -34.900000000000006) rectangle (60.00000000000001,-37.50000000000001);
\draw(60.10000000000001, -34.900000000000006) node[anchor=north west,align=left] {Elliptic \\ equations \\ with \\ infinity-Laplacian};
\draw (60.10000000000001, -34.900000000000006) rectangle (65.2,-37.00000000000001);
\draw(65.30000000000001, -34.900000000000006) node[anchor=north west,align=left] {Quasilinear\\ elliptic\\ equations};
\draw (65.30000000000001, -34.900000000000006) rectangle (68.65,-36.50000000000001);
\draw(54.900000000000006, -37.6) node[anchor=north west,align=left] {Laplace operator,\\ Helmholtz \\ equation (reduced\\ wave equation),\\ Poisson equation};
\draw (54.900000000000006, -37.6) rectangle (59.75000000000001,-40.2);
\draw(59.85000000000001, -37.6) node[anchor=north west,align=left] {Boundary \\ value problems\\ for \\ first-order \\ elliptic systems};
\draw (59.85000000000001, -37.6) rectangle (64.45,-40.2);
\draw(64.55000000000001, -37.6) node[anchor=north west,align=left] {Green’s \\ functions for\\ elliptic\\ equations};
\draw (64.55000000000001, -37.6) rectangle (68.4,-39.7);
\draw(54.900000000000006, -40.3) node[anchor=north west,align=left] {Boundary \\ value problems\\ for \\ second-order \\ elliptic systems};
\draw (54.900000000000006, -40.3) rectangle (59.50000000000001,-42.9);
\draw(59.60000000000001, -40.3) node[anchor=north west,align=left] {Boundary \\ value problems\\ for \\ higher-order \\ elliptic systems};
\draw (59.60000000000001, -40.3) rectangle (64.2,-42.9);
\draw(64.30000000000001, -40.3) node[anchor=north west,align=left] {Quasilinear\\ elliptic \\ equations with\\ \(p\)-Laplacian};
\draw (64.30000000000001, -40.3) rectangle (68.65,-42.4);
\draw(54.900000000000006, -43.0) node[anchor=north west,align=left] {Boundary values\\ of solutions\\ to elliptic \\ equations and \\ elliptic systems};
\draw (54.900000000000006, -43.0) rectangle (59.50000000000001,-45.6);
\draw(59.60000000000001, -43.0) node[anchor=north west,align=left] {Second-order\\ elliptic\\ equations};
\draw (59.60000000000001, -43.0) rectangle (63.20000000000001,-44.6);
\draw(63.300000000000004, -43.0) node[anchor=north west,align=left] {Variational\\ methods for\\ second-order\\ elliptic\\ equations};
\draw (63.300000000000004, -43.0) rectangle (66.9,-45.6);
\draw(54.900000000000006, -45.7) node[anchor=north west,align=left] {Higher-order\\ elliptic\\ equations};
\draw (54.900000000000006, -45.7) rectangle (58.50000000000001,-47.300000000000004);
\draw(58.60000000000001, -45.7) node[anchor=north west,align=left] {Variational\\ methods for\\ higher-order\\ elliptic\\ equations};
\draw (58.60000000000001, -45.7) rectangle (62.20000000000001,-48.300000000000004);
\draw(62.300000000000004, -45.7) node[anchor=north west,align=left] {Second-order\\ elliptic\\ systems};
\draw (62.300000000000004, -45.7) rectangle (65.9,-47.300000000000004);
\draw(54.900000000000006, -48.4) node[anchor=north west,align=left] {Higher-order\\ elliptic\\ systems};
\draw (54.900000000000006, -48.4) rectangle (58.50000000000001,-50.0);
\draw(58.60000000000001, -48.4) node[anchor=north west,align=left] {Variational\\ methods\\ for elliptic\\ systems};
\draw (58.60000000000001, -48.4) rectangle (62.20000000000001,-50.5);
\draw(62.300000000000004, -48.4) node[anchor=north west,align=left] {Monge-Ampère\\ equations};
\draw (62.300000000000004, -48.4) rectangle (65.9,-50.0);
\draw(54.900000000000006, -50.599999999999994) node[anchor=north west,align=left] {Degenerate\\ elliptic\\ equations};
\draw (54.900000000000006, -50.599999999999994) rectangle (58.00000000000001,-52.199999999999996);
\draw(58.10000000000001, -50.599999999999994) node[anchor=north west,align=left] {Nonlinear\\ elliptic\\ equations};
\draw (58.10000000000001, -50.599999999999994) rectangle (60.95000000000001,-52.199999999999996);
\draw(61.050000000000004, -50.599999999999994) node[anchor=north west,align=left] {Singular\\ elliptic\\ equations};
\draw (61.050000000000004, -50.599999999999994) rectangle (63.900000000000006,-52.199999999999996);
\draw(76.96999999999998, -1) node[anchor=north west,align=left] {\LARGE Ordinary differential equations};
\draw (76.96999999999998, -1) rectangle (137.07,-81.7);
\draw(77.96999999999998, -2) node[anchor=north west,align=left] {\large Functional-differential equations (including equations with delayed, advanced or state-dependent argument)};
\draw (77.96999999999998, -2) rectangle (117.66999999999999,-23.400000000000002);
\draw(78.96999999999998, -3) node[anchor=north west,align=left] {Functional-differentialinclusions};
\draw (78.96999999999998, -3) rectangle (87.81999999999998,-5.1);
\draw(87.91999999999999, -3) node[anchor=north west,align=left] {Functional-differentialequations\\ with fractional\\ derivatives};
\draw (87.91999999999999, -3) rectangle (96.51999999999998,-5.6);
\draw(96.61999999999998, -3) node[anchor=north west,align=left] {Functional-differentialequations\\ with state-dependent\\ arguments};
\draw (96.61999999999998, -3) rectangle (105.21999999999997,-5.6);
\draw(105.31999999999998, -3) node[anchor=north west,align=left] {Transformation\\ and reduction \\ of functional-differential\\ equations and systems,\\ normal forms};
\draw (105.31999999999998, -3) rectangle (112.41999999999997,-6.1);
\draw(78.96999999999998, -6.2) node[anchor=north west,align=left] {Symmetries,\\ invariants\\ of \\ functional-differential\\ equations};
\draw (78.96999999999998, -6.2) rectangle (85.31999999999998,-8.8);
\draw(85.41999999999999, -6.2) node[anchor=north west,align=left] {General theory\\ of \\ functional-differential\\ equations};
\draw (85.41999999999999, -6.2) rectangle (91.76999999999998,-8.3);
\draw(91.86999999999998, -6.2) node[anchor=north west,align=left] {Linear \\ functional-differential\\ equations};
\draw (91.86999999999998, -6.2) rectangle (98.21999999999997,-8.3);
\draw(98.32, -6.2) node[anchor=north west,align=left] {Theoretical \\ approximation of \\ solutions to \\ functional-differential\\ equations};
\draw (98.32, -6.2) rectangle (104.66999999999999,-8.8);
\draw(104.76999999999998, -6.2) node[anchor=north west,align=left] {Spectral theory\\ of \\ functional-differential\\ operators};
\draw (104.76999999999998, -6.2) rectangle (111.11999999999998,-8.3);
\draw(111.21999999999998, -6.2) node[anchor=north west,align=left] {Boundary value\\ problems\\ for \\ functional-differential\\ equations};
\draw (111.21999999999998, -6.2) rectangle (117.56999999999998,-8.8);
\draw(78.96999999999998, -8.9) node[anchor=north west,align=left] {Oscillation\\ theory of\\ functional-differential\\ equations};
\draw (78.96999999999998, -8.9) rectangle (85.31999999999998,-11.5);
\draw(85.41999999999999, -8.9) node[anchor=north west,align=left] {Growth, \\ boundedness, \\ comparison of \\ solutions to \\ functional-differential\\ equations};
\draw (85.41999999999999, -8.9) rectangle (91.76999999999998,-12.0);
\draw(91.86999999999998, -8.9) node[anchor=north west,align=left] {Periodic \\ solutions to\\ functional-differential\\ equations};
\draw (91.86999999999998, -8.9) rectangle (98.21999999999997,-11.5);
\draw(98.32, -8.9) node[anchor=north west,align=left] {Almost and \\ pseudo-almost periodic\\ solutions to \\ functional-differential\\ equations};
\draw (98.32, -8.9) rectangle (104.66999999999999,-11.5);
\draw(104.76999999999998, -8.9) node[anchor=north west,align=left] {Heteroclinic \\ and homoclinic\\ orbits of \\ functional-differential\\ equations};
\draw (104.76999999999998, -8.9) rectangle (111.11999999999998,-11.5);
\draw(111.21999999999998, -8.9) node[anchor=north west,align=left] {Bifurcation\\ theory of\\ functional-differential\\ equations};
\draw (111.21999999999998, -8.9) rectangle (117.56999999999998,-11.5);
\draw(78.96999999999998, -12.100000000000001) node[anchor=north west,align=left] {Invariant \\ manifolds of\\ functional-differential\\ equations};
\draw (78.96999999999998, -12.100000000000001) rectangle (85.31999999999998,-14.700000000000001);
\draw(85.41999999999999, -12.100000000000001) node[anchor=north west,align=left] {Stability \\ theory of \\ functional-differential\\ equations};
\draw (85.41999999999999, -12.100000000000001) rectangle (91.76999999999998,-14.200000000000001);
\draw(91.86999999999998, -12.100000000000001) node[anchor=north west,align=left] {Stationary\\ solutions \\ of \\ functional-differential\\ equations};
\draw (91.86999999999998, -12.100000000000001) rectangle (98.21999999999997,-14.700000000000001);
\draw(98.32, -12.100000000000001) node[anchor=north west,align=left] {Complex (chaotic)\\ behavior of\\ solutions to \\ functional-differential\\ equations};
\draw (98.32, -12.100000000000001) rectangle (104.66999999999999,-14.700000000000001);
\draw(104.76999999999998, -12.100000000000001) node[anchor=north west,align=left] {Synchronization\\ of \\ functional-differential\\ equations};
\draw (104.76999999999998, -12.100000000000001) rectangle (111.11999999999998,-14.200000000000001);
\draw(111.21999999999998, -12.100000000000001) node[anchor=north west,align=left] {Asymptotic\\ theory of\\ functional-differential\\ equations};
\draw (111.21999999999998, -12.100000000000001) rectangle (117.56999999999998,-14.700000000000001);
\draw(78.96999999999998, -14.8) node[anchor=north west,align=left] {Singular \\ perturbations\\ of \\ functional-differential\\ equations};
\draw (78.96999999999998, -14.8) rectangle (85.31999999999998,-17.400000000000002);
\draw(85.41999999999999, -14.8) node[anchor=north west,align=left] {Perturbations\\ of \\ functional-differential\\ equations};
\draw (85.41999999999999, -14.8) rectangle (91.76999999999998,-16.900000000000002);
\draw(91.86999999999998, -14.8) node[anchor=north west,align=left] {Inverse \\ problems for\\ functional-differential\\ equations};
\draw (91.86999999999998, -14.8) rectangle (98.21999999999997,-17.400000000000002);
\draw(98.32, -14.8) node[anchor=north west,align=left] {Functional-differential\\ equations in \\ abstract spaces};
\draw (98.32, -14.8) rectangle (104.66999999999999,-16.900000000000002);
\draw(104.76999999999998, -14.8) node[anchor=north west,align=left] {Lattice \\ functional-differential\\ equations};
\draw (104.76999999999998, -14.8) rectangle (111.11999999999998,-16.900000000000002);
\draw(111.21999999999998, -14.8) node[anchor=north west,align=left] {Implicit \\ functional-differential\\ equations};
\draw (111.21999999999998, -14.8) rectangle (117.56999999999998,-16.900000000000002);
\draw(78.96999999999998, -17.5) node[anchor=north west,align=left] {Averaging \\ for \\ functional-differential\\ equations};
\draw (78.96999999999998, -17.5) rectangle (85.31999999999998,-19.6);
\draw(85.41999999999999, -17.5) node[anchor=north west,align=left] {Hybrid systems\\ of \\ functional-differential\\ equations};
\draw (85.41999999999999, -17.5) rectangle (91.76999999999998,-19.6);
\draw(91.86999999999998, -17.5) node[anchor=north west,align=left] {Control \\ problems for\\ functional-differential\\ equations};
\draw (91.86999999999998, -17.5) rectangle (98.21999999999997,-20.1);
\draw(98.32, -17.5) node[anchor=north west,align=left] {Fuzzy \\ functional-differential\\ equations};
\draw (98.32, -17.5) rectangle (104.66999999999999,-19.6);
\draw(104.76999999999998, -17.5) node[anchor=north west,align=left] {Functional-differential\\ inequalities};
\draw (104.76999999999998, -17.5) rectangle (111.11999999999998,-19.6);
\draw(111.21999999999998, -17.5) node[anchor=north west,align=left] {Discontinuous\\ functional-differential\\ equations};
\draw (111.21999999999998, -17.5) rectangle (117.56999999999998,-19.6);
\draw(78.96999999999998, -20.2) node[anchor=north west,align=left] {Neutral \\ functional-differential\\ equations};
\draw (78.96999999999998, -20.2) rectangle (85.31999999999998,-22.3);
\draw(85.41999999999999, -20.2) node[anchor=north west,align=left] {Functional-differential\\ equations\\ in the \\ complex domain};
\draw (85.41999999999999, -20.2) rectangle (91.76999999999998,-22.8);
\draw(91.86999999999998, -20.2) node[anchor=north west,align=left] {Functional-differential\\ equations on \\ time scales or\\ measure chains};
\draw (91.86999999999998, -20.2) rectangle (98.21999999999997,-22.8);
\draw(98.32, -20.2) node[anchor=north west,align=left] {Functional-differential\\ equations\\ with impulses};
\draw (98.32, -20.2) rectangle (104.66999999999999,-22.3);
\draw(104.76999999999998, -20.2) node[anchor=north west,align=left] {Stochastic\\ functional-differential\\ equations};
\draw (104.76999999999998, -20.2) rectangle (111.11999999999998,-22.3);
\draw(111.21999999999998, -20.2) node[anchor=north west,align=left] {Qualitative \\ investigation and\\ simulation of \\ models involving\\ functional-differential\\ equations};
\draw (111.21999999999998, -20.2) rectangle (117.56999999999998,-23.3);
\draw(117.76999999999998, -2) node[anchor=north west,align=left] {\large General theory for ordinary differential equations};
\draw (117.76999999999998, -2) rectangle (136.86999999999998,-20.700000000000003);
\draw(118.76999999999998, -3) node[anchor=north west,align=left] {Analytical theory\\ of ordinary \\ differential \\ equations: series, \\ transformations, \\ transforms, \\ operational calculus, etc.};
\draw (118.76999999999998, -3) rectangle (125.86999999999998,-6.6);
\draw(125.96999999999998, -3) node[anchor=north west,align=left] {Initial value problems,\\ existence, \\ uniqueness, continuous\\ dependence and \\ continuation of solutions\\ to ordinary \\ differential equations};
\draw (125.96999999999998, -3) rectangle (132.82,-6.6);
\draw(132.92, -3) node[anchor=north west,align=left] {Discontinuous\\ ordinary\\ differential\\ equations};
\draw (132.92, -3) rectangle (136.76999999999998,-5.1);
\draw(118.76999999999998, -6.7) node[anchor=north west,align=left] {Generalized \\ ordinary differential\\ equations \\ (measure-differential\\ equations, \\ set-valued differential\\ equations, etc.)};
\draw (118.76999999999998, -6.7) rectangle (125.11999999999998,-10.3);
\draw(125.21999999999998, -6.7) node[anchor=north west,align=left] {Fractional \\ ordinary \\ differential equations\\ and \\ fractional differential\\ inclusions};
\draw (125.21999999999998, -6.7) rectangle (131.57,-9.8);
\draw(131.67, -6.7) node[anchor=north west,align=left] {Inverse \\ problems involving\\ ordinary\\ differential\\ equations};
\draw (131.67, -6.7) rectangle (136.76999999999998,-9.3);
\draw(118.76999999999998, -10.4) node[anchor=north west,align=left] {Implicit ordinary\\ differential\\ equations,\\ differential-algebraic\\ equations};
\draw (118.76999999999998, -10.4) rectangle (124.86999999999998,-13.0);
\draw(124.96999999999998, -10.4) node[anchor=north west,align=left] {Theoretical \\ approximation of\\ solutions to\\ ordinary \\ differential equations};
\draw (124.96999999999998, -10.4) rectangle (131.07,-13.0);
\draw(131.17, -10.4) node[anchor=north west,align=left] {Geometric \\ methods in ordinary\\ differential\\ equations};
\draw (131.17, -10.4) rectangle (136.51999999999998,-12.5);
\draw(118.76999999999998, -13.100000000000001) node[anchor=north west,align=left] {Explicit \\ solutions, first \\ integrals of \\ ordinary differential\\ equations};
\draw (118.76999999999998, -13.100000000000001) rectangle (124.61999999999998,-15.700000000000001);
\draw(124.71999999999998, -13.100000000000001) node[anchor=north west,align=left] {Nonlinear \\ ordinary differential\\ equations\\ and systems,\\ general theory};
\draw (124.71999999999998, -13.100000000000001) rectangle (130.57,-15.700000000000001);
\draw(130.67, -13.100000000000001) node[anchor=north west,align=left] {Differential \\ inequalities \\ involving functions\\ of a single\\ real variable};
\draw (130.67, -13.100000000000001) rectangle (136.01999999999998,-15.700000000000001);
\draw(118.76999999999998, -15.8) node[anchor=north west,align=left] {Linear \\ ordinary \\ differential \\ equations and \\ systems, general};
\draw (118.76999999999998, -15.8) rectangle (123.36999999999998,-18.400000000000002);
\draw(123.46999999999998, -15.8) node[anchor=north west,align=left] {Ordinary \\ differential \\ equations of \\ infinite order};
\draw (123.46999999999998, -15.8) rectangle (127.56999999999998,-17.900000000000002);
\draw(127.66999999999999, -15.8) node[anchor=north west,align=left] {Hybrid systems\\ of ordinary\\ differential\\ equations};
\draw (127.66999999999999, -15.8) rectangle (131.76999999999998,-17.900000000000002);
\draw(131.86999999999998, -15.8) node[anchor=north west,align=left] {Ordinary \\ differential\\ equations \\ with impulses};
\draw (131.86999999999998, -15.8) rectangle (135.71999999999997,-17.900000000000002);
\draw(118.76999999999998, -18.5) node[anchor=north west,align=left] {Fuzzy \\ ordinary \\ differential\\ equations};
\draw (118.76999999999998, -18.5) rectangle (122.36999999999998,-20.6);
\draw(122.46999999999998, -18.5) node[anchor=north west,align=left] {Ordinary\\ lattice \\ differential\\ equations};
\draw (122.46999999999998, -18.5) rectangle (126.06999999999998,-20.6);
\draw(126.16999999999999, -18.5) node[anchor=north west,align=left] {Ordinary \\ differential\\ inclusions};
\draw (126.16999999999999, -18.5) rectangle (129.76999999999998,-20.1);
\draw(117.76999999999998, -20.800000000000004) node[anchor=north west,align=left] {\large History of \\ ordinary differential\\ equations};
\draw (117.76999999999998, -20.800000000000004) rectangle (124.87999999999998,-22.400000000000006);
\draw(77.96999999999998, -23.500000000000004) node[anchor=north west,align=left] {\large Ordinary differential equations in the complex domain};
\draw (77.96999999999998, -23.500000000000004) rectangle (98.07,-44.7);
\draw(78.96999999999998, -24.500000000000004) node[anchor=north west,align=left] {Singular perturbation\\ problems for \\ ordinary differential\\ equations in the \\ complex domain (complex\\ WKB, turning \\ points, steepest descent)};
\draw (78.96999999999998, -24.500000000000004) rectangle (85.81999999999998,-28.100000000000005);
\draw(85.91999999999999, -24.500000000000004) node[anchor=north west,align=left] {Algebraic aspects \\ (differential-algebraic,\\ hypertranscendence,\\ group-theoretical)\\ of ordinary \\ differential equations\\ in the complex domain};
\draw (85.91999999999999, -24.500000000000004) rectangle (92.51999999999998,-28.100000000000005);
\draw(92.61999999999998, -24.500000000000004) node[anchor=north west,align=left] {Oscillation, \\ growth of solutions\\ to ordinary\\ differential\\ equations in\\ the complex domain};
\draw (92.61999999999998, -24.500000000000004) rectangle (97.96999999999997,-27.600000000000005);
\draw(78.96999999999998, -28.200000000000003) node[anchor=north west,align=left] {Painlevé and other\\ special ordinary\\ differential equations\\ in the complex\\ domain; classification,\\ hierarchies};
\draw (78.96999999999998, -28.200000000000003) rectangle (85.31999999999998,-31.300000000000004);
\draw(85.41999999999999, -28.200000000000003) node[anchor=north west,align=left] {Singularities, \\ monodromy and local \\ behavior of solutions\\ to ordinary \\ differential equations\\ in the complex \\ domain, normal forms};
\draw (85.41999999999999, -28.200000000000003) rectangle (91.51999999999998,-31.800000000000004);
\draw(91.61999999999998, -28.200000000000003) node[anchor=north west,align=left] {Entire and \\ meromorphic solutions\\ to ordinary\\ differential\\ equations in\\ the complex domain};
\draw (91.61999999999998, -28.200000000000003) rectangle (97.46999999999997,-31.300000000000004);
\draw(78.96999999999998, -31.900000000000006) node[anchor=north west,align=left] {Formal solutions\\ and transform \\ techniques for \\ ordinary differential\\ equations in\\ the complex domain};
\draw (78.96999999999998, -31.900000000000006) rectangle (84.81999999999998,-35.00000000000001);
\draw(84.91999999999999, -31.900000000000006) node[anchor=north west,align=left] {Stokes phenomena\\ and connection \\ problems (linear and\\ nonlinear) for\\ ordinary differential\\ equations in\\ the complex domain};
\draw (84.91999999999999, -31.900000000000006) rectangle (90.76999999999998,-35.50000000000001);
\draw(90.86999999999998, -31.900000000000006) node[anchor=north west,align=left] {Inverse problems \\ (Riemann-Hilbert, \\ inverse differential\\ Galois, etc.) for\\ ordinary differential\\ equations in\\ the complex domain};
\draw (90.86999999999998, -31.900000000000006) rectangle (96.71999999999997,-35.50000000000001);
\draw(78.96999999999998, -35.60000000000001) node[anchor=north west,align=left] {Topological \\ structure of \\ trajectories of \\ ordinary differential\\ equations in\\ the complex domain};
\draw (78.96999999999998, -35.60000000000001) rectangle (84.81999999999998,-38.70000000000001);
\draw(84.91999999999999, -35.60000000000001) node[anchor=north west,align=left] {Ordinary \\ differential \\ equations on complex\\ manifolds};
\draw (84.91999999999999, -35.60000000000001) rectangle (90.51999999999998,-37.70000000000001);
\draw(90.61999999999998, -35.60000000000001) node[anchor=north west,align=left] {Nonlinear ordinary\\ differential\\ equations and\\ systems in the\\ complex domain};
\draw (90.61999999999998, -35.60000000000001) rectangle (95.71999999999997,-38.20000000000001);
\draw(78.96999999999998, -38.800000000000004) node[anchor=north west,align=left] {Spectral theory\\ for ordinary\\ differential\\ operators in \\ the complex domain};
\draw (78.96999999999998, -38.800000000000004) rectangle (84.06999999999998,-41.400000000000006);
\draw(84.16999999999999, -38.800000000000004) node[anchor=north west,align=left] {Asymptotics and\\ summation methods\\ for ordinary\\ differential\\ equations in the\\ complex domain};
\draw (84.16999999999999, -38.800000000000004) rectangle (89.01999999999998,-41.900000000000006);
\draw(89.11999999999998, -38.800000000000004) node[anchor=north west,align=left] {Isomonodromic\\ deformations \\ for ordinary \\ differential \\ equations in the\\ complex domain};
\draw (89.11999999999998, -38.800000000000004) rectangle (93.71999999999997,-41.900000000000006);
\draw(78.96999999999998, -42.0) node[anchor=north west,align=left] {Linear ordinary\\ differential\\ equations and\\ systems in the\\ complex domain};
\draw (78.96999999999998, -42.0) rectangle (83.31999999999998,-44.6);
\draw(98.16999999999999, -23.500000000000004) node[anchor=north west,align=left] {\large Control problems including ordinary differential equations};
\draw (98.16999999999999, -23.500000000000004) rectangle (118.22,-27.200000000000003);
\draw(99.16999999999999, -24.500000000000004) node[anchor=north west,align=left] {Chaos control\\ for problems\\ involving \\ ordinary \\ differential equations};
\draw (99.16999999999999, -24.500000000000004) rectangle (105.26999999999998,-27.100000000000005);
\draw(105.36999999999999, -24.500000000000004) node[anchor=north west,align=left] {Control \\ problems involving\\ ordinary\\ differential\\ equations};
\draw (105.36999999999999, -24.500000000000004) rectangle (110.46999999999998,-27.100000000000005);
\draw(110.57, -24.500000000000004) node[anchor=north west,align=left] {Stabilization\\ of solutions\\ to ordinary\\ differential\\ equations};
\draw (110.57, -24.500000000000004) rectangle (114.41999999999999,-27.100000000000005);
\draw(114.51999999999998, -24.500000000000004) node[anchor=north west,align=left] {Bifurcation\\ control\\ of ordinary\\ differential\\ equations};
\draw (114.51999999999998, -24.500000000000004) rectangle (118.11999999999998,-27.100000000000005);
\draw(98.16999999999999, -27.300000000000004) node[anchor=north west,align=left] {\large Stability theory for ordinary differential equations};
\draw (98.16999999999999, -27.300000000000004) rectangle (117.76999999999998,-39.60000000000001);
\draw(99.16999999999999, -28.300000000000004) node[anchor=north west,align=left] {Structural \\ stability and analogous\\ concepts \\ of solutions to\\ ordinary \\ differential equations};
\draw (99.16999999999999, -28.300000000000004) rectangle (105.51999999999998,-31.400000000000006);
\draw(105.61999999999999, -28.300000000000004) node[anchor=north west,align=left] {Asymptotic \\ properties of \\ solutions to \\ ordinary \\ differential equations};
\draw (105.61999999999999, -28.300000000000004) rectangle (111.71999999999998,-30.900000000000006);
\draw(111.82, -28.300000000000004) node[anchor=north west,align=left] {Synchronization\\ of solutions\\ to \\ ordinary differential\\ equations};
\draw (111.82, -28.300000000000004) rectangle (117.66999999999999,-30.900000000000006);
\draw(99.16999999999999, -31.500000000000004) node[anchor=north west,align=left] {Characteristic\\ and Lyapunov\\ exponents of \\ ordinary \\ differential equations};
\draw (99.16999999999999, -31.500000000000004) rectangle (105.26999999999998,-34.1);
\draw(105.36999999999999, -31.500000000000004) node[anchor=north west,align=left] {Dichotomy, \\ trichotomy of \\ solutions to \\ ordinary \\ differential equations};
\draw (105.36999999999999, -31.500000000000004) rectangle (111.46999999999998,-34.1);
\draw(111.57, -31.500000000000004) node[anchor=north west,align=left] {Global stability\\ of \\ solutions to \\ ordinary \\ differential equations};
\draw (111.57, -31.500000000000004) rectangle (117.66999999999999,-34.1);
\draw(99.16999999999999, -34.2) node[anchor=north west,align=left] {Stability of \\ manifolds of \\ solutions to \\ ordinary differential\\ equations};
\draw (99.16999999999999, -34.2) rectangle (105.01999999999998,-36.800000000000004);
\draw(105.11999999999999, -34.2) node[anchor=north west,align=left] {Perturbations\\ of ordinary\\ differential\\ equations};
\draw (105.11999999999999, -34.2) rectangle (108.96999999999998,-36.300000000000004);
\draw(109.07, -34.2) node[anchor=north west,align=left] {Singular \\ perturbations\\ of ordinary\\ differential\\ equations};
\draw (109.07, -34.2) rectangle (112.91999999999999,-36.800000000000004);
\draw(113.01999999999998, -34.2) node[anchor=north west,align=left] {Stability \\ of solutions\\ to ordinary\\ differential\\ equations};
\draw (113.01999999999998, -34.2) rectangle (116.61999999999998,-36.800000000000004);
\draw(99.16999999999999, -36.900000000000006) node[anchor=north west,align=left] {Attractors\\ of solutions\\ to ordinary\\ differential\\ equations};
\draw (99.16999999999999, -36.900000000000006) rectangle (102.76999999999998,-39.50000000000001);
\draw(98.16999999999999, -39.70000000000001) node[anchor=north west,align=left] {\large Ordinary differential equations and systems with randomness};
\draw (98.16999999999999, -39.70000000000001) rectangle (117.05999999999999,-43.40000000000001);
\draw(99.16999999999999, -40.70000000000001) node[anchor=north west,align=left] {Bifurcation of \\ solutions to \\ ordinary differential\\ equations \\ involving randomness};
\draw (99.16999999999999, -40.70000000000001) rectangle (105.01999999999998,-43.30000000000001);
\draw(105.11999999999999, -40.70000000000001) node[anchor=north west,align=left] {Resonance phenomena\\ for ordinary\\ differential\\ equations \\ involving randomness};
\draw (105.11999999999999, -40.70000000000001) rectangle (110.71999999999998,-43.30000000000001);
\draw(110.82, -40.70000000000001) node[anchor=north west,align=left] {Ordinary \\ differential\\ equations \\ and systems\\ with randomness};
\draw (110.82, -40.70000000000001) rectangle (115.16999999999999,-43.30000000000001);
\draw(118.32, -23.500000000000004) node[anchor=north west,align=left] {\large Dynamic equations on time scales or measure chains};
\draw (118.32, -23.500000000000004) rectangle (134.42,-26.700000000000003);
\draw(119.32, -24.500000000000004) node[anchor=north west,align=left] {Dynamic \\ equations on time\\ scales or\\ measure chains};
\draw (119.32, -24.500000000000004) rectangle (124.16999999999999,-26.600000000000005);
\draw(77.96999999999998, -44.80000000000001) node[anchor=north west,align=left] {\large Boundary value problems for ordinary differential equations};
\draw (77.96999999999998, -44.80000000000001) rectangle (97.57,-63.500000000000014);
\draw(78.96999999999998, -45.80000000000001) node[anchor=north west,align=left] {Linear boundary\\ value problems \\ for ordinary \\ differential equations\\ with nonlinear\\ dependence on\\ the spectral parameter};
\draw (78.96999999999998, -45.80000000000001) rectangle (85.06999999999998,-49.40000000000001);
\draw(85.16999999999999, -45.80000000000001) node[anchor=north west,align=left] {Parameter dependent\\ boundary \\ value problems \\ for ordinary \\ differential equations};
\draw (85.16999999999999, -45.80000000000001) rectangle (91.26999999999998,-48.40000000000001);
\draw(91.36999999999998, -45.80000000000001) node[anchor=north west,align=left] {Boundary \\ eigenvalue \\ problems for \\ ordinary \\ differential equations};
\draw (91.36999999999998, -45.80000000000001) rectangle (97.46999999999997,-48.40000000000001);
\draw(78.96999999999998, -49.500000000000014) node[anchor=north west,align=left] {Nonlocal and\\ multipoint \\ boundary value\\ problems for\\ ordinary \\ differential equations};
\draw (78.96999999999998, -49.500000000000014) rectangle (85.06999999999998,-52.600000000000016);
\draw(85.16999999999999, -49.500000000000014) node[anchor=north west,align=left] {Nonlinear \\ boundary value \\ problems for \\ ordinary \\ differential equations};
\draw (85.16999999999999, -49.500000000000014) rectangle (91.26999999999998,-52.100000000000016);
\draw(91.36999999999998, -49.500000000000014) node[anchor=north west,align=left] {Singular nonlinear\\ boundary \\ value problems for\\ ordinary \\ differential equations};
\draw (91.36999999999998, -49.500000000000014) rectangle (97.46999999999997,-52.100000000000016);
\draw(78.96999999999998, -52.70000000000001) node[anchor=north west,align=left] {Positive solutions\\ to nonlinear\\ boundary value\\ problems for\\ ordinary \\ differential equations};
\draw (78.96999999999998, -52.70000000000001) rectangle (85.06999999999998,-55.80000000000001);
\draw(85.16999999999999, -52.70000000000001) node[anchor=north west,align=left] {Boundary value\\ problems with\\ impulses for\\ ordinary \\ differential equations};
\draw (85.16999999999999, -52.70000000000001) rectangle (91.26999999999998,-55.30000000000001);
\draw(91.36999999999998, -52.70000000000001) node[anchor=north west,align=left] {Boundary value\\ problems on\\ infinite \\ intervals for \\ ordinary \\ differential equations};
\draw (91.36999999999998, -52.70000000000001) rectangle (97.46999999999997,-55.80000000000001);
\draw(78.96999999999998, -55.90000000000001) node[anchor=north west,align=left] {Boundary value\\ problems on\\ graphs and \\ networks for \\ ordinary \\ differential equations};
\draw (78.96999999999998, -55.90000000000001) rectangle (85.06999999999998,-59.000000000000014);
\draw(85.16999999999999, -55.90000000000001) node[anchor=north west,align=left] {Applications\\ of boundary \\ value problems\\ involving \\ ordinary \\ differential equations};
\draw (85.16999999999999, -55.90000000000001) rectangle (91.26999999999998,-59.000000000000014);
\draw(91.36999999999998, -55.90000000000001) node[anchor=north west,align=left] {Linear boundary\\ value \\ problems for \\ ordinary differential\\ equations};
\draw (91.36999999999998, -55.90000000000001) rectangle (97.21999999999997,-58.500000000000014);
\draw(78.96999999999998, -59.10000000000001) node[anchor=north west,align=left] {Weyl theory and\\ its generalizations\\ for \\ ordinary differential\\ equations};
\draw (78.96999999999998, -59.10000000000001) rectangle (84.81999999999998,-61.70000000000001);
\draw(84.91999999999999, -59.10000000000001) node[anchor=north west,align=left] {Green’s functions\\ for \\ ordinary differential\\ equations};
\draw (84.91999999999999, -59.10000000000001) rectangle (90.76999999999998,-61.20000000000001);
\draw(90.86999999999998, -59.10000000000001) node[anchor=north west,align=left] {Special ordinary\\ differential\\ equations\\ (Mathieu, \\ Hill, Bessel, etc.)};
\draw (90.86999999999998, -59.10000000000001) rectangle (96.21999999999997,-61.70000000000001);
\draw(78.96999999999998, -61.80000000000001) node[anchor=north west,align=left] {Sturm-Liouville\\ theory};
\draw (78.96999999999998, -61.80000000000001) rectangle (83.31999999999998,-63.40000000000001);
\draw(97.66999999999999, -44.80000000000001) node[anchor=north west,align=left] {\large Qualitative theory for ordinary differential equations};
\draw (97.66999999999999, -44.80000000000001) rectangle (117.26999999999998,-67.70000000000002);
\draw(98.66999999999999, -45.80000000000001) node[anchor=north west,align=left] {Ordinary differential\\ equations and \\ connections with real \\ algebraic geometry \\ (fewnomials, desingularization,\\ zeros of \\ abelian integrals, etc.)};
\draw (98.66999999999999, -45.80000000000001) rectangle (107.01999999999998,-49.40000000000001);
\draw(107.11999999999999, -45.80000000000001) node[anchor=north west,align=left] {Theory of limit cycles\\ of polynomial and \\ analytic vector fields\\ (existence, uniqueness,\\ bounds, Hilbert’s\\ 16th problem and \\ ramifications) for ordinary\\ differential equations};
\draw (107.11999999999999, -45.80000000000001) rectangle (114.46999999999998,-49.90000000000001);
\draw(98.66999999999999, -50.000000000000014) node[anchor=north west,align=left] {Topological structure\\ of integral\\ curves, singular\\ points, limit \\ cycles of ordinary \\ differential equations};
\draw (98.66999999999999, -50.000000000000014) rectangle (104.76999999999998,-53.100000000000016);
\draw(104.86999999999999, -50.000000000000014) node[anchor=north west,align=left] {Oscillation theory,\\ zeros, \\ disconjugacy and \\ comparison theory for\\ ordinary \\ differential equations};
\draw (104.86999999999999, -50.000000000000014) rectangle (110.96999999999998,-53.100000000000016);
\draw(111.07, -50.000000000000014) node[anchor=north west,align=left] {Nonlinear \\ oscillations and\\ coupled \\ oscillators for \\ ordinary \\ differential equations};
\draw (111.07, -50.000000000000014) rectangle (117.16999999999999,-53.100000000000016);
\draw(98.66999999999999, -53.20000000000001) node[anchor=north west,align=left] {Almost and \\ pseudo-almost periodic\\ solutions \\ to ordinary \\ differential equations};
\draw (98.66999999999999, -53.20000000000001) rectangle (104.76999999999998,-55.80000000000001);
\draw(104.86999999999999, -53.20000000000001) node[anchor=north west,align=left] {Homoclinic and\\ heteroclinic\\ solutions to \\ ordinary \\ differential equations};
\draw (104.86999999999999, -53.20000000000001) rectangle (110.96999999999998,-55.80000000000001);
\draw(111.07, -53.20000000000001) node[anchor=north west,align=left] {Equivalence and\\ asymptotic \\ equivalence of\\ ordinary \\ differential equations};
\draw (111.07, -53.20000000000001) rectangle (117.16999999999999,-55.80000000000001);
\draw(98.66999999999999, -55.900000000000006) node[anchor=north west,align=left] {Growth and \\ boundedness of \\ solutions to \\ ordinary differential\\ equations};
\draw (98.66999999999999, -55.900000000000006) rectangle (104.51999999999998,-58.50000000000001);
\draw(104.61999999999999, -55.900000000000006) node[anchor=north west,align=left] {Transformation\\ and reduction\\ of ordinary \\ differential \\ equations and \\ systems, normal forms};
\draw (104.61999999999999, -55.900000000000006) rectangle (110.46999999999998,-59.00000000000001);
\draw(110.57, -55.900000000000006) node[anchor=north west,align=left] {Periodic \\ solutions to \\ ordinary differential\\ equations};
\draw (110.57, -55.900000000000006) rectangle (116.41999999999999,-58.00000000000001);
\draw(98.66999999999999, -59.10000000000001) node[anchor=north west,align=left] {Complex behavior\\ and chaotic\\ systems of \\ ordinary differential\\ equations};
\draw (98.66999999999999, -59.10000000000001) rectangle (104.51999999999998,-61.70000000000001);
\draw(104.61999999999999, -59.10000000000001) node[anchor=north west,align=left] {Averaging \\ method for ordinary\\ differential\\ equations};
\draw (104.61999999999999, -59.10000000000001) rectangle (109.96999999999998,-61.20000000000001);
\draw(110.07, -59.10000000000001) node[anchor=north west,align=left] {Monotone \\ systems involving\\ ordinary\\ differential\\ equations};
\draw (110.07, -59.10000000000001) rectangle (114.91999999999999,-61.70000000000001);
\draw(98.66999999999999, -61.80000000000001) node[anchor=north west,align=left] {Qualitative \\ investigation\\ and simulation\\ of ordinary\\ differential\\ equation models};
\draw (98.66999999999999, -61.80000000000001) rectangle (103.01999999999998,-64.9);
\draw(103.11999999999999, -61.80000000000001) node[anchor=north west,align=left] {Multifrequency\\ systems\\ of ordinary\\ differential\\ equations};
\draw (103.11999999999999, -61.80000000000001) rectangle (107.21999999999998,-64.4);
\draw(107.32, -61.80000000000001) node[anchor=north west,align=left] {Invariant \\ manifolds for\\ ordinary\\ differential\\ equations};
\draw (107.32, -61.80000000000001) rectangle (111.16999999999999,-64.4);
\draw(111.26999999999998, -61.80000000000001) node[anchor=north west,align=left] {Symmetries,\\ invariants\\ of ordinary\\ differential\\ equations};
\draw (111.26999999999998, -61.80000000000001) rectangle (114.86999999999998,-64.4);
\draw(98.66999999999999, -65.0) node[anchor=north west,align=left] {Bifurcation\\ theory \\ for ordinary\\ differential\\ equations};
\draw (98.66999999999999, -65.0) rectangle (102.26999999999998,-67.6);
\draw(102.36999999999999, -65.0) node[anchor=north west,align=left] {Relaxation\\ oscillations\\ for ordinary\\ differential\\ equations};
\draw (102.36999999999999, -65.0) rectangle (105.96999999999998,-67.6);
\draw(106.07, -65.0) node[anchor=north west,align=left] {Ordinary \\ differential\\ equations \\ and systems\\ on manifolds};
\draw (106.07, -65.0) rectangle (109.66999999999999,-67.6);
\draw(109.76999999999998, -65.0) node[anchor=north west,align=left] {Hysteresis\\ for ordinary\\ differential\\ equations};
\draw (109.76999999999998, -65.0) rectangle (113.36999999999998,-67.1);
\draw(77.96999999999998, -63.600000000000016) node[anchor=north west,align=left] {\large Differential equations in abstract spaces};
\draw (77.96999999999998, -63.600000000000016) rectangle (91.27999999999999,-66.80000000000001);
\draw(78.96999999999998, -64.60000000000002) node[anchor=north west,align=left] {Linear \\ differential \\ equations in \\ abstract spaces};
\draw (78.96999999999998, -64.60000000000002) rectangle (83.31999999999998,-66.70000000000002);
\draw(83.41999999999999, -64.60000000000002) node[anchor=north west,align=left] {Nonlinear \\ differential \\ equations in\\ abstract spaces};
\draw (83.41999999999999, -64.60000000000002) rectangle (87.76999999999998,-66.70000000000002);
\draw(87.86999999999998, -64.60000000000002) node[anchor=north west,align=left] {Evolution\\ inclusions};
\draw (87.86999999999998, -64.60000000000002) rectangle (90.96999999999997,-66.20000000000002);
\draw(117.36999999999999, -44.80000000000001) node[anchor=north west,align=left] {\large Asymptotic theory for ordinary differential equations};
\draw (117.36999999999999, -44.80000000000001) rectangle (136.97,-54.40000000000001);
\draw(118.36999999999999, -45.80000000000001) node[anchor=north west,align=left] {Asymptotic \\ expansions of \\ solutions to \\ ordinary \\ differential equations};
\draw (118.36999999999999, -45.80000000000001) rectangle (124.46999999999998,-48.40000000000001);
\draw(124.57, -45.80000000000001) node[anchor=north west,align=left] {Perturbations,\\ asymptotics \\ of solutions to\\ ordinary \\ differential equations};
\draw (124.57, -45.80000000000001) rectangle (130.67,-48.40000000000001);
\draw(130.76999999999998, -45.80000000000001) node[anchor=north west,align=left] {Singular \\ perturbations, general\\ theory for\\ ordinary \\ differential equations};
\draw (130.76999999999998, -45.80000000000001) rectangle (136.86999999999998,-48.40000000000001);
\draw(118.36999999999999, -48.500000000000014) node[anchor=north west,align=left] {Singular \\ perturbations, turning\\ point theory,\\ WKB methods for\\ ordinary \\ differential equations};
\draw (118.36999999999999, -48.500000000000014) rectangle (124.46999999999998,-51.600000000000016);
\draw(124.57, -48.500000000000014) node[anchor=north west,align=left] {Canard solutions\\ to \\ ordinary differential\\ equations};
\draw (124.57, -48.500000000000014) rectangle (130.42,-50.600000000000016);
\draw(130.51999999999998, -48.500000000000014) node[anchor=north west,align=left] {Methods of \\ nonstandard \\ analysis for \\ ordinary differential\\ equations};
\draw (130.51999999999998, -48.500000000000014) rectangle (136.36999999999998,-51.100000000000016);
\draw(118.36999999999999, -51.70000000000001) node[anchor=north west,align=left] {Multiple \\ scale methods\\ for ordinary\\ differential\\ equations};
\draw (118.36999999999999, -51.70000000000001) rectangle (122.21999999999998,-54.30000000000001);
\draw(77.96999999999998, -67.80000000000001) node[anchor=north west,align=left] {\large Ordinary differential operators};
\draw (77.96999999999998, -67.80000000000001) rectangle (91.36999999999998,-81.60000000000001);
\draw(78.96999999999998, -68.80000000000001) node[anchor=north west,align=left] {Eigenfunctions, \\ eigenfunction \\ expansions, completeness\\ of eigenfunctions\\ of ordinary\\ differential operators};
\draw (78.96999999999998, -68.80000000000001) rectangle (85.56999999999998,-71.9);
\draw(85.66999999999999, -68.80000000000001) node[anchor=north west,align=left] {General \\ spectral theory\\ of ordinary\\ differential\\ operators};
\draw (85.66999999999999, -68.80000000000001) rectangle (90.01999999999998,-71.4);
\draw(78.96999999999998, -72.00000000000001) node[anchor=north west,align=left] {Asymptotic distribution\\ of eigenvalues,\\ asymptotic \\ theory of eigenfunctions\\ for ordinary \\ differential operators};
\draw (78.96999999999998, -72.00000000000001) rectangle (85.56999999999998,-75.10000000000001);
\draw(85.66999999999999, -72.00000000000001) node[anchor=north west,align=left] {Nonlinear\\ ordinary \\ differential\\ operators};
\draw (85.66999999999999, -72.00000000000001) rectangle (89.26999999999998,-74.10000000000001);
\draw(78.96999999999998, -75.20000000000002) node[anchor=north west,align=left] {Numerical approximation\\ of eigenvalues\\ and of other \\ parts of the spectrum\\ of ordinary \\ differential operators};
\draw (78.96999999999998, -75.20000000000002) rectangle (85.31999999999998,-78.30000000000001);
\draw(85.41999999999999, -75.20000000000002) node[anchor=north west,align=left] {Particular \\ ordinary differential\\ operators\\ (Dirac, \\ one-dimensional \\ Schrödinger, etc.)};
\draw (85.41999999999999, -75.20000000000002) rectangle (91.26999999999998,-78.30000000000001);
\draw(78.96999999999998, -78.4) node[anchor=north west,align=left] {Eigenvalues, \\ estimation of \\ eigenvalues, upper and\\ lower bounds \\ of ordinary \\ differential operators};
\draw (78.96999999999998, -78.4) rectangle (85.06999999999998,-81.5);
\draw(85.16999999999999, -78.4) node[anchor=north west,align=left] {Scattering theory,\\ inverse \\ scattering involving\\ ordinary \\ differential operators};
\draw (85.16999999999999, -78.4) rectangle (91.26999999999998,-81.0);
\draw(1, -52.5) node[anchor=north west,align=left] {\LARGE Field theory and polynomials};
\draw (1, -52.5) rectangle (28.64,-79.4);
\draw(2, -53.5) node[anchor=north west,align=left] {\large Connections between field theory and logic};
\draw (2, -53.5) rectangle (17.299999999999997,-56.7);
\draw(3, -54.5) node[anchor=north west,align=left] {Ultraproducts\\ and \\ field theory};
\draw (3, -54.5) rectangle (6.85,-56.1);
\draw(6.949999999999999, -54.5) node[anchor=north west,align=left] {Decidability\\ and \\ field theory};
\draw (6.949999999999999, -54.5) rectangle (10.549999999999999,-56.1);
\draw(10.65, -54.5) node[anchor=north west,align=left] {Nonstandard\\ arithmetic\\ and\\ field theory};
\draw (10.65, -54.5) rectangle (14.25,-56.6);
\draw(14.35, -54.5) node[anchor=north west,align=left] {Model \\ theory \\ of fields};
\draw (14.35, -54.5) rectangle (17.2,-56.1);
\draw(17.4, -53.5) node[anchor=north west,align=left] {\large Homological methods (field theory)};
\draw (17.4, -53.5) rectangle (28.54,-56.7);
\draw(18.4, -54.5) node[anchor=north west,align=left] {Cohomological\\ dimension\\ of fields};
\draw (18.4, -54.5) rectangle (22.25,-56.6);
\draw(22.349999999999998, -54.5) node[anchor=north west,align=left] {Galois\\ cohomology};
\draw (22.349999999999998, -54.5) rectangle (25.45,-56.1);
\draw(2, -56.8) node[anchor=north west,align=left] {\large General field theory};
\draw (2, -56.8) rectangle (14.149999999999999,-64.39999999999999);
\draw(3, -57.8) node[anchor=north west,align=left] {Polynomials \\ in general \\ fields (irreducibility,\\ etc.)};
\draw (3, -57.8) rectangle (9.35,-59.9);
\draw(9.45, -57.8) node[anchor=north west,align=left] {Finite \\ fields \\ (field-theoretic\\ aspects)};
\draw (9.45, -57.8) rectangle (14.049999999999999,-59.9);
\draw(3, -60.0) node[anchor=north west,align=left] {Hilbertian \\ fields; Hilbert’s\\ irreducibility theorem};
\draw (3, -60.0) rectangle (9.1,-62.1);
\draw(9.2, -60.0) node[anchor=north west,align=left] {Skew fields,\\ division\\ rings};
\draw (9.2, -60.0) rectangle (12.799999999999999,-61.6);
\draw(3, -62.199999999999996) node[anchor=north west,align=left] {Special \\ polynomials\\ in general\\ fields};
\draw (3, -62.199999999999996) rectangle (6.35,-64.3);
\draw(6.449999999999999, -62.199999999999996) node[anchor=north west,align=left] {Equations\\ in general\\ fields};
\draw (6.449999999999999, -62.199999999999996) rectangle (9.549999999999999,-63.8);
\draw(9.649999999999999, -62.199999999999996) node[anchor=north west,align=left] {Field \\ arithmetic};
\draw (9.649999999999999, -62.199999999999996) rectangle (12.749999999999998,-63.3);
\draw(14.249999999999998, -56.8) node[anchor=north west,align=left] {\large Differential and difference algebra};
\draw (14.249999999999998, -56.8) rectangle (26.349999999999998,-61.699999999999996);
\draw(15.249999999999998, -57.8) node[anchor=north west,align=left] {Differential\\ algebra};
\draw (15.249999999999998, -57.8) rectangle (18.849999999999998,-59.4);
\draw(18.949999999999996, -57.8) node[anchor=north west,align=left] {Abstract \\ differential\\ equations};
\draw (18.949999999999996, -57.8) rectangle (22.549999999999997,-59.4);
\draw(22.65, -57.8) node[anchor=north west,align=left] {\(p\)-adic\\ differential\\ equations};
\draw (22.65, -57.8) rectangle (26.25,-59.9);
\draw(15.249999999999998, -60.0) node[anchor=north west,align=left] {Difference\\ algebra};
\draw (15.249999999999998, -60.0) rectangle (18.349999999999998,-61.6);
\draw(14.249999999999998, -61.8) node[anchor=north west,align=left] {\large Computational methods\\ for problems \\ pertaining to field theory};
\draw (14.249999999999998, -61.8) rectangle (22.909999999999997,-63.4);
\draw(2, -64.5) node[anchor=north west,align=left] {\large Real and complex fields};
\draw (2, -64.5) rectangle (12.399999999999999,-71.4);
\draw(3, -65.5) node[anchor=north west,align=left] {Polynomials in\\ real and complex\\ fields: location\\ of zeros \\ (algebraic theorems)};
\draw (3, -65.5) rectangle (8.6,-68.1);
\draw(3, -68.2) node[anchor=north west,align=left] {Fields related\\ with sums of\\ squares (formally\\ real fields,\\ Pythagorean\\ fields, etc.)};
\draw (3, -68.2) rectangle (7.85,-71.3);
\draw(7.949999999999999, -68.2) node[anchor=north west,align=left] {Polynomials \\ in real and \\ complex fields:\\ factorization};
\draw (7.949999999999999, -68.2) rectangle (12.299999999999999,-70.3);
\draw(12.499999999999998, -64.5) node[anchor=north west,align=left] {\large Generalizations of fields};
\draw (12.499999999999998, -64.5) rectangle (20.849999999999998,-66.7);
\draw(13.499999999999998, -65.5) node[anchor=north west,align=left] {Near-fields};
\draw (13.499999999999998, -65.5) rectangle (16.849999999999998,-66.6);
\draw(16.949999999999996, -65.5) node[anchor=north west,align=left] {Semifields};
\draw (16.949999999999996, -65.5) rectangle (20.049999999999997,-66.6);
\draw(12.499999999999998, -66.8) node[anchor=north west,align=left] {\large History of\\ field theory};
\draw (12.499999999999998, -66.8) rectangle (16.819999999999997,-67.89999999999999);
\draw(2, -71.5) node[anchor=north west,align=left] {\large Field extensions};
\draw (2, -71.5) rectangle (10.149999999999999,-78.1);
\draw(3, -72.5) node[anchor=north west,align=left] {Transcendental\\ field\\ extensions};
\draw (3, -72.5) rectangle (7.1,-74.1);
\draw(7.199999999999999, -72.5) node[anchor=north west,align=left] {Inverse\\ Galois\\ theory};
\draw (7.199999999999999, -72.5) rectangle (9.549999999999999,-74.1);
\draw(3, -74.2) node[anchor=north west,align=left] {Separable\\ extensions,\\ Galois theory};
\draw (3, -74.2) rectangle (6.85,-76.3);
\draw(6.949999999999999, -74.2) node[anchor=north west,align=left] {Algebraic\\ field\\ extensions};
\draw (6.949999999999999, -74.2) rectangle (10.049999999999999,-75.8);
\draw(3, -76.4) node[anchor=north west,align=left] {Inseparable\\ field\\ extensions};
\draw (3, -76.4) rectangle (6.35,-78.0);
\draw(10.249999999999998, -71.5) node[anchor=north west,align=left] {\large Topological fields};
\draw (10.249999999999998, -71.5) rectangle (18.4,-79.3);
\draw(11.249999999999998, -72.5) node[anchor=north west,align=left] {Non-Archimedean\\ valued fields};
\draw (11.249999999999998, -72.5) rectangle (15.599999999999998,-74.1);
\draw(15.699999999999998, -72.5) node[anchor=north west,align=left] {Ordered\\ fields};
\draw (15.699999999999998, -72.5) rectangle (18.049999999999997,-73.6);
\draw(11.249999999999998, -74.2) node[anchor=north west,align=left] {Krasner-Tate\\ algebras};
\draw (11.249999999999998, -74.2) rectangle (14.849999999999998,-75.8);
\draw(14.949999999999998, -74.2) node[anchor=north west,align=left] {Topological\\ semifields};
\draw (14.949999999999998, -74.2) rectangle (18.299999999999997,-75.8);
\draw(11.249999999999998, -75.9) node[anchor=north west,align=left] {Formally\\ \(p\)-adic\\ fields};
\draw (11.249999999999998, -75.9) rectangle (14.349999999999998,-77.5);
\draw(14.449999999999998, -75.9) node[anchor=north west,align=left] {General\\ valuation\\ theory\\ for fields};
\draw (14.449999999999998, -75.9) rectangle (17.549999999999997,-78.0);
\draw(11.249999999999998, -78.1) node[anchor=north west,align=left] {Normed\\ fields};
\draw (11.249999999999998, -78.1) rectangle (13.349999999999998,-79.19999999999999);
\draw(13.45, -78.1) node[anchor=north west,align=left] {Valued\\ fields};
\draw (13.45, -78.1) rectangle (15.549999999999999,-79.19999999999999);
\draw(1, -81.8) node[anchor=north west,align=left] {\LARGE Several complex variables and analytic spaces};
\draw (1, -81.8) rectangle (64.4,-145.3);
\draw(2, -82.8) node[anchor=north west,align=left] {\large Non-Archimedean analysis (should also be assigned at least one other classification number from Section 32-XX describing the type of problem)};
\draw (2, -82.8) rectangle (46.31,-87.5);
\draw(3, -83.8) node[anchor=north west,align=left] {Non-Archimedean \\ analysis (should also be\\ assigned at least \\ one other classification\\ number from Section\\ 32-XX describing\\ the type of problem)};
\draw (3, -83.8) rectangle (9.6,-87.39999999999999);
\draw(46.410000000000004, -82.8) node[anchor=north west,align=left] {\large Geometric convexity in several complex variables};
\draw (46.410000000000004, -82.8) rectangle (62.760000000000005,-88.7);
\draw(47.410000000000004, -83.8) node[anchor=north west,align=left] {Analytical \\ consequences of \\ geometric convexity\\ (vanishing\\ theorems, etc.)};
\draw (47.410000000000004, -83.8) rectangle (52.760000000000005,-86.39999999999999);
\draw(52.86, -83.8) node[anchor=north west,align=left] {Other notions\\ of convexity\\ in relation\\ to several \\ complex variables};
\draw (52.86, -83.8) rectangle (57.71,-86.39999999999999);
\draw(57.81, -83.8) node[anchor=north west,align=left] {Invariant \\ metrics and \\ pseudodistances \\ in several \\ complex variables};
\draw (57.81, -83.8) rectangle (62.660000000000004,-86.39999999999999);
\draw(47.410000000000004, -86.5) node[anchor=north west,align=left] {\(q\)-convexity,\\ \(q\)-concavity};
\draw (47.410000000000004, -86.5) rectangle (52.010000000000005,-88.1);
\draw(52.11, -86.5) node[anchor=north west,align=left] {Finite-type\\ conditions \\ for the boundary\\ of a domain};
\draw (52.11, -86.5) rectangle (56.71,-88.6);
\draw(56.81, -86.5) node[anchor=north west,align=left] {Topological\\ consequences\\ of geometric\\ convexity};
\draw (56.81, -86.5) rectangle (60.410000000000004,-88.6);
\draw(2, -88.8) node[anchor=north west,align=left] {\large Holomorphic functions of several complex variables};
\draw (2, -88.8) rectangle (21.1,-113.9);
\draw(3, -89.8) node[anchor=north west,align=left] {Other generalizations\\ of function theory\\ of one complex \\ variable (should also be\\ assigned at least \\ one classification \\ number from Section 30-XX)};
\draw (3, -89.8) rectangle (10.1,-93.39999999999999);
\draw(10.2, -89.8) node[anchor=north west,align=left] {Other spaces of \\ holomorphic functions of\\ several complex \\ variables (e.g., bounded\\ mean oscillation \\ (BMOA), vanishing mean\\ oscillation (VMOA))};
\draw (10.2, -89.8) rectangle (16.799999999999997,-93.39999999999999);
\draw(16.9, -89.8) node[anchor=north west,align=left] {Hyperfunctions};
\draw (16.9, -89.8) rectangle (21.0,-90.89999999999999);
\draw(16.9, -91.0) node[anchor=north west,align=left] {Residues\\ for several\\ complex\\ variables};
\draw (16.9, -91.0) rectangle (20.25,-93.1);
\draw(3, -93.5) node[anchor=north west,align=left] {Integral \\ representations, \\ constructed kernels\\ (e.g., \\ Cauchy, Fantappiè-type\\ kernels)};
\draw (3, -93.5) rectangle (9.1,-96.6);
\draw(9.2, -93.5) node[anchor=north west,align=left] {Normal families \\ of holomorphic \\ functions, mappings\\ of several complex\\ variables, and\\ related topics \\ (taut manifolds etc.)};
\draw (9.2, -93.5) rectangle (15.049999999999999,-97.1);
\draw(15.149999999999999, -93.5) node[anchor=north west,align=left] {Functional analysis\\ techniques \\ applied to functions\\ of several \\ complex variables};
\draw (15.149999999999999, -93.5) rectangle (20.75,-96.1);
\draw(3, -97.2) node[anchor=north west,align=left] {\(H^p\)-spaces,\\ Nevanlinna \\ spaces of functions\\ in several \\ complex variables};
\draw (3, -97.2) rectangle (8.35,-99.8);
\draw(8.45, -97.2) node[anchor=north west,align=left] {Banach algebra\\ techniques applied\\ to functions\\ of several \\ complex variables};
\draw (8.45, -97.2) rectangle (13.549999999999999,-99.8);
\draw(13.649999999999999, -97.2) node[anchor=north west,align=left] {Power series,\\ series of\\ functions of\\ several \\ complex variables};
\draw (13.649999999999999, -97.2) rectangle (18.5,-99.8);
\draw(3, -99.9) node[anchor=north west,align=left] {Polynomials\\ and rational\\ functions \\ of several \\ complex variables};
\draw (3, -99.9) rectangle (7.85,-102.5);
\draw(7.949999999999999, -99.9) node[anchor=north west,align=left] {Holomorphic\\ functions of\\ several \\ complex variables};
\draw (7.949999999999999, -99.9) rectangle (12.799999999999999,-102.0);
\draw(12.899999999999999, -99.9) node[anchor=north west,align=left] {Special \\ families of \\ functions of\\ several \\ complex variables};
\draw (12.899999999999999, -99.9) rectangle (17.75,-102.5);
\draw(3, -102.6) node[anchor=north west,align=left] {Bloch functions,\\ normal\\ functions of\\ several \\ complex variables};
\draw (3, -102.6) rectangle (7.85,-105.19999999999999);
\draw(7.949999999999999, -102.6) node[anchor=north west,align=left] {Meromorphic\\ functions of\\ several \\ complex variables};
\draw (7.949999999999999, -102.6) rectangle (12.799999999999999,-104.69999999999999);
\draw(12.899999999999999, -102.6) node[anchor=north west,align=left] {Nevanlinna \\ theory; growth \\ estimates; other\\ inequalities\\ of several \\ complex variables};
\draw (12.899999999999999, -102.6) rectangle (17.75,-105.69999999999999);
\draw(3, -105.8) node[anchor=north west,align=left] {Algebras of\\ holomorphic\\ functions of\\ several \\ complex variables};
\draw (3, -105.8) rectangle (7.85,-108.39999999999999);
\draw(7.949999999999999, -105.8) node[anchor=north west,align=left] {Boundary behavior\\ of holomorphic\\ functions\\ of several \\ complex variables};
\draw (7.949999999999999, -105.8) rectangle (12.799999999999999,-108.39999999999999);
\draw(12.899999999999999, -105.8) node[anchor=north west,align=left] {Zero sets of\\ holomorphic\\ functions \\ of several \\ complex variables};
\draw (12.899999999999999, -105.8) rectangle (17.75,-108.39999999999999);
\draw(3, -108.5) node[anchor=north west,align=left] {Integral \\ representations;\\ canonical \\ kernels (Szegő,\\ Bergman, etc.)};
\draw (3, -108.5) rectangle (7.6,-111.1);
\draw(7.699999999999999, -108.5) node[anchor=north west,align=left] {Multifunctions\\ of \\ several complex\\ variables};
\draw (7.699999999999999, -108.5) rectangle (12.049999999999999,-110.6);
\draw(12.149999999999999, -108.5) node[anchor=north west,align=left] {Entire \\ functions of \\ several complex\\ variables};
\draw (12.149999999999999, -108.5) rectangle (16.5,-110.6);
\draw(16.599999999999998, -108.5) node[anchor=north west,align=left] {Bergman \\ spaces of \\ functions in \\ several complex\\ variables};
\draw (16.599999999999998, -108.5) rectangle (20.949999999999996,-111.1);
\draw(3, -111.19999999999999) node[anchor=north west,align=left] {Harmonic \\ analysis of \\ several complex\\ variables};
\draw (3, -111.19999999999999) rectangle (7.35,-113.29999999999998);
\draw(7.449999999999999, -111.19999999999999) node[anchor=north west,align=left] {Singular \\ integrals of \\ functions in \\ several complex\\ variables};
\draw (7.449999999999999, -111.19999999999999) rectangle (11.799999999999999,-113.79999999999998);
\draw(21.200000000000003, -88.8) node[anchor=north west,align=left] {\large Differential operators in several variables};
\draw (21.200000000000003, -88.8) rectangle (36.35,-97.39999999999999);
\draw(22.200000000000003, -89.8) node[anchor=north west,align=left] {\(\overline\partial_b\) and\\ \(\overline\partial_b\)-Neumann\\ operators};
\draw (22.200000000000003, -89.8) rectangle (30.550000000000004,-92.39999999999999);
\draw(30.650000000000002, -89.8) node[anchor=north west,align=left] {Other partial \\ differential \\ equations of complex\\ analysis in\\ several variables};
\draw (30.650000000000002, -89.8) rectangle (36.25,-92.39999999999999);
\draw(22.200000000000003, -92.5) node[anchor=north west,align=left] {\(\overline\partial\)\\ and\\ \(\overline\partial\)-Neumann\\ operators};
\draw (22.200000000000003, -92.5) rectangle (30.050000000000004,-95.1);
\draw(30.150000000000002, -92.5) node[anchor=north west,align=left] {Pseudodifferential\\ operators in \\ several complex\\ variables};
\draw (30.150000000000002, -92.5) rectangle (35.25,-95.1);
\draw(22.200000000000003, -95.2) node[anchor=north west,align=left] {Complex \\ Monge-Ampère\\ operators};
\draw (22.200000000000003, -95.2) rectangle (25.800000000000004,-96.8);
\draw(25.900000000000002, -95.2) node[anchor=north west,align=left] {Heat kernels\\ in several\\ complex\\ variables};
\draw (25.900000000000002, -95.2) rectangle (29.500000000000004,-97.3);
\draw(21.200000000000003, -97.5) node[anchor=north west,align=left] {\large Holomorphic mappings and correspondences};
\draw (21.200000000000003, -97.5) rectangle (35.8,-110.8);
\draw(22.200000000000003, -98.5) node[anchor=north west,align=left] {Holomorphic \\ mappings, (holomorphic)\\ embeddings \\ and related questions\\ in several\\ complex variables};
\draw (22.200000000000003, -98.5) rectangle (28.550000000000004,-101.6);
\draw(28.650000000000002, -98.5) node[anchor=north west,align=left] {Proper \\ holomorphic mappings,\\ finiteness\\ theorems};
\draw (28.650000000000002, -98.5) rectangle (34.5,-100.6);
\draw(22.200000000000003, -101.7) node[anchor=north west,align=left] {Iteration of \\ holomorphic maps, \\ fixed points of \\ holomorphic maps\\ and related \\ problems for several\\ complex variables};
\draw (22.200000000000003, -101.7) rectangle (27.800000000000004,-105.3);
\draw(27.900000000000002, -101.7) node[anchor=north west,align=left] {Picard-type \\ theorems and \\ generalizations \\ for several \\ complex variables};
\draw (27.900000000000002, -101.7) rectangle (32.75,-104.3);
\draw(22.200000000000003, -105.4) node[anchor=north west,align=left] {Value \\ distribution \\ theory in higher\\ dimensions};
\draw (22.200000000000003, -105.4) rectangle (26.800000000000004,-107.5);
\draw(26.900000000000002, -105.4) node[anchor=north west,align=left] {Meromorphic\\ mappings in\\ several complex\\ variables};
\draw (26.900000000000002, -105.4) rectangle (31.25,-107.5);
\draw(31.35, -105.4) node[anchor=north west,align=left] {Boundary \\ uniqueness of \\ mappings in \\ several complex\\ variables};
\draw (31.35, -105.4) rectangle (35.7,-108.0);
\draw(22.200000000000003, -108.1) node[anchor=north west,align=left] {Boundary \\ regularity of \\ mappings in \\ several complex\\ variables};
\draw (22.200000000000003, -108.1) rectangle (26.550000000000004,-110.69999999999999);
\draw(21.200000000000003, -110.9) node[anchor=north west,align=left] {\large Computational methods \\ for problems pertaining\\ to several complex \\ variables and analytic spaces};
\draw (21.200000000000003, -110.9) rectangle (30.790000000000003,-113.0);
\draw(36.45, -88.8) node[anchor=north west,align=left] {\large Complex spaces with a group of automorphisms};
\draw (36.45, -88.8) rectangle (50.85,-97.89999999999999);
\draw(37.45, -89.8) node[anchor=north west,align=left] {Hermitian symmetric\\ spaces, bounded\\ symmetric \\ domains, Jordan \\ algebras (complex-analytic\\ aspects)};
\draw (37.45, -89.8) rectangle (44.550000000000004,-92.89999999999999);
\draw(44.650000000000006, -89.8) node[anchor=north west,align=left] {Complex vector\\ fields, \\ holomorphic \\ foliations, \\ \(\mathbb{C}\)-actions};
\draw (44.650000000000006, -89.8) rectangle (50.75000000000001,-92.39999999999999);
\draw(37.45, -93.0) node[anchor=north west,align=left] {Automorphism\\ groups of \\ \(\mathbb{C}^n\)\\ and affine\\ manifolds};
\draw (37.45, -93.0) rectangle (42.050000000000004,-95.6);
\draw(42.150000000000006, -93.0) node[anchor=north west,align=left] {Complex Lie\\ groups, group\\ actions on\\ complex spaces};
\draw (42.150000000000006, -93.0) rectangle (46.25000000000001,-95.1);
\draw(46.35, -93.0) node[anchor=north west,align=left] {Automorphism\\ groups\\ of other \\ complex spaces};
\draw (46.35, -93.0) rectangle (50.45,-95.1);
\draw(37.45, -95.7) node[anchor=north west,align=left] {Homogeneous\\ complex\\ manifolds};
\draw (37.45, -95.7) rectangle (40.800000000000004,-97.3);
\draw(40.900000000000006, -95.7) node[anchor=north west,align=left] {Almost \\ homogeneous\\ manifolds\\ and spaces};
\draw (40.900000000000006, -95.7) rectangle (44.25000000000001,-97.8);
\draw(50.95, -88.8) node[anchor=north west,align=left] {\large Deformations of analytic structures};
\draw (50.95, -88.8) rectangle (64.3,-100.6);
\draw(51.95, -89.8) node[anchor=north west,align=left] {Moduli and \\ deformations for \\ ordinary differential\\ equations (e.g.,\\ Knizhnik-Zamolodchikov\\ equation)};
\draw (51.95, -89.8) rectangle (58.050000000000004,-92.89999999999999);
\draw(58.150000000000006, -89.8) node[anchor=north west,align=left] {Moduli of Riemann\\ surfaces, \\ Teichmüller theory\\ (complex-analytic\\ aspects in\\ several variables)};
\draw (58.150000000000006, -89.8) rectangle (63.25000000000001,-92.89999999999999);
\draw(51.95, -93.0) node[anchor=north west,align=left] {Complex-analytic\\ moduli problems};
\draw (51.95, -93.0) rectangle (56.550000000000004,-94.6);
\draw(56.650000000000006, -93.0) node[anchor=north west,align=left] {Period matrices,\\ variation\\ of Hodge\\ structure; \\ degenerations};
\draw (56.650000000000006, -93.0) rectangle (61.25000000000001,-95.6);
\draw(51.95, -95.7) node[anchor=north west,align=left] {Applications\\ of deformations\\ of analytic\\ structures\\ to the sciences};
\draw (51.95, -95.7) rectangle (56.300000000000004,-98.3);
\draw(56.400000000000006, -95.7) node[anchor=north west,align=left] {Deformations\\ of \\ fiber bundles};
\draw (56.400000000000006, -95.7) rectangle (60.25000000000001,-97.3);
\draw(60.35, -95.7) node[anchor=north west,align=left] {Deformations\\ of \\ submanifolds \\ and subspaces};
\draw (60.35, -95.7) rectangle (64.2,-97.8);
\draw(51.95, -98.4) node[anchor=north west,align=left] {Deformations\\ of\\ complex\\ structures};
\draw (51.95, -98.4) rectangle (55.550000000000004,-100.5);
\draw(55.650000000000006, -98.4) node[anchor=north west,align=left] {Deformations\\ of special\\ (e.g., CR)\\ structures};
\draw (55.650000000000006, -98.4) rectangle (59.25000000000001,-100.5);
\draw(2, -114.0) node[anchor=north west,align=left] {\large Complex singularities};
\draw (2, -114.0) rectangle (16.099999999999998,-132.2);
\draw(3, -115.0) node[anchor=north west,align=left] {Monodromy; \\ relations with \\ differential equations\\ and \(D\)-modules\\ (complex-analytic aspects)};
\draw (3, -115.0) rectangle (10.1,-118.1);
\draw(10.2, -115.0) node[anchor=north west,align=left] {Mixed Hodge \\ theory of \\ singular varieties\\ (complex-analytic\\ aspects)};
\draw (10.2, -115.0) rectangle (15.299999999999999,-117.6);
\draw(3, -118.2) node[anchor=north west,align=left] {Topological aspects\\ of complex \\ singularities: Lefschetz\\ theorems, \\ topological \\ classification, invariants};
\draw (3, -118.2) rectangle (10.1,-121.3);
\draw(10.2, -118.2) node[anchor=north west,align=left] {Singularities\\ of \\ holomorphic vector\\ fields \\ and foliations};
\draw (10.2, -118.2) rectangle (15.299999999999999,-120.8);
\draw(3, -121.4) node[anchor=north west,align=left] {Stratifications;\\ constructible\\ sheaves; \\ intersection cohomology\\ (complex-analytic\\ aspects)};
\draw (3, -121.4) rectangle (9.35,-124.5);
\draw(9.45, -121.4) node[anchor=north west,align=left] {Modifications;\\ resolution\\ of singularities\\ (complex-analytic\\ aspects)};
\draw (9.45, -121.4) rectangle (14.299999999999999,-124.0);
\draw(3, -124.6) node[anchor=north west,align=left] {Equisingularity\\ (topological \\ and analytic)};
\draw (3, -124.6) rectangle (7.35,-126.69999999999999);
\draw(7.449999999999999, -124.6) node[anchor=north west,align=left] {Relations\\ with \\ arrangements of\\ hyperplanes};
\draw (7.449999999999999, -124.6) rectangle (11.799999999999999,-126.69999999999999);
\draw(11.899999999999999, -124.6) node[anchor=north west,align=left] {Global theory\\ of complex\\ singularities;\\ cohomological\\ properties};
\draw (11.899999999999999, -124.6) rectangle (15.999999999999998,-127.19999999999999);
\draw(3, -127.3) node[anchor=north west,align=left] {Deformations\\ of complex\\ singularities;\\ vanishing\\ cycles};
\draw (3, -127.3) rectangle (7.1,-129.9);
\draw(7.199999999999999, -127.3) node[anchor=north west,align=left] {Milnor \\ fibration; \\ relations with\\ knot theory};
\draw (7.199999999999999, -127.3) rectangle (11.299999999999999,-129.4);
\draw(11.399999999999999, -127.3) node[anchor=north west,align=left] {Local \\ complex \\ singularities};
\draw (11.399999999999999, -127.3) rectangle (15.249999999999998,-128.9);
\draw(3, -130.0) node[anchor=north west,align=left] {Complex \\ surface and \\ hypersurface\\ singularities};
\draw (3, -130.0) rectangle (6.85,-132.1);
\draw(6.949999999999999, -130.0) node[anchor=north west,align=left] {Other \\ operations on\\ complex \\ singularities};
\draw (6.949999999999999, -130.0) rectangle (10.799999999999999,-132.1);
\draw(10.9, -130.0) node[anchor=north west,align=left] {Invariants\\ of \\ analytic \\ local rings};
\draw (10.9, -130.0) rectangle (14.25,-132.1);
\draw(16.199999999999996, -114.0) node[anchor=north west,align=left] {\large Generalizations of analytic spaces};
\draw (16.199999999999996, -114.0) rectangle (28.849999999999994,-119.9);
\draw(17.199999999999996, -115.0) node[anchor=north west,align=left] {Differentiable\\ functions \\ on analytic \\ spaces, \\ differentiable spaces};
\draw (17.199999999999996, -115.0) rectangle (23.049999999999997,-117.6);
\draw(23.149999999999995, -115.0) node[anchor=north west,align=left] {Holomorphic\\ maps with \\ infinite-dimensional\\ arguments\\ or values};
\draw (23.149999999999995, -115.0) rectangle (28.749999999999993,-117.6);
\draw(17.199999999999996, -117.7) node[anchor=north west,align=left] {Formal and\\ graded \\ complex spaces};
\draw (17.199999999999996, -117.7) rectangle (21.299999999999997,-119.3);
\draw(21.399999999999995, -117.7) node[anchor=north west,align=left] {Banach \\ analytic \\ manifolds \\ and spaces};
\draw (21.399999999999995, -117.7) rectangle (24.499999999999996,-119.8);
\draw(16.199999999999996, -120.0) node[anchor=north west,align=left] {\large Holomorphic convexity};
\draw (16.199999999999996, -120.0) rectangle (28.599999999999994,-130.1);
\draw(17.199999999999996, -121.0) node[anchor=north west,align=left] {Holomorphic, \\ polynomial and rational\\ approximation, and\\ interpolation in\\ several complex \\ variables; Runge pairs};
\draw (17.199999999999996, -121.0) rectangle (23.549999999999997,-124.1);
\draw(23.649999999999995, -121.0) node[anchor=north west,align=left] {Holomorphically\\ convex\\ complex \\ spaces, reduction\\ theory};
\draw (23.649999999999995, -121.0) rectangle (28.499999999999993,-123.6);
\draw(17.199999999999996, -124.2) node[anchor=north west,align=left] {Polynomial \\ convexity, rational\\ convexity, \\ meromorphic convexity\\ in several\\ complex variables};
\draw (17.199999999999996, -124.2) rectangle (23.049999999999997,-127.3);
\draw(23.149999999999995, -124.2) node[anchor=north west,align=left] {Stein \\ spaces, Stein\\ manifolds};
\draw (23.149999999999995, -124.2) rectangle (26.999999999999996,-125.8);
\draw(23.149999999999995, -125.9) node[anchor=north west,align=left] {The Levi\\ problem};
\draw (23.149999999999995, -125.9) rectangle (25.749999999999996,-127.0);
\draw(17.199999999999996, -127.4) node[anchor=north west,align=left] {Global boundary\\ behavior of \\ holomorphic functions\\ of several\\ complex variables};
\draw (17.199999999999996, -127.4) rectangle (23.049999999999997,-130.0);
\draw(16.199999999999996, -130.2) node[anchor=north west,align=left] {\large History of several\\ complex variables\\ and analytic spaces};
\draw (16.199999999999996, -130.2) rectangle (22.689999999999994,-131.79999999999998);
\draw(28.949999999999996, -114.0) node[anchor=north west,align=left] {\large Complex manifolds};
\draw (28.949999999999996, -114.0) rectangle (41.599999999999994,-129.2);
\draw(29.949999999999996, -115.0) node[anchor=north west,align=left] {Kähler-Einsteinmanifolds};
\draw (29.949999999999996, -115.0) rectangle (36.55,-116.6);
\draw(36.64999999999999, -115.0) node[anchor=north west,align=left] {Calabi-Yau\\ theory \\ (complex-analytic\\ aspects)};
\draw (36.64999999999999, -115.0) rectangle (41.49999999999999,-117.1);
\draw(29.949999999999996, -117.2) node[anchor=north west,align=left] {Special domains\\ (Reinhardt, \\ Hartogs, circular, \\ tube, etc.) in \\ \(\mathbb{C}^n\) \\ and complex manifolds};
\draw (29.949999999999996, -117.2) rectangle (35.8,-120.3);
\draw(35.89999999999999, -117.2) node[anchor=north west,align=left] {Pseudoholomorphic\\ curves};
\draw (35.89999999999999, -117.2) rectangle (40.74999999999999,-118.8);
\draw(35.89999999999999, -118.9) node[anchor=north west,align=left] {Kähler \\ manifolds};
\draw (35.89999999999999, -118.9) rectangle (38.74999999999999,-120.0);
\draw(29.949999999999996, -120.4) node[anchor=north west,align=left] {Complex \\ manifolds as \\ subdomains of \\ Euclidean space};
\draw (29.949999999999996, -120.4) rectangle (34.3,-122.5);
\draw(34.39999999999999, -120.4) node[anchor=north west,align=left] {Uniformization\\ of complex\\ manifolds};
\draw (34.39999999999999, -120.4) rectangle (38.49999999999999,-122.5);
\draw(38.599999999999994, -120.4) node[anchor=north west,align=left] {Negative\\ curvature\\ complex\\ manifolds};
\draw (38.599999999999994, -120.4) rectangle (41.449999999999996,-122.5);
\draw(29.949999999999996, -122.6) node[anchor=north west,align=left] {Classification\\ theorems\\ for complex\\ manifolds};
\draw (29.949999999999996, -122.6) rectangle (34.05,-124.69999999999999);
\draw(34.14999999999999, -122.6) node[anchor=north west,align=left] {Hyperbolic\\ and Kobayashi\\ hyperbolic\\ manifolds};
\draw (34.14999999999999, -122.6) rectangle (37.99999999999999,-124.69999999999999);
\draw(38.099999999999994, -122.6) node[anchor=north west,align=left] {Notions of\\ stability\\ for complex\\ manifolds};
\draw (38.099999999999994, -122.6) rectangle (41.449999999999996,-124.69999999999999);
\draw(29.949999999999996, -124.8) node[anchor=north west,align=left] {Oka principle\\ and Oka\\ manifolds};
\draw (29.949999999999996, -124.8) rectangle (33.8,-126.39999999999999);
\draw(33.89999999999999, -124.8) node[anchor=north west,align=left] {Embedding\\ theorems \\ for complex\\ manifolds};
\draw (33.89999999999999, -124.8) rectangle (37.24999999999999,-126.89999999999999);
\draw(37.349999999999994, -124.8) node[anchor=north west,align=left] {Topological\\ aspects\\ of complex\\ manifolds};
\draw (37.349999999999994, -124.8) rectangle (40.699999999999996,-126.89999999999999);
\draw(29.949999999999996, -127.0) node[anchor=north west,align=left] {Positive\\ curvature\\ complex\\ manifolds};
\draw (29.949999999999996, -127.0) rectangle (32.8,-129.1);
\draw(32.9, -127.0) node[anchor=north west,align=left] {Stein \\ manifolds};
\draw (32.9, -127.0) rectangle (35.75,-128.1);
\draw(35.849999999999994, -127.0) node[anchor=north west,align=left] {Almost \\ complex \\ manifolds};
\draw (35.849999999999994, -127.0) rectangle (38.699999999999996,-128.6);
\draw(41.699999999999996, -114.0) node[anchor=north west,align=left] {\large Local analytic geometry};
\draw (41.699999999999996, -114.0) rectangle (53.099999999999994,-122.6);
\draw(42.699999999999996, -115.0) node[anchor=north west,align=left] {Triangulation and\\ topological \\ properties of \\ semi-analytic \\ andsubanalytic sets, \\ and related questions};
\draw (42.699999999999996, -115.0) rectangle (48.55,-118.1);
\draw(48.64999999999999, -115.0) node[anchor=north west,align=left] {Germs of \\ analytic sets,\\ local \\ parametrization};
\draw (48.64999999999999, -115.0) rectangle (52.99999999999999,-117.1);
\draw(42.699999999999996, -118.2) node[anchor=north west,align=left] {Analytic \\ algebras and\\ generalizations,\\ preparation theorems};
\draw (42.699999999999996, -118.2) rectangle (48.3,-120.8);
\draw(48.39999999999999, -118.2) node[anchor=north west,align=left] {Semi-analytic\\ sets, \\ subanalytic \\ sets, and \\ generalizations};
\draw (48.39999999999999, -118.2) rectangle (52.74999999999999,-120.8);
\draw(42.699999999999996, -120.9) node[anchor=north west,align=left] {Analytic \\ subsets of\\ affine space};
\draw (42.699999999999996, -120.9) rectangle (46.3,-122.5);
\draw(41.699999999999996, -122.7) node[anchor=north west,align=left] {\large Automorphic functions};
\draw (41.699999999999996, -122.7) rectangle (50.849999999999994,-128.6);
\draw(42.699999999999996, -123.7) node[anchor=north west,align=left] {General theory\\ of automorphic\\ functions\\ of several \\ complex variables};
\draw (42.699999999999996, -123.7) rectangle (47.55,-126.3);
\draw(42.699999999999996, -126.4) node[anchor=north west,align=left] {Automorphic\\ forms in \\ several complex\\ variables};
\draw (42.699999999999996, -126.4) rectangle (47.05,-128.5);
\draw(47.14999999999999, -126.4) node[anchor=north west,align=left] {Automorphic\\ functions\\ in symmetric\\ domains};
\draw (47.14999999999999, -126.4) rectangle (50.74999999999999,-128.5);
\draw(53.199999999999996, -114.0) node[anchor=north west,align=left] {\large Analytic spaces};
\draw (53.199999999999996, -114.0) rectangle (63.349999999999994,-132.6);
\draw(54.199999999999996, -115.0) node[anchor=north west,align=left] {The Levi \\ problem in complex\\ spaces;\\ generalizations};
\draw (54.199999999999996, -115.0) rectangle (59.3,-117.1);
\draw(59.39999999999999, -115.0) node[anchor=north west,align=left] {Real-analytic\\ manifolds,\\ real-analytic\\ spaces};
\draw (59.39999999999999, -115.0) rectangle (63.24999999999999,-117.1);
\draw(54.199999999999996, -117.2) node[anchor=north west,align=left] {Applications\\ of analytic \\ spaces to physics\\ and other\\ areas of science};
\draw (54.199999999999996, -117.2) rectangle (59.05,-119.8);
\draw(59.14999999999999, -117.2) node[anchor=north west,align=left] {Real-analytic\\ sets,\\ complex \\ Nash functions};
\draw (59.14999999999999, -117.2) rectangle (63.24999999999999,-119.3);
\draw(54.199999999999996, -119.9) node[anchor=north west,align=left] {Integration\\ on analytic\\ sets and \\ spaces, currents};
\draw (54.199999999999996, -119.9) rectangle (58.8,-122.0);
\draw(58.89999999999999, -119.9) node[anchor=north west,align=left] {Local \\ cohomology\\ of \\ analytic spaces};
\draw (58.89999999999999, -119.9) rectangle (63.24999999999999,-122.0);
\draw(54.199999999999996, -122.1) node[anchor=north west,align=left] {Analytic\\ sheaves \\ and cohomology\\ groups};
\draw (54.199999999999996, -122.1) rectangle (58.3,-124.19999999999999);
\draw(58.39999999999999, -122.1) node[anchor=north west,align=left] {Sheaves of \\ differential \\ operators and \\ their modules,\\ \(D\)-modules};
\draw (58.39999999999999, -122.1) rectangle (62.49999999999999,-124.69999999999999);
\draw(54.199999999999996, -124.8) node[anchor=north west,align=left] {Embedding\\ of \\ real-analytic\\ manifolds};
\draw (54.199999999999996, -124.8) rectangle (58.05,-126.89999999999999);
\draw(58.14999999999999, -124.8) node[anchor=north west,align=left] {Complex\\ supergeometry};
\draw (58.14999999999999, -124.8) rectangle (61.99999999999999,-126.39999999999999);
\draw(54.199999999999996, -127.0) node[anchor=north west,align=left] {Analytic\\ subsets\\ and \\ submanifolds};
\draw (54.199999999999996, -127.0) rectangle (57.8,-129.1);
\draw(57.89999999999999, -127.0) node[anchor=north west,align=left] {Duality \\ theorems \\ for analytic\\ spaces};
\draw (57.89999999999999, -127.0) rectangle (61.49999999999999,-129.1);
\draw(54.199999999999996, -129.2) node[anchor=north west,align=left] {Topology\\ of analytic\\ spaces};
\draw (54.199999999999996, -129.2) rectangle (57.55,-130.79999999999998);
\draw(57.64999999999999, -129.2) node[anchor=north west,align=left] {Embedding\\ of analytic\\ spaces};
\draw (57.64999999999999, -129.2) rectangle (60.99999999999999,-130.79999999999998);
\draw(54.199999999999996, -130.9) node[anchor=north west,align=left] {Normal\\ analytic\\ spaces};
\draw (54.199999999999996, -130.9) rectangle (56.8,-132.5);
\draw(56.9, -130.9) node[anchor=north west,align=left] {Complex\\ spaces};
\draw (56.9, -130.9) rectangle (59.25,-132.0);
\draw(2, -132.7) node[anchor=north west,align=left] {\large Pseudoconvex domains};
\draw (2, -132.7) rectangle (12.149999999999999,-139.79999999999998);
\draw(3, -133.7) node[anchor=north west,align=left] {Geometric and\\ analytic \\ invariants on \\ weakly pseudoconvex\\ boundaries};
\draw (3, -133.7) rectangle (8.35,-136.29999999999998);
\draw(8.45, -133.7) node[anchor=north west,align=left] {Strongly\\ pseudoconvex\\ domains};
\draw (8.45, -133.7) rectangle (12.049999999999999,-135.29999999999998);
\draw(3, -136.39999999999998) node[anchor=north west,align=left] {Finite-typedomains};
\draw (3, -136.39999999999998) rectangle (8.1,-137.99999999999997);
\draw(8.2, -136.39999999999998) node[anchor=north west,align=left] {Domains\\ of \\ holomorphy};
\draw (8.2, -136.39999999999998) rectangle (11.299999999999999,-137.99999999999997);
\draw(3, -138.1) node[anchor=north west,align=left] {Exhaustion\\ functions};
\draw (3, -138.1) rectangle (6.1,-139.7);
\draw(6.199999999999999, -138.1) node[anchor=north west,align=left] {Peak \\ functions};
\draw (6.199999999999999, -138.1) rectangle (9.049999999999999,-139.2);
\draw(9.149999999999999, -138.1) node[anchor=north west,align=left] {Worm \\ domains};
\draw (9.149999999999999, -138.1) rectangle (11.499999999999998,-139.2);
\draw(12.249999999999998, -132.7) node[anchor=north west,align=left] {\large Compact analytic spaces};
\draw (12.249999999999998, -132.7) rectangle (22.15,-143.0);
\draw(13.249999999999998, -133.7) node[anchor=north west,align=left] {Transcendental\\ methods of \\ algebraic geometry\\ (complex-analytic\\ aspects)};
\draw (13.249999999999998, -133.7) rectangle (18.349999999999998,-136.29999999999998);
\draw(18.449999999999996, -133.7) node[anchor=north west,align=left] {Applications\\ of compact\\ analytic\\ spaces to \\ the sciences};
\draw (18.449999999999996, -133.7) rectangle (22.049999999999997,-136.29999999999998);
\draw(13.249999999999998, -136.39999999999998) node[anchor=north west,align=left] {Compact \\ Kähler manifolds:\\ generalizations, \\ classification};
\draw (13.249999999999998, -136.39999999999998) rectangle (18.099999999999998,-138.99999999999997);
\draw(18.199999999999996, -136.39999999999998) node[anchor=north west,align=left] {Compact \\ complex \\ \(3\)-folds};
\draw (18.199999999999996, -136.39999999999998) rectangle (21.549999999999997,-137.99999999999997);
\draw(13.249999999999998, -139.1) node[anchor=north west,align=left] {Compactification\\ of \\ analytic spaces};
\draw (13.249999999999998, -139.1) rectangle (17.849999999999998,-141.2);
\draw(17.949999999999996, -139.1) node[anchor=north west,align=left] {Compact \\ complex \\ \(n\)-folds};
\draw (17.949999999999996, -139.1) rectangle (21.299999999999997,-140.7);
\draw(13.249999999999998, -141.29999999999998) node[anchor=north west,align=left] {Algebraic\\ dependence\\ theorems};
\draw (13.249999999999998, -141.29999999999998) rectangle (16.349999999999998,-142.89999999999998);
\draw(16.449999999999996, -141.29999999999998) node[anchor=north west,align=left] {Compact\\ complex\\ surfaces};
\draw (16.449999999999996, -141.29999999999998) rectangle (19.049999999999997,-142.89999999999998);
\draw(22.25, -132.7) node[anchor=north west,align=left] {\large Pluripotential theory};
\draw (22.25, -132.7) rectangle (31.65,-145.2);
\draw(23.25, -133.7) node[anchor=north west,align=left] {Plurisubharmonic\\ functions\\ and \\ generalizations};
\draw (23.25, -133.7) rectangle (27.85,-135.79999999999998);
\draw(27.95, -133.7) node[anchor=north west,align=left] {Currents};
\draw (27.95, -133.7) rectangle (30.55,-134.79999999999998);
\draw(23.25, -135.89999999999998) node[anchor=north west,align=left] {Plurisubharmonic\\ exhaustion\\ functions};
\draw (23.25, -135.89999999999998) rectangle (27.85,-137.99999999999997);
\draw(27.95, -135.89999999999998) node[anchor=north west,align=left] {Lelong\\ numbers};
\draw (27.95, -135.89999999999998) rectangle (30.3,-136.99999999999997);
\draw(23.25, -138.1) node[anchor=north west,align=left] {Plurisubharmonic\\ extremal\\ functions, \\ pluricomplex \\ Green functions};
\draw (23.25, -138.1) rectangle (27.85,-140.7);
\draw(23.25, -140.79999999999998) node[anchor=north west,align=left] {Capacity\\ theory\\ and \\ generalizations};
\draw (23.25, -140.79999999999998) rectangle (27.6,-142.89999999999998);
\draw(23.25, -143.0) node[anchor=north west,align=left] {General \\ pluripotential\\ theory};
\draw (23.25, -143.0) rectangle (27.35,-144.6);
\draw(27.45, -143.0) node[anchor=north west,align=left] {Removable\\ sets in\\ pluripotential\\ theory};
\draw (27.45, -143.0) rectangle (31.549999999999997,-145.1);
\draw(31.75, -132.7) node[anchor=north west,align=left] {\large CR manifolds};
\draw (31.75, -132.7) rectangle (40.9,-142.5);
\draw(32.75, -133.7) node[anchor=north west,align=left] {Extension of\\ functions and\\ other analytic\\ objects \\ from CR manifolds};
\draw (32.75, -133.7) rectangle (37.6,-136.29999999999998);
\draw(37.7, -133.7) node[anchor=north west,align=left] {Embeddings\\ of CR\\ manifolds};
\draw (37.7, -133.7) rectangle (40.800000000000004,-135.29999999999998);
\draw(32.75, -136.39999999999998) node[anchor=north west,align=left] {CR structures,\\ CR operators,\\ and \\ generalizations};
\draw (32.75, -136.39999999999998) rectangle (37.1,-138.49999999999997);
\draw(37.2, -136.39999999999998) node[anchor=north west,align=left] {CR manifolds\\ as \\ boundaries\\ of domains};
\draw (37.2, -136.39999999999998) rectangle (40.800000000000004,-138.49999999999997);
\draw(32.75, -138.6) node[anchor=north west,align=left] {Finite-type\\ conditions\\ on \\ CR manifolds};
\draw (32.75, -138.6) rectangle (36.35,-140.7);
\draw(36.45, -138.6) node[anchor=north west,align=left] {Real \\ submanifolds\\ in complex\\ manifolds};
\draw (36.45, -138.6) rectangle (40.050000000000004,-140.7);
\draw(32.75, -140.79999999999998) node[anchor=north west,align=left] {CR \\ functions};
\draw (32.75, -140.79999999999998) rectangle (35.6,-141.89999999999998);
\draw(35.7, -140.79999999999998) node[anchor=north west,align=left] {Analysis\\ on CR \\ manifolds};
\draw (35.7, -140.79999999999998) rectangle (38.550000000000004,-142.39999999999998);
\draw(41.0, -132.7) node[anchor=north west,align=left] {\large Analytic continuation};
\draw (41.0, -132.7) rectangle (49.9,-139.79999999999998);
\draw(42.0, -133.7) node[anchor=north west,align=left] {Removable \\ singularities in\\ several complex\\ variables};
\draw (42.0, -133.7) rectangle (46.6,-135.79999999999998);
\draw(46.7, -133.7) node[anchor=north west,align=left] {Domains\\ of \\ holomorphy};
\draw (46.7, -133.7) rectangle (49.800000000000004,-135.29999999999998);
\draw(42.0, -135.89999999999998) node[anchor=north west,align=left] {Continuation\\ of analytic\\ objects in\\ several complex\\ variables};
\draw (42.0, -135.89999999999998) rectangle (46.35,-138.49999999999997);
\draw(46.45, -135.89999999999998) node[anchor=north west,align=left] {Envelopes\\ of \\ holomorphy};
\draw (46.45, -135.89999999999998) rectangle (49.550000000000004,-137.49999999999997);
\draw(42.0, -138.6) node[anchor=north west,align=left] {Riemann\\ domains};
\draw (42.0, -138.6) rectangle (44.35,-139.7);
\draw(50.0, -132.7) node[anchor=north west,align=left] {\large Holomorphic fiber spaces};
\draw (50.0, -132.7) rectangle (58.9,-144.5);
\draw(51.0, -133.7) node[anchor=north west,align=left] {Twistor theory,\\ double \\ fibrations \\ (complex-analytic\\ aspects)};
\draw (51.0, -133.7) rectangle (55.85,-136.29999999999998);
\draw(55.95, -133.7) node[anchor=north west,align=left] {Bundle \\ convexity};
\draw (55.95, -133.7) rectangle (58.800000000000004,-134.79999999999998);
\draw(51.0, -136.39999999999998) node[anchor=north west,align=left] {Holomorphic\\ bundles\\ and \\ generalizations};
\draw (51.0, -136.39999999999998) rectangle (55.35,-138.49999999999997);
\draw(55.45, -136.39999999999998) node[anchor=north west,align=left] {Vanishing\\ theorems};
\draw (55.45, -136.39999999999998) rectangle (58.300000000000004,-137.99999999999997);
\draw(51.0, -138.6) node[anchor=north west,align=left] {Sheaves and \\ cohomology of\\ sections of \\ holomorphic \\ vector bundles,\\ general results};
\draw (51.0, -138.6) rectangle (55.35,-141.7);
\draw(51.0, -141.79999999999998) node[anchor=north west,align=left] {Applications\\ of holomorphic\\ fiber\\ spaces to\\ the sciences};
\draw (51.0, -141.79999999999998) rectangle (55.1,-144.39999999999998);
\draw(64.5, -81.8) node[anchor=north west,align=left] {\LARGE Associative rings and algebras};
\draw (64.5, -81.8) rectangle (122.9,-128.9);
\draw(65.5, -82.8) node[anchor=north west,align=left] {\large Chain conditions, growth conditions, and other forms of finiteness for associative rings and algebras};
\draw (65.5, -82.8) rectangle (98.95,-89.7);
\draw(66.5, -83.8) node[anchor=north west,align=left] {Chain conditions\\ on \\ annihilators and \\ summands: \\ Goldie-type conditions};
\draw (66.5, -83.8) rectangle (72.6,-86.39999999999999);
\draw(72.7, -83.8) node[anchor=north west,align=left] {Noetherian \\ rings and \\ modules (associative\\ rings\\ and algebras)};
\draw (72.7, -83.8) rectangle (78.3,-86.39999999999999);
\draw(78.4, -83.8) node[anchor=north west,align=left] {Chain conditions\\ on other classes\\ of submodules,\\ ideals, subrings,\\ etc.; coherence\\ (associative\\ rings and algebras)};
\draw (78.4, -83.8) rectangle (83.75,-87.39999999999999);
\draw(83.85, -83.8) node[anchor=north west,align=left] {Finite rings\\ and \\ finite-dimensional\\ associative\\ algebras};
\draw (83.85, -83.8) rectangle (88.94999999999999,-86.39999999999999);
\draw(89.05, -83.8) node[anchor=north west,align=left] {Artinian \\ rings and \\ modules \\ (associative rings\\ and algebras)};
\draw (89.05, -83.8) rectangle (94.14999999999999,-86.39999999999999);
\draw(94.25, -83.8) node[anchor=north west,align=left] {Localization\\ and \\ associative \\ Noetherian rings};
\draw (94.25, -83.8) rectangle (98.85,-85.89999999999999);
\draw(66.5, -87.5) node[anchor=north west,align=left] {Growth \\ rate, \\ Gelfand-Kirillov\\ dimension};
\draw (66.5, -87.5) rectangle (71.1,-89.6);
\draw(99.05000000000001, -82.8) node[anchor=north west,align=left] {\large Associative rings and algebras with additional structure};
\draw (99.05000000000001, -82.8) rectangle (119.80000000000001,-91.89999999999999);
\draw(100.05000000000001, -83.8) node[anchor=north west,align=left] {Actions of \\ groups and \\ semigroups; invariant\\ theory \\ (associative \\ rings and algebras)};
\draw (100.05000000000001, -83.8) rectangle (105.9,-86.89999999999999);
\draw(106.00000000000001, -83.8) node[anchor=north west,align=left] {Valuations, \\ completions, formal \\ power series and \\ related constructions\\ (associative \\ rings and algebras)};
\draw (106.00000000000001, -83.8) rectangle (111.85000000000001,-86.89999999999999);
\draw(111.95000000000002, -83.8) node[anchor=north west,align=left] {Filtered \\ associative \\ rings; filtrational\\ and \\ graded techniques};
\draw (111.95000000000002, -83.8) rectangle (117.30000000000001,-86.39999999999999);
\draw(100.05000000000001, -87.0) node[anchor=north west,align=left] {Rings with \\ involution; Lie,\\ Jordan and other\\ nonassociative\\ structures};
\draw (100.05000000000001, -87.0) rectangle (104.65,-89.6);
\draw(104.75000000000001, -87.0) node[anchor=north west,align=left] {Automorphisms\\ and \\ endomorphisms};
\draw (104.75000000000001, -87.0) rectangle (108.60000000000001,-88.6);
\draw(108.70000000000002, -87.0) node[anchor=north west,align=left] {Graded rings\\ and modules\\ (associative\\ rings\\ and algebras)};
\draw (108.70000000000002, -87.0) rectangle (112.55000000000001,-89.6);
\draw(112.65, -87.0) node[anchor=north west,align=left] {Derivations,\\ actions\\ of Lie\\ algebras};
\draw (112.65, -87.0) rectangle (116.25,-89.1);
\draw(116.35000000000001, -87.0) node[anchor=north west,align=left] {“Super” \\ (or “skew”)\\ structure};
\draw (116.35000000000001, -87.0) rectangle (119.7,-88.6);
\draw(100.05000000000001, -89.7) node[anchor=north west,align=left] {Topological\\ and ordered\\ rings\\ and modules};
\draw (100.05000000000001, -89.7) rectangle (103.4,-91.8);
\draw(65.5, -89.8) node[anchor=north west,align=left] {\large History of associative\\ rings and algebras};
\draw (65.5, -89.8) rectangle (72.92,-90.89999999999999);
\draw(65.5, -92.0) node[anchor=north west,align=left] {\large Associative rings and algebras arising under various constructions};
\draw (65.5, -92.0) rectangle (90.30000000000001,-102.8);
\draw(66.5, -93.0) node[anchor=north west,align=left] {Rings arising\\ from \\ noncommutative algebraic\\ geometry};
\draw (66.5, -93.0) rectangle (73.1,-95.1);
\draw(73.2, -93.0) node[anchor=north west,align=left] {Associative rings\\ determined by \\ universal properties \\ (free algebras, \\ coproducts, adjunction\\ of inverses, etc.)};
\draw (73.2, -93.0) rectangle (79.3,-96.1);
\draw(79.4, -93.0) node[anchor=north west,align=left] {Associative \\ rings of \\ functions, subdirect\\ products,\\ sheaves of rings};
\draw (79.4, -93.0) rectangle (85.0,-95.6);
\draw(85.1, -93.0) node[anchor=north west,align=left] {Finite generation,\\ finite \\ presentability,\\ normal forms \\ (diamond lemma,\\ term-rewriting)};
\draw (85.1, -93.0) rectangle (90.19999999999999,-96.1);
\draw(66.5, -96.2) node[anchor=north west,align=left] {Rings of \\ differential \\ operators \\ (associative \\ algebraic aspects)};
\draw (66.5, -96.2) rectangle (71.6,-98.8);
\draw(71.7, -96.2) node[anchor=north west,align=left] {Torsion theories;\\ radicals on\\ module categories\\ (associative\\ algebraic aspects)};
\draw (71.7, -96.2) rectangle (76.8,-98.8);
\draw(76.9, -96.2) node[anchor=north west,align=left] {Twisted and\\ skew group\\ rings, \\ crossed products};
\draw (76.9, -96.2) rectangle (81.5,-98.3);
\draw(81.6, -96.2) node[anchor=north west,align=left] {Ordinary \\ and skew \\ polynomial \\ rings and \\ semigroup rings};
\draw (81.6, -96.2) rectangle (85.94999999999999,-98.8);
\draw(86.05, -96.2) node[anchor=north west,align=left] {Extensions\\ of associative\\ rings\\ by ideals};
\draw (86.05, -96.2) rectangle (90.14999999999999,-98.3);
\draw(66.5, -98.9) node[anchor=north west,align=left] {Associative\\ rings of \\ fractions and \\ localizations};
\draw (66.5, -98.9) rectangle (70.6,-101.0);
\draw(70.7, -98.9) node[anchor=north west,align=left] {Centralizing\\ and \\ normalizing\\ extensions};
\draw (70.7, -98.9) rectangle (74.3,-101.0);
\draw(74.4, -98.9) node[anchor=north west,align=left] {Universal \\ enveloping \\ algebras of\\ Lie algebras};
\draw (74.4, -98.9) rectangle (78.0,-101.0);
\draw(78.1, -98.9) node[anchor=north west,align=left] {Smash \\ products of\\ general \\ Hopf actions};
\draw (78.1, -98.9) rectangle (81.69999999999999,-101.0);
\draw(81.8, -98.9) node[anchor=north west,align=left] {Endomorphism\\ rings;\\ matrix rings};
\draw (81.8, -98.9) rectangle (85.39999999999999,-100.5);
\draw(85.5, -98.9) node[anchor=north west,align=left] {Deformations\\ of\\ associative\\ rings};
\draw (85.5, -98.9) rectangle (89.1,-101.0);
\draw(66.5, -101.1) node[anchor=north west,align=left] {Quadratic\\ and Koszul\\ algebras};
\draw (66.5, -101.1) rectangle (69.6,-102.69999999999999);
\draw(69.7, -101.1) node[anchor=north west,align=left] {Leavitt\\ path \\ algebras};
\draw (69.7, -101.1) rectangle (72.3,-102.69999999999999);
\draw(72.4, -101.1) node[anchor=north west,align=left] {Group\\ rings};
\draw (72.4, -101.1) rectangle (74.25,-102.19999999999999);
\draw(90.4, -92.0) node[anchor=north west,align=left] {\large Radicals and radical properties of associative rings};
\draw (90.4, -92.0) rectangle (109.7,-95.7);
\draw(91.4, -93.0) node[anchor=north west,align=left] {Jacobson\\ radical,\\ quasimultiplication};
\draw (91.4, -93.0) rectangle (96.75,-95.1);
\draw(96.85000000000001, -93.0) node[anchor=north west,align=left] {Nil and \\ nilpotent \\ radicals, sets,\\ ideals, \\ associative rings};
\draw (96.85000000000001, -93.0) rectangle (101.7,-95.6);
\draw(101.80000000000001, -93.0) node[anchor=north west,align=left] {General \\ radicals \\ and associative\\ rings};
\draw (101.80000000000001, -93.0) rectangle (106.15,-95.1);
\draw(106.25, -93.0) node[anchor=north west,align=left] {Prime and\\ semiprime\\ associative\\ rings};
\draw (106.25, -93.0) rectangle (109.6,-95.1);
\draw(90.4, -95.8) node[anchor=north west,align=left] {\large Representation theory of associative rings and algebras};
\draw (90.4, -95.8) rectangle (108.75,-101.7);
\draw(91.4, -96.8) node[anchor=north west,align=left] {Auslander-Reiten\\ sequences (almost\\ split sequences)\\ and \\ Auslander-Reiten quivers};
\draw (91.4, -96.8) rectangle (98.0,-99.39999999999999);
\draw(98.10000000000001, -96.8) node[anchor=north west,align=left] {Representation\\ type (finite,\\ tame, wild, \\ etc.) of associative\\ algebras};
\draw (98.10000000000001, -96.8) rectangle (103.7,-99.39999999999999);
\draw(103.80000000000001, -96.8) node[anchor=north west,align=left] {Representations\\ of orders,\\ lattices, \\ algebras over \\ commutative rings};
\draw (103.80000000000001, -96.8) rectangle (108.65,-99.39999999999999);
\draw(91.4, -99.5) node[anchor=north west,align=left] {Representations\\ of \\ associative \\ Artinian rings};
\draw (91.4, -99.5) rectangle (95.75,-101.6);
\draw(95.85000000000001, -99.5) node[anchor=north west,align=left] {Representations\\ of quivers\\ and partially\\ ordered sets};
\draw (95.85000000000001, -99.5) rectangle (100.2,-101.6);
\draw(100.30000000000001, -99.5) node[anchor=north west,align=left] {Cohen-Macaulay\\ modules\\ in associative\\ algebras};
\draw (100.30000000000001, -99.5) rectangle (104.4,-101.6);
\draw(109.80000000000001, -92.0) node[anchor=north west,align=left] {\large Associative algebras and orders};
\draw (109.80000000000001, -92.0) rectangle (121.45000000000002,-97.4);
\draw(110.80000000000001, -93.0) node[anchor=north west,align=left] {Separable \\ algebras (e.g., \\ quaternion algebras,\\ Azumaya \\ algebras, etc.)};
\draw (110.80000000000001, -93.0) rectangle (116.4,-95.6);
\draw(116.50000000000001, -93.0) node[anchor=north west,align=left] {Commutativeorders};
\draw (116.50000000000001, -93.0) rectangle (121.35000000000001,-94.6);
\draw(110.80000000000001, -95.7) node[anchor=north west,align=left] {Orders in\\ separable\\ algebras};
\draw (110.80000000000001, -95.7) rectangle (113.65,-97.3);
\draw(113.75000000000001, -95.7) node[anchor=north west,align=left] {Lattices\\ over\\ orders};
\draw (113.75000000000001, -95.7) rectangle (116.35000000000001,-97.3);
\draw(65.5, -102.9) node[anchor=north west,align=left] {\large Modules, bimodules and ideals in associative algebras};
\draw (65.5, -102.9) rectangle (84.6,-113.5);
\draw(66.5, -103.9) node[anchor=north west,align=left] {Structure and \\ classification for \\ modules, bimodules and\\ ideals (except \\ as in 16Gxx), direct\\ sum decomposition\\ and cancellation\\ in associative algebras)};
\draw (66.5, -103.9) rectangle (73.1,-108.0);
\draw(73.2, -103.9) node[anchor=north west,align=left] {Infinite-dimensional\\ simple \\ rings (except\\ as in 16Kxx)};
\draw (73.2, -103.9) rectangle (78.8,-106.5);
\draw(78.9, -103.9) node[anchor=north west,align=left] {Free, projective,\\ and flat\\ modules and\\ ideals in \\ associative algebras};
\draw (78.9, -103.9) rectangle (84.5,-106.5);
\draw(66.5, -108.10000000000001) node[anchor=north west,align=left] {Simple and \\ semisimple \\ modules, primitive\\ rings and \\ ideals in \\ associative algebras};
\draw (66.5, -108.10000000000001) rectangle (72.1,-111.2);
\draw(72.2, -108.10000000000001) node[anchor=north west,align=left] {Other classes\\ of modules\\ and ideals\\ in \\ associative algebras};
\draw (72.2, -108.10000000000001) rectangle (77.8,-110.7);
\draw(77.9, -108.10000000000001) node[anchor=north west,align=left] {Injective \\ modules, \\ self-injective \\ associative rings};
\draw (77.9, -108.10000000000001) rectangle (82.75,-110.2);
\draw(66.5, -111.30000000000001) node[anchor=north west,align=left] {General \\ module theory\\ in associative\\ algebras};
\draw (66.5, -111.30000000000001) rectangle (70.6,-113.4);
\draw(70.7, -111.30000000000001) node[anchor=north west,align=left] {Module \\ categories in\\ associative\\ algebras};
\draw (70.7, -111.30000000000001) rectangle (74.55,-113.4);
\draw(74.65, -111.30000000000001) node[anchor=north west,align=left] {Bimodules\\ in \\ associative\\ algebras};
\draw (74.65, -111.30000000000001) rectangle (78.0,-113.4);
\draw(78.1, -111.30000000000001) node[anchor=north west,align=left] {Ideals in\\ associative\\ algebras};
\draw (78.1, -111.30000000000001) rectangle (81.44999999999999,-112.9);
\draw(84.7, -102.9) node[anchor=north west,align=left] {\large Homological methods in associative algebras};
\draw (84.7, -102.9) rectangle (101.05000000000001,-115.7);
\draw(85.7, -103.9) node[anchor=north west,align=left] {Homological conditions\\ on associative\\ rings (generalizations\\ of regular,\\ Gorenstein, \\ Cohen-Macaulay rings, etc.)};
\draw (85.7, -103.9) rectangle (93.05,-107.0);
\draw(93.15, -103.9) node[anchor=north west,align=left] {Homological \\ functors on modules\\ (Tor, Ext,\\ etc.) in \\ associative algebras};
\draw (93.15, -103.9) rectangle (98.75,-106.5);
\draw(85.7, -107.10000000000001) node[anchor=north west,align=left] {(Co)homology \\ of rings and \\ associative \\ algebras (e.g., \\ Hochschild, cyclic,\\ dihedral, etc.)};
\draw (85.7, -107.10000000000001) rectangle (91.05,-110.2);
\draw(91.15, -107.10000000000001) node[anchor=north west,align=left] {Grothendieck\\ groups,\\ \(K\)-theory, etc.};
\draw (91.15, -107.10000000000001) rectangle (96.25,-109.2);
\draw(96.35, -107.10000000000001) node[anchor=north west,align=left] {Derived \\ categories \\ and associative\\ algebras};
\draw (96.35, -107.10000000000001) rectangle (100.69999999999999,-109.2);
\draw(85.7, -110.30000000000001) node[anchor=north west,align=left] {Differential \\ graded algebras \\ and applications\\ (associative \\ algebraic aspects)};
\draw (85.7, -110.30000000000001) rectangle (90.8,-112.9);
\draw(90.9, -110.30000000000001) node[anchor=north west,align=left] {von Neumann \\ regular rings and\\ generalizations\\ (associative \\ algebraic aspects)};
\draw (90.9, -110.30000000000001) rectangle (96.0,-112.9);
\draw(96.1, -110.30000000000001) node[anchor=north west,align=left] {Semihereditary\\ and hereditary\\ rings, free ideal\\ rings, Sylvester\\ rings, etc.};
\draw (96.1, -110.30000000000001) rectangle (100.94999999999999,-112.9);
\draw(85.7, -113.0) node[anchor=north west,align=left] {Syzygies, \\ resolutions,\\ complexes\\ in associative\\ algebras};
\draw (85.7, -113.0) rectangle (89.8,-115.6);
\draw(89.9, -113.0) node[anchor=north west,align=left] {Homological\\ dimension\\ in associative\\ algebras};
\draw (89.9, -113.0) rectangle (94.0,-115.1);
\draw(101.15, -102.9) node[anchor=north west,align=left] {\large Hopf algebras, quantum groups and related topics};
\draw (101.15, -102.9) rectangle (117.45,-108.30000000000001);
\draw(102.15, -103.9) node[anchor=north west,align=left] {Ring-theoretic\\ aspects\\ of quantum\\ groups};
\draw (102.15, -103.9) rectangle (106.25,-106.0);
\draw(106.35000000000001, -103.9) node[anchor=north west,align=left] {Connections\\ of Hopf \\ algebras with\\ combinatorics};
\draw (106.35000000000001, -103.9) rectangle (110.2,-106.0);
\draw(110.30000000000001, -103.9) node[anchor=north west,align=left] {Hopf \\ algebras and\\ their \\ applications};
\draw (110.30000000000001, -103.9) rectangle (113.9,-106.0);
\draw(114.0, -103.9) node[anchor=north west,align=left] {Yang-Baxter\\ equations};
\draw (114.0, -103.9) rectangle (117.35,-105.5);
\draw(102.15, -106.10000000000001) node[anchor=north west,align=left] {Bialgebras};
\draw (102.15, -106.10000000000001) rectangle (105.25,-107.2);
\draw(105.35000000000001, -106.10000000000001) node[anchor=north west,align=left] {Coalgebras\\ and \\ comodules;\\ corings};
\draw (105.35000000000001, -106.10000000000001) rectangle (108.45,-108.2);
\draw(101.15, -108.4) node[anchor=north west,align=left] {\large Division rings and semisimple Artin rings};
\draw (101.15, -108.4) rectangle (116.25,-111.60000000000001);
\draw(102.15, -109.4) node[anchor=north west,align=left] {Infinite-dimensional\\ and general\\ division rings};
\draw (102.15, -109.4) rectangle (107.75,-111.5);
\draw(107.85000000000001, -109.4) node[anchor=north west,align=left] {Finite-dimensional\\ division\\ rings};
\draw (107.85000000000001, -109.4) rectangle (112.95,-111.5);
\draw(113.05000000000001, -109.4) node[anchor=north west,align=left] {Brauer \\ groups \\ (algebraic\\ aspects)};
\draw (113.05000000000001, -109.4) rectangle (116.15,-111.5);
\draw(101.15, -111.7) node[anchor=north west,align=left] {\large Computational aspects of associative rings};
\draw (101.15, -111.7) rectangle (114.77000000000001,-115.4);
\draw(102.15, -112.7) node[anchor=north west,align=left] {Computational\\ aspects\\ of associative\\ rings \\ (general theory)};
\draw (102.15, -112.7) rectangle (106.75,-115.3);
\draw(106.85000000000001, -112.7) node[anchor=north west,align=left] {Gröbner-Shirshov\\ bases};
\draw (106.85000000000001, -112.7) rectangle (111.45,-114.3);
\draw(117.55000000000001, -102.9) node[anchor=north west,align=left] {\large Generalizations};
\draw (117.55000000000001, -102.9) rectangle (122.80000000000001,-109.2);
\draw(118.55000000000001, -103.9) node[anchor=north west,align=left] {Hyperrings};
\draw (118.55000000000001, -103.9) rectangle (121.65,-105.0);
\draw(118.55000000000001, -105.10000000000001) node[anchor=north west,align=left] {Near-rings};
\draw (118.55000000000001, -105.10000000000001) rectangle (121.65,-106.2);
\draw(118.55000000000001, -106.30000000000001) node[anchor=north west,align=left] {\(\Gamma\)\\ and fuzzy\\ structures};
\draw (118.55000000000001, -106.30000000000001) rectangle (121.65,-107.9);
\draw(118.55000000000001, -108.0) node[anchor=north west,align=left] {Semirings};
\draw (118.55000000000001, -108.0) rectangle (121.4,-109.1);
\draw(65.5, -115.8) node[anchor=north west,align=left] {\large Rings with polynomial identity};
\draw (65.5, -115.8) rectangle (76.9,-124.89999999999999);
\draw(66.5, -116.8) node[anchor=north west,align=left] {Other kinds of\\ identities \\ (generalized \\ polynomial, rational,\\ involution)};
\draw (66.5, -116.8) rectangle (72.35,-119.39999999999999);
\draw(72.45, -116.8) node[anchor=north west,align=left] {Functional\\ identities\\ (associative\\ rings\\ and algebras)};
\draw (72.45, -116.8) rectangle (76.3,-119.39999999999999);
\draw(66.5, -119.5) node[anchor=north west,align=left] {\(T\)-ideals,\\ identities,\\ varieties of\\ associative \\ rings and algebras};
\draw (66.5, -119.5) rectangle (71.6,-122.1);
\draw(71.7, -119.5) node[anchor=north west,align=left] {Trace rings \\ and invariant\\ theory \\ (associative rings\\ and algebras)};
\draw (71.7, -119.5) rectangle (76.8,-122.1);
\draw(66.5, -122.2) node[anchor=north west,align=left] {Semiprime p.i.\\ rings, rings\\ embeddable in\\ matrices over\\ commutative rings};
\draw (66.5, -122.2) rectangle (71.35,-124.8);
\draw(71.45, -122.2) node[anchor=north west,align=left] {Identities \\ other than \\ those of matrices\\ over \\ commutative rings};
\draw (71.45, -122.2) rectangle (76.3,-124.8);
\draw(77.0, -115.8) node[anchor=north west,align=left] {\large Conditions on elements};
\draw (77.0, -115.8) rectangle (87.9,-128.8);
\draw(78.0, -116.8) node[anchor=north west,align=left] {Integral \\ domains (associative\\ rings\\ and algebras)};
\draw (78.0, -116.8) rectangle (83.6,-118.89999999999999);
\draw(83.7, -116.8) node[anchor=north west,align=left] {Ore rings, \\ multiplicative\\ sets, Ore\\ localization};
\draw (83.7, -116.8) rectangle (87.8,-118.89999999999999);
\draw(78.0, -119.0) node[anchor=north west,align=left] {Idempotent \\ elements \\ (associative rings\\ and algebras)};
\draw (78.0, -119.0) rectangle (83.1,-121.1);
\draw(83.2, -119.0) node[anchor=north west,align=left] {Divisibility,\\ noncommutative\\ UFDs};
\draw (83.2, -119.0) rectangle (87.3,-121.1);
\draw(78.0, -121.2) node[anchor=north west,align=left] {Center, normalizer\\ (invariant\\ elements) \\ (associative rings\\ and algebras)};
\draw (78.0, -121.2) rectangle (83.1,-123.8);
\draw(83.2, -121.2) node[anchor=north west,align=left] {Units, groups\\ of units\\ (associative\\ rings and\\ algebras)};
\draw (83.2, -121.2) rectangle (87.05,-123.8);
\draw(78.0, -123.9) node[anchor=north west,align=left] {Generalizations\\ of commutativity\\ (associative rings\\ and algebras)};
\draw (78.0, -123.9) rectangle (83.1,-126.5);
\draw(78.0, -126.6) node[anchor=north west,align=left] {Generalized \\ inverses \\ (associative rings\\ and algebras)};
\draw (78.0, -126.6) rectangle (83.1,-128.7);
\draw(65.5, -125.0) node[anchor=north west,align=left] {\large Local rings and generalizations};
\draw (65.5, -125.0) rectangle (75.71,-128.7);
\draw(66.5, -126.0) node[anchor=north west,align=left] {Quasi-Frobenius\\ rings};
\draw (66.5, -126.0) rectangle (70.85,-127.6);
\draw(70.95, -126.0) node[anchor=north west,align=left] {Noncommutative\\ local \\ and semilocal\\ rings, \\ perfect rings};
\draw (70.95, -126.0) rectangle (75.05,-128.6);
\draw(88.0, -115.8) node[anchor=north west,align=left] {\large General and miscellaneous};
\draw (88.0, -115.8) rectangle (96.35,-122.2);
\draw(89.0, -116.8) node[anchor=north west,align=left] {Category-theoretic\\ methods\\ and results\\ in associative\\ algebras \\ (except as in 16D90)};
\draw (89.0, -116.8) rectangle (94.6,-119.89999999999999);
\draw(89.0, -120.0) node[anchor=north west,align=left] {Applications\\ of logic\\ in associative\\ algebras};
\draw (89.0, -120.0) rectangle (93.1,-122.1);
\draw(137.17, -1) node[anchor=north west,align=left] {\LARGE Functions of a complex variable};
\draw (137.17, -1) rectangle (186.82,-50.6);
\draw(138.17, -2) node[anchor=north west,align=left] {\large Entire and meromorphic functions of one complex variable, and related topics};
\draw (138.17, -2) rectangle (166.17,-9.4);
\draw(139.17, -3) node[anchor=north west,align=left] {Functional equations\\ in the complex\\ plane, iteration\\ and composition\\ of analytic\\ functions of one\\ complex variable};
\draw (139.17, -3) rectangle (144.76999999999998,-6.6);
\draw(144.86999999999998, -3) node[anchor=north west,align=left] {Representations\\ of entire\\ functions of\\ one complex \\ variable by \\ series and integrals};
\draw (144.86999999999998, -3) rectangle (150.46999999999997,-6.1);
\draw(150.57, -3) node[anchor=north west,align=left] {Special classes\\ of entire functions\\ of one complex\\ variable and\\ growth estimates};
\draw (150.57, -3) rectangle (155.92,-5.6);
\draw(156.01999999999998, -3) node[anchor=north west,align=left] {Value distribution\\ of meromorphic\\ functions\\ of one complex\\ variable, \\ Nevanlinna theory};
\draw (156.01999999999998, -3) rectangle (161.11999999999998,-6.1);
\draw(161.22, -3) node[anchor=north west,align=left] {Cluster \\ sets, prime\\ ends, \\ boundary behavior};
\draw (161.22, -3) rectangle (166.07,-5.1);
\draw(139.17, -6.7) node[anchor=north west,align=left] {Entire \\ functions of one\\ complex \\ variable, \\ general theory};
\draw (139.17, -6.7) rectangle (143.76999999999998,-9.3);
\draw(143.86999999999998, -6.7) node[anchor=north west,align=left] {Normal \\ functions of one\\ complex \\ variable, \\ normal families};
\draw (143.86999999999998, -6.7) rectangle (148.46999999999997,-9.3);
\draw(148.57, -6.7) node[anchor=north west,align=left] {Quasi-analytic\\ and other \\ classes of \\ functions of one\\ complex variable};
\draw (148.57, -6.7) rectangle (153.17,-9.3);
\draw(153.26999999999998, -6.7) node[anchor=north west,align=left] {Meromorphic\\ functions of\\ one complex\\ variable, \\ general theory};
\draw (153.26999999999998, -6.7) rectangle (157.36999999999998,-9.3);
\draw(166.26999999999998, -2) node[anchor=north west,align=left] {\large Series expansions of functions of one complex variable};
\draw (166.26999999999998, -2) rectangle (186.07,-8.4);
\draw(167.26999999999998, -3) node[anchor=north west,align=left] {Boundary behavior\\ of power \\ series in one complex\\ variable; \\ over-convergence};
\draw (167.26999999999998, -3) rectangle (173.11999999999998,-5.6);
\draw(173.21999999999997, -3) node[anchor=north west,align=left] {Completeness \\ problems, closure\\ of a system of\\ functions of \\ one complex variable};
\draw (173.21999999999997, -3) rectangle (178.81999999999996,-5.6);
\draw(178.92, -3) node[anchor=north west,align=left] {Dirichlet series,\\ exponential\\ series and other\\ series in one\\ complex variable};
\draw (178.92, -3) rectangle (183.76999999999998,-5.6);
\draw(167.26999999999998, -5.7) node[anchor=north west,align=left] {Power series\\ (including\\ lacunary \\ series) in one\\ complex variable};
\draw (167.26999999999998, -5.7) rectangle (171.86999999999998,-8.3);
\draw(171.96999999999997, -5.7) node[anchor=north west,align=left] {Random power\\ series\\ in one \\ complex variable};
\draw (171.96999999999997, -5.7) rectangle (176.56999999999996,-7.800000000000001);
\draw(176.67, -5.7) node[anchor=north west,align=left] {Analytic \\ continuation\\ of functions\\ of one \\ complex variable};
\draw (176.67, -5.7) rectangle (181.26999999999998,-8.3);
\draw(181.36999999999998, -5.7) node[anchor=north west,align=left] {Continued \\ fractions; \\ complex-analytic\\ aspects};
\draw (181.36999999999998, -5.7) rectangle (185.96999999999997,-7.800000000000001);
\draw(138.17, -9.5) node[anchor=north west,align=left] {\large Spaces and algebras of analytic functions of one complex variable};
\draw (138.17, -9.5) rectangle (160.67,-14.9);
\draw(139.17, -10.5) node[anchor=north west,align=left] {Spaces of \\ bounded analytic\\ functions\\ of one complex\\ variable};
\draw (139.17, -10.5) rectangle (143.76999999999998,-13.1);
\draw(143.86999999999998, -10.5) node[anchor=north west,align=left] {Algebras \\ of analytic\\ functions\\ of one \\ complex variable};
\draw (143.86999999999998, -10.5) rectangle (148.46999999999997,-13.1);
\draw(148.57, -10.5) node[anchor=north west,align=left] {de \\ Branges-Rovnyak\\ spaces};
\draw (148.57, -10.5) rectangle (152.92,-12.1);
\draw(153.01999999999998, -10.5) node[anchor=north west,align=left] {Besov spaces\\ and \\ \(Q_p\)-spaces};
\draw (153.01999999999998, -10.5) rectangle (157.11999999999998,-12.1);
\draw(157.21999999999997, -10.5) node[anchor=north west,align=left] {Nevanlinna\\ spaces\\ and Smirnov\\ spaces};
\draw (157.21999999999997, -10.5) rectangle (160.56999999999996,-12.6);
\draw(139.17, -13.2) node[anchor=north west,align=left] {Bergman \\ spaces and \\ Fock spaces};
\draw (139.17, -13.2) rectangle (142.51999999999998,-14.799999999999999);
\draw(142.61999999999998, -13.2) node[anchor=north west,align=left] {BMO-spaces};
\draw (142.61999999999998, -13.2) rectangle (145.71999999999997,-14.299999999999999);
\draw(145.82, -13.2) node[anchor=north west,align=left] {Corona\\ theorems};
\draw (145.82, -13.2) rectangle (148.42,-14.299999999999999);
\draw(148.51999999999998, -13.2) node[anchor=north west,align=left] {Zygmund\\ spaces};
\draw (148.51999999999998, -13.2) rectangle (150.86999999999998,-14.299999999999999);
\draw(150.97, -13.2) node[anchor=north west,align=left] {Hardy\\ spaces};
\draw (150.97, -13.2) rectangle (153.07,-14.299999999999999);
\draw(153.17, -13.2) node[anchor=north west,align=left] {Bloch\\ spaces};
\draw (153.17, -13.2) rectangle (155.26999999999998,-14.299999999999999);
\draw(160.76999999999998, -9.5) node[anchor=north west,align=left] {\large Miscellaneous topics of analysis in the complex plane};
\draw (160.76999999999998, -9.5) rectangle (180.57,-15.9);
\draw(161.76999999999998, -10.5) node[anchor=north west,align=left] {Integration, \\ integrals of Cauchy \\ type, integral \\ representations of \\ analytic functions\\ in the complex plane};
\draw (161.76999999999998, -10.5) rectangle (167.36999999999998,-13.6);
\draw(167.46999999999997, -10.5) node[anchor=north west,align=left] {Moment problems\\ and \\ interpolation \\ problems in the\\ complex plane};
\draw (167.46999999999997, -10.5) rectangle (171.81999999999996,-13.1);
\draw(171.92, -10.5) node[anchor=north west,align=left] {Asymptotic\\ representations\\ in the\\ complex plane};
\draw (171.92, -10.5) rectangle (176.26999999999998,-12.6);
\draw(176.36999999999998, -10.5) node[anchor=north west,align=left] {Boundary \\ value problems\\ in the \\ complex plane};
\draw (176.36999999999998, -10.5) rectangle (180.46999999999997,-12.6);
\draw(161.76999999999998, -13.7) node[anchor=north west,align=left] {Approximation\\ in\\ the \\ complex plane};
\draw (161.76999999999998, -13.7) rectangle (165.61999999999998,-15.799999999999999);
\draw(180.67, -9.5) node[anchor=north west,align=left] {\large History of \\ functions of a \\ complex variable};
\draw (180.67, -9.5) rectangle (186.23,-11.1);
\draw(138.17, -16.0) node[anchor=north west,align=left] {\large Universal holomorphic functions of one complex variable};
\draw (138.17, -16.0) rectangle (157.22,-19.2);
\draw(139.17, -17.0) node[anchor=north west,align=left] {Universal \\ Taylor series\\ in one \\ complex variable};
\draw (139.17, -17.0) rectangle (143.76999999999998,-19.1);
\draw(143.86999999999998, -17.0) node[anchor=north west,align=left] {Universal \\ Dirichlet series\\ in one \\ complex variable};
\draw (143.86999999999998, -17.0) rectangle (148.46999999999997,-19.1);
\draw(148.57, -17.0) node[anchor=north west,align=left] {Universal\\ functions\\ of one \\ complex variable};
\draw (148.57, -17.0) rectangle (153.17,-19.1);
\draw(153.26999999999998, -17.0) node[anchor=north west,align=left] {Compositional\\ universality};
\draw (153.26999999999998, -17.0) rectangle (157.11999999999998,-18.6);
\draw(157.32, -16.0) node[anchor=north west,align=left] {\large General properties of functions of one complex variable};
\draw (157.32, -16.0) rectangle (174.97,-19.7);
\draw(158.32, -17.0) node[anchor=north west,align=left] {Monogenic \\ and polygenic\\ functions\\ of one \\ complex variable};
\draw (158.32, -17.0) rectangle (162.92,-19.6);
\draw(163.01999999999998, -17.0) node[anchor=north west,align=left] {Inequalities\\ in\\ the complex\\ plane};
\draw (163.01999999999998, -17.0) rectangle (166.61999999999998,-19.1);
\draw(175.07, -16.0) node[anchor=north west,align=left] {\large Riemann surfaces};
\draw (175.07, -16.0) rectangle (186.72,-28.5);
\draw(176.07, -17.0) node[anchor=north west,align=left] {Conformal \\ metrics \\ (hyperbolic, Poincaré,\\ distance\\ functions)};
\draw (176.07, -17.0) rectangle (182.17,-19.6);
\draw(182.26999999999998, -17.0) node[anchor=north west,align=left] {Ideal \\ boundary theory\\ for Riemann\\ surfaces};
\draw (182.26999999999998, -17.0) rectangle (186.61999999999998,-19.1);
\draw(176.07, -19.7) node[anchor=north west,align=left] {Fuchsian groups\\ and automorphic\\ functions (aspects\\ of compact \\ Riemann surfaces \\ and uniformization)};
\draw (176.07, -19.7) rectangle (181.42,-22.8);
\draw(181.51999999999998, -19.7) node[anchor=north west,align=left] {Kleinian groups\\ (aspects of\\ compact Riemann\\ surfaces and\\ uniformization)};
\draw (181.51999999999998, -19.7) rectangle (185.86999999999998,-22.3);
\draw(176.07, -22.9) node[anchor=north west,align=left] {Compact \\ Riemann \\ surfaces and \\ uniformization};
\draw (176.07, -22.9) rectangle (180.17,-25.0);
\draw(180.26999999999998, -22.9) node[anchor=north west,align=left] {Classification\\ theory\\ of Riemann\\ surfaces};
\draw (180.26999999999998, -22.9) rectangle (184.36999999999998,-25.0);
\draw(176.07, -25.1) node[anchor=north west,align=left] {Differentials\\ on\\ Riemann\\ surfaces};
\draw (176.07, -25.1) rectangle (179.92,-27.200000000000003);
\draw(180.01999999999998, -25.1) node[anchor=north west,align=left] {Teichmüller\\ theory\\ for Riemann\\ surfaces};
\draw (180.01999999999998, -25.1) rectangle (183.36999999999998,-27.200000000000003);
\draw(183.47, -25.1) node[anchor=north west,align=left] {Harmonic\\ functions\\ on Riemann\\ surfaces};
\draw (183.47, -25.1) rectangle (186.57,-27.200000000000003);
\draw(176.07, -27.3) node[anchor=north west,align=left] {Klein \\ surfaces};
\draw (176.07, -27.3) rectangle (178.67,-28.400000000000002);
\draw(157.32, -19.8) node[anchor=north west,align=left] {\large Computational methods for\\ problems pertaining to \\ functions of a complex variable};
\draw (157.32, -19.8) rectangle (167.53,-21.400000000000002);
\draw(157.32, -21.5) node[anchor=north west,align=left] {\large Analysis on metric spaces};
\draw (157.32, -21.5) rectangle (166.47,-26.4);
\draw(158.32, -22.5) node[anchor=north west,align=left] {Quasiconformal\\ mappings\\ in \\ metric spaces};
\draw (158.32, -22.5) rectangle (162.42,-24.6);
\draw(162.51999999999998, -22.5) node[anchor=north west,align=left] {Geometric\\ embeddings\\ of \\ metric spaces};
\draw (162.51999999999998, -22.5) rectangle (166.36999999999998,-24.6);
\draw(158.32, -24.7) node[anchor=north west,align=left] {Inequalities\\ in \\ metric spaces};
\draw (158.32, -24.7) rectangle (162.17,-26.3);
\draw(138.17, -28.6) node[anchor=north west,align=left] {\large Geometric function theory};
\draw (138.17, -28.6) rectangle (151.57,-50.5);
\draw(139.17, -29.6) node[anchor=north west,align=left] {Schwarz-Christoffel-type\\ mappings};
\draw (139.17, -29.6) rectangle (145.76999999999998,-31.700000000000003);
\draw(145.86999999999998, -29.6) node[anchor=north west,align=left] {Maximum principle,\\ Schwarz’s \\ lemma, Lindelöf \\ principle, analogues\\ and generalizations;\\ subordination};
\draw (145.86999999999998, -29.6) rectangle (151.46999999999997,-32.7);
\draw(139.17, -32.800000000000004) node[anchor=north west,align=left] {Extremal problems\\ for conformal\\ and \\ quasiconformal mappings,\\ other methods};
\draw (139.17, -32.800000000000004) rectangle (145.76999999999998,-35.400000000000006);
\draw(145.86999999999998, -32.800000000000004) node[anchor=north west,align=left] {Extremal problems\\ for conformal\\ and quasiconformal\\ mappings, \\ variational methods};
\draw (145.86999999999998, -32.800000000000004) rectangle (151.21999999999997,-35.400000000000006);
\draw(139.17, -35.5) node[anchor=north west,align=left] {Zeros of polynomials,\\ rational functions,\\ and other analytic\\ functions of one\\ complex variable\\ (e.g., zeros of \\ functions with bounded\\ Dirichlet integral)};
\draw (139.17, -35.5) rectangle (145.26999999999998,-39.6);
\draw(145.36999999999998, -35.5) node[anchor=north west,align=left] {Special classes \\ of univalent and \\ multivalent functions\\ of one complex\\ variable (starlike,\\ convex, bounded\\ rotation, etc.)};
\draw (145.36999999999998, -35.5) rectangle (151.21999999999997,-39.1);
\draw(139.17, -39.7) node[anchor=north west,align=left] {Coefficient \\ problems for \\ univalent and \\ multivalent functions\\ of one \\ complex variable};
\draw (139.17, -39.7) rectangle (145.01999999999998,-42.800000000000004);
\draw(145.11999999999998, -39.7) node[anchor=north west,align=left] {Quasiconformal\\ mappings in\\ \(\mathbb{R}^n\),\\ other \\ generalizations};
\draw (145.11999999999998, -39.7) rectangle (149.96999999999997,-42.300000000000004);
\draw(139.17, -42.900000000000006) node[anchor=north west,align=left] {Polynomials\\ and rational\\ functions\\ of one \\ complex variable};
\draw (139.17, -42.900000000000006) rectangle (143.76999999999998,-45.50000000000001);
\draw(143.86999999999998, -42.900000000000006) node[anchor=north west,align=left] {Kernel \\ functions in one\\ complex \\ variable and\\ applications};
\draw (143.86999999999998, -42.900000000000006) rectangle (148.46999999999997,-45.50000000000001);
\draw(148.57, -42.900000000000006) node[anchor=north west,align=left] {General \\ theory of\\ conformal\\ mappings};
\draw (148.57, -42.900000000000006) rectangle (151.42,-45.00000000000001);
\draw(139.17, -45.6) node[anchor=north west,align=left] {General theory\\ of univalent and\\ multivalent \\ functions of one\\ complex variable};
\draw (139.17, -45.6) rectangle (143.76999999999998,-48.2);
\draw(143.86999999999998, -45.6) node[anchor=north west,align=left] {Covering \\ theorems in \\ conformal \\ mapping theory};
\draw (143.86999999999998, -45.6) rectangle (147.96999999999997,-47.7);
\draw(148.07, -45.6) node[anchor=north west,align=left] {Conformal\\ mappings\\ of special\\ domains};
\draw (148.07, -45.6) rectangle (151.17,-47.7);
\draw(139.17, -48.3) node[anchor=north west,align=left] {Quasiconformal\\ mappings\\ in the \\ complex plane};
\draw (139.17, -48.3) rectangle (143.26999999999998,-50.4);
\draw(143.36999999999998, -48.3) node[anchor=north west,align=left] {Capacity and\\ harmonic \\ measure in the\\ complex plane};
\draw (143.36999999999998, -48.3) rectangle (147.46999999999997,-50.4);
\draw(151.67, -28.6) node[anchor=north west,align=left] {\large Generalized function theory};
\draw (151.67, -28.6) rectangle (163.07,-38.2);
\draw(152.67, -29.6) node[anchor=north west,align=left] {Finely \\ holomorphic functions\\ and \\ topological \\ function theory};
\draw (152.67, -29.6) rectangle (158.51999999999998,-32.2);
\draw(158.61999999999998, -29.6) node[anchor=north west,align=left] {Non-Archimedean\\ function theory};
\draw (158.61999999999998, -29.6) rectangle (162.96999999999997,-31.200000000000003);
\draw(152.67, -32.300000000000004) node[anchor=north west,align=left] {Generalizations\\ of Bers and \\ Vekua type \\ (pseudoanalytic, \\ \(p\)-analytic, etc.)};
\draw (152.67, -32.300000000000004) rectangle (158.51999999999998,-34.900000000000006);
\draw(158.61999999999998, -32.300000000000004) node[anchor=north west,align=left] {Functions of\\ hypercomplex\\ variables\\ and generalized\\ variables};
\draw (158.61999999999998, -32.300000000000004) rectangle (162.96999999999997,-34.900000000000006);
\draw(152.67, -35.0) node[anchor=north west,align=left] {Other \\ generalizations of \\ analytic functions\\ (including\\ abstract-valued\\ functions)};
\draw (152.67, -35.0) rectangle (158.01999999999998,-38.1);
\draw(158.11999999999998, -35.0) node[anchor=north west,align=left] {Discrete\\ analytic\\ functions};
\draw (158.11999999999998, -35.0) rectangle (160.96999999999997,-36.6);
\draw(151.67, -38.300000000000004) node[anchor=north west,align=left] {\large Function theory on the disc};
\draw (151.67, -38.300000000000004) rectangle (160.64,-43.2);
\draw(152.67, -39.300000000000004) node[anchor=north west,align=left] {Singular inner\\ functions\\ of one complex\\ variable};
\draw (152.67, -39.300000000000004) rectangle (156.76999999999998,-41.400000000000006);
\draw(156.86999999999998, -39.300000000000004) node[anchor=north west,align=left] {Inner \\ functions of\\ one complex\\ variable};
\draw (156.86999999999998, -39.300000000000004) rectangle (160.46999999999997,-41.400000000000006);
\draw(152.67, -41.50000000000001) node[anchor=north west,align=left] {Blaschke\\ products};
\draw (152.67, -41.50000000000001) rectangle (155.26999999999998,-43.10000000000001);
\draw(137.17, -50.7) node[anchor=north west,align=left] {\LARGE Number theory};
\draw (137.17, -50.7) rectangle (184.9,-145.7);
\draw(138.17, -51.7) node[anchor=north west,align=left] {\large Probabilistic theory: distribution modulo \(1\); metric theory of algorithms};
\draw (138.17, -51.7) rectangle (164.82,-58.6);
\draw(139.17, -52.7) node[anchor=north west,align=left] {Normal numbers,\\ radix expansions,\\ Pisot \\ numbers, Salem \\ numbers, good \\ lattice points, etc.};
\draw (139.17, -52.7) rectangle (144.76999999999998,-55.800000000000004);
\draw(144.86999999999998, -52.7) node[anchor=north west,align=left] {Metric theory\\ of other \\ algorithms and \\ expansions; \\ measure and \\ Hausdorff dimension};
\draw (144.86999999999998, -52.7) rectangle (150.21999999999997,-55.800000000000004);
\draw(150.32, -52.7) node[anchor=north west,align=left] {Harmonic analysis\\ and almost\\ periodicity \\ in probabilistic\\ number theory};
\draw (150.32, -52.7) rectangle (155.17,-55.300000000000004);
\draw(155.26999999999998, -52.7) node[anchor=north west,align=left] {Well-distributed\\ sequences\\ and \\ other variations};
\draw (155.26999999999998, -52.7) rectangle (159.86999999999998,-54.800000000000004);
\draw(159.97, -52.7) node[anchor=north west,align=left] {Irregularities\\ of \\ distribution,\\ discrepancy};
\draw (159.97, -52.7) rectangle (164.07,-54.800000000000004);
\draw(139.17, -55.900000000000006) node[anchor=north west,align=left] {Diophantine\\ approximation\\ in \\ probabilistic \\ number theory};
\draw (139.17, -55.900000000000006) rectangle (143.26999999999998,-58.50000000000001);
\draw(143.36999999999998, -55.900000000000006) node[anchor=north west,align=left] {Pseudo-random\\ numbers;\\ Monte \\ Carlo methods};
\draw (143.36999999999998, -55.900000000000006) rectangle (147.21999999999997,-58.00000000000001);
\draw(147.32, -55.900000000000006) node[anchor=north west,align=left] {Arithmetic \\ functions in\\ probabilistic\\ number theory};
\draw (147.32, -55.900000000000006) rectangle (151.17,-58.00000000000001);
\draw(151.26999999999998, -55.900000000000006) node[anchor=north west,align=left] {General \\ theory of \\ distribution\\ modulo \(1\)};
\draw (151.26999999999998, -55.900000000000006) rectangle (154.86999999999998,-58.00000000000001);
\draw(154.97, -55.900000000000006) node[anchor=north west,align=left] {Continuous,\\ \(p\)-adic\\ and abstract\\ analogues};
\draw (154.97, -55.900000000000006) rectangle (158.57,-58.00000000000001);
\draw(158.67, -55.900000000000006) node[anchor=north west,align=left] {Metric \\ theory of \\ continued\\ fractions};
\draw (158.67, -55.900000000000006) rectangle (161.76999999999998,-58.00000000000001);
\draw(161.86999999999998, -55.900000000000006) node[anchor=north west,align=left] {Special\\ sequences};
\draw (161.86999999999998, -55.900000000000006) rectangle (164.71999999999997,-57.50000000000001);
\draw(164.92, -51.7) node[anchor=north west,align=left] {\large Arithmetic algebraic geometry (Diophantine geometry)};
\draw (164.92, -51.7) rectangle (184.72,-63.0);
\draw(165.92, -52.7) node[anchor=north west,align=left] {\(L\)-functions\\ of varieties\\ over global\\ fields; \\ Birch-Swinnerton-Dyer\\ conjecture};
\draw (165.92, -52.7) rectangle (171.76999999999998,-55.800000000000004);
\draw(171.86999999999998, -52.7) node[anchor=north west,align=left] {Drinfel’d \\ modules; \\ higher-dimensional\\ motives, etc.};
\draw (171.86999999999998, -52.7) rectangle (176.96999999999997,-54.800000000000004);
\draw(177.07, -52.7) node[anchor=north west,align=left] {Arithmetic \\ aspects of dessins\\ d’enfants,\\ Belyĭ theory};
\draw (177.07, -52.7) rectangle (182.17,-54.800000000000004);
\draw(182.26999999999998, -52.7) node[anchor=north west,align=left] {Heights};
\draw (182.26999999999998, -52.7) rectangle (184.61999999999998,-53.800000000000004);
\draw(165.92, -55.900000000000006) node[anchor=north west,align=left] {Complex \\ multiplication\\ and \\ moduli of \\ abelian varieties};
\draw (165.92, -55.900000000000006) rectangle (170.76999999999998,-58.50000000000001);
\draw(170.86999999999998, -55.900000000000006) node[anchor=north west,align=left] {Arithmetic \\ aspects of \\ modular and \\ Shimura varieties};
\draw (170.86999999999998, -55.900000000000006) rectangle (175.71999999999997,-58.00000000000001);
\draw(175.82, -55.900000000000006) node[anchor=north west,align=left] {Curves of \\ arbitrary genus\\ or genus \\ \(\ne~1\) over\\ global fields};
\draw (175.82, -55.900000000000006) rectangle (180.17,-58.50000000000001);
\draw(180.26999999999998, -55.900000000000006) node[anchor=north west,align=left] {Polylogarithms\\ and \\ relations with\\ \(K\)-theory};
\draw (180.26999999999998, -55.900000000000006) rectangle (184.36999999999998,-58.00000000000001);
\draw(165.92, -58.6) node[anchor=north west,align=left] {Elliptic\\ curves\\ over \\ global fields};
\draw (165.92, -58.6) rectangle (169.76999999999998,-60.7);
\draw(169.86999999999998, -58.6) node[anchor=north west,align=left] {Elliptic\\ and \\ modular units};
\draw (169.86999999999998, -58.6) rectangle (173.71999999999997,-60.2);
\draw(173.82, -58.6) node[anchor=north west,align=left] {Varieties\\ over \\ global fields};
\draw (173.82, -58.6) rectangle (177.67,-60.2);
\draw(177.76999999999998, -58.6) node[anchor=north west,align=left] {Abelian \\ varieties\\ of dimension\\ \(>~1\)};
\draw (177.76999999999998, -58.6) rectangle (181.36999999999998,-60.7);
\draw(181.47, -58.6) node[anchor=north west,align=left] {Elliptic\\ curves\\ over local\\ fields};
\draw (181.47, -58.6) rectangle (184.57,-60.7);
\draw(165.92, -60.800000000000004) node[anchor=north west,align=left] {Curves \\ over finite\\ and \\ local fields};
\draw (165.92, -60.800000000000004) rectangle (169.51999999999998,-62.900000000000006);
\draw(169.61999999999998, -60.800000000000004) node[anchor=north west,align=left] {Varieties\\ over \\ finite and \\ local fields};
\draw (169.61999999999998, -60.800000000000004) rectangle (173.21999999999997,-62.900000000000006);
\draw(173.32, -60.800000000000004) node[anchor=north west,align=left] {Geometric\\ class \\ field theory};
\draw (173.32, -60.800000000000004) rectangle (176.92,-62.400000000000006);
\draw(177.01999999999998, -60.800000000000004) node[anchor=north west,align=left] {Arithmetic\\ mirror\\ symmetry};
\draw (177.01999999999998, -60.800000000000004) rectangle (180.11999999999998,-62.400000000000006);
\draw(138.17, -58.7) node[anchor=north west,align=left] {\large Miscellaneous applications of number theory};
\draw (138.17, -58.7) rectangle (152.1,-61.900000000000006);
\draw(139.17, -59.7) node[anchor=north west,align=left] {Miscellaneous\\ applications of\\ number theory};
\draw (139.17, -59.7) rectangle (143.51999999999998,-61.800000000000004);
\draw(138.17, -63.1) node[anchor=north west,align=left] {\large Finite fields and commutative rings (number-theoretic aspects)};
\draw (138.17, -63.1) rectangle (159.22,-69.5);
\draw(139.17, -64.1) node[anchor=north west,align=left] {Structure \\ theory for finite\\ fields and\\ commutative \\ rings \\ (number-theoretic aspects)};
\draw (139.17, -64.1) rectangle (146.26999999999998,-67.19999999999999);
\draw(146.36999999999998, -64.1) node[anchor=north west,align=left] {Algebraic \\ coding theory;\\ cryptography\\ (number-theoretic\\ aspects)};
\draw (146.36999999999998, -64.1) rectangle (151.21999999999997,-66.69999999999999);
\draw(151.32, -64.1) node[anchor=north west,align=left] {Polynomials\\ over \\ finite fields};
\draw (151.32, -64.1) rectangle (155.17,-65.69999999999999);
\draw(155.26999999999998, -64.1) node[anchor=north west,align=left] {Arithmetic\\ theory of \\ polynomial \\ rings over \\ finite fields};
\draw (155.26999999999998, -64.1) rectangle (159.11999999999998,-66.69999999999999);
\draw(139.17, -67.3) node[anchor=north west,align=left] {Exponential\\ sums};
\draw (139.17, -67.3) rectangle (142.51999999999998,-68.39999999999999);
\draw(142.61999999999998, -67.3) node[anchor=north west,align=left] {Finite \\ upper \\ half-planes};
\draw (142.61999999999998, -67.3) rectangle (145.96999999999997,-68.89999999999999);
\draw(146.07, -67.3) node[anchor=north west,align=left] {Other \\ character\\ sums and\\ Gauss sums};
\draw (146.07, -67.3) rectangle (149.17,-69.39999999999999);
\draw(149.26999999999998, -67.3) node[anchor=north west,align=left] {Cyclotomy};
\draw (149.26999999999998, -67.3) rectangle (152.11999999999998,-68.39999999999999);
\draw(159.32, -63.1) node[anchor=north west,align=left] {\large Diophantine approximation, transcendental number theory};
\draw (159.32, -63.1) rectangle (180.07,-76.1);
\draw(160.32, -64.1) node[anchor=north west,align=left] {Transcendence\\ theory\\ of elliptic\\ and \\ abelian functions};
\draw (160.32, -64.1) rectangle (165.17,-66.69999999999999);
\draw(165.26999999999998, -64.1) node[anchor=north west,align=left] {Transcendence\\ theory \\ of other \\ special functions};
\draw (165.26999999999998, -64.1) rectangle (170.11999999999998,-66.19999999999999);
\draw(170.22, -64.1) node[anchor=north west,align=left] {Small \\ fractional parts\\ of polynomials\\ and \\ generalizations};
\draw (170.22, -64.1) rectangle (174.82,-66.69999999999999);
\draw(174.92, -64.1) node[anchor=north west,align=left] {Number-theoretic\\ analogues\\ of methods \\ in Nevanlinna\\ theory (work\\ of Vojta et al.)};
\draw (174.92, -64.1) rectangle (179.51999999999998,-67.19999999999999);
\draw(160.32, -67.3) node[anchor=north west,align=left] {Markov and\\ Lagrange \\ spectra and \\ generalizations};
\draw (160.32, -67.3) rectangle (164.67,-69.39999999999999);
\draw(164.76999999999998, -67.3) node[anchor=north west,align=left] {Approximation\\ in \\ non-Archimedean\\ valuations};
\draw (164.76999999999998, -67.3) rectangle (169.11999999999998,-69.39999999999999);
\draw(169.22, -67.3) node[anchor=north west,align=left] {Continued\\ fractions\\ and \\ generalizations};
\draw (169.22, -67.3) rectangle (173.57,-69.39999999999999);
\draw(173.67, -67.3) node[anchor=north west,align=left] {Homogeneous\\ approximation \\ to one number};
\draw (173.67, -67.3) rectangle (177.76999999999998,-69.39999999999999);
\draw(177.87, -67.3) node[anchor=north west,align=left] {Metric\\ theory};
\draw (177.87, -67.3) rectangle (179.97,-68.39999999999999);
\draw(160.32, -69.5) node[anchor=north west,align=left] {Simultaneous\\ homogeneous \\ approximation,\\ linear forms};
\draw (160.32, -69.5) rectangle (164.42,-71.6);
\draw(164.51999999999998, -69.5) node[anchor=north west,align=left] {Irrationality;\\ linear\\ independence\\ over a field};
\draw (164.51999999999998, -69.5) rectangle (168.61999999999998,-71.6);
\draw(168.72, -69.5) node[anchor=north west,align=left] {Linear forms\\ in \\ logarithms; \\ Baker’s method};
\draw (168.72, -69.5) rectangle (172.82,-71.6);
\draw(172.92, -69.5) node[anchor=north west,align=left] {Approximation\\ by \\ numbers from\\ a fixed field};
\draw (172.92, -69.5) rectangle (176.76999999999998,-71.6);
\draw(176.87, -69.5) node[anchor=north west,align=left] {Results\\ involving\\ abelian\\ varieties};
\draw (176.87, -69.5) rectangle (179.72,-71.6);
\draw(160.32, -71.7) node[anchor=north west,align=left] {Inhomogeneous\\ linear forms};
\draw (160.32, -71.7) rectangle (164.17,-73.3);
\draw(164.26999999999998, -71.7) node[anchor=north west,align=left] {Approximation\\ to\\ algebraic\\ numbers};
\draw (164.26999999999998, -71.7) rectangle (168.11999999999998,-73.8);
\draw(168.22, -71.7) node[anchor=north west,align=left] {Transcendence\\ (general\\ theory)};
\draw (168.22, -71.7) rectangle (172.07,-73.3);
\draw(172.17, -71.7) node[anchor=north west,align=left] {Measures of\\ irrationality\\ and of\\ transcendence};
\draw (172.17, -71.7) rectangle (176.01999999999998,-73.8);
\draw(176.12, -71.7) node[anchor=north west,align=left] {Algebraic\\ independence;\\ Gel’fond’s\\ method};
\draw (176.12, -71.7) rectangle (179.97,-73.8);
\draw(160.32, -73.9) node[anchor=north west,align=left] {Transcendence\\ theory of\\ Drinfel’d and\\ \(t\)-modules};
\draw (160.32, -73.9) rectangle (164.17,-76.0);
\draw(164.26999999999998, -73.9) node[anchor=north west,align=left] {Diophantine\\ inequalities};
\draw (164.26999999999998, -73.9) rectangle (167.86999999999998,-75.5);
\draw(167.97, -73.9) node[anchor=north west,align=left] {Distribution\\ modulo one};
\draw (167.97, -73.9) rectangle (171.57,-75.5);
\draw(171.67, -73.9) node[anchor=north west,align=left] {Schmidt \\ Subspace \\ Theorem and \\ applications};
\draw (171.67, -73.9) rectangle (175.26999999999998,-76.0);
\draw(138.17, -69.60000000000001) node[anchor=north west,align=left] {\large Connections of number theory and logic};
\draw (138.17, -69.60000000000001) rectangle (150.54999999999998,-75.00000000000001);
\draw(139.17, -70.60000000000001) node[anchor=north west,align=left] {Decidability\\ (number-theoretic\\ aspects)};
\draw (139.17, -70.60000000000001) rectangle (144.01999999999998,-72.7);
\draw(144.11999999999998, -70.60000000000001) node[anchor=north west,align=left] {Ultraproducts\\ (number-theoretic\\ aspects)};
\draw (144.11999999999998, -70.60000000000001) rectangle (148.96999999999997,-72.7);
\draw(139.17, -72.80000000000001) node[anchor=north west,align=left] {Model \\ theory \\ (number-theoretic\\ aspects)};
\draw (139.17, -72.80000000000001) rectangle (144.01999999999998,-74.9);
\draw(144.11999999999998, -72.80000000000001) node[anchor=north west,align=left] {Nonstandard\\ arithmetic \\ (number-theoretic\\ aspects)};
\draw (144.11999999999998, -72.80000000000001) rectangle (148.96999999999997,-74.9);
\draw(180.17, -63.1) node[anchor=north west,align=left] {\large History of \\ number theory};
\draw (180.17, -63.1) rectangle (184.79999999999998,-64.2);
\draw(138.17, -76.2) node[anchor=north west,align=left] {\large Algebraic number theory: local and \(p\)-adic fields};
\draw (138.17, -76.2) rectangle (157.97,-86.5);
\draw(139.17, -77.2) node[anchor=north west,align=left] {Non-Archimedeandynamical\\ systems};
\draw (139.17, -77.2) rectangle (145.76999999999998,-79.3);
\draw(145.86999999999998, -77.2) node[anchor=north west,align=left] {Other analytic\\ theory (analogues\\ of beta and\\ gamma functions,\\ \(p\)-adic \\ integration, etc.)};
\draw (145.86999999999998, -77.2) rectangle (150.96999999999997,-80.3);
\draw(151.07, -77.2) node[anchor=north west,align=left] {Langlands-Weil\\ conjectures,\\ nonabelian class\\ field theory};
\draw (151.07, -77.2) rectangle (155.67,-79.8);
\draw(155.76999999999998, -77.2) node[anchor=north west,align=left] {Galois\\ theory};
\draw (155.76999999999998, -77.2) rectangle (157.86999999999998,-78.3);
\draw(139.17, -80.4) node[anchor=north west,align=left] {Integral\\ representations};
\draw (139.17, -80.4) rectangle (143.51999999999998,-82.0);
\draw(143.61999999999998, -80.4) node[anchor=north west,align=left] {Zeta \\ functions \\ and \\ \(L\)-functions};
\draw (143.61999999999998, -80.4) rectangle (147.96999999999997,-82.5);
\draw(148.07, -80.4) node[anchor=north west,align=left] {Algebras and\\ orders, \\ and their \\ zeta functions};
\draw (148.07, -80.4) rectangle (152.17,-82.5);
\draw(152.26999999999998, -80.4) node[anchor=north west,align=left] {Prehomogeneous\\ vector spaces};
\draw (152.26999999999998, -80.4) rectangle (156.36999999999998,-82.0);
\draw(139.17, -82.60000000000001) node[anchor=north west,align=left] {Class field\\ theory; \\ \(p\)-adic \\ formal groups};
\draw (139.17, -82.60000000000001) rectangle (143.01999999999998,-84.7);
\draw(143.11999999999998, -82.60000000000001) node[anchor=north west,align=left] {Ramification\\ and\\ extension\\ theory};
\draw (143.11999999999998, -82.60000000000001) rectangle (146.71999999999997,-84.7);
\draw(146.82, -82.60000000000001) node[anchor=north west,align=left] {\(K\)-theory\\ of \\ local fields};
\draw (146.82, -82.60000000000001) rectangle (150.42,-84.2);
\draw(150.51999999999998, -82.60000000000001) node[anchor=north west,align=left] {Polynomials};
\draw (150.51999999999998, -82.60000000000001) rectangle (153.86999999999998,-83.7);
\draw(153.97, -82.60000000000001) node[anchor=north west,align=left] {Other \\ nonanalytic\\ theory};
\draw (153.97, -82.60000000000001) rectangle (157.32,-84.2);
\draw(139.17, -84.80000000000001) node[anchor=north west,align=left] {Galois\\ cohomology};
\draw (139.17, -84.80000000000001) rectangle (142.26999999999998,-86.4);
\draw(158.07, -76.2) node[anchor=north west,align=left] {\large Discontinuous groups and automorphic forms};
\draw (158.07, -76.2) rectangle (174.17,-105.2);
\draw(159.07, -77.2) node[anchor=north west,align=left] {Automorphic forms on\\ \(\mbox{GL}(2)\); \\ Hilbert and Hilbert-Siegel\\ modular groups\\ and their modular \\ and automorphic forms;\\ Hilbert modular surfaces};
\draw (159.07, -77.2) rectangle (166.17,-80.8);
\draw(166.26999999999998, -77.2) node[anchor=north west,align=left] {Representation-theoretic\\ methods;\\ automorphic\\ representations\\ over local and\\ global fields};
\draw (166.26999999999998, -77.2) rectangle (172.86999999999998,-80.3);
\draw(159.07, -80.9) node[anchor=north west,align=left] {Special values \\ of automorphic \\ \(L\)-series, periods\\ of automorphic\\ forms, cohomology,\\ modular symbols};
\draw (159.07, -80.9) rectangle (164.92,-84.0);
\draw(165.01999999999998, -80.9) node[anchor=north west,align=left] {Langlands \\ \(L\)-functions;\\ one variable\\ Dirichlet \\ series and \\ functional equations};
\draw (165.01999999999998, -80.9) rectangle (170.61999999999998,-84.0);
\draw(170.72, -80.9) node[anchor=north west,align=left] {Modular\\ and \\ automorphic\\ functions};
\draw (170.72, -80.9) rectangle (174.07,-83.0);
\draw(159.07, -84.10000000000001) node[anchor=north west,align=left] {Other groups \\ and their modular\\ and automorphic\\ forms \\ (several variables)};
\draw (159.07, -84.10000000000001) rectangle (164.42,-86.7);
\draw(164.51999999999998, -84.10000000000001) node[anchor=north west,align=left] {Hecke-Petersson\\ operators,\\ differential\\ operators \\ (several variables)};
\draw (164.51999999999998, -84.10000000000001) rectangle (169.86999999999998,-86.7);
\draw(169.97, -84.10000000000001) node[anchor=north west,align=left] {Relations \\ with algebraic\\ geometry\\ and topology};
\draw (169.97, -84.10000000000001) rectangle (174.07,-86.2);
\draw(159.07, -86.80000000000001) node[anchor=north west,align=left] {Dirichlet series\\ in several complex\\ variables \\ associated to \\ automorphic forms; \\ Weyl group multiple\\ Dirichlet series};
\draw (159.07, -86.80000000000001) rectangle (164.42,-90.4);
\draw(164.51999999999998, -86.80000000000001) node[anchor=north west,align=left] {Siegel modular\\ groups; Siegel\\ and Hilbert-Siegel\\ modular and\\ automorphic forms};
\draw (164.51999999999998, -86.80000000000001) rectangle (169.61999999999998,-89.4);
\draw(169.72, -86.80000000000001) node[anchor=north west,align=left] {Holomorphic\\ modular \\ forms of \\ integral weight};
\draw (169.72, -86.80000000000001) rectangle (174.07,-88.9);
\draw(159.07, -90.5) node[anchor=north west,align=left] {Structure of\\ modular groups\\ and \\ generalizations; \\ arithmetic groups};
\draw (159.07, -90.5) rectangle (163.92,-93.1);
\draw(164.01999999999998, -90.5) node[anchor=north west,align=left] {Automorphic\\ forms and \\ their relations\\ with \\ perfectoid spaces};
\draw (164.01999999999998, -90.5) rectangle (168.86999999999998,-93.1);
\draw(168.97, -90.5) node[anchor=north west,align=left] {Modular \\ correspondences,\\ etc.};
\draw (168.97, -90.5) rectangle (173.57,-92.1);
\draw(159.07, -93.2) node[anchor=north west,align=left] {Relationship\\ to Lie algebras\\ and finite\\ simple groups};
\draw (159.07, -93.2) rectangle (163.42,-95.3);
\draw(163.51999999999998, -93.2) node[anchor=north west,align=left] {Hecke-Petersson\\ operators,\\ differential\\ operators\\ (one variable)};
\draw (163.51999999999998, -93.2) rectangle (167.86999999999998,-95.8);
\draw(167.97, -93.2) node[anchor=north west,align=left] {Theta series;\\ Weil \\ representation;\\ theta \\ correspondences};
\draw (167.97, -93.2) rectangle (172.32,-95.8);
\draw(159.07, -95.9) node[anchor=north west,align=left] {Congruences \\ for modular and\\ \(p\)-adic\\ modular forms};
\draw (159.07, -95.9) rectangle (163.42,-98.0);
\draw(163.51999999999998, -95.9) node[anchor=north west,align=left] {Galois\\ representations};
\draw (163.51999999999998, -95.9) rectangle (167.86999999999998,-97.5);
\draw(167.97, -95.9) node[anchor=north west,align=left] {Fourier \\ coefficients\\ of automorphic\\ forms};
\draw (167.97, -95.9) rectangle (172.07,-98.0);
\draw(159.07, -98.1) node[anchor=north west,align=left] {Forms of \\ half-integer\\ weight; \\ nonholomorphic\\ modular forms};
\draw (159.07, -98.1) rectangle (163.17,-100.69999999999999);
\draw(163.26999999999998, -98.1) node[anchor=north west,align=left] {Dedekind\\ eta function,\\ Dedekind sums};
\draw (163.26999999999998, -98.1) rectangle (167.11999999999998,-100.19999999999999);
\draw(167.22, -98.1) node[anchor=north west,align=left] {Modular forms\\ associated\\ to Drinfel’d\\ modules};
\draw (167.22, -98.1) rectangle (171.07,-100.19999999999999);
\draw(171.17, -98.1) node[anchor=north west,align=left] {Jacobi\\ forms};
\draw (171.17, -98.1) rectangle (173.26999999999998,-99.19999999999999);
\draw(159.07, -100.8) node[anchor=north west,align=left] {Spectral \\ theory; trace\\ formulas \\ (e.g., that\\ of Selberg)};
\draw (159.07, -100.8) rectangle (162.92,-103.39999999999999);
\draw(163.01999999999998, -100.8) node[anchor=north west,align=left] {Cohomology\\ of arithmetic\\ groups};
\draw (163.01999999999998, -100.8) rectangle (166.86999999999998,-102.39999999999999);
\draw(166.97, -100.8) node[anchor=north west,align=left] {Automorphic\\ forms, \\ one variable};
\draw (166.97, -100.8) rectangle (170.57,-102.39999999999999);
\draw(159.07, -103.5) node[anchor=north west,align=left] {\(p\)-adic\\ theory, \\ local fields};
\draw (159.07, -103.5) rectangle (162.67,-105.1);
\draw(138.17, -86.6) node[anchor=north west,align=left] {\large Zeta and \(L\)-functions: analytic theory};
\draw (138.17, -86.6) rectangle (153.51999999999998,-98.89999999999999);
\draw(139.17, -87.6) node[anchor=north west,align=left] {Selberg zeta functions\\ and regularized \\ determinants; applications\\ to spectral theory,\\ Dirichlet series, \\ Eisenstein series, etc.\\ (explicit formulas)};
\draw (139.17, -87.6) rectangle (146.26999999999998,-91.19999999999999);
\draw(146.36999999999998, -87.6) node[anchor=north west,align=left] {Nonreal zeros \\ of \(\zeta~(s)\)\\ and \(L(s,~\chi)\);\\ Riemann \\ and other hypotheses};
\draw (146.36999999999998, -87.6) rectangle (151.96999999999997,-90.19999999999999);
\draw(139.17, -91.3) node[anchor=north west,align=left] {Real zeros\\ of \(L(s,~\chi)\);\\ results on \\ \(L(1,~\chi)\)};
\draw (139.17, -91.3) rectangle (144.26999999999998,-93.89999999999999);
\draw(144.36999999999998, -91.3) node[anchor=north west,align=left] {Zeta and \\ \(L\)-functions\\ in characteristic\\ \(p\)};
\draw (144.36999999999998, -91.3) rectangle (149.21999999999997,-93.39999999999999);
\draw(149.32, -91.3) node[anchor=north west,align=left] {\(\zeta~(s)\)\\ and \\ \(L(s,~\chi)\)};
\draw (149.32, -91.3) rectangle (153.42,-92.89999999999999);
\draw(139.17, -94.0) node[anchor=north west,align=left] {Multiple \\ Dirichlet series\\ and zeta \\ functions and \\ multizeta values};
\draw (139.17, -94.0) rectangle (143.76999999999998,-96.6);
\draw(143.86999999999998, -94.0) node[anchor=north west,align=left] {Other \\ Dirichlet series\\ and zeta\\ functions};
\draw (143.86999999999998, -94.0) rectangle (148.46999999999997,-96.1);
\draw(148.57, -94.0) node[anchor=north west,align=left] {Hurwitz and\\ Lerch \\ zeta functions};
\draw (148.57, -94.0) rectangle (152.67,-95.6);
\draw(139.17, -96.69999999999999) node[anchor=north west,align=left] {Relations\\ with \\ noncommutative\\ geometry};
\draw (139.17, -96.69999999999999) rectangle (143.26999999999998,-98.79999999999998);
\draw(143.36999999999998, -96.69999999999999) node[anchor=north west,align=left] {Relations\\ with random\\ matrices};
\draw (143.36999999999998, -96.69999999999999) rectangle (146.71999999999997,-98.29999999999998);
\draw(146.82, -96.69999999999999) node[anchor=north west,align=left] {Tauberian\\ theorems};
\draw (146.82, -96.69999999999999) rectangle (149.67,-98.29999999999998);
\draw(138.17, -99.0) node[anchor=north west,align=left] {\large Polynomials and matrices};
\draw (138.17, -99.0) rectangle (147.07,-102.2);
\draw(139.17, -100.0) node[anchor=north west,align=left] {Polynomials\\ in \\ number theory};
\draw (139.17, -100.0) rectangle (143.01999999999998,-101.6);
\draw(143.11999999999998, -100.0) node[anchor=north west,align=left] {Matrices,\\ determinants\\ in \\ number theory};
\draw (143.11999999999998, -100.0) rectangle (146.96999999999997,-102.1);
\draw(174.26999999999998, -76.2) node[anchor=north west,align=left] {\large Sequences and sets};
\draw (174.26999999999998, -76.2) rectangle (184.67,-92.1);
\draw(175.26999999999998, -77.2) node[anchor=north west,align=left] {Farey sequences;\\ the \\ sequences \\ \(1^k,~2^k,~\dots\)};
\draw (175.26999999999998, -77.2) rectangle (180.61999999999998,-79.3);
\draw(180.71999999999997, -77.2) node[anchor=north west,align=left] {Other \\ combinatorial\\ number\\ theory};
\draw (180.71999999999997, -77.2) rectangle (184.56999999999996,-79.3);
\draw(175.26999999999998, -79.4) node[anchor=north west,align=left] {Arithmetic \\ combinatorics;\\ higher \\ degree uniformity};
\draw (175.26999999999998, -79.4) rectangle (180.11999999999998,-81.5);
\draw(180.21999999999997, -79.4) node[anchor=north west,align=left] {Fibonacci and\\ Lucas numbers\\ and \\ polynomials and\\ generalizations};
\draw (180.21999999999997, -79.4) rectangle (184.56999999999996,-82.0);
\draw(175.26999999999998, -82.10000000000001) node[anchor=north west,align=left] {Binomial \\ coefficients; \\ factorials; \\ \(q\)-identities};
\draw (175.26999999999998, -82.10000000000001) rectangle (179.86999999999998,-84.2);
\draw(179.96999999999997, -82.10000000000001) node[anchor=north west,align=left] {Representation\\ functions};
\draw (179.96999999999997, -82.10000000000001) rectangle (184.06999999999996,-83.7);
\draw(175.26999999999998, -84.30000000000001) node[anchor=north west,align=left] {Arithmetic\\ progressions};
\draw (175.26999999999998, -84.30000000000001) rectangle (178.86999999999998,-85.9);
\draw(178.96999999999997, -84.30000000000001) node[anchor=north west,align=left] {Recurrences};
\draw (178.96999999999997, -84.30000000000001) rectangle (182.31999999999996,-85.4);
\draw(175.26999999999998, -86.0) node[anchor=north west,align=left] {Bernoulli\\ and Euler\\ numbers and\\ polynomials};
\draw (175.26999999999998, -86.0) rectangle (178.61999999999998,-88.1);
\draw(178.71999999999997, -86.0) node[anchor=north west,align=left] {Special\\ sequences\\ and \\ polynomials};
\draw (178.71999999999997, -86.0) rectangle (182.06999999999996,-88.1);
\draw(175.26999999999998, -88.2) node[anchor=north west,align=left] {Additive\\ bases,\\ including\\ sumsets};
\draw (175.26999999999998, -88.2) rectangle (178.11999999999998,-90.3);
\draw(178.21999999999997, -88.2) node[anchor=north west,align=left] {Sequences\\ (mod\\ \(m\))};
\draw (178.21999999999997, -88.2) rectangle (181.06999999999996,-89.8);
\draw(181.17, -88.2) node[anchor=north west,align=left] {Automata\\ sequences};
\draw (181.17, -88.2) rectangle (184.01999999999998,-89.8);
\draw(175.26999999999998, -90.4) node[anchor=north west,align=left] {Density,\\ gaps,\\ topology};
\draw (175.26999999999998, -90.4) rectangle (177.86999999999998,-92.0);
\draw(177.96999999999997, -90.4) node[anchor=north west,align=left] {Bell and\\ Stirling\\ numbers};
\draw (177.96999999999997, -90.4) rectangle (180.56999999999996,-92.0);
\draw(174.26999999999998, -92.2) node[anchor=north west,align=left] {\large Computational number theory};
\draw (174.26999999999998, -92.2) rectangle (184.67,-104.7);
\draw(175.26999999999998, -93.2) node[anchor=north west,align=left] {Continued \\ fraction \\ calculations \\ (number-theoretic\\ aspects)};
\draw (175.26999999999998, -93.2) rectangle (180.11999999999998,-95.8);
\draw(180.21999999999997, -93.2) node[anchor=north west,align=left] {Factorization};
\draw (180.21999999999997, -93.2) rectangle (184.06999999999996,-94.3);
\draw(180.21999999999997, -94.4) node[anchor=north west,align=left] {Primality};
\draw (180.21999999999997, -94.4) rectangle (183.06999999999996,-95.5);
\draw(175.26999999999998, -95.9) node[anchor=north west,align=left] {Number-theoretic\\ algorithms;\\ complexity};
\draw (175.26999999999998, -95.9) rectangle (179.86999999999998,-98.0);
\draw(179.96999999999997, -95.9) node[anchor=north west,align=left] {Evaluation\\ of \\ number-theoretic\\ constants};
\draw (179.96999999999997, -95.9) rectangle (184.56999999999996,-98.0);
\draw(175.26999999999998, -98.10000000000001) node[anchor=north west,align=left] {Analytic\\ computations};
\draw (175.26999999999998, -98.10000000000001) rectangle (178.86999999999998,-99.7);
\draw(178.96999999999997, -98.10000000000001) node[anchor=north west,align=left] {Algebraic\\ number \\ theory \\ computations};
\draw (178.96999999999997, -98.10000000000001) rectangle (182.56999999999996,-100.2);
\draw(175.26999999999998, -100.30000000000001) node[anchor=north west,align=left] {Computer \\ solution of\\ Diophantine\\ equations};
\draw (175.26999999999998, -100.30000000000001) rectangle (178.61999999999998,-102.4);
\draw(178.71999999999997, -100.30000000000001) node[anchor=north west,align=left] {Calculation\\ of integer\\ sequences};
\draw (178.71999999999997, -100.30000000000001) rectangle (182.06999999999996,-101.9);
\draw(175.26999999999998, -102.5) node[anchor=north west,align=left] {Values of\\ arithmetic\\ functions;\\ tables};
\draw (175.26999999999998, -102.5) rectangle (178.36999999999998,-104.6);
\draw(138.17, -105.30000000000001) node[anchor=north west,align=left] {\large Algebraic number theory: global fields};
\draw (138.17, -105.30000000000001) rectangle (152.76999999999998,-127.10000000000001);
\draw(139.17, -106.30000000000001) node[anchor=north west,align=left] {PV-numbers and\\ generalizations;\\ other special\\ algebraic \\ numbers; Mahler measure};
\draw (139.17, -106.30000000000001) rectangle (145.51999999999998,-108.9);
\draw(145.61999999999998, -106.30000000000001) node[anchor=north west,align=left] {Integral \\ representations related\\ to algebraic \\ numbers; Galois \\ module structure \\ of rings of integers};
\draw (145.61999999999998, -106.30000000000001) rectangle (151.96999999999997,-109.4);
\draw(139.17, -109.50000000000001) node[anchor=north west,align=left] {Zeta functions\\ and \\ \(L\)-functions of \\ function fields};
\draw (139.17, -109.50000000000001) rectangle (144.51999999999998,-111.60000000000001);
\draw(144.61999999999998, -109.50000000000001) node[anchor=north west,align=left] {Polynomials\\ (irreducibility,\\ etc.)};
\draw (144.61999999999998, -109.50000000000001) rectangle (149.21999999999997,-111.60000000000001);
\draw(149.32, -109.50000000000001) node[anchor=north west,align=left] {Other \\ abelian and\\ metabelian\\ extensions};
\draw (149.32, -109.50000000000001) rectangle (152.67,-111.60000000000001);
\draw(139.17, -111.70000000000002) node[anchor=north west,align=left] {Langlands-Weil\\ conjectures,\\ nonabelian class\\ field theory};
\draw (139.17, -111.70000000000002) rectangle (143.76999999999998,-114.30000000000001);
\draw(143.86999999999998, -111.70000000000002) node[anchor=north west,align=left] {Zeta functions\\ and \\ \(L\)-functions \\ of number fields};
\draw (143.86999999999998, -111.70000000000002) rectangle (148.46999999999997,-113.80000000000001);
\draw(148.57, -111.70000000000002) node[anchor=north west,align=left] {Algebraic \\ numbers; rings\\ of algebraic\\ integers};
\draw (148.57, -111.70000000000002) rectangle (152.67,-113.80000000000001);
\draw(139.17, -114.4) node[anchor=north west,align=left] {Other algebras\\ and orders,\\ and their\\ zeta and \\ \(L\)-functions};
\draw (139.17, -114.4) rectangle (143.51999999999998,-117.0);
\draw(143.61999999999998, -114.4) node[anchor=north west,align=left] {Arithmetic\\ theory of \\ algebraic \\ function fields};
\draw (143.61999999999998, -114.4) rectangle (147.96999999999997,-116.5);
\draw(148.07, -114.4) node[anchor=north west,align=left] {Cyclotomic \\ function fields\\ (class groups,\\ Bernoulli\\ objects, etc.)};
\draw (148.07, -114.4) rectangle (152.42,-117.0);
\draw(139.17, -117.10000000000001) node[anchor=north west,align=left] {Class \\ numbers, class\\ groups, \\ discriminants};
\draw (139.17, -117.10000000000001) rectangle (143.26999999999998,-119.2);
\draw(143.36999999999998, -117.10000000000001) node[anchor=north west,align=left] {Quaternion and\\ other division\\ algebras:\\ arithmetic,\\ zeta functions};
\draw (143.36999999999998, -117.10000000000001) rectangle (147.46999999999997,-119.7);
\draw(147.57, -117.10000000000001) node[anchor=north west,align=left] {Units \\ and \\ factorization};
\draw (147.57, -117.10000000000001) rectangle (151.42,-118.7);
\draw(139.17, -119.80000000000001) node[anchor=north west,align=left] {Class groups\\ and \\ Picard groups\\ of orders};
\draw (139.17, -119.80000000000001) rectangle (143.01999999999998,-121.9);
\draw(143.11999999999998, -119.80000000000001) node[anchor=north west,align=left] {\(K\)-theory\\ of \\ global fields};
\draw (143.11999999999998, -119.80000000000001) rectangle (146.96999999999997,-121.4);
\draw(147.07, -119.80000000000001) node[anchor=north west,align=left] {Distribution\\ of \\ prime ideals};
\draw (147.07, -119.80000000000001) rectangle (150.67,-121.4);
\draw(139.17, -122.00000000000001) node[anchor=north west,align=left] {Quadratic\\ extensions};
\draw (139.17, -122.00000000000001) rectangle (142.26999999999998,-123.60000000000001);
\draw(142.36999999999998, -122.00000000000001) node[anchor=north west,align=left] {Cubic and\\ quartic\\ extensions};
\draw (142.36999999999998, -122.00000000000001) rectangle (145.46999999999997,-123.60000000000001);
\draw(145.57, -122.00000000000001) node[anchor=north west,align=left] {Cyclotomic\\ extensions};
\draw (145.57, -122.00000000000001) rectangle (148.67,-123.60000000000001);
\draw(148.76999999999998, -122.00000000000001) node[anchor=north west,align=left] {Galois\\ cohomology};
\draw (148.76999999999998, -122.00000000000001) rectangle (151.86999999999998,-123.60000000000001);
\draw(139.17, -123.70000000000002) node[anchor=north west,align=left] {Adèle \\ rings and\\ groups};
\draw (139.17, -123.70000000000002) rectangle (142.01999999999998,-125.30000000000001);
\draw(142.11999999999998, -123.70000000000002) node[anchor=north west,align=left] {Density\\ theorems};
\draw (142.11999999999998, -123.70000000000002) rectangle (144.71999999999997,-124.80000000000001);
\draw(144.82, -123.70000000000002) node[anchor=north west,align=left] {Other \\ analytic\\ theory};
\draw (144.82, -123.70000000000002) rectangle (147.42,-125.30000000000001);
\draw(147.51999999999998, -123.70000000000002) node[anchor=north west,align=left] {Iwasawa\\ theory};
\draw (147.51999999999998, -123.70000000000002) rectangle (149.86999999999998,-124.80000000000001);
\draw(149.97, -123.70000000000002) node[anchor=north west,align=left] {Totally\\ real\\ fields};
\draw (149.97, -123.70000000000002) rectangle (152.32,-125.30000000000001);
\draw(139.17, -125.4) node[anchor=north west,align=left] {Other\\ number\\ fields};
\draw (139.17, -125.4) rectangle (141.26999999999998,-127.0);
\draw(141.36999999999998, -125.4) node[anchor=north west,align=left] {Galois\\ theory};
\draw (141.36999999999998, -125.4) rectangle (143.46999999999997,-126.5);
\draw(143.57, -125.4) node[anchor=north west,align=left] {Class\\ field\\ theory};
\draw (143.57, -125.4) rectangle (145.67,-127.0);
\draw(152.86999999999998, -105.30000000000001) node[anchor=north west,align=left] {\large Exponential sums and character sums};
\draw (152.86999999999998, -105.30000000000001) rectangle (165.96999999999997,-112.9);
\draw(153.86999999999998, -106.30000000000001) node[anchor=north west,align=left] {Estimates\\ on \\ exponential sums};
\draw (153.86999999999998, -106.30000000000001) rectangle (158.46999999999997,-107.9);
\draw(158.56999999999996, -106.30000000000001) node[anchor=north west,align=left] {Gauss and\\ Kloosterman\\ sums; \\ generalizations};
\draw (158.56999999999996, -106.30000000000001) rectangle (162.91999999999996,-108.4);
\draw(163.01999999999998, -106.30000000000001) node[anchor=north west,align=left] {Sums over\\ primes};
\draw (163.01999999999998, -106.30000000000001) rectangle (165.86999999999998,-107.4);
\draw(153.86999999999998, -108.50000000000001) node[anchor=north west,align=left] {Jacobsthal\\ and Brewer\\ sums; other\\ complete \\ character sums};
\draw (153.86999999999998, -108.50000000000001) rectangle (157.96999999999997,-111.10000000000001);
\draw(158.06999999999996, -108.50000000000001) node[anchor=north west,align=left] {Trigonometric\\ and \\ exponential\\ sums, general};
\draw (158.06999999999996, -108.50000000000001) rectangle (161.91999999999996,-110.60000000000001);
\draw(162.01999999999998, -108.50000000000001) node[anchor=north west,align=left] {Estimates\\ on character\\ sums};
\draw (162.01999999999998, -108.50000000000001) rectangle (165.61999999999998,-110.10000000000001);
\draw(153.86999999999998, -111.20000000000002) node[anchor=north west,align=left] {Sums over\\ arbitrary\\ intervals};
\draw (153.86999999999998, -111.20000000000002) rectangle (156.71999999999997,-112.80000000000001);
\draw(156.81999999999996, -111.20000000000002) node[anchor=north west,align=left] {Weyl\\ sums};
\draw (156.81999999999996, -111.20000000000002) rectangle (158.41999999999996,-112.30000000000001);
\draw(152.86999999999998, -113.0) node[anchor=north west,align=left] {\large Additive number theory; partitions};
\draw (152.86999999999998, -113.0) rectangle (165.71999999999997,-123.3);
\draw(153.86999999999998, -114.0) node[anchor=north west,align=left] {Partition \\ identities;\\ identities\\ of \\ Rogers-Ramanujan type};
\draw (153.86999999999998, -114.0) rectangle (159.71999999999997,-116.6);
\draw(159.81999999999996, -114.0) node[anchor=north west,align=left] {Inverse \\ problems of \\ additive number\\ theory, \\ including sumsets};
\draw (159.81999999999996, -114.0) rectangle (164.66999999999996,-116.6);
\draw(153.86999999999998, -116.7) node[anchor=north west,align=left] {Goldbach-type\\ theorems; \\ other additive\\ questions \\ involving primes};
\draw (153.86999999999998, -116.7) rectangle (158.46999999999997,-119.3);
\draw(158.56999999999996, -116.7) node[anchor=north west,align=left] {Applications\\ of the\\ Hardy-Littlewood\\ method};
\draw (158.56999999999996, -116.7) rectangle (163.16999999999996,-118.8);
\draw(153.86999999999998, -119.4) node[anchor=north west,align=left] {Partitions; \\ congruences and\\ congruential\\ restrictions};
\draw (153.86999999999998, -119.4) rectangle (158.21999999999997,-121.5);
\draw(158.31999999999996, -119.4) node[anchor=north west,align=left] {Waring’s\\ problem \\ and variants};
\draw (158.31999999999996, -119.4) rectangle (161.91999999999996,-121.0);
\draw(162.01999999999998, -119.4) node[anchor=north west,align=left] {Lattice\\ points \\ in specified\\ regions};
\draw (162.01999999999998, -119.4) rectangle (165.61999999999998,-121.5);
\draw(153.86999999999998, -121.6) node[anchor=north west,align=left] {Elementary\\ theory of\\ partitions};
\draw (153.86999999999998, -121.6) rectangle (156.96999999999997,-123.19999999999999);
\draw(157.06999999999996, -121.6) node[anchor=north west,align=left] {Analytic\\ theory of\\ partitions};
\draw (157.06999999999996, -121.6) rectangle (160.16999999999996,-123.19999999999999);
\draw(166.07, -105.30000000000001) node[anchor=north west,align=left] {\large Multiplicative number theory};
\draw (166.07, -105.30000000000001) rectangle (178.97,-122.50000000000001);
\draw(167.07, -106.30000000000001) node[anchor=north west,align=left] {Distribution \\ functions \\ associated with \\ additive and \\ positive multiplicative\\ functions};
\draw (167.07, -106.30000000000001) rectangle (173.42,-109.4);
\draw(173.51999999999998, -106.30000000000001) node[anchor=north west,align=left] {Primes represented\\ by \\ polynomials; other \\ multiplicative\\ structures of\\ polynomial values};
\draw (173.51999999999998, -106.30000000000001) rectangle (178.86999999999998,-109.4);
\draw(167.07, -109.50000000000001) node[anchor=north west,align=left] {Other results \\ on the distribution\\ of values \\ or the characterization\\ of \\ arithmetic functions};
\draw (167.07, -109.50000000000001) rectangle (173.42,-112.60000000000001);
\draw(173.51999999999998, -109.50000000000001) node[anchor=north west,align=left] {Asymptotic \\ results on \\ counting functions\\ for algebraic\\ and topological\\ structures};
\draw (173.51999999999998, -109.50000000000001) rectangle (178.61999999999998,-112.60000000000001);
\draw(167.07, -112.70000000000002) node[anchor=north west,align=left] {Applications \\ of automorphic\\ functions and\\ forms to \\ multiplicative problems};
\draw (167.07, -112.70000000000002) rectangle (173.42,-115.30000000000001);
\draw(173.51999999999998, -112.70000000000002) node[anchor=north west,align=left] {Distribution\\ of integers\\ in special\\ residue classes};
\draw (173.51999999999998, -112.70000000000002) rectangle (177.86999999999998,-114.80000000000001);
\draw(167.07, -115.4) node[anchor=north west,align=left] {Distribution\\ of integers \\ with specified\\ multiplicative\\ constraints};
\draw (167.07, -115.4) rectangle (171.17,-118.0);
\draw(171.26999999999998, -115.4) node[anchor=north west,align=left] {Applications\\ of \\ sieve methods};
\draw (171.26999999999998, -115.4) rectangle (175.11999999999998,-117.0);
\draw(175.22, -115.4) node[anchor=north west,align=left] {Distribution\\ of primes};
\draw (175.22, -115.4) rectangle (178.82,-117.0);
\draw(167.07, -118.10000000000001) node[anchor=north west,align=left] {Asymptotic\\ results \\ on arithmetic\\ functions};
\draw (167.07, -118.10000000000001) rectangle (170.92,-120.2);
\draw(171.01999999999998, -118.10000000000001) node[anchor=north west,align=left] {Generalized\\ primes \\ and integers};
\draw (171.01999999999998, -118.10000000000001) rectangle (174.61999999999998,-119.7);
\draw(174.72, -118.10000000000001) node[anchor=north west,align=left] {Primes in\\ congruence\\ classes};
\draw (174.72, -118.10000000000001) rectangle (177.82,-119.7);
\draw(167.07, -120.30000000000001) node[anchor=north west,align=left] {Rate of \\ growth of\\ arithmetic\\ functions};
\draw (167.07, -120.30000000000001) rectangle (170.17,-122.4);
\draw(170.26999999999998, -120.30000000000001) node[anchor=north west,align=left] {Turán\\ theory};
\draw (170.26999999999998, -120.30000000000001) rectangle (172.36999999999998,-121.4);
\draw(172.47, -120.30000000000001) node[anchor=north west,align=left] {Sieves};
\draw (172.47, -120.30000000000001) rectangle (174.57,-121.4);
\draw(138.17, -127.2) node[anchor=north west,align=left] {\large Forms and linear algebraic groups};
\draw (138.17, -127.2) rectangle (150.76999999999998,-145.6);
\draw(139.17, -128.2) node[anchor=north west,align=left] {Analytic theory\\ (Epstein zeta\\ functions; \\ relations with\\ automorphic \\ forms and functions)};
\draw (139.17, -128.2) rectangle (144.76999999999998,-131.29999999999998);
\draw(144.86999999999998, -128.2) node[anchor=north west,align=left] {Sums of squares\\ and representations\\ by other\\ particular\\ quadratic forms};
\draw (144.86999999999998, -128.2) rectangle (150.21999999999997,-130.79999999999998);
\draw(139.17, -131.4) node[anchor=north west,align=left] {General ternary\\ and quaternary \\ quadratic forms;\\ forms of more \\ than two variables};
\draw (139.17, -131.4) rectangle (144.26999999999998,-134.0);
\draw(144.36999999999998, -131.4) node[anchor=north west,align=left] {General \\ binary quadratic\\ forms};
\draw (144.36999999999998, -131.4) rectangle (148.96999999999997,-133.0);
\draw(139.17, -134.1) node[anchor=north west,align=left] {Galois \\ cohomology of \\ linear algebraic\\ groups};
\draw (139.17, -134.1) rectangle (143.76999999999998,-136.2);
\draw(143.86999999999998, -134.1) node[anchor=north west,align=left] {Algebraic \\ theory of \\ quadratic \\ forms; Witt \\ groups and rings};
\draw (143.86999999999998, -134.1) rectangle (148.46999999999997,-136.7);
\draw(139.17, -136.8) node[anchor=north west,align=left] {Class numbers\\ of quadratic\\ and \\ Hermitian forms};
\draw (139.17, -136.8) rectangle (143.51999999999998,-138.9);
\draw(143.61999999999998, -136.8) node[anchor=north west,align=left] {\(K\)-theory\\ of quadratic\\ and \\ Hermitian forms};
\draw (143.61999999999998, -136.8) rectangle (147.96999999999997,-138.9);
\draw(139.17, -139.0) node[anchor=north west,align=left] {Quadratic\\ forms\\ over \\ general fields};
\draw (139.17, -139.0) rectangle (143.26999999999998,-141.1);
\draw(143.36999999999998, -139.0) node[anchor=north west,align=left] {Bilinear\\ and Hermitian\\ forms};
\draw (143.36999999999998, -139.0) rectangle (147.21999999999997,-140.6);
\draw(147.32, -139.0) node[anchor=north west,align=left] {Quadratic\\ forms over\\ local rings\\ and fields};
\draw (147.32, -139.0) rectangle (150.67,-141.1);
\draw(139.17, -141.2) node[anchor=north west,align=left] {Forms of \\ degree higher\\ than two};
\draw (139.17, -141.2) rectangle (143.01999999999998,-142.79999999999998);
\draw(143.11999999999998, -141.2) node[anchor=north west,align=left] {Quadratic \\ forms over \\ global rings\\ and fields};
\draw (143.11999999999998, -141.2) rectangle (146.71999999999997,-143.29999999999998);
\draw(146.82, -141.2) node[anchor=north west,align=left] {\(p\)-adic\\ theory};
\draw (146.82, -141.2) rectangle (149.92,-142.79999999999998);
\draw(139.17, -143.4) node[anchor=north west,align=left] {Forms \\ over real\\ fields};
\draw (139.17, -143.4) rectangle (142.01999999999998,-145.0);
\draw(142.11999999999998, -143.4) node[anchor=north west,align=left] {Classical\\ groups};
\draw (142.11999999999998, -143.4) rectangle (144.96999999999997,-144.5);
\draw(145.07, -143.4) node[anchor=north west,align=left] {Quadratic\\ spaces;\\ Clifford\\ algebras};
\draw (145.07, -143.4) rectangle (147.92,-145.5);
\draw(150.86999999999998, -127.2) node[anchor=north west,align=left] {\large Elementary number theory};
\draw (150.86999999999998, -127.2) rectangle (161.76999999999998,-137.5);
\draw(151.86999999999998, -128.2) node[anchor=north west,align=left] {Multiplicative\\ structure; \\ Euclidean algorithm;\\ greatest\\ common divisors};
\draw (151.86999999999998, -128.2) rectangle (157.46999999999997,-130.79999999999998);
\draw(157.56999999999996, -128.2) node[anchor=north west,align=left] {Factorization;\\ primality};
\draw (157.56999999999996, -128.2) rectangle (161.66999999999996,-129.79999999999998);
\draw(151.86999999999998, -130.9) node[anchor=north west,align=left] {Arithmetic\\ functions;\\ related \\ numbers; inversion\\ formulas};
\draw (151.86999999999998, -130.9) rectangle (156.96999999999997,-133.5);
\draw(157.06999999999996, -130.9) node[anchor=north west,align=left] {Congruences;\\ primitive\\ roots; \\ residue systems};
\draw (157.06999999999996, -130.9) rectangle (161.41999999999996,-133.0);
\draw(151.86999999999998, -133.6) node[anchor=north west,align=left] {Radix \\ representation;\\ digital\\ problems};
\draw (151.86999999999998, -133.6) rectangle (156.21999999999997,-135.7);
\draw(156.31999999999996, -133.6) node[anchor=north west,align=left] {Other \\ number \\ representations};
\draw (156.31999999999996, -133.6) rectangle (160.66999999999996,-135.2);
\draw(151.86999999999998, -135.8) node[anchor=north west,align=left] {Power \\ residues, \\ reciprocity};
\draw (151.86999999999998, -135.8) rectangle (155.21999999999997,-137.4);
\draw(155.31999999999996, -135.8) node[anchor=north west,align=left] {Continued\\ fractions};
\draw (155.31999999999996, -135.8) rectangle (158.16999999999996,-137.4);
\draw(158.26999999999998, -135.8) node[anchor=north west,align=left] {Primes};
\draw (158.26999999999998, -135.8) rectangle (160.36999999999998,-136.9);
\draw(161.86999999999998, -127.2) node[anchor=north west,align=left] {\large Geometry of numbers};
\draw (161.86999999999998, -127.2) rectangle (171.76999999999998,-137.5);
\draw(162.86999999999998, -128.2) node[anchor=north west,align=left] {Lattices and\\ convex bodies\\ (number-theoretic\\ aspects)};
\draw (162.86999999999998, -128.2) rectangle (167.71999999999997,-130.29999999999998);
\draw(167.81999999999996, -128.2) node[anchor=north west,align=left] {Relations\\ with \\ coding theory};
\draw (167.81999999999996, -128.2) rectangle (171.66999999999996,-129.79999999999998);
\draw(162.86999999999998, -130.4) node[anchor=north west,align=left] {Lattice \\ packing and \\ covering \\ (number-theoretic\\ aspects)};
\draw (162.86999999999998, -130.4) rectangle (167.71999999999997,-133.0);
\draw(167.81999999999996, -130.4) node[anchor=north west,align=left] {Automorphism\\ groups\\ of lattices};
\draw (167.81999999999996, -130.4) rectangle (171.41999999999996,-132.0);
\draw(162.86999999999998, -133.1) node[anchor=north west,align=left] {Quadratic \\ forms (reduction\\ theory,\\ extreme \\ forms, etc.)};
\draw (162.86999999999998, -133.1) rectangle (167.46999999999997,-135.7);
\draw(167.56999999999996, -133.1) node[anchor=north west,align=left] {Mean value\\ and transfer\\ theorems};
\draw (167.56999999999996, -133.1) rectangle (171.16999999999996,-134.7);
\draw(162.86999999999998, -135.8) node[anchor=north west,align=left] {Nonconvex\\ bodies};
\draw (162.86999999999998, -135.8) rectangle (165.71999999999997,-136.9);
\draw(165.81999999999996, -135.8) node[anchor=north west,align=left] {Products\\ of linear\\ forms};
\draw (165.81999999999996, -135.8) rectangle (168.66999999999996,-137.4);
\draw(168.76999999999998, -135.8) node[anchor=north west,align=left] {Minima\\ of forms};
\draw (168.76999999999998, -135.8) rectangle (171.36999999999998,-136.9);
\draw(171.86999999999998, -127.2) node[anchor=north west,align=left] {\large Diophantine equations};
\draw (171.86999999999998, -127.2) rectangle (181.01999999999998,-144.8);
\draw(172.86999999999998, -128.2) node[anchor=north west,align=left] {Higher \\ degree equations;\\ Fermat’s\\ equation};
\draw (172.86999999999998, -128.2) rectangle (177.71999999999997,-130.29999999999998);
\draw(177.81999999999996, -128.2) node[anchor=north west,align=left] {Rational\\ numbers \\ as sums of\\ fractions};
\draw (177.81999999999996, -128.2) rectangle (180.91999999999996,-130.29999999999998);
\draw(172.86999999999998, -130.4) node[anchor=north west,align=left] {Counting \\ solutions \\ of Diophantine\\ equations};
\draw (172.86999999999998, -130.4) rectangle (176.96999999999997,-132.5);
\draw(177.06999999999996, -130.4) node[anchor=north west,align=left] {\(p\)-adic\\ and \\ power \\ series fields};
\draw (177.06999999999996, -130.4) rectangle (180.91999999999996,-132.5);
\draw(172.86999999999998, -132.6) node[anchor=north west,align=left] {Multiplicative\\ and\\ norm form\\ equations};
\draw (172.86999999999998, -132.6) rectangle (176.96999999999997,-134.7);
\draw(177.06999999999996, -132.6) node[anchor=north west,align=left] {Quadratic \\ and bilinear\\ Diophantine\\ equations};
\draw (177.06999999999996, -132.6) rectangle (180.66999999999996,-134.7);
\draw(172.86999999999998, -134.8) node[anchor=north west,align=left] {Diophantine\\ equations\\ in \\ many variables};
\draw (172.86999999999998, -134.8) rectangle (176.96999999999997,-136.9);
\draw(177.06999999999996, -134.8) node[anchor=north west,align=left] {Diophantine\\ inequalities};
\draw (177.06999999999996, -134.8) rectangle (180.66999999999996,-136.4);
\draw(172.86999999999998, -137.0) node[anchor=north west,align=left] {Representation\\ problems};
\draw (172.86999999999998, -137.0) rectangle (176.96999999999997,-138.6);
\draw(177.06999999999996, -137.0) node[anchor=north west,align=left] {Linear \\ Diophantine\\ equations};
\draw (177.06999999999996, -137.0) rectangle (180.41999999999996,-138.6);
\draw(172.86999999999998, -138.7) node[anchor=north west,align=left] {Cubic and\\ quartic\\ Diophantine\\ equations};
\draw (172.86999999999998, -138.7) rectangle (176.21999999999997,-140.79999999999998);
\draw(176.31999999999996, -138.7) node[anchor=north west,align=left] {Thue-Mahler\\ equations};
\draw (176.31999999999996, -138.7) rectangle (179.66999999999996,-140.29999999999998);
\draw(172.86999999999998, -140.9) node[anchor=north west,align=left] {Exponential\\ Diophantine\\ equations};
\draw (172.86999999999998, -140.9) rectangle (176.21999999999997,-143.0);
\draw(176.31999999999996, -140.9) node[anchor=north west,align=left] {Congruences\\ in many\\ variables};
\draw (176.31999999999996, -140.9) rectangle (179.66999999999996,-142.5);
\draw(172.86999999999998, -143.1) node[anchor=north west,align=left] {The \\ Frobenius\\ problem};
\draw (172.86999999999998, -143.1) rectangle (175.71999999999997,-144.7);
\draw(186.92, -1) node[anchor=north west,align=left] {\LARGE Group theory and generalizations};
\draw (186.92, -1) rectangle (232.76999999999998,-68.6);
\draw(187.92, -2) node[anchor=north west,align=left] {\large Groupoids (i.e. small categories in which all morphisms are isomorphisms)};
\draw (187.92, -2) rectangle (211.14999999999998,-5.7);
\draw(188.92, -3) node[anchor=north west,align=left] {Groupoids (i.e.\\ small categories\\ in which\\ all morphisms\\ are isomorphisms)};
\draw (188.92, -3) rectangle (193.76999999999998,-5.6);
\draw(211.25, -2) node[anchor=north west,align=left] {\large Structure and classification of infinite or finite groups};
\draw (211.25, -2) rectangle (232.55,-11.1);
\draw(212.25, -3) node[anchor=north west,align=left] {Free products of \\ groups, free products\\ with amalgamation,\\ Higman-Neumann-Neumann\\ extensions,\\ and generalizations};
\draw (212.25, -3) rectangle (218.35,-6.1);
\draw(218.45, -3) node[anchor=north west,align=left] {Chains and \\ lattices of \\ subgroups, subnormal\\ subgroups};
\draw (218.45, -3) rectangle (224.04999999999998,-5.1);
\draw(224.15, -3) node[anchor=north west,align=left] {Extensions,\\ wreath products,\\ and other\\ compositions\\ of groups};
\draw (224.15, -3) rectangle (228.75,-5.6);
\draw(228.85, -3) node[anchor=north west,align=left] {Groups \\ with a \\ \(BN\)-pair;\\ buildings};
\draw (228.85, -3) rectangle (232.45,-5.1);
\draw(212.25, -6.2) node[anchor=north west,align=left] {Residual \\ properties and \\ generalizations;\\ residually\\ finite groups};
\draw (212.25, -6.2) rectangle (216.85,-8.8);
\draw(216.95, -6.2) node[anchor=north west,align=left] {Automorphisms\\ of \\ infinite groups};
\draw (216.95, -6.2) rectangle (221.29999999999998,-7.800000000000001);
\draw(221.4, -6.2) node[anchor=north west,align=left] {Quasivarieties\\ and\\ varieties\\ of groups};
\draw (221.4, -6.2) rectangle (225.5,-8.3);
\draw(225.6, -6.2) node[anchor=north west,align=left] {Free \\ nonabelian\\ groups};
\draw (225.6, -6.2) rectangle (228.7,-7.800000000000001);
\draw(228.8, -6.2) node[anchor=north west,align=left] {Local \\ properties\\ of groups};
\draw (228.8, -6.2) rectangle (231.9,-7.800000000000001);
\draw(212.25, -8.9) node[anchor=north west,align=left] {General \\ structure\\ theorems\\ for groups};
\draw (212.25, -8.9) rectangle (215.35,-11.0);
\draw(215.45, -8.9) node[anchor=north west,align=left] {Conjugacy\\ classes\\ for groups};
\draw (215.45, -8.9) rectangle (218.54999999999998,-10.5);
\draw(218.65, -8.9) node[anchor=north west,align=left] {Subgroup\\ theorems;\\ subgroup\\ growth};
\draw (218.65, -8.9) rectangle (221.5,-11.0);
\draw(221.6, -8.9) node[anchor=north west,align=left] {Limits,\\ profinite\\ groups};
\draw (221.6, -8.9) rectangle (224.45,-10.5);
\draw(224.55, -8.9) node[anchor=north west,align=left] {Maximal\\ subgroups};
\draw (224.55, -8.9) rectangle (227.4,-10.5);
\draw(227.5, -8.9) node[anchor=north west,align=left] {Groups\\ acting\\ on trees};
\draw (227.5, -8.9) rectangle (230.1,-10.5);
\draw(230.2, -8.9) node[anchor=north west,align=left] {Simple\\ groups};
\draw (230.2, -8.9) rectangle (232.29999999999998,-10.0);
\draw(187.92, -5.8) node[anchor=north west,align=left] {\large Connections of group theory with homological algebra and category theory};
\draw (187.92, -5.8) rectangle (210.83999999999997,-9.0);
\draw(188.92, -6.8) node[anchor=north west,align=left] {Homological\\ methods\\ in \\ group theory};
\draw (188.92, -6.8) rectangle (192.51999999999998,-8.9);
\draw(192.61999999999998, -6.8) node[anchor=north west,align=left] {Cohomology\\ of groups};
\draw (192.61999999999998, -6.8) rectangle (195.71999999999997,-8.4);
\draw(195.82, -6.8) node[anchor=north west,align=left] {Category\\ of\\ groups};
\draw (195.82, -6.8) rectangle (198.42,-8.4);
\draw(187.92, -9.1) node[anchor=north west,align=left] {\large Computational methods\\ for problems \\ pertaining to group theory};
\draw (187.92, -9.1) rectangle (196.57999999999998,-10.7);
\draw(187.92, -11.2) node[anchor=north west,align=left] {\large Special aspects of infinite or finite groups};
\draw (187.92, -11.2) rectangle (204.76999999999998,-31.8);
\draw(188.92, -12.2) node[anchor=north west,align=left] {Word problems, \\ other decision \\ problems, connections\\ with logic \\ and automata \\ (group-theoretic aspects)};
\draw (188.92, -12.2) rectangle (195.76999999999998,-15.299999999999999);
\draw(195.86999999999998, -12.2) node[anchor=north west,align=left] {Cancellation\\ theory of \\ groups; application\\ of van \\ Kampen diagrams};
\draw (195.86999999999998, -12.2) rectangle (201.21999999999997,-14.799999999999999);
\draw(201.32, -12.2) node[anchor=north west,align=left] {Groups of\\ finite \\ Morley rank};
\draw (201.32, -12.2) rectangle (204.67,-13.799999999999999);
\draw(188.92, -15.399999999999999) node[anchor=north west,align=left] {Generators,\\ relations, \\ and presentations\\ of groups};
\draw (188.92, -15.399999999999999) rectangle (193.76999999999998,-17.5);
\draw(193.86999999999998, -15.399999999999999) node[anchor=north west,align=left] {Representations\\ of groups\\ as automorphism\\ groups of\\ algebraic systems};
\draw (193.86999999999998, -15.399999999999999) rectangle (198.71999999999997,-18.0);
\draw(198.82, -15.399999999999999) node[anchor=north west,align=left] {Algebraic \\ geometry over \\ groups; equations\\ over groups};
\draw (198.82, -15.399999999999999) rectangle (203.67,-17.5);
\draw(188.92, -18.1) node[anchor=north west,align=left] {Generalizations\\ of solvable\\ and \\ nilpotent groups};
\draw (188.92, -18.1) rectangle (193.51999999999998,-20.200000000000003);
\draw(193.61999999999998, -18.1) node[anchor=north west,align=left] {Fundamental \\ groups and their\\ automorphisms\\ (group-theoretic\\ aspects)};
\draw (193.61999999999998, -18.1) rectangle (198.21999999999997,-20.700000000000003);
\draw(198.32, -18.1) node[anchor=north west,align=left] {Reflection\\ and Coxeter\\ groups \\ (group-theoretic\\ aspects)};
\draw (198.32, -18.1) rectangle (202.92,-20.700000000000003);
\draw(188.92, -20.8) node[anchor=north west,align=left] {Ordered \\ groups \\ (group-theoretic\\ aspects)};
\draw (188.92, -20.8) rectangle (193.51999999999998,-22.900000000000002);
\draw(193.61999999999998, -20.8) node[anchor=north west,align=left] {Derived series,\\ central\\ series, and\\ generalizations\\ for groups};
\draw (193.61999999999998, -20.8) rectangle (197.96999999999997,-23.400000000000002);
\draw(198.07, -20.8) node[anchor=north west,align=left] {Other classes\\ of groups\\ defined by \\ subgroup chains};
\draw (198.07, -20.8) rectangle (202.42,-22.900000000000002);
\draw(188.92, -23.5) node[anchor=north west,align=left] {FC-groups\\ and \\ their \\ generalizations};
\draw (188.92, -23.5) rectangle (193.26999999999998,-25.6);
\draw(193.36999999999998, -23.5) node[anchor=north west,align=left] {Solvable\\ groups, \\ supersolvable\\ groups};
\draw (193.36999999999998, -23.5) rectangle (197.21999999999997,-25.6);
\draw(197.32, -23.5) node[anchor=north west,align=left] {Periodic\\ groups; \\ locally \\ finite groups};
\draw (197.32, -23.5) rectangle (201.17,-25.6);
\draw(201.26999999999998, -23.5) node[anchor=north west,align=left] {Associated\\ Lie \\ structures \\ for groups};
\draw (201.26999999999998, -23.5) rectangle (204.61999999999998,-25.6);
\draw(188.92, -25.7) node[anchor=north west,align=left] {Hyperbolic \\ groups and \\ nonpositively\\ curved groups};
\draw (188.92, -25.7) rectangle (192.76999999999998,-27.8);
\draw(192.86999999999998, -25.7) node[anchor=north west,align=left] {Automorphism\\ groups\\ of groups};
\draw (192.86999999999998, -25.7) rectangle (196.46999999999997,-27.3);
\draw(196.57, -25.7) node[anchor=north west,align=left] {Braid \\ groups; \\ Artin groups};
\draw (196.57, -25.7) rectangle (200.17,-27.3);
\draw(200.26999999999998, -25.7) node[anchor=north west,align=left] {Other groups\\ related \\ to topology\\ or analysis};
\draw (200.26999999999998, -25.7) rectangle (203.86999999999998,-27.8);
\draw(188.92, -27.9) node[anchor=north west,align=left] {Commutator\\ calculus};
\draw (188.92, -27.9) rectangle (192.01999999999998,-29.5);
\draw(192.11999999999998, -27.9) node[anchor=north west,align=left] {Formations\\ of groups,\\ Fitting\\ classes};
\draw (192.11999999999998, -27.9) rectangle (195.21999999999997,-30.0);
\draw(195.32, -27.9) node[anchor=north west,align=left] {Engel \\ conditions};
\draw (195.32, -27.9) rectangle (198.42,-29.0);
\draw(198.51999999999998, -27.9) node[anchor=north west,align=left] {Asymptotic\\ properties\\ of groups};
\draw (198.51999999999998, -27.9) rectangle (201.61999999999998,-29.5);
\draw(201.72, -27.9) node[anchor=north west,align=left] {Nilpotent\\ groups};
\draw (201.72, -27.9) rectangle (204.57,-29.0);
\draw(188.92, -30.099999999999998) node[anchor=north west,align=left] {Geometric\\ group\\ theory};
\draw (188.92, -30.099999999999998) rectangle (191.76999999999998,-31.7);
\draw(204.87, -11.2) node[anchor=north west,align=left] {\large Linear algebraic groups and related topics};
\draw (204.87, -11.2) rectangle (220.97,-24.2);
\draw(205.87, -12.2) node[anchor=north west,align=left] {Quantum groups\\ (quantized \\ function algebras)\\ and their \\ representations};
\draw (205.87, -12.2) rectangle (210.97,-14.799999999999999);
\draw(211.07, -12.2) node[anchor=north west,align=left] {Linear algebraic\\ groups\\ over adèles\\ and other \\ rings and schemes};
\draw (211.07, -12.2) rectangle (215.92,-14.799999999999999);
\draw(216.02, -12.2) node[anchor=north west,align=left] {Exceptionalgroups};
\draw (216.02, -12.2) rectangle (220.87,-13.799999999999999);
\draw(205.87, -14.899999999999999) node[anchor=north west,align=left] {Representation\\ theory for\\ linear \\ algebraic groups};
\draw (205.87, -14.899999999999999) rectangle (210.47,-17.0);
\draw(210.57, -14.899999999999999) node[anchor=north west,align=left] {Structure \\ theory for \\ linear algebraic\\ groups};
\draw (210.57, -14.899999999999999) rectangle (215.17,-17.0);
\draw(215.27, -14.899999999999999) node[anchor=north west,align=left] {Cohomology\\ theory for\\ linear \\ algebraic groups};
\draw (215.27, -14.899999999999999) rectangle (219.87,-17.0);
\draw(205.87, -17.1) node[anchor=north west,align=left] {Linear \\ algebraic groups\\ over \\ arbitrary fields};
\draw (205.87, -17.1) rectangle (210.47,-19.200000000000003);
\draw(210.57, -17.1) node[anchor=north west,align=left] {Linear algebraic\\ groups over\\ the reals, the\\ complexes, \\ the quaternions};
\draw (210.57, -17.1) rectangle (215.17,-19.700000000000003);
\draw(215.27, -17.1) node[anchor=north west,align=left] {Linear algebraic\\ groups\\ over local \\ fields and \\ their integers};
\draw (215.27, -17.1) rectangle (219.87,-19.700000000000003);
\draw(205.87, -19.8) node[anchor=north west,align=left] {Linear algebraic\\ groups\\ over global\\ fields and \\ their integers};
\draw (205.87, -19.8) rectangle (210.47,-22.400000000000002);
\draw(210.57, -19.8) node[anchor=north west,align=left] {Linear \\ algebraic \\ groups over \\ finite fields};
\draw (210.57, -19.8) rectangle (214.42,-21.900000000000002);
\draw(214.52, -19.8) node[anchor=north west,align=left] {Applications\\ of linear\\ algebraic\\ groups to \\ the sciences};
\draw (214.52, -19.8) rectangle (218.12,-22.400000000000002);
\draw(205.87, -22.5) node[anchor=north west,align=left] {Schur and\\ \(q\)-Schur\\ algebras};
\draw (205.87, -22.5) rectangle (209.22,-24.1);
\draw(209.32, -22.5) node[anchor=north west,align=left] {Kac-Moody\\ groups};
\draw (209.32, -22.5) rectangle (212.17,-23.6);
\draw(204.87, -24.3) node[anchor=north west,align=left] {\large Probabilistic methods in group theory};
\draw (204.87, -24.3) rectangle (216.94,-27.5);
\draw(205.87, -25.3) node[anchor=north west,align=left] {Probabilistic\\ methods\\ in \\ group theory};
\draw (205.87, -25.3) rectangle (209.72,-27.400000000000002);
\draw(204.87, -27.6) node[anchor=north west,align=left] {\large History of\\ group theory};
\draw (204.87, -27.6) rectangle (209.19,-28.700000000000003);
\draw(221.07, -11.2) node[anchor=north west,align=left] {\large Other generalizations of groups};
\draw (221.07, -11.2) rectangle (232.67,-17.299999999999997);
\draw(222.07, -12.2) node[anchor=north west,align=left] {Sets with a\\ single \\ binary operation\\ (groupoids)};
\draw (222.07, -12.2) rectangle (226.67,-14.299999999999999);
\draw(226.76999999999998, -12.2) node[anchor=north west,align=left] {Ternary systems\\ (heaps,\\ semiheaps, \\ heapoids, etc.)};
\draw (226.76999999999998, -12.2) rectangle (231.11999999999998,-14.299999999999999);
\draw(222.07, -14.399999999999999) node[anchor=north west,align=left] {\(n\)-ary\\ systems \\ \((n\ge~3)\)};
\draw (222.07, -14.399999999999999) rectangle (225.67,-15.999999999999998);
\draw(225.76999999999998, -14.399999999999999) node[anchor=north west,align=left] {Loops,\\ quasigroups};
\draw (225.76999999999998, -14.399999999999999) rectangle (229.11999999999998,-15.999999999999998);
\draw(229.22, -14.399999999999999) node[anchor=north west,align=left] {Hypergroups};
\draw (229.22, -14.399999999999999) rectangle (232.57,-15.499999999999998);
\draw(222.07, -16.1) node[anchor=north west,align=left] {Fuzzy\\ groups};
\draw (222.07, -16.1) rectangle (224.17,-17.200000000000003);
\draw(221.07, -17.4) node[anchor=north west,align=left] {\large Foundations};
\draw (221.07, -17.4) rectangle (229.97,-22.799999999999997);
\draw(222.07, -18.4) node[anchor=north west,align=left] {Metamathematical\\ considerations \\ in group theory};
\draw (222.07, -18.4) rectangle (226.67,-20.5);
\draw(222.07, -20.599999999999998) node[anchor=north west,align=left] {Axiomatics\\ and elementary\\ properties\\ of groups};
\draw (222.07, -20.599999999999998) rectangle (226.17,-22.7);
\draw(226.26999999999998, -20.599999999999998) node[anchor=north west,align=left] {Applications\\ of \\ logic to \\ group theory};
\draw (226.26999999999998, -20.599999999999998) rectangle (229.86999999999998,-22.7);
\draw(187.92, -31.900000000000002) node[anchor=north west,align=left] {\large Representation theory of groups};
\draw (187.92, -31.900000000000002) rectangle (201.57,-49.300000000000004);
\draw(188.92, -32.900000000000006) node[anchor=north west,align=left] {Group rings of\\ infinite groups\\ and their \\ modules \\ (group-theoretic aspects)};
\draw (188.92, -32.900000000000006) rectangle (195.76999999999998,-35.50000000000001);
\draw(195.86999999999998, -32.900000000000006) node[anchor=north west,align=left] {Representations\\ of \\ infinite symmetric\\ groups};
\draw (195.86999999999998, -32.900000000000006) rectangle (200.96999999999997,-35.00000000000001);
\draw(188.92, -35.6) node[anchor=north west,align=left] {Group rings of\\ finite groups\\ and their \\ modules (group-theoretic\\ aspects)};
\draw (188.92, -35.6) rectangle (195.51999999999998,-38.2);
\draw(195.61999999999998, -35.6) node[anchor=north west,align=left] {Applications of\\ group representations\\ to physics\\ and other \\ areas of science};
\draw (195.61999999999998, -35.6) rectangle (201.46999999999997,-38.2);
\draw(188.92, -38.300000000000004) node[anchor=north west,align=left] {Representationsof\\ sporadic\\ groups};
\draw (188.92, -38.300000000000004) rectangle (193.76999999999998,-40.400000000000006);
\draw(193.86999999999998, -38.300000000000004) node[anchor=north west,align=left] {Representations\\ of \\ finite \\ symmetric groups};
\draw (193.86999999999998, -38.300000000000004) rectangle (198.46999999999997,-40.400000000000006);
\draw(188.92, -40.5) node[anchor=north west,align=left] {Hecke \\ algebras and \\ their \\ representations};
\draw (188.92, -40.5) rectangle (193.26999999999998,-42.6);
\draw(193.36999999999998, -40.5) node[anchor=north west,align=left] {Integral \\ representations\\ of \\ finite groups};
\draw (193.36999999999998, -40.5) rectangle (197.71999999999997,-42.6);
\draw(188.92, -42.7) node[anchor=north west,align=left] {\(p\)-adic\\ representations\\ of\\ finite groups};
\draw (188.92, -42.7) rectangle (193.26999999999998,-44.800000000000004);
\draw(193.36999999999998, -42.7) node[anchor=north west,align=left] {Integral \\ representations\\ of \\ infinite groups};
\draw (193.36999999999998, -42.7) rectangle (197.71999999999997,-44.800000000000004);
\draw(188.92, -44.900000000000006) node[anchor=north west,align=left] {Ordinary\\ representations\\ and\\ characters};
\draw (188.92, -44.900000000000006) rectangle (193.26999999999998,-47.00000000000001);
\draw(193.36999999999998, -44.900000000000006) node[anchor=north west,align=left] {Modular \\ representations\\ and\\ characters};
\draw (193.36999999999998, -44.900000000000006) rectangle (197.71999999999997,-47.00000000000001);
\draw(188.92, -47.1) node[anchor=north west,align=left] {Projective\\ representations\\ and\\ multipliers};
\draw (188.92, -47.1) rectangle (193.26999999999998,-49.2);
\draw(193.36999999999998, -47.1) node[anchor=north west,align=left] {Representations\\ of \\ finite groups\\ of Lie type};
\draw (193.36999999999998, -47.1) rectangle (197.71999999999997,-49.2);
\draw(201.67, -31.900000000000002) node[anchor=north west,align=left] {\large Permutation groups};
\draw (201.67, -31.900000000000002) rectangle (213.07,-43.400000000000006);
\draw(202.67, -32.900000000000006) node[anchor=north west,align=left] {Finite automorphism\\ groups of\\ algebraic, \\ geometric, or \\ combinatorial structures};
\draw (202.67, -32.900000000000006) rectangle (209.26999999999998,-35.50000000000001);
\draw(209.36999999999998, -32.900000000000006) node[anchor=north west,align=left] {Infinite\\ automorphism\\ groups};
\draw (209.36999999999998, -32.900000000000006) rectangle (212.96999999999997,-34.50000000000001);
\draw(202.67, -35.6) node[anchor=north west,align=left] {General \\ theory for \\ finite permutation\\ groups};
\draw (202.67, -35.6) rectangle (207.76999999999998,-37.7);
\draw(207.86999999999998, -35.6) node[anchor=north west,align=left] {General \\ theory for \\ infinite \\ permutation groups};
\draw (207.86999999999998, -35.6) rectangle (212.96999999999997,-37.7);
\draw(202.67, -37.800000000000004) node[anchor=north west,align=left] {Characterization\\ theorems\\ for permutation\\ groups};
\draw (202.67, -37.800000000000004) rectangle (207.26999999999998,-39.900000000000006);
\draw(207.36999999999998, -37.800000000000004) node[anchor=north west,align=left] {Multiply\\ transitive\\ infinite groups};
\draw (207.36999999999998, -37.800000000000004) rectangle (211.71999999999997,-39.900000000000006);
\draw(202.67, -40.0) node[anchor=north west,align=left] {Subgroups\\ of symmetric\\ groups};
\draw (202.67, -40.0) rectangle (206.26999999999998,-41.6);
\draw(206.36999999999998, -40.0) node[anchor=north west,align=left] {Multiply\\ transitive\\ finite\\ groups};
\draw (206.36999999999998, -40.0) rectangle (209.46999999999997,-42.1);
\draw(209.57, -40.0) node[anchor=north west,align=left] {Primitive\\ groups};
\draw (209.57, -40.0) rectangle (212.42,-41.1);
\draw(202.67, -42.2) node[anchor=north west,align=left] {Symmetric\\ groups};
\draw (202.67, -42.2) rectangle (205.51999999999998,-43.300000000000004);
\draw(213.17, -31.900000000000002) node[anchor=north west,align=left] {\large Abstract finite groups};
\draw (213.17, -31.900000000000002) rectangle (224.32,-48.10000000000001);
\draw(214.17, -32.900000000000006) node[anchor=north west,align=left] {Finite solvable\\ groups, theory \\ of formations, \\ Schunck classes, \\ Fitting classes,\\ \(\pi\)-length, ranks};
\draw (214.17, -32.900000000000006) rectangle (220.01999999999998,-36.00000000000001);
\draw(220.11999999999998, -32.900000000000006) node[anchor=north west,align=left] {Finite simple\\ groups \\ and their \\ classification};
\draw (220.11999999999998, -32.900000000000006) rectangle (224.21999999999997,-35.00000000000001);
\draw(214.17, -36.1) node[anchor=north west,align=left] {Arithmetic and\\ combinatorial\\ problems \\ involving abstract\\ finite groups};
\draw (214.17, -36.1) rectangle (219.26999999999998,-38.7);
\draw(219.36999999999998, -36.1) node[anchor=north west,align=left] {Sylow subgroups,\\ Sylow \\ properties, \\ \(\pi\)-groups, \\ \(\pi\)-structure};
\draw (219.36999999999998, -36.1) rectangle (224.21999999999997,-38.7);
\draw(214.17, -38.800000000000004) node[anchor=north west,align=left] {Simple \\ groups: sporadic\\ groups};
\draw (214.17, -38.800000000000004) rectangle (218.76999999999998,-40.400000000000006);
\draw(218.86999999999998, -38.800000000000004) node[anchor=north west,align=left] {Simple groups:\\ alternating\\ groups\\ and groups\\ of Lie type};
\draw (218.86999999999998, -38.800000000000004) rectangle (222.96999999999997,-41.400000000000006);
\draw(214.17, -41.5) node[anchor=north west,align=left] {Special \\ subgroups \\ (Frattini, \\ Fitting, etc.)};
\draw (214.17, -41.5) rectangle (218.26999999999998,-43.6);
\draw(218.36999999999998, -41.5) node[anchor=north west,align=left] {Subnormal \\ subgroups of\\ abstract \\ finite groups};
\draw (218.36999999999998, -41.5) rectangle (222.21999999999997,-43.6);
\draw(214.17, -43.7) node[anchor=north west,align=left] {Products of\\ subgroups\\ of abstract\\ finite groups};
\draw (214.17, -43.7) rectangle (218.01999999999998,-45.800000000000004);
\draw(218.11999999999998, -43.7) node[anchor=north west,align=left] {Automorphisms\\ of \\ abstract \\ finite groups};
\draw (218.11999999999998, -43.7) rectangle (221.96999999999997,-45.800000000000004);
\draw(214.17, -45.900000000000006) node[anchor=north west,align=left] {Finite \\ nilpotent \\ groups, \\ \(p\)-groups};
\draw (214.17, -45.900000000000006) rectangle (217.76999999999998,-48.00000000000001);
\draw(217.86999999999998, -45.900000000000006) node[anchor=north west,align=left] {Series and\\ lattices\\ of subgroups};
\draw (217.86999999999998, -45.900000000000006) rectangle (221.46999999999997,-47.50000000000001);
\draw(187.92, -49.400000000000006) node[anchor=north west,align=left] {\large Semigroups};
\draw (187.92, -49.400000000000006) rectangle (198.57,-68.5);
\draw(188.92, -50.400000000000006) node[anchor=north west,align=left] {Semigroup \\ rings, \\ multiplicative \\ semigroups of rings};
\draw (188.92, -50.400000000000006) rectangle (194.26999999999998,-52.50000000000001);
\draw(194.36999999999998, -50.400000000000006) node[anchor=north west,align=left] {Ideal \\ theory for \\ semigroups};
\draw (194.36999999999998, -50.400000000000006) rectangle (197.71999999999997,-52.00000000000001);
\draw(188.92, -52.60000000000001) node[anchor=north west,align=left] {Representation\\ of \\ semigroups; \\ actions of \\ semigroups on sets};
\draw (188.92, -52.60000000000001) rectangle (194.01999999999998,-55.20000000000001);
\draw(194.11999999999998, -52.60000000000001) node[anchor=north west,align=left] {Varieties\\ and \\ pseudovarieties\\ of semigroups};
\draw (194.11999999999998, -52.60000000000001) rectangle (198.46999999999997,-54.70000000000001);
\draw(188.92, -55.300000000000004) node[anchor=north west,align=left] {Semigroups\\ of \\ transformations, \\ relations, \\ partitions, etc.};
\draw (188.92, -55.300000000000004) rectangle (193.76999999999998,-57.900000000000006);
\draw(193.86999999999998, -55.300000000000004) node[anchor=north west,align=left] {Free semigroups,\\ generators and \\ relations, \\ word problems};
\draw (193.86999999999998, -55.300000000000004) rectangle (198.46999999999997,-57.900000000000006);
\draw(188.92, -58.00000000000001) node[anchor=north west,align=left] {Semigroups \\ in automata \\ theory, \\ linguistics, etc.};
\draw (188.92, -58.00000000000001) rectangle (193.76999999999998,-60.10000000000001);
\draw(193.86999999999998, -58.00000000000001) node[anchor=north west,align=left] {Algebraicmonoids};
\draw (193.86999999999998, -58.00000000000001) rectangle (198.46999999999997,-59.60000000000001);
\draw(188.92, -60.2) node[anchor=north west,align=left] {Connections of\\ semigroups \\ with homological\\ algebra and \\ category theory};
\draw (188.92, -60.2) rectangle (193.51999999999998,-62.800000000000004);
\draw(193.61999999999998, -60.2) node[anchor=north west,align=left] {Generalizations\\ of\\ semigroups};
\draw (193.61999999999998, -60.2) rectangle (197.96999999999997,-61.800000000000004);
\draw(188.92, -62.900000000000006) node[anchor=north west,align=left] {Commutative\\ semigroups};
\draw (188.92, -62.900000000000006) rectangle (192.26999999999998,-64.5);
\draw(192.36999999999998, -62.900000000000006) node[anchor=north west,align=left] {General \\ structure\\ theory for\\ semigroups};
\draw (192.36999999999998, -62.900000000000006) rectangle (195.46999999999997,-65.0);
\draw(188.92, -65.10000000000001) node[anchor=north west,align=left] {Radical \\ theory for\\ semigroups};
\draw (188.92, -65.10000000000001) rectangle (192.01999999999998,-66.7);
\draw(192.11999999999998, -65.10000000000001) node[anchor=north west,align=left] {Arithmetic\\ theory of\\ semigroups};
\draw (192.11999999999998, -65.10000000000001) rectangle (195.21999999999997,-66.7);
\draw(195.32, -65.10000000000001) node[anchor=north west,align=left] {Mappings\\ of \\ semigroups};
\draw (195.32, -65.10000000000001) rectangle (198.42,-66.7);
\draw(188.92, -66.80000000000001) node[anchor=north west,align=left] {Regular\\ semigroups};
\draw (188.92, -66.80000000000001) rectangle (192.01999999999998,-68.4);
\draw(192.11999999999998, -66.80000000000001) node[anchor=north west,align=left] {Inverse\\ semigroups};
\draw (192.11999999999998, -66.80000000000001) rectangle (195.21999999999997,-68.4);
\draw(195.32, -66.80000000000001) node[anchor=north west,align=left] {Orthodox\\ semigroups};
\draw (195.32, -66.80000000000001) rectangle (198.42,-68.4);
\draw(198.67, -49.400000000000006) node[anchor=north west,align=left] {\large Other groups of matrices};
\draw (198.67, -49.400000000000006) rectangle (208.57,-58.50000000000001);
\draw(199.67, -50.400000000000006) node[anchor=north west,align=left] {Unimodular \\ groups, congruence\\ subgroups\\ (group-theoretic\\ aspects)};
\draw (199.67, -50.400000000000006) rectangle (204.76999999999998,-53.00000000000001);
\draw(204.86999999999998, -50.400000000000006) node[anchor=north west,align=left] {Other matrix\\ groups\\ over fields};
\draw (204.86999999999998, -50.400000000000006) rectangle (208.46999999999997,-52.00000000000001);
\draw(199.67, -53.10000000000001) node[anchor=north west,align=left] {Fuchsian groups\\ and their\\ generalizations\\ (group-theoretic\\ aspects)};
\draw (199.67, -53.10000000000001) rectangle (204.26999999999998,-55.70000000000001);
\draw(204.36999999999998, -53.10000000000001) node[anchor=north west,align=left] {Other \\ matrix groups\\ over \\ finite fields};
\draw (204.36999999999998, -53.10000000000001) rectangle (208.21999999999997,-55.20000000000001);
\draw(199.67, -55.800000000000004) node[anchor=north west,align=left] {Other geometric\\ groups,\\ including\\ crystallographic\\ groups};
\draw (199.67, -55.800000000000004) rectangle (204.26999999999998,-58.400000000000006);
\draw(204.36999999999998, -55.800000000000004) node[anchor=north west,align=left] {Other matrix\\ groups\\ over rings};
\draw (204.36999999999998, -55.800000000000004) rectangle (207.96999999999997,-57.400000000000006);
\draw(208.67, -49.400000000000006) node[anchor=north west,align=left] {\large Abelian groups};
\draw (208.67, -49.400000000000006) rectangle (217.82,-64.10000000000001);
\draw(209.67, -50.400000000000006) node[anchor=north west,align=left] {Direct sums,\\ direct products,\\ etc. for\\ abelian groups};
\draw (209.67, -50.400000000000006) rectangle (214.26999999999998,-52.50000000000001);
\draw(214.36999999999998, -50.400000000000006) node[anchor=north west,align=left] {Subgroups\\ of abelian\\ groups};
\draw (214.36999999999998, -50.400000000000006) rectangle (217.46999999999997,-52.00000000000001);
\draw(209.67, -52.60000000000001) node[anchor=north west,align=left] {Torsion groups,\\ primary\\ groups and \\ generalized \\ primary groups};
\draw (209.67, -52.60000000000001) rectangle (214.01999999999998,-55.20000000000001);
\draw(214.11999999999998, -52.60000000000001) node[anchor=north west,align=left] {Torsion-free\\ groups,\\ finite rank};
\draw (214.11999999999998, -52.60000000000001) rectangle (217.71999999999997,-54.20000000000001);
\draw(209.67, -55.300000000000004) node[anchor=north west,align=left] {Automorphisms,\\ homomorphisms,\\ endomorphisms,\\ etc. for\\ abelian groups};
\draw (209.67, -55.300000000000004) rectangle (213.76999999999998,-57.900000000000006);
\draw(213.86999999999998, -55.300000000000004) node[anchor=north west,align=left] {Torsion-free\\ groups, \\ infinite rank};
\draw (213.86999999999998, -55.300000000000004) rectangle (217.71999999999997,-57.400000000000006);
\draw(209.67, -58.00000000000001) node[anchor=north west,align=left] {Homological\\ and \\ categorical \\ methods for\\ abelian groups};
\draw (209.67, -58.00000000000001) rectangle (213.76999999999998,-60.60000000000001);
\draw(213.86999999999998, -58.00000000000001) node[anchor=north west,align=left] {Extensions\\ of abelian\\ groups};
\draw (213.86999999999998, -58.00000000000001) rectangle (216.96999999999997,-59.60000000000001);
\draw(209.67, -60.7) node[anchor=north west,align=left] {Topological\\ methods\\ for \\ abelian groups};
\draw (209.67, -60.7) rectangle (213.76999999999998,-62.800000000000004);
\draw(213.86999999999998, -60.7) node[anchor=north west,align=left] {Finite\\ abelian\\ groups};
\draw (213.86999999999998, -60.7) rectangle (216.21999999999997,-62.300000000000004);
\draw(209.67, -62.900000000000006) node[anchor=north west,align=left] {Mixed\\ groups};
\draw (209.67, -62.900000000000006) rectangle (211.76999999999998,-64.0);
\draw(186.92, -68.69999999999999) node[anchor=north west,align=left] {\LARGE Category theory; homological algebra};
\draw (186.92, -68.69999999999999) rectangle (229.36999999999998,-130.5);
\draw(187.92, -69.69999999999999) node[anchor=north west,align=left] {\large Homological algebra in category theory, derived categories and functors};
\draw (187.92, -69.69999999999999) rectangle (214.42,-78.79999999999998);
\draw(188.92, -70.69999999999999) node[anchor=north west,align=left] {\(A_{\infty}\)-categories,\\ relations \\ with homological\\ mirror symmetry};
\draw (188.92, -70.69999999999999) rectangle (196.01999999999998,-73.29999999999998);
\draw(196.11999999999998, -70.69999999999999) node[anchor=north west,align=left] {Relative homological\\ algebra,\\ projective classes\\ (category-theoretic\\ aspects)};
\draw (196.11999999999998, -70.69999999999999) rectangle (201.71999999999997,-73.29999999999998);
\draw(201.82, -70.69999999999999) node[anchor=north west,align=left] {Projectives\\ and injectives\\ (category-theoretic\\ aspects)};
\draw (201.82, -70.69999999999999) rectangle (207.17,-73.29999999999998);
\draw(207.26999999999998, -70.69999999999999) node[anchor=north west,align=left] {Resolutions;\\ derived \\ functors \\ (category-theoretic\\ aspects)};
\draw (207.26999999999998, -70.69999999999999) rectangle (212.61999999999998,-73.29999999999998);
\draw(188.92, -73.39999999999999) node[anchor=north west,align=left] {Ext and Tor, \\ generalizations,\\ Künneth formula\\ (category-theoretic\\ aspects)};
\draw (188.92, -73.39999999999999) rectangle (194.26999999999998,-75.99999999999999);
\draw(194.36999999999998, -73.39999999999999) node[anchor=north west,align=left] {Homological\\ dimension \\ (category-theoretic\\ aspects)};
\draw (194.36999999999998, -73.39999999999999) rectangle (199.71999999999997,-75.49999999999999);
\draw(199.82, -73.39999999999999) node[anchor=north west,align=left] {Chain complexes\\ (category-theoretic\\ aspects),\\ dg categories};
\draw (199.82, -73.39999999999999) rectangle (205.17,-75.99999999999999);
\draw(205.26999999999998, -73.39999999999999) node[anchor=north west,align=left] {Nonabelian\\ homological\\ algebra \\ (category-theoretic\\ aspects)};
\draw (205.26999999999998, -73.39999999999999) rectangle (210.61999999999998,-75.99999999999999);
\draw(210.72, -73.39999999999999) node[anchor=north west,align=left] {Derived \\ categories,\\ triangulated\\ categories};
\draw (210.72, -73.39999999999999) rectangle (214.32,-75.49999999999999);
\draw(188.92, -76.1) node[anchor=north west,align=left] {Other \\ (co)homology \\ theories \\ (category-theoretic\\ aspects)};
\draw (188.92, -76.1) rectangle (194.26999999999998,-78.69999999999999);
\draw(194.36999999999998, -76.1) node[anchor=north west,align=left] {2-groups, \\ crossed \\ modules, crossed\\ complexes};
\draw (194.36999999999998, -76.1) rectangle (198.96999999999997,-78.19999999999999);
\draw(199.07, -76.1) node[anchor=north west,align=left] {Spectral\\ sequences,\\ hypercohomology};
\draw (199.07, -76.1) rectangle (203.42,-78.19999999999999);
\draw(203.51999999999998, -76.1) node[anchor=north west,align=left] {Simplicial\\ modules and\\ Dold-Kan \\ correspondence};
\draw (203.51999999999998, -76.1) rectangle (207.61999999999998,-78.19999999999999);
\draw(207.71999999999997, -76.1) node[anchor=north west,align=left] {Stable \\ module \\ categories};
\draw (207.71999999999997, -76.1) rectangle (210.81999999999996,-77.69999999999999);
\draw(210.92, -76.1) node[anchor=north west,align=left] {Graph \\ complexes\\ and graph\\ homology};
\draw (210.92, -76.1) rectangle (213.76999999999998,-78.19999999999999);
\draw(214.51999999999998, -69.69999999999999) node[anchor=north west,align=left] {\large Categories in geometry and topology};
\draw (214.51999999999998, -69.69999999999999) rectangle (228.67,-83.19999999999999);
\draw(215.51999999999998, -70.69999999999999) node[anchor=north west,align=left] {Presheaves and\\ sheaves, stacks,\\ descent \\ conditions \\ (category-theoretic aspects)};
\draw (215.51999999999998, -70.69999999999999) rectangle (223.11999999999998,-73.29999999999998);
\draw(223.21999999999997, -70.69999999999999) node[anchor=north west,align=left] {Abstract \\ manifolds and fiber\\ bundles \\ (category-theoretic\\ aspects)};
\draw (223.21999999999997, -70.69999999999999) rectangle (228.56999999999996,-73.29999999999998);
\draw(215.51999999999998, -73.39999999999999) node[anchor=north west,align=left] {Synthetic \\ differential geometry,\\ tangent \\ categories, differential\\ categories};
\draw (215.51999999999998, -73.39999999999999) rectangle (222.11999999999998,-75.99999999999999);
\draw(222.21999999999997, -73.39999999999999) node[anchor=north west,align=left] {Algebraic \\ \(K\)-theory and\\ \(L\)-theory\\ (category-theoretic\\ aspects)};
\draw (222.21999999999997, -73.39999999999999) rectangle (227.56999999999996,-75.99999999999999);
\draw(215.51999999999998, -76.1) node[anchor=north west,align=left] {Grothendieck\\ groups \\ (category-theoretic\\ aspects)};
\draw (215.51999999999998, -76.1) rectangle (220.86999999999998,-78.19999999999999);
\draw(220.96999999999997, -76.1) node[anchor=north west,align=left] {Grothendieck\\ topologies\\ and \\ Grothendieck topoi};
\draw (220.96999999999997, -76.1) rectangle (226.06999999999996,-78.19999999999999);
\draw(215.51999999999998, -78.29999999999998) node[anchor=north west,align=left] {Frames and \\ locales, pointfree\\ topology,\\ Stone duality};
\draw (215.51999999999998, -78.29999999999998) rectangle (220.61999999999998,-80.39999999999998);
\draw(220.71999999999997, -78.29999999999998) node[anchor=north west,align=left] {Categories\\ of topological\\ spaces\\ and continuous\\ mappings};
\draw (220.71999999999997, -78.29999999999998) rectangle (224.81999999999996,-80.89999999999998);
\draw(224.92, -78.29999999999998) node[anchor=north west,align=left] {Local \\ categories \\ and functors};
\draw (224.92, -78.29999999999998) rectangle (228.51999999999998,-79.89999999999998);
\draw(215.51999999999998, -80.99999999999999) node[anchor=north west,align=left] {Goodwillie\\ calculus\\ and functor\\ calculus};
\draw (215.51999999999998, -80.99999999999999) rectangle (218.86999999999998,-83.09999999999998);
\draw(218.96999999999997, -80.99999999999999) node[anchor=north west,align=left] {Quantales};
\draw (218.96999999999997, -80.99999999999999) rectangle (221.81999999999996,-82.09999999999998);
\draw(187.92, -78.89999999999999) node[anchor=north west,align=left] {\large Computational methods\\ for problems pertaining\\ to category theory};
\draw (187.92, -78.89999999999999) rectangle (195.64999999999998,-80.49999999999999);
\draw(187.92, -80.6) node[anchor=north west,align=left] {\large History of \\ category theory};
\draw (187.92, -80.6) rectangle (193.17,-81.69999999999999);
\draw(187.92, -83.29999999999998) node[anchor=north west,align=left] {\large Higher categories and homotopical algebra};
\draw (187.92, -83.29999999999998) rectangle (203.57,-97.79999999999998);
\draw(188.92, -84.29999999999998) node[anchor=north west,align=left] {\((\infty,~n)\)-categories\\ and \\ \((\infty,\infty)\)-categories};
\draw (188.92, -84.29999999999998) rectangle (197.01999999999998,-86.89999999999998);
\draw(197.11999999999998, -84.29999999999998) node[anchor=north west,align=left] {Categories of \\ fibrations, \\ relations to \\ \(K\)-theory, relations\\ to type theory};
\draw (197.11999999999998, -84.29999999999998) rectangle (203.46999999999997,-86.89999999999998);
\draw(188.92, -86.99999999999999) node[anchor=north west,align=left] {\((\infty,1)\)-categories\\ (quasi-categories,\\ Segal \\ spaces, etc.); \\ \(\infty\)-topoi, stable\\ \(\infty\)-categories};
\draw (188.92, -86.99999999999999) rectangle (195.76999999999998,-90.09999999999998);
\draw(195.86999999999998, -86.99999999999999) node[anchor=north west,align=left] {Localizations\\ (e.g., \\ simplicial localization,\\ Bousfield\\ localization)};
\draw (195.86999999999998, -86.99999999999999) rectangle (202.46999999999997,-89.59999999999998);
\draw(188.92, -90.19999999999999) node[anchor=north west,align=left] {Tricategories,\\ weak \\ \(n\)-categories, \\ coherence, \\ semi-strictification};
\draw (188.92, -90.19999999999999) rectangle (194.51999999999998,-92.79999999999998);
\draw(194.61999999999998, -90.19999999999999) node[anchor=north west,align=left] {\(\infty\)-operads\\ and higher\\ algebra};
\draw (194.61999999999998, -90.19999999999999) rectangle (199.71999999999997,-92.29999999999998);
\draw(199.82, -90.19999999999999) node[anchor=north west,align=left] {Simplicial\\ sets,\\ simplicial\\ objects};
\draw (199.82, -90.19999999999999) rectangle (202.92,-92.29999999999998);
\draw(188.92, -92.89999999999998) node[anchor=north west,align=left] {Strict \\ omega-categories,\\ computads,\\ polygraphs};
\draw (188.92, -92.89999999999998) rectangle (193.76999999999998,-94.99999999999997);
\draw(193.86999999999998, -92.89999999999998) node[anchor=north west,align=left] {Categorification};
\draw (193.86999999999998, -92.89999999999998) rectangle (198.46999999999997,-93.99999999999997);
\draw(198.57, -92.89999999999998) node[anchor=north west,align=left] {2-categories,\\ bicategories,\\ double\\ categories};
\draw (198.57, -92.89999999999998) rectangle (202.42,-94.99999999999997);
\draw(188.92, -95.09999999999998) node[anchor=north west,align=left] {2-dimensional\\ monad theory};
\draw (188.92, -95.09999999999998) rectangle (192.76999999999998,-96.69999999999997);
\draw(192.86999999999998, -95.09999999999998) node[anchor=north west,align=left] {Homotopical\\ algebra, \\ Quillen model\\ categories,\\ derivators};
\draw (192.86999999999998, -95.09999999999998) rectangle (196.71999999999997,-97.69999999999997);
\draw(203.67, -83.29999999999998) node[anchor=north west,align=left] {\large General theory of categories and functors};
\draw (203.67, -83.29999999999998) rectangle (219.01999999999998,-99.29999999999998);
\draw(204.67, -84.29999999999998) node[anchor=north west,align=left] {Factorization \\ systems, substructures,\\ quotient \\ structures, \\ congruences, amalgams};
\draw (204.67, -84.29999999999998) rectangle (211.01999999999998,-86.89999999999998);
\draw(211.11999999999998, -84.29999999999998) node[anchor=north west,align=left] {Limits and colimits\\ (products, sums,\\ directed limits,\\ pushouts, fiber\\ products, \\ equalizers, kernels, \\ ends and coends, etc.)};
\draw (211.11999999999998, -84.29999999999998) rectangle (217.21999999999997,-87.89999999999998);
\draw(204.67, -87.99999999999999) node[anchor=north west,align=left] {Categories \\ admitting limits \\ (complete categories),\\ functors\\ preserving limits,\\ completions};
\draw (204.67, -87.99999999999999) rectangle (210.76999999999998,-91.09999999999998);
\draw(210.86999999999998, -87.99999999999999) node[anchor=north west,align=left] {Special \\ properties of \\ functors (faithful,\\ full, etc.)};
\draw (210.86999999999998, -87.99999999999999) rectangle (216.21999999999997,-90.09999999999998);
\draw(204.67, -91.19999999999999) node[anchor=north west,align=left] {Adjoint functors\\ (universal \\ constructions, \\ reflective \\ subcategories, Kan \\ extensions, etc.)};
\draw (204.67, -91.19999999999999) rectangle (210.01999999999998,-94.29999999999998);
\draw(210.11999999999998, -91.19999999999999) node[anchor=north west,align=left] {Foundations,\\ relations to\\ logic and \\ deductive systems};
\draw (210.11999999999998, -91.19999999999999) rectangle (214.96999999999997,-93.29999999999998);
\draw(215.07, -91.19999999999999) node[anchor=north west,align=left] {Graphs, \\ diagram \\ schemes, \\ precategories};
\draw (215.07, -91.19999999999999) rectangle (218.92,-93.29999999999998);
\draw(204.67, -94.39999999999998) node[anchor=north west,align=left] {Definitions\\ and \\ generalizations \\ in theory of\\ categories};
\draw (204.67, -94.39999999999998) rectangle (209.26999999999998,-96.99999999999997);
\draw(209.36999999999998, -94.39999999999998) node[anchor=north west,align=left] {Epimorphisms,\\ monomorphisms,\\ special classes\\ of morphisms,\\ null morphisms};
\draw (209.36999999999998, -94.39999999999998) rectangle (213.71999999999997,-96.99999999999997);
\draw(213.82, -94.39999999999998) node[anchor=north west,align=left] {Functor \\ categories,\\ comma \\ categories};
\draw (213.82, -94.39999999999998) rectangle (217.17,-96.49999999999997);
\draw(204.67, -97.09999999999998) node[anchor=north west,align=left] {Natural \\ morphisms,\\ dinatural\\ morphisms};
\draw (204.67, -97.09999999999998) rectangle (207.76999999999998,-99.19999999999997);
\draw(207.86999999999998, -97.09999999999998) node[anchor=north west,align=left] {Graded \\ categories\\ (general)};
\draw (207.86999999999998, -97.09999999999998) rectangle (210.96999999999997,-98.69999999999997);
\draw(219.11999999999998, -83.29999999999998) node[anchor=north west,align=left] {\large Categories and theories};
\draw (219.11999999999998, -83.29999999999998) rectangle (229.26999999999998,-95.09999999999998);
\draw(220.11999999999998, -84.29999999999998) node[anchor=north west,align=left] {Monads (= standard\\ construction, \\ triple or triad), \\ algebras for monads,\\ homology and derived\\ functors for monads};
\draw (220.11999999999998, -84.29999999999998) rectangle (225.71999999999997,-87.39999999999998);
\draw(225.81999999999996, -84.29999999999998) node[anchor=north west,align=left] {Accessible\\ and locally\\ presentable\\ categories};
\draw (225.81999999999996, -84.29999999999998) rectangle (229.16999999999996,-86.39999999999998);
\draw(220.11999999999998, -87.49999999999999) node[anchor=north west,align=left] {Theories \\ (e.g., algebraic\\ theories),\\ structure,\\ and semantics};
\draw (220.11999999999998, -87.49999999999999) rectangle (224.71999999999997,-90.09999999999998);
\draw(224.81999999999996, -87.49999999999999) node[anchor=north west,align=left] {Eilenberg-Moore\\ and Kleisli\\ constructions\\ for monads};
\draw (224.81999999999996, -87.49999999999999) rectangle (229.16999999999996,-89.59999999999998);
\draw(220.11999999999998, -90.19999999999999) node[anchor=north west,align=left] {Sketches\\ and \\ generalizations};
\draw (220.11999999999998, -90.19999999999999) rectangle (224.46999999999997,-91.79999999999998);
\draw(224.56999999999996, -90.19999999999999) node[anchor=north west,align=left] {Structured\\ objects \\ in a category\\ (group\\ objects, etc.)};
\draw (224.56999999999996, -90.19999999999999) rectangle (228.66999999999996,-92.79999999999998);
\draw(220.11999999999998, -92.89999999999998) node[anchor=north west,align=left] {Categorical\\ semantics\\ of formal\\ languages};
\draw (220.11999999999998, -92.89999999999998) rectangle (223.46999999999997,-94.99999999999997);
\draw(223.56999999999996, -92.89999999999998) node[anchor=north west,align=left] {Equational\\ categories};
\draw (223.56999999999996, -92.89999999999998) rectangle (226.66999999999996,-94.49999999999997);
\draw(187.92, -99.39999999999999) node[anchor=north west,align=left] {\large Monoidal categories and operads};
\draw (187.92, -99.39999999999999) rectangle (200.32,-119.5);
\draw(188.92, -100.39999999999999) node[anchor=north west,align=left] {Polycategories/dioperads,\\ properads, PROPs, \\ cyclic operads,\\ modular operads};
\draw (188.92, -100.39999999999999) rectangle (195.76999999999998,-102.99999999999999);
\draw(195.86999999999998, -100.39999999999999) node[anchor=north west,align=left] {Algebraic \\ operads, \\ cooperads, and \\ Koszul duality};
\draw (195.86999999999998, -100.39999999999999) rectangle (200.21999999999997,-102.49999999999999);
\draw(188.92, -103.1) node[anchor=north west,align=left] {Braided \\ monoidal categories\\ and ribbon\\ categories};
\draw (188.92, -103.1) rectangle (194.26999999999998,-105.19999999999999);
\draw(194.36999999999998, -103.1) node[anchor=north west,align=left] {Fusion \\ categories, modular\\ tensor \\ categories, \\ modular functors};
\draw (194.36999999999998, -103.1) rectangle (199.71999999999997,-105.69999999999999);
\draw(188.92, -105.8) node[anchor=north west,align=left] {Dagger \\ categories, \\ categorical quantum\\ mechanics};
\draw (188.92, -105.8) rectangle (194.26999999999998,-107.89999999999999);
\draw(194.36999999999998, -105.8) node[anchor=north west,align=left] {Monoidal \\ categories, \\ symmetric monoidal\\ categories};
\draw (194.36999999999998, -105.8) rectangle (199.46999999999997,-107.89999999999999);
\draw(188.92, -108.0) node[anchor=north west,align=left] {Traced monoidal\\ categories,\\ compact \\ closed categories,\\ star-autonomous\\ categories};
\draw (188.92, -108.0) rectangle (194.01999999999998,-111.1);
\draw(194.11999999999998, -108.0) node[anchor=north west,align=left] {Categories\\ of networks\\ and \\ processes, \\ compositionality};
\draw (194.11999999999998, -108.0) rectangle (198.71999999999997,-110.6);
\draw(188.92, -111.19999999999999) node[anchor=north west,align=left] {Non-symmetric\\ operads, \\ multicategories,\\ generalized\\ multicategories};
\draw (188.92, -111.19999999999999) rectangle (193.51999999999998,-113.79999999999998);
\draw(193.61999999999998, -111.19999999999999) node[anchor=north west,align=left] {Bimonoidal,\\ skew-monoidal,\\ duoidal\\ categories};
\draw (193.61999999999998, -111.19999999999999) rectangle (197.71999999999997,-113.29999999999998);
\draw(188.92, -113.89999999999999) node[anchor=north west,align=left] {Species, \\ Hopf monoids,\\ operads in\\ combinatorics};
\draw (188.92, -113.89999999999999) rectangle (192.76999999999998,-115.99999999999999);
\draw(192.86999999999998, -113.89999999999999) node[anchor=north west,align=left] {String \\ diagrams and\\ graphical\\ calculi};
\draw (192.86999999999998, -113.89999999999999) rectangle (196.46999999999997,-115.99999999999999);
\draw(196.57, -113.89999999999999) node[anchor=north west,align=left] {Categorical\\ aspects\\ of \\ linear logic};
\draw (196.57, -113.89999999999999) rectangle (200.17,-115.99999999999999);
\draw(188.92, -116.1) node[anchor=north west,align=left] {Topological\\ and\\ simplicial\\ operads};
\draw (188.92, -116.1) rectangle (192.26999999999998,-118.19999999999999);
\draw(192.36999999999998, -116.1) node[anchor=north west,align=left] {Tannakian\\ categories};
\draw (192.36999999999998, -116.1) rectangle (195.46999999999997,-117.69999999999999);
\draw(195.57, -116.1) node[anchor=north west,align=left] {Operads\\ (general)};
\draw (195.57, -116.1) rectangle (198.42,-117.69999999999999);
\draw(188.92, -118.29999999999998) node[anchor=north west,align=left] {Globular\\ operads};
\draw (188.92, -118.29999999999998) rectangle (191.51999999999998,-119.39999999999998);
\draw(200.42, -99.39999999999999) node[anchor=north west,align=left] {\large Categorical structures};
\draw (200.42, -99.39999999999999) rectangle (211.32,-110.69999999999999);
\draw(201.42, -100.39999999999999) node[anchor=north west,align=left] {Proarrow equipments,\\ Yoneda \\ structures, KZ \\ doctrines (lax \\ idempotent monads)};
\draw (201.42, -100.39999999999999) rectangle (207.01999999999998,-102.99999999999999);
\draw(207.11999999999998, -100.39999999999999) node[anchor=north west,align=left] {Enriched \\ categories \\ (over closed\\ or monoidal\\ categories)};
\draw (207.11999999999998, -100.39999999999999) rectangle (210.71999999999997,-102.99999999999999);
\draw(201.42, -103.1) node[anchor=north west,align=left] {Actions of a\\ monoidal \\ category, \\ tensorial strength};
\draw (201.42, -103.1) rectangle (206.51999999999998,-105.19999999999999);
\draw(206.61999999999998, -103.1) node[anchor=north west,align=left] {Profunctors\\ (= \\ correspondences,\\ distributors,\\ modules)};
\draw (206.61999999999998, -103.1) rectangle (211.21999999999997,-105.69999999999999);
\draw(201.42, -105.8) node[anchor=north west,align=left] {Closed categories\\ (closed \\ monoidal and \\ Cartesian closed\\ categories, etc.)};
\draw (201.42, -105.8) rectangle (206.26999999999998,-108.39999999999999);
\draw(206.36999999999998, -105.8) node[anchor=north west,align=left] {Fibered\\ categories};
\draw (206.36999999999998, -105.8) rectangle (209.46999999999997,-107.39999999999999);
\draw(201.42, -108.5) node[anchor=north west,align=left] {Internal\\ categories\\ and\\ groupoids};
\draw (201.42, -108.5) rectangle (204.51999999999998,-110.6);
\draw(204.61999999999998, -108.5) node[anchor=north west,align=left] {Formal\\ category\\ theory};
\draw (204.61999999999998, -108.5) rectangle (207.21999999999997,-110.1);
\draw(211.42, -99.39999999999999) node[anchor=north west,align=left] {\large Categorical algebra};
\draw (211.42, -99.39999999999999) rectangle (221.57,-111.39999999999999);
\draw(212.42, -100.39999999999999) node[anchor=north west,align=left] {Protomodular\\ categories, \\ semi-abelian \\ categories, \\ Mal’tsev categories};
\draw (212.42, -100.39999999999999) rectangle (217.76999999999998,-102.99999999999999);
\draw(217.86999999999998, -100.39999999999999) node[anchor=north west,align=left] {Preadditive,\\ additive\\ categories};
\draw (217.86999999999998, -100.39999999999999) rectangle (221.46999999999997,-101.99999999999999);
\draw(212.42, -103.1) node[anchor=north west,align=left] {Definable \\ subcategories\\ and \\ connections with\\ model theory};
\draw (212.42, -103.1) rectangle (217.01999999999998,-105.69999999999999);
\draw(217.11999999999998, -103.1) node[anchor=north west,align=left] {Localization\\ of categories,\\ calculus\\ of fractions};
\draw (217.11999999999998, -103.1) rectangle (221.21999999999997,-105.19999999999999);
\draw(212.42, -105.8) node[anchor=north west,align=left] {Abelian \\ categories,\\ Grothendieck\\ categories};
\draw (212.42, -105.8) rectangle (216.01999999999998,-107.89999999999999);
\draw(216.11999999999998, -105.8) node[anchor=north west,align=left] {Regular \\ categories,\\ Barr-exact\\ categories};
\draw (216.11999999999998, -105.8) rectangle (219.46999999999997,-107.89999999999999);
\draw(212.42, -108.0) node[anchor=north west,align=left] {Categorical\\ embedding\\ theorems};
\draw (212.42, -108.0) rectangle (215.76999999999998,-109.6);
\draw(215.86999999999998, -108.0) node[anchor=north west,align=left] {Categorical\\ Galois\\ theory};
\draw (215.86999999999998, -108.0) rectangle (219.21999999999997,-109.6);
\draw(212.42, -109.69999999999999) node[anchor=north west,align=left] {Torsion\\ theories,\\ radicals};
\draw (212.42, -109.69999999999999) rectangle (215.26999999999998,-111.29999999999998);
\draw(187.92, -119.6) node[anchor=north west,align=left] {\large Special categories};
\draw (187.92, -119.6) rectangle (197.82,-130.4);
\draw(188.92, -120.6) node[anchor=north west,align=left] {Categories\\ of sets,\\ characterizations};
\draw (188.92, -120.6) rectangle (193.76999999999998,-122.69999999999999);
\draw(193.86999999999998, -120.6) node[anchor=north west,align=left] {Extensive, \\ distributive,\\ and adhesive\\ categories};
\draw (193.86999999999998, -120.6) rectangle (197.71999999999997,-122.69999999999999);
\draw(188.92, -122.8) node[anchor=north west,align=left] {Categories\\ of spans/cospans,\\ relations, or \\ partial maps};
\draw (188.92, -122.8) rectangle (193.76999999999998,-125.39999999999999);
\draw(193.86999999999998, -122.8) node[anchor=north west,align=left] {Categories\\ of machines,\\ automata};
\draw (193.86999999999998, -122.8) rectangle (197.46999999999997,-124.39999999999999);
\draw(193.86999999999998, -124.5) node[anchor=north west,align=left] {Topoi};
\draw (193.86999999999998, -124.5) rectangle (195.71999999999997,-125.1);
\draw(188.92, -125.5) node[anchor=north west,align=left] {Preorders, \\ orders, domains\\ and lattices\\ (viewed\\ as categories)};
\draw (188.92, -125.5) rectangle (193.26999999999998,-128.1);
\draw(193.36999999999998, -125.5) node[anchor=north west,align=left] {Groupoids, \\ semigroupoids,\\ semigroups, \\ groups (viewed\\ as categories)};
\draw (193.36999999999998, -125.5) rectangle (197.46999999999997,-128.1);
\draw(188.92, -128.2) node[anchor=north west,align=left] {Embedding\\ theorems,\\ universal\\ categories};
\draw (188.92, -128.2) rectangle (192.01999999999998,-130.29999999999998);
\draw(232.86999999999998, -1) node[anchor=north west,align=left] {\LARGE Measure and integration};
\draw (232.86999999999998, -1) rectangle (274.40999999999997,-32.300000000000004);
\draw(233.86999999999998, -2) node[anchor=north west,align=left] {\large Set functions, measures and integrals with values in abstract spaces};
\draw (233.86999999999998, -2) rectangle (255.54999999999998,-6.199999999999999);
\draw(234.86999999999998, -3) node[anchor=north west,align=left] {Set-valued set \\ functions and \\ measures; integration\\ of set-valued\\ functions; \\ measurable selections};
\draw (234.86999999999998, -3) rectangle (240.71999999999997,-6.1);
\draw(240.81999999999996, -3) node[anchor=north west,align=left] {Group- or \\ semigroup-valued\\ set \\ functions, measures\\ and integrals};
\draw (240.81999999999996, -3) rectangle (246.16999999999996,-5.6);
\draw(246.26999999999998, -3) node[anchor=north west,align=left] {Vector-valued\\ set functions,\\ measures\\ and integrals};
\draw (246.26999999999998, -3) rectangle (250.36999999999998,-5.1);
\draw(250.46999999999997, -3) node[anchor=north west,align=left] {Set functions,\\ measures and\\ integrals \\ with values in\\ ordered spaces};
\draw (250.46999999999997, -3) rectangle (254.56999999999996,-5.6);
\draw(255.64999999999998, -2) node[anchor=north west,align=left] {\large Miscellaneous topics in measure theory};
\draw (255.64999999999998, -2) rectangle (268.03,-5.2);
\draw(256.65, -3) node[anchor=north west,align=left] {Other \\ connections \\ with logic \\ and set theory};
\draw (256.65, -3) rectangle (260.75,-5.1);
\draw(260.84999999999997, -3) node[anchor=north west,align=left] {Nonstandard\\ measure\\ theory};
\draw (260.84999999999997, -3) rectangle (264.2,-4.6);
\draw(264.29999999999995, -3) node[anchor=north west,align=left] {Fuzzy\\ measure\\ theory};
\draw (264.29999999999995, -3) rectangle (266.65,-4.6);
\draw(268.13, -2) node[anchor=north west,align=left] {\large History of measure\\ and integration};
\draw (268.13, -2) rectangle (274.31,-3.1);
\draw(233.86999999999998, -6.299999999999999) node[anchor=north west,align=left] {\large Set functions and measures on spaces with additional structure};
\draw (233.86999999999998, -6.299999999999999) rectangle (253.71999999999997,-13.7);
\draw(234.86999999999998, -7.299999999999999) node[anchor=north west,align=left] {Set functions and\\ measures and integrals\\ in \\ infinite-dimensional spaces\\ (Wiener measure, \\ Gaussian measure, etc.)};
\draw (234.86999999999998, -7.299999999999999) rectangle (242.21999999999997,-10.399999999999999);
\draw(242.31999999999996, -7.299999999999999) node[anchor=north west,align=left] {Integration theory\\ via linear \\ functionals (Radon \\ measures, Daniell \\ integrals, etc.),\\ representing set\\ functions and measures};
\draw (242.31999999999996, -7.299999999999999) rectangle (248.41999999999996,-10.899999999999999);
\draw(248.51999999999998, -7.299999999999999) node[anchor=north west,align=left] {Set functions and\\ measures on \\ topological groups\\ or semigroups,\\ Haar measures, \\ invariant measures};
\draw (248.51999999999998, -7.299999999999999) rectangle (253.61999999999998,-10.399999999999999);
\draw(234.86999999999998, -11.0) node[anchor=north west,align=left] {Set functions \\ and measures on\\ topological \\ spaces (regularity\\ of measures, etc.)};
\draw (234.86999999999998, -11.0) rectangle (239.96999999999997,-13.6);
\draw(253.82, -6.299999999999999) node[anchor=north west,align=left] {\large Measure-theoretic ergodic theory};
\draw (253.82, -6.299999999999999) rectangle (265.46999999999997,-12.2);
\draw(254.82, -7.299999999999999) node[anchor=north west,align=left] {General groups\\ of \\ measure-preserving \\ transformations};
\draw (254.82, -7.299999999999999) rectangle (260.17,-9.399999999999999);
\draw(260.27, -7.299999999999999) node[anchor=north west,align=left] {Measure-preserving\\ transformations};
\draw (260.27, -7.299999999999999) rectangle (265.37,-9.399999999999999);
\draw(254.82, -9.5) node[anchor=north west,align=left] {One-parameter\\ continuous \\ families of \\ measure-preserving\\ transformations};
\draw (254.82, -9.5) rectangle (259.92,-12.1);
\draw(260.02, -9.5) node[anchor=north west,align=left] {Entropy \\ and other\\ invariants};
\draw (260.02, -9.5) rectangle (263.12,-11.1);
\draw(265.57, -6.299999999999999) node[anchor=north west,align=left] {\large Computational methods for\\ problems pertaining to\\ measure and integration};
\draw (265.57, -6.299999999999999) rectangle (273.92,-7.899999999999999);
\draw(233.86999999999998, -13.799999999999999) node[anchor=north west,align=left] {\large Classical measure theory};
\draw (233.86999999999998, -13.799999999999999) rectangle (245.01999999999998,-32.2);
\draw(234.86999999999998, -14.799999999999999) node[anchor=north west,align=left] {Measurable and\\ nonmeasurable \\ functions, sequences\\ of measurable\\ functions,\\ modes of convergence};
\draw (234.86999999999998, -14.799999999999999) rectangle (240.46999999999997,-17.9);
\draw(240.56999999999996, -14.799999999999999) node[anchor=north west,align=left] {Contents, \\ measures, \\ outer measures,\\ capacities};
\draw (240.56999999999996, -14.799999999999999) rectangle (244.91999999999996,-16.9);
\draw(234.86999999999998, -18.0) node[anchor=north west,align=left] {Classes of sets\\ (Borel fields, \\ \(\sigma\)-rings,\\ etc.), measurable\\ sets, Suslin\\ sets, analytic sets};
\draw (234.86999999999998, -18.0) rectangle (240.21999999999997,-21.1);
\draw(240.31999999999996, -18.0) node[anchor=north west,align=left] {Measures \\ on Boolean\\ rings, \\ measure algebras};
\draw (240.31999999999996, -18.0) rectangle (244.91999999999996,-20.1);
\draw(234.86999999999998, -21.2) node[anchor=north west,align=left] {Abstract \\ differentiation \\ theory, \\ differentiation of\\ set functions};
\draw (234.86999999999998, -21.2) rectangle (239.96999999999997,-23.8);
\draw(240.06999999999996, -21.2) node[anchor=north west,align=left] {Real- or\\ complex-valued\\ set\\ functions};
\draw (240.06999999999996, -21.2) rectangle (244.16999999999996,-23.3);
\draw(234.86999999999998, -23.9) node[anchor=north west,align=left] {Integration\\ and \\ disintegration\\ of measures};
\draw (234.86999999999998, -23.9) rectangle (238.96999999999997,-26.0);
\draw(239.06999999999996, -23.9) node[anchor=north west,align=left] {Length, area,\\ volume, other\\ geometric \\ measure theory};
\draw (239.06999999999996, -23.9) rectangle (243.16999999999996,-26.0);
\draw(234.86999999999998, -26.1) node[anchor=north west,align=left] {Integration\\ with respect\\ to measures\\ and other \\ set functions};
\draw (234.86999999999998, -26.1) rectangle (238.71999999999997,-28.700000000000003);
\draw(238.81999999999996, -26.1) node[anchor=north west,align=left] {Measures \\ and integrals\\ in product\\ spaces};
\draw (238.81999999999996, -26.1) rectangle (242.66999999999996,-28.200000000000003);
\draw(234.86999999999998, -28.799999999999997) node[anchor=north west,align=left] {Spaces of\\ measures,\\ convergence\\ of measures};
\draw (234.86999999999998, -28.799999999999997) rectangle (238.21999999999997,-30.9);
\draw(238.31999999999996, -28.799999999999997) node[anchor=north west,align=left] {Hausdorff\\ and packing\\ measures};
\draw (238.31999999999996, -28.799999999999997) rectangle (241.66999999999996,-30.4);
\draw(241.76999999999998, -28.799999999999997) node[anchor=north west,align=left] {Fractals};
\draw (241.76999999999998, -28.799999999999997) rectangle (244.36999999999998,-29.9);
\draw(234.86999999999998, -31.0) node[anchor=north west,align=left] {Lifting\\ theory};
\draw (234.86999999999998, -31.0) rectangle (237.21999999999997,-32.1);
\draw(232.86999999999998, -32.400000000000006) node[anchor=north west,align=left] {\LARGE Algebraic geometry};
\draw (232.86999999999998, -32.400000000000006) rectangle (273.46999999999997,-120.80000000000001);
\draw(233.86999999999998, -33.400000000000006) node[anchor=north west,align=left] {\large Arithmetic problems in algebraic geometry; Diophantine geometry};
\draw (233.86999999999998, -33.400000000000006) rectangle (257.66999999999996,-42.50000000000001);
\draw(234.86999999999998, -34.400000000000006) node[anchor=north west,align=left] {Zeta functions \\ and related questions\\ in algebraic\\ geometry (e.g.,\\ Birch-Swinnerton-Dyer\\ conjecture)};
\draw (234.86999999999998, -34.400000000000006) rectangle (240.71999999999997,-37.50000000000001);
\draw(240.81999999999996, -34.400000000000006) node[anchor=north west,align=left] {Hasse principle,\\ weak and \\ strong approximation,\\ Brauer-Manin\\ obstruction};
\draw (240.81999999999996, -34.400000000000006) rectangle (246.66999999999996,-37.00000000000001);
\draw(246.76999999999998, -34.400000000000006) node[anchor=north west,align=left] {Universal profinite\\ groups (relationship\\ to moduli\\ spaces, projective\\ and moduli towers,\\ Galois theory)};
\draw (246.76999999999998, -34.400000000000006) rectangle (252.36999999999998,-37.50000000000001);
\draw(252.46999999999997, -34.400000000000006) node[anchor=north west,align=left] {Positive \\ characteristic\\ ground \\ fields in \\ algebraic geometry};
\draw (252.46999999999997, -34.400000000000006) rectangle (257.57,-37.00000000000001);
\draw(234.86999999999998, -37.60000000000001) node[anchor=north west,align=left] {Other \\ nonalgebraically \\ closed ground \\ fields in algebraic\\ geometry};
\draw (234.86999999999998, -37.60000000000001) rectangle (240.21999999999997,-40.20000000000001);
\draw(240.31999999999996, -37.60000000000001) node[anchor=north west,align=left] {Applications\\ to coding \\ theory and \\ cryptography of \\ arithmetic geometry};
\draw (240.31999999999996, -37.60000000000001) rectangle (245.66999999999996,-40.20000000000001);
\draw(245.76999999999998, -37.60000000000001) node[anchor=north west,align=left] {Arithmetic\\ varieties \\ and schemes;\\ Arakelov \\ theory; heights};
\draw (245.76999999999998, -37.60000000000001) rectangle (250.11999999999998,-40.20000000000001);
\draw(250.21999999999997, -37.60000000000001) node[anchor=north west,align=left] {Perfectoid\\ spaces and\\ mixed \\ characteristic};
\draw (250.21999999999997, -37.60000000000001) rectangle (254.31999999999996,-39.70000000000001);
\draw(254.42, -37.60000000000001) node[anchor=north west,align=left] {Rational\\ points};
\draw (254.42, -37.60000000000001) rectangle (257.02,-38.70000000000001);
\draw(234.86999999999998, -40.300000000000004) node[anchor=north west,align=left] {Finite \\ ground fields\\ in algebraic\\ geometry};
\draw (234.86999999999998, -40.300000000000004) rectangle (238.71999999999997,-42.400000000000006);
\draw(238.81999999999996, -40.300000000000004) node[anchor=north west,align=left] {Global \\ ground fields\\ in algebraic\\ geometry};
\draw (238.81999999999996, -40.300000000000004) rectangle (242.66999999999996,-42.400000000000006);
\draw(242.76999999999998, -40.300000000000004) node[anchor=north west,align=left] {Local ground\\ fields\\ in algebraic\\ geometry};
\draw (242.76999999999998, -40.300000000000004) rectangle (246.36999999999998,-42.400000000000006);
\draw(246.46999999999997, -40.300000000000004) node[anchor=north west,align=left] {Modular \\ and Shimura\\ varieties};
\draw (246.46999999999997, -40.300000000000004) rectangle (249.81999999999996,-41.900000000000006);
\draw(249.92, -40.300000000000004) node[anchor=north west,align=left] {Rigid \\ analytic\\ geometry};
\draw (249.92, -40.300000000000004) rectangle (252.51999999999998,-41.900000000000006);
\draw(257.77, -33.400000000000006) node[anchor=north west,align=left] {\large (Co)homology theory in algebraic geometry};
\draw (257.77, -33.400000000000006) rectangle (273.37,-47.400000000000006);
\draw(258.77, -34.400000000000006) node[anchor=north west,align=left] {Other algebro-geometric\\ (co)homologies\\ (e.g., \\ intersection, equivariant,\\ Lawson, Deligne\\ (co)homologies)};
\draw (258.77, -34.400000000000006) rectangle (265.87,-37.50000000000001);
\draw(265.96999999999997, -34.400000000000006) node[anchor=north west,align=left] {Differentials\\ and other special\\ sheaves; \\ D-modules; \\ Bernstein-Sato \\ ideals and polynomials};
\draw (265.96999999999997, -34.400000000000006) rectangle (272.07,-37.50000000000001);
\draw(258.77, -37.60000000000001) node[anchor=north west,align=left] {Derived categories\\ of sheaves,\\ dg categories,\\ and related \\ constructions in\\ algebraic geometry};
\draw (258.77, -37.60000000000001) rectangle (263.87,-40.70000000000001);
\draw(263.96999999999997, -37.60000000000001) node[anchor=north west,align=left] {Homotopy \\ theory and \\ fundamental groups\\ in algebraic\\ geometry};
\draw (263.96999999999997, -37.60000000000001) rectangle (269.07,-40.20000000000001);
\draw(269.16999999999996, -37.60000000000001) node[anchor=north west,align=left] {Étale and \\ other \\ Grothendieck \\ topologies and\\ (co)homologies};
\draw (269.16999999999996, -37.60000000000001) rectangle (273.27,-40.20000000000001);
\draw(258.77, -40.800000000000004) node[anchor=north west,align=left] {Classical \\ real and complex\\ (co)homology\\ in algebraic\\ geometry};
\draw (258.77, -40.800000000000004) rectangle (263.37,-43.400000000000006);
\draw(263.46999999999997, -40.800000000000004) node[anchor=north west,align=left] {Motivic \\ cohomology;\\ motivic \\ homotopy theory};
\draw (263.46999999999997, -40.800000000000004) rectangle (267.82,-42.900000000000006);
\draw(267.91999999999996, -40.800000000000004) node[anchor=north west,align=left] {de Rham \\ cohomology \\ and algebraic\\ geometry};
\draw (267.91999999999996, -40.800000000000004) rectangle (271.77,-42.900000000000006);
\draw(258.77, -43.50000000000001) node[anchor=north west,align=left] {Sheaves \\ in algebraic\\ geometry};
\draw (258.77, -43.50000000000001) rectangle (262.37,-45.10000000000001);
\draw(262.46999999999997, -43.50000000000001) node[anchor=north west,align=left] {Vanishing\\ theorems \\ in algebraic\\ geometry};
\draw (262.46999999999997, -43.50000000000001) rectangle (266.07,-45.60000000000001);
\draw(266.16999999999996, -43.50000000000001) node[anchor=north west,align=left] {Topological\\ properties\\ in algebraic\\ geometry};
\draw (266.16999999999996, -43.50000000000001) rectangle (269.77,-45.60000000000001);
\draw(269.87, -43.50000000000001) node[anchor=north west,align=left] {\(p\)-adic\\ cohomology,\\ crystalline\\ cohomology};
\draw (269.87, -43.50000000000001) rectangle (273.22,-45.60000000000001);
\draw(258.77, -45.7) node[anchor=north west,align=left] {Multiplier\\ ideals};
\draw (258.77, -45.7) rectangle (261.87,-47.300000000000004);
\draw(261.96999999999997, -45.7) node[anchor=north west,align=left] {Brauer \\ groups of\\ schemes};
\draw (261.96999999999997, -45.7) rectangle (264.82,-47.300000000000004);
\draw(233.86999999999998, -42.60000000000001) node[anchor=north west,align=left] {\large History of \\ algebraic geometry};
\draw (233.86999999999998, -42.60000000000001) rectangle (240.04999999999998,-43.70000000000001);
\draw(233.86999999999998, -47.50000000000001) node[anchor=north west,align=left] {\large Projective and enumerative algebraic geometry};
\draw (233.86999999999998, -47.50000000000001) rectangle (250.71999999999997,-57.10000000000001);
\draw(234.86999999999998, -48.50000000000001) node[anchor=north west,align=left] {Gromov-Witten \\ invariants, quantum \\ cohomology, Gopakumar-Vafa\\ invariants,\\ Donaldson-Thomas\\ invariants \\ (algebro-geometric aspects)};
\draw (234.86999999999998, -48.50000000000001) rectangle (242.21999999999997,-52.10000000000001);
\draw(242.31999999999996, -48.50000000000001) node[anchor=north west,align=left] {Enumerative \\ problems \\ (combinatorial \\ problems) in \\ algebraic geometry};
\draw (242.31999999999996, -48.50000000000001) rectangle (247.41999999999996,-51.10000000000001);
\draw(247.51999999999998, -48.50000000000001) node[anchor=north west,align=left] {Varieties\\ of \\ low degree};
\draw (247.51999999999998, -48.50000000000001) rectangle (250.61999999999998,-50.10000000000001);
\draw(234.86999999999998, -52.20000000000001) node[anchor=north west,align=left] {Secant \\ varieties, tensor\\ rank, \\ varieties of\\ sums of powers};
\draw (234.86999999999998, -52.20000000000001) rectangle (239.71999999999997,-54.80000000000001);
\draw(239.81999999999996, -52.20000000000001) node[anchor=north west,align=left] {Configurations\\ and \\ arrangements of \\ linear subspaces};
\draw (239.81999999999996, -52.20000000000001) rectangle (244.41999999999996,-54.30000000000001);
\draw(244.51999999999998, -52.20000000000001) node[anchor=north west,align=left] {Projective\\ techniques\\ in algebraic\\ geometry};
\draw (244.51999999999998, -52.20000000000001) rectangle (248.11999999999998,-54.30000000000001);
\draw(234.86999999999998, -54.900000000000006) node[anchor=north west,align=left] {Adjunction\\ problems};
\draw (234.86999999999998, -54.900000000000006) rectangle (237.96999999999997,-56.50000000000001);
\draw(238.06999999999996, -54.900000000000006) node[anchor=north west,align=left] {Classical\\ problems,\\ Schubert\\ calculus};
\draw (238.06999999999996, -54.900000000000006) rectangle (240.91999999999996,-57.00000000000001);
\draw(250.82, -47.50000000000001) node[anchor=north west,align=left] {\large Families, fibrations in algebraic geometry};
\draw (250.82, -47.50000000000001) rectangle (266.92,-60.30000000000001);
\draw(251.82, -48.50000000000001) node[anchor=north west,align=left] {Applications of \\ vector bundles and\\ moduli spaces in\\ mathematical \\ physics (twistor \\ theory, instantons,\\ quantum field theory)};
\draw (251.82, -48.50000000000001) rectangle (257.67,-52.10000000000001);
\draw(257.77, -48.50000000000001) node[anchor=north west,align=left] {Structure \\ of families\\ (Picard-Lefschetz,\\ monodromy, etc.)};
\draw (257.77, -48.50000000000001) rectangle (262.87,-51.10000000000001);
\draw(262.96999999999997, -48.50000000000001) node[anchor=north west,align=left] {Fine and\\ coarse \\ moduli spaces};
\draw (262.96999999999997, -48.50000000000001) rectangle (266.82,-50.10000000000001);
\draw(251.82, -52.20000000000001) node[anchor=north west,align=left] {Fibrations,\\ degenerations\\ in \\ algebraic geometry};
\draw (251.82, -52.20000000000001) rectangle (256.92,-54.30000000000001);
\draw(257.02, -52.20000000000001) node[anchor=north west,align=left] {Variation \\ of Hodge \\ structures \\ (algebro-geometric\\ aspects)};
\draw (257.02, -52.20000000000001) rectangle (262.12,-54.80000000000001);
\draw(262.21999999999997, -52.20000000000001) node[anchor=north west,align=left] {Algebraic \\ moduli problems,\\ moduli of\\ vector bundles};
\draw (262.21999999999997, -52.20000000000001) rectangle (266.82,-54.30000000000001);
\draw(251.82, -54.900000000000006) node[anchor=north west,align=left] {Formal methods\\ and deformations\\ in \\ algebraic geometry};
\draw (251.82, -54.900000000000006) rectangle (256.92,-57.00000000000001);
\draw(257.02, -54.900000000000006) node[anchor=north west,align=left] {Geometric \\ Langlands \\ program \\ (algebro-geometric\\ aspects)};
\draw (257.02, -54.900000000000006) rectangle (262.12,-57.50000000000001);
\draw(262.21999999999997, -54.900000000000006) node[anchor=north west,align=left] {Stacks \\ and moduli\\ problems};
\draw (262.21999999999997, -54.900000000000006) rectangle (265.32,-56.50000000000001);
\draw(251.82, -57.60000000000001) node[anchor=north west,align=left] {Arithmetic ground\\ fields (finite,\\ local, global)\\ and families\\ or fibrations};
\draw (251.82, -57.60000000000001) rectangle (256.67,-60.20000000000001);
\draw(233.86999999999998, -60.400000000000006) node[anchor=north west,align=left] {\large Surfaces and higher-dimensional varieties};
\draw (233.86999999999998, -60.400000000000006) rectangle (249.46999999999997,-77.30000000000001);
\draw(234.86999999999998, -61.400000000000006) node[anchor=north west,align=left] {Arithmetic \\ ground fields \\ for surfaces \\ or higher-dimensional\\ varieties};
\draw (234.86999999999998, -61.400000000000006) rectangle (240.71999999999997,-64.0);
\draw(240.81999999999996, -61.400000000000006) node[anchor=north west,align=left] {Topology of \\ surfaces (Donaldson\\ polynomials,\\ Seiberg-Witten\\ invariants)};
\draw (240.81999999999996, -61.400000000000006) rectangle (246.16999999999996,-64.0);
\draw(246.26999999999998, -61.400000000000006) node[anchor=north west,align=left] {Surfaces\\ of general\\ type};
\draw (246.26999999999998, -61.400000000000006) rectangle (249.36999999999998,-63.00000000000001);
\draw(234.86999999999998, -64.10000000000001) node[anchor=north west,align=left] {Moduli, \\ classification: \\ analytic theory;\\ relations \\ with modular forms};
\draw (234.86999999999998, -64.10000000000001) rectangle (239.96999999999997,-66.7);
\draw(240.06999999999996, -64.10000000000001) node[anchor=north west,align=left] {Singularities\\ of surfaces\\ or \\ higher-dimensional\\ varieties};
\draw (240.06999999999996, -64.10000000000001) rectangle (245.16999999999996,-66.7);
\draw(245.26999999999998, -64.10000000000001) node[anchor=north west,align=left] {Hypersurfaces\\ and\\ algebraic\\ geometry};
\draw (245.26999999999998, -64.10000000000001) rectangle (249.11999999999998,-66.2);
\draw(234.86999999999998, -66.80000000000001) node[anchor=north west,align=left] {Elliptic \\ surfaces, elliptic\\ or Calabi-Yau\\ fibrations};
\draw (234.86999999999998, -66.80000000000001) rectangle (239.96999999999997,-68.9);
\draw(240.06999999999996, -66.80000000000001) node[anchor=north west,align=left] {Calabi-Yau \\ manifolds \\ (algebro-geometric\\ aspects)};
\draw (240.06999999999996, -66.80000000000001) rectangle (245.16999999999996,-68.9);
\draw(245.26999999999998, -66.80000000000001) node[anchor=north west,align=left] {Relationships\\ with physics};
\draw (245.26999999999998, -66.80000000000001) rectangle (249.11999999999998,-68.4);
\draw(234.86999999999998, -69.0) node[anchor=north west,align=left] {Mirror \\ symmetry \\ (algebro-geometric\\ aspects)};
\draw (234.86999999999998, -69.0) rectangle (239.96999999999997,-71.1);
\draw(240.06999999999996, -69.0) node[anchor=north west,align=left] {Automorphisms\\ of surfaces\\ and \\ higher-dimensional\\ varieties};
\draw (240.06999999999996, -69.0) rectangle (245.16999999999996,-71.6);
\draw(245.26999999999998, -69.0) node[anchor=north west,align=left] {\(K3\) \\ surfaces and\\ Enriques\\ surfaces};
\draw (245.26999999999998, -69.0) rectangle (248.86999999999998,-71.1);
\draw(234.86999999999998, -71.7) node[anchor=north west,align=left] {Vector bundles\\ on surfaces and\\ higher-dimensional\\ varieties,\\ and their moduli};
\draw (234.86999999999998, -71.7) rectangle (239.96999999999997,-74.3);
\draw(240.06999999999996, -71.7) node[anchor=north west,align=left] {Families, \\ moduli, \\ classification: \\ algebraic theory};
\draw (240.06999999999996, -71.7) rectangle (244.66999999999996,-73.8);
\draw(244.76999999999998, -71.7) node[anchor=north west,align=left] {Holomorphic\\ symplectic\\ varieties,\\ hyper-Kähler\\ varieties};
\draw (244.76999999999998, -71.7) rectangle (248.36999999999998,-74.3);
\draw(234.86999999999998, -74.4) node[anchor=north west,align=left] {\(3\)-folds};
\draw (234.86999999999998, -74.4) rectangle (238.21999999999997,-75.5);
\draw(238.31999999999996, -74.4) node[anchor=north west,align=left] {\(4\)-folds};
\draw (238.31999999999996, -74.4) rectangle (241.66999999999996,-75.5);
\draw(241.76999999999998, -74.4) node[anchor=north west,align=left] {\(n\)-folds\\ (\(n>4\))};
\draw (241.76999999999998, -74.4) rectangle (245.11999999999998,-76.0);
\draw(245.21999999999997, -74.4) node[anchor=north west,align=left] {Rational\\ and ruled\\ surfaces};
\draw (245.21999999999997, -74.4) rectangle (248.06999999999996,-76.0);
\draw(234.86999999999998, -76.10000000000001) node[anchor=north west,align=left] {Fano \\ varieties};
\draw (234.86999999999998, -76.10000000000001) rectangle (237.71999999999997,-77.2);
\draw(237.81999999999996, -76.10000000000001) node[anchor=north west,align=left] {Special\\ surfaces};
\draw (237.81999999999996, -76.10000000000001) rectangle (240.41999999999996,-77.2);
\draw(249.56999999999996, -60.400000000000006) node[anchor=north west,align=left] {\large Computational aspects in algebraic geometry};
\draw (249.56999999999996, -60.400000000000006) rectangle (265.16999999999996,-68.5);
\draw(250.56999999999996, -61.400000000000006) node[anchor=north west,align=left] {Geometric \\ aspects of \\ numerical algebraic\\ geometry};
\draw (250.56999999999996, -61.400000000000006) rectangle (255.91999999999996,-63.50000000000001);
\draw(256.02, -61.400000000000006) node[anchor=north west,align=left] {Computational\\ aspects of \\ higher-dimensional\\ varieties};
\draw (256.02, -61.400000000000006) rectangle (261.12,-63.50000000000001);
\draw(261.21999999999997, -61.400000000000006) node[anchor=north west,align=left] {Computational\\ aspects\\ of algebraic\\ curves};
\draw (261.21999999999997, -61.400000000000006) rectangle (265.07,-63.50000000000001);
\draw(250.56999999999996, -63.60000000000001) node[anchor=north west,align=left] {Effectivity, \\ complexity and\\ computational\\ aspects of \\ algebraic geometry};
\draw (250.56999999999996, -63.60000000000001) rectangle (255.66999999999996,-66.2);
\draw(255.76999999999995, -63.60000000000001) node[anchor=north west,align=left] {Computational\\ aspects\\ of algebraic\\ surfaces};
\draw (255.76999999999995, -63.60000000000001) rectangle (259.61999999999995,-65.7);
\draw(259.71999999999997, -63.60000000000001) node[anchor=north west,align=left] {Computational\\ algebraic\\ geometry over\\ arithmetic\\ ground fields};
\draw (259.71999999999997, -63.60000000000001) rectangle (263.57,-66.2);
\draw(250.56999999999996, -66.30000000000001) node[anchor=north west,align=left] {Computational\\ real\\ algebraic\\ geometry};
\draw (250.56999999999996, -66.30000000000001) rectangle (254.41999999999996,-68.4);
\draw(249.56999999999996, -68.60000000000001) node[anchor=north west,align=left] {\large Real algebraic and real-analytic geometry};
\draw (249.56999999999996, -68.60000000000001) rectangle (262.87999999999994,-74.00000000000001);
\draw(250.56999999999996, -69.60000000000001) node[anchor=north west,align=left] {Semialgebraic\\ sets\\ and related\\ spaces};
\draw (250.56999999999996, -69.60000000000001) rectangle (254.41999999999996,-71.7);
\draw(254.51999999999995, -69.60000000000001) node[anchor=north west,align=left] {Real-analytic\\ and\\ semi-analytic\\ sets};
\draw (254.51999999999995, -69.60000000000001) rectangle (258.36999999999995,-71.7);
\draw(258.46999999999997, -69.60000000000001) node[anchor=north west,align=left] {Nash \\ functions and\\ manifolds};
\draw (258.46999999999997, -69.60000000000001) rectangle (262.32,-71.2);
\draw(250.56999999999996, -71.80000000000001) node[anchor=north west,align=left] {Real \\ algebraic\\ sets};
\draw (250.56999999999996, -71.80000000000001) rectangle (253.41999999999996,-73.4);
\draw(253.51999999999995, -71.80000000000001) node[anchor=north west,align=left] {Topology\\ of real \\ algebraic\\ varieties};
\draw (253.51999999999995, -71.80000000000001) rectangle (256.36999999999995,-73.9);
\draw(265.27, -60.400000000000006) node[anchor=north west,align=left] {\large Tropical geometry};
\draw (265.27, -60.400000000000006) rectangle (273.16999999999996,-70.7);
\draw(266.27, -61.400000000000006) node[anchor=north west,align=left] {Foundations\\ of tropical\\ geometry\\ and relations\\ with algebra};
\draw (266.27, -61.400000000000006) rectangle (270.12,-64.0);
\draw(266.27, -64.10000000000001) node[anchor=north west,align=left] {Combinatorial\\ aspects\\ of tropical\\ varieties};
\draw (266.27, -64.10000000000001) rectangle (270.12,-66.2);
\draw(266.27, -66.30000000000001) node[anchor=north west,align=left] {Applications\\ of\\ tropical\\ geometry};
\draw (266.27, -66.30000000000001) rectangle (269.87,-68.4);
\draw(266.27, -68.5) node[anchor=north west,align=left] {Geometric\\ aspects\\ of tropical\\ varieties};
\draw (266.27, -68.5) rectangle (269.62,-70.6);
\draw(269.71999999999997, -68.5) node[anchor=north west,align=left] {Arithmetic\\ aspects \\ of tropical\\ varieties};
\draw (269.71999999999997, -68.5) rectangle (273.07,-70.6);
\draw(233.86999999999998, -77.4) node[anchor=north west,align=left] {\large Local theory in algebraic geometry};
\draw (233.86999999999998, -77.4) rectangle (246.71999999999997,-85.5);
\draw(234.86999999999998, -78.4) node[anchor=north west,align=left] {Local deformation\\ theory,\\ Artin \\ approximation, etc.};
\draw (234.86999999999998, -78.4) rectangle (240.21999999999997,-80.5);
\draw(240.31999999999996, -78.4) node[anchor=north west,align=left] {Local structure\\ of morphisms\\ in algebraic\\ geometry:\\ étale, flat, etc.};
\draw (240.31999999999996, -78.4) rectangle (245.16999999999996,-81.0);
\draw(234.86999999999998, -81.10000000000001) node[anchor=north west,align=left] {Singularities\\ in\\ algebraic\\ geometry};
\draw (234.86999999999998, -81.10000000000001) rectangle (238.71999999999997,-83.2);
\draw(238.81999999999996, -81.10000000000001) node[anchor=north west,align=left] {Deformations\\ of \\ singularities};
\draw (238.81999999999996, -81.10000000000001) rectangle (242.66999999999996,-82.7);
\draw(242.76999999999998, -81.10000000000001) node[anchor=north west,align=left] {Infinitesimal\\ methods\\ in algebraic\\ geometry};
\draw (242.76999999999998, -81.10000000000001) rectangle (246.61999999999998,-83.2);
\draw(234.86999999999998, -83.30000000000001) node[anchor=north west,align=left] {Local \\ cohomology \\ and algebraic\\ geometry};
\draw (234.86999999999998, -83.30000000000001) rectangle (238.71999999999997,-85.4);
\draw(238.81999999999996, -83.30000000000001) node[anchor=north west,align=left] {Formal \\ neighborhoods\\ in algebraic\\ geometry};
\draw (238.81999999999996, -83.30000000000001) rectangle (242.66999999999996,-85.4);
\draw(246.81999999999996, -77.4) node[anchor=north west,align=left] {\large Foundations of algebraic geometry};
\draw (246.81999999999996, -77.4) rectangle (259.21999999999997,-87.0);
\draw(247.81999999999996, -78.4) node[anchor=north west,align=left] {Fundamental constructions\\ in algebraic\\ geometry involving\\ higher and derived\\ categories (homotopical\\ algebraic \\ geometry, derived \\ algebraic geometry, etc.)};
\draw (247.81999999999996, -78.4) rectangle (254.66999999999996,-82.5);
\draw(254.76999999999995, -78.4) node[anchor=north west,align=left] {Generalizations\\ (algebraic \\ spaces, stacks)};
\draw (254.76999999999995, -78.4) rectangle (259.11999999999995,-80.5);
\draw(254.76999999999995, -80.60000000000001) node[anchor=north west,align=left] {Relevant\\ commutative\\ algebra};
\draw (254.76999999999995, -80.60000000000001) rectangle (258.11999999999995,-82.2);
\draw(247.81999999999996, -82.60000000000001) node[anchor=north west,align=left] {Noncommutative\\ algebraic\\ geometry};
\draw (247.81999999999996, -82.60000000000001) rectangle (251.91999999999996,-84.7);
\draw(252.01999999999995, -82.60000000000001) node[anchor=north west,align=left] {Elementary\\ questions\\ in algebraic\\ geometry};
\draw (252.01999999999995, -82.60000000000001) rectangle (255.61999999999995,-84.7);
\draw(255.71999999999997, -82.60000000000001) node[anchor=north west,align=left] {Logarithmic\\ algebraic\\ geometry,\\ log schemes};
\draw (255.71999999999997, -82.60000000000001) rectangle (259.07,-84.7);
\draw(247.81999999999996, -84.80000000000001) node[anchor=north west,align=left] {Geometry \\ over the \\ field with \\ one element};
\draw (247.81999999999996, -84.80000000000001) rectangle (251.16999999999996,-86.9);
\draw(251.26999999999995, -84.80000000000001) node[anchor=north west,align=left] {Varieties\\ and\\ morphisms};
\draw (251.26999999999995, -84.80000000000001) rectangle (254.11999999999995,-86.4);
\draw(254.21999999999997, -84.80000000000001) node[anchor=north west,align=left] {Schemes\\ and \\ morphisms};
\draw (254.21999999999997, -84.80000000000001) rectangle (257.07,-86.4);
\draw(259.32, -77.4) node[anchor=north west,align=left] {\large Special varieties};
\draw (259.32, -77.4) rectangle (271.71999999999997,-89.9);
\draw(260.32, -78.4) node[anchor=north west,align=left] {Varieties defined\\ by ring \\ conditions (factorial,\\ Cohen-Macaulay,\\ seminormal)};
\draw (260.32, -78.4) rectangle (266.42,-81.0);
\draw(266.52, -78.4) node[anchor=north west,align=left] {Compactifications;\\ symmetric\\ and spherical\\ varieties};
\draw (266.52, -78.4) rectangle (271.62,-80.5);
\draw(260.32, -81.10000000000001) node[anchor=north west,align=left] {Determinantalvarieties};
\draw (260.32, -81.10000000000001) rectangle (266.42,-82.7);
\draw(266.52, -81.10000000000001) node[anchor=north west,align=left] {Toric varieties,\\ Newton\\ polyhedra, \\ Okounkov bodies};
\draw (266.52, -81.10000000000001) rectangle (271.12,-83.2);
\draw(260.32, -83.30000000000001) node[anchor=north west,align=left] {Low codimension\\ problems\\ in algebraic\\ geometry};
\draw (260.32, -83.30000000000001) rectangle (264.67,-85.4);
\draw(264.77, -83.30000000000001) node[anchor=north west,align=left] {Homogeneous\\ spaces\\ and \\ generalizations};
\draw (264.77, -83.30000000000001) rectangle (269.12,-85.4);
\draw(269.21999999999997, -83.30000000000001) node[anchor=north west,align=left] {Linkage};
\draw (269.21999999999997, -83.30000000000001) rectangle (271.57,-84.4);
\draw(260.32, -85.5) node[anchor=north west,align=left] {Grassmannians,\\ Schubert\\ varieties, \\ flag manifolds};
\draw (260.32, -85.5) rectangle (264.42,-87.6);
\draw(264.52, -85.5) node[anchor=north west,align=left] {Supervarieties};
\draw (264.52, -85.5) rectangle (268.62,-86.6);
\draw(268.71999999999997, -85.5) node[anchor=north west,align=left] {Character\\ varieties};
\draw (268.71999999999997, -85.5) rectangle (271.57,-87.1);
\draw(260.32, -87.7) node[anchor=north west,align=left] {Complete\\ intersections};
\draw (260.32, -87.7) rectangle (264.17,-89.3);
\draw(264.27, -87.7) node[anchor=north west,align=left] {Rational\\ and \\ unirational\\ varieties};
\draw (264.27, -87.7) rectangle (267.62,-89.8);
\draw(267.71999999999997, -87.7) node[anchor=north west,align=left] {Rationally\\ connected\\ varieties};
\draw (267.71999999999997, -87.7) rectangle (270.82,-89.3);
\draw(233.86999999999998, -90.0) node[anchor=north west,align=left] {\large Curves in algebraic geometry};
\draw (233.86999999999998, -90.0) rectangle (245.71999999999997,-108.1);
\draw(234.86999999999998, -91.0) node[anchor=north west,align=left] {Special \\ divisors on \\ curves (gonality,\\ Brill-Noether theory)};
\draw (234.86999999999998, -91.0) rectangle (240.71999999999997,-93.6);
\draw(240.81999999999996, -91.0) node[anchor=north west,align=left] {Theta functions\\ and \\ curves; Schottky\\ problem};
\draw (240.81999999999996, -91.0) rectangle (245.41999999999996,-93.1);
\draw(234.86999999999998, -93.7) node[anchor=north west,align=left] {Riemann \\ surfaces; Weierstrass\\ points;\\ gap sequences};
\draw (234.86999999999998, -93.7) rectangle (240.71999999999997,-95.8);
\draw(240.81999999999996, -93.7) node[anchor=north west,align=left] {Special \\ algebraic curves\\ and curves\\ of low genus};
\draw (240.81999999999996, -93.7) rectangle (245.41999999999996,-95.8);
\draw(234.86999999999998, -95.9) node[anchor=north west,align=left] {Algebraic \\ functions and\\ function \\ fields in algebraic\\ geometry};
\draw (234.86999999999998, -95.9) rectangle (240.21999999999997,-98.5);
\draw(240.31999999999996, -95.9) node[anchor=north west,align=left] {Relationships\\ between \\ algebraic curves\\ and \\ integrable systems};
\draw (240.31999999999996, -95.9) rectangle (245.41999999999996,-98.5);
\draw(234.86999999999998, -98.6) node[anchor=north west,align=left] {Relationships\\ between \\ algebraic curves\\ and physics};
\draw (234.86999999999998, -98.6) rectangle (239.46999999999997,-100.69999999999999);
\draw(239.56999999999996, -98.6) node[anchor=north west,align=left] {Automorphismsof\\ curves};
\draw (239.56999999999996, -98.6) rectangle (243.91999999999996,-100.19999999999999);
\draw(234.86999999999998, -100.8) node[anchor=north west,align=left] {Singularities\\ of \\ curves, \\ local rings};
\draw (234.86999999999998, -100.8) rectangle (238.71999999999997,-102.89999999999999);
\draw(238.81999999999996, -100.8) node[anchor=north west,align=left] {Vector \\ bundles on \\ curves and \\ their moduli};
\draw (238.81999999999996, -100.8) rectangle (242.41999999999996,-102.89999999999999);
\draw(242.51999999999998, -100.8) node[anchor=north west,align=left] {Families,\\ moduli \\ of curves\\ (analytic)};
\draw (242.51999999999998, -100.8) rectangle (245.61999999999998,-102.89999999999999);
\draw(234.86999999999998, -103.0) node[anchor=north west,align=left] {Families,\\ moduli \\ of curves\\ (algebraic)};
\draw (234.86999999999998, -103.0) rectangle (238.21999999999997,-105.1);
\draw(238.31999999999996, -103.0) node[anchor=north west,align=left] {Coverings\\ of curves,\\ fundamental\\ group};
\draw (238.31999999999996, -103.0) rectangle (241.66999999999996,-105.1);
\draw(241.76999999999998, -103.0) node[anchor=north west,align=left] {Arithmetic\\ ground\\ fields\\ for curves};
\draw (241.76999999999998, -103.0) rectangle (244.86999999999998,-105.1);
\draw(234.86999999999998, -105.2) node[anchor=north west,align=left] {Jacobians,\\ Prym\\ varieties};
\draw (234.86999999999998, -105.2) rectangle (237.96999999999997,-106.8);
\draw(238.06999999999996, -105.2) node[anchor=north west,align=left] {Plane \\ and space\\ curves};
\draw (238.06999999999996, -105.2) rectangle (240.91999999999996,-106.8);
\draw(241.01999999999998, -105.2) node[anchor=north west,align=left] {Dessins\\ d’enfants\\ theory};
\draw (241.01999999999998, -105.2) rectangle (243.86999999999998,-106.8);
\draw(234.86999999999998, -106.9) node[anchor=north west,align=left] {Elliptic\\ curves};
\draw (234.86999999999998, -106.9) rectangle (237.46999999999997,-108.0);
\draw(245.81999999999996, -90.0) node[anchor=north west,align=left] {\large Cycles and subschemes};
\draw (245.81999999999996, -90.0) rectangle (257.21999999999997,-102.5);
\draw(246.81999999999996, -91.0) node[anchor=north west,align=left] {Intersection \\ theory, characteristic\\ classes,\\ intersection \\ multiplicities in\\ algebraic geometry};
\draw (246.81999999999996, -91.0) rectangle (252.91999999999996,-94.1);
\draw(253.01999999999995, -91.0) node[anchor=north west,align=left] {(Equivariant)\\ Chow \\ groups and \\ rings; motives};
\draw (253.01999999999995, -91.0) rectangle (257.11999999999995,-93.1);
\draw(246.81999999999996, -94.2) node[anchor=north west,align=left] {Transcendental\\ methods,\\ Hodge theory\\ (algebro-geometric\\ aspects)};
\draw (246.81999999999996, -94.2) rectangle (251.91999999999996,-96.8);
\draw(252.01999999999995, -94.2) node[anchor=north west,align=left] {Applications \\ of methods of \\ algebraic \\ \(K\)-theory in \\ algebraic geometry};
\draw (252.01999999999995, -94.2) rectangle (257.11999999999995,-96.8);
\draw(246.81999999999996, -96.9) node[anchor=north west,align=left] {Parametrization\\ (Chow\\ and Hilbert\\ schemes)};
\draw (246.81999999999996, -96.9) rectangle (251.16999999999996,-99.0);
\draw(251.26999999999995, -96.9) node[anchor=north west,align=left] {Divisors, \\ linear systems,\\ invertible\\ sheaves};
\draw (251.26999999999995, -96.9) rectangle (255.61999999999995,-99.0);
\draw(246.81999999999996, -99.1) node[anchor=north west,align=left] {Pencils, \\ nets, webs\\ in algebraic\\ geometry};
\draw (246.81999999999996, -99.1) rectangle (250.41999999999996,-101.19999999999999);
\draw(250.51999999999995, -99.1) node[anchor=north west,align=left] {Riemann-Roch\\ theorems};
\draw (250.51999999999995, -99.1) rectangle (254.11999999999995,-100.69999999999999);
\draw(254.21999999999997, -99.1) node[anchor=north west,align=left] {Algebraic\\ cycles};
\draw (254.21999999999997, -99.1) rectangle (257.07,-100.19999999999999);
\draw(246.81999999999996, -101.3) node[anchor=north west,align=left] {Torelli\\ problem};
\draw (246.81999999999996, -101.3) rectangle (249.16999999999996,-102.39999999999999);
\draw(249.26999999999995, -101.3) node[anchor=north west,align=left] {Picard\\ groups};
\draw (249.26999999999995, -101.3) rectangle (251.36999999999995,-102.39999999999999);
\draw(257.32, -90.0) node[anchor=north west,align=left] {\large Abelian varieties and schemes};
\draw (257.32, -90.0) rectangle (268.46999999999997,-100.3);
\draw(258.32, -91.0) node[anchor=north west,align=left] {Analytic theory\\ of abelian \\ varieties; abelian\\ integrals\\ and differentials};
\draw (258.32, -91.0) rectangle (263.42,-93.6);
\draw(263.52, -91.0) node[anchor=north west,align=left] {Algebraic \\ moduli of abelian\\ varieties,\\ classification};
\draw (263.52, -91.0) rectangle (268.37,-93.1);
\draw(258.32, -93.7) node[anchor=north west,align=left] {Picard \\ schemes, higher\\ Jacobians};
\draw (258.32, -93.7) rectangle (262.67,-95.3);
\draw(262.77, -93.7) node[anchor=north west,align=left] {Complex \\ multiplication\\ and abelian\\ varieties};
\draw (262.77, -93.7) rectangle (266.87,-95.8);
\draw(258.32, -95.9) node[anchor=north west,align=left] {Arithmetic\\ ground fields\\ for abelian\\ varieties};
\draw (258.32, -95.9) rectangle (262.17,-98.0);
\draw(262.27, -95.9) node[anchor=north west,align=left] {Theta \\ functions and\\ abelian\\ varieties};
\draw (262.27, -95.9) rectangle (266.12,-98.0);
\draw(258.32, -98.1) node[anchor=north west,align=left] {Subvarieties\\ of\\ abelian\\ varieties};
\draw (258.32, -98.1) rectangle (261.92,-100.19999999999999);
\draw(262.02, -98.1) node[anchor=north west,align=left] {Algebraic\\ theory \\ of abelian\\ varieties};
\draw (262.02, -98.1) rectangle (265.12,-100.19999999999999);
\draw(265.21999999999997, -98.1) node[anchor=north west,align=left] {Isogeny};
\draw (265.21999999999997, -98.1) rectangle (267.57,-99.19999999999999);
\draw(233.86999999999998, -108.20000000000002) node[anchor=north west,align=left] {\large Affine geometry};
\draw (233.86999999999998, -108.20000000000002) rectangle (244.01999999999998,-115.80000000000001);
\draw(234.86999999999998, -109.20000000000002) node[anchor=north west,align=left] {Affine spaces\\ (automorphisms,\\ embeddings,\\ exotic \\ structures, \\ cancellation problem)};
\draw (234.86999999999998, -109.20000000000002) rectangle (240.71999999999997,-112.30000000000001);
\draw(240.81999999999996, -109.20000000000002) node[anchor=north west,align=left] {Group \\ actions on\\ affine\\ varieties};
\draw (240.81999999999996, -109.20000000000002) rectangle (243.91999999999996,-111.30000000000001);
\draw(234.86999999999998, -112.40000000000002) node[anchor=north west,align=left] {Classification\\ of affine\\ varieties};
\draw (234.86999999999998, -112.40000000000002) rectangle (238.96999999999997,-114.50000000000001);
\draw(239.06999999999996, -112.40000000000002) node[anchor=north west,align=left] {Affine\\ fibrations};
\draw (239.06999999999996, -112.40000000000002) rectangle (242.16999999999996,-114.00000000000001);
\draw(234.86999999999998, -114.60000000000002) node[anchor=north west,align=left] {Jacobian\\ problem};
\draw (234.86999999999998, -114.60000000000002) rectangle (237.46999999999997,-115.70000000000002);
\draw(244.11999999999998, -108.20000000000002) node[anchor=north west,align=left] {\large Birational geometry};
\draw (244.11999999999998, -108.20000000000002) rectangle (253.76999999999998,-120.70000000000002);
\draw(245.11999999999998, -109.20000000000002) node[anchor=north west,align=left] {Global theory\\ and resolution\\ of singularities\\ (algebro-geometric\\ aspects)};
\draw (245.11999999999998, -109.20000000000002) rectangle (250.21999999999997,-111.80000000000001);
\draw(250.31999999999996, -109.20000000000002) node[anchor=north west,align=left] {Arcs and\\ motivic \\ integration};
\draw (250.31999999999996, -109.20000000000002) rectangle (253.66999999999996,-110.80000000000001);
\draw(245.11999999999998, -111.90000000000002) node[anchor=north west,align=left] {Rational\\ and \\ birational maps};
\draw (245.11999999999998, -111.90000000000002) rectangle (249.46999999999997,-113.50000000000001);
\draw(249.56999999999996, -111.90000000000002) node[anchor=north west,align=left] {McKay\\ correspondence};
\draw (249.56999999999996, -111.90000000000002) rectangle (253.66999999999996,-113.50000000000001);
\draw(245.11999999999998, -113.60000000000002) node[anchor=north west,align=left] {Birational\\ automorphisms,\\ Cremona\\ group and\\ generalizations};
\draw (245.11999999999998, -113.60000000000002) rectangle (249.46999999999997,-116.20000000000002);
\draw(249.56999999999996, -113.60000000000002) node[anchor=north west,align=left] {Minimal model\\ program \\ (Mori theory,\\ extremal rays)};
\draw (249.56999999999996, -113.60000000000002) rectangle (253.66999999999996,-115.70000000000002);
\draw(245.11999999999998, -116.30000000000001) node[anchor=north west,align=left] {Rationality\\ questions\\ in algebraic\\ geometry};
\draw (245.11999999999998, -116.30000000000001) rectangle (248.71999999999997,-118.4);
\draw(248.81999999999996, -116.30000000000001) node[anchor=north west,align=left] {Coverings\\ in algebraic\\ geometry};
\draw (248.81999999999996, -116.30000000000001) rectangle (252.41999999999996,-117.9);
\draw(245.11999999999998, -118.50000000000001) node[anchor=north west,align=left] {Ramification\\ problems\\ in algebraic\\ geometry};
\draw (245.11999999999998, -118.50000000000001) rectangle (248.71999999999997,-120.60000000000001);
\draw(248.81999999999996, -118.50000000000001) node[anchor=north west,align=left] {Embeddings\\ in algebraic\\ geometry};
\draw (248.81999999999996, -118.50000000000001) rectangle (252.41999999999996,-120.10000000000001);
\draw(253.86999999999998, -108.20000000000002) node[anchor=north west,align=left] {\large Algebraic groups};
\draw (253.86999999999998, -108.20000000000002) rectangle (263.52,-118.00000000000001);
\draw(254.86999999999998, -109.20000000000002) node[anchor=north west,align=left] {Classical \\ groups \\ (algebro-geometric\\ aspects)};
\draw (254.86999999999998, -109.20000000000002) rectangle (259.96999999999997,-111.30000000000001);
\draw(260.07, -109.20000000000002) node[anchor=north west,align=left] {Group \\ varieties};
\draw (260.07, -109.20000000000002) rectangle (262.92,-110.30000000000001);
\draw(254.86999999999998, -111.40000000000002) node[anchor=north west,align=left] {Affine algebraic\\ groups,\\ hyperalgebra\\ constructions};
\draw (254.86999999999998, -111.40000000000002) rectangle (259.46999999999997,-113.50000000000001);
\draw(259.57, -111.40000000000002) node[anchor=north west,align=left] {Group actions\\ on varieties\\ or schemes\\ (quotients)};
\draw (259.57, -111.40000000000002) rectangle (263.42,-113.50000000000001);
\draw(254.86999999999998, -113.60000000000002) node[anchor=north west,align=left] {Other \\ algebraic groups\\ (geometric\\ aspects)};
\draw (254.86999999999998, -113.60000000000002) rectangle (259.46999999999997,-115.70000000000002);
\draw(259.57, -113.60000000000002) node[anchor=north west,align=left] {Geometric\\ invariant\\ theory};
\draw (259.57, -113.60000000000002) rectangle (262.42,-115.20000000000002);
\draw(254.86999999999998, -115.80000000000001) node[anchor=north west,align=left] {Formal \\ groups, \\ \(p\)-divisible\\ groups};
\draw (254.86999999999998, -115.80000000000001) rectangle (259.21999999999997,-117.9);
\draw(259.32, -115.80000000000001) node[anchor=north west,align=left] {Group\\ schemes};
\draw (259.32, -115.80000000000001) rectangle (261.67,-116.9);
\draw(274.51, -1) node[anchor=north west,align=left] {\LARGE Commutative algebra};
\draw (274.51, -1) rectangle (314.82,-49.6);
\draw(275.51, -2) node[anchor=north west,align=left] {\large Chain conditions, finiteness conditions in commutative ring theory};
\draw (275.51, -2) rectangle (296.57,-6.199999999999999);
\draw(276.51, -3) node[anchor=north west,align=left] {Commutative \\ rings and modules\\ of finite \\ generation or \\ presentation; \\ number of generators};
\draw (276.51, -3) rectangle (282.11,-6.1);
\draw(282.21, -3) node[anchor=north west,align=left] {Commutative \\ Artinian rings\\ and modules,\\ finite-dimensional\\ algebras};
\draw (282.21, -3) rectangle (287.31,-5.6);
\draw(287.40999999999997, -3) node[anchor=north west,align=left] {Commutative\\ Noetherian\\ rings \\ and modules};
\draw (287.40999999999997, -3) rectangle (290.76,-5.1);
\draw(296.67, -2) node[anchor=north west,align=left] {\large Theory of modules and ideals in commutative rings};
\draw (296.67, -2) rectangle (314.72,-13.3);
\draw(297.67, -3) node[anchor=north west,align=left] {Structure, \\ classification \\ theorems for modules\\ and ideals in\\ commutative rings};
\draw (297.67, -3) rectangle (303.27000000000004,-5.6);
\draw(303.37, -3) node[anchor=north west,align=left] {Dimension \\ theory, depth, \\ related commutative\\ rings \\ (catenary, etc.)};
\draw (303.37, -3) rectangle (308.72,-5.6);
\draw(308.82, -3) node[anchor=north west,align=left] {Projective\\ and free \\ modules and \\ ideals in \\ commutative rings};
\draw (308.82, -3) rectangle (313.67,-5.6);
\draw(297.67, -5.7) node[anchor=north west,align=left] {Injective \\ and flat \\ modules and \\ ideals in \\ commutative rings};
\draw (297.67, -5.7) rectangle (302.52000000000004,-8.3);
\draw(302.62, -5.7) node[anchor=north west,align=left] {Torsion \\ modules and \\ ideals in \\ commutative rings};
\draw (302.62, -5.7) rectangle (307.47,-7.800000000000001);
\draw(307.57, -5.7) node[anchor=north west,align=left] {Other special\\ types of \\ modules and \\ ideals in \\ commutative rings};
\draw (307.57, -5.7) rectangle (312.42,-8.3);
\draw(312.52000000000004, -5.7) node[anchor=north west,align=left] {Class\\ groups};
\draw (312.52000000000004, -5.7) rectangle (314.62000000000006,-6.800000000000001);
\draw(297.67, -8.4) node[anchor=north west,align=left] {Linkage, \\ complete \\ intersections \\ and determinantal\\ ideals};
\draw (297.67, -8.4) rectangle (302.52000000000004,-11.0);
\draw(302.62, -8.4) node[anchor=north west,align=left] {Module \\ categories\\ and \\ commutative rings};
\draw (302.62, -8.4) rectangle (307.47,-10.5);
\draw(307.57, -8.4) node[anchor=north west,align=left] {Theory of modules\\ and ideals\\ in commutative\\ rings described\\ by combinatorial\\ properties};
\draw (307.57, -8.4) rectangle (312.42,-11.5);
\draw(297.67, -11.600000000000001) node[anchor=north west,align=left] {Cohen-Macaulay\\ modules};
\draw (297.67, -11.600000000000001) rectangle (301.77000000000004,-13.200000000000001);
\draw(275.51, -6.299999999999999) node[anchor=north west,align=left] {\large Computational aspects and applications of commutative rings};
\draw (275.51, -6.299999999999999) rectangle (296.06,-12.7);
\draw(276.51, -7.299999999999999) node[anchor=north west,align=left] {Applications of\\ commutative \\ algebra (e.g., to\\ statistics, \\ control theory, \\ optimization, etc.)};
\draw (276.51, -7.299999999999999) rectangle (281.86,-10.399999999999999);
\draw(281.96, -7.299999999999999) node[anchor=north west,align=left] {Gröbner bases; \\ other bases for \\ ideals and modules\\ (e.g., Janet \\ and border bases)};
\draw (281.96, -7.299999999999999) rectangle (287.06,-9.899999999999999);
\draw(287.15999999999997, -7.299999999999999) node[anchor=north west,align=left] {Polynomials,\\ factorization\\ in \\ commutative rings};
\draw (287.15999999999997, -7.299999999999999) rectangle (292.01,-9.399999999999999);
\draw(292.11, -7.299999999999999) node[anchor=north west,align=left] {Computational\\ homological\\ algebra};
\draw (292.11, -7.299999999999999) rectangle (295.96000000000004,-9.399999999999999);
\draw(276.51, -10.5) node[anchor=north west,align=left] {Solving \\ polynomial\\ systems;\\ resultants};
\draw (276.51, -10.5) rectangle (279.61,-12.6);
\draw(275.51, -13.4) node[anchor=north west,align=left] {\large Arithmetic rings and other special commutative rings};
\draw (275.51, -13.4) rectangle (295.31,-23.700000000000003);
\draw(276.51, -14.4) node[anchor=north west,align=left] {Commutative rings\\ defined by \\ monomial ideals;\\ Stanley-Reisner\\ face rings; \\ simplicial complexes};
\draw (276.51, -14.4) rectangle (282.11,-17.5);
\draw(282.21, -14.4) node[anchor=north west,align=left] {Commutative rings\\ defined by \\ factorization \\ properties (e.g.,\\ atomic, factorial,\\ half-factorial)};
\draw (282.21, -14.4) rectangle (287.31,-17.5);
\draw(287.40999999999997, -14.4) node[anchor=north west,align=left] {Polynomial \\ rings and \\ ideals; rings \\ of integer-valued\\ polynomials};
\draw (287.40999999999997, -14.4) rectangle (292.26,-17.0);
\draw(292.36, -14.4) node[anchor=north west,align=left] {Principal\\ ideal\\ rings};
\draw (292.36, -14.4) rectangle (295.21000000000004,-16.0);
\draw(276.51, -17.6) node[anchor=north west,align=left] {Commutative\\ rings defined\\ by binomial\\ ideals, \\ toric rings, etc.};
\draw (276.51, -17.6) rectangle (281.36,-20.200000000000003);
\draw(281.46, -17.6) node[anchor=north west,align=left] {Other \\ commutative rings\\ defined \\ by combinatorial\\ properties};
\draw (281.46, -17.6) rectangle (286.31,-20.200000000000003);
\draw(286.40999999999997, -17.6) node[anchor=north west,align=left] {Dedekind, \\ Prüfer, Krull\\ and Mori rings\\ and their\\ generalizations};
\draw (286.40999999999997, -17.6) rectangle (290.76,-20.200000000000003);
\draw(290.86, -17.6) node[anchor=north west,align=left] {Euclidean\\ rings\\ and \\ generalizations};
\draw (290.86, -17.6) rectangle (295.21000000000004,-19.700000000000003);
\draw(276.51, -20.3) node[anchor=north west,align=left] {Rings with\\ straightening\\ laws, \\ Hodge algebras};
\draw (276.51, -20.3) rectangle (280.61,-22.400000000000002);
\draw(280.71, -20.3) node[anchor=north west,align=left] {Witt vectors\\ and \\ related rings};
\draw (280.71, -20.3) rectangle (284.56,-21.900000000000002);
\draw(284.65999999999997, -20.3) node[anchor=north west,align=left] {Formal \\ power \\ series rings};
\draw (284.65999999999997, -20.3) rectangle (288.26,-21.900000000000002);
\draw(288.36, -20.3) node[anchor=north west,align=left] {Seminormal\\ rings};
\draw (288.36, -20.3) rectangle (291.46000000000004,-21.400000000000002);
\draw(291.56, -20.3) node[anchor=north west,align=left] {Valuation\\ rings};
\draw (291.56, -20.3) rectangle (294.41,-21.400000000000002);
\draw(276.51, -22.5) node[anchor=north west,align=left] {Excellent\\ rings};
\draw (276.51, -22.5) rectangle (279.36,-23.6);
\draw(279.46, -22.5) node[anchor=north west,align=left] {Cluster\\ algebras};
\draw (279.46, -22.5) rectangle (282.06,-23.6);
\draw(295.40999999999997, -13.4) node[anchor=north west,align=left] {\large Commutative ring extensions and related topics};
\draw (295.40999999999997, -13.4) rectangle (312.96,-22.5);
\draw(296.40999999999997, -14.4) node[anchor=north west,align=left] {Integral \\ closure of \\ commutative rings\\ and ideals};
\draw (296.40999999999997, -14.4) rectangle (301.26,-16.5);
\draw(301.35999999999996, -14.4) node[anchor=north west,align=left] {Rings of \\ fractions and\\ localization\\ for \\ commutative rings};
\draw (301.35999999999996, -14.4) rectangle (306.21,-17.0);
\draw(306.30999999999995, -14.4) node[anchor=north west,align=left] {Galois theory\\ and \\ commutative ring\\ extensions};
\draw (306.30999999999995, -14.4) rectangle (310.90999999999997,-16.5);
\draw(296.40999999999997, -17.1) node[anchor=north west,align=left] {Étale and \\ flat extensions;\\ Henselization;\\ Artin\\ approximation};
\draw (296.40999999999997, -17.1) rectangle (301.01,-19.700000000000003);
\draw(301.10999999999996, -17.1) node[anchor=north west,align=left] {Extension\\ theory \\ of commutative\\ rings};
\draw (301.10999999999996, -17.1) rectangle (305.21,-19.200000000000003);
\draw(305.30999999999995, -17.1) node[anchor=north west,align=left] {Morphisms\\ of commutative\\ rings};
\draw (305.30999999999995, -17.1) rectangle (309.40999999999997,-18.700000000000003);
\draw(309.51, -17.1) node[anchor=north west,align=left] {Polynomials\\ over\\ commutative\\ rings};
\draw (309.51, -17.1) rectangle (312.86,-19.200000000000003);
\draw(296.40999999999997, -19.8) node[anchor=north west,align=left] {Integral \\ dependence in \\ commutative \\ rings; going\\ up, going down};
\draw (296.40999999999997, -19.8) rectangle (300.51,-22.400000000000002);
\draw(300.60999999999996, -19.8) node[anchor=north west,align=left] {Completion\\ of commutative\\ rings};
\draw (300.60999999999996, -19.8) rectangle (304.71,-21.400000000000002);
\draw(295.40999999999997, -22.6) node[anchor=north west,align=left] {\large History of \\ commutative algebra};
\draw (295.40999999999997, -22.6) rectangle (301.9,-23.700000000000003);
\draw(275.51, -23.8) node[anchor=north west,align=left] {\large Homological methods in commutative ring theory};
\draw (275.51, -23.8) rectangle (292.61,-36.1);
\draw(276.51, -24.8) node[anchor=north west,align=left] {Homological \\ conjectures \\ (intersection \\ theorems) in \\ commutative ring theory};
\draw (276.51, -24.8) rectangle (282.86,-27.400000000000002);
\draw(282.96, -24.8) node[anchor=north west,align=left] {(Co)homology of \\ commutative rings\\ and algebras (e.g.,\\ Hochschild, \\ André-Quillen, cyclic,\\ dihedral, etc.)};
\draw (282.96, -24.8) rectangle (289.06,-27.900000000000002);
\draw(289.15999999999997, -24.8) node[anchor=north west,align=left] {Torsion \\ theory for\\ commutative\\ rings};
\draw (289.15999999999997, -24.8) rectangle (292.51,-26.900000000000002);
\draw(276.51, -28.0) node[anchor=north west,align=left] {Syzygies,\\ resolutions,\\ complexes\\ and \\ commutative rings};
\draw (276.51, -28.0) rectangle (281.36,-30.6);
\draw(281.46, -28.0) node[anchor=north west,align=left] {Homological\\ dimension\\ and \\ commutative rings};
\draw (281.46, -28.0) rectangle (286.31,-30.1);
\draw(286.40999999999997, -28.0) node[anchor=north west,align=left] {Homological \\ functors on \\ modules of \\ commutative rings\\ (Tor, Ext, etc.)};
\draw (286.40999999999997, -28.0) rectangle (291.26,-30.6);
\draw(276.51, -30.700000000000003) node[anchor=north west,align=left] {Deformations\\ and infinitesimal\\ methods\\ in commutative\\ ring theory};
\draw (276.51, -30.700000000000003) rectangle (281.36,-33.300000000000004);
\draw(281.46, -30.700000000000003) node[anchor=north west,align=left] {Derived \\ categories \\ and commutative\\ rings};
\draw (281.46, -30.700000000000003) rectangle (285.81,-32.800000000000004);
\draw(285.90999999999997, -30.700000000000003) node[anchor=north west,align=left] {Grothendieck\\ groups, \\ \(K\)-theory\\ and commutative\\ rings};
\draw (285.90999999999997, -30.700000000000003) rectangle (290.26,-33.300000000000004);
\draw(276.51, -33.400000000000006) node[anchor=north west,align=left] {Hilbert-Samuel\\ and \\ Hilbert-Kunz \\ functions; \\ Poincaré series};
\draw (276.51, -33.400000000000006) rectangle (280.86,-36.00000000000001);
\draw(280.96, -33.400000000000006) node[anchor=north west,align=left] {Local \\ cohomology \\ and commutative\\ rings};
\draw (280.96, -33.400000000000006) rectangle (285.31,-35.50000000000001);
\draw(292.71, -23.8) node[anchor=north west,align=left] {\large Applications of logic to commutative algebra};
\draw (292.71, -23.8) rectangle (306.95,-27.0);
\draw(293.71, -24.8) node[anchor=north west,align=left] {Applications\\ of logic\\ to commutative\\ algebra};
\draw (293.71, -24.8) rectangle (297.81,-26.900000000000002);
\draw(292.71, -27.1) node[anchor=north west,align=left] {\large Topological rings and modules};
\draw (292.71, -27.1) rectangle (303.76,-31.5);
\draw(293.71, -28.1) node[anchor=north west,align=left] {Global \\ topological\\ rings};
\draw (293.71, -28.1) rectangle (297.06,-29.700000000000003);
\draw(297.15999999999997, -28.1) node[anchor=north west,align=left] {Analytical\\ algebras\\ and rings};
\draw (297.15999999999997, -28.1) rectangle (300.26,-29.700000000000003);
\draw(300.35999999999996, -28.1) node[anchor=north west,align=left] {Complete\\ rings,\\ completion};
\draw (300.35999999999996, -28.1) rectangle (303.46,-29.700000000000003);
\draw(293.71, -29.8) node[anchor=north west,align=left] {Henselian\\ rings};
\draw (293.71, -29.8) rectangle (296.56,-30.900000000000002);
\draw(296.65999999999997, -29.8) node[anchor=north west,align=left] {Ordered\\ rings};
\draw (296.65999999999997, -29.8) rectangle (299.01,-30.900000000000002);
\draw(299.10999999999996, -29.8) node[anchor=north west,align=left] {Real \\ algebra};
\draw (299.10999999999996, -29.8) rectangle (301.46,-30.900000000000002);
\draw(301.56, -29.8) node[anchor=north west,align=left] {Power\\ series\\ rings};
\draw (301.56, -29.8) rectangle (303.66,-31.400000000000002);
\draw(292.71, -31.6) node[anchor=north west,align=left] {\large Finite commutative rings};
\draw (292.71, -31.6) rectangle (300.75,-34.800000000000004);
\draw(293.71, -32.6) node[anchor=north west,align=left] {Structure\\ of finite\\ commutative\\ rings};
\draw (293.71, -32.6) rectangle (297.06,-34.7);
\draw(297.15999999999997, -32.6) node[anchor=north west,align=left] {Polynomials\\ and finite\\ commutative\\ rings};
\draw (297.15999999999997, -32.6) rectangle (300.51,-34.7);
\draw(307.05, -23.8) node[anchor=north west,align=left] {\large Differential algebra};
\draw (307.05, -23.8) rectangle (313.85,-31.4);
\draw(308.05, -24.8) node[anchor=north west,align=left] {Modules\\ of \\ differentials};
\draw (308.05, -24.8) rectangle (311.90000000000003,-26.400000000000002);
\draw(308.05, -26.5) node[anchor=north west,align=left] {Commutative\\ rings of \\ differential \\ operators and\\ their modules};
\draw (308.05, -26.5) rectangle (311.90000000000003,-29.1);
\draw(308.05, -29.200000000000003) node[anchor=north west,align=left] {Derivations\\ and\\ commutative\\ rings};
\draw (308.05, -29.200000000000003) rectangle (311.40000000000003,-31.300000000000004);
\draw(275.51, -36.2) node[anchor=north west,align=left] {\large General commutative ring theory};
\draw (275.51, -36.2) rectangle (288.15999999999997,-49.5);
\draw(276.51, -37.2) node[anchor=north west,align=left] {Characteristic \\ \(p\) methods \\ (Frobenius endomorphism)\\ and reduction\\ to characteristic\\ \(p\); tight closure};
\draw (276.51, -37.2) rectangle (283.11,-40.300000000000004);
\draw(283.21, -37.2) node[anchor=north west,align=left] {Ideals and\\ multiplicative\\ ideal \\ theory in \\ commutative rings};
\draw (283.21, -37.2) rectangle (288.06,-39.800000000000004);
\draw(276.51, -40.400000000000006) node[anchor=north west,align=left] {General commutative\\ ring theory and\\ combinatorics \\ (zero-divisor graphs,\\ annihilating-ideal\\ graphs, etc.)};
\draw (276.51, -40.400000000000006) rectangle (282.36,-43.50000000000001);
\draw(282.46, -40.400000000000006) node[anchor=north west,align=left] {Divisibility\\ and factorizations\\ in \\ commutative rings};
\draw (282.46, -40.400000000000006) rectangle (287.56,-42.50000000000001);
\draw(276.51, -43.6) node[anchor=north west,align=left] {Associated graded\\ rings of \\ ideals (Rees ring,\\ form ring), \\ analytic spread \\ and related topics};
\draw (276.51, -43.6) rectangle (281.61,-46.7);
\draw(281.71, -43.6) node[anchor=north west,align=left] {Valuations\\ and their \\ generalizations\\ for \\ commutative rings};
\draw (281.71, -43.6) rectangle (286.56,-46.2);
\draw(276.51, -46.800000000000004) node[anchor=north west,align=left] {Actions of\\ groups on \\ commutative\\ rings; \\ invariant theory};
\draw (276.51, -46.800000000000004) rectangle (281.11,-49.400000000000006);
\draw(281.21, -46.800000000000004) node[anchor=north west,align=left] {Graded\\ rings};
\draw (281.21, -46.800000000000004) rectangle (283.31,-47.900000000000006);
\draw(288.26, -36.2) node[anchor=north west,align=left] {\large Local rings and semilocal rings};
\draw (288.26, -36.2) rectangle (298.46999999999997,-41.6);
\draw(289.26, -37.2) node[anchor=north west,align=left] {Special types\\ (Cohen-Macaulay, \\ Gorenstein, \\ Buchsbaum, etc.)};
\draw (289.26, -37.2) rectangle (294.11,-39.800000000000004);
\draw(294.21, -37.2) node[anchor=north west,align=left] {Multiplicity\\ theory\\ and \\ related topics};
\draw (294.21, -37.2) rectangle (298.31,-39.300000000000004);
\draw(289.26, -39.900000000000006) node[anchor=north west,align=left] {Regular\\ local\\ rings};
\draw (289.26, -39.900000000000006) rectangle (291.61,-41.50000000000001);
\draw(288.26, -41.7) node[anchor=north west,align=left] {\large Integral domains};
\draw (288.26, -41.7) rectangle (293.82,-43.900000000000006);
\draw(289.26, -42.7) node[anchor=north west,align=left] {Integral\\ domains};
\draw (289.26, -42.7) rectangle (291.86,-43.800000000000004);
\draw(274.51, -49.7) node[anchor=north west,align=left] {\LARGE Real functions};
\draw (274.51, -49.7) rectangle (305.96,-88.0);
\draw(275.51, -50.7) node[anchor=north west,align=left] {\large Miscellaneous topics in real functions};
\draw (275.51, -50.7) rectangle (291.90999999999997,-60.5);
\draw(276.51, -51.7) node[anchor=north west,align=left] {\(C^\infty\)-functions,quasi-analytic\\ functions};
\draw (276.51, -51.7) rectangle (286.36,-53.800000000000004);
\draw(286.46, -51.7) node[anchor=north west,align=left] {Nonstandardanalysis};
\draw (286.46, -51.7) rectangle (291.81,-53.300000000000004);
\draw(276.51, -53.900000000000006) node[anchor=north west,align=left] {Non-Archimedeananalysis};
\draw (276.51, -53.900000000000006) rectangle (282.86,-55.50000000000001);
\draw(282.96, -53.900000000000006) node[anchor=north west,align=left] {Real-analyticfunctions};
\draw (282.96, -53.900000000000006) rectangle (289.06,-55.50000000000001);
\draw(289.15999999999997, -53.900000000000006) node[anchor=north west,align=left] {Fuzzy\\ real \\ analysis};
\draw (289.15999999999997, -53.900000000000006) rectangle (291.76,-55.50000000000001);
\draw(276.51, -55.6) node[anchor=north west,align=left] {Calculus of \\ functions on \\ infinite-dimensional\\ spaces};
\draw (276.51, -55.6) rectangle (282.11,-57.7);
\draw(282.21, -55.6) node[anchor=north west,align=left] {Calculus of \\ functions taking\\ values in\\ infinite-dimensional\\ spaces};
\draw (282.21, -55.6) rectangle (287.81,-58.2);
\draw(287.90999999999997, -55.6) node[anchor=north west,align=left] {Constructive\\ real\\ analysis};
\draw (287.90999999999997, -55.6) rectangle (291.51,-57.2);
\draw(287.90999999999997, -57.300000000000004) node[anchor=north west,align=left] {Means};
\draw (287.90999999999997, -57.300000000000004) rectangle (289.76,-57.900000000000006);
\draw(276.51, -58.300000000000004) node[anchor=north west,align=left] {Real analysis\\ on time\\ scales or \\ measure chains};
\draw (276.51, -58.300000000000004) rectangle (280.61,-60.400000000000006);
\draw(280.71, -58.300000000000004) node[anchor=north west,align=left] {Set-valued\\ functions};
\draw (280.71, -58.300000000000004) rectangle (283.81,-59.900000000000006);
\draw(292.01, -50.7) node[anchor=north west,align=left] {\large Functions of one variable};
\draw (292.01, -50.7) rectangle (305.86,-73.30000000000001);
\draw(293.01, -51.7) node[anchor=north west,align=left] {Continuity and \\ related questions \\ (modulus of continuity,\\ semicontinuity,\\ discontinuities,\\ etc.) for real \\ functions in one variable};
\draw (293.01, -51.7) rectangle (299.86,-55.300000000000004);
\draw(299.96, -51.7) node[anchor=north west,align=left] {Foundations: \\ limits and \\ generalizations, \\ elementary \\ topology of the line};
\draw (299.96, -51.7) rectangle (305.56,-54.300000000000004);
\draw(293.01, -55.400000000000006) node[anchor=north west,align=left] {Nondifferentiability\\ (nondifferentiable\\ functions, \\ points of \\ nondifferentiability), \\ discontinuous derivatives};
\draw (293.01, -55.400000000000006) rectangle (299.86,-58.50000000000001);
\draw(299.96, -55.400000000000006) node[anchor=north west,align=left] {Singular functions,\\ Cantor \\ functions, functions\\ with other \\ special properties};
\draw (299.96, -55.400000000000006) rectangle (305.56,-58.00000000000001);
\draw(293.01, -58.6) node[anchor=north west,align=left] {Differentiation \\ (real functions of \\ one variable): general\\ theory, generalized\\ derivatives,\\ mean value theorems};
\draw (293.01, -58.6) rectangle (299.11,-61.7);
\draw(299.21, -58.6) node[anchor=north west,align=left] {Antidifferentiation};
\draw (299.21, -58.6) rectangle (304.56,-60.2);
\draw(293.01, -61.800000000000004) node[anchor=north west,align=left] {Classification\\ of real functions;\\ Baire \\ classification of \\ sets and functions};
\draw (293.01, -61.800000000000004) rectangle (298.11,-64.4);
\draw(298.21, -61.800000000000004) node[anchor=north west,align=left] {Rate of growth\\ of functions,\\ orders of \\ infinity, slowly\\ varying functions};
\draw (298.21, -61.800000000000004) rectangle (303.06,-64.4);
\draw(293.01, -64.5) node[anchor=north west,align=left] {Denjoy and \\ Perron integrals,\\ other special\\ integrals};
\draw (293.01, -64.5) rectangle (297.86,-66.6);
\draw(297.96, -64.5) node[anchor=north west,align=left] {Functions \\ of bounded \\ variation, \\ generalizations};
\draw (297.96, -64.5) rectangle (302.31,-66.6);
\draw(302.40999999999997, -64.5) node[anchor=north west,align=left] {Fractional\\ derivatives\\ and\\ integrals};
\draw (302.40999999999997, -64.5) rectangle (305.76,-66.6);
\draw(293.01, -66.7) node[anchor=north west,align=left] {Absolutely \\ continuous real\\ functions in\\ one variable};
\draw (293.01, -66.7) rectangle (297.36,-68.8);
\draw(297.46, -66.7) node[anchor=north west,align=left] {Monotonic\\ functions,\\ generalizations};
\draw (297.46, -66.7) rectangle (301.81,-68.8);
\draw(301.90999999999997, -66.7) node[anchor=north west,align=left] {Integrals of\\ Riemann, \\ Stieltjes and\\ Lebesgue type};
\draw (301.90999999999997, -66.7) rectangle (305.76,-68.8);
\draw(293.01, -68.9) node[anchor=north west,align=left] {Convexity of\\ real functions\\ in one \\ variable, \\ generalizations};
\draw (293.01, -68.9) rectangle (297.36,-71.5);
\draw(297.46, -68.9) node[anchor=north west,align=left] {One-variable\\ calculus};
\draw (297.46, -68.9) rectangle (301.06,-70.5);
\draw(301.15999999999997, -68.9) node[anchor=north west,align=left] {Iteration\\ of real \\ functions in\\ one variable};
\draw (301.15999999999997, -68.9) rectangle (304.76,-71.0);
\draw(293.01, -71.6) node[anchor=north west,align=left] {Elementary\\ functions};
\draw (293.01, -71.6) rectangle (296.11,-73.19999999999999);
\draw(296.21, -71.6) node[anchor=north west,align=left] {Lipschitz\\ (Hölder)\\ classes};
\draw (296.21, -71.6) rectangle (299.06,-73.19999999999999);
\draw(275.51, -60.6) node[anchor=north west,align=left] {\large Polynomials, rational functions in real analysis};
\draw (275.51, -60.6) rectangle (290.99,-63.800000000000004);
\draw(276.51, -61.6) node[anchor=north west,align=left] {Real \\ polynomials: \\ analytic \\ properties, etc.};
\draw (276.51, -61.6) rectangle (281.11,-63.7);
\draw(281.21, -61.6) node[anchor=north west,align=left] {Real \\ polynomials:\\ location\\ of zeros};
\draw (281.21, -61.6) rectangle (284.81,-63.7);
\draw(284.90999999999997, -61.6) node[anchor=north west,align=left] {Real \\ rational\\ functions};
\draw (284.90999999999997, -61.6) rectangle (287.76,-63.2);
\draw(275.51, -63.900000000000006) node[anchor=north west,align=left] {\large Inequalities in real analysis};
\draw (275.51, -63.900000000000006) rectangle (285.90999999999997,-71.5);
\draw(276.51, -64.9) node[anchor=north west,align=left] {Inequalities \\ involving \\ derivatives and \\ differential and\\ integral operators};
\draw (276.51, -64.9) rectangle (281.61,-67.5);
\draw(281.71, -64.9) node[anchor=north west,align=left] {Inequalities\\ for \\ trigonometric \\ functions and\\ polynomials};
\draw (281.71, -64.9) rectangle (285.81,-67.5);
\draw(276.51, -67.60000000000001) node[anchor=north west,align=left] {Inequalities\\ for sums,\\ series\\ and integrals};
\draw (276.51, -67.60000000000001) rectangle (280.36,-69.7);
\draw(280.46, -67.60000000000001) node[anchor=north west,align=left] {Inequalities\\ involving\\ other types\\ of functions};
\draw (280.46, -67.60000000000001) rectangle (284.06,-69.7);
\draw(276.51, -69.80000000000001) node[anchor=north west,align=left] {Other \\ analytical \\ inequalities};
\draw (276.51, -69.80000000000001) rectangle (280.11,-71.4);
\draw(275.51, -71.60000000000001) node[anchor=north west,align=left] {\large Computational methods\\ for problems pertaining\\ to real functions};
\draw (275.51, -71.60000000000001) rectangle (283.24,-73.2);
\draw(275.51, -73.4) node[anchor=north west,align=left] {\large Functions of several variables};
\draw (275.51, -73.4) rectangle (287.15999999999997,-87.9);
\draw(276.51, -74.4) node[anchor=north west,align=left] {Implicit function\\ theorems, \\ Jacobians, \\ transformations with\\ several variables};
\draw (276.51, -74.4) rectangle (282.11,-77.0);
\draw(282.21, -74.4) node[anchor=north west,align=left] {Integral formulas\\ of real \\ functions of \\ several variables\\ (Stokes, Gauss,\\ Green, etc.)};
\draw (282.21, -74.4) rectangle (287.06,-77.5);
\draw(276.51, -77.60000000000001) node[anchor=north west,align=left] {Absolutely \\ continuous real \\ functions of \\ several variables,\\ functions \\ of bounded variation};
\draw (276.51, -77.60000000000001) rectangle (282.11,-80.7);
\draw(282.21, -77.60000000000001) node[anchor=north west,align=left] {Continuity\\ and \\ differentiation\\ questions};
\draw (282.21, -77.60000000000001) rectangle (286.56,-79.7);
\draw(276.51, -80.80000000000001) node[anchor=north west,align=left] {Integration of\\ real functions\\ of several \\ variables: length,\\ area, volume};
\draw (276.51, -80.80000000000001) rectangle (281.61,-83.4);
\draw(281.71, -80.80000000000001) node[anchor=north west,align=left] {Special properties\\ of functions\\ of several \\ variables, Hölder\\ conditions, etc.};
\draw (281.71, -80.80000000000001) rectangle (286.81,-83.4);
\draw(276.51, -83.5) node[anchor=north west,align=left] {Convexity of\\ real functions\\ of several\\ variables, \\ generalizations};
\draw (276.51, -83.5) rectangle (280.86,-86.1);
\draw(280.96, -83.5) node[anchor=north west,align=left] {Representation\\ and \\ superposition\\ of functions};
\draw (280.96, -83.5) rectangle (285.06,-85.6);
\draw(276.51, -86.2) node[anchor=north west,align=left] {Calculus\\ of vector\\ functions};
\draw (276.51, -86.2) rectangle (279.36,-87.8);
\draw(287.26, -73.4) node[anchor=north west,align=left] {\large History of \\ real functions};
\draw (287.26, -73.4) rectangle (292.2,-74.5);
\draw(274.51, -88.10000000000001) node[anchor=north west,align=left] {\LARGE Nonassociative rings and algebras};
\draw (274.51, -88.10000000000001) rectangle (305.86,-142.20000000000002);
\draw(275.51, -89.10000000000001) node[anchor=north west,align=left] {\large Jordan algebras (algebras, triples and pairs)};
\draw (275.51, -89.10000000000001) rectangle (292.56,-98.9);
\draw(276.51, -90.10000000000001) node[anchor=north west,align=left] {Jordan \\ structures associated\\ with \\ other structures};
\draw (276.51, -90.10000000000001) rectangle (282.36,-92.2);
\draw(282.46, -90.10000000000001) node[anchor=north west,align=left] {Finite-dimensional\\ structures of \\ Jordan algebras};
\draw (282.46, -90.10000000000001) rectangle (287.56,-92.2);
\draw(287.65999999999997, -90.10000000000001) node[anchor=north west,align=left] {Associated \\ groups, \\ automorphisms of\\ Jordan algebras};
\draw (287.65999999999997, -90.10000000000001) rectangle (292.26,-92.2);
\draw(276.51, -92.30000000000001) node[anchor=north west,align=left] {Associated\\ manifolds\\ of \\ Jordan algebras};
\draw (276.51, -92.30000000000001) rectangle (280.86,-94.4);
\draw(280.96, -92.30000000000001) node[anchor=north west,align=left] {Idempotents,\\ Peirce \\ decompositions};
\draw (280.96, -92.30000000000001) rectangle (285.06,-94.4);
\draw(285.15999999999997, -92.30000000000001) node[anchor=north west,align=left] {Jordan \\ structures on \\ Banach spaces\\ and algebras};
\draw (285.15999999999997, -92.30000000000001) rectangle (289.26,-94.4);
\draw(289.36, -92.30000000000001) node[anchor=north west,align=left] {Identities\\ and free\\ Jordan\\ structures};
\draw (289.36, -92.30000000000001) rectangle (292.46000000000004,-94.4);
\draw(276.51, -94.50000000000001) node[anchor=north west,align=left] {Applications\\ of Jordan \\ algebras to \\ physics, etc.};
\draw (276.51, -94.50000000000001) rectangle (280.36,-96.60000000000001);
\draw(280.46, -94.50000000000001) node[anchor=north west,align=left] {Exceptional\\ Jordan\\ structures};
\draw (280.46, -94.50000000000001) rectangle (283.81,-96.10000000000001);
\draw(283.90999999999997, -94.50000000000001) node[anchor=north west,align=left] {Structure\\ theory \\ for Jordan\\ algebras};
\draw (283.90999999999997, -94.50000000000001) rectangle (287.01,-96.60000000000001);
\draw(287.11, -94.50000000000001) node[anchor=north west,align=left] {Simple,\\ semisimple\\ Jordan\\ algebras};
\draw (287.11, -94.50000000000001) rectangle (290.21000000000004,-96.60000000000001);
\draw(276.51, -96.7) node[anchor=north west,align=left] {Associated\\ geometries\\ of Jordan\\ algebras};
\draw (276.51, -96.7) rectangle (279.61,-98.8);
\draw(279.71, -96.7) node[anchor=north west,align=left] {Division\\ algebras\\ and Jordan\\ algebras};
\draw (279.71, -96.7) rectangle (282.81,-98.8);
\draw(282.90999999999997, -96.7) node[anchor=north west,align=left] {Super \\ structures};
\draw (282.90999999999997, -96.7) rectangle (286.01,-97.8);
\draw(286.11, -96.7) node[anchor=north west,align=left] {Radicals\\ in Jordan\\ algebras};
\draw (286.11, -96.7) rectangle (288.96000000000004,-98.3);
\draw(292.65999999999997, -89.10000000000001) node[anchor=north west,align=left] {\large Lie algebras and Lie superalgebras};
\draw (292.65999999999997, -89.10000000000001) rectangle (305.76,-122.2);
\draw(293.65999999999997, -90.10000000000001) node[anchor=north west,align=left] {Lie (super)algebras\\ associated \\ with other structures\\ (associative,\\ Jordan, etc.)};
\draw (293.65999999999997, -90.10000000000001) rectangle (299.51,-92.7);
\draw(299.60999999999996, -90.10000000000001) node[anchor=north west,align=left] {Kac-Moody \\ (super)algebras; \\ extended affine Lie\\ algebras; \\ toroidal Lie algebras};
\draw (299.60999999999996, -90.10000000000001) rectangle (305.46,-92.7);
\draw(293.65999999999997, -92.80000000000001) node[anchor=north west,align=left] {Quantum groups\\ (quantized \\ enveloping algebras)\\ and related\\ deformations};
\draw (293.65999999999997, -92.80000000000001) rectangle (299.26,-95.4);
\draw(299.35999999999996, -92.80000000000001) node[anchor=north west,align=left] {Infinite-dimensional\\ Lie \\ (super)algebras};
\draw (299.35999999999996, -92.80000000000001) rectangle (304.96,-94.9);
\draw(293.65999999999997, -95.50000000000001) node[anchor=north west,align=left] {Automorphisms,\\ derivations, \\ other operators \\ for Lie algebras\\ and super algebras};
\draw (293.65999999999997, -95.50000000000001) rectangle (298.76,-98.10000000000001);
\draw(298.85999999999996, -95.50000000000001) node[anchor=north west,align=left] {Applications\\ of Lie algebras\\ and \\ superalgebras to \\ integrable systems};
\draw (298.85999999999996, -95.50000000000001) rectangle (303.96,-98.10000000000001);
\draw(293.65999999999997, -98.20000000000002) node[anchor=north west,align=left] {Applications\\ of Lie \\ (super)algebras to\\ physics, etc.};
\draw (293.65999999999997, -98.20000000000002) rectangle (298.76,-100.30000000000001);
\draw(298.85999999999996, -98.20000000000002) node[anchor=north west,align=left] {Vertex operators;\\ vertex \\ operator algebras\\ and related\\ structures};
\draw (298.85999999999996, -98.20000000000002) rectangle (303.71,-100.80000000000001);
\draw(293.65999999999997, -100.9) node[anchor=north west,align=left] {Representations\\ of Lie \\ algebras and Lie\\ superalgebras,\\ algebraic\\ theory (weights)};
\draw (293.65999999999997, -100.9) rectangle (298.26,-104.0);
\draw(298.35999999999996, -100.9) node[anchor=north west,align=left] {Lie algebras\\ of \\ linear \\ algebraic groups};
\draw (298.35999999999996, -100.9) rectangle (302.96,-103.0);
\draw(303.05999999999995, -100.9) node[anchor=north west,align=left] {Poisson\\ algebras};
\draw (303.05999999999995, -100.9) rectangle (305.65999999999997,-102.0);
\draw(293.65999999999997, -104.10000000000001) node[anchor=north west,align=left] {Lie algebras\\ of vector\\ fields and\\ related \\ (super) algebras};
\draw (293.65999999999997, -104.10000000000001) rectangle (298.26,-106.7);
\draw(298.35999999999996, -104.10000000000001) node[anchor=north west,align=left] {Identities,\\ free \\ Lie \\ (super)algebras};
\draw (298.35999999999996, -104.10000000000001) rectangle (302.71,-106.2);
\draw(302.80999999999995, -104.10000000000001) node[anchor=north west,align=left] {Coadjoint\\ orbits;\\ nilpotent\\ varieties};
\draw (302.80999999999995, -104.10000000000001) rectangle (305.65999999999997,-106.2);
\draw(293.65999999999997, -106.80000000000001) node[anchor=north west,align=left] {Representations\\ of Lie algebras\\ and Lie \\ superalgebras, \\ analytic theory};
\draw (293.65999999999997, -106.80000000000001) rectangle (298.01,-109.4);
\draw(298.10999999999996, -106.80000000000001) node[anchor=north west,align=left] {Simple, \\ semisimple, \\ reductive \\ (super)algebras};
\draw (298.10999999999996, -106.80000000000001) rectangle (302.46,-108.9);
\draw(302.55999999999995, -106.80000000000001) node[anchor=north west,align=left] {Root \\ systems};
\draw (302.55999999999995, -106.80000000000001) rectangle (304.90999999999997,-107.9);
\draw(293.65999999999997, -109.5) node[anchor=north west,align=left] {Exceptional\\ (super)algebras};
\draw (293.65999999999997, -109.5) rectangle (298.01,-111.1);
\draw(298.10999999999996, -109.5) node[anchor=north west,align=left] {Solvable,\\ nilpotent\\ (super)algebras};
\draw (298.10999999999996, -109.5) rectangle (302.46,-111.6);
\draw(293.65999999999997, -111.7) node[anchor=north west,align=left] {Universal\\ enveloping\\ (super)algebras};
\draw (293.65999999999997, -111.7) rectangle (298.01,-113.8);
\draw(298.10999999999996, -111.7) node[anchor=north west,align=left] {Yang-Baxter\\ equations\\ and Rota-Baxter\\ operators};
\draw (298.10999999999996, -111.7) rectangle (302.46,-113.8);
\draw(293.65999999999997, -113.9) node[anchor=north west,align=left] {Modular \\ Lie \\ (super)algebras};
\draw (293.65999999999997, -113.9) rectangle (298.01,-115.5);
\draw(298.10999999999996, -113.9) node[anchor=north west,align=left] {Homological\\ methods\\ in Lie \\ (super)algebras};
\draw (298.10999999999996, -113.9) rectangle (302.46,-116.0);
\draw(293.65999999999997, -116.10000000000001) node[anchor=north west,align=left] {Cohomology\\ of Lie\\ (super)algebras};
\draw (293.65999999999997, -116.10000000000001) rectangle (298.01,-118.2);
\draw(298.10999999999996, -116.10000000000001) node[anchor=north west,align=left] {Lie bialgebras;\\ Lie\\ coalgebras};
\draw (298.10999999999996, -116.10000000000001) rectangle (302.46,-117.7);
\draw(293.65999999999997, -118.30000000000001) node[anchor=north west,align=left] {Graded \\ Lie \\ (super)algebras};
\draw (293.65999999999997, -118.30000000000001) rectangle (298.01,-119.9);
\draw(298.10999999999996, -118.30000000000001) node[anchor=north west,align=left] {Color Lie\\ (super)algebras};
\draw (298.10999999999996, -118.30000000000001) rectangle (302.46,-119.9);
\draw(293.65999999999997, -120.0) node[anchor=north west,align=left] {Structure \\ theory for Lie\\ algebras and\\ superalgebras};
\draw (293.65999999999997, -120.0) rectangle (297.76,-122.1);
\draw(297.85999999999996, -120.0) node[anchor=north west,align=left] {Hom-Lie \\ and related\\ algebras};
\draw (297.85999999999996, -120.0) rectangle (301.21,-121.6);
\draw(301.30999999999995, -120.0) node[anchor=north west,align=left] {Virasoro\\ and related\\ algebras};
\draw (301.30999999999995, -120.0) rectangle (304.65999999999997,-121.6);
\draw(275.51, -99.00000000000001) node[anchor=north west,align=left] {\large Other nonassociative rings and algebras};
\draw (275.51, -99.00000000000001) rectangle (289.56,-106.10000000000001);
\draw(276.51, -100.00000000000001) node[anchor=north west,align=left] {\((\gamma,~\delta)\)-rings,\\ including \\ \((1,-1)\)-rings};
\draw (276.51, -100.00000000000001) rectangle (283.86,-102.60000000000001);
\draw(283.96, -100.00000000000001) node[anchor=north west,align=left] {Lie-admissible\\ algebras};
\draw (283.96, -100.00000000000001) rectangle (288.06,-101.60000000000001);
\draw(276.51, -102.70000000000002) node[anchor=north west,align=left] {Alternative\\ rings};
\draw (276.51, -102.70000000000002) rectangle (279.86,-104.30000000000001);
\draw(279.96, -102.70000000000002) node[anchor=north west,align=left] {Right \\ alternative\\ rings};
\draw (279.96, -102.70000000000002) rectangle (283.31,-104.30000000000001);
\draw(283.40999999999997, -102.70000000000002) node[anchor=north west,align=left] {(non-Lie)\\ Hom \\ algebras \\ and topics};
\draw (283.40999999999997, -102.70000000000002) rectangle (286.51,-104.80000000000001);
\draw(286.61, -102.70000000000002) node[anchor=north west,align=left] {Mal’tsev\\ rings and\\ algebras};
\draw (286.61, -102.70000000000002) rectangle (289.46000000000004,-104.30000000000001);
\draw(276.51, -104.90000000000002) node[anchor=north west,align=left] {Genetic\\ algebras};
\draw (276.51, -104.90000000000002) rectangle (279.11,-106.00000000000001);
\draw(275.51, -106.20000000000002) node[anchor=north west,align=left] {\large Computational methods\\ for problems \\ pertaining to nonassociative\\ rings and algebras};
\draw (275.51, -106.20000000000002) rectangle (284.78999999999996,-108.30000000000001);
\draw(275.51, -108.4) node[anchor=north west,align=left] {\large History of \\ nonassociative \\ rings and algebras};
\draw (275.51, -108.4) rectangle (281.69,-110.0);
\draw(275.51, -122.30000000000001) node[anchor=north west,align=left] {\large General nonassociative rings};
\draw (275.51, -122.30000000000001) rectangle (286.65999999999997,-142.10000000000002);
\draw(276.51, -123.30000000000001) node[anchor=north west,align=left] {General theory\\ of \\ nonassociative rings\\ and algebras};
\draw (276.51, -123.30000000000001) rectangle (282.11,-125.4);
\draw(282.21, -123.30000000000001) node[anchor=north west,align=left] {Radical theory\\ (nonassociative\\ rings\\ and algebras)};
\draw (282.21, -123.30000000000001) rectangle (286.56,-125.4);
\draw(276.51, -125.50000000000001) node[anchor=north west,align=left] {Automorphisms,\\ derivations, \\ other operators \\ (nonassociative\\ rings and algebras)};
\draw (276.51, -125.50000000000001) rectangle (281.86,-128.10000000000002);
\draw(281.96, -125.50000000000001) node[anchor=north west,align=left] {Nonassociative\\ algebras\\ satisfying \\ other identities};
\draw (281.96, -125.50000000000001) rectangle (286.56,-127.60000000000001);
\draw(276.51, -128.20000000000002) node[anchor=north west,align=left] {Compositionalgebras};
\draw (276.51, -128.20000000000002) rectangle (281.86,-129.8);
\draw(281.96, -128.20000000000002) node[anchor=north west,align=left] {Quadratic\\ algebras \\ (but not \\ quadratic \\ Jordan algebras)};
\draw (281.96, -128.20000000000002) rectangle (286.56,-130.8);
\draw(276.51, -130.9) node[anchor=north west,align=left] {Power-associative\\ rings};
\draw (276.51, -130.9) rectangle (281.36,-132.5);
\draw(281.46, -130.9) node[anchor=north west,align=left] {Gröbner-Shirshov\\ bases \\ in nonassociative\\ algebras};
\draw (281.46, -130.9) rectangle (286.31,-133.0);
\draw(276.51, -133.10000000000002) node[anchor=north west,align=left] {Noncommutative\\ Jordan\\ algebras};
\draw (276.51, -133.10000000000002) rectangle (280.61,-134.70000000000002);
\draw(280.71, -133.10000000000002) node[anchor=north west,align=left] {Nonassociative\\ division\\ algebras};
\draw (280.71, -133.10000000000002) rectangle (284.81,-134.70000000000002);
\draw(276.51, -134.8) node[anchor=north west,align=left] {Free \\ nonassociative\\ algebras};
\draw (276.51, -134.8) rectangle (280.61,-136.4);
\draw(280.71, -134.8) node[anchor=north west,align=left] {Structure \\ theory for \\ nonassociative\\ algebras};
\draw (280.71, -134.8) rectangle (284.81,-136.9);
\draw(276.51, -137.0) node[anchor=north west,align=left] {Other \\ \(n\)-ary \\ compositions \\ \((n~\ge~3)\)};
\draw (276.51, -137.0) rectangle (280.36,-139.1);
\draw(280.46, -137.0) node[anchor=north west,align=left] {Superalgebras};
\draw (280.46, -137.0) rectangle (284.31,-138.1);
\draw(276.51, -139.20000000000002) node[anchor=north west,align=left] {Ternary\\ compositions};
\draw (276.51, -139.20000000000002) rectangle (280.11,-140.8);
\draw(280.21, -139.20000000000002) node[anchor=north west,align=left] {Flexible\\ algebras};
\draw (280.21, -139.20000000000002) rectangle (282.81,-140.8);
\draw(282.90999999999997, -139.20000000000002) node[anchor=north west,align=left] {Leibniz\\ algebras};
\draw (282.90999999999997, -139.20000000000002) rectangle (285.51,-140.3);
\draw(276.51, -140.9) node[anchor=north west,align=left] {Valued\\ algebras};
\draw (276.51, -140.9) rectangle (279.11,-142.0);
\draw(314.92, -1) node[anchor=north west,align=left] {\LARGE Topological groups, Lie groups};
\draw (314.92, -1) rectangle (345.90000000000003,-37.6);
\draw(315.92, -2) node[anchor=north west,align=left] {\large Topological and differentiable algebraic systems};
\draw (315.92, -2) rectangle (331.77000000000004,-10.6);
\draw(316.92, -3) node[anchor=north west,align=left] {Topological \\ groupoids \\ (including \\ differentiable and\\ Lie groupoids)};
\draw (316.92, -3) rectangle (322.02000000000004,-5.6);
\draw(322.12, -3) node[anchor=north west,align=left] {Other topological\\ algebraic\\ systems\\ and their \\ representations};
\draw (322.12, -3) rectangle (326.97,-5.6);
\draw(327.07, -3) node[anchor=north west,align=left] {Topological\\ semilattices,\\ lattices \\ and applications};
\draw (327.07, -3) rectangle (331.67,-5.1);
\draw(316.92, -5.7) node[anchor=north west,align=left] {Representations\\ of general\\ topological\\ groups and\\ semigroups};
\draw (316.92, -5.7) rectangle (321.27000000000004,-8.3);
\draw(321.37, -5.7) node[anchor=north west,align=left] {Structure\\ of general\\ topological\\ groups};
\draw (321.37, -5.7) rectangle (324.72,-7.800000000000001);
\draw(324.82, -5.7) node[anchor=north west,align=left] {Analysis\\ on general\\ topological\\ groups};
\draw (324.82, -5.7) rectangle (328.17,-7.800000000000001);
\draw(328.27000000000004, -5.7) node[anchor=north west,align=left] {Structure\\ of \\ topological\\ semigroups};
\draw (328.27000000000004, -5.7) rectangle (331.62000000000006,-7.800000000000001);
\draw(316.92, -8.4) node[anchor=north west,align=left] {Analysis\\ on \\ topological\\ semigroups};
\draw (316.92, -8.4) rectangle (320.27000000000004,-10.5);
\draw(331.87, -2) node[anchor=north west,align=left] {\large Locally compact abelian groups (LCA groups)};
\draw (331.87, -2) rectangle (345.8,-5.2);
\draw(332.87, -3) node[anchor=north west,align=left] {General \\ properties and\\ structure\\ of LCA groups};
\draw (332.87, -3) rectangle (336.97,-5.1);
\draw(337.07, -3) node[anchor=north west,align=left] {Structure\\ of group \\ algebras of\\ LCA groups};
\draw (337.07, -3) rectangle (340.42,-5.1);
\draw(331.87, -5.3) node[anchor=north west,align=left] {\large Computational methods\\ for problems pertaining\\ to topological groups};
\draw (331.87, -5.3) rectangle (339.6,-6.9);
\draw(331.87, -7.0) node[anchor=north west,align=left] {\large History of \\ topological groups};
\draw (331.87, -7.0) rectangle (338.05,-8.1);
\draw(331.87, -8.2) node[anchor=north west,align=left] {\large Compact groups};
\draw (331.87, -8.2) rectangle (336.81,-10.399999999999999);
\draw(332.87, -9.2) node[anchor=north west,align=left] {Compact\\ groups};
\draw (332.87, -9.2) rectangle (335.22,-10.299999999999999);
\draw(315.92, -10.7) node[anchor=north west,align=left] {\large Locally compact groups and their algebras};
\draw (315.92, -10.7) rectangle (330.77000000000004,-22.0);
\draw(316.92, -11.7) node[anchor=north west,align=left] {\(C^*\)-algebras\\ and \\ \(W^*\)-algebras in \\ relation to group\\ representations};
\draw (316.92, -11.7) rectangle (322.52000000000004,-14.299999999999999);
\draw(322.62, -11.7) node[anchor=north west,align=left] {Representationsof\\ group\\ algebras};
\draw (322.62, -11.7) rectangle (327.47,-13.799999999999999);
\draw(327.57, -11.7) node[anchor=north west,align=left] {Rigidity\\ in locally\\ compact\\ groups};
\draw (327.57, -11.7) rectangle (330.67,-13.799999999999999);
\draw(316.92, -14.399999999999999) node[anchor=north west,align=left] {Kazhdan’s \\ property (T), the\\ Haagerup \\ property, and \\ generalizations};
\draw (316.92, -14.399999999999999) rectangle (321.77000000000004,-17.0);
\draw(321.87, -14.399999999999999) node[anchor=north west,align=left] {Unitary \\ representations\\ of locally \\ compact groups};
\draw (321.87, -14.399999999999999) rectangle (326.22,-16.5);
\draw(326.32, -14.399999999999999) node[anchor=north west,align=left] {Other \\ representations\\ of locally\\ compact groups};
\draw (326.32, -14.399999999999999) rectangle (330.67,-16.5);
\draw(316.92, -17.1) node[anchor=north west,align=left] {Group \\ algebras of \\ locally compact\\ groups};
\draw (316.92, -17.1) rectangle (321.27000000000004,-19.200000000000003);
\draw(321.37, -17.1) node[anchor=north west,align=left] {Induced \\ representations\\ for locally\\ compact groups};
\draw (321.37, -17.1) rectangle (325.72,-19.200000000000003);
\draw(325.82, -17.1) node[anchor=north west,align=left] {General \\ properties and\\ structure\\ of locally\\ compact groups};
\draw (325.82, -17.1) rectangle (329.92,-19.700000000000003);
\draw(316.92, -19.8) node[anchor=north west,align=left] {Duality \\ theorems for\\ locally \\ compact groups};
\draw (316.92, -19.8) rectangle (321.02000000000004,-21.900000000000002);
\draw(321.12, -19.8) node[anchor=north west,align=left] {Automorphism\\ groups of\\ locally \\ compact groups};
\draw (321.12, -19.8) rectangle (325.22,-21.900000000000002);
\draw(325.32, -19.8) node[anchor=north west,align=left] {Ergodic\\ theory \\ on groups};
\draw (325.32, -19.8) rectangle (328.17,-21.400000000000002);
\draw(330.87, -10.7) node[anchor=north west,align=left] {\large Lie groups};
\draw (330.87, -10.7) rectangle (343.77,-37.5);
\draw(331.87, -11.7) node[anchor=north west,align=left] {Geometric \\ Langlands \\ program: \\ representation-theoretic\\ aspects};
\draw (331.87, -11.7) rectangle (338.47,-14.299999999999999);
\draw(338.57, -11.7) node[anchor=north west,align=left] {General \\ properties and \\ structure of other\\ Lie groups};
\draw (338.57, -11.7) rectangle (343.67,-13.799999999999999);
\draw(331.87, -14.399999999999999) node[anchor=north west,align=left] {Representations\\ of nilpotent and\\ solvable Lie \\ groups (special \\ orbital integrals,\\ non-type I \\ representations, etc.)};
\draw (331.87, -14.399999999999999) rectangle (337.97,-18.0);
\draw(338.07, -14.399999999999999) node[anchor=north west,align=left] {Infinite-dimensional\\ Lie \\ groups and their\\ Lie algebras:\\ general properties};
\draw (338.07, -14.399999999999999) rectangle (343.67,-17.0);
\draw(331.87, -18.1) node[anchor=north west,align=left] {Representations\\ of Lie and \\ real algebraic\\ groups: algebraic\\ methods \\ (Verma modules, etc.)};
\draw (331.87, -18.1) rectangle (337.72,-21.200000000000003);
\draw(337.82, -18.1) node[anchor=north west,align=left] {Analysis on \\ and representations\\ of \\ infinite-dimensional\\ Lie groups};
\draw (337.82, -18.1) rectangle (343.42,-20.700000000000003);
\draw(331.87, -21.3) node[anchor=north west,align=left] {Representations\\ of Lie and\\ linear algebraic\\ groups \\ over local fields};
\draw (331.87, -21.3) rectangle (336.72,-23.900000000000002);
\draw(336.82, -21.3) node[anchor=north west,align=left] {Representations\\ of Lie and \\ linear algebraic\\ groups over\\ real fields: \\ analytic methods};
\draw (336.82, -21.3) rectangle (341.42,-24.400000000000002);
\draw(331.87, -24.5) node[anchor=north west,align=left] {Representations\\ of Lie and \\ linear algebraic\\ groups over\\ global fields\\ and adèle rings};
\draw (331.87, -24.5) rectangle (336.47,-27.6);
\draw(336.57, -24.5) node[anchor=north west,align=left] {Applications\\ of Lie groups\\ to the sciences;\\ explicit\\ representations};
\draw (336.57, -24.5) rectangle (341.17,-27.1);
\draw(331.87, -27.7) node[anchor=north west,align=left] {General \\ properties and \\ structure of\\ real Lie groups};
\draw (331.87, -27.7) rectangle (336.22,-29.8);
\draw(336.32, -27.7) node[anchor=north west,align=left] {Semisimple\\ Lie groups\\ and their \\ representations};
\draw (336.32, -27.7) rectangle (340.67,-29.8);
\draw(340.77, -27.7) node[anchor=north west,align=left] {Local Lie\\ groups};
\draw (340.77, -27.7) rectangle (343.62,-28.8);
\draw(331.87, -29.9) node[anchor=north west,align=left] {Loop groups\\ and related\\ constructions,\\ group-theoretic\\ treatment};
\draw (331.87, -29.9) rectangle (336.22,-32.5);
\draw(336.32, -29.9) node[anchor=north west,align=left] {Structure and\\ representation\\ of the\\ Lorentz group};
\draw (336.32, -29.9) rectangle (340.42,-32.0);
\draw(340.52, -29.9) node[anchor=north west,align=left] {Nilpotent\\ and \\ solvable \\ Lie groups};
\draw (340.52, -29.9) rectangle (343.62,-32.0);
\draw(331.87, -32.599999999999994) node[anchor=north west,align=left] {General \\ properties \\ and structure\\ of complex\\ Lie groups};
\draw (331.87, -32.599999999999994) rectangle (335.72,-35.199999999999996);
\draw(335.82, -32.599999999999994) node[anchor=north west,align=left] {Discrete \\ subgroups \\ of Lie groups};
\draw (335.82, -32.599999999999994) rectangle (339.67,-34.199999999999996);
\draw(339.77, -32.599999999999994) node[anchor=north west,align=left] {Continuous\\ cohomologyof\\ Lie groups};
\draw (339.77, -32.599999999999994) rectangle (343.37,-34.699999999999996);
\draw(331.87, -35.3) node[anchor=north west,align=left] {Analysis\\ on real \\ and complex\\ Lie groups};
\draw (331.87, -35.3) rectangle (335.22,-37.4);
\draw(335.32, -35.3) node[anchor=north west,align=left] {Analysis\\ on \\ \(p\)-adic \\ Lie groups};
\draw (335.32, -35.3) rectangle (338.67,-37.4);
\draw(338.77, -35.3) node[anchor=north west,align=left] {Lie \\ algebras of\\ Lie groups};
\draw (338.77, -35.3) rectangle (342.12,-36.9);
\draw(315.92, -22.1) node[anchor=north west,align=left] {\large Noncompact transformation groups};
\draw (315.92, -22.1) rectangle (327.32,-27.5);
\draw(316.92, -23.1) node[anchor=north west,align=left] {General \\ theory of group\\ and \\ pseudogroup actions};
\draw (316.92, -23.1) rectangle (322.27000000000004,-25.200000000000003);
\draw(322.37, -23.1) node[anchor=north west,align=left] {Homogeneousspaces};
\draw (322.37, -23.1) rectangle (327.22,-24.700000000000003);
\draw(316.92, -25.3) node[anchor=north west,align=left] {Groups as\\ automorphisms\\ of other\\ structures};
\draw (316.92, -25.3) rectangle (320.77000000000004,-27.400000000000002);
\draw(320.87, -25.3) node[anchor=north west,align=left] {Measurable\\ group\\ actions};
\draw (320.87, -25.3) rectangle (323.97,-26.900000000000002);
\draw(314.92, -37.7) node[anchor=north west,align=left] {\LARGE Mathematical logic and foundations};
\draw (314.92, -37.7) rectangle (344.32,-117.39999999999999);
\draw(315.92, -38.7) node[anchor=north west,align=left] {\large Proof theory and constructive mathematics};
\draw (315.92, -38.7) rectangle (331.52000000000004,-51.7);
\draw(316.92, -39.7) node[anchor=north west,align=left] {Provability \\ logics and related\\ algebras \\ (e.g., diagonalizable\\ algebras)};
\draw (316.92, -39.7) rectangle (322.77000000000004,-42.300000000000004);
\draw(322.87, -39.7) node[anchor=north west,align=left] {Proof-theoretic\\ aspects of \\ linear logic and\\ other \\ substructural logics};
\draw (322.87, -39.7) rectangle (328.47,-42.300000000000004);
\draw(328.57, -39.7) node[anchor=north west,align=left] {Structure\\ of\\ proofs};
\draw (328.57, -39.7) rectangle (331.42,-41.300000000000004);
\draw(316.92, -42.400000000000006) node[anchor=north west,align=left] {Proof theory,\\ general\\ (including\\ proof-theoretic\\ semantics)};
\draw (316.92, -42.400000000000006) rectangle (321.27000000000004,-45.00000000000001);
\draw(321.37, -42.400000000000006) node[anchor=north west,align=left] {Cut-elimination\\ and\\ normal-form\\ theorems};
\draw (321.37, -42.400000000000006) rectangle (325.72,-44.50000000000001);
\draw(325.82, -42.400000000000006) node[anchor=north west,align=left] {Relative \\ consistency\\ and \\ interpretations};
\draw (325.82, -42.400000000000006) rectangle (330.17,-44.50000000000001);
\draw(316.92, -45.1) node[anchor=north west,align=left] {Gödel \\ numberings and \\ issues of \\ incompleteness};
\draw (316.92, -45.1) rectangle (321.27000000000004,-47.2);
\draw(321.37, -45.1) node[anchor=north west,align=left] {Metamathematics\\ of\\ constructive\\ systems};
\draw (321.37, -45.1) rectangle (325.72,-47.2);
\draw(325.82, -45.1) node[anchor=north west,align=left] {Intuitionistic\\ mathematics};
\draw (325.82, -45.1) rectangle (329.92,-46.7);
\draw(316.92, -47.300000000000004) node[anchor=north west,align=left] {Second- and\\ higher-order\\ arithmetic\\ and fragments};
\draw (316.92, -47.300000000000004) rectangle (320.77000000000004,-49.400000000000006);
\draw(320.87, -47.300000000000004) node[anchor=north west,align=left] {Other \\ constructive \\ mathematics};
\draw (320.87, -47.300000000000004) rectangle (324.72,-48.900000000000006);
\draw(324.82, -47.300000000000004) node[anchor=north west,align=left] {Constructive\\ and\\ recursive\\ analysis};
\draw (324.82, -47.300000000000004) rectangle (328.42,-49.400000000000006);
\draw(316.92, -49.5) node[anchor=north west,align=left] {Functionals\\ in proof\\ theory};
\draw (316.92, -49.5) rectangle (320.27000000000004,-51.1);
\draw(320.37, -49.5) node[anchor=north west,align=left] {Recursive\\ ordinals \\ and ordinal\\ notations};
\draw (320.37, -49.5) rectangle (323.72,-51.6);
\draw(323.82, -49.5) node[anchor=north west,align=left] {First-order\\ arithmetic\\ and\\ fragments};
\draw (323.82, -49.5) rectangle (327.17,-51.6);
\draw(327.27000000000004, -49.5) node[anchor=north west,align=left] {Complexity\\ of proofs};
\draw (327.27000000000004, -49.5) rectangle (330.37000000000006,-51.1);
\draw(331.62, -38.7) node[anchor=north west,align=left] {\large Computability and recursion theory};
\draw (331.62, -38.7) rectangle (344.22,-63.0);
\draw(332.62, -39.7) node[anchor=north west,align=left] {Computability\\ and recursion\\ theory on \\ ordinals, \\ admissible sets, etc.};
\draw (332.62, -39.7) rectangle (338.47,-42.300000000000004);
\draw(338.57, -39.7) node[anchor=north west,align=left] {Complexity of\\ computation \\ (including implicit\\ computational\\ complexity)};
\draw (338.57, -39.7) rectangle (343.92,-42.300000000000004);
\draw(332.62, -42.400000000000006) node[anchor=north west,align=left] {Computation\\ over the \\ reals, \\ computable analysis};
\draw (332.62, -42.400000000000006) rectangle (337.97,-44.50000000000001);
\draw(338.07, -42.400000000000006) node[anchor=north west,align=left] {Other degrees\\ and reducibilities\\ in \\ computability and \\ recursion theory};
\draw (338.07, -42.400000000000006) rectangle (343.17,-45.00000000000001);
\draw(332.62, -45.1) node[anchor=north west,align=left] {Recursive \\ equivalence\\ types of \\ sets and \\ structures, isols};
\draw (332.62, -45.1) rectangle (337.47,-47.7);
\draw(337.57, -45.1) node[anchor=north west,align=left] {Word problems,\\ etc. in\\ computability\\ and \\ recursion theory};
\draw (337.57, -45.1) rectangle (342.17,-47.7);
\draw(332.62, -47.800000000000004) node[anchor=north west,align=left] {Hierarchies\\ of computability\\ and\\ definability};
\draw (332.62, -47.800000000000004) rectangle (337.22,-49.900000000000006);
\draw(337.32, -47.800000000000004) node[anchor=north west,align=left] {Abstract and\\ axiomatic\\ computability\\ and \\ recursion theory};
\draw (337.32, -47.800000000000004) rectangle (341.92,-50.400000000000006);
\draw(332.62, -50.5) node[anchor=north west,align=left] {Applications\\ of computability\\ and \\ recursion theory};
\draw (332.62, -50.5) rectangle (337.22,-52.6);
\draw(337.32, -50.5) node[anchor=north west,align=left] {Automata and\\ formal grammars\\ in connection\\ with logical\\ questions};
\draw (337.32, -50.5) rectangle (341.67,-53.1);
\draw(332.62, -53.2) node[anchor=north west,align=left] {Recursively \\ (computably) \\ enumerable sets\\ and degrees};
\draw (332.62, -53.2) rectangle (336.97,-55.300000000000004);
\draw(337.07, -53.2) node[anchor=north west,align=left] {Undecidability\\ and \\ degrees of sets\\ of sentences};
\draw (337.07, -53.2) rectangle (341.42,-55.300000000000004);
\draw(332.62, -55.400000000000006) node[anchor=north west,align=left] {Thue and\\ Post \\ systems, etc.};
\draw (332.62, -55.400000000000006) rectangle (336.47,-57.00000000000001);
\draw(336.57, -55.400000000000006) node[anchor=north west,align=left] {Recursive \\ functions and\\ relations,\\ subrecursive\\ hierarchies};
\draw (336.57, -55.400000000000006) rectangle (340.42,-58.00000000000001);
\draw(340.52, -55.400000000000006) node[anchor=north west,align=left] {Other Turing\\ degree\\ structures};
\draw (340.52, -55.400000000000006) rectangle (344.12,-57.00000000000001);
\draw(332.62, -58.1) node[anchor=north west,align=left] {Theory of \\ numerations,\\ effectively\\ presented\\ structures};
\draw (332.62, -58.1) rectangle (336.22,-60.7);
\draw(336.32, -58.1) node[anchor=north west,align=left] {Inductive\\ definability};
\draw (336.32, -58.1) rectangle (339.92,-59.7);
\draw(340.02, -58.1) node[anchor=north west,align=left] {Turing \\ machines\\ and related\\ notions};
\draw (340.02, -58.1) rectangle (343.37,-60.2);
\draw(332.62, -60.8) node[anchor=north west,align=left] {Algorithmic\\ randomness\\ and\\ dimension};
\draw (332.62, -60.8) rectangle (335.97,-62.9);
\draw(336.07, -60.8) node[anchor=north west,align=left] {Higher-type\\ and set\\ recursion\\ theory};
\draw (336.07, -60.8) rectangle (339.42,-62.9);
\draw(315.92, -51.800000000000004) node[anchor=north west,align=left] {\large Philosophical aspects of logic and foundations};
\draw (315.92, -51.800000000000004) rectangle (330.78000000000003,-55.50000000000001);
\draw(316.92, -52.800000000000004) node[anchor=north west,align=left] {Philosophical\\ and \\ critical aspects\\ of logic\\ and foundations};
\draw (316.92, -52.800000000000004) rectangle (321.52000000000004,-55.400000000000006);
\draw(321.62, -52.800000000000004) node[anchor=north west,align=left] {Logic in\\ the \\ philosophy\\ of science};
\draw (321.62, -52.800000000000004) rectangle (324.72,-54.900000000000006);
\draw(315.92, -55.60000000000001) node[anchor=north west,align=left] {\large Computational methods\\ for problems \\ pertaining to mathematical\\ logic and foundations};
\draw (315.92, -55.60000000000001) rectangle (324.58000000000004,-57.70000000000001);
\draw(315.92, -57.800000000000004) node[anchor=north west,align=left] {\large History of \\ mathematical logic\\ and foundations};
\draw (315.92, -57.800000000000004) rectangle (322.1,-59.400000000000006);
\draw(315.92, -63.10000000000001) node[anchor=north west,align=left] {\large General logic};
\draw (315.92, -63.10000000000001) rectangle (328.07,-83.9);
\draw(316.92, -64.10000000000001) node[anchor=north west,align=left] {Substructural \\ logics (including \\ relevance, entailment,\\ linear logic,\\ Lambek calculus,\\ BCK and BCI logics)};
\draw (316.92, -64.10000000000001) rectangle (323.02000000000004,-67.2);
\draw(323.12, -64.10000000000001) node[anchor=north west,align=left] {Probability\\ and \\ inductive logic};
\draw (323.12, -64.10000000000001) rectangle (327.47,-65.7);
\draw(323.12, -65.80000000000001) node[anchor=north west,align=left] {Temporal\\ logic};
\draw (323.12, -65.80000000000001) rectangle (325.72,-66.9);
\draw(316.92, -67.30000000000001) node[anchor=north west,align=left] {Subsystems \\ of classical\\ logic (including\\ intuitionistic logic)};
\draw (316.92, -67.30000000000001) rectangle (322.77000000000004,-69.9);
\draw(322.87, -67.30000000000001) node[anchor=north west,align=left] {Foundations \\ of classical \\ theories \\ (including reverse\\ mathematics)};
\draw (322.87, -67.30000000000001) rectangle (327.97,-69.9);
\draw(316.92, -70.00000000000001) node[anchor=north west,align=left] {Logics of \\ knowledge and\\ belief \\ (including \\ belief change)};
\draw (316.92, -70.00000000000001) rectangle (321.02000000000004,-72.60000000000001);
\draw(321.12, -70.00000000000001) node[anchor=north west,align=left] {Paraconsistent\\ logics};
\draw (321.12, -70.00000000000001) rectangle (325.22,-71.60000000000001);
\draw(325.32, -70.00000000000001) node[anchor=north west,align=left] {Combined\\ logics};
\draw (325.32, -70.00000000000001) rectangle (327.92,-71.10000000000001);
\draw(316.92, -72.70000000000002) node[anchor=north west,align=left] {Classical\\ propositional\\ logic};
\draw (316.92, -72.70000000000002) rectangle (320.77000000000004,-74.30000000000001);
\draw(320.87, -72.70000000000002) node[anchor=north west,align=left] {Mechanization\\ of proofs\\ and logical\\ operations};
\draw (320.87, -72.70000000000002) rectangle (324.72,-74.80000000000001);
\draw(324.82, -72.70000000000002) node[anchor=north west,align=left] {Abstract\\ deductive\\ systems};
\draw (324.82, -72.70000000000002) rectangle (327.67,-74.30000000000001);
\draw(316.92, -74.9) node[anchor=north west,align=left] {Higher-order\\ logic};
\draw (316.92, -74.9) rectangle (320.52000000000004,-76.5);
\draw(320.62, -74.9) node[anchor=north west,align=left] {Decidability\\ of theories\\ and sets\\ of sentences};
\draw (320.62, -74.9) rectangle (324.22,-77.0);
\draw(324.32, -74.9) node[anchor=north west,align=left] {Fuzzy logic;\\ logic \\ of vagueness};
\draw (324.32, -74.9) rectangle (327.92,-76.5);
\draw(316.92, -77.10000000000001) node[anchor=north west,align=left] {Intermediate\\ logics};
\draw (316.92, -77.10000000000001) rectangle (320.52000000000004,-78.7);
\draw(320.62, -77.10000000000001) node[anchor=north west,align=left] {Other \\ nonclassical\\ logic};
\draw (320.62, -77.10000000000001) rectangle (324.22,-78.7);
\draw(324.32, -77.10000000000001) node[anchor=north west,align=left] {Other \\ applications\\ of logic};
\draw (324.32, -77.10000000000001) rectangle (327.92,-78.7);
\draw(316.92, -78.80000000000001) node[anchor=north west,align=left] {Classical\\ first-order\\ logic};
\draw (316.92, -78.80000000000001) rectangle (320.27000000000004,-80.4);
\draw(320.37, -78.80000000000001) node[anchor=north west,align=left] {Combinatory\\ logic\\ and lambda\\ calculus};
\draw (320.37, -78.80000000000001) rectangle (323.72,-80.9);
\draw(323.82, -78.80000000000001) node[anchor=north west,align=left] {Modal logic\\ (including\\ the logic\\ of norms)};
\draw (323.82, -78.80000000000001) rectangle (327.17,-80.9);
\draw(316.92, -81.0) node[anchor=north west,align=left] {Many-valued\\ logic};
\draw (316.92, -81.0) rectangle (320.27000000000004,-82.6);
\draw(320.37, -81.0) node[anchor=north west,align=left] {Logic of\\ natural\\ languages};
\draw (320.37, -81.0) rectangle (323.22,-82.6);
\draw(323.32, -81.0) node[anchor=north west,align=left] {Logic in\\ computer\\ science};
\draw (323.32, -81.0) rectangle (325.92,-82.6);
\draw(316.92, -82.7) node[anchor=north west,align=left] {Type\\ theory};
\draw (316.92, -82.7) rectangle (319.02000000000004,-83.8);
\draw(328.17, -63.10000000000001) node[anchor=north west,align=left] {\large Set theory};
\draw (328.17, -63.10000000000001) rectangle (340.02000000000004,-82.70000000000002);
\draw(329.17, -64.10000000000001) node[anchor=north west,align=left] {Other classical\\ set theory \\ (including functions,\\ relations,\\ and set algebra)};
\draw (329.17, -64.10000000000001) rectangle (335.02000000000004,-66.7);
\draw(335.12, -64.10000000000001) node[anchor=north west,align=left] {Cardinal \\ characteristics\\ of the\\ continuum};
\draw (335.12, -64.10000000000001) rectangle (339.47,-66.2);
\draw(329.17, -66.80000000000001) node[anchor=north west,align=left] {Other aspects\\ of forcing\\ and \\ Boolean-valued models};
\draw (329.17, -66.80000000000001) rectangle (335.02000000000004,-68.9);
\draw(335.12, -66.80000000000001) node[anchor=north west,align=left] {Continuum\\ hypothesis\\ and \\ Martin’s axiom};
\draw (335.12, -66.80000000000001) rectangle (339.22,-68.9);
\draw(329.17, -69.00000000000001) node[anchor=north west,align=left] {Inner models, \\ including \\ constructibility, \\ ordinal definability,\\ and core models};
\draw (329.17, -69.00000000000001) rectangle (335.02000000000004,-71.60000000000001);
\draw(335.12, -69.00000000000001) node[anchor=north west,align=left] {Generic \\ absoluteness\\ and \\ forcing axioms};
\draw (335.12, -69.00000000000001) rectangle (339.22,-71.10000000000001);
\draw(329.17, -71.70000000000002) node[anchor=north west,align=left] {Ordered sets\\ and their\\ cofinalities;\\ pcf theory};
\draw (329.17, -71.70000000000002) rectangle (333.02000000000004,-73.80000000000001);
\draw(333.12, -71.70000000000002) node[anchor=north west,align=left] {Other \\ combinatorial\\ set theory};
\draw (333.12, -71.70000000000002) rectangle (336.97,-73.30000000000001);
\draw(337.07, -71.70000000000002) node[anchor=north west,align=left] {Partition\\ relations};
\draw (337.07, -71.70000000000002) rectangle (339.92,-73.30000000000001);
\draw(329.17, -73.9) node[anchor=north west,align=left] {Axiomatics of\\ classical set\\ theory and\\ its fragments};
\draw (329.17, -73.9) rectangle (333.02000000000004,-76.0);
\draw(333.12, -73.9) node[anchor=north west,align=left] {Other notions\\ of \\ set-theoretic\\ definability};
\draw (333.12, -73.9) rectangle (336.97,-76.0);
\draw(337.07, -73.9) node[anchor=north west,align=left] {Large \\ cardinals};
\draw (337.07, -73.9) rectangle (339.92,-75.0);
\draw(329.17, -76.10000000000001) node[anchor=north west,align=left] {Other \\ set-theoretic\\ hypotheses\\ and axioms};
\draw (329.17, -76.10000000000001) rectangle (333.02000000000004,-78.2);
\draw(333.12, -76.10000000000001) node[anchor=north west,align=left] {Ordinal \\ and cardinal\\ numbers};
\draw (333.12, -76.10000000000001) rectangle (336.72,-77.7);
\draw(336.82, -76.10000000000001) node[anchor=north west,align=left] {Theory \\ of fuzzy\\ sets, etc.};
\draw (336.82, -76.10000000000001) rectangle (339.92,-77.7);
\draw(329.17, -78.30000000000001) node[anchor=north west,align=left] {Axiom of \\ choice and\\ related \\ propositions};
\draw (329.17, -78.30000000000001) rectangle (332.77000000000004,-80.4);
\draw(332.87, -78.30000000000001) node[anchor=north west,align=left] {Consistency\\ and \\ independence\\ results};
\draw (332.87, -78.30000000000001) rectangle (336.47,-80.4);
\draw(336.57, -78.30000000000001) node[anchor=north west,align=left] {Descriptive\\ set\\ theory};
\draw (336.57, -78.30000000000001) rectangle (339.92,-79.9);
\draw(329.17, -80.5) node[anchor=north west,align=left] {Nonclassical\\ and \\ second-order\\ set theories};
\draw (329.17, -80.5) rectangle (332.77000000000004,-82.6);
\draw(332.87, -80.5) node[anchor=north west,align=left] {Applications\\ of\\ set theory};
\draw (332.87, -80.5) rectangle (336.47,-82.1);
\draw(336.57, -80.5) node[anchor=north west,align=left] {Determinacy\\ principles};
\draw (336.57, -80.5) rectangle (339.92,-82.1);
\draw(315.92, -84.0) node[anchor=north west,align=left] {\large Model theory};
\draw (315.92, -84.0) rectangle (326.07,-117.3);
\draw(316.92, -85.0) node[anchor=north west,align=left] {Equational \\ classes, universal\\ algebra \\ in model theory};
\draw (316.92, -85.0) rectangle (322.02000000000004,-87.1);
\draw(322.12, -85.0) node[anchor=north west,align=left] {Ultraproducts\\ and\\ related \\ constructions};
\draw (322.12, -85.0) rectangle (325.97,-87.1);
\draw(316.92, -87.2) node[anchor=north west,align=left] {Quantifier \\ elimination,\\ model \\ completeness and \\ related topics};
\draw (316.92, -87.2) rectangle (321.77000000000004,-89.8);
\draw(321.87, -87.2) node[anchor=north west,align=left] {Basic \\ properties of \\ first-order \\ languages and\\ structures};
\draw (321.87, -87.2) rectangle (325.97,-89.8);
\draw(316.92, -89.9) node[anchor=north west,align=left] {Computable \\ structure theory,\\ computable\\ model theory};
\draw (316.92, -89.9) rectangle (321.77000000000004,-92.0);
\draw(321.87, -89.9) node[anchor=north west,align=left] {Model theory\\ of denumerable\\ and separable\\ structures};
\draw (321.87, -89.9) rectangle (325.97,-92.0);
\draw(316.92, -92.1) node[anchor=north west,align=left] {Continuous\\ model theory,\\ model \\ theory of \\ metric structures};
\draw (316.92, -92.1) rectangle (321.77000000000004,-94.69999999999999);
\draw(321.87, -92.1) node[anchor=north west,align=left] {Interpolation,\\ preservation, \\ definability};
\draw (321.87, -92.1) rectangle (325.97,-94.19999999999999);
\draw(316.92, -94.8) node[anchor=north west,align=left] {Classification\\ theory, \\ stability and \\ related concepts\\ in model theory};
\draw (316.92, -94.8) rectangle (321.52000000000004,-97.39999999999999);
\draw(321.62, -94.8) node[anchor=north west,align=left] {Model-theoretic\\ forcing};
\draw (321.62, -94.8) rectangle (325.97,-96.39999999999999);
\draw(316.92, -97.5) node[anchor=north west,align=left] {Nonclassical\\ models \\ (Boolean-valued,\\ sheaf, etc.)};
\draw (316.92, -97.5) rectangle (321.52000000000004,-99.6);
\draw(321.62, -97.5) node[anchor=north west,align=left] {Model-theoretic\\ algebra};
\draw (321.62, -97.5) rectangle (325.97,-99.1);
\draw(316.92, -99.7) node[anchor=north west,align=left] {Abstract \\ elementary \\ classes and \\ related topics};
\draw (316.92, -99.7) rectangle (321.02000000000004,-101.8);
\draw(321.12, -99.7) node[anchor=north west,align=left] {Other \\ model \\ constructions};
\draw (321.12, -99.7) rectangle (324.97,-101.3);
\draw(316.92, -101.9) node[anchor=north west,align=left] {Set-theoretic\\ model theory};
\draw (316.92, -101.9) rectangle (320.77000000000004,-103.5);
\draw(320.87, -101.9) node[anchor=north west,align=left] {Models \\ of arithmetic\\ and\\ set theory};
\draw (320.87, -101.9) rectangle (324.72,-104.0);
\draw(316.92, -104.1) node[anchor=north west,align=left] {Logic with\\ extra \\ quantifiers \\ and operators};
\draw (316.92, -104.1) rectangle (320.77000000000004,-106.19999999999999);
\draw(320.87, -104.1) node[anchor=north west,align=left] {Second- \\ and \\ higher-order \\ model theory};
\draw (320.87, -104.1) rectangle (324.72,-106.19999999999999);
\draw(316.92, -106.3) node[anchor=north west,align=left] {Categoricity\\ and \\ completeness\\ of theories};
\draw (316.92, -106.3) rectangle (320.52000000000004,-108.39999999999999);
\draw(320.62, -106.3) node[anchor=north west,align=left] {Models with\\ special \\ properties \\ (saturated, \\ rigid, etc.)};
\draw (320.62, -106.3) rectangle (324.22,-108.89999999999999);
\draw(316.92, -109.0) node[anchor=north west,align=left] {Model theory\\ of ordered\\ structures;\\ o-minimality};
\draw (316.92, -109.0) rectangle (320.52000000000004,-111.1);
\draw(320.62, -109.0) node[anchor=north west,align=left] {Models of\\ other \\ mathematical\\ theories};
\draw (320.62, -109.0) rectangle (324.22,-111.1);
\draw(316.92, -111.19999999999999) node[anchor=north west,align=left] {Other \\ classical \\ first-order \\ model theory};
\draw (316.92, -111.19999999999999) rectangle (320.52000000000004,-113.29999999999998);
\draw(320.62, -111.19999999999999) node[anchor=north west,align=left] {Applications\\ of \\ model theory};
\draw (320.62, -111.19999999999999) rectangle (324.22,-112.79999999999998);
\draw(316.92, -113.39999999999999) node[anchor=north west,align=left] {Model \\ theory of\\ finite \\ structures};
\draw (316.92, -113.39999999999999) rectangle (320.02000000000004,-115.49999999999999);
\draw(320.12, -113.39999999999999) node[anchor=north west,align=left] {Properties\\ of classes\\ of models};
\draw (320.12, -113.39999999999999) rectangle (323.22,-114.99999999999999);
\draw(323.32, -113.39999999999999) node[anchor=north west,align=left] {Abstract\\ model\\ theory};
\draw (323.32, -113.39999999999999) rectangle (325.92,-114.99999999999999);
\draw(316.92, -115.6) node[anchor=north west,align=left] {Logic on\\ admissible\\ sets};
\draw (316.92, -115.6) rectangle (320.02000000000004,-117.19999999999999);
\draw(320.12, -115.6) node[anchor=north west,align=left] {Other \\ infinitary\\ logic};
\draw (320.12, -115.6) rectangle (323.22,-117.19999999999999);
\draw(326.17, -84.0) node[anchor=north west,align=left] {\large Algebraic logic};
\draw (326.17, -84.0) rectangle (335.57,-93.3);
\draw(327.17, -85.0) node[anchor=north west,align=left] {Cylindric and\\ polyadic \\ algebras; \\ relation algebras};
\draw (327.17, -85.0) rectangle (332.02000000000004,-87.1);
\draw(332.12, -85.0) node[anchor=north west,align=left] {Categorical\\ logic,\\ topoi};
\draw (332.12, -85.0) rectangle (335.47,-86.6);
\draw(327.17, -87.2) node[anchor=north west,align=left] {Logical aspects\\ of \\ Łukasiewicz and \\ Post algebras};
\draw (327.17, -87.2) rectangle (331.77000000000004,-89.3);
\draw(331.87, -87.2) node[anchor=north west,align=left] {Logical\\ aspects\\ of Boolean\\ algebras};
\draw (331.87, -87.2) rectangle (334.97,-89.3);
\draw(327.17, -89.4) node[anchor=north west,align=left] {Other \\ algebras related\\ to logic};
\draw (327.17, -89.4) rectangle (331.77000000000004,-91.0);
\draw(331.87, -89.4) node[anchor=north west,align=left] {Abstract\\ algebraic\\ logic};
\draw (331.87, -89.4) rectangle (334.72,-91.0);
\draw(327.17, -91.1) node[anchor=north west,align=left] {Logical aspects\\ of lattices\\ and related\\ structures};
\draw (327.17, -91.1) rectangle (331.52000000000004,-93.19999999999999);
\draw(331.62, -91.1) node[anchor=north west,align=left] {Quantum\\ logic};
\draw (331.62, -91.1) rectangle (333.97,-92.19999999999999);
\draw(326.17, -93.4) node[anchor=north west,align=left] {\large Nonstandard models};
\draw (326.17, -93.4) rectangle (334.57,-99.30000000000001);
\draw(327.17, -94.4) node[anchor=north west,align=left] {Other applications\\ of \\ nonstandard models\\ (economics,\\ physics, etc.)};
\draw (327.17, -94.4) rectangle (332.27000000000004,-97.0);
\draw(327.17, -97.10000000000001) node[anchor=north west,align=left] {Nonstandard\\ models \\ of arithmetic};
\draw (327.17, -97.10000000000001) rectangle (331.02000000000004,-98.7);
\draw(331.12, -97.10000000000001) node[anchor=north west,align=left] {Nonstandard\\ models\\ in \\ mathematics};
\draw (331.12, -97.10000000000001) rectangle (334.47,-99.2);
\draw(346.00000000000006, -1) node[anchor=north west,align=left] {\LARGE K-Theory};
\draw (346.00000000000006, -1) rectangle (374.45000000000005,-38.6);
\draw(347.00000000000006, -2) node[anchor=north west,align=left] {\large Higher algebraic \(K\)-theory};
\draw (347.00000000000006, -2) rectangle (362.40000000000003,-10.1);
\draw(348.00000000000006, -3) node[anchor=north west,align=left] {Karoubi-Villamayor-Gersten\(K\)-theory};
\draw (348.00000000000006, -3) rectangle (358.1000000000001,-5.1);
\draw(358.20000000000005, -3) node[anchor=north west,align=left] {\(K\)-theory\\ and homology;\\ cyclic\\ homology \\ and cohomology};
\draw (358.20000000000005, -3) rectangle (362.30000000000007,-5.6);
\draw(348.00000000000006, -5.7) node[anchor=north west,align=left] {\(Q\)- and\\ plus-constructions};
\draw (348.00000000000006, -5.7) rectangle (353.1000000000001,-7.300000000000001);
\draw(353.20000000000005, -5.7) node[anchor=north west,align=left] {Negative\\ \(K\)-theory,\\ NK and Nil};
\draw (353.20000000000005, -5.7) rectangle (357.05000000000007,-7.800000000000001);
\draw(357.15000000000003, -5.7) node[anchor=north west,align=left] {Algebraic\\ \(K\)-theory\\ of spaces};
\draw (357.15000000000003, -5.7) rectangle (360.75000000000006,-7.300000000000001);
\draw(348.00000000000006, -7.9) node[anchor=north west,align=left] {Higher \\ symbols,\\ Milnor \\ \(K\)-theory};
\draw (348.00000000000006, -7.9) rectangle (351.6000000000001,-10.0);
\draw(351.70000000000005, -7.9) node[anchor=north west,align=left] {Computations\\ of higher\\ \(K\)-theory\\ of rings};
\draw (351.70000000000005, -7.9) rectangle (355.30000000000007,-10.0);
\draw(355.40000000000003, -7.9) node[anchor=north west,align=left] {Symmetric\\ monoidal\\ categories};
\draw (355.40000000000003, -7.9) rectangle (358.50000000000006,-9.5);
\draw(362.50000000000006, -2) node[anchor=north west,align=left] {\large \(K\)-theory and operator algebras};
\draw (362.50000000000006, -2) rectangle (374.3500000000001,-6.9);
\draw(363.50000000000006, -3) node[anchor=north west,align=left] {Kasparov \\ theory \\ (\(KK\)-theory)};
\draw (363.50000000000006, -3) rectangle (367.8500000000001,-4.6);
\draw(367.95000000000005, -3) node[anchor=north west,align=left] {Ext and\\ \(K\)-homology};
\draw (367.95000000000005, -3) rectangle (372.05000000000007,-4.6);
\draw(372.15000000000003, -3) node[anchor=north west,align=left] {Index\\ theory};
\draw (372.15000000000003, -3) rectangle (374.25000000000006,-4.1);
\draw(363.50000000000006, -4.7) node[anchor=north west,align=left] {\(K_0\)\\ as an \\ ordered \\ group, traces};
\draw (363.50000000000006, -4.7) rectangle (367.3500000000001,-6.800000000000001);
\draw(362.50000000000006, -7.0) node[anchor=north west,align=left] {\large Computational methods\\ for problems \\ pertaining to \(K\)-theory};
\draw (362.50000000000006, -7.0) rectangle (371.1600000000001,-8.6);
\draw(362.50000000000006, -8.7) node[anchor=north west,align=left] {\large History of\\ \(K\)-theory};
\draw (362.50000000000006, -8.7) rectangle (366.82000000000005,-9.799999999999999);
\draw(347.00000000000006, -10.2) node[anchor=north west,align=left] {\large Miscellaneous applications of \(K\)-theory};
\draw (347.00000000000006, -10.2) rectangle (360.62000000000006,-13.399999999999999);
\draw(348.00000000000006, -11.2) node[anchor=north west,align=left] {Miscellaneous\\ applications of\\ \(K\)-theory};
\draw (348.00000000000006, -11.2) rectangle (352.3500000000001,-13.299999999999999);
\draw(360.72, -10.2) node[anchor=north west,align=left] {\large Grothendieck groups and \(K_0\)};
\draw (360.72, -10.2) rectangle (371.62,-16.1);
\draw(361.72, -11.2) node[anchor=north west,align=left] {Frobenius\\ induction,\\ Burnside\\ and \\ representation rings};
\draw (361.72, -11.2) rectangle (367.32000000000005,-13.799999999999999);
\draw(367.42, -11.2) node[anchor=north west,align=left] {Stability\\ for projective\\ modules};
\draw (367.42, -11.2) rectangle (371.52000000000004,-12.799999999999999);
\draw(361.72, -13.899999999999999) node[anchor=north west,align=left] {Efficient\\ generation\\ of modules};
\draw (361.72, -13.899999999999999) rectangle (364.82000000000005,-15.499999999999998);
\draw(364.92, -13.899999999999999) node[anchor=north west,align=left] {\(K_0\)\\ of group\\ rings \\ and orders};
\draw (364.92, -13.899999999999999) rectangle (368.02000000000004,-15.999999999999998);
\draw(368.12, -13.899999999999999) node[anchor=north west,align=left] {\(K_0\)\\ of other\\ rings};
\draw (368.12, -13.899999999999999) rectangle (370.72,-15.499999999999998);
\draw(347.00000000000006, -16.2) node[anchor=north west,align=left] {\large Whitehead groups and \(K_1\)};
\draw (347.00000000000006, -16.2) rectangle (357.6000000000001,-21.1);
\draw(348.00000000000006, -17.2) node[anchor=north west,align=left] {Stable\\ range \\ conditions};
\draw (348.00000000000006, -17.2) rectangle (351.1000000000001,-18.8);
\draw(351.20000000000005, -17.2) node[anchor=north west,align=left] {Stability\\ for linear\\ groups};
\draw (351.20000000000005, -17.2) rectangle (354.30000000000007,-18.8);
\draw(354.40000000000003, -17.2) node[anchor=north west,align=left] {\(K_1\)\\ of group\\ rings \\ and orders};
\draw (354.40000000000003, -17.2) rectangle (357.50000000000006,-19.3);
\draw(348.00000000000006, -19.4) node[anchor=north west,align=left] {Congruence\\ subgroup\\ problems};
\draw (348.00000000000006, -19.4) rectangle (351.1000000000001,-21.0);
\draw(357.70000000000005, -16.2) node[anchor=north west,align=left] {\large \(K\)-theory in number theory};
\draw (357.70000000000005, -16.2) rectangle (368.1,-23.1);
\draw(358.70000000000005, -17.2) node[anchor=north west,align=left] {Étale cohomology,\\ higher \\ regulators, zeta\\ and \(L\)-functions\\ (\(K\)-theoretic aspects)};
\draw (358.70000000000005, -17.2) rectangle (365.55000000000007,-20.3);
\draw(358.70000000000005, -20.4) node[anchor=north west,align=left] {Generalized\\ class field\\ theory \\ (\(K\)-theoretic\\ aspects)};
\draw (358.70000000000005, -20.4) rectangle (363.30000000000007,-23.0);
\draw(363.40000000000003, -20.4) node[anchor=north west,align=left] {Symbols and\\ arithmetic \\ (\(K\)-theoretic\\ aspects)};
\draw (363.40000000000003, -20.4) rectangle (368.00000000000006,-22.5);
\draw(347.00000000000006, -23.2) node[anchor=north west,align=left] {\large \(K\)-theory in geometry};
\draw (347.00000000000006, -23.2) rectangle (357.15000000000003,-29.1);
\draw(348.00000000000006, -24.2) node[anchor=north west,align=left] {Relations of\\ \(K\)-theory\\ with \\ cohomology theories};
\draw (348.00000000000006, -24.2) rectangle (353.3500000000001,-26.3);
\draw(353.45000000000005, -24.2) node[anchor=north west,align=left] {\(K\)-theory\\ of schemes};
\draw (353.45000000000005, -24.2) rectangle (357.05000000000007,-25.8);
\draw(348.00000000000006, -26.4) node[anchor=north west,align=left] {Algebraic \\ cycles and motivic\\ cohomology\\ (\(K\)-theoretic\\ aspects)};
\draw (348.00000000000006, -26.4) rectangle (353.1000000000001,-29.0);
\draw(357.25000000000006, -23.2) node[anchor=north west,align=left] {\large Obstructions from topology};
\draw (357.25000000000006, -23.2) rectangle (366.90000000000003,-29.1);
\draw(358.25000000000006, -24.2) node[anchor=north west,align=left] {Surgery \\ obstructions \\ (\(K\)-theoretic\\ aspects)};
\draw (358.25000000000006, -24.2) rectangle (362.8500000000001,-26.3);
\draw(362.95000000000005, -24.2) node[anchor=north west,align=left] {Whitehead\\ (and related)\\ torsion};
\draw (362.95000000000005, -24.2) rectangle (366.80000000000007,-25.8);
\draw(358.25000000000006, -26.4) node[anchor=north west,align=left] {Obstructions\\ to group\\ actions \\ (\(K\)-theoretic\\ aspects)};
\draw (358.25000000000006, -26.4) rectangle (362.8500000000001,-29.0);
\draw(362.95000000000005, -26.4) node[anchor=north west,align=left] {Finiteness\\ and other \\ obstructions\\ in \(K_0\)};
\draw (362.95000000000005, -26.4) rectangle (366.55000000000007,-28.5);
\draw(347.00000000000006, -29.2) node[anchor=north west,align=left] {\large Steinberg groups and \(K_2\)};
\draw (347.00000000000006, -29.2) rectangle (356.28000000000003,-34.6);
\draw(348.00000000000006, -30.2) node[anchor=north west,align=left] {Symbols, \\ presentations\\ and stability\\ of \(K_2\)};
\draw (348.00000000000006, -30.2) rectangle (351.8500000000001,-32.3);
\draw(351.95000000000005, -30.2) node[anchor=north west,align=left] {\(K_2\) \\ and the \\ Brauer group};
\draw (351.95000000000005, -30.2) rectangle (355.55000000000007,-31.8);
\draw(348.00000000000006, -32.4) node[anchor=north west,align=left] {Central \\ extensions \\ and Schur \\ multipliers};
\draw (348.00000000000006, -32.4) rectangle (351.3500000000001,-34.5);
\draw(351.45000000000005, -32.4) node[anchor=north west,align=left] {Excision\\ for\\ \(K_2\)};
\draw (351.45000000000005, -32.4) rectangle (354.05000000000007,-34.0);
\draw(356.38000000000005, -29.2) node[anchor=north west,align=left] {\large Topological \(K\)-theory};
\draw (356.38000000000005, -29.2) rectangle (365.53000000000003,-38.5);
\draw(357.38000000000005, -30.2) node[anchor=north west,align=left] {\(J\)-homomorphism,\\ Adams \\ operations};
\draw (357.38000000000005, -30.2) rectangle (362.7300000000001,-32.3);
\draw(357.38000000000005, -32.4) node[anchor=north west,align=left] {Geometric \\ applications \\ of topological\\ \(K\)-theory};
\draw (357.38000000000005, -32.4) rectangle (361.4800000000001,-34.5);
\draw(361.58000000000004, -32.4) node[anchor=north west,align=left] {Connective\\ \(K\)-theory,\\ cobordism};
\draw (361.58000000000004, -32.4) rectangle (365.43000000000006,-34.5);
\draw(357.38000000000005, -34.6) node[anchor=north west,align=left] {Twisted \\ \(K\)-theory;\\ differential\\ \(K\)-theory};
\draw (357.38000000000005, -34.6) rectangle (361.2300000000001,-36.7);
\draw(361.33000000000004, -34.6) node[anchor=north west,align=left] {Riemann-Roch\\ theorems,\\ Chern\\ characters};
\draw (361.33000000000004, -34.6) rectangle (364.93000000000006,-36.7);
\draw(357.38000000000005, -36.8) node[anchor=north west,align=left] {Equivariant\\ \(K\)-theory};
\draw (357.38000000000005, -36.8) rectangle (360.9800000000001,-38.4);
\draw(365.63000000000005, -29.2) node[anchor=north west,align=left] {\large \(K\)-theory of forms};
\draw (365.63000000000005, -29.2) rectangle (374.28000000000003,-34.6);
\draw(366.63000000000005, -30.2) node[anchor=north west,align=left] {Hermitian \\ \(K\)-theory,\\ relations \\ with \(K\)-theory\\ of rings};
\draw (366.63000000000005, -30.2) rectangle (371.4800000000001,-32.8);
\draw(371.58000000000004, -30.2) node[anchor=north west,align=left] {Witt \\ groups\\ of rings};
\draw (371.58000000000004, -30.2) rectangle (374.18000000000006,-31.8);
\draw(366.63000000000005, -32.9) node[anchor=north west,align=left] {Stability\\ for quadratic\\ modules};
\draw (366.63000000000005, -32.9) rectangle (370.4800000000001,-34.5);
\draw(370.58000000000004, -32.9) node[anchor=north west,align=left] {\(L\)-theory\\ of \\ group rings};
\draw (370.58000000000004, -32.9) rectangle (374.18000000000006,-34.5);
\draw(346.00000000000006, -38.7) node[anchor=north west,align=left] {\LARGE Special functions};
\draw (346.00000000000006, -38.7) rectangle (373.50000000000006,-81.60000000000002);
\draw(347.00000000000006, -39.7) node[anchor=north west,align=left] {\large Basic hypergeometric functions};
\draw (347.00000000000006, -39.7) rectangle (361.40000000000003,-56.2);
\draw(348.00000000000006, -40.7) node[anchor=north west,align=left] {Connections of basic\\ hypergeometric \\ functions with quantum\\ groups, Chevalley\\ groups, \(p\)-adic\\ groups, Hecke \\ algebras, and related topics};
\draw (348.00000000000006, -40.7) rectangle (355.6000000000001,-44.300000000000004);
\draw(355.70000000000005, -40.7) node[anchor=north west,align=left] {Basic orthogonal\\ polynomials and\\ functions \\ associated with root\\ systems (Macdonald\\ polynomials, etc.)};
\draw (355.70000000000005, -40.7) rectangle (361.30000000000007,-43.800000000000004);
\draw(348.00000000000006, -44.400000000000006) node[anchor=north west,align=left] {Orthogonal polynomials\\ and functions\\ in several variables\\ expressible in\\ terms of basic \\ hypergeometric functions\\ in one variable};
\draw (348.00000000000006, -44.400000000000006) rectangle (354.6000000000001,-48.00000000000001);
\draw(354.70000000000005, -44.400000000000006) node[anchor=north west,align=left] {Basic orthogonal\\ polynomials \\ and functions \\ (Askey-Wilson \\ polynomials, etc.)};
\draw (354.70000000000005, -44.400000000000006) rectangle (359.80000000000007,-47.00000000000001);
\draw(348.00000000000006, -48.1) node[anchor=north west,align=left] {Basic \\ hypergeometric \\ functions \\ associated \\ with root systems};
\draw (348.00000000000006, -48.1) rectangle (352.8500000000001,-50.7);
\draw(352.95000000000005, -48.1) node[anchor=north west,align=left] {Other basic \\ hypergeometric \\ functions and \\ integrals in \\ several variables};
\draw (352.95000000000005, -48.1) rectangle (357.80000000000007,-50.7);
\draw(348.00000000000006, -50.800000000000004) node[anchor=north west,align=left] {Basic \\ hypergeometric \\ functions in one\\ variable,\\ \({}_r\phi_s\)};
\draw (348.00000000000006, -50.800000000000004) rectangle (352.6000000000001,-53.400000000000006);
\draw(352.70000000000005, -50.800000000000004) node[anchor=north west,align=left] {Basic \\ hypergeometric \\ integrals and \\ functions \\ defined by them};
\draw (352.70000000000005, -50.800000000000004) rectangle (357.05000000000007,-53.400000000000006);
\draw(357.15000000000003, -50.800000000000004) node[anchor=north west,align=left] {Applications\\ of basic\\ hypergeometric\\ functions};
\draw (357.15000000000003, -50.800000000000004) rectangle (361.25000000000006,-52.900000000000006);
\draw(348.00000000000006, -53.5) node[anchor=north west,align=left] {\(q\)-gamma\\ functions,\\ \(q\)-beta\\ functions\\ and integrals};
\draw (348.00000000000006, -53.5) rectangle (351.8500000000001,-56.1);
\draw(351.95000000000005, -53.5) node[anchor=north west,align=left] {Bibasic \\ functions\\ and multiple\\ bases};
\draw (351.95000000000005, -53.5) rectangle (355.55000000000007,-55.6);
\draw(361.50000000000006, -39.7) node[anchor=north west,align=left] {\large Other special functions};
\draw (361.50000000000006, -39.7) rectangle (373.40000000000003,-49.5);
\draw(362.50000000000006, -40.7) node[anchor=north west,align=left] {Painlevé-typefunctions};
\draw (362.50000000000006, -40.7) rectangle (368.6000000000001,-42.300000000000004);
\draw(368.70000000000005, -40.7) node[anchor=north west,align=left] {Lamé, Mathieu,\\ and \\ spheroidal wave\\ functions};
\draw (368.70000000000005, -40.7) rectangle (373.05000000000007,-42.800000000000004);
\draw(362.50000000000006, -42.900000000000006) node[anchor=north west,align=left] {Special functions\\ in \\ characteristic \(p\)\\ (gamma \\ functions, etc.)};
\draw (362.50000000000006, -42.900000000000006) rectangle (368.1000000000001,-45.50000000000001);
\draw(368.20000000000005, -42.900000000000006) node[anchor=north west,align=left] {Other functions\\ coming from \\ differential, \\ difference and \\ integral equations};
\draw (368.20000000000005, -42.900000000000006) rectangle (373.30000000000007,-45.50000000000001);
\draw(362.50000000000006, -45.6) node[anchor=north west,align=left] {Mittag-Leffler\\ functions\\ and \\ generalizations};
\draw (362.50000000000006, -45.6) rectangle (366.8500000000001,-47.7);
\draw(366.95000000000005, -45.6) node[anchor=north west,align=left] {Other functions\\ defined\\ by series\\ and integrals};
\draw (366.95000000000005, -45.6) rectangle (371.30000000000007,-47.7);
\draw(362.50000000000006, -47.800000000000004) node[anchor=north west,align=left] {Elliptic \\ functions \\ and integrals};
\draw (362.50000000000006, -47.800000000000004) rectangle (366.3500000000001,-49.400000000000006);
\draw(366.45000000000005, -47.800000000000004) node[anchor=north west,align=left] {Other\\ wave \\ functions};
\draw (366.45000000000005, -47.800000000000004) rectangle (369.30000000000007,-49.400000000000006);
\draw(361.50000000000006, -49.6) node[anchor=north west,align=left] {\large Elementary classical functions};
\draw (361.50000000000006, -49.6) rectangle (372.65000000000003,-56.0);
\draw(362.50000000000006, -50.6) node[anchor=north west,align=left] {Incomplete beta\\ and gamma \\ functions (error \\ functions, probability\\ integral,\\ Fresnel integrals)};
\draw (362.50000000000006, -50.6) rectangle (368.6000000000001,-53.7);
\draw(368.70000000000005, -50.6) node[anchor=north west,align=left] {Exponential\\ and \\ trigonometric\\ functions};
\draw (368.70000000000005, -50.6) rectangle (372.55000000000007,-52.7);
\draw(362.50000000000006, -53.800000000000004) node[anchor=north west,align=left] {Gamma, \\ beta and\\ polygamma\\ functions};
\draw (362.50000000000006, -53.800000000000004) rectangle (365.3500000000001,-55.900000000000006);
\draw(365.45000000000005, -53.800000000000004) node[anchor=north west,align=left] {Higher \\ logarithm\\ functions};
\draw (365.45000000000005, -53.800000000000004) rectangle (368.30000000000007,-55.400000000000006);
\draw(347.00000000000006, -56.30000000000001) node[anchor=north west,align=left] {\large Hypergeometric functions};
\draw (347.00000000000006, -56.30000000000001) rectangle (360.65000000000003,-77.20000000000002);
\draw(348.00000000000006, -57.30000000000001) node[anchor=north west,align=left] {Orthogonal \\ polynomials and functions\\ in several\\ variables expressible\\ in terms \\ of special functions\\ in one variable};
\draw (348.00000000000006, -57.30000000000001) rectangle (354.8500000000001,-60.90000000000001);
\draw(354.95000000000005, -57.30000000000001) node[anchor=north west,align=left] {Confluent \\ hypergeometric \\ functions, \\ Whittaker functions,\\ \({}_1F_1\)};
\draw (354.95000000000005, -57.30000000000001) rectangle (360.55000000000007,-59.90000000000001);
\draw(348.00000000000006, -61.000000000000014) node[anchor=north west,align=left] {Other \\ hypergeometric functions\\ and \\ integrals in several\\ variables};
\draw (348.00000000000006, -61.000000000000014) rectangle (354.6000000000001,-63.600000000000016);
\draw(354.70000000000005, -61.000000000000014) node[anchor=north west,align=left] {Hypergeometric \\ integrals and \\ functions defined \\ by them (\(E\), \\ \(G\), \(H\) and\\ \(I\) functions)};
\draw (354.70000000000005, -61.000000000000014) rectangle (359.80000000000007,-64.10000000000001);
\draw(348.00000000000006, -64.20000000000002) node[anchor=north west,align=left] {Orthogonal polynomials\\ and functions\\ of hypergeometric\\ type (Jacobi,\\ Laguerre, Hermite,\\ Askey scheme, etc.)};
\draw (348.00000000000006, -64.20000000000002) rectangle (354.1000000000001,-67.30000000000001);
\draw(354.20000000000005, -64.20000000000002) node[anchor=north west,align=left] {Bessel and\\ Airy functions,\\ cylinder\\ functions,\\ \({}_0F_1\)};
\draw (354.20000000000005, -64.20000000000002) rectangle (358.55000000000007,-66.80000000000001);
\draw(348.00000000000006, -67.4) node[anchor=north west,align=left] {Orthogonal \\ polynomials and\\ functions \\ associated with\\ root systems};
\draw (348.00000000000006, -67.4) rectangle (352.3500000000001,-70.0);
\draw(352.45000000000005, -67.4) node[anchor=north west,align=left] {Hypergeometric\\ functions \\ associated with\\ root systems};
\draw (352.45000000000005, -67.4) rectangle (356.80000000000007,-69.5);
\draw(356.90000000000003, -67.4) node[anchor=north west,align=left] {Appell, \\ Horn and \\ Lauricella\\ functions};
\draw (356.90000000000003, -67.4) rectangle (360.00000000000006,-69.5);
\draw(348.00000000000006, -70.10000000000001) node[anchor=north west,align=left] {Connections of\\ hypergeometric\\ functions \\ with groups and\\ algebras, and\\ related topics};
\draw (348.00000000000006, -70.10000000000001) rectangle (352.3500000000001,-73.2);
\draw(352.45000000000005, -70.10000000000001) node[anchor=north west,align=left] {Classical \\ hypergeometric\\ functions,\\ \({}_2F_1\)};
\draw (352.45000000000005, -70.10000000000001) rectangle (356.55000000000007,-72.2);
\draw(356.65000000000003, -70.10000000000001) node[anchor=north west,align=left] {Other special\\ orthogonal\\ polynomials\\ and functions};
\draw (356.65000000000003, -70.10000000000001) rectangle (360.50000000000006,-72.2);
\draw(348.00000000000006, -73.30000000000001) node[anchor=north west,align=left] {Generalized\\ hypergeometric\\ series,\\ \({}_pF_q\)};
\draw (348.00000000000006, -73.30000000000001) rectangle (352.1000000000001,-75.4);
\draw(352.20000000000005, -73.30000000000001) node[anchor=north west,align=left] {Elliptic \\ integrals as\\ hypergeometric\\ functions};
\draw (352.20000000000005, -73.30000000000001) rectangle (356.30000000000007,-75.4);
\draw(356.40000000000003, -73.30000000000001) node[anchor=north west,align=left] {Applications\\ of \\ hypergeometric\\ functions};
\draw (356.40000000000003, -73.30000000000001) rectangle (360.50000000000006,-75.4);
\draw(348.00000000000006, -75.50000000000001) node[anchor=north west,align=left] {Spherical\\ harmonics};
\draw (348.00000000000006, -75.50000000000001) rectangle (350.8500000000001,-77.10000000000001);
\draw(360.75000000000006, -56.30000000000001) node[anchor=north west,align=left] {\large History of \\ special functions};
\draw (360.75000000000006, -56.30000000000001) rectangle (366.62000000000006,-57.40000000000001);
\draw(347.00000000000006, -77.30000000000001) node[anchor=north west,align=left] {\large Computational aspects of special functions};
\draw (347.00000000000006, -77.30000000000001) rectangle (360.62000000000006,-81.50000000000001);
\draw(348.00000000000006, -78.30000000000001) node[anchor=north west,align=left] {Numerical \\ approximation\\ and \\ evaluation of \\ special functions};
\draw (348.00000000000006, -78.30000000000001) rectangle (352.8500000000001,-80.9);
\draw(352.95000000000005, -78.30000000000001) node[anchor=north west,align=left] {Symbolic \\ computation of \\ special functions\\ (Gosper and\\ Zeilberger\\ algorithms, etc.)};
\draw (352.95000000000005, -78.30000000000001) rectangle (357.80000000000007,-81.4);
\draw(346.00000000000006, -81.70000000000002) node[anchor=north west,align=left] {\LARGE Potential theory};
\draw (346.00000000000006, -81.70000000000002) rectangle (372.70000000000005,-112.80000000000003);
\draw(347.00000000000006, -82.70000000000002) node[anchor=north west,align=left] {\large Potential theory on fractals and metric spaces};
\draw (347.00000000000006, -82.70000000000002) rectangle (361.86000000000007,-85.90000000000002);
\draw(348.00000000000006, -83.70000000000002) node[anchor=north west,align=left] {Potential \\ theory on \\ fractals and\\ metric spaces};
\draw (348.00000000000006, -83.70000000000002) rectangle (351.8500000000001,-85.80000000000001);
\draw(361.96000000000004, -82.70000000000002) node[anchor=north west,align=left] {\large Axiomatic potential theory};
\draw (361.96000000000004, -82.70000000000002) rectangle (370.62000000000006,-85.40000000000002);
\draw(362.96000000000004, -83.70000000000002) node[anchor=north west,align=left] {Axiomatic\\ potential\\ theory};
\draw (362.96000000000004, -83.70000000000002) rectangle (365.81000000000006,-85.30000000000001);
\draw(347.00000000000006, -86.00000000000001) node[anchor=north west,align=left] {\large Generalizations of potential theory};
\draw (347.00000000000006, -86.00000000000001) rectangle (360.1000000000001,-95.10000000000001);
\draw(348.00000000000006, -87.00000000000001) node[anchor=north west,align=left] {Pluriharmonic\\ and \\ plurisubharmonic\\ functions};
\draw (348.00000000000006, -87.00000000000001) rectangle (352.6000000000001,-89.10000000000001);
\draw(352.70000000000005, -87.00000000000001) node[anchor=north west,align=left] {Harmonic, \\ subharmonic, \\ superharmonic\\ functions \\ on other spaces};
\draw (352.70000000000005, -87.00000000000001) rectangle (357.05000000000007,-89.60000000000001);
\draw(357.15000000000003, -87.00000000000001) node[anchor=north west,align=left] {Discrete\\ potential\\ theory};
\draw (357.15000000000003, -87.00000000000001) rectangle (360.00000000000006,-88.60000000000001);
\draw(348.00000000000006, -89.70000000000002) node[anchor=north west,align=left] {Fine potential\\ theory; \\ fine properties\\ of sets \\ and functions};
\draw (348.00000000000006, -89.70000000000002) rectangle (352.3500000000001,-92.30000000000001);
\draw(352.45000000000005, -89.70000000000002) node[anchor=north west,align=left] {Other \\ generalizations\\ (nonlinear\\ potential \\ theory, etc.)};
\draw (352.45000000000005, -89.70000000000002) rectangle (356.80000000000007,-92.30000000000001);
\draw(356.90000000000003, -89.70000000000002) node[anchor=north west,align=left] {Dirichlet\\ forms};
\draw (356.90000000000003, -89.70000000000002) rectangle (359.75000000000006,-90.80000000000001);
\draw(348.00000000000006, -92.40000000000002) node[anchor=north west,align=left] {Potentials\\ and \\ capacities on \\ other spaces};
\draw (348.00000000000006, -92.40000000000002) rectangle (352.1000000000001,-94.50000000000001);
\draw(352.20000000000005, -92.40000000000002) node[anchor=north west,align=left] {Potential \\ theory on \\ Riemannian \\ manifolds and\\ other spaces};
\draw (352.20000000000005, -92.40000000000002) rectangle (356.05000000000007,-95.00000000000001);
\draw(356.15000000000003, -92.40000000000002) node[anchor=north west,align=left] {Martin\\ boundary\\ theory};
\draw (356.15000000000003, -92.40000000000002) rectangle (358.75000000000006,-94.00000000000001);
\draw(360.20000000000005, -86.00000000000001) node[anchor=north west,align=left] {\large Two-dimensional potential theory};
\draw (360.20000000000005, -86.00000000000001) rectangle (372.6,-99.30000000000001);
\draw(361.20000000000005, -87.00000000000001) node[anchor=north west,align=left] {Connections of\\ harmonic functions\\ with \\ differential equations\\ in two dimensions};
\draw (361.20000000000005, -87.00000000000001) rectangle (367.30000000000007,-89.60000000000001);
\draw(367.40000000000003, -87.00000000000001) node[anchor=north west,align=left] {Potentials and \\ capacity, harmonic\\ measure, extremal\\ length and \\ related notions \\ in two dimensions};
\draw (367.40000000000003, -87.00000000000001) rectangle (372.50000000000006,-90.10000000000001);
\draw(361.20000000000005, -90.20000000000002) node[anchor=north west,align=left] {Boundary value\\ and inverse \\ problems for harmonic\\ functions \\ in two dimensions};
\draw (361.20000000000005, -90.20000000000002) rectangle (367.05000000000007,-92.80000000000001);
\draw(367.15000000000003, -90.20000000000002) node[anchor=north west,align=left] {Integral \\ representations, \\ integral operators,\\ integral equations\\ methods in\\ two dimensions};
\draw (367.15000000000003, -90.20000000000002) rectangle (372.50000000000006,-93.30000000000001);
\draw(361.20000000000005, -93.40000000000002) node[anchor=north west,align=left] {Biharmonic, \\ polyharmonic \\ functions and \\ equations, Poisson’s\\ equation \\ in two dimensions};
\draw (361.20000000000005, -93.40000000000002) rectangle (366.80000000000007,-96.50000000000001);
\draw(366.90000000000003, -93.40000000000002) node[anchor=north west,align=left] {Boundary behavior\\ (theorems\\ of Fatou type,\\ etc.) of \\ harmonic functions\\ in two dimensions};
\draw (366.90000000000003, -93.40000000000002) rectangle (372.00000000000006,-96.50000000000001);
\draw(361.20000000000005, -96.60000000000002) node[anchor=north west,align=left] {Harmonic, \\ subharmonic, \\ superharmonic\\ functions \\ in two dimensions};
\draw (361.20000000000005, -96.60000000000002) rectangle (366.05000000000007,-99.20000000000002);
\draw(347.00000000000006, -95.20000000000002) node[anchor=north west,align=left] {\large Computational methods\\ for problems pertaining\\ to potential theory};
\draw (347.00000000000006, -95.20000000000002) rectangle (354.7300000000001,-96.80000000000001);
\draw(347.00000000000006, -96.90000000000002) node[anchor=north west,align=left] {\large History of \\ potential theory};
\draw (347.00000000000006, -96.90000000000002) rectangle (352.56000000000006,-98.00000000000001);
\draw(347.00000000000006, -99.40000000000002) node[anchor=north west,align=left] {\large Higher-dimensional potential theory};
\draw (347.00000000000006, -99.40000000000002) rectangle (359.40000000000003,-112.70000000000002);
\draw(348.00000000000006, -100.40000000000002) node[anchor=north west,align=left] {Boundary value\\ and inverse \\ problems for harmonic\\ functions \\ in higher dimensions};
\draw (348.00000000000006, -100.40000000000002) rectangle (353.8500000000001,-103.00000000000001);
\draw(353.95000000000005, -100.40000000000002) node[anchor=north west,align=left] {Integral \\ representations, \\ integral operators,\\ integral equations\\ methods in\\ higher dimensions};
\draw (353.95000000000005, -100.40000000000002) rectangle (359.30000000000007,-103.50000000000001);
\draw(348.00000000000006, -103.60000000000002) node[anchor=north west,align=left] {Potentials and\\ capacities,\\ extremal \\ length and related\\ notions in\\ higher dimensions};
\draw (348.00000000000006, -103.60000000000002) rectangle (353.1000000000001,-106.70000000000002);
\draw(353.20000000000005, -103.60000000000002) node[anchor=north west,align=left] {Boundary \\ behavior of \\ harmonic functions\\ in higher\\ dimensions};
\draw (353.20000000000005, -103.60000000000002) rectangle (358.30000000000007,-106.20000000000002);
\draw(348.00000000000006, -106.80000000000003) node[anchor=north west,align=left] {Harmonic, \\ subharmonic, \\ superharmonic \\ functions in \\ higher dimensions};
\draw (348.00000000000006, -106.80000000000003) rectangle (352.8500000000001,-109.40000000000002);
\draw(352.95000000000005, -106.80000000000003) node[anchor=north west,align=left] {Biharmonic and\\ polyharmonic\\ equations and\\ functions in \\ higher dimensions};
\draw (352.95000000000005, -106.80000000000003) rectangle (357.80000000000007,-109.40000000000002);
\draw(348.00000000000006, -109.50000000000003) node[anchor=north west,align=left] {Connections \\ of harmonic \\ functions with\\ differential\\ equations in\\ higher dimensions};
\draw (348.00000000000006, -109.50000000000003) rectangle (352.8500000000001,-112.60000000000002);
\draw(346.00000000000006, -112.90000000000002) node[anchor=north west,align=left] {\LARGE Order, lattices, ordered algebraic structures};
\draw (346.00000000000006, -112.90000000000002) rectangle (371.95000000000005,-148.00000000000003);
\draw(347.00000000000006, -113.90000000000002) node[anchor=north west,align=left] {\large Modular lattices, complemented lattices};
\draw (347.00000000000006, -113.90000000000002) rectangle (361.6000000000001,-119.80000000000003);
\draw(348.00000000000006, -114.90000000000002) node[anchor=north west,align=left] {Complemented\\ lattices, \\ orthocomplemented\\ lattices\\ and posets};
\draw (348.00000000000006, -114.90000000000002) rectangle (352.8500000000001,-117.50000000000001);
\draw(352.95000000000005, -114.90000000000002) node[anchor=north west,align=left] {Complemented\\ modular lattices,\\ continuous\\ geometries};
\draw (352.95000000000005, -114.90000000000002) rectangle (357.80000000000007,-117.00000000000001);
\draw(357.90000000000003, -114.90000000000002) node[anchor=north west,align=left] {Modular \\ lattices,\\ Desarguesian\\ lattices};
\draw (357.90000000000003, -114.90000000000002) rectangle (361.50000000000006,-117.00000000000001);
\draw(348.00000000000006, -117.60000000000002) node[anchor=north west,align=left] {Semimodular\\ lattices,\\ geometric\\ lattices};
\draw (348.00000000000006, -117.60000000000002) rectangle (351.3500000000001,-119.70000000000002);
\draw(361.70000000000005, -113.90000000000002) node[anchor=north west,align=left] {\large Ordered structures};
\draw (361.70000000000005, -113.90000000000002) rectangle (371.85,-124.20000000000002);
\draw(362.70000000000005, -114.90000000000002) node[anchor=north west,align=left] {Ordered \\ topological \\ structures (aspects\\ of ordered\\ structures)};
\draw (362.70000000000005, -114.90000000000002) rectangle (368.05000000000007,-117.50000000000001);
\draw(368.15000000000003, -114.90000000000002) node[anchor=north west,align=left] {Ordered \\ semigroups \\ and monoids};
\draw (368.15000000000003, -114.90000000000002) rectangle (371.50000000000006,-116.50000000000001);
\draw(362.70000000000005, -117.60000000000002) node[anchor=north west,align=left] {Ordered \\ rings, algebras,\\ modules};
\draw (362.70000000000005, -117.60000000000002) rectangle (367.30000000000007,-119.20000000000002);
\draw(367.40000000000003, -117.60000000000002) node[anchor=north west,align=left] {Ordered \\ abelian groups,\\ Riesz \\ groups, ordered\\ linear spaces};
\draw (367.40000000000003, -117.60000000000002) rectangle (371.75000000000006,-120.20000000000002);
\draw(362.70000000000005, -120.30000000000003) node[anchor=north west,align=left] {BCK-algebras,\\ BCI-algebras\\ (aspects\\ of ordered\\ structures)};
\draw (362.70000000000005, -120.30000000000003) rectangle (366.55000000000007,-122.90000000000002);
\draw(366.65000000000003, -120.30000000000003) node[anchor=north west,align=left] {Quantales};
\draw (366.65000000000003, -120.30000000000003) rectangle (369.50000000000006,-121.40000000000002);
\draw(366.65000000000003, -121.50000000000001) node[anchor=north west,align=left] {Noether\\ lattices};
\draw (366.65000000000003, -121.50000000000001) rectangle (369.25000000000006,-122.60000000000001);
\draw(362.70000000000005, -123.00000000000003) node[anchor=north west,align=left] {Ordered\\ groups};
\draw (362.70000000000005, -123.00000000000003) rectangle (365.05000000000007,-124.10000000000002);
\draw(347.00000000000006, -119.90000000000002) node[anchor=north west,align=left] {\large Computational methods\\ for problems pertaining\\ to ordered structures};
\draw (347.00000000000006, -119.90000000000002) rectangle (354.7300000000001,-121.50000000000001);
\draw(347.00000000000006, -121.60000000000002) node[anchor=north west,align=left] {\large History of \\ ordered structures};
\draw (347.00000000000006, -121.60000000000002) rectangle (353.18000000000006,-122.70000000000002);
\draw(347.00000000000006, -124.30000000000003) node[anchor=north west,align=left] {\large Distributive lattices};
\draw (347.00000000000006, -124.30000000000003) rectangle (358.15000000000003,-136.8);
\draw(348.00000000000006, -125.30000000000003) node[anchor=north west,align=left] {De Morgan \\ algebras, Łukasiewicz\\ algebras\\ (lattice-theoretic\\ aspects)};
\draw (348.00000000000006, -125.30000000000003) rectangle (353.8500000000001,-127.90000000000002);
\draw(353.95000000000005, -125.30000000000003) node[anchor=north west,align=left] {Complete\\ distributivity};
\draw (353.95000000000005, -125.30000000000003) rectangle (358.05000000000007,-126.90000000000002);
\draw(348.00000000000006, -128.00000000000003) node[anchor=north west,align=left] {Pseudocomplemented\\ lattices};
\draw (348.00000000000006, -128.00000000000003) rectangle (353.1000000000001,-129.60000000000002);
\draw(353.20000000000005, -128.00000000000003) node[anchor=north west,align=left] {Structure and\\ representation\\ theory\\ of distributive\\ lattices};
\draw (353.20000000000005, -128.00000000000003) rectangle (357.55000000000007,-130.60000000000002);
\draw(348.00000000000006, -130.70000000000002) node[anchor=north west,align=left] {Heyting \\ algebras \\ (lattice-theoretic\\ aspects)};
\draw (348.00000000000006, -130.70000000000002) rectangle (353.1000000000001,-132.8);
\draw(353.20000000000005, -130.70000000000002) node[anchor=north west,align=left] {Other \\ generalizations\\ of distributive\\ lattices};
\draw (353.20000000000005, -130.70000000000002) rectangle (357.55000000000007,-132.8);
\draw(348.00000000000006, -132.90000000000003) node[anchor=north west,align=left] {Post algebras\\ (lattice-theoretic\\ aspects)};
\draw (348.00000000000006, -132.90000000000003) rectangle (353.1000000000001,-135.00000000000003);
\draw(353.20000000000005, -132.90000000000003) node[anchor=north west,align=left] {Fuzzy lattices\\ (soft \\ algebras) and \\ related topics};
\draw (353.20000000000005, -132.90000000000003) rectangle (357.30000000000007,-135.00000000000003);
\draw(348.00000000000006, -135.10000000000002) node[anchor=north west,align=left] {MV-algebras};
\draw (348.00000000000006, -135.10000000000002) rectangle (351.3500000000001,-136.20000000000002);
\draw(351.45000000000005, -135.10000000000002) node[anchor=north west,align=left] {Lattices\\ and\\ duality};
\draw (351.45000000000005, -135.10000000000002) rectangle (354.05000000000007,-136.70000000000002);
\draw(354.15000000000003, -135.10000000000002) node[anchor=north west,align=left] {Frames,\\ locales};
\draw (354.15000000000003, -135.10000000000002) rectangle (356.50000000000006,-136.20000000000002);
\draw(358.25000000000006, -124.30000000000003) node[anchor=north west,align=left] {\large Ordered sets};
\draw (358.25000000000006, -124.30000000000003) rectangle (368.90000000000003,-131.90000000000003);
\draw(359.25000000000006, -125.30000000000003) node[anchor=north west,align=left] {Galois \\ correspondences, closure\\ operators\\ (in relation \\ to ordered sets)};
\draw (359.25000000000006, -125.30000000000003) rectangle (365.8500000000001,-127.90000000000002);
\draw(365.95000000000005, -125.30000000000003) node[anchor=north west,align=left] {Algebraic\\ aspects\\ of posets};
\draw (365.95000000000005, -125.30000000000003) rectangle (368.80000000000007,-126.90000000000002);
\draw(359.25000000000006, -128.00000000000003) node[anchor=north west,align=left] {Generalizations\\ of \\ ordered sets};
\draw (359.25000000000006, -128.00000000000003) rectangle (363.6000000000001,-129.60000000000002);
\draw(363.70000000000005, -128.00000000000003) node[anchor=north west,align=left] {Combinatorics\\ of \\ partially\\ ordered sets};
\draw (363.70000000000005, -128.00000000000003) rectangle (367.55000000000007,-130.10000000000002);
\draw(359.25000000000006, -130.20000000000002) node[anchor=north west,align=left] {Semilattices};
\draw (359.25000000000006, -130.20000000000002) rectangle (362.8500000000001,-131.3);
\draw(362.95000000000005, -130.20000000000002) node[anchor=north west,align=left] {Partial\\ orders,\\ general};
\draw (362.95000000000005, -130.20000000000002) rectangle (365.30000000000007,-131.8);
\draw(365.40000000000003, -130.20000000000002) node[anchor=north west,align=left] {Total\\ orders};
\draw (365.40000000000003, -130.20000000000002) rectangle (367.50000000000006,-131.3);
\draw(347.00000000000006, -136.90000000000003) node[anchor=north west,align=left] {\large Boolean algebras (Boolean rings)};
\draw (347.00000000000006, -136.90000000000003) rectangle (357.65000000000003,-145.00000000000003);
\draw(348.00000000000006, -137.90000000000003) node[anchor=north west,align=left] {Generalizationsof\\ Boolean\\ algebras};
\draw (348.00000000000006, -137.90000000000003) rectangle (352.8500000000001,-140.00000000000003);
\draw(352.95000000000005, -137.90000000000003) node[anchor=north west,align=left] {Stone spaces\\ (Boolean spaces)\\ and related\\ structures};
\draw (352.95000000000005, -137.90000000000003) rectangle (357.55000000000007,-140.00000000000003);
\draw(348.00000000000006, -140.10000000000002) node[anchor=north west,align=left] {Boolean algebras\\ with additional\\ operations \\ (diagonalizable\\ algebras, etc.)};
\draw (348.00000000000006, -140.10000000000002) rectangle (352.6000000000001,-142.70000000000002);
\draw(352.70000000000005, -140.10000000000002) node[anchor=north west,align=left] {Ring-theoretic\\ properties\\ of Boolean\\ algebras};
\draw (352.70000000000005, -140.10000000000002) rectangle (356.80000000000007,-142.20000000000002);
\draw(348.00000000000006, -142.80000000000004) node[anchor=north west,align=left] {Chain \\ conditions,\\ complete\\ algebras};
\draw (348.00000000000006, -142.80000000000004) rectangle (351.3500000000001,-144.90000000000003);
\draw(351.45000000000005, -142.80000000000004) node[anchor=north west,align=left] {Structure\\ theory \\ of Boolean\\ algebras};
\draw (351.45000000000005, -142.80000000000004) rectangle (354.55000000000007,-144.90000000000003);
\draw(354.65000000000003, -142.80000000000004) node[anchor=north west,align=left] {Boolean\\ functions};
\draw (354.65000000000003, -142.80000000000004) rectangle (357.50000000000006,-144.40000000000003);
\draw(357.75000000000006, -136.90000000000003) node[anchor=north west,align=left] {\large Lattices};
\draw (357.75000000000006, -136.90000000000003) rectangle (367.65000000000003,-147.90000000000003);
\draw(358.75000000000006, -137.90000000000003) node[anchor=north west,align=left] {Topologicallattices};
\draw (358.75000000000006, -137.90000000000003) rectangle (364.1000000000001,-139.50000000000003);
\draw(364.20000000000005, -137.90000000000003) node[anchor=north west,align=left] {Structure\\ theory \\ of lattices};
\draw (364.20000000000005, -137.90000000000003) rectangle (367.55000000000007,-139.50000000000003);
\draw(358.75000000000006, -139.60000000000002) node[anchor=north west,align=left] {Generalizations\\ of\\ lattices};
\draw (358.75000000000006, -139.60000000000002) rectangle (363.1000000000001,-141.20000000000002);
\draw(363.20000000000005, -139.60000000000002) node[anchor=north west,align=left] {Representation\\ theory of\\ lattices};
\draw (363.20000000000005, -139.60000000000002) rectangle (367.30000000000007,-141.70000000000002);
\draw(358.75000000000006, -141.80000000000004) node[anchor=north west,align=left] {Free lattices,\\ projective\\ lattices,\\ word problems};
\draw (358.75000000000006, -141.80000000000004) rectangle (362.8500000000001,-143.90000000000003);
\draw(362.95000000000005, -141.80000000000004) node[anchor=north west,align=left] {Continuous\\ lattices and\\ posets, \\ applications};
\draw (362.95000000000005, -141.80000000000004) rectangle (366.55000000000007,-143.90000000000003);
\draw(358.75000000000006, -144.00000000000003) node[anchor=north west,align=left] {Complete \\ lattices, \\ completions};
\draw (358.75000000000006, -144.00000000000003) rectangle (362.1000000000001,-145.60000000000002);
\draw(362.20000000000005, -144.00000000000003) node[anchor=north west,align=left] {Lattice \\ ideals, \\ congruence\\ relations};
\draw (362.20000000000005, -144.00000000000003) rectangle (365.30000000000007,-146.10000000000002);
\draw(358.75000000000006, -146.20000000000005) node[anchor=north west,align=left] {Varieties\\ of \\ lattices};
\draw (358.75000000000006, -146.20000000000005) rectangle (361.6000000000001,-147.80000000000004);
\draw(374.55000000000007, -1) node[anchor=north west,align=left] {\LARGE Combinatorics};
\draw (374.55000000000007, -1) rectangle (397.75000000000006,-52.50000000000001);
\draw(375.55000000000007, -2) node[anchor=north west,align=left] {\large Graph theory};
\draw (375.55000000000007, -2) rectangle (388.1500000000001,-38.300000000000004);
\draw(376.55000000000007, -3) node[anchor=north west,align=left] {Isomorphism \\ problems in graph \\ theory (reconstruction\\ conjecture,\\ etc.) and \\ homomorphisms (subgraph\\ embedding, etc.)};
\draw (376.55000000000007, -3) rectangle (382.9000000000001,-6.6);
\draw(383.00000000000006, -3) node[anchor=north west,align=left] {Graphs and\\ abstract \\ algebra (groups,\\ rings,\\ fields, etc.)};
\draw (383.00000000000006, -3) rectangle (387.6000000000001,-5.6);
\draw(383.00000000000006, -5.7) node[anchor=north west,align=left] {Trees};
\draw (383.00000000000006, -5.7) rectangle (384.8500000000001,-6.3);
\draw(376.55000000000007, -6.7) node[anchor=north west,align=left] {Graph representations\\ (geometric\\ and \\ intersection \\ representations, etc.)};
\draw (376.55000000000007, -6.7) rectangle (382.6500000000001,-9.3);
\draw(382.75000000000006, -6.7) node[anchor=north west,align=left] {Games on \\ graphs \\ (graph-theoretic\\ aspects)};
\draw (382.75000000000006, -6.7) rectangle (387.3500000000001,-8.8);
\draw(376.55000000000007, -9.4) node[anchor=north west,align=left] {Edge subsets with\\ special properties\\ (factorization,\\ matching, \\ partitioning, covering\\ and packing, etc.)};
\draw (376.55000000000007, -9.4) rectangle (382.6500000000001,-12.5);
\draw(382.75000000000006, -9.4) node[anchor=north west,align=left] {Structural \\ characterization\\ of families\\ of graphs};
\draw (382.75000000000006, -9.4) rectangle (387.3500000000001,-11.5);
\draw(376.55000000000007, -12.600000000000001) node[anchor=north west,align=left] {Vertex subsets\\ with special\\ properties \\ (dominating sets,\\ independent \\ sets, cliques, etc.)};
\draw (376.55000000000007, -12.600000000000001) rectangle (382.1500000000001,-15.700000000000001);
\draw(382.25000000000006, -12.600000000000001) node[anchor=north west,align=left] {Graph \\ operations (line\\ graphs, \\ products, etc.)};
\draw (382.25000000000006, -12.600000000000001) rectangle (386.8500000000001,-14.700000000000001);
\draw(376.55000000000007, -15.8) node[anchor=north west,align=left] {Graph labelling\\ (graceful\\ graphs, \\ bandwidth, etc.)};
\draw (376.55000000000007, -15.8) rectangle (381.1500000000001,-17.900000000000002);
\draw(381.25000000000006, -15.8) node[anchor=north west,align=left] {Random \\ graphs \\ (graph-theoretic\\ aspects)};
\draw (381.25000000000006, -15.8) rectangle (385.8500000000001,-17.900000000000002);
\draw(385.95000000000005, -15.8) node[anchor=north west,align=left] {Graph\\ minors};
\draw (385.95000000000005, -15.8) rectangle (388.05000000000007,-16.900000000000002);
\draw(376.55000000000007, -18.0) node[anchor=north west,align=left] {Small world\\ graphs, complex\\ networks\\ (graph-theoretic\\ aspects)};
\draw (376.55000000000007, -18.0) rectangle (381.1500000000001,-20.6);
\draw(381.25000000000006, -18.0) node[anchor=north west,align=left] {Graph \\ algorithms \\ (graph-theoretic\\ aspects)};
\draw (381.25000000000006, -18.0) rectangle (385.8500000000001,-20.1);
\draw(376.55000000000007, -20.7) node[anchor=north west,align=left] {Graphical \\ indices (Wiener\\ index, Zagreb\\ index, Randić\\ index, etc.)};
\draw (376.55000000000007, -20.7) rectangle (380.9000000000001,-23.3);
\draw(381.00000000000006, -20.7) node[anchor=north west,align=left] {Planar graphs;\\ geometric\\ and topological\\ aspects\\ of graph theory};
\draw (381.00000000000006, -20.7) rectangle (385.3500000000001,-23.3);
\draw(385.45000000000005, -20.7) node[anchor=north west,align=left] {Distance\\ in\\ graphs};
\draw (385.45000000000005, -20.7) rectangle (388.05000000000007,-22.3);
\draw(376.55000000000007, -23.4) node[anchor=north west,align=left] {Eulerian \\ and Hamiltonian\\ graphs};
\draw (376.55000000000007, -23.4) rectangle (380.9000000000001,-25.0);
\draw(381.00000000000006, -23.4) node[anchor=north west,align=left] {Graphs and\\ linear algebra\\ (matrices,\\ eigenvalues,\\ etc.)};
\draw (381.00000000000006, -23.4) rectangle (385.1000000000001,-26.0);
\draw(385.20000000000005, -23.4) node[anchor=north west,align=left] {Flows \\ in graphs};
\draw (385.20000000000005, -23.4) rectangle (388.05000000000007,-24.5);
\draw(385.20000000000005, -24.6) node[anchor=north west,align=left] {Paths and\\ cycles};
\draw (385.20000000000005, -24.6) rectangle (388.05000000000007,-25.700000000000003);
\draw(376.55000000000007, -26.099999999999998) node[anchor=north west,align=left] {Graph designs\\ and \\ isomorphic \\ decomposition};
\draw (376.55000000000007, -26.099999999999998) rectangle (380.4000000000001,-28.2);
\draw(380.50000000000006, -26.099999999999998) node[anchor=north west,align=left] {Fractional\\ graph \\ theory, fuzzy\\ graph theory};
\draw (380.50000000000006, -26.099999999999998) rectangle (384.3500000000001,-28.2);
\draw(384.45000000000005, -26.099999999999998) node[anchor=north west,align=left] {Signed \\ and weighted\\ graphs};
\draw (384.45000000000005, -26.099999999999998) rectangle (388.05000000000007,-27.7);
\draw(376.55000000000007, -28.299999999999997) node[anchor=north west,align=left] {Connectivity};
\draw (376.55000000000007, -28.299999999999997) rectangle (380.1500000000001,-29.4);
\draw(380.25000000000006, -28.299999999999997) node[anchor=north west,align=left] {Applications\\ of \\ graph theory};
\draw (380.25000000000006, -28.299999999999997) rectangle (383.8500000000001,-29.9);
\draw(383.95000000000005, -28.299999999999997) node[anchor=north west,align=left] {Coloring\\ of graphs\\ and \\ hypergraphs};
\draw (383.95000000000005, -28.299999999999997) rectangle (387.30000000000007,-30.4);
\draw(376.55000000000007, -30.499999999999996) node[anchor=north west,align=left] {Directed\\ graphs \\ (digraphs),\\ tournaments};
\draw (376.55000000000007, -30.499999999999996) rectangle (379.9000000000001,-32.599999999999994);
\draw(380.00000000000006, -30.499999999999996) node[anchor=north west,align=left] {Enumeration\\ in graph\\ theory};
\draw (380.00000000000006, -30.499999999999996) rectangle (383.3500000000001,-32.099999999999994);
\draw(383.45000000000005, -30.499999999999996) node[anchor=north west,align=left] {Graph\\ polynomials};
\draw (383.45000000000005, -30.499999999999996) rectangle (386.80000000000007,-32.099999999999994);
\draw(376.55000000000007, -32.699999999999996) node[anchor=north west,align=left] {Density\\ (toughness,\\ etc.)};
\draw (376.55000000000007, -32.699999999999996) rectangle (379.9000000000001,-34.3);
\draw(380.00000000000006, -32.699999999999996) node[anchor=north west,align=left] {Generalized\\ Ramsey\\ theory};
\draw (380.00000000000006, -32.699999999999996) rectangle (383.3500000000001,-34.3);
\draw(383.45000000000005, -32.699999999999996) node[anchor=north west,align=left] {Hypergraphs};
\draw (383.45000000000005, -32.699999999999996) rectangle (386.80000000000007,-33.8);
\draw(376.55000000000007, -34.4) node[anchor=north west,align=left] {Random\\ walks \\ on graphs};
\draw (376.55000000000007, -34.4) rectangle (379.4000000000001,-36.0);
\draw(379.50000000000006, -34.4) node[anchor=north west,align=left] {Extremal\\ problems\\ in graph\\ theory};
\draw (379.50000000000006, -34.4) rectangle (382.1000000000001,-36.5);
\draw(382.20000000000005, -34.4) node[anchor=north west,align=left] {Expander\\ graphs};
\draw (382.20000000000005, -34.4) rectangle (384.80000000000007,-35.5);
\draw(384.9000000000001, -34.4) node[anchor=north west,align=left] {Infinite\\ graphs};
\draw (384.9000000000001, -34.4) rectangle (387.5000000000001,-35.5);
\draw(376.55000000000007, -36.6) node[anchor=north west,align=left] {Chemical\\ graph\\ theory};
\draw (376.55000000000007, -36.6) rectangle (379.1500000000001,-38.2);
\draw(379.25000000000006, -36.6) node[anchor=north west,align=left] {Vertex\\ degrees};
\draw (379.25000000000006, -36.6) rectangle (381.6000000000001,-37.7);
\draw(381.70000000000005, -36.6) node[anchor=north west,align=left] {Perfect\\ graphs};
\draw (381.70000000000005, -36.6) rectangle (384.05000000000007,-37.7);
\draw(388.25000000000006, -2) node[anchor=north west,align=left] {\large Enumerative combinatorics};
\draw (388.25000000000006, -2) rectangle (397.65000000000003,-14.0);
\draw(389.25000000000006, -3) node[anchor=north west,align=left] {Exact enumeration\\ problems,\\ generating\\ functions};
\draw (389.25000000000006, -3) rectangle (394.1000000000001,-5.1);
\draw(394.20000000000005, -3) node[anchor=north west,align=left] {Asymptotic\\ enumeration};
\draw (394.20000000000005, -3) rectangle (397.55000000000007,-4.6);
\draw(389.25000000000006, -5.2) node[anchor=north west,align=left] {\(q\)-calculus\\ and related\\ topics};
\draw (389.25000000000006, -5.2) rectangle (393.3500000000001,-7.300000000000001);
\draw(393.45000000000005, -5.2) node[anchor=north west,align=left] {Permutations,\\ words,\\ matrices};
\draw (393.45000000000005, -5.2) rectangle (397.30000000000007,-6.800000000000001);
\draw(389.25000000000006, -7.4) node[anchor=north west,align=left] {Factorials,\\ binomial \\ coefficients,\\ combinatorial\\ functions};
\draw (389.25000000000006, -7.4) rectangle (393.1000000000001,-10.0);
\draw(393.20000000000005, -7.4) node[anchor=north west,align=left] {Combinatorial\\ aspects\\ of partitions\\ of integers};
\draw (393.20000000000005, -7.4) rectangle (397.05000000000007,-9.5);
\draw(389.25000000000006, -10.100000000000001) node[anchor=north west,align=left] {Combinatorial\\ identities,\\ bijective\\ combinatorics};
\draw (389.25000000000006, -10.100000000000001) rectangle (393.1000000000001,-12.200000000000001);
\draw(393.20000000000005, -10.100000000000001) node[anchor=north west,align=left] {Combinatorial\\ inequalities};
\draw (393.20000000000005, -10.100000000000001) rectangle (397.05000000000007,-11.700000000000001);
\draw(389.25000000000006, -12.3) node[anchor=north west,align=left] {Partitions\\ of sets};
\draw (389.25000000000006, -12.3) rectangle (392.3500000000001,-13.9);
\draw(392.45000000000005, -12.3) node[anchor=north west,align=left] {Umbral\\ calculus};
\draw (392.45000000000005, -12.3) rectangle (395.05000000000007,-13.4);
\draw(388.25000000000006, -14.1) node[anchor=north west,align=left] {\large Extremal combinatorics};
\draw (388.25000000000006, -14.1) rectangle (397.65000000000003,-20.5);
\draw(389.25000000000006, -15.1) node[anchor=north west,align=left] {Probabilistic \\ methods in extremal\\ combinatorics,\\ including \\ polynomial methods \\ (combinatorial \\ Nullstellensatz, etc.)};
\draw (389.25000000000006, -15.1) rectangle (395.3500000000001,-18.7);
\draw(395.45000000000005, -15.1) node[anchor=north west,align=left] {Ramsey\\ theory};
\draw (395.45000000000005, -15.1) rectangle (397.55000000000007,-16.2);
\draw(389.25000000000006, -18.8) node[anchor=north west,align=left] {Transversal\\ (matching)\\ theory};
\draw (389.25000000000006, -18.8) rectangle (392.6000000000001,-20.400000000000002);
\draw(392.70000000000005, -18.8) node[anchor=north west,align=left] {Extremal\\ set\\ theory};
\draw (392.70000000000005, -18.8) rectangle (395.30000000000007,-20.400000000000002);
\draw(388.25000000000006, -20.6) node[anchor=north west,align=left] {\large Algebraic combinatorics};
\draw (388.25000000000006, -20.6) rectangle (397.65000000000003,-32.6);
\draw(389.25000000000006, -21.6) node[anchor=north west,align=left] {Combinatorial\\ aspects\\ of representation\\ theory};
\draw (389.25000000000006, -21.6) rectangle (394.1000000000001,-23.700000000000003);
\draw(389.25000000000006, -23.8) node[anchor=north west,align=left] {Symmetric\\ functions\\ and \\ generalizations};
\draw (389.25000000000006, -23.8) rectangle (393.6000000000001,-25.900000000000002);
\draw(393.70000000000005, -23.8) node[anchor=north west,align=left] {Combinatorial\\ aspects\\ of algebraic\\ geometry};
\draw (393.70000000000005, -23.8) rectangle (397.55000000000007,-25.900000000000002);
\draw(389.25000000000006, -26.0) node[anchor=north west,align=left] {Association\\ schemes, \\ strongly \\ regular graphs};
\draw (389.25000000000006, -26.0) rectangle (393.3500000000001,-28.1);
\draw(393.45000000000005, -26.0) node[anchor=north west,align=left] {Combinatorial\\ aspects\\ of commutative\\ algebra};
\draw (393.45000000000005, -26.0) rectangle (397.55000000000007,-28.1);
\draw(389.25000000000006, -28.200000000000003) node[anchor=north west,align=left] {Combinatorial\\ aspects\\ of groups \\ and algebras};
\draw (389.25000000000006, -28.200000000000003) rectangle (393.1000000000001,-30.300000000000004);
\draw(393.20000000000005, -28.200000000000003) node[anchor=north west,align=left] {Group actions\\ on \\ combinatorial\\ structures};
\draw (393.20000000000005, -28.200000000000003) rectangle (397.05000000000007,-30.300000000000004);
\draw(389.25000000000006, -30.400000000000002) node[anchor=north west,align=left] {Combinatorial\\ aspects\\ of simplicial\\ complexes};
\draw (389.25000000000006, -30.400000000000002) rectangle (393.1000000000001,-32.5);
\draw(388.25000000000006, -32.7) node[anchor=north west,align=left] {\large Computational methods\\ for problems \\ pertaining to combinatorics};
\draw (388.25000000000006, -32.7) rectangle (397.22,-34.300000000000004);
\draw(388.25000000000006, -34.400000000000006) node[anchor=north west,align=left] {\large History of \\ combinatorics};
\draw (388.25000000000006, -34.400000000000006) rectangle (392.88000000000005,-35.50000000000001);
\draw(375.55000000000007, -38.400000000000006) node[anchor=north west,align=left] {\large Designs and configurations};
\draw (375.55000000000007, -38.400000000000006) rectangle (386.20000000000005,-52.400000000000006);
\draw(376.55000000000007, -39.400000000000006) node[anchor=north west,align=left] {Combinatorial\\ aspects of \\ difference sets\\ (number-theoretic,\\ group-theoretic, etc.)};
\draw (376.55000000000007, -39.400000000000006) rectangle (382.6500000000001,-42.50000000000001);
\draw(382.75000000000006, -39.400000000000006) node[anchor=north west,align=left] {Polyominoes};
\draw (382.75000000000006, -39.400000000000006) rectangle (386.1000000000001,-40.50000000000001);
\draw(382.75000000000006, -40.60000000000001) node[anchor=north west,align=left] {Triple\\ systems};
\draw (382.75000000000006, -40.60000000000001) rectangle (385.1000000000001,-41.70000000000001);
\draw(376.55000000000007, -42.60000000000001) node[anchor=north west,align=left] {Combinatorial\\ aspects of\\ matrices \\ (incidence, \\ Hadamard, etc.)};
\draw (376.55000000000007, -42.60000000000001) rectangle (380.9000000000001,-45.20000000000001);
\draw(381.00000000000006, -42.60000000000001) node[anchor=north west,align=left] {Combinatorial\\ aspects\\ of tessellation\\ and \\ tiling problems};
\draw (381.00000000000006, -42.60000000000001) rectangle (385.3500000000001,-45.20000000000001);
\draw(376.55000000000007, -45.300000000000004) node[anchor=north west,align=left] {Other \\ designs, \\ configurations};
\draw (376.55000000000007, -45.300000000000004) rectangle (380.6500000000001,-46.900000000000006);
\draw(380.75000000000006, -45.300000000000004) node[anchor=north west,align=left] {Combinatorial\\ aspects\\ of \\ block designs};
\draw (380.75000000000006, -45.300000000000004) rectangle (384.6000000000001,-47.400000000000006);
\draw(376.55000000000007, -47.50000000000001) node[anchor=north west,align=left] {Orthogonal\\ arrays, Latin\\ squares,\\ Room squares};
\draw (376.55000000000007, -47.50000000000001) rectangle (380.4000000000001,-49.60000000000001);
\draw(380.50000000000006, -47.50000000000001) node[anchor=north west,align=left] {Combinatorial\\ aspects\\ of finite\\ geometries};
\draw (380.50000000000006, -47.50000000000001) rectangle (384.3500000000001,-49.60000000000001);
\draw(376.55000000000007, -49.7) node[anchor=north west,align=left] {Combinatorial\\ aspects\\ of matroids\\ and geometric\\ lattices};
\draw (376.55000000000007, -49.7) rectangle (380.4000000000001,-52.300000000000004);
\draw(380.50000000000006, -49.7) node[anchor=north west,align=left] {Combinatorial\\ aspects\\ of packing\\ and covering};
\draw (380.50000000000006, -49.7) rectangle (384.3500000000001,-51.800000000000004);
\draw(374.55000000000007, -52.60000000000001) node[anchor=north west,align=left] {\LARGE General algebraic systems};
\draw (374.55000000000007, -52.60000000000001) rectangle (395.55000000000007,-76.10000000000001);
\draw(375.55000000000007, -53.60000000000001) node[anchor=north west,align=left] {\large Algebraic structures};
\draw (375.55000000000007, -53.60000000000001) rectangle (386.20000000000005,-68.30000000000001);
\draw(376.55000000000007, -54.60000000000001) node[anchor=north west,align=left] {Heterogeneousalgebras};
\draw (376.55000000000007, -54.60000000000001) rectangle (382.4000000000001,-56.20000000000001);
\draw(382.50000000000006, -54.60000000000001) node[anchor=north west,align=left] {Subalgebras,\\ congruence\\ relations};
\draw (382.50000000000006, -54.60000000000001) rectangle (386.1000000000001,-56.70000000000001);
\draw(376.55000000000007, -56.80000000000001) node[anchor=north west,align=left] {Applications\\ of universal\\ algebra in \\ computer science};
\draw (376.55000000000007, -56.80000000000001) rectangle (381.1500000000001,-58.90000000000001);
\draw(381.25000000000006, -56.80000000000001) node[anchor=north west,align=left] {Operations and\\ polynomials\\ in algebraic\\ structures,\\ primal algebras};
\draw (381.25000000000006, -56.80000000000001) rectangle (385.6000000000001,-59.40000000000001);
\draw(376.55000000000007, -59.50000000000001) node[anchor=north west,align=left] {Automorphisms\\ and \\ endomorphisms \\ of algebraic\\ structures};
\draw (376.55000000000007, -59.50000000000001) rectangle (380.6500000000001,-62.10000000000001);
\draw(380.75000000000006, -59.50000000000001) node[anchor=north west,align=left] {Word problems\\ (aspects\\ of algebraic\\ structures)};
\draw (380.75000000000006, -59.50000000000001) rectangle (384.6000000000001,-61.60000000000001);
\draw(376.55000000000007, -62.20000000000001) node[anchor=north west,align=left] {Relational\\ systems,\\ laws of\\ composition};
\draw (376.55000000000007, -62.20000000000001) rectangle (379.9000000000001,-64.30000000000001);
\draw(380.00000000000006, -62.20000000000001) node[anchor=north west,align=left] {Equational\\ compactness};
\draw (380.00000000000006, -62.20000000000001) rectangle (383.3500000000001,-63.80000000000001);
\draw(383.45000000000005, -62.20000000000001) node[anchor=north west,align=left] {Partial\\ algebras};
\draw (383.45000000000005, -62.20000000000001) rectangle (386.05000000000007,-63.30000000000001);
\draw(376.55000000000007, -64.4) node[anchor=north west,align=left] {Structure\\ theory of\\ algebraic\\ structures};
\draw (376.55000000000007, -64.4) rectangle (379.6500000000001,-66.5);
\draw(379.75000000000006, -64.4) node[anchor=north west,align=left] {Infinitary\\ algebras};
\draw (379.75000000000006, -64.4) rectangle (382.8500000000001,-66.0);
\draw(382.95000000000005, -64.4) node[anchor=north west,align=left] {Fuzzy \\ algebraic\\ structures};
\draw (382.95000000000005, -64.4) rectangle (386.05000000000007,-66.0);
\draw(376.55000000000007, -66.60000000000001) node[anchor=north west,align=left] {Unary \\ algebras};
\draw (376.55000000000007, -66.60000000000001) rectangle (379.1500000000001,-67.7);
\draw(379.25000000000006, -66.60000000000001) node[anchor=north west,align=left] {Finitary\\ algebras};
\draw (379.25000000000006, -66.60000000000001) rectangle (381.8500000000001,-68.2);
\draw(386.30000000000007, -53.60000000000001) node[anchor=north west,align=left] {\large Other classes of algebras};
\draw (386.30000000000007, -53.60000000000001) rectangle (395.45000000000005,-58.50000000000001);
\draw(387.30000000000007, -54.60000000000001) node[anchor=north west,align=left] {Quasivarieties};
\draw (387.30000000000007, -54.60000000000001) rectangle (391.4000000000001,-55.70000000000001);
\draw(391.50000000000006, -54.60000000000001) node[anchor=north west,align=left] {Natural \\ dualities for\\ classes\\ of algebras};
\draw (391.50000000000006, -54.60000000000001) rectangle (395.3500000000001,-56.70000000000001);
\draw(387.30000000000007, -56.80000000000001) node[anchor=north west,align=left] {Categories\\ of\\ algebras};
\draw (387.30000000000007, -56.80000000000001) rectangle (390.4000000000001,-58.40000000000001);
\draw(390.50000000000006, -56.80000000000001) node[anchor=north west,align=left] {Axiomatic\\ model\\ classes};
\draw (390.50000000000006, -56.80000000000001) rectangle (393.3500000000001,-58.40000000000001);
\draw(386.30000000000007, -58.60000000000001) node[anchor=north west,align=left] {\large Computational methods for\\ problems pertaining to\\ general algebraic systems};
\draw (386.30000000000007, -58.60000000000001) rectangle (394.6500000000001,-60.20000000000001);
\draw(386.30000000000007, -60.30000000000001) node[anchor=north west,align=left] {\large History of general\\ algebraic systems};
\draw (386.30000000000007, -60.30000000000001) rectangle (392.4800000000001,-61.40000000000001);
\draw(375.55000000000007, -68.4) node[anchor=north west,align=left] {\large Varieties};
\draw (375.55000000000007, -68.4) rectangle (385.70000000000005,-76.0);
\draw(376.55000000000007, -69.4) node[anchor=north west,align=left] {Products, \\ amalgamated products,\\ and other\\ kinds of limits\\ and colimits};
\draw (376.55000000000007, -69.4) rectangle (382.4000000000001,-72.0);
\draw(382.50000000000006, -69.4) node[anchor=north west,align=left] {Equational\\ logic,\\ Mal’tsev\\ conditions};
\draw (382.50000000000006, -69.4) rectangle (385.6000000000001,-71.5);
\draw(376.55000000000007, -72.10000000000001) node[anchor=north west,align=left] {Congruence \\ modularity, \\ congruence \\ distributivity};
\draw (376.55000000000007, -72.10000000000001) rectangle (380.6500000000001,-74.2);
\draw(380.75000000000006, -72.10000000000001) node[anchor=north west,align=left] {Subdirect \\ products and\\ subdirect\\ irreducibility};
\draw (380.75000000000006, -72.10000000000001) rectangle (384.8500000000001,-74.2);
\draw(376.55000000000007, -74.30000000000001) node[anchor=north west,align=left] {Injectives,\\ projectives};
\draw (376.55000000000007, -74.30000000000001) rectangle (379.9000000000001,-75.9);
\draw(380.00000000000006, -74.30000000000001) node[anchor=north west,align=left] {Lattices\\ of \\ varieties};
\draw (380.00000000000006, -74.30000000000001) rectangle (382.8500000000001,-75.9);
\draw(382.95000000000005, -74.30000000000001) node[anchor=north west,align=left] {Free \\ algebras};
\draw (382.95000000000005, -74.30000000000001) rectangle (385.55000000000007,-75.4);
\draw(374.55000000000007, -76.20000000000002) node[anchor=north west,align=left] {\LARGE Linear and multilinear algebra; matrix theory};
\draw (374.55000000000007, -76.20000000000002) rectangle (394.5900000000001,-122.70000000000002);
\draw(375.55000000000007, -77.20000000000002) node[anchor=north west,align=left] {\large Basic linear algebra};
\draw (375.55000000000007, -77.20000000000002) rectangle (386.70000000000005,-110.00000000000001);
\draw(376.55000000000007, -78.20000000000002) node[anchor=north west,align=left] {Theory of \\ matrix inversion\\ and \\ generalized inverses};
\draw (376.55000000000007, -78.20000000000002) rectangle (382.1500000000001,-80.30000000000001);
\draw(382.25000000000006, -78.20000000000002) node[anchor=north west,align=left] {Vector and\\ tensor \\ algebra, theory\\ of invariants};
\draw (382.25000000000006, -78.20000000000002) rectangle (386.6000000000001,-80.30000000000001);
\draw(376.55000000000007, -80.40000000000002) node[anchor=north west,align=left] {Norms of matrices,\\ numerical\\ range, \\ applications of \\ functional analysis\\ to matrix theory};
\draw (376.55000000000007, -80.40000000000002) rectangle (381.9000000000001,-83.50000000000001);
\draw(382.00000000000006, -80.40000000000002) node[anchor=north west,align=left] {Linear \\ transformations,\\ semilinear \\ transformations};
\draw (382.00000000000006, -80.40000000000002) rectangle (386.6000000000001,-82.50000000000001);
\draw(376.55000000000007, -83.60000000000002) node[anchor=north west,align=left] {Matrix exponential\\ and \\ similar functions\\ of matrices};
\draw (376.55000000000007, -83.60000000000002) rectangle (381.6500000000001,-85.70000000000002);
\draw(381.75000000000006, -83.60000000000002) node[anchor=north west,align=left] {Vector spaces,\\ linear \\ dependence, \\ rank, lineability};
\draw (381.75000000000006, -83.60000000000002) rectangle (386.6000000000001,-85.70000000000002);
\draw(376.55000000000007, -85.80000000000001) node[anchor=north west,align=left] {Linear \\ equations \\ (linear algebraic\\ aspects)};
\draw (376.55000000000007, -85.80000000000001) rectangle (381.4000000000001,-87.9);
\draw(381.50000000000006, -85.80000000000001) node[anchor=north west,align=left] {Determinants,\\ permanents,\\ traces, \\ other special\\ matrix functions};
\draw (381.50000000000006, -85.80000000000001) rectangle (386.1000000000001,-88.4);
\draw(376.55000000000007, -88.50000000000001) node[anchor=north west,align=left] {Diagonalization,\\ Jordan forms};
\draw (376.55000000000007, -88.50000000000001) rectangle (381.1500000000001,-90.10000000000001);
\draw(381.25000000000006, -88.50000000000001) node[anchor=north west,align=left] {Inequalities\\ involving \\ eigenvalues \\ and eigenvectors};
\draw (381.25000000000006, -88.50000000000001) rectangle (385.8500000000001,-90.60000000000001);
\draw(376.55000000000007, -90.70000000000002) node[anchor=north west,align=left] {Canonical\\ forms, \\ reductions, \\ classification};
\draw (376.55000000000007, -90.70000000000002) rectangle (380.6500000000001,-92.80000000000001);
\draw(380.75000000000006, -90.70000000000002) node[anchor=north west,align=left] {Matrices over\\ function rings\\ in one or\\ more variables};
\draw (380.75000000000006, -90.70000000000002) rectangle (384.8500000000001,-92.80000000000001);
\draw(376.55000000000007, -92.90000000000002) node[anchor=north west,align=left] {Factorization\\ of\\ matrices};
\draw (376.55000000000007, -92.90000000000002) rectangle (380.4000000000001,-94.50000000000001);
\draw(380.50000000000006, -92.90000000000002) node[anchor=north west,align=left] {Matrix \\ equations and\\ identities};
\draw (380.50000000000006, -92.90000000000002) rectangle (384.3500000000001,-94.50000000000001);
\draw(376.55000000000007, -94.60000000000002) node[anchor=north west,align=left] {Commutativity\\ of\\ matrices};
\draw (376.55000000000007, -94.60000000000002) rectangle (380.4000000000001,-96.20000000000002);
\draw(380.50000000000006, -94.60000000000002) node[anchor=north west,align=left] {Miscellaneous\\ inequalities\\ involving\\ matrices};
\draw (380.50000000000006, -94.60000000000002) rectangle (384.3500000000001,-96.70000000000002);
\draw(376.55000000000007, -96.80000000000001) node[anchor=north west,align=left] {Applications\\ of Clifford\\ algebras to\\ physics, etc.};
\draw (376.55000000000007, -96.80000000000001) rectangle (380.4000000000001,-98.9);
\draw(380.50000000000006, -96.80000000000001) node[anchor=north west,align=left] {Applications\\ of \\ generalized\\ inverses};
\draw (380.50000000000006, -96.80000000000001) rectangle (384.1000000000001,-98.9);
\draw(384.20000000000005, -96.80000000000001) node[anchor=north west,align=left] {Matrix\\ pencils};
\draw (384.20000000000005, -96.80000000000001) rectangle (386.55000000000007,-97.9);
\draw(376.55000000000007, -99.00000000000001) node[anchor=north west,align=left] {Conditioning\\ of\\ matrices};
\draw (376.55000000000007, -99.00000000000001) rectangle (380.1500000000001,-100.60000000000001);
\draw(380.25000000000006, -99.00000000000001) node[anchor=north west,align=left] {Eigenvalues,\\ singular\\ values, and\\ eigenvectors};
\draw (380.25000000000006, -99.00000000000001) rectangle (383.8500000000001,-101.10000000000001);
\draw(376.55000000000007, -101.20000000000002) node[anchor=north west,align=left] {Linear \\ inequalities\\ of matrices};
\draw (376.55000000000007, -101.20000000000002) rectangle (380.1500000000001,-102.80000000000001);
\draw(380.25000000000006, -101.20000000000002) node[anchor=north west,align=left] {Quadratic \\ and bilinear\\ forms, inner\\ products};
\draw (380.25000000000006, -101.20000000000002) rectangle (383.8500000000001,-103.30000000000001);
\draw(376.55000000000007, -103.4) node[anchor=north west,align=left] {Algebraic\\ systems\\ of matrices};
\draw (376.55000000000007, -103.4) rectangle (379.9000000000001,-105.0);
\draw(380.00000000000006, -103.4) node[anchor=north west,align=left] {Multilinear\\ algebra,\\ tensor\\ calculus};
\draw (380.00000000000006, -103.4) rectangle (383.3500000000001,-105.5);
\draw(383.45000000000005, -103.4) node[anchor=north west,align=left] {Other \\ algebras \\ built from\\ modules};
\draw (383.45000000000005, -103.4) rectangle (386.55000000000007,-105.5);
\draw(376.55000000000007, -105.60000000000001) node[anchor=north west,align=left] {Max-plus\\ and related\\ algebras};
\draw (376.55000000000007, -105.60000000000001) rectangle (379.9000000000001,-107.2);
\draw(380.00000000000006, -105.60000000000001) node[anchor=north west,align=left] {Matrix \\ completion\\ problems};
\draw (380.00000000000006, -105.60000000000001) rectangle (383.1000000000001,-107.2);
\draw(383.20000000000005, -105.60000000000001) node[anchor=north west,align=left] {Inverse\\ problems\\ in linear\\ algebra};
\draw (383.20000000000005, -105.60000000000001) rectangle (386.05000000000007,-107.7);
\draw(376.55000000000007, -107.80000000000001) node[anchor=north west,align=left] {Clifford\\ algebras,\\ spinors};
\draw (376.55000000000007, -107.80000000000001) rectangle (379.4000000000001,-109.4);
\draw(379.50000000000006, -107.80000000000001) node[anchor=north west,align=left] {Exterior\\ algebra,\\ Grassmann\\ algebras};
\draw (379.50000000000006, -107.80000000000001) rectangle (382.3500000000001,-109.9);
\draw(382.45000000000005, -107.80000000000001) node[anchor=north west,align=left] {Linear \\ preserver\\ problems};
\draw (382.45000000000005, -107.80000000000001) rectangle (385.30000000000007,-109.4);
\draw(386.80000000000007, -77.20000000000002) node[anchor=north west,align=left] {\large History of \\ linear algebra};
\draw (386.80000000000007, -77.20000000000002) rectangle (391.74000000000007,-78.30000000000001);
\draw(375.55000000000007, -110.10000000000002) node[anchor=north west,align=left] {\large Special matrices};
\draw (375.55000000000007, -110.10000000000002) rectangle (386.70000000000005,-122.60000000000002);
\draw(376.55000000000007, -111.10000000000002) node[anchor=north west,align=left] {Positive \\ matrices and \\ their \\ generalizations; cones\\ of matrices};
\draw (376.55000000000007, -111.10000000000002) rectangle (382.6500000000001,-113.70000000000002);
\draw(382.75000000000006, -111.10000000000002) node[anchor=north west,align=left] {Boolean \\ and Hadamard\\ matrices};
\draw (382.75000000000006, -111.10000000000002) rectangle (386.3500000000001,-112.70000000000002);
\draw(376.55000000000007, -113.80000000000003) node[anchor=north west,align=left] {Matrices over\\ special \\ rings (quaternions,\\ finite\\ fields, etc.)};
\draw (376.55000000000007, -113.80000000000003) rectangle (381.9000000000001,-116.40000000000002);
\draw(382.00000000000006, -113.80000000000003) node[anchor=north west,align=left] {Hermitian,\\ skew-Hermitian,\\ and \\ related matrices};
\draw (382.00000000000006, -113.80000000000003) rectangle (386.6000000000001,-115.90000000000002);
\draw(376.55000000000007, -116.50000000000003) node[anchor=north west,align=left] {Toeplitz,\\ Cauchy,\\ and related\\ matrices};
\draw (376.55000000000007, -116.50000000000003) rectangle (379.9000000000001,-118.60000000000002);
\draw(380.00000000000006, -116.50000000000003) node[anchor=north west,align=left] {Orthogonal\\ matrices};
\draw (380.00000000000006, -116.50000000000003) rectangle (383.1000000000001,-118.10000000000002);
\draw(383.20000000000005, -116.50000000000003) node[anchor=north west,align=left] {Stochastic\\ matrices};
\draw (383.20000000000005, -116.50000000000003) rectangle (386.30000000000007,-118.10000000000002);
\draw(376.55000000000007, -118.70000000000002) node[anchor=north west,align=left] {Random \\ matrices\\ (algebraic\\ aspects)};
\draw (376.55000000000007, -118.70000000000002) rectangle (379.6500000000001,-120.80000000000001);
\draw(379.75000000000006, -118.70000000000002) node[anchor=north west,align=left] {Fuzzy \\ matrices};
\draw (379.75000000000006, -118.70000000000002) rectangle (382.3500000000001,-119.80000000000001);
\draw(382.45000000000005, -118.70000000000002) node[anchor=north west,align=left] {Matrix\\ Lie \\ algebras};
\draw (382.45000000000005, -118.70000000000002) rectangle (385.05000000000007,-120.30000000000001);
\draw(376.55000000000007, -120.90000000000002) node[anchor=north west,align=left] {Sign \\ pattern \\ matrices};
\draw (376.55000000000007, -120.90000000000002) rectangle (379.1500000000001,-122.50000000000001);
\draw(379.25000000000006, -120.90000000000002) node[anchor=north west,align=left] {Matrices\\ of\\ integers};
\draw (379.25000000000006, -120.90000000000002) rectangle (381.8500000000001,-122.50000000000001);
\draw(374.55000000000007, -122.80000000000001) node[anchor=north west,align=left] {\LARGE History and biography};
\draw (374.55000000000007, -122.80000000000001) rectangle (391.25000000000006,-148.4);
\draw(375.55000000000007, -123.80000000000001) node[anchor=north west,align=left] {\large History of mathematics and mathematicians};
\draw (375.55000000000007, -123.80000000000001) rectangle (391.1500000000001,-148.3);
\draw(376.55000000000007, -124.80000000000001) node[anchor=north west,align=left] {Bibliographicstudies};
\draw (376.55000000000007, -124.80000000000001) rectangle (382.1500000000001,-126.4);
\draw(382.25000000000006, -124.80000000000001) node[anchor=north west,align=left] {History of \\ mathematics of the\\ indigenous \\ cultures of Africa,\\ Asia, and Oceania};
\draw (382.25000000000006, -124.80000000000001) rectangle (387.6000000000001,-127.4);
\draw(387.70000000000005, -124.80000000000001) node[anchor=north west,align=left] {History of\\ mathematics\\ in Ancient\\ Babylon};
\draw (387.70000000000005, -124.80000000000001) rectangle (391.05000000000007,-126.9);
\draw(376.55000000000007, -127.50000000000001) node[anchor=north west,align=left] {History of \\ mathematics of the\\ indigenous \\ cultures of Europe\\ (pre-Greek, etc.)};
\draw (376.55000000000007, -127.50000000000001) rectangle (381.6500000000001,-130.10000000000002);
\draw(381.75000000000006, -127.50000000000001) node[anchor=north west,align=left] {Ethnomathematics,\\ general};
\draw (381.75000000000006, -127.50000000000001) rectangle (386.6000000000001,-129.10000000000002);
\draw(386.70000000000005, -127.50000000000001) node[anchor=north west,align=left] {History of\\ mathematics\\ in Paleolithic\\ and \\ Neolithic times};
\draw (386.70000000000005, -127.50000000000001) rectangle (391.05000000000007,-130.10000000000002);
\draw(376.55000000000007, -130.20000000000002) node[anchor=north west,align=left] {History of \\ mathematics \\ in late antiquity\\ and \\ medieval Europe};
\draw (376.55000000000007, -130.20000000000002) rectangle (381.4000000000001,-132.8);
\draw(381.50000000000006, -130.20000000000002) node[anchor=north west,align=left] {History of \\ mathematics at\\ institutions\\ and academies\\ (non-university)};
\draw (381.50000000000006, -130.20000000000002) rectangle (386.1000000000001,-132.8);
\draw(386.20000000000005, -130.20000000000002) node[anchor=north west,align=left] {History of \\ mathematics \\ in Ancient \\ Greece and Rome};
\draw (386.20000000000005, -130.20000000000002) rectangle (390.55000000000007,-132.3);
\draw(376.55000000000007, -132.9) node[anchor=north west,align=left] {History of \\ mathematics in\\ the 15th and\\ 16th centuries,\\ Renaissance};
\draw (376.55000000000007, -132.9) rectangle (380.9000000000001,-135.5);
\draw(381.00000000000006, -132.9) node[anchor=north west,align=left] {Collected or\\ selected works;\\ reprintings\\ or translations\\ of classics};
\draw (381.00000000000006, -132.9) rectangle (385.3500000000001,-135.5);
\draw(385.45000000000005, -132.9) node[anchor=north west,align=left] {History of \\ mathematics of\\ the indigenous\\ cultures of\\ the Americas};
\draw (385.45000000000005, -132.9) rectangle (389.55000000000007,-135.5);
\draw(376.55000000000007, -135.60000000000002) node[anchor=north west,align=left] {History \\ of mathematics\\ in \\ Ancient Egypt};
\draw (376.55000000000007, -135.60000000000002) rectangle (380.6500000000001,-137.70000000000002);
\draw(380.75000000000006, -135.60000000000002) node[anchor=north west,align=left] {History of\\ mathematics\\ in \\ Southeast Asia};
\draw (380.75000000000006, -135.60000000000002) rectangle (384.8500000000001,-137.70000000000002);
\draw(384.95000000000005, -135.60000000000002) node[anchor=north west,align=left] {Biographies,\\ obituaries,\\ personalia,\\ bibliographies};
\draw (384.95000000000005, -135.60000000000002) rectangle (389.05000000000007,-137.70000000000002);
\draw(376.55000000000007, -137.8) node[anchor=north west,align=left] {Sociology\\ (and \\ profession) of\\ mathematics};
\draw (376.55000000000007, -137.8) rectangle (380.6500000000001,-139.9);
\draw(380.75000000000006, -137.8) node[anchor=north west,align=left] {Historiography};
\draw (380.75000000000006, -137.8) rectangle (384.8500000000001,-138.9);
\draw(384.95000000000005, -137.8) node[anchor=north west,align=left] {History of \\ mathematics \\ in the Golden\\ Age of Islam};
\draw (384.95000000000005, -137.8) rectangle (388.80000000000007,-139.9);
\draw(376.55000000000007, -140.0) node[anchor=north west,align=left] {General \\ histories, \\ source books};
\draw (376.55000000000007, -140.0) rectangle (380.1500000000001,-141.6);
\draw(380.25000000000006, -140.0) node[anchor=north west,align=left] {History of\\ mathematics\\ in the\\ 17th century};
\draw (380.25000000000006, -140.0) rectangle (383.8500000000001,-142.1);
\draw(383.95000000000005, -140.0) node[anchor=north west,align=left] {History of\\ mathematics\\ in the\\ 18th century};
\draw (383.95000000000005, -140.0) rectangle (387.55000000000007,-142.1);
\draw(387.6500000000001, -140.0) node[anchor=north west,align=left] {History of\\ mathematics\\ in China};
\draw (387.6500000000001, -140.0) rectangle (391.0000000000001,-141.6);
\draw(376.55000000000007, -142.20000000000002) node[anchor=north west,align=left] {History of\\ mathematics\\ in the\\ 19th century};
\draw (376.55000000000007, -142.20000000000002) rectangle (380.1500000000001,-144.3);
\draw(380.25000000000006, -142.20000000000002) node[anchor=north west,align=left] {History of\\ mathematics\\ in the\\ 20th century};
\draw (380.25000000000006, -142.20000000000002) rectangle (383.8500000000001,-144.3);
\draw(383.95000000000005, -142.20000000000002) node[anchor=north west,align=left] {History of\\ mathematics\\ in the\\ 21st century};
\draw (383.95000000000005, -142.20000000000002) rectangle (387.55000000000007,-144.3);
\draw(387.6500000000001, -142.20000000000002) node[anchor=north west,align=left] {History of\\ mathematics\\ in Japan};
\draw (387.6500000000001, -142.20000000000002) rectangle (391.0000000000001,-143.8);
\draw(376.55000000000007, -144.4) node[anchor=north west,align=left] {Development\\ of \\ contemporary\\ mathematics};
\draw (376.55000000000007, -144.4) rectangle (380.1500000000001,-146.5);
\draw(380.25000000000006, -144.4) node[anchor=north west,align=left] {Future \\ perspectives\\ in \\ mathematics};
\draw (380.25000000000006, -144.4) rectangle (383.8500000000001,-146.5);
\draw(383.95000000000005, -144.4) node[anchor=north west,align=left] {History of\\ mathematics\\ at specific\\ universities};
\draw (383.95000000000005, -144.4) rectangle (387.55000000000007,-146.5);
\draw(387.6500000000001, -144.4) node[anchor=north west,align=left] {History of\\ mathematics\\ in India};
\draw (387.6500000000001, -144.4) rectangle (391.0000000000001,-146.0);
\draw(376.55000000000007, -146.60000000000002) node[anchor=north west,align=left] {Schools\\ of \\ mathematics};
\draw (376.55000000000007, -146.60000000000002) rectangle (379.9000000000001,-148.20000000000002);
\end{tikzpicture}

\end{document}