\documentclass[12pt]{article}
\usepackage[utf8]{inputenc}
\usepackage{pgf,tikz,pgfplots}
\pgfplotsset{compat=1.15}
\usepackage{mathrsfs}
\usetikzlibrary{arrows}
\usepackage{fontspec}
\setmainfont[Renderer=ICU,Mapping=tex-text]{Cousine}
\usepackage{amssymb}
\usepackage[paperwidth=398.9400000000001cm,paperheight=151.5cm,left=0.1cm,right=0.1cm,top=0.1cm,bottom=0.1cm]{geometry}
\begin{document}\begin{tikzpicture}[line cap=round,line join=round,>=triangle 45,x=1cm,y=1cm]
\clip(0, 0)rectangle(394.9400000000001, -147.5);

\draw(0, 0) node[anchor=north west,align=left] {\Huge half};
\draw (0, 0) rectangle (394.9400000000001,-147.5);
\draw(1, -1) node[anchor=north west,align=left] {\LARGE Partial differential equations};
\draw (1, -1) rectangle (75.86999999999999,-51.9);
\draw(2, -2) node[anchor=north west,align=left] {\large General higher-order partial differential equations and systems of higher-order partial differential equations};
\draw (2, -2) rectangle (42.6,-8.4);
\draw(3, -3) node[anchor=north west,align=left] {Initial value\\  problems\\ for linear\\ higher-order PDEs};
\draw (3, -3) rectangle (7.85,-5.1);
\draw(7.949999999999999, -3) node[anchor=north west,align=left] {Boundary value\\ problems for\\  linear\\ higher-order PDEs};
\draw (7.949999999999999, -3) rectangle (12.799999999999999,-5.1);
\draw(12.899999999999999, -3) node[anchor=north west,align=left] {Initial-boundary\\  value\\ problems for\\ linear\\ higher-order PDEs};
\draw (12.899999999999999, -3) rectangle (17.75,-5.6);
\draw(17.849999999999998, -3) node[anchor=north west,align=left] {Initial\\ value problems\\  for\\ nonlinear\\ higher-order PDEs};
\draw (17.849999999999998, -3) rectangle (22.699999999999996,-5.6);
\draw(22.799999999999997, -3) node[anchor=north west,align=left] {Boundary\\ value problems\\ for nonlinear\\ higher-order PDEs};
\draw (22.799999999999997, -3) rectangle (27.65,-5.1);
\draw(27.749999999999996, -3) node[anchor=north west,align=left] {Initial-boundary\\  value\\ problems for\\ nonlinear\\ higher-order PDEs};
\draw (27.749999999999996, -3) rectangle (32.599999999999994,-5.6);
\draw(32.699999999999996, -3) node[anchor=north west,align=left] {Systems\\ of linear\\ higher-order PDEs};
\draw (32.699999999999996, -3) rectangle (37.55,-4.6);
\draw(37.65, -3) node[anchor=north west,align=left] {Initial value\\  problems\\ for systems\\ of linear\\ higher-order PDEs};
\draw (37.65, -3) rectangle (42.5,-5.6);
\draw(3, -5.7) node[anchor=north west,align=left] {Boundary value\\  problems\\ for systems\\ of linear\\ higher-order PDEs};
\draw (3, -5.7) rectangle (7.85,-8.3);
\draw(7.949999999999999, -5.7) node[anchor=north west,align=left] {Initial-boundary\\ value problems\\ for systems\\  of linear\\ higher-order PDEs};
\draw (7.949999999999999, -5.7) rectangle (12.799999999999999,-8.3);
\draw(12.899999999999999, -5.7) node[anchor=north west,align=left] {Systems\\ of nonlinear\\ higher-order PDEs};
\draw (12.899999999999999, -5.7) rectangle (17.75,-7.300000000000001);
\draw(17.849999999999998, -5.7) node[anchor=north west,align=left] {Initial value\\ problems for\\  systems of\\ nonlinear\\ higher-order PDEs};
\draw (17.849999999999998, -5.7) rectangle (22.699999999999996,-8.3);
\draw(22.799999999999997, -5.7) node[anchor=north west,align=left] {Boundary value\\ problems for\\  systems of\\ nonlinear\\ higher-order PDEs};
\draw (22.799999999999997, -5.7) rectangle (27.65,-8.3);
\draw(27.749999999999996, -5.7) node[anchor=north west,align=left] {Initial-boundary\\ value problems\\ for systems\\ of nonlinear\\ higher-order PDEs};
\draw (27.749999999999996, -5.7) rectangle (32.599999999999994,-8.3);
\draw(32.699999999999996, -5.7) node[anchor=north west,align=left] {Linear\\ higher-order\\ PDEs};
\draw (32.699999999999996, -5.7) rectangle (36.3,-7.300000000000001);
\draw(36.4, -5.7) node[anchor=north west,align=left] {Nonlinear\\ higher-order\\ PDEs};
\draw (36.4, -5.7) rectangle (40.0,-7.300000000000001);
\draw(42.7, -2) node[anchor=north west,align=left] {\large Partial differential equations and systems of partial differential equations with constant coefficients};
\draw (42.7, -2) rectangle (75.23,-6.199999999999999);
\draw(43.7, -3) node[anchor=north west,align=left] {Convexity\\ properties of\\ solutions to PDEs\\ and systems of\\  PDEs with\\ constant coefficients};
\draw (43.7, -3) rectangle (49.550000000000004,-6.1);
\draw(49.650000000000006, -3) node[anchor=north west,align=left] {Initial value\\ problems for PDEs\\  and systems\\ of PDEs with\\ constant coefficients};
\draw (49.650000000000006, -3) rectangle (55.50000000000001,-5.6);
\draw(55.6, -3) node[anchor=north west,align=left] {General theory\\ of PDEs and\\  systems of\\ PDEs with\\ constant coefficients};
\draw (55.6, -3) rectangle (61.45,-5.6);
\draw(61.550000000000004, -3) node[anchor=north west,align=left] {Fundamental\\ solutions to PDEs\\ and systems of\\ PDEs with constant\\ coefficients};
\draw (61.550000000000004, -3) rectangle (66.65,-5.6);
\draw(42.7, -6.299999999999999) node[anchor=north west,align=left] {\large History of partial\\ differential equations};
\draw (42.7, -6.299999999999999) rectangle (50.120000000000005,-7.399999999999999);
\draw(2, -8.5) node[anchor=north west,align=left] {\large General first-order partial differential equations and systems of first-order partial differential equations};
\draw (2, -8.5) rectangle (40.6,-14.9);
\draw(3, -9.5) node[anchor=north west,align=left] {Initial value\\  problems\\ for linear\\ first-order PDEs};
\draw (3, -9.5) rectangle (7.6,-11.6);
\draw(7.699999999999999, -9.5) node[anchor=north west,align=left] {Boundary value\\  problems\\ for linear\\ first-order PDEs};
\draw (7.699999999999999, -9.5) rectangle (12.299999999999999,-11.6);
\draw(12.399999999999999, -9.5) node[anchor=north west,align=left] {Initial-boundary\\ value\\  problems\\ for linear\\ first-order PDEs};
\draw (12.399999999999999, -9.5) rectangle (17.0,-12.1);
\draw(17.099999999999998, -9.5) node[anchor=north west,align=left] {Initial value\\ problems for\\  nonlinear\\ first-order PDEs};
\draw (17.099999999999998, -9.5) rectangle (21.699999999999996,-11.6);
\draw(21.799999999999997, -9.5) node[anchor=north west,align=left] {Boundary\\ value problems\\  for\\ nonlinear\\ first-order PDEs};
\draw (21.799999999999997, -9.5) rectangle (26.4,-12.1);
\draw(26.499999999999996, -9.5) node[anchor=north west,align=left] {Initial-boundary\\  value\\ problems for\\ nonlinear\\ first-order PDEs};
\draw (26.499999999999996, -9.5) rectangle (31.099999999999994,-12.1);
\draw(31.199999999999996, -9.5) node[anchor=north west,align=left] {Initial value\\  problems\\ for systems\\ of linear\\ first-order PDEs};
\draw (31.199999999999996, -9.5) rectangle (35.8,-12.1);
\draw(35.9, -9.5) node[anchor=north west,align=left] {Boundary value\\  problems\\ for systems\\ of linear\\ first-order PDEs};
\draw (35.9, -9.5) rectangle (40.5,-12.1);
\draw(3, -12.2) node[anchor=north west,align=left] {Initial-boundary\\ value problems\\ for systems\\  of linear\\ first-order PDEs};
\draw (3, -12.2) rectangle (7.6,-14.799999999999999);
\draw(7.699999999999999, -12.2) node[anchor=north west,align=left] {Initial value\\  problems\\ for systems\\ of nonlinear\\ first-order PDEs};
\draw (7.699999999999999, -12.2) rectangle (12.299999999999999,-14.799999999999999);
\draw(12.399999999999999, -12.2) node[anchor=north west,align=left] {Boundary value\\ problems for\\  systems of\\ nonlinear\\ first-order PDEs};
\draw (12.399999999999999, -12.2) rectangle (17.0,-14.799999999999999);
\draw(17.099999999999998, -12.2) node[anchor=north west,align=left] {Initial-boundary\\ value problems\\ for systems\\ of nonlinear\\ first-order PDEs};
\draw (17.099999999999998, -12.2) rectangle (21.699999999999996,-14.799999999999999);
\draw(21.799999999999997, -12.2) node[anchor=north west,align=left] {Hamilton-Jacobi\\ equations};
\draw (21.799999999999997, -12.2) rectangle (26.15,-13.299999999999999);
\draw(26.249999999999996, -12.2) node[anchor=north west,align=left] {Systems\\ of nonlinear\\ first-order\\ PDEs};
\draw (26.249999999999996, -12.2) rectangle (29.849999999999998,-14.299999999999999);
\draw(29.949999999999996, -12.2) node[anchor=north west,align=left] {Linear\\ first-order\\ PDEs};
\draw (29.949999999999996, -12.2) rectangle (33.3,-13.799999999999999);
\draw(33.39999999999999, -12.2) node[anchor=north west,align=left] {Nonlinear\\ first-order\\ PDEs};
\draw (33.39999999999999, -12.2) rectangle (36.74999999999999,-13.799999999999999);
\draw(36.849999999999994, -12.2) node[anchor=north west,align=left] {Systems\\ of linear\\ first-order\\ PDEs};
\draw (36.849999999999994, -12.2) rectangle (40.199999999999996,-14.299999999999999);
\draw(40.7, -8.5) node[anchor=north west,align=left] {\large Qualitative properties of solutions to partial differential equations};
\draw (40.7, -8.5) rectangle (66.7,-21.0);
\draw(41.7, -9.5) node[anchor=north west,align=left] {Oscillation,\\ zeros of solutions,\\  mean value\\ theorems, etc.\\ in context of PDEs};
\draw (41.7, -9.5) rectangle (47.050000000000004,-12.1);
\draw(47.150000000000006, -9.5) node[anchor=north west,align=left] {Critical points\\ of functionals\\  in context of\\ PDEs (e.g.,\\ energy functionals)};
\draw (47.150000000000006, -9.5) rectangle (52.50000000000001,-12.1);
\draw(52.6, -9.5) node[anchor=north west,align=left] {Homogenization\\ in context of\\ PDEs; PDEs in\\  media with\\ periodic structure};
\draw (52.6, -9.5) rectangle (57.7,-12.1);
\draw(57.800000000000004, -9.5) node[anchor=north west,align=left] {Dependence of\\ solutions to PDEs\\  on initial\\ and/or boundary\\ data and/or on\\ parameters of PDEs};
\draw (57.800000000000004, -9.5) rectangle (62.900000000000006,-12.6);
\draw(63.0, -9.5) node[anchor=north west,align=left] {Bifurcations\\ in context\\ of PDEs};
\draw (63.0, -9.5) rectangle (66.6,-11.1);
\draw(63.0, -11.2) node[anchor=north west,align=left] {Attractors};
\draw (63.0, -11.2) rectangle (66.1,-11.799999999999999);
\draw(41.7, -12.7) node[anchor=north west,align=left] {Almost and\\ pseudo-almost\\  periodic\\ solutions to PDEs};
\draw (41.7, -12.7) rectangle (46.550000000000004,-14.799999999999999);
\draw(46.650000000000006, -12.7) node[anchor=north west,align=left] {Liouville\\ theorems and\\ Phragmén-Lindelöf\\ theorems in\\ context of PDEs};
\draw (46.650000000000006, -12.7) rectangle (51.50000000000001,-15.299999999999999);
\draw(51.6, -12.7) node[anchor=north west,align=left] {Continuation\\ and prolongation\\  of\\ solutions to PDEs};
\draw (51.6, -12.7) rectangle (56.45,-14.799999999999999);
\draw(56.55, -12.7) node[anchor=north west,align=left] {Smoothness\\ and regularity\\  of\\ solutions to PDEs};
\draw (56.55, -12.7) rectangle (61.4,-14.799999999999999);
\draw(61.5, -12.7) node[anchor=north west,align=left] {Symmetries,\\ invariants,\\  etc. in\\ context of PDEs};
\draw (61.5, -12.7) rectangle (65.85,-14.799999999999999);
\draw(41.7, -15.4) node[anchor=north west,align=left] {Perturbations\\  in\\ context of PDEs};
\draw (41.7, -15.4) rectangle (46.050000000000004,-17.0);
\draw(46.150000000000006, -15.4) node[anchor=north west,align=left] {Singular\\ perturbations\\ in context\\ of PDEs};
\draw (46.150000000000006, -15.4) rectangle (50.00000000000001,-17.5);
\draw(50.1, -15.4) node[anchor=north west,align=left] {Asymptotic\\  behavior\\ of solutions\\ to PDEs};
\draw (50.1, -15.4) rectangle (53.7,-17.5);
\draw(53.800000000000004, -15.4) node[anchor=north west,align=left] {Critical\\ exponents\\ in context\\ of PDEs};
\draw (53.800000000000004, -15.4) rectangle (56.900000000000006,-17.5);
\draw(57.0, -15.4) node[anchor=north west,align=left] {Resonance\\ in context\\ of PDEs};
\draw (57.0, -15.4) rectangle (60.1,-17.0);
\draw(60.2, -15.4) node[anchor=north west,align=left] {Stability\\ in context\\ of PDEs};
\draw (60.2, -15.4) rectangle (63.300000000000004,-17.0);
\draw(63.400000000000006, -15.4) node[anchor=north west,align=left] {Pattern\\ formations\\ in context\\ of PDEs};
\draw (63.400000000000006, -15.4) rectangle (66.5,-17.5);
\draw(41.7, -17.6) node[anchor=north west,align=left] {Blow-up\\ in context\\ of PDEs};
\draw (41.7, -17.6) rectangle (44.800000000000004,-19.200000000000003);
\draw(44.900000000000006, -17.6) node[anchor=north west,align=left] {A priori\\ estimates\\ in context\\ of PDEs};
\draw (44.900000000000006, -17.6) rectangle (48.00000000000001,-19.700000000000003);
\draw(48.1, -17.6) node[anchor=north west,align=left] {Maximum\\ principles\\ in context\\ of PDEs};
\draw (48.1, -17.6) rectangle (51.2,-19.700000000000003);
\draw(51.300000000000004, -17.6) node[anchor=north west,align=left] {Comparison\\ principles\\ in context\\ of PDEs};
\draw (51.300000000000004, -17.6) rectangle (54.400000000000006,-19.700000000000003);
\draw(54.5, -17.6) node[anchor=north west,align=left] {Axially\\ symmetric\\ solutions\\ to PDEs};
\draw (54.5, -17.6) rectangle (57.35,-19.700000000000003);
\draw(57.45, -17.6) node[anchor=north west,align=left] {Entire\\ solutions\\ to PDEs};
\draw (57.45, -17.6) rectangle (60.300000000000004,-19.200000000000003);
\draw(60.400000000000006, -17.6) node[anchor=north west,align=left] {Positive\\ solutions\\ to PDEs};
\draw (60.400000000000006, -17.6) rectangle (63.25000000000001,-19.200000000000003);
\draw(63.35, -17.6) node[anchor=north west,align=left] {Periodic\\ solutions\\ to PDEs};
\draw (63.35, -17.6) rectangle (66.2,-19.200000000000003);
\draw(41.7, -19.8) node[anchor=north west,align=left] {Inertial\\ manifolds};
\draw (41.7, -19.8) rectangle (44.550000000000004,-20.900000000000002);
\draw(2, -15.0) node[anchor=north west,align=left] {\large Representations of solutions to partial differential equations};
\draw (2, -15.0) rectangle (25.450000000000003,-19.9);
\draw(3, -16.0) node[anchor=north west,align=left] {Integral\\ representations\\ of solutions\\ to PDEs};
\draw (3, -16.0) rectangle (7.35,-18.1);
\draw(7.449999999999999, -16.0) node[anchor=north west,align=left] {Trigonometric\\ solutions\\ to PDEs};
\draw (7.449999999999999, -16.0) rectangle (11.299999999999999,-17.6);
\draw(11.4, -16.0) node[anchor=north west,align=left] {Self-similar\\ solutions\\ to PDEs};
\draw (11.4, -16.0) rectangle (15.0,-17.6);
\draw(15.1, -16.0) node[anchor=north west,align=left] {Asymptotic\\ expansions\\ of solutions\\ to PDEs};
\draw (15.1, -16.0) rectangle (18.7,-18.1);
\draw(18.8, -16.0) node[anchor=north west,align=left] {Solutions\\ to PDEs in\\ closed form};
\draw (18.8, -16.0) rectangle (22.150000000000002,-17.6);
\draw(22.25, -16.0) node[anchor=north west,align=left] {Polynomial\\ solutions\\ to PDEs};
\draw (22.25, -16.0) rectangle (25.35,-17.6);
\draw(3, -18.2) node[anchor=north west,align=left] {Traveling\\  wave\\ solutions};
\draw (3, -18.2) rectangle (5.85,-19.8);
\draw(5.95, -18.2) node[anchor=north west,align=left] {Soliton\\ solutions};
\draw (5.95, -18.2) rectangle (8.8,-19.3);
\draw(8.9, -18.2) node[anchor=north west,align=left] {Series\\ solutions\\ to PDEs};
\draw (8.9, -18.2) rectangle (11.75,-19.8);
\draw(66.8, -8.5) node[anchor=north west,align=left] {\large Close-to-elliptic equations};
\draw (66.8, -8.5) rectangle (75.77,-11.9);
\draw(67.8, -9.5) node[anchor=north west,align=left] {Quasielliptic\\ equations};
\draw (67.8, -9.5) rectangle (71.64999999999999,-10.6);
\draw(71.75, -9.5) node[anchor=north west,align=left] {Hypoelliptic\\ equations};
\draw (71.75, -9.5) rectangle (75.35,-10.6);
\draw(67.8, -10.7) node[anchor=north west,align=left] {Subelliptic\\ equations};
\draw (67.8, -10.7) rectangle (71.14999999999999,-11.799999999999999);
\draw(2, -21.1) node[anchor=north west,align=left] {\large General topics in partial differential equations};
\draw (2, -21.1) rectangle (20.35,-35.6);
\draw(3, -22.1) node[anchor=north west,align=left] {Inequalities\\ applied to PDEs\\ involving derivatives,\\ differential\\  and integral\\ operators, or integrals};
\draw (3, -22.1) rectangle (9.35,-25.200000000000003);
\draw(9.45, -22.1) node[anchor=north west,align=left] {Cauchy-Kovalevskaya\\ theorems};
\draw (9.45, -22.1) rectangle (14.799999999999999,-23.200000000000003);
\draw(9.45, -23.3) node[anchor=north west,align=left] {Parametrices\\ in context\\ of PDEs};
\draw (9.45, -23.3) rectangle (13.049999999999999,-24.900000000000002);
\draw(14.899999999999999, -22.1) node[anchor=north west,align=left] {Microlocal\\ methods and methods\\  of sheaf\\ theory and\\ homological algebra\\ applied to PDEs};
\draw (14.899999999999999, -22.1) rectangle (20.25,-25.200000000000003);
\draw(3, -25.3) node[anchor=north west,align=left] {Existence problems\\  for PDEs:\\ global existence,\\ local existence,\\ non-existence};
\draw (3, -25.3) rectangle (8.1,-27.900000000000002);
\draw(8.2, -25.3) node[anchor=north west,align=left] {Geometric\\ theory,\\ characteristics,\\ transformations\\ in context of PDEs};
\draw (8.2, -25.3) rectangle (13.299999999999999,-27.900000000000002);
\draw(13.399999999999999, -25.3) node[anchor=north west,align=left] {Uniqueness\\ problems for\\ PDEs: global\\ uniqueness, local\\ uniqueness,\\ non-uniqueness};
\draw (13.399999999999999, -25.3) rectangle (18.25,-28.400000000000002);
\draw(3, -28.5) node[anchor=north west,align=left] {Topological\\ and monotonicity\\  methods\\ applied to PDEs};
\draw (3, -28.5) rectangle (7.6,-30.6);
\draw(7.699999999999999, -28.5) node[anchor=north west,align=left] {Transform\\ methods (e.g.,\\  integral\\ transforms)\\ applied to PDEs};
\draw (7.699999999999999, -28.5) rectangle (12.049999999999999,-31.1);
\draw(12.149999999999999, -28.5) node[anchor=north west,align=left] {Methods of\\ ordinary\\ differential\\ equations\\ applied to PDEs};
\draw (12.149999999999999, -28.5) rectangle (16.5,-31.1);
\draw(16.599999999999998, -28.5) node[anchor=north west,align=left] {Fundamental\\ solutions\\ to PDEs};
\draw (16.599999999999998, -28.5) rectangle (19.95,-30.1);
\draw(3, -31.200000000000003) node[anchor=north west,align=left] {Other\\ special methods\\ applied\\ to PDEs};
\draw (3, -31.200000000000003) rectangle (7.35,-33.300000000000004);
\draw(7.449999999999999, -31.200000000000003) node[anchor=north west,align=left] {Theoretical\\ approximation\\ in context\\ of PDEs};
\draw (7.449999999999999, -31.200000000000003) rectangle (11.299999999999999,-33.300000000000004);
\draw(11.4, -31.200000000000003) node[anchor=north west,align=left] {Variational\\ methods\\ applied\\ to PDEs};
\draw (11.4, -31.200000000000003) rectangle (14.75,-33.300000000000004);
\draw(14.85, -31.200000000000003) node[anchor=north west,align=left] {Analyticity\\ in context\\ of PDEs};
\draw (14.85, -31.200000000000003) rectangle (18.2,-32.800000000000004);
\draw(3, -33.400000000000006) node[anchor=north west,align=left] {Singularity\\ in context\\ of PDEs};
\draw (3, -33.400000000000006) rectangle (6.35,-35.00000000000001);
\draw(6.449999999999999, -33.400000000000006) node[anchor=north west,align=left] {Wave front\\  sets\\ in context\\ of PDEs};
\draw (6.449999999999999, -33.400000000000006) rectangle (9.549999999999999,-35.50000000000001);
\draw(9.649999999999999, -33.400000000000006) node[anchor=north west,align=left] {Classical\\ solutions\\ to PDEs};
\draw (9.649999999999999, -33.400000000000006) rectangle (12.499999999999998,-35.00000000000001);
\draw(20.450000000000003, -21.1) node[anchor=north west,align=left] {\large Generalized solutions to partial differential equations};
\draw (20.450000000000003, -21.1) rectangle (38.10000000000001,-23.8);
\draw(21.450000000000003, -22.1) node[anchor=north west,align=left] {Weak\\ solutions\\ to PDEs};
\draw (21.450000000000003, -22.1) rectangle (24.300000000000004,-23.700000000000003);
\draw(24.400000000000002, -22.1) node[anchor=north west,align=left] {Strong\\ solutions\\ to PDEs};
\draw (24.400000000000002, -22.1) rectangle (27.250000000000004,-23.700000000000003);
\draw(27.35, -22.1) node[anchor=north west,align=left] {Viscosity\\ solutions\\ to PDEs};
\draw (27.35, -22.1) rectangle (30.200000000000003,-23.700000000000003);
\draw(20.450000000000003, -23.900000000000002) node[anchor=north west,align=left] {\large Hyperbolic equations and hyperbolic systems};
\draw (20.450000000000003, -23.900000000000002) rectangle (36.55,-31.5);
\draw(21.450000000000003, -24.900000000000002) node[anchor=north west,align=left] {Initial value\\ problems\\ for first-order\\ hyperbolic equations};
\draw (21.450000000000003, -24.900000000000002) rectangle (27.050000000000004,-27.000000000000004);
\draw(27.150000000000002, -24.900000000000002) node[anchor=north west,align=left] {Initial-boundary\\  value\\ problems for\\ first-order\\ hyperbolic equations};
\draw (27.150000000000002, -24.900000000000002) rectangle (32.75,-27.500000000000004);
\draw(32.85, -24.900000000000002) node[anchor=north west,align=left] {Second-order\\ hyperbolic\\ equations};
\draw (32.85, -24.900000000000002) rectangle (36.45,-26.500000000000004);
\draw(21.450000000000003, -27.6) node[anchor=north west,align=left] {Initial-boundary\\  value\\ problems for\\ second-order\\ hyperbolic equations};
\draw (21.450000000000003, -27.6) rectangle (27.050000000000004,-30.200000000000003);
\draw(27.150000000000002, -27.6) node[anchor=north west,align=left] {Initial value\\  problems\\ for second-order\\ hyperbolic\\ equations};
\draw (27.150000000000002, -27.6) rectangle (31.75,-30.200000000000003);
\draw(31.85, -27.6) node[anchor=north west,align=left] {First-order\\ hyperbolic\\ equations};
\draw (31.85, -27.6) rectangle (35.2,-29.200000000000003);
\draw(21.450000000000003, -30.300000000000004) node[anchor=north west,align=left] {Wave\\ equation};
\draw (21.450000000000003, -30.300000000000004) rectangle (24.050000000000004,-31.400000000000006);
\draw(38.20000000000001, -21.1) node[anchor=north west,align=left] {\large Parabolic equations and parabolic systems};
\draw (38.20000000000001, -21.1) rectangle (53.80000000000001,-49.1);
\draw(39.20000000000001, -22.1) node[anchor=north west,align=left] {Unilateral problems\\ for nonlinear\\  parabolic\\ equations and variational\\ inequalities\\ with nonlinear\\ parabolic operators};
\draw (39.20000000000001, -22.1) rectangle (46.05000000000001,-25.700000000000003);
\draw(46.150000000000006, -22.1) node[anchor=north west,align=left] {Unilateral problems\\ for linear parabolic\\ equations and\\  variational\\ inequalities with linear\\ parabolic operators};
\draw (46.150000000000006, -22.1) rectangle (52.75000000000001,-25.200000000000003);
\draw(39.20000000000001, -25.8) node[anchor=north west,align=left] {Initial value\\ problems\\  for\\ second-order\\ parabolic equations};
\draw (39.20000000000001, -25.8) rectangle (44.55000000000001,-28.400000000000002);
\draw(44.650000000000006, -25.8) node[anchor=north west,align=left] {Initial-boundary\\  value\\ problems for\\ second-order\\ parabolic equations};
\draw (44.650000000000006, -25.8) rectangle (50.00000000000001,-28.400000000000002);
\draw(50.10000000000001, -25.8) node[anchor=north west,align=left] {Second-order\\ parabolic\\ equations};
\draw (50.10000000000001, -25.8) rectangle (53.70000000000001,-27.400000000000002);
\draw(39.20000000000001, -28.5) node[anchor=north west,align=left] {Initial value\\ problems\\  for\\ higher-order\\ parabolic equations};
\draw (39.20000000000001, -28.5) rectangle (44.55000000000001,-31.1);
\draw(44.650000000000006, -28.5) node[anchor=north west,align=left] {Initial-boundary\\  value\\ problems for\\ higher-order\\ parabolic equations};
\draw (44.650000000000006, -28.5) rectangle (50.00000000000001,-31.1);
\draw(50.10000000000001, -28.5) node[anchor=north west,align=left] {Higher-order\\ parabolic\\ equations};
\draw (50.10000000000001, -28.5) rectangle (53.70000000000001,-30.1);
\draw(39.20000000000001, -31.200000000000003) node[anchor=north west,align=left] {Nonlinear initial,\\ boundary and\\ initial-boundary\\ value problems\\  for linear\\ parabolic equations};
\draw (39.20000000000001, -31.200000000000003) rectangle (44.55000000000001,-34.300000000000004);
\draw(44.650000000000006, -31.200000000000003) node[anchor=north west,align=left] {Nonlinear initial,\\  boundary and\\ initial-boundary\\  value problems\\ for nonlinear\\ parabolic equations};
\draw (44.650000000000006, -31.200000000000003) rectangle (50.00000000000001,-34.300000000000004);
\draw(50.10000000000001, -31.200000000000003) node[anchor=north west,align=left] {Second-order\\ parabolic\\ systems};
\draw (50.10000000000001, -31.200000000000003) rectangle (53.70000000000001,-32.800000000000004);
\draw(50.10000000000001, -32.900000000000006) node[anchor=north west,align=left] {Heat\\ equation};
\draw (50.10000000000001, -32.900000000000006) rectangle (52.70000000000001,-34.00000000000001);
\draw(39.20000000000001, -34.400000000000006) node[anchor=north west,align=left] {Unilateral problems\\  for parabolic\\ systems and systems\\ of variational\\ inequalities with\\ parabolic operators};
\draw (39.20000000000001, -34.400000000000006) rectangle (44.55000000000001,-37.50000000000001);
\draw(44.650000000000006, -34.400000000000006) node[anchor=north west,align=left] {Semilinear\\ parabolic equations\\ with Laplacian,\\  bi-Laplacian\\ or poly-Laplacian};
\draw (44.650000000000006, -34.400000000000006) rectangle (50.00000000000001,-37.00000000000001);
\draw(50.10000000000001, -34.400000000000006) node[anchor=north west,align=left] {Higher-order\\ parabolic\\ systems};
\draw (50.10000000000001, -34.400000000000006) rectangle (53.70000000000001,-36.00000000000001);
\draw(39.20000000000001, -37.6) node[anchor=north west,align=left] {Reaction-diffusion\\ equations};
\draw (39.20000000000001, -37.6) rectangle (44.30000000000001,-38.7);
\draw(44.400000000000006, -37.6) node[anchor=north west,align=left] {Initial value\\ problems\\  for\\ second-order\\ parabolic systems};
\draw (44.400000000000006, -37.6) rectangle (49.25000000000001,-40.2);
\draw(39.20000000000001, -38.800000000000004) node[anchor=north west,align=left] {Heat\\ kernel};
\draw (39.20000000000001, -38.800000000000004) rectangle (41.30000000000001,-39.900000000000006);
\draw(49.35000000000001, -37.6) node[anchor=north west,align=left] {Ultraparabolic\\ equations,\\ pseudoparabolic\\ equations, etc.};
\draw (49.35000000000001, -37.6) rectangle (53.70000000000001,-39.7);
\draw(39.20000000000001, -40.3) node[anchor=north west,align=left] {Initial value\\ problems\\  for\\ higher-order\\ parabolic systems};
\draw (39.20000000000001, -40.3) rectangle (44.05000000000001,-42.9);
\draw(44.150000000000006, -40.3) node[anchor=north west,align=left] {Initial-boundary\\  value\\ problems for\\ second-order\\ parabolic systems};
\draw (44.150000000000006, -40.3) rectangle (49.00000000000001,-42.9);
\draw(49.10000000000001, -40.3) node[anchor=north west,align=left] {Quasilinear\\ parabolic\\ equations with\\ \(p\)-Laplacian};
\draw (49.10000000000001, -40.3) rectangle (53.20000000000001,-42.4);
\draw(39.20000000000001, -43.0) node[anchor=north west,align=left] {Initial-boundary\\  value\\ problems for\\ higher-order\\ parabolic systems};
\draw (39.20000000000001, -43.0) rectangle (44.05000000000001,-45.6);
\draw(44.150000000000006, -43.0) node[anchor=north west,align=left] {Quasilinear\\ parabolic\\ equations with\\ mean curvature\\ operator};
\draw (44.150000000000006, -43.0) rectangle (48.25000000000001,-45.6);
\draw(48.35000000000001, -43.0) node[anchor=north west,align=left] {Parabolic\\ Monge-Ampère\\ equations};
\draw (48.35000000000001, -43.0) rectangle (51.95000000000001,-44.6);
\draw(39.20000000000001, -45.7) node[anchor=north west,align=left] {Quasilinear\\ parabolic\\ equations};
\draw (39.20000000000001, -45.7) rectangle (42.55000000000001,-47.300000000000004);
\draw(42.650000000000006, -45.7) node[anchor=north west,align=left] {Semilinear\\ parabolic\\ equations};
\draw (42.650000000000006, -45.7) rectangle (45.75000000000001,-47.300000000000004);
\draw(45.85000000000001, -45.7) node[anchor=north west,align=left] {Degenerate\\ parabolic\\ equations};
\draw (45.85000000000001, -45.7) rectangle (48.95000000000001,-47.300000000000004);
\draw(49.05000000000001, -45.7) node[anchor=north west,align=left] {Nonlinear\\ parabolic\\ equations};
\draw (49.05000000000001, -45.7) rectangle (51.90000000000001,-47.300000000000004);
\draw(39.20000000000001, -47.4) node[anchor=north west,align=left] {Singular\\ parabolic\\ equations};
\draw (39.20000000000001, -47.4) rectangle (42.05000000000001,-49.0);
\draw(42.15000000000001, -47.4) node[anchor=north west,align=left] {Abstract\\ parabolic\\ equations};
\draw (42.15000000000001, -47.4) rectangle (45.000000000000014,-49.0);
\draw(53.900000000000006, -21.1) node[anchor=north west,align=left] {\large Elliptic equations and elliptic systems};
\draw (53.900000000000006, -21.1) rectangle (68.75,-51.8);
\draw(54.900000000000006, -22.1) node[anchor=north west,align=left] {Unilateral problems\\ for linear elliptic\\  equations and\\ variational\\ inequalities with linear\\ elliptic operators};
\draw (54.900000000000006, -22.1) rectangle (61.50000000000001,-25.200000000000003);
\draw(61.60000000000001, -22.1) node[anchor=north west,align=left] {Unilateral problems\\ for nonlinear\\ elliptic equations\\ and variational\\ inequalities\\ with nonlinear\\ elliptic operators};
\draw (61.60000000000001, -22.1) rectangle (66.95,-25.700000000000003);
\draw(54.900000000000006, -25.8) node[anchor=north west,align=left] {Unilateral problems\\  for elliptic\\ systems and systems\\  of variational\\ inequalities with\\ elliptic operators};
\draw (54.900000000000006, -25.8) rectangle (60.25000000000001,-28.900000000000002);
\draw(60.35000000000001, -25.8) node[anchor=north west,align=left] {Boundary\\ value problems\\  for\\ second-order\\ elliptic equations};
\draw (60.35000000000001, -25.8) rectangle (65.45,-28.400000000000002);
\draw(65.55000000000001, -25.8) node[anchor=north west,align=left] {Semilinear\\ elliptic\\ equations};
\draw (65.55000000000001, -25.8) rectangle (68.65,-27.400000000000002);
\draw(54.900000000000006, -29.0) node[anchor=north west,align=left] {Boundary\\ value problems\\  for\\ higher-order\\ elliptic equations};
\draw (54.900000000000006, -29.0) rectangle (60.00000000000001,-31.6);
\draw(60.10000000000001, -29.0) node[anchor=north west,align=left] {Nonlinear\\ boundary value\\ problems for\\  linear\\ elliptic equations};
\draw (60.10000000000001, -29.0) rectangle (65.2,-31.6);
\draw(65.30000000000001, -29.0) node[anchor=north west,align=left] {Schrödinger\\ operator,\\ Schrödinger\\ equation};
\draw (65.30000000000001, -29.0) rectangle (68.65,-31.1);
\draw(54.900000000000006, -31.700000000000003) node[anchor=north west,align=left] {Nonlinear\\ boundary value\\ problems for\\ nonlinear\\ elliptic equations};
\draw (54.900000000000006, -31.700000000000003) rectangle (60.00000000000001,-34.300000000000004);
\draw(60.10000000000001, -31.700000000000003) node[anchor=north west,align=left] {Semilinear\\ elliptic equations\\ with Laplacian,\\  bi-Laplacian\\ or poly-Laplacian};
\draw (60.10000000000001, -31.700000000000003) rectangle (65.2,-34.300000000000004);
\draw(65.30000000000001, -31.700000000000003) node[anchor=north west,align=left] {First-order\\ elliptic\\ systems};
\draw (65.30000000000001, -31.700000000000003) rectangle (68.65,-33.300000000000004);
\draw(54.900000000000006, -34.400000000000006) node[anchor=north west,align=left] {Quasilinear\\  elliptic\\ equations\\ with mean\\ curvature operator};
\draw (54.900000000000006, -34.400000000000006) rectangle (60.00000000000001,-37.00000000000001);
\draw(60.10000000000001, -34.400000000000006) node[anchor=north west,align=left] {Elliptic\\ equations\\ with\\ infinity-Laplacian};
\draw (60.10000000000001, -34.400000000000006) rectangle (65.2,-36.50000000000001);
\draw(65.30000000000001, -34.400000000000006) node[anchor=north west,align=left] {Quasilinear\\ elliptic\\ equations};
\draw (65.30000000000001, -34.400000000000006) rectangle (68.65,-36.00000000000001);
\draw(54.900000000000006, -37.1) node[anchor=north west,align=left] {Laplace operator,\\  Helmholtz\\ equation (reduced\\ wave equation),\\ Poisson equation};
\draw (54.900000000000006, -37.1) rectangle (59.75000000000001,-39.7);
\draw(59.85000000000001, -37.1) node[anchor=north west,align=left] {Boundary\\ value problems\\  for\\ first-order\\ elliptic systems};
\draw (59.85000000000001, -37.1) rectangle (64.45,-39.7);
\draw(64.55000000000001, -37.1) node[anchor=north west,align=left] {Green’s\\ functions for\\ elliptic\\ equations};
\draw (64.55000000000001, -37.1) rectangle (68.4,-39.2);
\draw(54.900000000000006, -39.8) node[anchor=north west,align=left] {Boundary\\ value problems\\  for\\ second-order\\ elliptic systems};
\draw (54.900000000000006, -39.8) rectangle (59.50000000000001,-42.4);
\draw(59.60000000000001, -39.8) node[anchor=north west,align=left] {Boundary\\ value problems\\  for\\ higher-order\\ elliptic systems};
\draw (59.60000000000001, -39.8) rectangle (64.2,-42.4);
\draw(64.30000000000001, -39.8) node[anchor=north west,align=left] {Second-order\\ elliptic\\ equations};
\draw (64.30000000000001, -39.8) rectangle (67.9,-41.4);
\draw(54.900000000000006, -42.5) node[anchor=north west,align=left] {Boundary values\\ of solutions\\  to elliptic\\ equations and\\ elliptic systems};
\draw (54.900000000000006, -42.5) rectangle (59.50000000000001,-45.1);
\draw(59.60000000000001, -42.5) node[anchor=north west,align=left] {Quasilinear\\  elliptic\\ equations\\ with \(p\)-Laplacian};
\draw (59.60000000000001, -42.5) rectangle (64.2,-44.6);
\draw(64.30000000000001, -42.5) node[anchor=north west,align=left] {Variational\\ methods for\\ second-order\\ elliptic\\ equations};
\draw (64.30000000000001, -42.5) rectangle (67.9,-45.1);
\draw(54.900000000000006, -45.2) node[anchor=north west,align=left] {Higher-order\\ elliptic\\ equations};
\draw (54.900000000000006, -45.2) rectangle (58.50000000000001,-46.800000000000004);
\draw(58.60000000000001, -45.2) node[anchor=north west,align=left] {Variational\\ methods for\\ higher-order\\ elliptic\\ equations};
\draw (58.60000000000001, -45.2) rectangle (62.20000000000001,-47.800000000000004);
\draw(62.300000000000004, -45.2) node[anchor=north west,align=left] {Second-order\\ elliptic\\ systems};
\draw (62.300000000000004, -45.2) rectangle (65.9,-46.800000000000004);
\draw(54.900000000000006, -47.9) node[anchor=north west,align=left] {Higher-order\\ elliptic\\ systems};
\draw (54.900000000000006, -47.9) rectangle (58.50000000000001,-49.5);
\draw(58.60000000000001, -47.9) node[anchor=north west,align=left] {Variational\\  methods\\ for elliptic\\ systems};
\draw (58.60000000000001, -47.9) rectangle (62.20000000000001,-50.0);
\draw(62.300000000000004, -47.9) node[anchor=north west,align=left] {Monge-Ampère\\ equations};
\draw (62.300000000000004, -47.9) rectangle (65.9,-49.0);
\draw(54.900000000000006, -50.099999999999994) node[anchor=north west,align=left] {Degenerate\\ elliptic\\ equations};
\draw (54.900000000000006, -50.099999999999994) rectangle (58.00000000000001,-51.699999999999996);
\draw(58.10000000000001, -50.099999999999994) node[anchor=north west,align=left] {Nonlinear\\ elliptic\\ equations};
\draw (58.10000000000001, -50.099999999999994) rectangle (60.95000000000001,-51.699999999999996);
\draw(61.050000000000004, -50.099999999999994) node[anchor=north west,align=left] {Singular\\ elliptic\\ equations};
\draw (61.050000000000004, -50.099999999999994) rectangle (63.900000000000006,-51.699999999999996);
\draw(75.96999999999998, -1) node[anchor=north west,align=left] {\LARGE Ordinary differential equations};
\draw (75.96999999999998, -1) rectangle (136.07,-80.2);
\draw(76.96999999999998, -2) node[anchor=north west,align=left] {\large Functional-differential equations (including equations with delayed, advanced or state-dependent argument)};
\draw (76.96999999999998, -2) rectangle (116.66999999999999,-21.900000000000002);
\draw(77.96999999999998, -3) node[anchor=north west,align=left] {Transformation\\ and reduction\\ of functional-differential\\ equations and systems,\\ normal forms};
\draw (77.96999999999998, -3) rectangle (85.06999999999998,-5.6);
\draw(85.16999999999999, -3) node[anchor=north west,align=left] {Symmetries,\\ invariants\\  of\\ functional-differential\\ equations};
\draw (85.16999999999999, -3) rectangle (91.51999999999998,-5.6);
\draw(91.61999999999998, -3) node[anchor=north west,align=left] {General theory\\  of\\ functional-differential\\ equations};
\draw (91.61999999999998, -3) rectangle (97.96999999999997,-5.1);
\draw(98.07, -3) node[anchor=north west,align=left] {Linear\\ functional-differential\\ equations};
\draw (98.07, -3) rectangle (104.41999999999999,-4.6);
\draw(104.51999999999998, -3) node[anchor=north west,align=left] {Theoretical\\ approximation of\\ solutions to\\ functional-differential\\ equations};
\draw (104.51999999999998, -3) rectangle (110.86999999999998,-5.6);
\draw(77.96999999999998, -5.7) node[anchor=north west,align=left] {Spectral theory\\  of\\ functional-differential\\ operators};
\draw (77.96999999999998, -5.7) rectangle (84.31999999999998,-7.800000000000001);
\draw(84.41999999999999, -5.7) node[anchor=north west,align=left] {Functional-differential\\ inclusions};
\draw (84.41999999999999, -5.7) rectangle (90.76999999999998,-6.800000000000001);
\draw(90.86999999999998, -5.7) node[anchor=north west,align=left] {Boundary value\\  problems\\ for\\ functional-differential\\ equations};
\draw (90.86999999999998, -5.7) rectangle (97.21999999999997,-8.3);
\draw(84.41999999999999, -6.9) node[anchor=north west,align=left] {Functional-differential\\ inequalities};
\draw (84.41999999999999, -6.9) rectangle (90.76999999999998,-8.0);
\draw(97.32, -5.7) node[anchor=north west,align=left] {Oscillation\\  theory of\\ functional-differential\\ equations};
\draw (97.32, -5.7) rectangle (103.66999999999999,-7.800000000000001);
\draw(103.76999999999998, -5.7) node[anchor=north west,align=left] {Growth,\\ boundedness,\\ comparison of\\ solutions to\\ functional-differential\\ equations};
\draw (103.76999999999998, -5.7) rectangle (110.11999999999998,-8.8);
\draw(110.21999999999998, -5.7) node[anchor=north west,align=left] {Periodic\\ solutions to\\ functional-differential\\ equations};
\draw (110.21999999999998, -5.7) rectangle (116.56999999999998,-7.800000000000001);
\draw(77.96999999999998, -8.9) node[anchor=north west,align=left] {Almost and\\ pseudo-almost periodic\\  solutions to\\ functional-differential\\ equations};
\draw (77.96999999999998, -8.9) rectangle (84.31999999999998,-11.5);
\draw(84.41999999999999, -8.9) node[anchor=north west,align=left] {Heteroclinic\\ and homoclinic\\ orbits of\\ functional-differential\\ equations};
\draw (84.41999999999999, -8.9) rectangle (90.76999999999998,-11.5);
\draw(90.86999999999998, -8.9) node[anchor=north west,align=left] {Bifurcation\\  theory of\\ functional-differential\\ equations};
\draw (90.86999999999998, -8.9) rectangle (97.21999999999997,-11.0);
\draw(97.32, -8.9) node[anchor=north west,align=left] {Invariant\\ manifolds of\\ functional-differential\\ equations};
\draw (97.32, -8.9) rectangle (103.66999999999999,-11.0);
\draw(103.76999999999998, -8.9) node[anchor=north west,align=left] {Stability\\ theory of\\ functional-differential\\ equations};
\draw (103.76999999999998, -8.9) rectangle (110.11999999999998,-11.0);
\draw(110.21999999999998, -8.9) node[anchor=north west,align=left] {Stationary\\ solutions\\ of\\ functional-differential\\ equations};
\draw (110.21999999999998, -8.9) rectangle (116.56999999999998,-11.5);
\draw(77.96999999999998, -11.600000000000001) node[anchor=north west,align=left] {Complex (chaotic)\\  behavior of\\ solutions to\\ functional-differential\\ equations};
\draw (77.96999999999998, -11.600000000000001) rectangle (84.31999999999998,-14.200000000000001);
\draw(84.41999999999999, -11.600000000000001) node[anchor=north west,align=left] {Synchronization\\  of\\ functional-differential\\ equations};
\draw (84.41999999999999, -11.600000000000001) rectangle (90.76999999999998,-13.700000000000001);
\draw(90.86999999999998, -11.600000000000001) node[anchor=north west,align=left] {Asymptotic\\ theory of\\ functional-differential\\ equations};
\draw (90.86999999999998, -11.600000000000001) rectangle (97.21999999999997,-13.700000000000001);
\draw(97.32, -11.600000000000001) node[anchor=north west,align=left] {Singular\\ perturbations\\  of\\ functional-differential\\ equations};
\draw (97.32, -11.600000000000001) rectangle (103.66999999999999,-14.200000000000001);
\draw(103.76999999999998, -11.600000000000001) node[anchor=north west,align=left] {Perturbations\\  of\\ functional-differential\\ equations};
\draw (103.76999999999998, -11.600000000000001) rectangle (110.11999999999998,-13.700000000000001);
\draw(110.21999999999998, -11.600000000000001) node[anchor=north west,align=left] {Inverse\\ problems for\\ functional-differential\\ equations};
\draw (110.21999999999998, -11.600000000000001) rectangle (116.56999999999998,-13.700000000000001);
\draw(77.96999999999998, -14.3) node[anchor=north west,align=left] {Functional-differential\\ equations in\\ abstract spaces};
\draw (77.96999999999998, -14.3) rectangle (84.31999999999998,-15.9);
\draw(84.41999999999999, -14.3) node[anchor=north west,align=left] {Lattice\\ functional-differential\\ equations};
\draw (84.41999999999999, -14.3) rectangle (90.76999999999998,-15.9);
\draw(90.86999999999998, -14.3) node[anchor=north west,align=left] {Implicit\\ functional-differential\\ equations};
\draw (90.86999999999998, -14.3) rectangle (97.21999999999997,-15.9);
\draw(97.32, -14.3) node[anchor=north west,align=left] {Averaging\\ for\\ functional-differential\\ equations};
\draw (97.32, -14.3) rectangle (103.66999999999999,-16.400000000000002);
\draw(103.76999999999998, -14.3) node[anchor=north west,align=left] {Hybrid systems\\  of\\ functional-differential\\ equations};
\draw (103.76999999999998, -14.3) rectangle (110.11999999999998,-16.400000000000002);
\draw(110.21999999999998, -14.3) node[anchor=north west,align=left] {Control\\ problems for\\ functional-differential\\ equations};
\draw (110.21999999999998, -14.3) rectangle (116.56999999999998,-16.400000000000002);
\draw(77.96999999999998, -16.5) node[anchor=north west,align=left] {Fuzzy\\ functional-differential\\ equations};
\draw (77.96999999999998, -16.5) rectangle (84.31999999999998,-18.1);
\draw(84.41999999999999, -16.5) node[anchor=north west,align=left] {Functional-differential\\ equations\\ with fractional\\ derivatives};
\draw (84.41999999999999, -16.5) rectangle (90.76999999999998,-18.6);
\draw(90.86999999999998, -16.5) node[anchor=north west,align=left] {Discontinuous\\ functional-differential\\ equations};
\draw (90.86999999999998, -16.5) rectangle (97.21999999999997,-18.1);
\draw(97.32, -16.5) node[anchor=north west,align=left] {Neutral\\ functional-differential\\ equations};
\draw (97.32, -16.5) rectangle (103.66999999999999,-18.1);
\draw(103.76999999999998, -16.5) node[anchor=north west,align=left] {Functional-differential\\ equations\\  in the\\ complex domain};
\draw (103.76999999999998, -16.5) rectangle (110.11999999999998,-18.6);
\draw(110.21999999999998, -16.5) node[anchor=north west,align=left] {Functional-differential\\ equations on\\ time scales or\\ measure chains};
\draw (110.21999999999998, -16.5) rectangle (116.56999999999998,-18.6);
\draw(77.96999999999998, -18.7) node[anchor=north west,align=left] {Functional-differential\\ equations\\ with state-dependent\\ arguments};
\draw (77.96999999999998, -18.7) rectangle (84.31999999999998,-20.8);
\draw(84.41999999999999, -18.7) node[anchor=north west,align=left] {Functional-differential\\ equations\\ with impulses};
\draw (84.41999999999999, -18.7) rectangle (90.76999999999998,-20.3);
\draw(90.86999999999998, -18.7) node[anchor=north west,align=left] {Stochastic\\ functional-differential\\ equations};
\draw (90.86999999999998, -18.7) rectangle (97.21999999999997,-20.3);
\draw(97.32, -18.7) node[anchor=north west,align=left] {Qualitative\\ investigation and\\ simulation of\\ models involving\\ functional-differential\\ equations};
\draw (97.32, -18.7) rectangle (103.66999999999999,-21.8);
\draw(116.76999999999998, -2) node[anchor=north west,align=left] {\large General theory for ordinary differential equations};
\draw (116.76999999999998, -2) rectangle (135.86999999999998,-20.700000000000003);
\draw(117.76999999999998, -3) node[anchor=north west,align=left] {Analytical theory\\  of ordinary\\ differential\\ equations: series,\\ transformations,\\ transforms,\\ operational calculus, etc.};
\draw (117.76999999999998, -3) rectangle (124.86999999999998,-6.6);
\draw(124.96999999999998, -3) node[anchor=north west,align=left] {Initial value problems,\\  existence,\\ uniqueness, continuous\\  dependence and\\ continuation of solutions\\  to ordinary\\ differential equations};
\draw (124.96999999999998, -3) rectangle (131.82,-6.6);
\draw(131.92, -3) node[anchor=north west,align=left] {Discontinuous\\ ordinary\\ differential\\ equations};
\draw (131.92, -3) rectangle (135.76999999999998,-5.1);
\draw(117.76999999999998, -6.7) node[anchor=north west,align=left] {Generalized\\ ordinary differential\\  equations\\ (measure-differential\\  equations,\\ set-valued differential\\ equations, etc.)};
\draw (117.76999999999998, -6.7) rectangle (124.11999999999998,-10.3);
\draw(124.21999999999998, -6.7) node[anchor=north west,align=left] {Fractional\\ ordinary\\ differential equations\\  and\\ fractional differential\\ inclusions};
\draw (124.21999999999998, -6.7) rectangle (130.57,-9.8);
\draw(130.67, -6.7) node[anchor=north west,align=left] {Inverse\\ problems involving\\ ordinary\\ differential\\ equations};
\draw (130.67, -6.7) rectangle (135.76999999999998,-9.3);
\draw(117.76999999999998, -10.4) node[anchor=north west,align=left] {Implicit ordinary\\ differential\\  equations,\\ differential-algebraic\\ equations};
\draw (117.76999999999998, -10.4) rectangle (123.86999999999998,-13.0);
\draw(123.96999999999998, -10.4) node[anchor=north west,align=left] {Theoretical\\ approximation of\\  solutions to\\ ordinary\\ differential equations};
\draw (123.96999999999998, -10.4) rectangle (130.07,-13.0);
\draw(130.17, -10.4) node[anchor=north west,align=left] {Geometric\\ methods in ordinary\\ differential\\ equations};
\draw (130.17, -10.4) rectangle (135.51999999999998,-12.5);
\draw(117.76999999999998, -13.100000000000001) node[anchor=north west,align=left] {Explicit\\ solutions, first\\ integrals of\\ ordinary differential\\ equations};
\draw (117.76999999999998, -13.100000000000001) rectangle (123.61999999999998,-15.700000000000001);
\draw(123.71999999999998, -13.100000000000001) node[anchor=north west,align=left] {Nonlinear\\ ordinary differential\\ equations\\ and systems,\\ general theory};
\draw (123.71999999999998, -13.100000000000001) rectangle (129.57,-15.700000000000001);
\draw(129.67, -13.100000000000001) node[anchor=north west,align=left] {Differential\\ inequalities\\ involving functions\\ of a single\\ real variable};
\draw (129.67, -13.100000000000001) rectangle (135.01999999999998,-15.700000000000001);
\draw(117.76999999999998, -15.8) node[anchor=north west,align=left] {Linear\\ ordinary\\ differential\\ equations and\\ systems, general};
\draw (117.76999999999998, -15.8) rectangle (122.36999999999998,-18.400000000000002);
\draw(122.46999999999998, -15.8) node[anchor=north west,align=left] {Ordinary\\ differential\\ equations of\\ infinite order};
\draw (122.46999999999998, -15.8) rectangle (126.56999999999998,-17.900000000000002);
\draw(126.66999999999999, -15.8) node[anchor=north west,align=left] {Hybrid systems\\ of ordinary\\ differential\\ equations};
\draw (126.66999999999999, -15.8) rectangle (130.76999999999998,-17.900000000000002);
\draw(130.86999999999998, -15.8) node[anchor=north west,align=left] {Ordinary\\ differential\\ equations\\ with impulses};
\draw (130.86999999999998, -15.8) rectangle (134.71999999999997,-17.900000000000002);
\draw(117.76999999999998, -18.5) node[anchor=north west,align=left] {Fuzzy\\ ordinary\\ differential\\ equations};
\draw (117.76999999999998, -18.5) rectangle (121.36999999999998,-20.6);
\draw(121.46999999999998, -18.5) node[anchor=north west,align=left] {Ordinary\\ lattice\\ differential\\ equations};
\draw (121.46999999999998, -18.5) rectangle (125.06999999999998,-20.6);
\draw(125.16999999999999, -18.5) node[anchor=north west,align=left] {Ordinary\\ differential\\ inclusions};
\draw (125.16999999999999, -18.5) rectangle (128.76999999999998,-20.1);
\draw(76.96999999999998, -22.000000000000004) node[anchor=north west,align=left] {\large Ordinary differential equations in the complex domain};
\draw (76.96999999999998, -22.000000000000004) rectangle (97.07,-43.2);
\draw(77.96999999999998, -23.000000000000004) node[anchor=north west,align=left] {Singular perturbation\\  problems for\\ ordinary differential\\ equations in the\\ complex domain (complex\\  WKB, turning\\ points, steepest descent)};
\draw (77.96999999999998, -23.000000000000004) rectangle (84.81999999999998,-26.600000000000005);
\draw(84.91999999999999, -23.000000000000004) node[anchor=north west,align=left] {Algebraic aspects\\ (differential-algebraic,\\ hypertranscendence,\\ group-theoretical)\\  of ordinary\\ differential equations\\ in the complex domain};
\draw (84.91999999999999, -23.000000000000004) rectangle (91.51999999999998,-26.600000000000005);
\draw(91.61999999999998, -23.000000000000004) node[anchor=north west,align=left] {Oscillation,\\ growth of solutions\\ to ordinary\\ differential\\ equations in\\ the complex domain};
\draw (91.61999999999998, -23.000000000000004) rectangle (96.96999999999997,-26.100000000000005);
\draw(77.96999999999998, -26.700000000000003) node[anchor=north west,align=left] {Painlevé and other\\  special ordinary\\ differential equations\\ in the complex\\ domain; classification,\\ hierarchies};
\draw (77.96999999999998, -26.700000000000003) rectangle (84.31999999999998,-29.800000000000004);
\draw(84.41999999999999, -26.700000000000003) node[anchor=north west,align=left] {Singularities,\\ monodromy and local\\ behavior of solutions\\  to ordinary\\ differential equations\\  in the complex\\ domain, normal forms};
\draw (84.41999999999999, -26.700000000000003) rectangle (90.51999999999998,-30.300000000000004);
\draw(90.61999999999998, -26.700000000000003) node[anchor=north west,align=left] {Entire and\\ meromorphic solutions\\ to ordinary\\ differential\\ equations in\\ the complex domain};
\draw (90.61999999999998, -26.700000000000003) rectangle (96.46999999999997,-29.800000000000004);
\draw(77.96999999999998, -30.400000000000006) node[anchor=north west,align=left] {Formal solutions\\  and transform\\ techniques for\\ ordinary differential\\ equations in\\ the complex domain};
\draw (77.96999999999998, -30.400000000000006) rectangle (83.81999999999998,-33.50000000000001);
\draw(83.91999999999999, -30.400000000000006) node[anchor=north west,align=left] {Stokes phenomena\\ and connection\\ problems (linear and\\  nonlinear) for\\ ordinary differential\\ equations in\\ the complex domain};
\draw (83.91999999999999, -30.400000000000006) rectangle (89.76999999999998,-34.00000000000001);
\draw(89.86999999999998, -30.400000000000006) node[anchor=north west,align=left] {Inverse problems\\ (Riemann-Hilbert,\\ inverse differential\\ Galois, etc.) for\\ ordinary differential\\ equations in\\ the complex domain};
\draw (89.86999999999998, -30.400000000000006) rectangle (95.71999999999997,-34.00000000000001);
\draw(77.96999999999998, -34.10000000000001) node[anchor=north west,align=left] {Topological\\ structure of\\ trajectories of\\ ordinary differential\\ equations in\\ the complex domain};
\draw (77.96999999999998, -34.10000000000001) rectangle (83.81999999999998,-37.20000000000001);
\draw(83.91999999999999, -34.10000000000001) node[anchor=north west,align=left] {Ordinary\\ differential\\ equations on complex\\ manifolds};
\draw (83.91999999999999, -34.10000000000001) rectangle (89.51999999999998,-36.20000000000001);
\draw(89.61999999999998, -34.10000000000001) node[anchor=north west,align=left] {Nonlinear ordinary\\ differential\\ equations and\\ systems in the\\ complex domain};
\draw (89.61999999999998, -34.10000000000001) rectangle (94.71999999999997,-36.70000000000001);
\draw(77.96999999999998, -37.300000000000004) node[anchor=north west,align=left] {Spectral theory\\ for ordinary\\  differential\\ operators in\\ the complex domain};
\draw (77.96999999999998, -37.300000000000004) rectangle (83.06999999999998,-39.900000000000006);
\draw(83.16999999999999, -37.300000000000004) node[anchor=north west,align=left] {Asymptotics and\\ summation methods\\ for ordinary\\  differential\\ equations in the\\ complex domain};
\draw (83.16999999999999, -37.300000000000004) rectangle (88.01999999999998,-40.400000000000006);
\draw(88.11999999999998, -37.300000000000004) node[anchor=north west,align=left] {Isomonodromic\\ deformations\\ for ordinary\\ differential\\ equations in the\\ complex domain};
\draw (88.11999999999998, -37.300000000000004) rectangle (92.71999999999997,-40.400000000000006);
\draw(77.96999999999998, -40.5) node[anchor=north west,align=left] {Linear ordinary\\ differential\\ equations and\\ systems in the\\ complex domain};
\draw (77.96999999999998, -40.5) rectangle (82.31999999999998,-43.1);
\draw(97.16999999999999, -22.000000000000004) node[anchor=north west,align=left] {\large Control problems including ordinary differential equations};
\draw (97.16999999999999, -22.000000000000004) rectangle (117.22,-25.700000000000003);
\draw(98.16999999999999, -23.000000000000004) node[anchor=north west,align=left] {Chaos control\\ for problems\\  involving\\ ordinary\\ differential equations};
\draw (98.16999999999999, -23.000000000000004) rectangle (104.26999999999998,-25.600000000000005);
\draw(104.36999999999999, -23.000000000000004) node[anchor=north west,align=left] {Control\\ problems involving\\ ordinary\\ differential\\ equations};
\draw (104.36999999999999, -23.000000000000004) rectangle (109.46999999999998,-25.600000000000005);
\draw(109.57, -23.000000000000004) node[anchor=north west,align=left] {Stabilization\\ of solutions\\ to ordinary\\ differential\\ equations};
\draw (109.57, -23.000000000000004) rectangle (113.41999999999999,-25.600000000000005);
\draw(113.51999999999998, -23.000000000000004) node[anchor=north west,align=left] {Bifurcation\\  control\\ of ordinary\\ differential\\ equations};
\draw (113.51999999999998, -23.000000000000004) rectangle (117.11999999999998,-25.600000000000005);
\draw(97.16999999999999, -25.800000000000004) node[anchor=north west,align=left] {\large Stability theory for ordinary differential equations};
\draw (97.16999999999999, -25.800000000000004) rectangle (116.76999999999998,-38.10000000000001);
\draw(98.16999999999999, -26.800000000000004) node[anchor=north west,align=left] {Structural\\ stability and analogous\\  concepts\\ of solutions to\\ ordinary\\ differential equations};
\draw (98.16999999999999, -26.800000000000004) rectangle (104.51999999999998,-29.900000000000006);
\draw(104.61999999999999, -26.800000000000004) node[anchor=north west,align=left] {Asymptotic\\ properties of\\ solutions to\\ ordinary\\ differential equations};
\draw (104.61999999999999, -26.800000000000004) rectangle (110.71999999999998,-29.400000000000006);
\draw(110.82, -26.800000000000004) node[anchor=north west,align=left] {Synchronization\\ of solutions\\  to\\ ordinary differential\\ equations};
\draw (110.82, -26.800000000000004) rectangle (116.66999999999999,-29.400000000000006);
\draw(98.16999999999999, -30.000000000000004) node[anchor=north west,align=left] {Characteristic\\  and Lyapunov\\ exponents of\\ ordinary\\ differential equations};
\draw (98.16999999999999, -30.000000000000004) rectangle (104.26999999999998,-32.6);
\draw(104.36999999999999, -30.000000000000004) node[anchor=north west,align=left] {Dichotomy,\\ trichotomy of\\ solutions to\\ ordinary\\ differential equations};
\draw (104.36999999999999, -30.000000000000004) rectangle (110.46999999999998,-32.6);
\draw(110.57, -30.000000000000004) node[anchor=north west,align=left] {Global stability\\  of\\ solutions to\\ ordinary\\ differential equations};
\draw (110.57, -30.000000000000004) rectangle (116.66999999999999,-32.6);
\draw(98.16999999999999, -32.7) node[anchor=north west,align=left] {Stability of\\ manifolds of\\ solutions to\\ ordinary differential\\ equations};
\draw (98.16999999999999, -32.7) rectangle (104.01999999999998,-35.300000000000004);
\draw(104.11999999999999, -32.7) node[anchor=north west,align=left] {Perturbations\\ of ordinary\\ differential\\ equations};
\draw (104.11999999999999, -32.7) rectangle (107.96999999999998,-34.800000000000004);
\draw(108.07, -32.7) node[anchor=north west,align=left] {Singular\\ perturbations\\ of ordinary\\ differential\\ equations};
\draw (108.07, -32.7) rectangle (111.91999999999999,-35.300000000000004);
\draw(112.01999999999998, -32.7) node[anchor=north west,align=left] {Stability\\ of solutions\\ to ordinary\\ differential\\ equations};
\draw (112.01999999999998, -32.7) rectangle (115.61999999999998,-35.300000000000004);
\draw(98.16999999999999, -35.400000000000006) node[anchor=north west,align=left] {Attractors\\ of solutions\\ to ordinary\\ differential\\ equations};
\draw (98.16999999999999, -35.400000000000006) rectangle (101.76999999999998,-38.00000000000001);
\draw(97.16999999999999, -38.20000000000001) node[anchor=north west,align=left] {\large Ordinary differential equations and systems with randomness};
\draw (97.16999999999999, -38.20000000000001) rectangle (116.05999999999999,-41.90000000000001);
\draw(98.16999999999999, -39.20000000000001) node[anchor=north west,align=left] {Bifurcation of\\ solutions to\\ ordinary differential\\  equations\\ involving randomness};
\draw (98.16999999999999, -39.20000000000001) rectangle (104.01999999999998,-41.80000000000001);
\draw(104.11999999999999, -39.20000000000001) node[anchor=north west,align=left] {Resonance phenomena\\ for ordinary\\ differential\\  equations\\ involving randomness};
\draw (104.11999999999999, -39.20000000000001) rectangle (109.71999999999998,-41.80000000000001);
\draw(109.82, -39.20000000000001) node[anchor=north west,align=left] {Ordinary\\ differential\\ equations\\ and systems\\ with randomness};
\draw (109.82, -39.20000000000001) rectangle (114.16999999999999,-41.80000000000001);
\draw(117.32, -22.000000000000004) node[anchor=north west,align=left] {\large Dynamic equations on time scales or measure chains};
\draw (117.32, -22.000000000000004) rectangle (133.42,-25.200000000000003);
\draw(118.32, -23.000000000000004) node[anchor=north west,align=left] {Dynamic\\ equations on time\\  scales or\\ measure chains};
\draw (118.32, -23.000000000000004) rectangle (123.16999999999999,-25.100000000000005);
\draw(76.96999999999998, -43.30000000000001) node[anchor=north west,align=left] {\large Boundary value problems for ordinary differential equations};
\draw (76.96999999999998, -43.30000000000001) rectangle (96.57,-61.500000000000014);
\draw(77.96999999999998, -44.30000000000001) node[anchor=north west,align=left] {Linear boundary\\ value problems\\ for ordinary\\ differential equations\\ with nonlinear\\  dependence on\\ the spectral parameter};
\draw (77.96999999999998, -44.30000000000001) rectangle (84.06999999999998,-47.90000000000001);
\draw(84.16999999999999, -44.30000000000001) node[anchor=north west,align=left] {Parameter dependent\\  boundary\\ value problems\\ for ordinary\\ differential equations};
\draw (84.16999999999999, -44.30000000000001) rectangle (90.26999999999998,-46.90000000000001);
\draw(90.36999999999998, -44.30000000000001) node[anchor=north west,align=left] {Boundary\\ eigenvalue\\ problems for\\ ordinary\\ differential equations};
\draw (90.36999999999998, -44.30000000000001) rectangle (96.46999999999997,-46.90000000000001);
\draw(77.96999999999998, -48.000000000000014) node[anchor=north west,align=left] {Nonlocal and\\ multipoint\\ boundary value\\ problems for\\ ordinary\\ differential equations};
\draw (77.96999999999998, -48.000000000000014) rectangle (84.06999999999998,-51.100000000000016);
\draw(84.16999999999999, -48.000000000000014) node[anchor=north west,align=left] {Nonlinear\\ boundary value\\ problems for\\ ordinary\\ differential equations};
\draw (84.16999999999999, -48.000000000000014) rectangle (90.26999999999998,-50.600000000000016);
\draw(90.36999999999998, -48.000000000000014) node[anchor=north west,align=left] {Singular nonlinear\\  boundary\\ value problems for\\  ordinary\\ differential equations};
\draw (90.36999999999998, -48.000000000000014) rectangle (96.46999999999997,-50.600000000000016);
\draw(77.96999999999998, -51.20000000000001) node[anchor=north west,align=left] {Positive solutions\\ to nonlinear\\ boundary value\\  problems for\\ ordinary\\ differential equations};
\draw (77.96999999999998, -51.20000000000001) rectangle (84.06999999999998,-54.30000000000001);
\draw(84.16999999999999, -51.20000000000001) node[anchor=north west,align=left] {Boundary value\\ problems with\\  impulses for\\ ordinary\\ differential equations};
\draw (84.16999999999999, -51.20000000000001) rectangle (90.26999999999998,-53.80000000000001);
\draw(90.36999999999998, -51.20000000000001) node[anchor=north west,align=left] {Boundary value\\ problems on\\  infinite\\ intervals for\\ ordinary\\ differential equations};
\draw (90.36999999999998, -51.20000000000001) rectangle (96.46999999999997,-54.30000000000001);
\draw(77.96999999999998, -54.40000000000001) node[anchor=north west,align=left] {Boundary value\\ problems on\\  graphs and\\ networks for\\ ordinary\\ differential equations};
\draw (77.96999999999998, -54.40000000000001) rectangle (84.06999999999998,-57.500000000000014);
\draw(84.16999999999999, -54.40000000000001) node[anchor=north west,align=left] {Applications\\ of boundary\\ value problems\\  involving\\ ordinary\\ differential equations};
\draw (84.16999999999999, -54.40000000000001) rectangle (90.26999999999998,-57.500000000000014);
\draw(90.36999999999998, -54.40000000000001) node[anchor=north west,align=left] {Linear boundary\\  value\\ problems for\\ ordinary differential\\ equations};
\draw (90.36999999999998, -54.40000000000001) rectangle (96.21999999999997,-57.000000000000014);
\draw(77.96999999999998, -57.60000000000001) node[anchor=north west,align=left] {Weyl theory and\\ its generalizations\\  for\\ ordinary differential\\ equations};
\draw (77.96999999999998, -57.60000000000001) rectangle (83.81999999999998,-60.20000000000001);
\draw(83.91999999999999, -57.60000000000001) node[anchor=north west,align=left] {Green’s functions\\  for\\ ordinary differential\\ equations};
\draw (83.91999999999999, -57.60000000000001) rectangle (89.76999999999998,-59.70000000000001);
\draw(89.86999999999998, -57.60000000000001) node[anchor=north west,align=left] {Special ordinary\\ differential\\ equations\\  (Mathieu,\\ Hill, Bessel, etc.)};
\draw (89.86999999999998, -57.60000000000001) rectangle (95.21999999999997,-60.20000000000001);
\draw(77.96999999999998, -60.30000000000001) node[anchor=north west,align=left] {Sturm-Liouville\\ theory};
\draw (77.96999999999998, -60.30000000000001) rectangle (82.31999999999998,-61.40000000000001);
\draw(96.66999999999999, -43.30000000000001) node[anchor=north west,align=left] {\large Qualitative theory for ordinary differential equations};
\draw (96.66999999999999, -43.30000000000001) rectangle (116.26999999999998,-66.20000000000002);
\draw(97.66999999999999, -44.30000000000001) node[anchor=north west,align=left] {Ordinary differential\\  equations and\\ connections with real\\ algebraic geometry\\ (fewnomials, desingularization,\\  zeros of\\ abelian integrals, etc.)};
\draw (97.66999999999999, -44.30000000000001) rectangle (106.01999999999998,-47.90000000000001);
\draw(106.11999999999999, -44.30000000000001) node[anchor=north west,align=left] {Theory of limit cycles\\  of polynomial and\\ analytic vector fields\\ (existence, uniqueness,\\ bounds, Hilbert’s\\  16th problem and\\ ramifications) for ordinary\\ differential equations};
\draw (106.11999999999999, -44.30000000000001) rectangle (113.46999999999998,-48.40000000000001);
\draw(97.66999999999999, -48.500000000000014) node[anchor=north west,align=left] {Topological structure\\  of integral\\ curves, singular\\ points, limit\\ cycles of ordinary\\ differential equations};
\draw (97.66999999999999, -48.500000000000014) rectangle (103.76999999999998,-51.600000000000016);
\draw(103.86999999999999, -48.500000000000014) node[anchor=north west,align=left] {Oscillation theory,\\  zeros,\\ disconjugacy and\\ comparison theory for\\  ordinary\\ differential equations};
\draw (103.86999999999999, -48.500000000000014) rectangle (109.96999999999998,-51.600000000000016);
\draw(110.07, -48.500000000000014) node[anchor=north west,align=left] {Nonlinear\\ oscillations and\\  coupled\\ oscillators for\\ ordinary\\ differential equations};
\draw (110.07, -48.500000000000014) rectangle (116.16999999999999,-51.600000000000016);
\draw(97.66999999999999, -51.70000000000001) node[anchor=north west,align=left] {Almost and\\ pseudo-almost periodic\\  solutions\\ to ordinary\\ differential equations};
\draw (97.66999999999999, -51.70000000000001) rectangle (103.76999999999998,-54.30000000000001);
\draw(103.86999999999999, -51.70000000000001) node[anchor=north west,align=left] {Homoclinic and\\  heteroclinic\\ solutions to\\ ordinary\\ differential equations};
\draw (103.86999999999999, -51.70000000000001) rectangle (109.96999999999998,-54.30000000000001);
\draw(110.07, -51.70000000000001) node[anchor=north west,align=left] {Equivalence and\\  asymptotic\\ equivalence of\\ ordinary\\ differential equations};
\draw (110.07, -51.70000000000001) rectangle (116.16999999999999,-54.30000000000001);
\draw(97.66999999999999, -54.400000000000006) node[anchor=north west,align=left] {Growth and\\ boundedness of\\ solutions to\\ ordinary differential\\ equations};
\draw (97.66999999999999, -54.400000000000006) rectangle (103.51999999999998,-57.00000000000001);
\draw(103.61999999999999, -54.400000000000006) node[anchor=north west,align=left] {Transformation\\ and reduction\\  of ordinary\\ differential\\ equations and\\ systems, normal forms};
\draw (103.61999999999999, -54.400000000000006) rectangle (109.46999999999998,-57.50000000000001);
\draw(109.57, -54.400000000000006) node[anchor=north west,align=left] {Periodic\\ solutions to\\ ordinary differential\\ equations};
\draw (109.57, -54.400000000000006) rectangle (115.41999999999999,-56.50000000000001);
\draw(97.66999999999999, -57.60000000000001) node[anchor=north west,align=left] {Complex behavior\\ and chaotic\\  systems of\\ ordinary differential\\ equations};
\draw (97.66999999999999, -57.60000000000001) rectangle (103.51999999999998,-60.20000000000001);
\draw(103.61999999999999, -57.60000000000001) node[anchor=north west,align=left] {Averaging\\ method for ordinary\\ differential\\ equations};
\draw (103.61999999999999, -57.60000000000001) rectangle (108.96999999999998,-59.70000000000001);
\draw(109.07, -57.60000000000001) node[anchor=north west,align=left] {Monotone\\ systems involving\\ ordinary\\ differential\\ equations};
\draw (109.07, -57.60000000000001) rectangle (113.91999999999999,-60.20000000000001);
\draw(97.66999999999999, -60.30000000000001) node[anchor=north west,align=left] {Qualitative\\ investigation\\ and simulation\\ of ordinary\\ differential\\ equation models};
\draw (97.66999999999999, -60.30000000000001) rectangle (102.01999999999998,-63.40000000000001);
\draw(102.11999999999999, -60.30000000000001) node[anchor=north west,align=left] {Multifrequency\\ systems\\ of ordinary\\ differential\\ equations};
\draw (102.11999999999999, -60.30000000000001) rectangle (106.21999999999998,-62.90000000000001);
\draw(106.32, -60.30000000000001) node[anchor=north west,align=left] {Invariant\\ manifolds for\\  ordinary\\ differential\\ equations};
\draw (106.32, -60.30000000000001) rectangle (110.16999999999999,-62.90000000000001);
\draw(110.26999999999998, -60.30000000000001) node[anchor=north west,align=left] {Symmetries,\\ invariants\\ of ordinary\\ differential\\ equations};
\draw (110.26999999999998, -60.30000000000001) rectangle (113.86999999999998,-62.90000000000001);
\draw(97.66999999999999, -63.50000000000001) node[anchor=north west,align=left] {Bifurcation\\  theory\\ for ordinary\\ differential\\ equations};
\draw (97.66999999999999, -63.50000000000001) rectangle (101.26999999999998,-66.10000000000001);
\draw(101.36999999999999, -63.50000000000001) node[anchor=north west,align=left] {Relaxation\\ oscillations\\ for ordinary\\ differential\\ equations};
\draw (101.36999999999999, -63.50000000000001) rectangle (104.96999999999998,-66.10000000000001);
\draw(105.07, -63.50000000000001) node[anchor=north west,align=left] {Ordinary\\ differential\\ equations\\ and systems\\ on manifolds};
\draw (105.07, -63.50000000000001) rectangle (108.66999999999999,-66.10000000000001);
\draw(108.76999999999998, -63.50000000000001) node[anchor=north west,align=left] {Hysteresis\\ for ordinary\\ differential\\ equations};
\draw (108.76999999999998, -63.50000000000001) rectangle (112.36999999999998,-65.60000000000001);
\draw(76.96999999999998, -61.600000000000016) node[anchor=north west,align=left] {\large Differential equations in abstract spaces};
\draw (76.96999999999998, -61.600000000000016) rectangle (90.27999999999999,-64.80000000000001);
\draw(77.96999999999998, -62.600000000000016) node[anchor=north west,align=left] {Linear\\ differential\\ equations in\\ abstract spaces};
\draw (77.96999999999998, -62.600000000000016) rectangle (82.31999999999998,-64.70000000000002);
\draw(82.41999999999999, -62.600000000000016) node[anchor=north west,align=left] {Nonlinear\\ differential\\ equations in\\ abstract spaces};
\draw (82.41999999999999, -62.600000000000016) rectangle (86.76999999999998,-64.70000000000002);
\draw(86.86999999999998, -62.600000000000016) node[anchor=north west,align=left] {Evolution\\ inclusions};
\draw (86.86999999999998, -62.600000000000016) rectangle (89.96999999999997,-63.70000000000002);
\draw(116.36999999999999, -43.30000000000001) node[anchor=north west,align=left] {\large Asymptotic theory for ordinary differential equations};
\draw (116.36999999999999, -43.30000000000001) rectangle (135.97,-52.90000000000001);
\draw(117.36999999999999, -44.30000000000001) node[anchor=north west,align=left] {Asymptotic\\ expansions of\\ solutions to\\ ordinary\\ differential equations};
\draw (117.36999999999999, -44.30000000000001) rectangle (123.46999999999998,-46.90000000000001);
\draw(123.57, -44.30000000000001) node[anchor=north west,align=left] {Perturbations,\\  asymptotics\\ of solutions to\\  ordinary\\ differential equations};
\draw (123.57, -44.30000000000001) rectangle (129.67,-46.90000000000001);
\draw(129.76999999999998, -44.30000000000001) node[anchor=north west,align=left] {Singular\\ perturbations, general\\ theory for\\  ordinary\\ differential equations};
\draw (129.76999999999998, -44.30000000000001) rectangle (135.86999999999998,-46.90000000000001);
\draw(117.36999999999999, -47.000000000000014) node[anchor=north west,align=left] {Singular\\ perturbations, turning\\ point theory,\\ WKB methods for\\  ordinary\\ differential equations};
\draw (117.36999999999999, -47.000000000000014) rectangle (123.46999999999998,-50.100000000000016);
\draw(123.57, -47.000000000000014) node[anchor=north west,align=left] {Canard solutions\\  to\\ ordinary differential\\ equations};
\draw (123.57, -47.000000000000014) rectangle (129.42,-49.100000000000016);
\draw(129.51999999999998, -47.000000000000014) node[anchor=north west,align=left] {Methods of\\ nonstandard\\ analysis for\\ ordinary differential\\ equations};
\draw (129.51999999999998, -47.000000000000014) rectangle (135.36999999999998,-49.600000000000016);
\draw(117.36999999999999, -50.20000000000001) node[anchor=north west,align=left] {Multiple\\ scale methods\\ for ordinary\\ differential\\ equations};
\draw (117.36999999999999, -50.20000000000001) rectangle (121.21999999999998,-52.80000000000001);
\draw(116.36999999999999, -53.000000000000014) node[anchor=north west,align=left] {\large History of\\ ordinary differential\\ equations};
\draw (116.36999999999999, -53.000000000000014) rectangle (123.47999999999999,-54.600000000000016);
\draw(76.96999999999998, -66.30000000000001) node[anchor=north west,align=left] {\large Ordinary differential operators};
\draw (76.96999999999998, -66.30000000000001) rectangle (90.36999999999998,-80.10000000000001);
\draw(77.96999999999998, -67.30000000000001) node[anchor=north west,align=left] {Eigenfunctions,\\ eigenfunction\\ expansions, completeness\\ of eigenfunctions\\  of ordinary\\ differential operators};
\draw (77.96999999999998, -67.30000000000001) rectangle (84.56999999999998,-70.4);
\draw(84.66999999999999, -67.30000000000001) node[anchor=north west,align=left] {General\\ spectral theory\\ of ordinary\\ differential\\ operators};
\draw (84.66999999999999, -67.30000000000001) rectangle (89.01999999999998,-69.9);
\draw(77.96999999999998, -70.50000000000001) node[anchor=north west,align=left] {Asymptotic distribution\\ of eigenvalues,\\  asymptotic\\ theory of eigenfunctions\\  for ordinary\\ differential operators};
\draw (77.96999999999998, -70.50000000000001) rectangle (84.56999999999998,-73.60000000000001);
\draw(84.66999999999999, -70.50000000000001) node[anchor=north west,align=left] {Nonlinear\\ ordinary\\ differential\\ operators};
\draw (84.66999999999999, -70.50000000000001) rectangle (88.26999999999998,-72.60000000000001);
\draw(77.96999999999998, -73.70000000000002) node[anchor=north west,align=left] {Numerical approximation\\ of eigenvalues\\  and of other\\ parts of the spectrum\\  of ordinary\\ differential operators};
\draw (77.96999999999998, -73.70000000000002) rectangle (84.31999999999998,-76.80000000000001);
\draw(84.41999999999999, -73.70000000000002) node[anchor=north west,align=left] {Particular\\ ordinary differential\\ operators\\  (Dirac,\\ one-dimensional\\ Schrödinger, etc.)};
\draw (84.41999999999999, -73.70000000000002) rectangle (90.26999999999998,-76.80000000000001);
\draw(77.96999999999998, -76.9) node[anchor=north west,align=left] {Eigenvalues,\\ estimation of\\ eigenvalues, upper and\\  lower bounds\\ of ordinary\\ differential operators};
\draw (77.96999999999998, -76.9) rectangle (84.06999999999998,-80.0);
\draw(84.16999999999999, -76.9) node[anchor=north west,align=left] {Scattering theory,\\  inverse\\ scattering involving\\  ordinary\\ differential operators};
\draw (84.16999999999999, -76.9) rectangle (90.26999999999998,-79.5);
\draw(1, -52.0) node[anchor=north west,align=left] {\LARGE Field theory and polynomials};
\draw (1, -52.0) rectangle (28.64,-77.9);
\draw(2, -53.0) node[anchor=north west,align=left] {\large Connections between field theory and logic};
\draw (2, -53.0) rectangle (17.299999999999997,-56.2);
\draw(3, -54.0) node[anchor=north west,align=left] {Ultraproducts\\  and\\ field theory};
\draw (3, -54.0) rectangle (6.85,-55.6);
\draw(6.949999999999999, -54.0) node[anchor=north west,align=left] {Decidability\\  and\\ field theory};
\draw (6.949999999999999, -54.0) rectangle (10.549999999999999,-55.6);
\draw(10.65, -54.0) node[anchor=north west,align=left] {Nonstandard\\ arithmetic\\  and\\ field theory};
\draw (10.65, -54.0) rectangle (14.25,-56.1);
\draw(14.35, -54.0) node[anchor=north west,align=left] {Model\\ theory\\ of fields};
\draw (14.35, -54.0) rectangle (17.2,-55.6);
\draw(17.4, -53.0) node[anchor=north west,align=left] {\large Homological methods (field theory)};
\draw (17.4, -53.0) rectangle (28.54,-55.7);
\draw(18.4, -54.0) node[anchor=north west,align=left] {Cohomological\\ dimension\\ of fields};
\draw (18.4, -54.0) rectangle (22.25,-55.6);
\draw(22.349999999999998, -54.0) node[anchor=north west,align=left] {Galois\\ cohomology};
\draw (22.349999999999998, -54.0) rectangle (25.45,-55.1);
\draw(2, -56.3) node[anchor=north west,align=left] {\large General field theory};
\draw (2, -56.3) rectangle (14.149999999999999,-63.4);
\draw(3, -57.3) node[anchor=north west,align=left] {Polynomials\\ in general\\ fields (irreducibility,\\ etc.)};
\draw (3, -57.3) rectangle (9.35,-59.4);
\draw(9.45, -57.3) node[anchor=north west,align=left] {Finite\\ fields\\ (field-theoretic\\ aspects)};
\draw (9.45, -57.3) rectangle (14.049999999999999,-59.4);
\draw(3, -59.5) node[anchor=north west,align=left] {Hilbertian\\ fields; Hilbert’s\\ irreducibility theorem};
\draw (3, -59.5) rectangle (9.1,-61.1);
\draw(9.2, -59.5) node[anchor=north west,align=left] {Skew fields,\\ division\\ rings};
\draw (9.2, -59.5) rectangle (12.799999999999999,-61.1);
\draw(3, -61.199999999999996) node[anchor=north west,align=left] {Special\\ polynomials\\ in general\\ fields};
\draw (3, -61.199999999999996) rectangle (6.35,-63.3);
\draw(6.449999999999999, -61.199999999999996) node[anchor=north west,align=left] {Equations\\ in general\\ fields};
\draw (6.449999999999999, -61.199999999999996) rectangle (9.549999999999999,-62.8);
\draw(9.649999999999999, -61.199999999999996) node[anchor=north west,align=left] {Field\\ arithmetic};
\draw (9.649999999999999, -61.199999999999996) rectangle (12.749999999999998,-62.3);
\draw(14.249999999999998, -56.3) node[anchor=north west,align=left] {\large Differential and difference algebra};
\draw (14.249999999999998, -56.3) rectangle (26.349999999999998,-60.199999999999996);
\draw(15.249999999999998, -57.3) node[anchor=north west,align=left] {Differential\\ algebra};
\draw (15.249999999999998, -57.3) rectangle (18.849999999999998,-58.4);
\draw(18.949999999999996, -57.3) node[anchor=north west,align=left] {Abstract\\ differential\\ equations};
\draw (18.949999999999996, -57.3) rectangle (22.549999999999997,-58.9);
\draw(22.65, -57.3) node[anchor=north west,align=left] {\(p\)-adic\\ differential\\ equations};
\draw (22.65, -57.3) rectangle (26.25,-58.9);
\draw(15.249999999999998, -59.0) node[anchor=north west,align=left] {Difference\\ algebra};
\draw (15.249999999999998, -59.0) rectangle (18.349999999999998,-60.1);
\draw(14.249999999999998, -60.3) node[anchor=north west,align=left] {\large Computational methods\\  for problems\\ pertaining to field theory};
\draw (14.249999999999998, -60.3) rectangle (22.909999999999997,-61.9);
\draw(14.249999999999998, -62.0) node[anchor=north west,align=left] {\large History of\\ field theory};
\draw (14.249999999999998, -62.0) rectangle (18.569999999999997,-63.1);
\draw(2, -63.5) node[anchor=north west,align=left] {\large Real and complex fields};
\draw (2, -63.5) rectangle (12.399999999999999,-70.4);
\draw(3, -64.5) node[anchor=north west,align=left] {Polynomials in\\ real and complex\\ fields: location\\  of zeros\\ (algebraic theorems)};
\draw (3, -64.5) rectangle (8.6,-67.1);
\draw(3, -67.2) node[anchor=north west,align=left] {Fields related\\  with sums of\\ squares (formally\\ real fields,\\ Pythagorean\\ fields, etc.)};
\draw (3, -67.2) rectangle (7.85,-70.3);
\draw(7.949999999999999, -67.2) node[anchor=north west,align=left] {Polynomials\\ in real and\\ complex fields:\\ factorization};
\draw (7.949999999999999, -67.2) rectangle (12.299999999999999,-69.3);
\draw(12.499999999999998, -63.5) node[anchor=north west,align=left] {\large Generalizations of fields};
\draw (12.499999999999998, -63.5) rectangle (20.849999999999998,-65.2);
\draw(13.499999999999998, -64.5) node[anchor=north west,align=left] {Near-fields};
\draw (13.499999999999998, -64.5) rectangle (16.849999999999998,-65.1);
\draw(16.949999999999996, -64.5) node[anchor=north west,align=left] {Semifields};
\draw (16.949999999999996, -64.5) rectangle (20.049999999999997,-65.1);
\draw(2, -70.5) node[anchor=north west,align=left] {\large Field extensions};
\draw (2, -70.5) rectangle (10.149999999999999,-76.6);
\draw(3, -71.5) node[anchor=north west,align=left] {Transcendental\\ field\\ extensions};
\draw (3, -71.5) rectangle (7.1,-73.1);
\draw(7.199999999999999, -71.5) node[anchor=north west,align=left] {Inverse\\ Galois\\ theory};
\draw (7.199999999999999, -71.5) rectangle (9.549999999999999,-73.1);
\draw(3, -73.2) node[anchor=north west,align=left] {Separable\\ extensions,\\ Galois theory};
\draw (3, -73.2) rectangle (6.85,-74.8);
\draw(6.949999999999999, -73.2) node[anchor=north west,align=left] {Algebraic\\  field\\ extensions};
\draw (6.949999999999999, -73.2) rectangle (10.049999999999999,-74.8);
\draw(3, -74.9) node[anchor=north west,align=left] {Inseparable\\  field\\ extensions};
\draw (3, -74.9) rectangle (6.35,-76.5);
\draw(10.249999999999998, -70.5) node[anchor=north west,align=left] {\large Topological fields};
\draw (10.249999999999998, -70.5) rectangle (18.4,-77.8);
\draw(11.249999999999998, -71.5) node[anchor=north west,align=left] {Non-Archimedean\\ valued fields};
\draw (11.249999999999998, -71.5) rectangle (15.599999999999998,-72.6);
\draw(15.699999999999998, -71.5) node[anchor=north west,align=left] {Formally\\ \(p\)-adic\\ fields};
\draw (15.699999999999998, -71.5) rectangle (18.299999999999997,-73.1);
\draw(11.249999999999998, -73.2) node[anchor=north west,align=left] {Krasner-Tate\\ algebras};
\draw (11.249999999999998, -73.2) rectangle (14.849999999999998,-74.3);
\draw(14.949999999999998, -73.2) node[anchor=north west,align=left] {Topological\\ semifields};
\draw (14.949999999999998, -73.2) rectangle (18.299999999999997,-74.3);
\draw(11.249999999999998, -74.4) node[anchor=north west,align=left] {General\\ valuation\\ theory\\ for fields};
\draw (11.249999999999998, -74.4) rectangle (14.349999999999998,-76.5);
\draw(14.449999999999998, -74.4) node[anchor=north west,align=left] {Ordered\\ fields};
\draw (14.449999999999998, -74.4) rectangle (16.799999999999997,-75.5);
\draw(11.249999999999998, -76.6) node[anchor=north west,align=left] {Normed\\ fields};
\draw (11.249999999999998, -76.6) rectangle (13.349999999999998,-77.69999999999999);
\draw(13.45, -76.6) node[anchor=north west,align=left] {Valued\\ fields};
\draw (13.45, -76.6) rectangle (15.549999999999999,-77.69999999999999);
\draw(1, -80.3) node[anchor=north west,align=left] {\LARGE Several complex variables and analytic spaces};
\draw (1, -80.3) rectangle (63.85,-144.8);
\draw(2, -81.3) node[anchor=north west,align=left] {\large Non-Archimedean analysis (should also be assigned at least one other classification number from Section 32-XX describing the type of problem)};
\draw (2, -81.3) rectangle (46.31,-86.0);
\draw(3, -82.3) node[anchor=north west,align=left] {Non-Archimedean\\ analysis (should also be\\  assigned at least\\ one other classification\\ number from Section\\ 32-XX describing\\ the type of problem)};
\draw (3, -82.3) rectangle (9.6,-85.89999999999999);
\draw(46.410000000000004, -81.3) node[anchor=north west,align=left] {\large Complex spaces with a group of automorphisms};
\draw (46.410000000000004, -81.3) rectangle (63.010000000000005,-88.2);
\draw(47.410000000000004, -82.3) node[anchor=north west,align=left] {Hermitian symmetric\\ spaces, bounded\\  symmetric\\ domains, Jordan\\ algebras (complex-analytic\\ aspects)};
\draw (47.410000000000004, -82.3) rectangle (54.510000000000005,-85.39999999999999);
\draw(54.61, -82.3) node[anchor=north west,align=left] {Complex Lie\\ groups, group\\ actions on\\ complex spaces};
\draw (54.61, -82.3) rectangle (58.71,-84.39999999999999);
\draw(58.81, -82.3) node[anchor=north west,align=left] {Automorphism\\  groups\\ of other\\ complex spaces};
\draw (58.81, -82.3) rectangle (62.910000000000004,-84.39999999999999);
\draw(47.410000000000004, -85.5) node[anchor=north west,align=left] {Complex\\ vector fields,\\ holomorphic\\ foliations,\\ \(\mathbb{C}\)-actions};
\draw (47.410000000000004, -85.5) rectangle (51.510000000000005,-88.1);
\draw(51.61, -85.5) node[anchor=north west,align=left] {Automorphism\\ groups of\\ \(\mathbb{C}^n\) and affine\\ manifolds};
\draw (51.61, -85.5) rectangle (55.21,-87.6);
\draw(55.31, -85.5) node[anchor=north west,align=left] {Homogeneous\\ complex\\ manifolds};
\draw (55.31, -85.5) rectangle (58.660000000000004,-87.1);
\draw(58.760000000000005, -85.5) node[anchor=north west,align=left] {Almost\\ homogeneous\\ manifolds\\ and spaces};
\draw (58.760000000000005, -85.5) rectangle (62.11000000000001,-87.6);
\draw(2, -86.1) node[anchor=north west,align=left] {\large Computational methods\\ for problems pertaining\\ to several complex\\ variables and analytic spaces};
\draw (2, -86.1) rectangle (11.59,-88.19999999999999);
\draw(2, -88.3) node[anchor=north west,align=left] {\large Holomorphic functions of several complex variables};
\draw (2, -88.3) rectangle (21.1,-113.4);
\draw(3, -89.3) node[anchor=north west,align=left] {Other generalizations\\ of function theory\\  of one complex\\ variable (should also be\\  assigned at least\\ one classification\\ number from Section 30-XX)};
\draw (3, -89.3) rectangle (10.1,-92.89999999999999);
\draw(10.2, -89.3) node[anchor=north west,align=left] {Other spaces of\\ holomorphic functions of\\  several complex\\ variables (e.g., bounded\\  mean oscillation\\ (BMOA), vanishing mean\\ oscillation (VMOA))};
\draw (10.2, -89.3) rectangle (16.799999999999997,-92.89999999999999);
\draw(16.9, -89.3) node[anchor=north west,align=left] {Hyperfunctions};
\draw (16.9, -89.3) rectangle (21.0,-89.89999999999999);
\draw(16.9, -90.0) node[anchor=north west,align=left] {Residues\\ for several\\ complex\\ variables};
\draw (16.9, -90.0) rectangle (20.25,-92.1);
\draw(3, -93.0) node[anchor=north west,align=left] {Integral\\ representations,\\ constructed kernels\\  (e.g.,\\ Cauchy, Fantappiè-type\\ kernels)};
\draw (3, -93.0) rectangle (9.1,-96.1);
\draw(9.2, -93.0) node[anchor=north west,align=left] {Normal families\\ of holomorphic\\ functions, mappings\\ of several complex\\  variables, and\\ related topics\\ (taut manifolds etc.)};
\draw (9.2, -93.0) rectangle (15.049999999999999,-96.6);
\draw(15.149999999999999, -93.0) node[anchor=north west,align=left] {Functional analysis\\  techniques\\ applied to functions\\  of several\\ complex variables};
\draw (15.149999999999999, -93.0) rectangle (20.75,-95.6);
\draw(3, -96.7) node[anchor=north west,align=left] {Banach algebra\\ techniques applied\\ to functions\\  of several\\ complex variables};
\draw (3, -96.7) rectangle (8.1,-99.3);
\draw(8.2, -96.7) node[anchor=north west,align=left] {Power series,\\  series of\\ functions of\\  several\\ complex variables};
\draw (8.2, -96.7) rectangle (13.049999999999999,-99.3);
\draw(13.149999999999999, -96.7) node[anchor=north west,align=left] {Polynomials\\ and rational\\  functions\\ of several\\ complex variables};
\draw (13.149999999999999, -96.7) rectangle (18.0,-99.3);
\draw(3, -99.4) node[anchor=north west,align=left] {Holomorphic\\ functions of\\  several\\ complex variables};
\draw (3, -99.4) rectangle (7.85,-101.5);
\draw(7.949999999999999, -99.4) node[anchor=north west,align=left] {Special\\ families of\\ functions of\\ several\\ complex variables};
\draw (7.949999999999999, -99.4) rectangle (12.799999999999999,-102.0);
\draw(12.899999999999999, -99.4) node[anchor=north west,align=left] {Bloch functions,\\  normal\\ functions of\\  several\\ complex variables};
\draw (12.899999999999999, -99.4) rectangle (17.75,-102.0);
\draw(3, -102.1) node[anchor=north west,align=left] {Meromorphic\\ functions of\\  several\\ complex variables};
\draw (3, -102.1) rectangle (7.85,-104.19999999999999);
\draw(7.949999999999999, -102.1) node[anchor=north west,align=left] {Nevanlinna\\ theory; growth\\ estimates; other\\  inequalities\\ of several\\ complex variables};
\draw (7.949999999999999, -102.1) rectangle (12.799999999999999,-105.19999999999999);
\draw(12.899999999999999, -102.1) node[anchor=north west,align=left] {\(H^p\)-spaces,\\ Nevanlinna spaces\\ of functions\\  in several\\ complex variables};
\draw (12.899999999999999, -102.1) rectangle (17.75,-104.69999999999999);
\draw(3, -105.3) node[anchor=north west,align=left] {Algebras of\\ holomorphic\\ functions of\\  several\\ complex variables};
\draw (3, -105.3) rectangle (7.85,-107.89999999999999);
\draw(7.949999999999999, -105.3) node[anchor=north west,align=left] {Boundary behavior\\ of holomorphic\\ functions\\  of several\\ complex variables};
\draw (7.949999999999999, -105.3) rectangle (12.799999999999999,-107.89999999999999);
\draw(12.899999999999999, -105.3) node[anchor=north west,align=left] {Zero sets of\\ holomorphic\\  functions\\ of several\\ complex variables};
\draw (12.899999999999999, -105.3) rectangle (17.75,-107.89999999999999);
\draw(3, -108.0) node[anchor=north west,align=left] {Integral\\ representations;\\ canonical\\ kernels (Szegő,\\ Bergman, etc.)};
\draw (3, -108.0) rectangle (7.6,-110.6);
\draw(7.699999999999999, -108.0) node[anchor=north west,align=left] {Multifunctions\\  of\\ several complex\\ variables};
\draw (7.699999999999999, -108.0) rectangle (12.049999999999999,-110.1);
\draw(12.149999999999999, -108.0) node[anchor=north west,align=left] {Entire\\ functions of\\ several complex\\ variables};
\draw (12.149999999999999, -108.0) rectangle (16.5,-110.1);
\draw(16.599999999999998, -108.0) node[anchor=north west,align=left] {Bergman\\ spaces of\\ functions in\\ several complex\\ variables};
\draw (16.599999999999998, -108.0) rectangle (20.949999999999996,-110.6);
\draw(3, -110.69999999999999) node[anchor=north west,align=left] {Harmonic\\ analysis of\\ several complex\\ variables};
\draw (3, -110.69999999999999) rectangle (7.35,-112.79999999999998);
\draw(7.449999999999999, -110.69999999999999) node[anchor=north west,align=left] {Singular\\ integrals of\\ functions in\\ several complex\\ variables};
\draw (7.449999999999999, -110.69999999999999) rectangle (11.799999999999999,-113.29999999999998);
\draw(21.200000000000003, -88.3) node[anchor=north west,align=left] {\large Geometric convexity in several complex variables};
\draw (21.200000000000003, -88.3) rectangle (37.550000000000004,-94.2);
\draw(22.200000000000003, -89.3) node[anchor=north west,align=left] {Analytical\\ consequences of\\ geometric convexity\\ (vanishing\\ theorems, etc.)};
\draw (22.200000000000003, -89.3) rectangle (27.550000000000004,-91.89999999999999);
\draw(27.650000000000002, -89.3) node[anchor=north west,align=left] {Other notions\\ of convexity\\  in relation\\ to several\\ complex variables};
\draw (27.650000000000002, -89.3) rectangle (32.5,-91.89999999999999);
\draw(32.6, -89.3) node[anchor=north west,align=left] {Invariant\\ metrics and\\ pseudodistances\\ in several\\ complex variables};
\draw (32.6, -89.3) rectangle (37.45,-91.89999999999999);
\draw(22.200000000000003, -92.0) node[anchor=north west,align=left] {Finite-type\\ conditions\\ for the boundary\\ of a domain};
\draw (22.200000000000003, -92.0) rectangle (26.800000000000004,-94.1);
\draw(26.900000000000002, -92.0) node[anchor=north west,align=left] {\(q\)-convexity,\\ \(q\)-concavity};
\draw (26.900000000000002, -92.0) rectangle (30.500000000000004,-93.1);
\draw(30.6, -92.0) node[anchor=north west,align=left] {Topological\\ consequences\\ of geometric\\ convexity};
\draw (30.6, -92.0) rectangle (34.2,-94.1);
\draw(21.200000000000003, -94.3) node[anchor=north west,align=left] {\large Differential operators in several variables};
\draw (21.200000000000003, -94.3) rectangle (36.8,-100.2);
\draw(22.200000000000003, -95.3) node[anchor=north west,align=left] {Other partial\\ differential\\ equations of complex\\  analysis in\\ several variables};
\draw (22.200000000000003, -95.3) rectangle (27.800000000000004,-97.89999999999999);
\draw(27.900000000000002, -95.3) node[anchor=north west,align=left] {Pseudodifferential\\ operators in\\ several complex\\ variables};
\draw (27.900000000000002, -95.3) rectangle (33.0,-97.39999999999999);
\draw(33.1, -95.3) node[anchor=north west,align=left] {Complex\\ Monge-Ampère\\ operators};
\draw (33.1, -95.3) rectangle (36.7,-96.89999999999999);
\draw(22.200000000000003, -98.0) node[anchor=north west,align=left] {Heat kernels\\ in several\\ complex\\ variables};
\draw (22.200000000000003, -98.0) rectangle (25.800000000000004,-100.1);
\draw(25.900000000000002, -98.0) node[anchor=north west,align=left] {\(\overline\partial\) and\\ \(\overline\partial\)-Neumann\\ operators};
\draw (25.900000000000002, -98.0) rectangle (28.750000000000004,-99.6);
\draw(28.85, -98.0) node[anchor=north west,align=left] {\(\overline\partial_b\) and\\ \(\overline\partial_b\)-Neumann\\ operators};
\draw (28.85, -98.0) rectangle (31.700000000000003,-99.6);
\draw(21.200000000000003, -100.3) node[anchor=north west,align=left] {\large Deformations of analytic structures};
\draw (21.200000000000003, -100.3) rectangle (34.550000000000004,-112.1);
\draw(22.200000000000003, -101.3) node[anchor=north west,align=left] {Moduli and\\ deformations for\\ ordinary differential\\ equations (e.g.,\\ Knizhnik-Zamolodchikov\\ equation)};
\draw (22.200000000000003, -101.3) rectangle (28.300000000000004,-104.39999999999999);
\draw(28.400000000000002, -101.3) node[anchor=north west,align=left] {Moduli of Riemann\\  surfaces,\\ Teichmüller theory\\ (complex-analytic\\ aspects in\\ several variables)};
\draw (28.400000000000002, -101.3) rectangle (33.5,-104.39999999999999);
\draw(22.200000000000003, -104.5) node[anchor=north west,align=left] {Complex-analytic\\ moduli problems};
\draw (22.200000000000003, -104.5) rectangle (26.800000000000004,-105.6);
\draw(26.900000000000002, -104.5) node[anchor=north west,align=left] {Period matrices,\\ variation\\  of Hodge\\ structure;\\ degenerations};
\draw (26.900000000000002, -104.5) rectangle (31.5,-107.1);
\draw(22.200000000000003, -107.2) node[anchor=north west,align=left] {Applications\\ of deformations\\ of analytic\\  structures\\ to the sciences};
\draw (22.200000000000003, -107.2) rectangle (26.550000000000004,-109.8);
\draw(26.650000000000002, -107.2) node[anchor=north west,align=left] {Deformations\\  of\\ fiber bundles};
\draw (26.650000000000002, -107.2) rectangle (30.500000000000004,-108.8);
\draw(30.6, -107.2) node[anchor=north west,align=left] {Deformations\\  of\\ submanifolds\\ and subspaces};
\draw (30.6, -107.2) rectangle (34.45,-109.3);
\draw(22.200000000000003, -109.9) node[anchor=north west,align=left] {Deformations\\  of\\ complex\\ structures};
\draw (22.200000000000003, -109.9) rectangle (25.800000000000004,-112.0);
\draw(25.900000000000002, -109.9) node[anchor=north west,align=left] {Deformations\\ of special\\ (e.g., CR)\\ structures};
\draw (25.900000000000002, -109.9) rectangle (29.500000000000004,-112.0);
\draw(37.650000000000006, -88.3) node[anchor=north west,align=left] {\large Holomorphic mappings and correspondences};
\draw (37.650000000000006, -88.3) rectangle (52.25,-101.6);
\draw(38.650000000000006, -89.3) node[anchor=north west,align=left] {Holomorphic\\ mappings, (holomorphic)\\  embeddings\\ and related questions\\  in several\\ complex variables};
\draw (38.650000000000006, -89.3) rectangle (45.00000000000001,-92.39999999999999);
\draw(45.10000000000001, -89.3) node[anchor=north west,align=left] {Proper\\ holomorphic mappings,\\ finiteness\\ theorems};
\draw (45.10000000000001, -89.3) rectangle (50.95000000000001,-91.39999999999999);
\draw(38.650000000000006, -92.5) node[anchor=north west,align=left] {Iteration of\\ holomorphic maps,\\ fixed points of\\ holomorphic maps\\ and related\\ problems for several\\ complex variables};
\draw (38.650000000000006, -92.5) rectangle (44.25000000000001,-96.1);
\draw(44.35000000000001, -92.5) node[anchor=north west,align=left] {Picard-type\\ theorems and\\ generalizations\\ for several\\ complex variables};
\draw (44.35000000000001, -92.5) rectangle (49.20000000000001,-95.1);
\draw(38.650000000000006, -96.2) node[anchor=north west,align=left] {Value\\ distribution\\ theory in higher\\ dimensions};
\draw (38.650000000000006, -96.2) rectangle (43.25000000000001,-98.3);
\draw(43.35000000000001, -96.2) node[anchor=north west,align=left] {Meromorphic\\ mappings in\\ several complex\\ variables};
\draw (43.35000000000001, -96.2) rectangle (47.70000000000001,-98.3);
\draw(47.800000000000004, -96.2) node[anchor=north west,align=left] {Boundary\\ uniqueness of\\ mappings in\\ several complex\\ variables};
\draw (47.800000000000004, -96.2) rectangle (52.150000000000006,-98.8);
\draw(38.650000000000006, -98.9) node[anchor=north west,align=left] {Boundary\\ regularity of\\ mappings in\\ several complex\\ variables};
\draw (38.650000000000006, -98.9) rectangle (43.00000000000001,-101.5);
\draw(52.35, -88.3) node[anchor=north west,align=left] {\large Local analytic geometry};
\draw (52.35, -88.3) rectangle (63.75,-96.89999999999999);
\draw(53.35, -89.3) node[anchor=north west,align=left] {Triangulation and\\  topological\\ properties of\\ semi-analytic\\ andsubanalytic sets,\\ and related questions};
\draw (53.35, -89.3) rectangle (59.2,-92.39999999999999);
\draw(59.3, -89.3) node[anchor=north west,align=left] {Germs of\\ analytic sets,\\  local\\ parametrization};
\draw (59.3, -89.3) rectangle (63.65,-91.39999999999999);
\draw(53.35, -92.5) node[anchor=north west,align=left] {Analytic\\ algebras and\\ generalizations,\\ preparation theorems};
\draw (53.35, -92.5) rectangle (58.95,-94.6);
\draw(59.05, -92.5) node[anchor=north west,align=left] {Semi-analytic\\  sets,\\ subanalytic\\ sets, and\\ generalizations};
\draw (59.05, -92.5) rectangle (63.4,-95.1);
\draw(53.35, -95.2) node[anchor=north west,align=left] {Analytic\\ subsets of\\ affine space};
\draw (53.35, -95.2) rectangle (56.95,-96.8);
\draw(52.35, -97.0) node[anchor=north west,align=left] {\large History of several\\ complex variables\\ and analytic spaces};
\draw (52.35, -97.0) rectangle (58.84,-98.6);
\draw(2, -113.5) node[anchor=north west,align=left] {\large Complex singularities};
\draw (2, -113.5) rectangle (16.099999999999998,-131.7);
\draw(3, -114.5) node[anchor=north west,align=left] {Monodromy;\\ relations with\\ differential\\ equations and\\ \(D\)-modules\\ (complex-analytic aspects)};
\draw (3, -114.5) rectangle (10.1,-117.6);
\draw(10.2, -114.5) node[anchor=north west,align=left] {Mixed Hodge\\ theory of\\ singular varieties\\ (complex-analytic\\ aspects)};
\draw (10.2, -114.5) rectangle (15.299999999999999,-117.1);
\draw(3, -117.7) node[anchor=north west,align=left] {Topological aspects\\  of complex\\ singularities: Lefschetz\\  theorems,\\ topological\\ classification, invariants};
\draw (3, -117.7) rectangle (10.1,-120.8);
\draw(10.2, -117.7) node[anchor=north west,align=left] {Singularities\\  of\\ holomorphic vector\\  fields\\ and foliations};
\draw (10.2, -117.7) rectangle (15.299999999999999,-120.3);
\draw(3, -120.9) node[anchor=north west,align=left] {Stratifications;\\ constructible\\  sheaves;\\ intersection cohomology\\ (complex-analytic\\ aspects)};
\draw (3, -120.9) rectangle (9.35,-124.0);
\draw(9.45, -120.9) node[anchor=north west,align=left] {Modifications;\\  resolution\\ of singularities\\ (complex-analytic\\ aspects)};
\draw (9.45, -120.9) rectangle (14.299999999999999,-123.5);
\draw(3, -124.1) node[anchor=north west,align=left] {Equisingularity\\ (topological\\ and analytic)};
\draw (3, -124.1) rectangle (7.35,-125.69999999999999);
\draw(7.449999999999999, -124.1) node[anchor=north west,align=left] {Relations\\ with\\ arrangements of\\ hyperplanes};
\draw (7.449999999999999, -124.1) rectangle (11.799999999999999,-126.19999999999999);
\draw(11.899999999999999, -124.1) node[anchor=north west,align=left] {Global theory\\ of complex\\ singularities;\\ cohomological\\ properties};
\draw (11.899999999999999, -124.1) rectangle (15.999999999999998,-126.69999999999999);
\draw(3, -126.8) node[anchor=north west,align=left] {Deformations\\ of complex\\ singularities;\\ vanishing\\ cycles};
\draw (3, -126.8) rectangle (7.1,-129.4);
\draw(7.199999999999999, -126.8) node[anchor=north west,align=left] {Milnor\\ fibration;\\ relations with\\ knot theory};
\draw (7.199999999999999, -126.8) rectangle (11.299999999999999,-128.9);
\draw(11.399999999999999, -126.8) node[anchor=north west,align=left] {Local\\ complex\\ singularities};
\draw (11.399999999999999, -126.8) rectangle (15.249999999999998,-128.4);
\draw(3, -129.5) node[anchor=north west,align=left] {Complex\\ surface and\\ hypersurface\\ singularities};
\draw (3, -129.5) rectangle (6.85,-131.6);
\draw(6.949999999999999, -129.5) node[anchor=north west,align=left] {Other\\ operations on\\ complex\\ singularities};
\draw (6.949999999999999, -129.5) rectangle (10.799999999999999,-131.6);
\draw(10.9, -129.5) node[anchor=north west,align=left] {Invariants\\  of\\ analytic\\ local rings};
\draw (10.9, -129.5) rectangle (14.25,-131.6);
\draw(16.199999999999996, -113.5) node[anchor=north west,align=left] {\large Generalizations of analytic spaces};
\draw (16.199999999999996, -113.5) rectangle (28.849999999999994,-119.4);
\draw(17.199999999999996, -114.5) node[anchor=north west,align=left] {Differentiable\\  functions\\ on analytic\\ spaces,\\ differentiable spaces};
\draw (17.199999999999996, -114.5) rectangle (23.049999999999997,-117.1);
\draw(23.149999999999995, -114.5) node[anchor=north west,align=left] {Holomorphic\\ maps with\\ infinite-dimensional\\ arguments\\ or values};
\draw (23.149999999999995, -114.5) rectangle (28.749999999999993,-117.1);
\draw(17.199999999999996, -117.2) node[anchor=north west,align=left] {Formal and\\  graded\\ complex spaces};
\draw (17.199999999999996, -117.2) rectangle (21.299999999999997,-118.8);
\draw(21.399999999999995, -117.2) node[anchor=north west,align=left] {Banach\\ analytic\\ manifolds\\ and spaces};
\draw (21.399999999999995, -117.2) rectangle (24.499999999999996,-119.3);
\draw(16.199999999999996, -119.5) node[anchor=north west,align=left] {\large Holomorphic convexity};
\draw (16.199999999999996, -119.5) rectangle (28.599999999999994,-129.6);
\draw(17.199999999999996, -120.5) node[anchor=north west,align=left] {Holomorphic,\\ polynomial and rational\\ approximation, and\\ interpolation in\\  several complex\\ variables; Runge pairs};
\draw (17.199999999999996, -120.5) rectangle (23.549999999999997,-123.6);
\draw(23.649999999999995, -120.5) node[anchor=north west,align=left] {Holomorphically\\ convex\\  complex\\ spaces, reduction\\ theory};
\draw (23.649999999999995, -120.5) rectangle (28.499999999999993,-123.1);
\draw(17.199999999999996, -123.7) node[anchor=north west,align=left] {Polynomial\\ convexity, rational\\  convexity,\\ meromorphic convexity\\  in several\\ complex variables};
\draw (17.199999999999996, -123.7) rectangle (23.049999999999997,-126.8);
\draw(23.149999999999995, -123.7) node[anchor=north west,align=left] {Stein\\ spaces, Stein\\ manifolds};
\draw (23.149999999999995, -123.7) rectangle (26.999999999999996,-125.3);
\draw(23.149999999999995, -125.4) node[anchor=north west,align=left] {The Levi\\ problem};
\draw (23.149999999999995, -125.4) rectangle (25.749999999999996,-126.5);
\draw(17.199999999999996, -126.9) node[anchor=north west,align=left] {Global boundary\\  behavior of\\ holomorphic functions\\ of several\\ complex variables};
\draw (17.199999999999996, -126.9) rectangle (23.049999999999997,-129.5);
\draw(28.949999999999996, -113.5) node[anchor=north west,align=left] {\large Analytic spaces};
\draw (28.949999999999996, -113.5) rectangle (39.349999999999994,-132.1);
\draw(29.949999999999996, -114.5) node[anchor=north west,align=left] {Sheaves of\\ differential\\ operators and\\  their\\ modules, \(D\)-modules};
\draw (29.949999999999996, -114.5) rectangle (35.05,-117.1);
\draw(35.14999999999999, -114.5) node[anchor=north west,align=left] {Real-analytic\\  sets,\\ complex\\ Nash functions};
\draw (35.14999999999999, -114.5) rectangle (39.24999999999999,-116.6);
\draw(29.949999999999996, -117.2) node[anchor=north west,align=left] {The Levi\\ problem in complex\\  spaces;\\ generalizations};
\draw (29.949999999999996, -117.2) rectangle (35.05,-119.3);
\draw(35.14999999999999, -117.2) node[anchor=north west,align=left] {Analytic\\ sheaves\\ and cohomology\\ groups};
\draw (35.14999999999999, -117.2) rectangle (39.24999999999999,-119.3);
\draw(29.949999999999996, -119.4) node[anchor=north west,align=left] {Applications\\ of analytic\\ spaces to physics\\ and other\\ areas of science};
\draw (29.949999999999996, -119.4) rectangle (34.8,-122.0);
\draw(34.89999999999999, -119.4) node[anchor=north west,align=left] {Local\\ cohomology\\  of\\ analytic spaces};
\draw (34.89999999999999, -119.4) rectangle (39.24999999999999,-121.5);
\draw(29.949999999999996, -122.1) node[anchor=north west,align=left] {Integration\\ on analytic\\ sets and\\ spaces, currents};
\draw (29.949999999999996, -122.1) rectangle (34.55,-124.19999999999999);
\draw(34.64999999999999, -122.1) node[anchor=north west,align=left] {Real-analytic\\ manifolds,\\ real-analytic\\ spaces};
\draw (34.64999999999999, -122.1) rectangle (38.49999999999999,-124.19999999999999);
\draw(29.949999999999996, -124.3) node[anchor=north west,align=left] {Embedding\\  of\\ real-analytic\\ manifolds};
\draw (29.949999999999996, -124.3) rectangle (33.8,-126.39999999999999);
\draw(33.89999999999999, -124.3) node[anchor=north west,align=left] {Complex\\ supergeometry};
\draw (33.89999999999999, -124.3) rectangle (37.74999999999999,-125.39999999999999);
\draw(29.949999999999996, -126.5) node[anchor=north west,align=left] {Analytic\\ subsets\\  and\\ submanifolds};
\draw (29.949999999999996, -126.5) rectangle (33.55,-128.6);
\draw(33.64999999999999, -126.5) node[anchor=north west,align=left] {Duality\\ theorems\\ for analytic\\ spaces};
\draw (33.64999999999999, -126.5) rectangle (37.24999999999999,-128.6);
\draw(29.949999999999996, -128.7) node[anchor=north west,align=left] {Topology\\ of analytic\\ spaces};
\draw (29.949999999999996, -128.7) rectangle (33.3,-130.29999999999998);
\draw(33.39999999999999, -128.7) node[anchor=north west,align=left] {Embedding\\ of analytic\\ spaces};
\draw (33.39999999999999, -128.7) rectangle (36.74999999999999,-130.29999999999998);
\draw(36.849999999999994, -128.7) node[anchor=north west,align=left] {Complex\\ spaces};
\draw (36.849999999999994, -128.7) rectangle (39.199999999999996,-129.79999999999998);
\draw(29.949999999999996, -130.4) node[anchor=north west,align=left] {Normal\\ analytic\\ spaces};
\draw (29.949999999999996, -130.4) rectangle (32.55,-132.0);
\draw(39.449999999999996, -113.5) node[anchor=north west,align=left] {\large Complex manifolds};
\draw (39.449999999999996, -113.5) rectangle (49.599999999999994,-132.6);
\draw(40.449999999999996, -114.5) node[anchor=north west,align=left] {Special domains\\  (Reinhardt,\\ Hartogs, circular,\\ tube, etc.)\\  in \(\mathbb{C}^n\) and\\ complex manifolds};
\draw (40.449999999999996, -114.5) rectangle (45.55,-117.6);
\draw(45.64999999999999, -114.5) node[anchor=north west,align=left] {Hyperbolic\\ and Kobayashi\\ hyperbolic\\ manifolds};
\draw (45.64999999999999, -114.5) rectangle (49.49999999999999,-116.6);
\draw(40.449999999999996, -117.7) node[anchor=north west,align=left] {Calabi-Yau\\ theory\\ (complex-analytic\\ aspects)};
\draw (40.449999999999996, -117.7) rectangle (45.3,-119.8);
\draw(45.39999999999999, -117.7) node[anchor=north west,align=left] {Uniformization\\ of complex\\ manifolds};
\draw (45.39999999999999, -117.7) rectangle (49.49999999999999,-119.3);
\draw(40.449999999999996, -119.9) node[anchor=north west,align=left] {Pseudoholomorphic\\ curves};
\draw (40.449999999999996, -119.9) rectangle (45.3,-121.0);
\draw(45.39999999999999, -119.9) node[anchor=north west,align=left] {Classification\\ theorems\\ for complex\\ manifolds};
\draw (45.39999999999999, -119.9) rectangle (49.49999999999999,-122.0);
\draw(40.449999999999996, -122.1) node[anchor=north west,align=left] {Kähler-Einstein\\ manifolds};
\draw (40.449999999999996, -122.1) rectangle (44.8,-123.19999999999999);
\draw(44.89999999999999, -122.1) node[anchor=north west,align=left] {Complex\\ manifolds as\\ subdomains of\\ Euclidean space};
\draw (44.89999999999999, -122.1) rectangle (49.24999999999999,-124.19999999999999);
\draw(40.449999999999996, -124.3) node[anchor=north west,align=left] {Oka principle\\ and Oka\\ manifolds};
\draw (40.449999999999996, -124.3) rectangle (44.3,-125.89999999999999);
\draw(44.39999999999999, -124.3) node[anchor=north west,align=left] {Notions of\\ stability\\ for complex\\ manifolds};
\draw (44.39999999999999, -124.3) rectangle (47.74999999999999,-126.39999999999999);
\draw(40.449999999999996, -126.5) node[anchor=north west,align=left] {Embedding\\ theorems\\ for complex\\ manifolds};
\draw (40.449999999999996, -126.5) rectangle (43.8,-128.6);
\draw(43.89999999999999, -126.5) node[anchor=north west,align=left] {Topological\\  aspects\\ of complex\\ manifolds};
\draw (43.89999999999999, -126.5) rectangle (47.24999999999999,-128.6);
\draw(40.449999999999996, -128.7) node[anchor=north west,align=left] {Negative\\ curvature\\  complex\\ manifolds};
\draw (40.449999999999996, -128.7) rectangle (43.3,-130.79999999999998);
\draw(43.4, -128.7) node[anchor=north west,align=left] {Positive\\ curvature\\  complex\\ manifolds};
\draw (43.4, -128.7) rectangle (46.25,-130.79999999999998);
\draw(46.349999999999994, -128.7) node[anchor=north west,align=left] {Kähler\\ manifolds};
\draw (46.349999999999994, -128.7) rectangle (49.199999999999996,-129.79999999999998);
\draw(40.449999999999996, -130.9) node[anchor=north west,align=left] {Stein\\ manifolds};
\draw (40.449999999999996, -130.9) rectangle (43.3,-132.0);
\draw(43.4, -130.9) node[anchor=north west,align=left] {Almost\\ complex\\ manifolds};
\draw (43.4, -130.9) rectangle (46.25,-132.5);
\draw(49.699999999999996, -113.5) node[anchor=north west,align=left] {\large Compact analytic spaces};
\draw (49.699999999999996, -113.5) rectangle (59.599999999999994,-122.8);
\draw(50.699999999999996, -114.5) node[anchor=north west,align=left] {Transcendental\\  methods of\\ algebraic geometry\\ (complex-analytic\\ aspects)};
\draw (50.699999999999996, -114.5) rectangle (55.8,-117.1);
\draw(55.89999999999999, -114.5) node[anchor=north west,align=left] {Applications\\ of compact\\  analytic\\ spaces to\\ the sciences};
\draw (55.89999999999999, -114.5) rectangle (59.49999999999999,-117.1);
\draw(50.699999999999996, -117.2) node[anchor=north west,align=left] {Compact\\ Kähler manifolds:\\ generalizations,\\ classification};
\draw (50.699999999999996, -117.2) rectangle (55.55,-119.3);
\draw(55.64999999999999, -117.2) node[anchor=north west,align=left] {Algebraic\\ dependence\\ theorems};
\draw (55.64999999999999, -117.2) rectangle (58.74999999999999,-118.8);
\draw(50.699999999999996, -119.4) node[anchor=north west,align=left] {Compactification\\  of\\ analytic spaces};
\draw (50.699999999999996, -119.4) rectangle (55.3,-121.0);
\draw(55.39999999999999, -119.4) node[anchor=north west,align=left] {Compact\\ complex\\ surfaces};
\draw (55.39999999999999, -119.4) rectangle (57.99999999999999,-121.0);
\draw(50.699999999999996, -121.1) node[anchor=north west,align=left] {Compact\\ complex\\ \(3\)-folds};
\draw (50.699999999999996, -121.1) rectangle (53.05,-122.69999999999999);
\draw(53.15, -121.1) node[anchor=north west,align=left] {Compact\\ complex\\ \(n\)-folds};
\draw (53.15, -121.1) rectangle (55.5,-122.69999999999999);
\draw(49.699999999999996, -122.9) node[anchor=north west,align=left] {\large Automorphic functions};
\draw (49.699999999999996, -122.9) rectangle (58.849999999999994,-128.8);
\draw(50.699999999999996, -123.9) node[anchor=north west,align=left] {General theory\\ of automorphic\\ functions\\  of several\\ complex variables};
\draw (50.699999999999996, -123.9) rectangle (55.55,-126.5);
\draw(50.699999999999996, -126.60000000000001) node[anchor=north west,align=left] {Automorphic\\  forms in\\ several complex\\ variables};
\draw (50.699999999999996, -126.60000000000001) rectangle (55.05,-128.70000000000002);
\draw(55.14999999999999, -126.60000000000001) node[anchor=north west,align=left] {Automorphic\\ functions\\ in symmetric\\ domains};
\draw (55.14999999999999, -126.60000000000001) rectangle (58.74999999999999,-128.70000000000002);
\draw(2, -132.7) node[anchor=north west,align=left] {\large Pluripotential theory};
\draw (2, -132.7) rectangle (11.399999999999999,-144.7);
\draw(3, -133.7) node[anchor=north west,align=left] {Plurisubharmonic\\ functions\\  and\\ generalizations};
\draw (3, -133.7) rectangle (7.6,-135.79999999999998);
\draw(7.699999999999999, -133.7) node[anchor=north west,align=left] {Currents};
\draw (7.699999999999999, -133.7) rectangle (10.299999999999999,-134.29999999999998);
\draw(7.699999999999999, -134.39999999999998) node[anchor=north west,align=left] {Lelong\\ numbers};
\draw (7.699999999999999, -134.39999999999998) rectangle (10.049999999999999,-135.49999999999997);
\draw(3, -135.89999999999998) node[anchor=north west,align=left] {Plurisubharmonic\\ exhaustion\\ functions};
\draw (3, -135.89999999999998) rectangle (7.6,-137.49999999999997);
\draw(3, -137.6) node[anchor=north west,align=left] {Plurisubharmonic\\  extremal\\ functions,\\ pluricomplex\\ Green functions};
\draw (3, -137.6) rectangle (7.6,-140.2);
\draw(3, -140.29999999999998) node[anchor=north west,align=left] {Capacity\\  theory\\ and\\ generalizations};
\draw (3, -140.29999999999998) rectangle (7.35,-142.39999999999998);
\draw(3, -142.5) node[anchor=north west,align=left] {General\\ pluripotential\\ theory};
\draw (3, -142.5) rectangle (7.1,-144.1);
\draw(7.199999999999999, -142.5) node[anchor=north west,align=left] {Removable\\  sets in\\ pluripotential\\ theory};
\draw (7.199999999999999, -142.5) rectangle (11.299999999999999,-144.6);
\draw(11.499999999999998, -132.7) node[anchor=north west,align=left] {\large CR manifolds};
\draw (11.499999999999998, -132.7) rectangle (20.65,-142.5);
\draw(12.499999999999998, -133.7) node[anchor=north west,align=left] {Extension of\\ functions and\\ other analytic\\  objects\\ from CR manifolds};
\draw (12.499999999999998, -133.7) rectangle (17.349999999999998,-136.29999999999998);
\draw(17.449999999999996, -133.7) node[anchor=north west,align=left] {Embeddings\\ of CR\\ manifolds};
\draw (17.449999999999996, -133.7) rectangle (20.549999999999997,-135.29999999999998);
\draw(12.499999999999998, -136.39999999999998) node[anchor=north west,align=left] {CR structures,\\ CR operators,\\  and\\ generalizations};
\draw (12.499999999999998, -136.39999999999998) rectangle (16.849999999999998,-138.49999999999997);
\draw(16.949999999999996, -136.39999999999998) node[anchor=north west,align=left] {CR manifolds\\  as\\ boundaries\\ of domains};
\draw (16.949999999999996, -136.39999999999998) rectangle (20.549999999999997,-138.49999999999997);
\draw(12.499999999999998, -138.6) node[anchor=north west,align=left] {Finite-type\\ conditions\\  on\\ CR manifolds};
\draw (12.499999999999998, -138.6) rectangle (16.099999999999998,-140.7);
\draw(16.199999999999996, -138.6) node[anchor=north west,align=left] {Real\\ submanifolds\\ in complex\\ manifolds};
\draw (16.199999999999996, -138.6) rectangle (19.799999999999997,-140.7);
\draw(12.499999999999998, -140.79999999999998) node[anchor=north west,align=left] {CR\\ functions};
\draw (12.499999999999998, -140.79999999999998) rectangle (15.349999999999998,-141.89999999999998);
\draw(15.45, -140.79999999999998) node[anchor=north west,align=left] {Analysis\\  on CR\\ manifolds};
\draw (15.45, -140.79999999999998) rectangle (18.3,-142.39999999999998);
\draw(20.75, -132.7) node[anchor=north west,align=left] {\large Analytic continuation};
\draw (20.75, -132.7) rectangle (29.65,-139.79999999999998);
\draw(21.75, -133.7) node[anchor=north west,align=left] {Removable\\ singularities in\\ several complex\\ variables};
\draw (21.75, -133.7) rectangle (26.35,-135.79999999999998);
\draw(26.45, -133.7) node[anchor=north west,align=left] {Domains\\  of\\ holomorphy};
\draw (26.45, -133.7) rectangle (29.55,-135.29999999999998);
\draw(21.75, -135.89999999999998) node[anchor=north west,align=left] {Continuation\\ of analytic\\  objects in\\ several complex\\ variables};
\draw (21.75, -135.89999999999998) rectangle (26.1,-138.49999999999997);
\draw(26.2, -135.89999999999998) node[anchor=north west,align=left] {Envelopes\\  of\\ holomorphy};
\draw (26.2, -135.89999999999998) rectangle (29.3,-137.49999999999997);
\draw(21.75, -138.6) node[anchor=north west,align=left] {Riemann\\ domains};
\draw (21.75, -138.6) rectangle (24.1,-139.7);
\draw(29.75, -132.7) node[anchor=north west,align=left] {\large Holomorphic fiber spaces};
\draw (29.75, -132.7) rectangle (38.65,-144.5);
\draw(30.75, -133.7) node[anchor=north west,align=left] {Twistor theory,\\  double\\ fibrations\\ (complex-analytic\\ aspects)};
\draw (30.75, -133.7) rectangle (35.6,-136.29999999999998);
\draw(35.7, -133.7) node[anchor=north west,align=left] {Bundle\\ convexity};
\draw (35.7, -133.7) rectangle (38.550000000000004,-134.79999999999998);
\draw(35.7, -134.89999999999998) node[anchor=north west,align=left] {Vanishing\\ theorems};
\draw (35.7, -134.89999999999998) rectangle (38.550000000000004,-135.99999999999997);
\draw(30.75, -136.39999999999998) node[anchor=north west,align=left] {Holomorphic\\ bundles\\  and\\ generalizations};
\draw (30.75, -136.39999999999998) rectangle (35.1,-138.49999999999997);
\draw(30.75, -138.6) node[anchor=north west,align=left] {Sheaves and\\ cohomology of\\ sections of\\ holomorphic\\ vector bundles,\\ general results};
\draw (30.75, -138.6) rectangle (35.1,-141.7);
\draw(30.75, -141.79999999999998) node[anchor=north west,align=left] {Applications\\ of holomorphic\\ fiber\\  spaces to\\ the sciences};
\draw (30.75, -141.79999999999998) rectangle (34.85,-144.39999999999998);
\draw(38.75, -132.7) node[anchor=north west,align=left] {\large Pseudoconvex domains};
\draw (38.75, -132.7) rectangle (47.65,-141.0);
\draw(39.75, -133.7) node[anchor=north west,align=left] {Geometric and\\  analytic\\ invariants on\\ weakly pseudoconvex\\ boundaries};
\draw (39.75, -133.7) rectangle (45.1,-136.29999999999998);
\draw(45.2, -133.7) node[anchor=north west,align=left] {Worm\\ domains};
\draw (45.2, -133.7) rectangle (47.550000000000004,-134.79999999999998);
\draw(39.75, -136.39999999999998) node[anchor=north west,align=left] {Strongly\\ pseudoconvex\\ domains};
\draw (39.75, -136.39999999999998) rectangle (43.35,-137.99999999999997);
\draw(43.45, -136.39999999999998) node[anchor=north west,align=left] {Finite-type\\ domains};
\draw (43.45, -136.39999999999998) rectangle (46.800000000000004,-137.49999999999997);
\draw(39.75, -138.1) node[anchor=north west,align=left] {Domains\\  of\\ holomorphy};
\draw (39.75, -138.1) rectangle (42.85,-139.7);
\draw(42.95, -138.1) node[anchor=north west,align=left] {Exhaustion\\ functions};
\draw (42.95, -138.1) rectangle (46.050000000000004,-139.2);
\draw(39.75, -139.79999999999998) node[anchor=north west,align=left] {Peak\\ functions};
\draw (39.75, -139.79999999999998) rectangle (42.6,-140.89999999999998);
\draw(63.95, -80.3) node[anchor=north west,align=left] {\LARGE Associative rings and algebras};
\draw (63.95, -80.3) rectangle (122.10000000000001,-126.4);
\draw(64.95, -81.3) node[anchor=north west,align=left] {\large Chain conditions, growth conditions, and other forms of finiteness for associative rings and algebras};
\draw (64.95, -81.3) rectangle (98.4,-88.2);
\draw(65.95, -82.3) node[anchor=north west,align=left] {Chain conditions\\  on\\ annihilators and\\ summands:\\ Goldie-type conditions};
\draw (65.95, -82.3) rectangle (72.05,-84.89999999999999);
\draw(72.15, -82.3) node[anchor=north west,align=left] {Noetherian\\ rings and\\ modules (associative\\  rings\\ and algebras)};
\draw (72.15, -82.3) rectangle (77.75,-84.89999999999999);
\draw(77.85, -82.3) node[anchor=north west,align=left] {Chain conditions\\ on other classes\\ of submodules,\\ ideals, subrings,\\ etc.; coherence\\  (associative\\ rings and algebras)};
\draw (77.85, -82.3) rectangle (83.19999999999999,-85.89999999999999);
\draw(83.30000000000001, -82.3) node[anchor=north west,align=left] {Finite rings\\  and\\ finite-dimensional\\ associative\\ algebras};
\draw (83.30000000000001, -82.3) rectangle (88.4,-84.89999999999999);
\draw(88.5, -82.3) node[anchor=north west,align=left] {Artinian\\ rings and\\ modules\\ (associative rings\\ and algebras)};
\draw (88.5, -82.3) rectangle (93.6,-84.89999999999999);
\draw(93.7, -82.3) node[anchor=north west,align=left] {Localization\\  and\\ associative\\ Noetherian rings};
\draw (93.7, -82.3) rectangle (98.3,-84.39999999999999);
\draw(65.95, -86.0) node[anchor=north west,align=left] {Growth\\ rate,\\ Gelfand-Kirillov\\ dimension};
\draw (65.95, -86.0) rectangle (70.55,-88.1);
\draw(98.5, -81.3) node[anchor=north west,align=left] {\large Associative rings and algebras with additional structure};
\draw (98.5, -81.3) rectangle (119.25,-90.39999999999999);
\draw(99.5, -82.3) node[anchor=north west,align=left] {Actions of\\ groups and\\ semigroups; invariant\\  theory\\ (associative\\ rings and algebras)};
\draw (99.5, -82.3) rectangle (105.35,-85.39999999999999);
\draw(105.45, -82.3) node[anchor=north west,align=left] {Valuations,\\ completions, formal\\ power series and\\ related constructions\\  (associative\\ rings and algebras)};
\draw (105.45, -82.3) rectangle (111.3,-85.39999999999999);
\draw(111.4, -82.3) node[anchor=north west,align=left] {Filtered\\ associative\\ rings; filtrational\\  and\\ graded techniques};
\draw (111.4, -82.3) rectangle (116.75,-84.89999999999999);
\draw(99.5, -85.5) node[anchor=north west,align=left] {Rings with\\ involution; Lie,\\ Jordan and other\\ nonassociative\\ structures};
\draw (99.5, -85.5) rectangle (104.1,-88.1);
\draw(104.2, -85.5) node[anchor=north west,align=left] {Automorphisms\\  and\\ endomorphisms};
\draw (104.2, -85.5) rectangle (108.05,-87.1);
\draw(108.15, -85.5) node[anchor=north west,align=left] {Graded rings\\ and modules\\ (associative\\  rings\\ and algebras)};
\draw (108.15, -85.5) rectangle (112.0,-88.1);
\draw(112.1, -85.5) node[anchor=north west,align=left] {Derivations,\\ actions\\ of Lie\\ algebras};
\draw (112.1, -85.5) rectangle (115.69999999999999,-87.6);
\draw(115.8, -85.5) node[anchor=north west,align=left] {“Super”\\ (or “skew”)\\ structure};
\draw (115.8, -85.5) rectangle (119.14999999999999,-87.1);
\draw(99.5, -88.2) node[anchor=north west,align=left] {Topological\\ and ordered\\  rings\\ and modules};
\draw (99.5, -88.2) rectangle (102.85,-90.3);
\draw(64.95, -88.3) node[anchor=north west,align=left] {\large History of associative\\ rings and algebras};
\draw (64.95, -88.3) rectangle (72.37,-89.39999999999999);
\draw(64.95, -90.5) node[anchor=north west,align=left] {\large Associative rings and algebras arising under various constructions};
\draw (64.95, -90.5) rectangle (89.75,-101.3);
\draw(65.95, -91.5) node[anchor=north west,align=left] {Rings arising\\  from\\ noncommutative algebraic\\ geometry};
\draw (65.95, -91.5) rectangle (72.55,-93.6);
\draw(72.65, -91.5) node[anchor=north west,align=left] {Associative rings\\ determined by\\ universal properties\\ (free algebras,\\ coproducts, adjunction\\ of inverses, etc.)};
\draw (72.65, -91.5) rectangle (78.75,-94.6);
\draw(78.85, -91.5) node[anchor=north west,align=left] {Associative\\ rings of\\ functions, subdirect\\  products,\\ sheaves of rings};
\draw (78.85, -91.5) rectangle (84.44999999999999,-94.1);
\draw(84.55000000000001, -91.5) node[anchor=north west,align=left] {Finite generation,\\  finite\\ presentability,\\ normal forms\\ (diamond lemma,\\ term-rewriting)};
\draw (84.55000000000001, -91.5) rectangle (89.65,-94.6);
\draw(65.95, -94.7) node[anchor=north west,align=left] {Rings of\\ differential\\ operators\\ (associative\\ algebraic aspects)};
\draw (65.95, -94.7) rectangle (71.05,-97.3);
\draw(71.15, -94.7) node[anchor=north west,align=left] {Torsion theories;\\  radicals on\\ module categories\\ (associative\\ algebraic aspects)};
\draw (71.15, -94.7) rectangle (76.25,-97.3);
\draw(76.35, -94.7) node[anchor=north west,align=left] {Twisted and\\ skew group\\  rings,\\ crossed products};
\draw (76.35, -94.7) rectangle (80.94999999999999,-96.8);
\draw(81.05, -94.7) node[anchor=north west,align=left] {Ordinary\\ and skew\\ polynomial\\ rings and\\ semigroup rings};
\draw (81.05, -94.7) rectangle (85.39999999999999,-97.3);
\draw(85.5, -94.7) node[anchor=north west,align=left] {Extensions\\ of associative\\ rings\\ by ideals};
\draw (85.5, -94.7) rectangle (89.6,-96.8);
\draw(65.95, -97.4) node[anchor=north west,align=left] {Associative\\ rings of\\ fractions and\\ localizations};
\draw (65.95, -97.4) rectangle (69.8,-99.5);
\draw(69.9, -97.4) node[anchor=north west,align=left] {Centralizing\\  and\\ normalizing\\ extensions};
\draw (69.9, -97.4) rectangle (73.5,-99.5);
\draw(73.60000000000001, -97.4) node[anchor=north west,align=left] {Universal\\ enveloping\\ algebras of\\ Lie algebras};
\draw (73.60000000000001, -97.4) rectangle (77.2,-99.5);
\draw(77.3, -97.4) node[anchor=north west,align=left] {Smash\\ products of\\ general\\ Hopf actions};
\draw (77.3, -97.4) rectangle (80.89999999999999,-99.5);
\draw(81.0, -97.4) node[anchor=north west,align=left] {Endomorphism\\  rings;\\ matrix rings};
\draw (81.0, -97.4) rectangle (84.6,-99.0);
\draw(84.7, -97.4) node[anchor=north west,align=left] {Deformations\\  of\\ associative\\ rings};
\draw (84.7, -97.4) rectangle (88.3,-99.5);
\draw(65.95, -99.6) node[anchor=north west,align=left] {Quadratic\\ and Koszul\\ algebras};
\draw (65.95, -99.6) rectangle (69.05,-101.19999999999999);
\draw(69.15, -99.6) node[anchor=north west,align=left] {Leavitt\\  path\\ algebras};
\draw (69.15, -99.6) rectangle (71.75,-101.19999999999999);
\draw(71.85000000000001, -99.6) node[anchor=north west,align=left] {Group\\ rings};
\draw (71.85000000000001, -99.6) rectangle (73.7,-100.69999999999999);
\draw(89.85000000000001, -90.5) node[anchor=north west,align=left] {\large Radicals and radical properties of associative rings};
\draw (89.85000000000001, -90.5) rectangle (109.15,-94.2);
\draw(90.85000000000001, -91.5) node[anchor=north west,align=left] {Jacobson\\ radical,\\ quasimultiplication};
\draw (90.85000000000001, -91.5) rectangle (96.2,-93.1);
\draw(96.30000000000001, -91.5) node[anchor=north west,align=left] {Nil and\\ nilpotent\\ radicals, sets,\\  ideals,\\ associative rings};
\draw (96.30000000000001, -91.5) rectangle (101.15,-94.1);
\draw(101.25, -91.5) node[anchor=north west,align=left] {General\\ radicals\\ and associative\\ rings};
\draw (101.25, -91.5) rectangle (105.6,-93.6);
\draw(105.7, -91.5) node[anchor=north west,align=left] {Prime and\\ semiprime\\ associative\\ rings};
\draw (105.7, -91.5) rectangle (109.05,-93.6);
\draw(89.85000000000001, -94.3) node[anchor=north west,align=left] {\large Representation theory of associative rings and algebras};
\draw (89.85000000000001, -94.3) rectangle (108.20000000000002,-100.2);
\draw(90.85000000000001, -95.3) node[anchor=north west,align=left] {Auslander-Reiten\\ sequences (almost\\ split sequences)\\  and\\ Auslander-Reiten quivers};
\draw (90.85000000000001, -95.3) rectangle (97.45,-97.89999999999999);
\draw(97.55000000000001, -95.3) node[anchor=north west,align=left] {Representation\\ type (finite,\\  tame, wild,\\ etc.) of associative\\ algebras};
\draw (97.55000000000001, -95.3) rectangle (103.15,-97.89999999999999);
\draw(103.25, -95.3) node[anchor=north west,align=left] {Representations\\ of orders,\\  lattices,\\ algebras over\\ commutative rings};
\draw (103.25, -95.3) rectangle (108.1,-97.89999999999999);
\draw(90.85000000000001, -98.0) node[anchor=north west,align=left] {Representations\\  of\\ associative\\ Artinian rings};
\draw (90.85000000000001, -98.0) rectangle (95.2,-100.1);
\draw(95.30000000000001, -98.0) node[anchor=north west,align=left] {Representations\\ of quivers\\ and partially\\ ordered sets};
\draw (95.30000000000001, -98.0) rectangle (99.65,-100.1);
\draw(99.75, -98.0) node[anchor=north west,align=left] {Cohen-Macaulay\\ modules\\ in associative\\ algebras};
\draw (99.75, -98.0) rectangle (103.85,-100.1);
\draw(109.25, -90.5) node[anchor=north west,align=left] {\large Rings with polynomial identity};
\draw (109.25, -90.5) rectangle (120.4,-99.6);
\draw(110.25, -91.5) node[anchor=north west,align=left] {Other kinds of\\  identities\\ (generalized\\ polynomial, rational,\\ involution)};
\draw (110.25, -91.5) rectangle (116.1,-94.1);
\draw(116.2, -91.5) node[anchor=north west,align=left] {Functional\\ identities\\ (associative\\ rings\\ and algebras)};
\draw (116.2, -91.5) rectangle (120.05,-94.1);
\draw(110.25, -94.2) node[anchor=north west,align=left] {Trace rings\\ and invariant\\ theory\\ (associative rings\\ and algebras)};
\draw (110.25, -94.2) rectangle (115.35,-96.8);
\draw(115.45, -94.2) node[anchor=north west,align=left] {\(T\)-ideals,\\ identities,\\ varieties of\\ associative rings\\ and algebras};
\draw (115.45, -94.2) rectangle (120.3,-96.8);
\draw(110.25, -96.9) node[anchor=north west,align=left] {Semiprime p.i.\\  rings, rings\\ embeddable in\\ matrices over\\ commutative rings};
\draw (110.25, -96.9) rectangle (115.1,-99.5);
\draw(115.2, -96.9) node[anchor=north west,align=left] {Identities\\ other than\\ those of matrices\\  over\\ commutative rings};
\draw (115.2, -96.9) rectangle (120.05,-99.5);
\draw(64.95, -101.4) node[anchor=north west,align=left] {\large Modules, bimodules and ideals in associative algebras};
\draw (64.95, -101.4) rectangle (84.05000000000001,-112.0);
\draw(65.95, -102.4) node[anchor=north west,align=left] {Structure and\\ classification for\\ modules, bimodules and\\  ideals (except\\ as in 16Gxx), direct\\ sum decomposition\\ and cancellation\\ in associative algebras)};
\draw (65.95, -102.4) rectangle (72.55,-106.5);
\draw(72.65, -102.4) node[anchor=north west,align=left] {Infinite-dimensional\\  simple\\ rings (except\\ as in 16Kxx)};
\draw (72.65, -102.4) rectangle (78.25,-104.5);
\draw(72.65, -104.60000000000001) node[anchor=north west,align=left] {Ideals in\\ associative\\ algebras};
\draw (72.65, -104.60000000000001) rectangle (76.0,-106.2);
\draw(78.35, -102.4) node[anchor=north west,align=left] {Free, projective,\\ and flat\\  modules and\\ ideals in\\ associative algebras};
\draw (78.35, -102.4) rectangle (83.94999999999999,-105.0);
\draw(65.95, -106.60000000000001) node[anchor=north west,align=left] {Simple and\\ semisimple\\ modules, primitive\\  rings and\\ ideals in\\ associative algebras};
\draw (65.95, -106.60000000000001) rectangle (71.55,-109.7);
\draw(71.65, -106.60000000000001) node[anchor=north west,align=left] {Other classes\\ of modules\\ and ideals\\  in\\ associative algebras};
\draw (71.65, -106.60000000000001) rectangle (77.25,-109.2);
\draw(77.35, -106.60000000000001) node[anchor=north west,align=left] {Injective\\ modules,\\ self-injective\\ associative rings};
\draw (77.35, -106.60000000000001) rectangle (82.19999999999999,-108.7);
\draw(65.95, -109.80000000000001) node[anchor=north west,align=left] {General\\ module theory\\ in associative\\ algebras};
\draw (65.95, -109.80000000000001) rectangle (70.05,-111.9);
\draw(70.15, -109.80000000000001) node[anchor=north west,align=left] {Module\\ categories in\\ associative\\ algebras};
\draw (70.15, -109.80000000000001) rectangle (74.0,-111.9);
\draw(74.10000000000001, -109.80000000000001) node[anchor=north west,align=left] {Bimodules\\  in\\ associative\\ algebras};
\draw (74.10000000000001, -109.80000000000001) rectangle (77.45,-111.9);
\draw(84.15, -101.4) node[anchor=north west,align=left] {\large Hopf algebras, quantum groups and related topics};
\draw (84.15, -101.4) rectangle (100.45,-106.80000000000001);
\draw(85.15, -102.4) node[anchor=north west,align=left] {Ring-theoretic\\ aspects\\ of quantum\\ groups};
\draw (85.15, -102.4) rectangle (89.25,-104.5);
\draw(89.35000000000001, -102.4) node[anchor=north west,align=left] {Connections\\  of Hopf\\ algebras with\\ combinatorics};
\draw (89.35000000000001, -102.4) rectangle (93.2,-104.5);
\draw(93.30000000000001, -102.4) node[anchor=north west,align=left] {Hopf\\ algebras and\\ their\\ applications};
\draw (93.30000000000001, -102.4) rectangle (96.9,-104.5);
\draw(97.0, -102.4) node[anchor=north west,align=left] {Yang-Baxter\\ equations};
\draw (97.0, -102.4) rectangle (100.35,-103.5);
\draw(97.0, -103.60000000000001) node[anchor=north west,align=left] {Bialgebras};
\draw (97.0, -103.60000000000001) rectangle (100.1,-104.2);
\draw(85.15, -104.60000000000001) node[anchor=north west,align=left] {Coalgebras\\  and\\ comodules;\\ corings};
\draw (85.15, -104.60000000000001) rectangle (88.25,-106.7);
\draw(84.15, -106.9) node[anchor=north west,align=left] {\large Division rings and semisimple Artin rings};
\draw (84.15, -106.9) rectangle (99.25,-110.10000000000001);
\draw(85.15, -107.9) node[anchor=north west,align=left] {Infinite-dimensional\\ and general\\ division rings};
\draw (85.15, -107.9) rectangle (90.75,-109.5);
\draw(90.85000000000001, -107.9) node[anchor=north west,align=left] {Finite-dimensional\\ division\\ rings};
\draw (90.85000000000001, -107.9) rectangle (95.95,-109.5);
\draw(96.05000000000001, -107.9) node[anchor=north west,align=left] {Brauer\\ groups\\ (algebraic\\ aspects)};
\draw (96.05000000000001, -107.9) rectangle (99.15,-110.0);
\draw(100.55000000000001, -101.4) node[anchor=north west,align=left] {\large Homological methods in associative algebras};
\draw (100.55000000000001, -101.4) rectangle (116.65,-113.7);
\draw(101.55000000000001, -102.4) node[anchor=north west,align=left] {Homological conditions\\ on associative\\ rings (generalizations\\ of regular,\\  Gorenstein,\\ Cohen-Macaulay rings, etc.)};
\draw (101.55000000000001, -102.4) rectangle (108.9,-105.5);
\draw(109.00000000000001, -102.4) node[anchor=north west,align=left] {Homological\\ functors on modules\\ (Tor, Ext,\\  etc.) in\\ associative algebras};
\draw (109.00000000000001, -102.4) rectangle (114.60000000000001,-105.0);
\draw(101.55000000000001, -105.60000000000001) node[anchor=north west,align=left] {(Co)homology\\ of rings and\\ associative\\ algebras (e.g.,\\ Hochschild, cyclic,\\ dihedral, etc.)};
\draw (101.55000000000001, -105.60000000000001) rectangle (106.9,-108.7);
\draw(107.00000000000001, -105.60000000000001) node[anchor=north west,align=left] {Differential\\ graded algebras\\ and applications\\  (associative\\ algebraic aspects)};
\draw (107.00000000000001, -105.60000000000001) rectangle (112.10000000000001,-108.2);
\draw(112.20000000000002, -105.60000000000001) node[anchor=north west,align=left] {Derived\\ categories\\ and associative\\ algebras};
\draw (112.20000000000002, -105.60000000000001) rectangle (116.55000000000001,-107.7);
\draw(101.55000000000001, -108.80000000000001) node[anchor=north west,align=left] {von Neumann\\ regular rings and\\ generalizations\\  (associative\\ algebraic aspects)};
\draw (101.55000000000001, -108.80000000000001) rectangle (106.65,-111.4);
\draw(106.75000000000001, -108.80000000000001) node[anchor=north west,align=left] {Semihereditary\\ and hereditary\\ rings, free ideal\\ rings, Sylvester\\ rings, etc.};
\draw (106.75000000000001, -108.80000000000001) rectangle (111.60000000000001,-111.4);
\draw(111.70000000000002, -108.80000000000001) node[anchor=north west,align=left] {Syzygies,\\ resolutions,\\  complexes\\ in associative\\ algebras};
\draw (111.70000000000002, -108.80000000000001) rectangle (115.80000000000001,-111.4);
\draw(101.55000000000001, -111.5) node[anchor=north west,align=left] {Homological\\  dimension\\ in associative\\ algebras};
\draw (101.55000000000001, -111.5) rectangle (105.65,-113.6);
\draw(105.75000000000001, -111.5) node[anchor=north west,align=left] {Grothendieck\\ groups,\\ \(K\)-theory, etc.};
\draw (105.75000000000001, -111.5) rectangle (109.85000000000001,-113.1);
\draw(116.75, -101.4) node[anchor=north west,align=left] {\large Generalizations};
\draw (116.75, -101.4) rectangle (122.0,-106.2);
\draw(117.75, -102.4) node[anchor=north west,align=left] {Hyperrings};
\draw (117.75, -102.4) rectangle (120.85,-103.0);
\draw(117.75, -103.10000000000001) node[anchor=north west,align=left] {Near-rings};
\draw (117.75, -103.10000000000001) rectangle (120.85,-103.7);
\draw(117.75, -103.80000000000001) node[anchor=north west,align=left] {\(\Gamma\) and\\ fuzzy\\ structures};
\draw (117.75, -103.80000000000001) rectangle (120.85,-105.4);
\draw(117.75, -105.5) node[anchor=north west,align=left] {Semirings};
\draw (117.75, -105.5) rectangle (120.6,-106.1);
\draw(64.95, -113.8) node[anchor=north west,align=left] {\large Computational aspects of associative rings};
\draw (64.95, -113.8) rectangle (78.57000000000001,-117.5);
\draw(65.95, -114.8) node[anchor=north west,align=left] {Computational\\  aspects\\ of associative\\  rings\\ (general theory)};
\draw (65.95, -114.8) rectangle (70.55,-117.39999999999999);
\draw(70.65, -114.8) node[anchor=north west,align=left] {Gröbner-Shirshov\\ bases};
\draw (70.65, -114.8) rectangle (75.25,-115.89999999999999);
\draw(78.67, -113.8) node[anchor=north west,align=left] {\large Conditions on elements};
\draw (78.67, -113.8) rectangle (89.57,-126.3);
\draw(79.67, -114.8) node[anchor=north west,align=left] {Integral\\ domains (associative\\  rings\\ and algebras)};
\draw (79.67, -114.8) rectangle (85.27,-116.89999999999999);
\draw(85.37, -114.8) node[anchor=north west,align=left] {Ore rings,\\ multiplicative\\  sets, Ore\\ localization};
\draw (85.37, -114.8) rectangle (89.47,-116.89999999999999);
\draw(79.67, -117.0) node[anchor=north west,align=left] {Idempotent\\ elements\\ (associative rings\\ and algebras)};
\draw (79.67, -117.0) rectangle (84.77,-119.1);
\draw(84.87, -117.0) node[anchor=north west,align=left] {Divisibility,\\ noncommutative\\ UFDs};
\draw (84.87, -117.0) rectangle (88.97,-118.6);
\draw(79.67, -119.2) node[anchor=north west,align=left] {Center, normalizer\\ (invariant\\  elements)\\ (associative rings\\ and algebras)};
\draw (79.67, -119.2) rectangle (84.77,-121.8);
\draw(84.87, -119.2) node[anchor=north west,align=left] {Units, groups\\ of units\\ (associative\\ rings and\\ algebras)};
\draw (84.87, -119.2) rectangle (88.72,-121.8);
\draw(79.67, -121.9) node[anchor=north west,align=left] {Generalizations\\ of commutativity\\ (associative rings\\ and algebras)};
\draw (79.67, -121.9) rectangle (84.77,-124.0);
\draw(79.67, -124.1) node[anchor=north west,align=left] {Generalized\\ inverses\\ (associative rings\\ and algebras)};
\draw (79.67, -124.1) rectangle (84.77,-126.19999999999999);
\draw(64.95, -117.6) node[anchor=north west,align=left] {\large Associative algebras and orders};
\draw (64.95, -117.6) rectangle (75.16,-123.0);
\draw(65.95, -118.6) node[anchor=north west,align=left] {Separable\\ algebras (e.g.,\\ quaternion algebras,\\  Azumaya\\ algebras, etc.)};
\draw (65.95, -118.6) rectangle (71.55,-121.19999999999999);
\draw(71.65, -118.6) node[anchor=north west,align=left] {Commutative\\ orders};
\draw (71.65, -118.6) rectangle (75.0,-119.69999999999999);
\draw(65.95, -121.3) node[anchor=north west,align=left] {Orders in\\ separable\\ algebras};
\draw (65.95, -121.3) rectangle (68.8,-122.89999999999999);
\draw(68.9, -121.3) node[anchor=north west,align=left] {Lattices\\ over\\ orders};
\draw (68.9, -121.3) rectangle (71.5,-122.89999999999999);
\draw(89.67, -113.8) node[anchor=north west,align=left] {\large Local rings and generalizations};
\draw (89.67, -113.8) rectangle (99.88,-117.5);
\draw(90.67, -114.8) node[anchor=north west,align=left] {Quasi-Frobenius\\ rings};
\draw (90.67, -114.8) rectangle (95.02,-115.89999999999999);
\draw(95.12, -114.8) node[anchor=north west,align=left] {Noncommutative\\  local\\ and semilocal\\  rings,\\ perfect rings};
\draw (95.12, -114.8) rectangle (99.22,-117.39999999999999);
\draw(89.67, -117.6) node[anchor=north west,align=left] {\large General and miscellaneous};
\draw (89.67, -117.6) rectangle (98.02,-124.0);
\draw(90.67, -118.6) node[anchor=north west,align=left] {Category-theoretic\\ methods\\  and results\\ in associative\\  algebras\\ (except as in 16D90)};
\draw (90.67, -118.6) rectangle (96.27,-121.69999999999999);
\draw(90.67, -121.8) node[anchor=north west,align=left] {Applications\\  of logic\\ in associative\\ algebras};
\draw (90.67, -121.8) rectangle (94.77,-123.89999999999999);
\draw(136.17, -1) node[anchor=north west,align=left] {\LARGE Functions of a complex variable};
\draw (136.17, -1) rectangle (187.57,-47.2);
\draw(137.17, -2) node[anchor=north west,align=left] {\large Entire and meromorphic functions of one complex variable, and related topics};
\draw (137.17, -2) rectangle (165.17,-9.4);
\draw(138.17, -3) node[anchor=north west,align=left] {Functional equations\\ in the complex\\ plane, iteration\\ and composition\\ of analytic\\ functions of one\\ complex variable};
\draw (138.17, -3) rectangle (143.76999999999998,-6.6);
\draw(143.86999999999998, -3) node[anchor=north west,align=left] {Representations\\  of entire\\ functions of\\ one complex\\ variable by\\ series and integrals};
\draw (143.86999999999998, -3) rectangle (149.46999999999997,-6.1);
\draw(149.57, -3) node[anchor=north west,align=left] {Special classes\\ of entire functions\\ of one complex\\  variable and\\ growth estimates};
\draw (149.57, -3) rectangle (154.92,-5.6);
\draw(155.01999999999998, -3) node[anchor=north west,align=left] {Value distribution\\ of meromorphic\\ functions\\ of one complex\\  variable,\\ Nevanlinna theory};
\draw (155.01999999999998, -3) rectangle (160.11999999999998,-6.1);
\draw(160.22, -3) node[anchor=north west,align=left] {Cluster\\ sets, prime\\ ends,\\ boundary behavior};
\draw (160.22, -3) rectangle (165.07,-5.1);
\draw(138.17, -6.7) node[anchor=north west,align=left] {Entire\\ functions of one\\  complex\\ variable,\\ general theory};
\draw (138.17, -6.7) rectangle (142.76999999999998,-9.3);
\draw(142.86999999999998, -6.7) node[anchor=north west,align=left] {Normal\\ functions of one\\  complex\\ variable,\\ normal families};
\draw (142.86999999999998, -6.7) rectangle (147.46999999999997,-9.3);
\draw(147.57, -6.7) node[anchor=north west,align=left] {Quasi-analytic\\  and other\\ classes of\\ functions of one\\ complex variable};
\draw (147.57, -6.7) rectangle (152.17,-9.3);
\draw(152.26999999999998, -6.7) node[anchor=north west,align=left] {Meromorphic\\ functions of\\ one complex\\  variable,\\ general theory};
\draw (152.26999999999998, -6.7) rectangle (156.36999999999998,-9.3);
\draw(165.26999999999998, -2) node[anchor=north west,align=left] {\large Series expansions of functions of one complex variable};
\draw (165.26999999999998, -2) rectangle (185.07,-8.4);
\draw(166.26999999999998, -3) node[anchor=north west,align=left] {Boundary behavior\\  of power\\ series in one complex\\  variable;\\ over-convergence};
\draw (166.26999999999998, -3) rectangle (172.11999999999998,-5.6);
\draw(172.21999999999997, -3) node[anchor=north west,align=left] {Completeness\\ problems, closure\\ of a system of\\  functions of\\ one complex variable};
\draw (172.21999999999997, -3) rectangle (177.81999999999996,-5.6);
\draw(177.92, -3) node[anchor=north west,align=left] {Dirichlet series,\\  exponential\\ series and other\\ series in one\\ complex variable};
\draw (177.92, -3) rectangle (182.76999999999998,-5.6);
\draw(166.26999999999998, -5.7) node[anchor=north west,align=left] {Power series\\  (including\\ lacunary\\ series) in one\\ complex variable};
\draw (166.26999999999998, -5.7) rectangle (170.86999999999998,-8.3);
\draw(170.96999999999997, -5.7) node[anchor=north west,align=left] {Random power\\  series\\ in one\\ complex variable};
\draw (170.96999999999997, -5.7) rectangle (175.56999999999996,-7.800000000000001);
\draw(175.67, -5.7) node[anchor=north west,align=left] {Analytic\\ continuation\\ of functions\\  of one\\ complex variable};
\draw (175.67, -5.7) rectangle (180.26999999999998,-8.3);
\draw(180.36999999999998, -5.7) node[anchor=north west,align=left] {Continued\\ fractions;\\ complex-analytic\\ aspects};
\draw (180.36999999999998, -5.7) rectangle (184.96999999999997,-7.800000000000001);
\draw(137.17, -9.5) node[anchor=north west,align=left] {\large Spaces and algebras of analytic functions of one complex variable};
\draw (137.17, -9.5) rectangle (161.62,-14.9);
\draw(138.17, -10.5) node[anchor=north west,align=left] {Spaces of\\ bounded analytic\\ functions\\ of one complex\\ variable};
\draw (138.17, -10.5) rectangle (142.76999999999998,-13.1);
\draw(142.86999999999998, -10.5) node[anchor=north west,align=left] {Algebras\\ of analytic\\ functions\\  of one\\ complex variable};
\draw (142.86999999999998, -10.5) rectangle (147.46999999999997,-13.1);
\draw(147.57, -10.5) node[anchor=north west,align=left] {de\\ Branges-Rovnyak\\ spaces};
\draw (147.57, -10.5) rectangle (151.92,-12.1);
\draw(147.57, -12.2) node[anchor=north west,align=left] {BMO-spaces};
\draw (147.57, -12.2) rectangle (150.67,-12.799999999999999);
\draw(152.01999999999998, -10.5) node[anchor=north west,align=left] {Nevanlinna\\  spaces\\ and Smirnov\\ spaces};
\draw (152.01999999999998, -10.5) rectangle (155.36999999999998,-12.6);
\draw(155.47, -10.5) node[anchor=north west,align=left] {Bergman\\ spaces and\\ Fock spaces};
\draw (155.47, -10.5) rectangle (158.82,-12.1);
\draw(158.92, -10.5) node[anchor=north west,align=left] {Corona\\ theorems};
\draw (158.92, -10.5) rectangle (161.51999999999998,-11.6);
\draw(138.17, -13.2) node[anchor=north west,align=left] {Besov\\ spaces and\\ \(Q_p\)-spaces};
\draw (138.17, -13.2) rectangle (141.26999999999998,-14.799999999999999);
\draw(141.36999999999998, -13.2) node[anchor=north west,align=left] {Zygmund\\ spaces};
\draw (141.36999999999998, -13.2) rectangle (143.71999999999997,-14.299999999999999);
\draw(143.82, -13.2) node[anchor=north west,align=left] {Hardy\\ spaces};
\draw (143.82, -13.2) rectangle (145.92,-14.299999999999999);
\draw(146.01999999999998, -13.2) node[anchor=north west,align=left] {Bloch\\ spaces};
\draw (146.01999999999998, -13.2) rectangle (148.11999999999998,-14.299999999999999);
\draw(161.72, -9.5) node[anchor=north west,align=left] {\large Miscellaneous topics of analysis in the complex plane};
\draw (161.72, -9.5) rectangle (181.51999999999998,-15.9);
\draw(162.72, -10.5) node[anchor=north west,align=left] {Integration,\\ integrals of Cauchy\\ type, integral\\ representations of\\ analytic functions\\ in the complex plane};
\draw (162.72, -10.5) rectangle (168.32,-13.6);
\draw(168.42, -10.5) node[anchor=north west,align=left] {Moment problems\\  and\\ interpolation\\ problems in the\\ complex plane};
\draw (168.42, -10.5) rectangle (172.76999999999998,-13.1);
\draw(172.87, -10.5) node[anchor=north west,align=left] {Asymptotic\\ representations\\ in the\\ complex plane};
\draw (172.87, -10.5) rectangle (177.22,-12.6);
\draw(177.32, -10.5) node[anchor=north west,align=left] {Boundary\\ value problems\\  in the\\ complex plane};
\draw (177.32, -10.5) rectangle (181.42,-12.6);
\draw(162.72, -13.7) node[anchor=north west,align=left] {Approximation\\ in\\  the\\ complex plane};
\draw (162.72, -13.7) rectangle (166.57,-15.799999999999999);
\draw(181.62, -9.5) node[anchor=north west,align=left] {\large History of\\ functions of a\\ complex variable};
\draw (181.62, -9.5) rectangle (187.18,-11.1);
\draw(137.17, -16.0) node[anchor=north west,align=left] {\large Universal holomorphic functions of one complex variable};
\draw (137.17, -16.0) rectangle (156.22,-19.2);
\draw(138.17, -17.0) node[anchor=north west,align=left] {Universal\\ Taylor series\\  in one\\ complex variable};
\draw (138.17, -17.0) rectangle (142.76999999999998,-19.1);
\draw(142.86999999999998, -17.0) node[anchor=north west,align=left] {Universal\\ Dirichlet series\\  in one\\ complex variable};
\draw (142.86999999999998, -17.0) rectangle (147.46999999999997,-19.1);
\draw(147.57, -17.0) node[anchor=north west,align=left] {Universal\\ functions\\ of one\\ complex variable};
\draw (147.57, -17.0) rectangle (152.17,-19.1);
\draw(152.26999999999998, -17.0) node[anchor=north west,align=left] {Compositional\\ universality};
\draw (152.26999999999998, -17.0) rectangle (156.11999999999998,-18.1);
\draw(156.32, -16.0) node[anchor=north west,align=left] {\large General properties of functions of one complex variable};
\draw (156.32, -16.0) rectangle (173.97,-19.7);
\draw(157.32, -17.0) node[anchor=north west,align=left] {Monogenic\\ and polygenic\\ functions\\  of one\\ complex variable};
\draw (157.32, -17.0) rectangle (161.92,-19.6);
\draw(162.01999999999998, -17.0) node[anchor=north west,align=left] {Inequalities\\  in\\ the complex\\ plane};
\draw (162.01999999999998, -17.0) rectangle (165.61999999999998,-19.1);
\draw(174.07, -16.0) node[anchor=north west,align=left] {\large Geometric function theory};
\draw (174.07, -16.0) rectangle (187.47,-37.400000000000006);
\draw(175.07, -17.0) node[anchor=north west,align=left] {Schwarz-Christoffel-type\\ mappings};
\draw (175.07, -17.0) rectangle (181.67,-18.1);
\draw(181.76999999999998, -17.0) node[anchor=north west,align=left] {Maximum principle,\\  Schwarz’s\\ lemma, Lindelöf\\ principle, analogues\\ and generalizations;\\ subordination};
\draw (181.76999999999998, -17.0) rectangle (187.36999999999998,-20.1);
\draw(175.07, -20.2) node[anchor=north west,align=left] {Extremal problems\\ for conformal\\  and\\ quasiconformal mappings,\\ other methods};
\draw (175.07, -20.2) rectangle (181.67,-22.8);
\draw(181.76999999999998, -20.2) node[anchor=north west,align=left] {Extremal problems\\  for conformal\\ and quasiconformal\\  mappings,\\ variational methods};
\draw (181.76999999999998, -20.2) rectangle (187.11999999999998,-22.8);
\draw(175.07, -22.9) node[anchor=north west,align=left] {Zeros of polynomials,\\ rational functions,\\ and other analytic\\ functions of one\\  complex variable\\ (e.g., zeros of\\ functions with bounded\\ Dirichlet integral)};
\draw (175.07, -22.9) rectangle (181.17,-27.0);
\draw(181.26999999999998, -22.9) node[anchor=north west,align=left] {Special classes\\ of univalent and\\ multivalent functions\\ of one complex\\ variable (starlike,\\ convex, bounded\\ rotation, etc.)};
\draw (181.26999999999998, -22.9) rectangle (187.11999999999998,-26.5);
\draw(175.07, -27.1) node[anchor=north west,align=left] {Coefficient\\ problems for\\ univalent and\\ multivalent functions\\  of one\\ complex variable};
\draw (175.07, -27.1) rectangle (180.92,-30.200000000000003);
\draw(181.01999999999998, -27.1) node[anchor=north west,align=left] {Polynomials\\ and rational\\ functions\\  of one\\ complex variable};
\draw (181.01999999999998, -27.1) rectangle (185.61999999999998,-29.700000000000003);
\draw(175.07, -30.3) node[anchor=north west,align=left] {Kernel\\ functions in one\\  complex\\ variable and\\ applications};
\draw (175.07, -30.3) rectangle (179.67,-32.9);
\draw(179.76999999999998, -30.3) node[anchor=north west,align=left] {General theory\\ of univalent and\\  multivalent\\ functions of one\\ complex variable};
\draw (179.76999999999998, -30.3) rectangle (184.36999999999998,-32.9);
\draw(184.47, -30.3) node[anchor=north west,align=left] {General\\ theory of\\ conformal\\ mappings};
\draw (184.47, -30.3) rectangle (187.32,-32.4);
\draw(175.07, -33.0) node[anchor=north west,align=left] {Quasiconformal\\  mappings\\ in \(\mathbb{R}^n\), other\\ generalizations};
\draw (175.07, -33.0) rectangle (179.42,-35.1);
\draw(179.51999999999998, -33.0) node[anchor=north west,align=left] {Covering\\ theorems in\\ conformal\\ mapping theory};
\draw (179.51999999999998, -33.0) rectangle (183.61999999999998,-35.1);
\draw(183.72, -33.0) node[anchor=north west,align=left] {Conformal\\ mappings\\ of special\\ domains};
\draw (183.72, -33.0) rectangle (186.82,-35.1);
\draw(175.07, -35.2) node[anchor=north west,align=left] {Quasiconformal\\ mappings\\  in the\\ complex plane};
\draw (175.07, -35.2) rectangle (179.17,-37.300000000000004);
\draw(179.26999999999998, -35.2) node[anchor=north west,align=left] {Capacity and\\  harmonic\\ measure in the\\ complex plane};
\draw (179.26999999999998, -35.2) rectangle (183.36999999999998,-37.300000000000004);
\draw(156.32, -19.8) node[anchor=north west,align=left] {\large Riemann surfaces};
\draw (156.32, -19.8) rectangle (167.97,-32.3);
\draw(157.32, -20.8) node[anchor=north west,align=left] {Conformal\\ metrics\\ (hyperbolic, Poincaré,\\ distance\\ functions)};
\draw (157.32, -20.8) rectangle (163.42,-23.400000000000002);
\draw(163.51999999999998, -20.8) node[anchor=north west,align=left] {Ideal\\ boundary theory\\ for Riemann\\ surfaces};
\draw (163.51999999999998, -20.8) rectangle (167.86999999999998,-22.900000000000002);
\draw(157.32, -23.5) node[anchor=north west,align=left] {Fuchsian groups\\ and automorphic\\ functions (aspects\\  of compact\\ Riemann surfaces\\ and uniformization)};
\draw (157.32, -23.5) rectangle (162.67,-26.6);
\draw(162.76999999999998, -23.5) node[anchor=north west,align=left] {Kleinian groups\\  (aspects of\\ compact Riemann\\ surfaces and\\ uniformization)};
\draw (162.76999999999998, -23.5) rectangle (167.11999999999998,-26.1);
\draw(157.32, -26.700000000000003) node[anchor=north west,align=left] {Compact\\ Riemann\\ surfaces and\\ uniformization};
\draw (157.32, -26.700000000000003) rectangle (161.42,-28.800000000000004);
\draw(161.51999999999998, -26.700000000000003) node[anchor=north west,align=left] {Classification\\ theory\\ of Riemann\\ surfaces};
\draw (161.51999999999998, -26.700000000000003) rectangle (165.61999999999998,-28.800000000000004);
\draw(157.32, -28.900000000000002) node[anchor=north west,align=left] {Differentials\\ on\\ Riemann\\ surfaces};
\draw (157.32, -28.900000000000002) rectangle (161.17,-31.000000000000004);
\draw(161.26999999999998, -28.900000000000002) node[anchor=north west,align=left] {Teichmüller\\ theory\\ for Riemann\\ surfaces};
\draw (161.26999999999998, -28.900000000000002) rectangle (164.61999999999998,-31.000000000000004);
\draw(164.72, -28.900000000000002) node[anchor=north west,align=left] {Harmonic\\ functions\\ on Riemann\\ surfaces};
\draw (164.72, -28.900000000000002) rectangle (167.82,-31.000000000000004);
\draw(157.32, -31.1) node[anchor=north west,align=left] {Klein\\ surfaces};
\draw (157.32, -31.1) rectangle (159.92,-32.2);
\draw(156.32, -32.400000000000006) node[anchor=north west,align=left] {\large Computational methods for\\ problems pertaining to\\ functions of a complex variable};
\draw (156.32, -32.400000000000006) rectangle (166.53,-34.00000000000001);
\draw(137.17, -37.5) node[anchor=north west,align=left] {\large Generalized function theory};
\draw (137.17, -37.5) rectangle (148.32,-47.1);
\draw(138.17, -38.5) node[anchor=north west,align=left] {Finely\\ holomorphic functions\\  and\\ topological\\ function theory};
\draw (138.17, -38.5) rectangle (144.01999999999998,-41.1);
\draw(144.11999999999998, -38.5) node[anchor=north west,align=left] {Discrete\\ analytic\\ functions};
\draw (144.11999999999998, -38.5) rectangle (146.96999999999997,-40.1);
\draw(138.17, -41.2) node[anchor=north west,align=left] {Other\\ generalizations of\\ analytic functions\\ (including\\ abstract-valued\\ functions)};
\draw (138.17, -41.2) rectangle (143.26999999999998,-44.300000000000004);
\draw(143.36999999999998, -41.2) node[anchor=north west,align=left] {Generalizations\\  of Bers and\\ Vekua type\\ (pseudoanalytic,\\ \(p\)-analytic, etc.)};
\draw (143.36999999999998, -41.2) rectangle (148.21999999999997,-43.800000000000004);
\draw(138.17, -44.4) node[anchor=north west,align=left] {Non-Archimedean\\ function theory};
\draw (138.17, -44.4) rectangle (142.51999999999998,-45.5);
\draw(142.61999999999998, -44.4) node[anchor=north west,align=left] {Functions of\\ hypercomplex\\  variables\\ and generalized\\ variables};
\draw (142.61999999999998, -44.4) rectangle (146.96999999999997,-47.0);
\draw(148.42, -37.5) node[anchor=north west,align=left] {\large Analysis on metric spaces};
\draw (148.42, -37.5) rectangle (157.57,-42.4);
\draw(149.42, -38.5) node[anchor=north west,align=left] {Quasiconformal\\ mappings\\  in\\ metric spaces};
\draw (149.42, -38.5) rectangle (153.51999999999998,-40.6);
\draw(153.61999999999998, -38.5) node[anchor=north west,align=left] {Geometric\\ embeddings\\  of\\ metric spaces};
\draw (153.61999999999998, -38.5) rectangle (157.46999999999997,-40.6);
\draw(149.42, -40.7) node[anchor=north west,align=left] {Inequalities\\  in\\ metric spaces};
\draw (149.42, -40.7) rectangle (153.26999999999998,-42.300000000000004);
\draw(148.42, -42.5) node[anchor=north west,align=left] {\large Function theory on the disc};
\draw (148.42, -42.5) rectangle (157.39,-46.9);
\draw(149.42, -43.5) node[anchor=north west,align=left] {Singular inner\\ functions\\ of one complex\\ variable};
\draw (149.42, -43.5) rectangle (153.51999999999998,-45.6);
\draw(153.61999999999998, -43.5) node[anchor=north west,align=left] {Inner\\ functions of\\ one complex\\ variable};
\draw (153.61999999999998, -43.5) rectangle (157.21999999999997,-45.6);
\draw(149.42, -45.7) node[anchor=north west,align=left] {Blaschke\\ products};
\draw (149.42, -45.7) rectangle (152.01999999999998,-46.800000000000004);
\draw(136.17, -47.300000000000004) node[anchor=north west,align=left] {\LARGE Number theory};
\draw (136.17, -47.300000000000004) rectangle (183.32,-140.10000000000002);
\draw(137.17, -48.300000000000004) node[anchor=north west,align=left] {\large Probabilistic theory: distribution modulo \(1\); metric theory of algorithms};
\draw (137.17, -48.300000000000004) rectangle (163.32,-55.2);
\draw(138.17, -49.300000000000004) node[anchor=north west,align=left] {Normal numbers,\\ radix expansions,\\  Pisot\\ numbers, Salem\\ numbers, good\\ lattice points, etc.};
\draw (138.17, -49.300000000000004) rectangle (143.76999999999998,-52.400000000000006);
\draw(143.86999999999998, -49.300000000000004) node[anchor=north west,align=left] {Metric theory\\  of other\\ algorithms and\\ expansions;\\ measure and\\ Hausdorff dimension};
\draw (143.86999999999998, -49.300000000000004) rectangle (149.21999999999997,-52.400000000000006);
\draw(149.32, -49.300000000000004) node[anchor=north west,align=left] {Harmonic analysis\\ and almost\\  periodicity\\ in probabilistic\\ number theory};
\draw (149.32, -49.300000000000004) rectangle (154.17,-51.900000000000006);
\draw(154.26999999999998, -49.300000000000004) node[anchor=north west,align=left] {Well-distributed\\ sequences\\  and\\ other variations};
\draw (154.26999999999998, -49.300000000000004) rectangle (158.86999999999998,-51.400000000000006);
\draw(158.97, -49.300000000000004) node[anchor=north west,align=left] {Irregularities\\  of\\ distribution,\\ discrepancy};
\draw (158.97, -49.300000000000004) rectangle (163.07,-51.400000000000006);
\draw(138.17, -52.50000000000001) node[anchor=north west,align=left] {Pseudo-random\\ numbers;\\  Monte\\ Carlo methods};
\draw (138.17, -52.50000000000001) rectangle (142.01999999999998,-54.60000000000001);
\draw(142.11999999999998, -52.50000000000001) node[anchor=north west,align=left] {Diophantine\\ approximation\\  in\\ probabilistic\\ number theory};
\draw (142.11999999999998, -52.50000000000001) rectangle (145.96999999999997,-55.10000000000001);
\draw(146.07, -52.50000000000001) node[anchor=north west,align=left] {Arithmetic\\ functions in\\ probabilistic\\ number theory};
\draw (146.07, -52.50000000000001) rectangle (149.92,-54.60000000000001);
\draw(150.01999999999998, -52.50000000000001) node[anchor=north west,align=left] {General\\ theory of\\ distribution\\ modulo \(1\)};
\draw (150.01999999999998, -52.50000000000001) rectangle (153.61999999999998,-54.60000000000001);
\draw(153.72, -52.50000000000001) node[anchor=north west,align=left] {Continuous,\\  \(p\)-adic\\ and abstract\\ analogues};
\draw (153.72, -52.50000000000001) rectangle (157.32,-54.60000000000001);
\draw(157.42, -52.50000000000001) node[anchor=north west,align=left] {Special\\ sequences};
\draw (157.42, -52.50000000000001) rectangle (160.26999999999998,-53.60000000000001);
\draw(160.36999999999998, -52.50000000000001) node[anchor=north west,align=left] {Metric\\ theory of\\ continued\\ fractions};
\draw (160.36999999999998, -52.50000000000001) rectangle (163.21999999999997,-54.60000000000001);
\draw(163.42, -48.300000000000004) node[anchor=north west,align=left] {\large Arithmetic algebraic geometry (Diophantine geometry)};
\draw (163.42, -48.300000000000004) rectangle (183.22,-59.10000000000001);
\draw(164.42, -49.300000000000004) node[anchor=north west,align=left] {\(L\)-functions of\\ varieties over\\ global fields;\\ Birch-Swinnerton-Dyer\\ conjecture};
\draw (164.42, -49.300000000000004) rectangle (170.26999999999998,-51.900000000000006);
\draw(170.36999999999998, -49.300000000000004) node[anchor=north west,align=left] {Drinfel’d\\ modules;\\ higher-dimensional\\ motives, etc.};
\draw (170.36999999999998, -49.300000000000004) rectangle (175.46999999999997,-51.400000000000006);
\draw(175.57, -49.300000000000004) node[anchor=north west,align=left] {Arithmetic\\ aspects of dessins\\ d’enfants,\\ Belyĭ theory};
\draw (175.57, -49.300000000000004) rectangle (180.67,-51.400000000000006);
\draw(180.76999999999998, -49.300000000000004) node[anchor=north west,align=left] {Heights};
\draw (180.76999999999998, -49.300000000000004) rectangle (183.11999999999998,-49.900000000000006);
\draw(164.42, -52.00000000000001) node[anchor=north west,align=left] {Complex\\ multiplication\\  and\\ moduli of\\ abelian varieties};
\draw (164.42, -52.00000000000001) rectangle (169.26999999999998,-54.60000000000001);
\draw(169.36999999999998, -52.00000000000001) node[anchor=north west,align=left] {Arithmetic\\ aspects of\\ modular and\\ Shimura varieties};
\draw (169.36999999999998, -52.00000000000001) rectangle (174.21999999999997,-54.10000000000001);
\draw(174.32, -52.00000000000001) node[anchor=north west,align=left] {Abelian\\ varieties\\  of\\ dimension \(>~1\)};
\draw (174.32, -52.00000000000001) rectangle (178.42,-54.10000000000001);
\draw(178.51999999999998, -52.00000000000001) node[anchor=north west,align=left] {Curves of\\ arbitrary\\ genus or genus\\ \(\ne~1\) over\\ global fields};
\draw (178.51999999999998, -52.00000000000001) rectangle (182.61999999999998,-54.60000000000001);
\draw(164.42, -54.7) node[anchor=north west,align=left] {Polylogarithms\\  and\\ relations\\ with \(K\)-theory};
\draw (164.42, -54.7) rectangle (168.51999999999998,-56.800000000000004);
\draw(168.61999999999998, -54.7) node[anchor=north west,align=left] {Elliptic\\  curves\\ over\\ global fields};
\draw (168.61999999999998, -54.7) rectangle (172.46999999999997,-56.800000000000004);
\draw(172.57, -54.7) node[anchor=north west,align=left] {Elliptic\\  and\\ modular units};
\draw (172.57, -54.7) rectangle (176.42,-56.300000000000004);
\draw(176.51999999999998, -54.7) node[anchor=north west,align=left] {Varieties\\  over\\ global fields};
\draw (176.51999999999998, -54.7) rectangle (180.36999999999998,-56.300000000000004);
\draw(164.42, -56.900000000000006) node[anchor=north west,align=left] {Curves\\ over finite\\  and\\ local fields};
\draw (164.42, -56.900000000000006) rectangle (168.01999999999998,-59.00000000000001);
\draw(168.11999999999998, -56.900000000000006) node[anchor=north west,align=left] {Varieties\\  over\\ finite and\\ local fields};
\draw (168.11999999999998, -56.900000000000006) rectangle (171.71999999999997,-59.00000000000001);
\draw(171.82, -56.900000000000006) node[anchor=north west,align=left] {Geometric\\  class\\ field theory};
\draw (171.82, -56.900000000000006) rectangle (175.42,-58.50000000000001);
\draw(175.51999999999998, -56.900000000000006) node[anchor=north west,align=left] {Elliptic\\  curves\\ over local\\ fields};
\draw (175.51999999999998, -56.900000000000006) rectangle (178.61999999999998,-59.00000000000001);
\draw(178.72, -56.900000000000006) node[anchor=north west,align=left] {Arithmetic\\ mirror\\ symmetry};
\draw (178.72, -56.900000000000006) rectangle (181.82,-58.50000000000001);
\draw(137.17, -55.300000000000004) node[anchor=north west,align=left] {\large Miscellaneous applications of number theory};
\draw (137.17, -55.300000000000004) rectangle (151.1,-58.00000000000001);
\draw(138.17, -56.300000000000004) node[anchor=north west,align=left] {Miscellaneous\\ applications of\\ number theory};
\draw (138.17, -56.300000000000004) rectangle (142.51999999999998,-57.900000000000006);
\draw(137.17, -59.2) node[anchor=north west,align=left] {\large Finite fields and commutative rings (number-theoretic aspects)};
\draw (137.17, -59.2) rectangle (158.22,-65.60000000000001);
\draw(138.17, -60.2) node[anchor=north west,align=left] {Structure\\ theory for finite\\  fields and\\ commutative\\ rings\\ (number-theoretic aspects)};
\draw (138.17, -60.2) rectangle (145.26999999999998,-63.300000000000004);
\draw(145.36999999999998, -60.2) node[anchor=north west,align=left] {Algebraic\\ coding theory;\\ cryptography\\ (number-theoretic\\ aspects)};
\draw (145.36999999999998, -60.2) rectangle (150.21999999999997,-62.800000000000004);
\draw(150.32, -60.2) node[anchor=north west,align=left] {Polynomials\\  over\\ finite fields};
\draw (150.32, -60.2) rectangle (154.17,-61.800000000000004);
\draw(150.32, -61.900000000000006) node[anchor=north west,align=left] {Cyclotomy};
\draw (150.32, -61.900000000000006) rectangle (153.17,-62.50000000000001);
\draw(154.26999999999998, -60.2) node[anchor=north west,align=left] {Arithmetic\\ theory of\\ polynomial\\ rings over\\ finite fields};
\draw (154.26999999999998, -60.2) rectangle (158.11999999999998,-62.800000000000004);
\draw(138.17, -63.400000000000006) node[anchor=north west,align=left] {Exponential\\ sums};
\draw (138.17, -63.400000000000006) rectangle (141.51999999999998,-64.5);
\draw(141.61999999999998, -63.400000000000006) node[anchor=north west,align=left] {Finite\\ upper\\ half-planes};
\draw (141.61999999999998, -63.400000000000006) rectangle (144.96999999999997,-65.0);
\draw(145.07, -63.400000000000006) node[anchor=north west,align=left] {Other\\ character\\ sums and\\ Gauss sums};
\draw (145.07, -63.400000000000006) rectangle (148.17,-65.5);
\draw(158.32, -59.2) node[anchor=north west,align=left] {\large Diophantine approximation, transcendental number theory};
\draw (158.32, -59.2) rectangle (179.07,-72.2);
\draw(159.32, -60.2) node[anchor=north west,align=left] {Transcendence\\ theory\\ of elliptic\\  and\\ abelian functions};
\draw (159.32, -60.2) rectangle (164.17,-62.800000000000004);
\draw(164.26999999999998, -60.2) node[anchor=north west,align=left] {Transcendence\\  theory\\ of other\\ special functions};
\draw (164.26999999999998, -60.2) rectangle (169.11999999999998,-62.300000000000004);
\draw(169.22, -60.2) node[anchor=north west,align=left] {Small\\ fractional parts\\ of polynomials\\  and\\ generalizations};
\draw (169.22, -60.2) rectangle (173.82,-62.800000000000004);
\draw(173.92, -60.2) node[anchor=north west,align=left] {Number-theoretic\\ analogues\\  of methods\\ in Nevanlinna\\ theory (work\\ of Vojta et al.)};
\draw (173.92, -60.2) rectangle (178.51999999999998,-63.300000000000004);
\draw(159.32, -63.400000000000006) node[anchor=north west,align=left] {Markov and\\ Lagrange\\ spectra and\\ generalizations};
\draw (159.32, -63.400000000000006) rectangle (163.67,-65.5);
\draw(163.76999999999998, -63.400000000000006) node[anchor=north west,align=left] {Approximation\\  in\\ non-Archimedean\\ valuations};
\draw (163.76999999999998, -63.400000000000006) rectangle (168.11999999999998,-65.5);
\draw(168.22, -63.400000000000006) node[anchor=north west,align=left] {Continued\\ fractions\\  and\\ generalizations};
\draw (168.22, -63.400000000000006) rectangle (172.57,-65.5);
\draw(172.67, -63.400000000000006) node[anchor=north west,align=left] {Simultaneous\\ homogeneous\\ approximation,\\ linear forms};
\draw (172.67, -63.400000000000006) rectangle (176.76999999999998,-65.5);
\draw(176.87, -63.400000000000006) node[anchor=north west,align=left] {Metric\\ theory};
\draw (176.87, -63.400000000000006) rectangle (178.97,-64.5);
\draw(159.32, -65.60000000000001) node[anchor=north west,align=left] {Irrationality;\\  linear\\ independence\\ over a field};
\draw (159.32, -65.60000000000001) rectangle (163.42,-67.7);
\draw(163.51999999999998, -65.60000000000001) node[anchor=north west,align=left] {Linear forms\\  in\\ logarithms;\\ Baker’s method};
\draw (163.51999999999998, -65.60000000000001) rectangle (167.61999999999998,-67.7);
\draw(167.72, -65.60000000000001) node[anchor=north west,align=left] {Homogeneous\\ approximation\\ to one number};
\draw (167.72, -65.60000000000001) rectangle (171.57,-67.2);
\draw(171.67, -65.60000000000001) node[anchor=north west,align=left] {Approximation\\  by\\ numbers from\\ a fixed field};
\draw (171.67, -65.60000000000001) rectangle (175.51999999999998,-67.7);
\draw(175.62, -65.60000000000001) node[anchor=north west,align=left] {Results\\ involving\\ abelian\\ varieties};
\draw (175.62, -65.60000000000001) rectangle (178.47,-67.7);
\draw(159.32, -67.80000000000001) node[anchor=north west,align=left] {Inhomogeneous\\ linear forms};
\draw (159.32, -67.80000000000001) rectangle (163.17,-68.9);
\draw(163.26999999999998, -67.80000000000001) node[anchor=north west,align=left] {Approximation\\ to\\ algebraic\\ numbers};
\draw (163.26999999999998, -67.80000000000001) rectangle (167.11999999999998,-69.9);
\draw(167.22, -67.80000000000001) node[anchor=north west,align=left] {Transcendence\\ (general\\ theory)};
\draw (167.22, -67.80000000000001) rectangle (171.07,-69.4);
\draw(171.17, -67.80000000000001) node[anchor=north west,align=left] {Measures of\\ irrationality\\  and of\\ transcendence};
\draw (171.17, -67.80000000000001) rectangle (175.01999999999998,-69.9);
\draw(175.12, -67.80000000000001) node[anchor=north west,align=left] {Algebraic\\ independence;\\ Gel’fond’s\\ method};
\draw (175.12, -67.80000000000001) rectangle (178.97,-69.9);
\draw(159.32, -70.0) node[anchor=north west,align=left] {Transcendence\\  theory\\ of Drinfel’d\\ and \(t\)-modules};
\draw (159.32, -70.0) rectangle (163.17,-72.1);
\draw(163.26999999999998, -70.0) node[anchor=north west,align=left] {Diophantine\\ inequalities};
\draw (163.26999999999998, -70.0) rectangle (166.86999999999998,-71.1);
\draw(166.97, -70.0) node[anchor=north west,align=left] {Distribution\\ modulo one};
\draw (166.97, -70.0) rectangle (170.57,-71.1);
\draw(170.67, -70.0) node[anchor=north west,align=left] {Schmidt\\ Subspace\\ Theorem and\\ applications};
\draw (170.67, -70.0) rectangle (174.26999999999998,-72.1);
\draw(137.17, -65.7) node[anchor=north west,align=left] {\large Connections of number theory and logic};
\draw (137.17, -65.7) rectangle (149.54999999999998,-70.60000000000001);
\draw(138.17, -66.7) node[anchor=north west,align=left] {Decidability\\ (number-theoretic\\ aspects)};
\draw (138.17, -66.7) rectangle (143.01999999999998,-68.3);
\draw(143.11999999999998, -66.7) node[anchor=north west,align=left] {Ultraproducts\\ (number-theoretic\\ aspects)};
\draw (143.11999999999998, -66.7) rectangle (147.96999999999997,-68.3);
\draw(138.17, -68.4) node[anchor=north west,align=left] {Model\\ theory\\ (number-theoretic\\ aspects)};
\draw (138.17, -68.4) rectangle (143.01999999999998,-70.5);
\draw(143.11999999999998, -68.4) node[anchor=north west,align=left] {Nonstandard\\ arithmetic\\ (number-theoretic\\ aspects)};
\draw (143.11999999999998, -68.4) rectangle (147.96999999999997,-70.5);
\draw(137.17, -70.7) node[anchor=north west,align=left] {\large History of\\ number theory};
\draw (137.17, -70.7) rectangle (141.79999999999998,-71.8);
\draw(137.17, -72.30000000000001) node[anchor=north west,align=left] {\large Algebraic number theory: local and \(p\)-adic fields};
\draw (137.17, -72.30000000000001) rectangle (156.96999999999997,-80.9);
\draw(138.17, -73.30000000000001) node[anchor=north west,align=left] {Other analytic\\ theory (analogues\\  of beta\\ and gamma\\ functions, \(p\)-adic\\ integration, etc.)};
\draw (138.17, -73.30000000000001) rectangle (143.26999999999998,-76.4);
\draw(143.36999999999998, -73.30000000000001) node[anchor=north west,align=left] {Langlands-Weil\\ conjectures,\\ nonabelian class\\ field theory};
\draw (143.36999999999998, -73.30000000000001) rectangle (147.96999999999997,-75.4);
\draw(143.36999999999998, -75.50000000000001) node[anchor=north west,align=left] {Polynomials};
\draw (143.36999999999998, -75.50000000000001) rectangle (146.71999999999997,-76.10000000000001);
\draw(148.07, -73.30000000000001) node[anchor=north west,align=left] {Integral\\ representations};
\draw (148.07, -73.30000000000001) rectangle (152.42,-74.4);
\draw(152.51999999999998, -73.30000000000001) node[anchor=north west,align=left] {Non-Archimedean\\ dynamical\\ systems};
\draw (152.51999999999998, -73.30000000000001) rectangle (156.86999999999998,-74.9);
\draw(138.17, -76.50000000000001) node[anchor=north west,align=left] {Zeta functions\\  and\\ \(L\)-functions};
\draw (138.17, -76.50000000000001) rectangle (142.26999999999998,-78.10000000000001);
\draw(142.36999999999998, -76.50000000000001) node[anchor=north west,align=left] {Algebras and\\  orders,\\ and their\\ zeta functions};
\draw (142.36999999999998, -76.50000000000001) rectangle (146.46999999999997,-78.60000000000001);
\draw(146.57, -76.50000000000001) node[anchor=north west,align=left] {Prehomogeneous\\ vector spaces};
\draw (146.57, -76.50000000000001) rectangle (150.67,-77.60000000000001);
\draw(150.76999999999998, -76.50000000000001) node[anchor=north west,align=left] {Class field\\  theory;\\ \(p\)-adic\\ formal groups};
\draw (150.76999999999998, -76.50000000000001) rectangle (154.61999999999998,-78.60000000000001);
\draw(154.72, -76.50000000000001) node[anchor=north west,align=left] {Galois\\ theory};
\draw (154.72, -76.50000000000001) rectangle (156.82,-77.60000000000001);
\draw(138.17, -78.70000000000002) node[anchor=north west,align=left] {Ramification\\ and\\ extension\\ theory};
\draw (138.17, -78.70000000000002) rectangle (141.76999999999998,-80.80000000000001);
\draw(141.86999999999998, -78.70000000000002) node[anchor=north west,align=left] {Other\\ nonanalytic\\ theory};
\draw (141.86999999999998, -78.70000000000002) rectangle (145.21999999999997,-80.30000000000001);
\draw(145.32, -78.70000000000002) node[anchor=north west,align=left] {Galois\\ cohomology};
\draw (145.32, -78.70000000000002) rectangle (148.42,-79.80000000000001);
\draw(148.51999999999998, -78.70000000000002) node[anchor=north west,align=left] {\(K\)-theory\\ of local\\ fields};
\draw (148.51999999999998, -78.70000000000002) rectangle (151.11999999999998,-80.30000000000001);
\draw(157.07, -72.30000000000001) node[anchor=north west,align=left] {\large Discontinuous groups and automorphic forms};
\draw (157.07, -72.30000000000001) rectangle (173.17,-99.60000000000001);
\draw(158.07, -73.30000000000001) node[anchor=north west,align=left] {Automorphic forms\\ on \(\mbox{GL}(2)\); Hilbert and\\ Hilbert-Siegel\\ modular groups and\\ their modular and\\ automorphic forms;\\ Hilbert modular surfaces};
\draw (158.07, -73.30000000000001) rectangle (164.67,-76.9);
\draw(164.76999999999998, -73.30000000000001) node[anchor=north west,align=left] {Representation-theoretic\\ methods;\\ automorphic\\ representations\\ over local and\\ global fields};
\draw (164.76999999999998, -73.30000000000001) rectangle (171.36999999999998,-76.4);
\draw(158.07, -77.00000000000001) node[anchor=north west,align=left] {Langlands\\ \(L\)-functions; one\\ variable Dirichlet\\  series and\\ functional equations};
\draw (158.07, -77.00000000000001) rectangle (163.67,-79.60000000000001);
\draw(163.76999999999998, -77.00000000000001) node[anchor=north west,align=left] {Other groups\\ and their modular\\ and automorphic\\  forms\\ (several variables)};
\draw (163.76999999999998, -77.00000000000001) rectangle (169.11999999999998,-79.60000000000001);
\draw(169.22, -77.00000000000001) node[anchor=north west,align=left] {Dedekind\\ eta function,\\ Dedekind sums};
\draw (169.22, -77.00000000000001) rectangle (173.07,-78.60000000000001);
\draw(158.07, -79.70000000000002) node[anchor=north west,align=left] {Hecke-Petersson\\ operators,\\ differential\\  operators\\ (several variables)};
\draw (158.07, -79.70000000000002) rectangle (163.42,-82.30000000000001);
\draw(163.51999999999998, -79.70000000000002) node[anchor=north west,align=left] {Dirichlet series\\ in several complex\\  variables\\ associated to\\ automorphic forms;\\ Weyl group multiple\\ Dirichlet series};
\draw (163.51999999999998, -79.70000000000002) rectangle (168.86999999999998,-83.30000000000001);
\draw(168.97, -79.70000000000002) node[anchor=north west,align=left] {Relations\\ with algebraic\\ geometry\\ and topology};
\draw (168.97, -79.70000000000002) rectangle (173.07,-81.80000000000001);
\draw(168.97, -81.9) node[anchor=north west,align=left] {Jacobi\\ forms};
\draw (168.97, -81.9) rectangle (171.07,-83.0);
\draw(158.07, -83.4) node[anchor=north west,align=left] {Siegel modular\\ groups; Siegel\\ and Hilbert-Siegel\\ modular and\\ automorphic forms};
\draw (158.07, -83.4) rectangle (163.17,-86.0);
\draw(163.26999999999998, -83.4) node[anchor=north west,align=left] {Special values\\ of automorphic\\ \(L\)-series, periods\\ of automorphic\\ forms, cohomology,\\ modular symbols};
\draw (163.26999999999998, -83.4) rectangle (168.36999999999998,-86.5);
\draw(168.47, -83.4) node[anchor=north west,align=left] {Modular\\ correspondences,\\ etc.};
\draw (168.47, -83.4) rectangle (173.07,-85.0);
\draw(168.47, -85.10000000000001) node[anchor=north west,align=left] {Galois\\ representations};
\draw (168.47, -85.10000000000001) rectangle (172.82,-86.2);
\draw(158.07, -86.60000000000001) node[anchor=north west,align=left] {Structure of\\ modular groups\\  and\\ generalizations;\\ arithmetic groups};
\draw (158.07, -86.60000000000001) rectangle (162.92,-89.2);
\draw(163.01999999999998, -86.60000000000001) node[anchor=north west,align=left] {Automorphic\\ forms and\\ their relations\\  with\\ perfectoid spaces};
\draw (163.01999999999998, -86.60000000000001) rectangle (167.86999999999998,-89.2);
\draw(167.97, -86.60000000000001) node[anchor=north west,align=left] {Holomorphic\\  modular\\ forms of\\ integral weight};
\draw (167.97, -86.60000000000001) rectangle (172.32,-88.7);
\draw(158.07, -89.30000000000001) node[anchor=north west,align=left] {Relationship\\ to Lie algebras\\ and finite\\ simple groups};
\draw (158.07, -89.30000000000001) rectangle (162.42,-91.4);
\draw(162.51999999999998, -89.30000000000001) node[anchor=north west,align=left] {Hecke-Petersson\\ operators,\\ differential\\ operators\\ (one variable)};
\draw (162.51999999999998, -89.30000000000001) rectangle (166.86999999999998,-91.9);
\draw(166.97, -89.30000000000001) node[anchor=north west,align=left] {Theta series;\\  Weil\\ representation;\\  theta\\ correspondences};
\draw (166.97, -89.30000000000001) rectangle (171.32,-91.9);
\draw(158.07, -92.00000000000001) node[anchor=north west,align=left] {Fourier\\ coefficients\\ of automorphic\\ forms};
\draw (158.07, -92.00000000000001) rectangle (162.17,-94.10000000000001);
\draw(162.26999999999998, -92.00000000000001) node[anchor=north west,align=left] {Forms of\\ half-integer\\ weight;\\ nonholomorphic\\ modular forms};
\draw (162.26999999999998, -92.00000000000001) rectangle (166.36999999999998,-94.60000000000001);
\draw(166.47, -92.00000000000001) node[anchor=north west,align=left] {Congruences\\ for modular\\ and \(p\)-adic\\ modular forms};
\draw (166.47, -92.00000000000001) rectangle (170.32,-94.10000000000001);
\draw(158.07, -94.70000000000002) node[anchor=north west,align=left] {Modular forms\\ associated\\ to Drinfel’d\\ modules};
\draw (158.07, -94.70000000000002) rectangle (161.92,-96.80000000000001);
\draw(162.01999999999998, -94.70000000000002) node[anchor=north west,align=left] {Spectral\\ theory; trace\\  formulas\\ (e.g., that\\ of Selberg)};
\draw (162.01999999999998, -94.70000000000002) rectangle (165.86999999999998,-97.30000000000001);
\draw(165.97, -94.70000000000002) node[anchor=north west,align=left] {Cohomology\\ of arithmetic\\ groups};
\draw (165.97, -94.70000000000002) rectangle (169.82,-96.30000000000001);
\draw(158.07, -97.4) node[anchor=north west,align=left] {\(p\)-adic\\ theory, local\\ fields};
\draw (158.07, -97.4) rectangle (161.92,-99.0);
\draw(162.01999999999998, -97.4) node[anchor=north west,align=left] {Automorphic\\  forms,\\ one variable};
\draw (162.01999999999998, -97.4) rectangle (165.61999999999998,-99.0);
\draw(165.72, -97.4) node[anchor=north west,align=left] {Modular\\ and\\ automorphic\\ functions};
\draw (165.72, -97.4) rectangle (169.07,-99.5);
\draw(137.17, -81.0) node[anchor=north west,align=left] {\large Zeta and \(L\)-functions: analytic theory};
\draw (137.17, -81.0) rectangle (152.76999999999998,-92.3);
\draw(138.17, -82.0) node[anchor=north west,align=left] {Selberg zeta functions\\  and regularized\\ determinants; applications\\ to spectral theory,\\  Dirichlet series,\\ Eisenstein series, etc.\\ (explicit formulas)};
\draw (138.17, -82.0) rectangle (145.26999999999998,-85.6);
\draw(145.36999999999998, -82.0) node[anchor=north west,align=left] {Nonreal zeros\\ of \(\zeta~(s)\) and\\  \(L(s,~\chi)\);\\ Riemann and\\ other hypotheses};
\draw (145.36999999999998, -82.0) rectangle (149.96999999999997,-84.6);
\draw(150.07, -82.0) node[anchor=north west,align=left] {\(\zeta~(s)\)\\  and\\ \(L(s,~\chi)\)};
\draw (150.07, -82.0) rectangle (152.67,-83.6);
\draw(138.17, -85.7) node[anchor=north west,align=left] {Multiple\\ Dirichlet series\\ and zeta\\ functions and\\ multizeta values};
\draw (138.17, -85.7) rectangle (142.76999999999998,-88.3);
\draw(142.86999999999998, -85.7) node[anchor=north west,align=left] {Zeta and\\ \(L\)-functions\\  in\\ characteristic \(p\)};
\draw (142.86999999999998, -85.7) rectangle (147.46999999999997,-87.8);
\draw(147.57, -85.7) node[anchor=north west,align=left] {Other\\ Dirichlet series\\ and zeta\\ functions};
\draw (147.57, -85.7) rectangle (152.17,-87.8);
\draw(138.17, -88.4) node[anchor=north west,align=left] {Hurwitz and\\  Lerch\\ zeta functions};
\draw (138.17, -88.4) rectangle (142.26999999999998,-90.0);
\draw(142.36999999999998, -88.4) node[anchor=north west,align=left] {Relations\\  with\\ noncommutative\\ geometry};
\draw (142.36999999999998, -88.4) rectangle (146.46999999999997,-90.5);
\draw(146.57, -88.4) node[anchor=north west,align=left] {Real zeros\\ of \(L(s,~\chi)\);\\ results\\ on \(L(1,~\chi)\)};
\draw (146.57, -88.4) rectangle (150.17,-90.5);
\draw(138.17, -90.6) node[anchor=north west,align=left] {Relations\\ with random\\ matrices};
\draw (138.17, -90.6) rectangle (141.51999999999998,-92.19999999999999);
\draw(141.61999999999998, -90.6) node[anchor=north west,align=left] {Tauberian\\ theorems};
\draw (141.61999999999998, -90.6) rectangle (144.46999999999997,-91.69999999999999);
\draw(137.17, -92.4) node[anchor=north west,align=left] {\large Polynomials and matrices};
\draw (137.17, -92.4) rectangle (146.07,-95.60000000000001);
\draw(138.17, -93.4) node[anchor=north west,align=left] {Polynomials\\  in\\ number theory};
\draw (138.17, -93.4) rectangle (142.01999999999998,-95.0);
\draw(142.11999999999998, -93.4) node[anchor=north west,align=left] {Matrices,\\ determinants\\  in\\ number theory};
\draw (142.11999999999998, -93.4) rectangle (145.96999999999997,-95.5);
\draw(173.26999999999998, -72.30000000000001) node[anchor=north west,align=left] {\large Geometry of numbers};
\draw (173.26999999999998, -72.30000000000001) rectangle (183.17,-82.60000000000001);
\draw(174.26999999999998, -73.30000000000001) node[anchor=north west,align=left] {Lattices and\\ convex bodies\\ (number-theoretic\\ aspects)};
\draw (174.26999999999998, -73.30000000000001) rectangle (179.11999999999998,-75.4);
\draw(179.21999999999997, -73.30000000000001) node[anchor=north west,align=left] {Relations\\  with\\ coding theory};
\draw (179.21999999999997, -73.30000000000001) rectangle (183.06999999999996,-74.9);
\draw(174.26999999999998, -75.50000000000001) node[anchor=north west,align=left] {Lattice\\ packing and\\ covering\\ (number-theoretic\\ aspects)};
\draw (174.26999999999998, -75.50000000000001) rectangle (179.11999999999998,-78.10000000000001);
\draw(179.21999999999997, -75.50000000000001) node[anchor=north west,align=left] {Automorphism\\  groups\\ of lattices};
\draw (179.21999999999997, -75.50000000000001) rectangle (182.81999999999996,-77.10000000000001);
\draw(174.26999999999998, -78.20000000000002) node[anchor=north west,align=left] {Quadratic\\ forms (reduction\\ theory,\\  extreme\\ forms, etc.)};
\draw (174.26999999999998, -78.20000000000002) rectangle (178.86999999999998,-80.80000000000001);
\draw(178.96999999999997, -78.20000000000002) node[anchor=north west,align=left] {Mean value\\ and transfer\\ theorems};
\draw (178.96999999999997, -78.20000000000002) rectangle (182.56999999999996,-79.80000000000001);
\draw(174.26999999999998, -80.9) node[anchor=north west,align=left] {Nonconvex\\ bodies};
\draw (174.26999999999998, -80.9) rectangle (177.11999999999998,-82.0);
\draw(177.21999999999997, -80.9) node[anchor=north west,align=left] {Products\\ of linear\\ forms};
\draw (177.21999999999997, -80.9) rectangle (180.06999999999996,-82.5);
\draw(180.17, -80.9) node[anchor=north west,align=left] {Minima\\ of forms};
\draw (180.17, -80.9) rectangle (182.76999999999998,-82.0);
\draw(173.26999999999998, -82.7) node[anchor=north west,align=left] {\large Sequences and sets};
\draw (173.26999999999998, -82.7) rectangle (182.67,-99.10000000000001);
\draw(174.26999999999998, -83.7) node[anchor=north west,align=left] {Arithmetic\\ combinatorics;\\  higher\\ degree uniformity};
\draw (174.26999999999998, -83.7) rectangle (179.11999999999998,-85.8);
\draw(179.21999999999997, -83.7) node[anchor=north west,align=left] {Recurrences};
\draw (179.21999999999997, -83.7) rectangle (182.56999999999996,-84.3);
\draw(179.21999999999997, -84.4) node[anchor=north west,align=left] {Sequences\\ (mod \(m\))};
\draw (179.21999999999997, -84.4) rectangle (182.06999999999996,-85.5);
\draw(174.26999999999998, -85.9) node[anchor=north west,align=left] {Fibonacci and\\ Lucas numbers\\  and\\ polynomials and\\ generalizations};
\draw (174.26999999999998, -85.9) rectangle (178.61999999999998,-88.5);
\draw(178.71999999999997, -85.9) node[anchor=north west,align=left] {Binomial\\ coefficients;\\ factorials;\\ \(q\)-identities};
\draw (178.71999999999997, -85.9) rectangle (182.56999999999996,-88.0);
\draw(174.26999999999998, -88.60000000000001) node[anchor=north west,align=left] {Representation\\ functions};
\draw (174.26999999999998, -88.60000000000001) rectangle (178.36999999999998,-89.7);
\draw(178.46999999999997, -88.60000000000001) node[anchor=north west,align=left] {Farey\\ sequences; the\\ sequences\\ \(1^k,~2^k,~\dots\)};
\draw (178.46999999999997, -88.60000000000001) rectangle (182.56999999999996,-90.7);
\draw(174.26999999999998, -90.80000000000001) node[anchor=north west,align=left] {Other\\ combinatorial\\ number\\ theory};
\draw (174.26999999999998, -90.80000000000001) rectangle (178.11999999999998,-92.9);
\draw(178.21999999999997, -90.80000000000001) node[anchor=north west,align=left] {Arithmetic\\ progressions};
\draw (178.21999999999997, -90.80000000000001) rectangle (181.81999999999996,-91.9);
\draw(174.26999999999998, -93.0) node[anchor=north west,align=left] {Bernoulli\\ and Euler\\ numbers and\\ polynomials};
\draw (174.26999999999998, -93.0) rectangle (177.61999999999998,-95.1);
\draw(177.71999999999997, -93.0) node[anchor=north west,align=left] {Special\\ sequences\\  and\\ polynomials};
\draw (177.71999999999997, -93.0) rectangle (181.06999999999996,-95.1);
\draw(174.26999999999998, -95.2) node[anchor=north west,align=left] {Additive\\  bases,\\ including\\ sumsets};
\draw (174.26999999999998, -95.2) rectangle (177.11999999999998,-97.3);
\draw(177.21999999999997, -95.2) node[anchor=north west,align=left] {Automata\\ sequences};
\draw (177.21999999999997, -95.2) rectangle (180.06999999999996,-96.3);
\draw(174.26999999999998, -97.4) node[anchor=north west,align=left] {Density,\\ gaps,\\ topology};
\draw (174.26999999999998, -97.4) rectangle (176.86999999999998,-99.0);
\draw(176.96999999999997, -97.4) node[anchor=north west,align=left] {Bell and\\ Stirling\\ numbers};
\draw (176.96999999999997, -97.4) rectangle (179.56999999999996,-99.0);
\draw(137.17, -99.70000000000002) node[anchor=north west,align=left] {\large Algebraic number theory: global fields};
\draw (137.17, -99.70000000000002) rectangle (151.76999999999998,-121.50000000000001);
\draw(138.17, -100.70000000000002) node[anchor=north west,align=left] {PV-numbers and\\ generalizations;\\ other special\\  algebraic\\ numbers; Mahler measure};
\draw (138.17, -100.70000000000002) rectangle (144.51999999999998,-103.30000000000001);
\draw(144.61999999999998, -100.70000000000002) node[anchor=north west,align=left] {Integral\\ representations related\\  to algebraic\\ numbers; Galois\\ module structure\\ of rings of integers};
\draw (144.61999999999998, -100.70000000000002) rectangle (150.96999999999997,-103.80000000000001);
\draw(138.17, -103.90000000000002) node[anchor=north west,align=left] {Polynomials\\ (irreducibility,\\ etc.)};
\draw (138.17, -103.90000000000002) rectangle (142.76999999999998,-105.50000000000001);
\draw(142.86999999999998, -103.90000000000002) node[anchor=north west,align=left] {Langlands-Weil\\ conjectures,\\ nonabelian class\\ field theory};
\draw (142.86999999999998, -103.90000000000002) rectangle (147.46999999999997,-106.00000000000001);
\draw(147.57, -103.90000000000002) node[anchor=north west,align=left] {Algebraic\\ numbers; rings\\ of algebraic\\ integers};
\draw (147.57, -103.90000000000002) rectangle (151.67,-106.00000000000001);
\draw(138.17, -106.10000000000002) node[anchor=north west,align=left] {Arithmetic\\ theory of\\ algebraic\\ function fields};
\draw (138.17, -106.10000000000002) rectangle (142.51999999999998,-108.20000000000002);
\draw(142.61999999999998, -106.10000000000002) node[anchor=north west,align=left] {Zeta functions\\  and\\ \(L\)-functions of\\ function fields};
\draw (142.61999999999998, -106.10000000000002) rectangle (146.96999999999997,-108.20000000000002);
\draw(147.07, -106.10000000000002) node[anchor=north west,align=left] {Cyclotomic\\ function fields\\ (class groups,\\  Bernoulli\\ objects, etc.)};
\draw (147.07, -106.10000000000002) rectangle (151.42,-108.70000000000002);
\draw(138.17, -108.80000000000001) node[anchor=north west,align=left] {Class\\ numbers, class\\  groups,\\ discriminants};
\draw (138.17, -108.80000000000001) rectangle (142.26999999999998,-110.9);
\draw(142.36999999999998, -108.80000000000001) node[anchor=north west,align=left] {Zeta functions\\  and\\ \(L\)-functions of\\ number fields};
\draw (142.36999999999998, -108.80000000000001) rectangle (146.46999999999997,-110.9);
\draw(146.57, -108.80000000000001) node[anchor=north west,align=left] {Quaternion and\\ other division\\ algebras:\\  arithmetic,\\ zeta functions};
\draw (146.57, -108.80000000000001) rectangle (150.67,-111.4);
\draw(138.17, -111.50000000000001) node[anchor=north west,align=left] {Other algebras\\  and\\ orders, and\\ their zeta and\\ \(L\)-functions};
\draw (138.17, -111.50000000000001) rectangle (142.26999999999998,-114.10000000000001);
\draw(142.36999999999998, -111.50000000000001) node[anchor=north west,align=left] {Units\\ and\\ factorization};
\draw (142.36999999999998, -111.50000000000001) rectangle (146.21999999999997,-113.10000000000001);
\draw(146.32, -111.50000000000001) node[anchor=north west,align=left] {Class groups\\  and\\ Picard groups\\ of orders};
\draw (146.32, -111.50000000000001) rectangle (150.17,-113.60000000000001);
\draw(138.17, -114.20000000000002) node[anchor=north west,align=left] {Distribution\\  of\\ prime ideals};
\draw (138.17, -114.20000000000002) rectangle (141.76999999999998,-115.80000000000001);
\draw(141.86999999999998, -114.20000000000002) node[anchor=north west,align=left] {Other\\ abelian and\\ metabelian\\ extensions};
\draw (141.86999999999998, -114.20000000000002) rectangle (145.21999999999997,-116.30000000000001);
\draw(145.32, -114.20000000000002) node[anchor=north west,align=left] {Quadratic\\ extensions};
\draw (145.32, -114.20000000000002) rectangle (148.42,-115.30000000000001);
\draw(148.51999999999998, -114.20000000000002) node[anchor=north west,align=left] {Cubic and\\  quartic\\ extensions};
\draw (148.51999999999998, -114.20000000000002) rectangle (151.61999999999998,-115.80000000000001);
\draw(138.17, -116.40000000000002) node[anchor=north west,align=left] {Cyclotomic\\ extensions};
\draw (138.17, -116.40000000000002) rectangle (141.26999999999998,-117.50000000000001);
\draw(141.36999999999998, -116.40000000000002) node[anchor=north west,align=left] {Galois\\ cohomology};
\draw (141.36999999999998, -116.40000000000002) rectangle (144.46999999999997,-117.50000000000001);
\draw(144.57, -116.40000000000002) node[anchor=north west,align=left] {Adèle\\ rings and\\ groups};
\draw (144.57, -116.40000000000002) rectangle (147.42,-118.00000000000001);
\draw(147.51999999999998, -116.40000000000002) node[anchor=north west,align=left] {\(K\)-theory\\ of global\\ fields};
\draw (147.51999999999998, -116.40000000000002) rectangle (150.36999999999998,-118.00000000000001);
\draw(138.17, -118.10000000000002) node[anchor=north west,align=left] {Density\\ theorems};
\draw (138.17, -118.10000000000002) rectangle (140.76999999999998,-119.20000000000002);
\draw(140.86999999999998, -118.10000000000002) node[anchor=north west,align=left] {Other\\ analytic\\ theory};
\draw (140.86999999999998, -118.10000000000002) rectangle (143.46999999999997,-119.70000000000002);
\draw(143.57, -118.10000000000002) node[anchor=north west,align=left] {Iwasawa\\ theory};
\draw (143.57, -118.10000000000002) rectangle (145.92,-119.20000000000002);
\draw(146.01999999999998, -118.10000000000002) node[anchor=north west,align=left] {Totally\\ real\\ fields};
\draw (146.01999999999998, -118.10000000000002) rectangle (148.36999999999998,-119.70000000000002);
\draw(148.47, -118.10000000000002) node[anchor=north west,align=left] {Other\\ number\\ fields};
\draw (148.47, -118.10000000000002) rectangle (150.57,-119.70000000000002);
\draw(138.17, -119.80000000000001) node[anchor=north west,align=left] {Galois\\ theory};
\draw (138.17, -119.80000000000001) rectangle (140.26999999999998,-120.9);
\draw(140.36999999999998, -119.80000000000001) node[anchor=north west,align=left] {Class\\ field\\ theory};
\draw (140.36999999999998, -119.80000000000001) rectangle (142.46999999999997,-121.4);
\draw(151.86999999999998, -99.70000000000002) node[anchor=north west,align=left] {\large Exponential sums and character sums};
\draw (151.86999999999998, -99.70000000000002) rectangle (164.96999999999997,-107.30000000000001);
\draw(152.86999999999998, -100.70000000000002) node[anchor=north west,align=left] {Estimates\\  on\\ exponential sums};
\draw (152.86999999999998, -100.70000000000002) rectangle (157.46999999999997,-102.30000000000001);
\draw(157.56999999999996, -100.70000000000002) node[anchor=north west,align=left] {Gauss and\\ Kloosterman\\  sums;\\ generalizations};
\draw (157.56999999999996, -100.70000000000002) rectangle (161.91999999999996,-102.80000000000001);
\draw(162.01999999999998, -100.70000000000002) node[anchor=north west,align=left] {Sums over\\ primes};
\draw (162.01999999999998, -100.70000000000002) rectangle (164.86999999999998,-101.80000000000001);
\draw(152.86999999999998, -102.90000000000002) node[anchor=north west,align=left] {Jacobsthal\\ and Brewer\\ sums; other\\  complete\\ character sums};
\draw (152.86999999999998, -102.90000000000002) rectangle (156.96999999999997,-105.50000000000001);
\draw(157.06999999999996, -102.90000000000002) node[anchor=north west,align=left] {Trigonometric\\  and\\ exponential\\ sums, general};
\draw (157.06999999999996, -102.90000000000002) rectangle (160.91999999999996,-105.00000000000001);
\draw(161.01999999999998, -102.90000000000002) node[anchor=north west,align=left] {Estimates\\ on character\\ sums};
\draw (161.01999999999998, -102.90000000000002) rectangle (164.61999999999998,-104.50000000000001);
\draw(152.86999999999998, -105.60000000000002) node[anchor=north west,align=left] {Sums over\\ arbitrary\\ intervals};
\draw (152.86999999999998, -105.60000000000002) rectangle (155.71999999999997,-107.20000000000002);
\draw(155.81999999999996, -105.60000000000002) node[anchor=north west,align=left] {Weyl\\ sums};
\draw (155.81999999999996, -105.60000000000002) rectangle (157.41999999999996,-106.70000000000002);
\draw(151.86999999999998, -107.4) node[anchor=north west,align=left] {\large Additive number theory; partitions};
\draw (151.86999999999998, -107.4) rectangle (164.71999999999997,-117.7);
\draw(152.86999999999998, -108.4) node[anchor=north west,align=left] {Partition\\ identities;\\ identities\\ of\\ Rogers-Ramanujan type};
\draw (152.86999999999998, -108.4) rectangle (158.71999999999997,-111.0);
\draw(158.81999999999996, -108.4) node[anchor=north west,align=left] {Inverse\\ problems of\\ additive number\\ theory,\\ including sumsets};
\draw (158.81999999999996, -108.4) rectangle (163.66999999999996,-111.0);
\draw(152.86999999999998, -111.10000000000001) node[anchor=north west,align=left] {Goldbach-type\\  theorems;\\ other additive\\ questions\\ involving primes};
\draw (152.86999999999998, -111.10000000000001) rectangle (157.46999999999997,-113.7);
\draw(157.56999999999996, -111.10000000000001) node[anchor=north west,align=left] {Applications\\  of the\\ Hardy-Littlewood\\ method};
\draw (157.56999999999996, -111.10000000000001) rectangle (162.16999999999996,-113.2);
\draw(152.86999999999998, -113.80000000000001) node[anchor=north west,align=left] {Partitions;\\ congruences and\\ congruential\\ restrictions};
\draw (152.86999999999998, -113.80000000000001) rectangle (157.21999999999997,-115.9);
\draw(157.31999999999996, -113.80000000000001) node[anchor=north west,align=left] {Waring’s\\ problem\\ and variants};
\draw (157.31999999999996, -113.80000000000001) rectangle (160.91999999999996,-115.4);
\draw(161.01999999999998, -113.80000000000001) node[anchor=north west,align=left] {Lattice\\ points\\ in specified\\ regions};
\draw (161.01999999999998, -113.80000000000001) rectangle (164.61999999999998,-115.9);
\draw(152.86999999999998, -116.0) node[anchor=north west,align=left] {Elementary\\ theory of\\ partitions};
\draw (152.86999999999998, -116.0) rectangle (155.96999999999997,-117.6);
\draw(156.06999999999996, -116.0) node[anchor=north west,align=left] {Analytic\\ theory of\\ partitions};
\draw (156.06999999999996, -116.0) rectangle (159.16999999999996,-117.6);
\draw(165.07, -99.70000000000002) node[anchor=north west,align=left] {\large Multiplicative number theory};
\draw (165.07, -99.70000000000002) rectangle (177.92,-116.90000000000002);
\draw(166.07, -100.70000000000002) node[anchor=north west,align=left] {Distribution\\ functions\\ associated with\\ additive and\\ positive multiplicative\\ functions};
\draw (166.07, -100.70000000000002) rectangle (172.42,-103.80000000000001);
\draw(172.51999999999998, -100.70000000000002) node[anchor=north west,align=left] {Primes represented\\  by\\ polynomials; other\\ multiplicative\\ structures of\\ polynomial values};
\draw (172.51999999999998, -100.70000000000002) rectangle (177.61999999999998,-103.80000000000001);
\draw(166.07, -103.90000000000002) node[anchor=north west,align=left] {Other results\\ on the distribution\\  of values\\ or the characterization\\  of\\ arithmetic functions};
\draw (166.07, -103.90000000000002) rectangle (172.42,-107.00000000000001);
\draw(172.51999999999998, -103.90000000000002) node[anchor=north west,align=left] {Asymptotic\\ results on\\ counting functions\\ for algebraic\\ and topological\\ structures};
\draw (172.51999999999998, -103.90000000000002) rectangle (177.61999999999998,-107.00000000000001);
\draw(166.07, -107.10000000000002) node[anchor=north west,align=left] {Applications\\ of automorphic\\ functions and\\ forms to\\ multiplicative problems};
\draw (166.07, -107.10000000000002) rectangle (172.42,-109.70000000000002);
\draw(172.51999999999998, -107.10000000000002) node[anchor=north west,align=left] {Distribution\\ of integers\\  in special\\ residue classes};
\draw (172.51999999999998, -107.10000000000002) rectangle (176.86999999999998,-109.20000000000002);
\draw(166.07, -109.80000000000001) node[anchor=north west,align=left] {Distribution\\ of integers\\ with specified\\ multiplicative\\ constraints};
\draw (166.07, -109.80000000000001) rectangle (170.17,-112.4);
\draw(170.26999999999998, -109.80000000000001) node[anchor=north west,align=left] {Applications\\  of\\ sieve methods};
\draw (170.26999999999998, -109.80000000000001) rectangle (174.11999999999998,-111.4);
\draw(170.26999999999998, -111.50000000000001) node[anchor=north west,align=left] {Sieves};
\draw (170.26999999999998, -111.50000000000001) rectangle (172.36999999999998,-112.10000000000001);
\draw(174.22, -109.80000000000001) node[anchor=north west,align=left] {Distribution\\ of primes};
\draw (174.22, -109.80000000000001) rectangle (177.82,-110.9);
\draw(166.07, -112.50000000000001) node[anchor=north west,align=left] {Asymptotic\\  results\\ on arithmetic\\ functions};
\draw (166.07, -112.50000000000001) rectangle (169.92,-114.60000000000001);
\draw(170.01999999999998, -112.50000000000001) node[anchor=north west,align=left] {Generalized\\  primes\\ and integers};
\draw (170.01999999999998, -112.50000000000001) rectangle (173.61999999999998,-114.10000000000001);
\draw(173.72, -112.50000000000001) node[anchor=north west,align=left] {Primes in\\ congruence\\ classes};
\draw (173.72, -112.50000000000001) rectangle (176.82,-114.10000000000001);
\draw(166.07, -114.70000000000002) node[anchor=north west,align=left] {Rate of\\ growth of\\ arithmetic\\ functions};
\draw (166.07, -114.70000000000002) rectangle (169.17,-116.80000000000001);
\draw(169.26999999999998, -114.70000000000002) node[anchor=north west,align=left] {Turán\\ theory};
\draw (169.26999999999998, -114.70000000000002) rectangle (171.36999999999998,-115.80000000000001);
\draw(137.17, -121.60000000000002) node[anchor=north west,align=left] {\large Forms and linear algebraic groups};
\draw (137.17, -121.60000000000002) rectangle (149.76999999999998,-140.00000000000003);
\draw(138.17, -122.60000000000002) node[anchor=north west,align=left] {Analytic theory\\ (Epstein zeta\\  functions;\\ relations with\\ automorphic\\ forms and functions)};
\draw (138.17, -122.60000000000002) rectangle (143.76999999999998,-125.70000000000002);
\draw(143.86999999999998, -122.60000000000002) node[anchor=north west,align=left] {Sums of squares\\ and representations\\ by other\\  particular\\ quadratic forms};
\draw (143.86999999999998, -122.60000000000002) rectangle (149.21999999999997,-125.20000000000002);
\draw(138.17, -125.80000000000003) node[anchor=north west,align=left] {General ternary\\ and quaternary\\ quadratic forms;\\ forms of more\\ than two variables};
\draw (138.17, -125.80000000000003) rectangle (143.26999999999998,-128.40000000000003);
\draw(143.36999999999998, -125.80000000000003) node[anchor=north west,align=left] {General\\ binary quadratic\\ forms};
\draw (143.36999999999998, -125.80000000000003) rectangle (147.96999999999997,-127.40000000000002);
\draw(138.17, -128.50000000000003) node[anchor=north west,align=left] {Galois\\ cohomology of\\ linear algebraic\\ groups};
\draw (138.17, -128.50000000000003) rectangle (142.76999999999998,-130.60000000000002);
\draw(142.86999999999998, -128.50000000000003) node[anchor=north west,align=left] {Algebraic\\ theory of\\ quadratic\\ forms; Witt\\ groups and rings};
\draw (142.86999999999998, -128.50000000000003) rectangle (147.46999999999997,-131.10000000000002);
\draw(147.57, -128.50000000000003) node[anchor=north west,align=left] {\(p\)-adic\\ theory};
\draw (147.57, -128.50000000000003) rectangle (149.67,-129.60000000000002);
\draw(138.17, -131.20000000000002) node[anchor=north west,align=left] {Class numbers\\ of quadratic\\  and\\ Hermitian forms};
\draw (138.17, -131.20000000000002) rectangle (142.51999999999998,-133.3);
\draw(142.61999999999998, -131.20000000000002) node[anchor=north west,align=left] {\(K\)-theory\\ of quadratic\\  and\\ Hermitian forms};
\draw (142.61999999999998, -131.20000000000002) rectangle (146.96999999999997,-133.3);
\draw(138.17, -133.40000000000003) node[anchor=north west,align=left] {Quadratic\\  forms\\ over\\ general fields};
\draw (138.17, -133.40000000000003) rectangle (142.26999999999998,-135.50000000000003);
\draw(142.36999999999998, -133.40000000000003) node[anchor=north west,align=left] {Bilinear\\ and Hermitian\\ forms};
\draw (142.36999999999998, -133.40000000000003) rectangle (146.21999999999997,-135.00000000000003);
\draw(146.32, -133.40000000000003) node[anchor=north west,align=left] {Quadratic\\ forms over\\ local rings\\ and fields};
\draw (146.32, -133.40000000000003) rectangle (149.67,-135.50000000000003);
\draw(138.17, -135.60000000000002) node[anchor=north west,align=left] {Forms of\\ degree higher\\ than two};
\draw (138.17, -135.60000000000002) rectangle (142.01999999999998,-137.20000000000002);
\draw(142.11999999999998, -135.60000000000002) node[anchor=north west,align=left] {Quadratic\\ forms over\\ global rings\\ and fields};
\draw (142.11999999999998, -135.60000000000002) rectangle (145.71999999999997,-137.70000000000002);
\draw(145.82, -135.60000000000002) node[anchor=north west,align=left] {Forms\\ over real\\ fields};
\draw (145.82, -135.60000000000002) rectangle (148.67,-137.20000000000002);
\draw(138.17, -137.8) node[anchor=north west,align=left] {Classical\\ groups};
\draw (138.17, -137.8) rectangle (141.01999999999998,-138.9);
\draw(141.11999999999998, -137.8) node[anchor=north west,align=left] {Quadratic\\ spaces;\\ Clifford\\ algebras};
\draw (141.11999999999998, -137.8) rectangle (143.96999999999997,-139.9);
\draw(149.86999999999998, -121.60000000000002) node[anchor=north west,align=left] {\large Elementary number theory};
\draw (149.86999999999998, -121.60000000000002) rectangle (160.76999999999998,-131.90000000000003);
\draw(150.86999999999998, -122.60000000000002) node[anchor=north west,align=left] {Multiplicative\\  structure;\\ Euclidean algorithm;\\  greatest\\ common divisors};
\draw (150.86999999999998, -122.60000000000002) rectangle (156.46999999999997,-125.20000000000002);
\draw(156.56999999999996, -122.60000000000002) node[anchor=north west,align=left] {Factorization;\\ primality};
\draw (156.56999999999996, -122.60000000000002) rectangle (160.66999999999996,-123.70000000000002);
\draw(156.56999999999996, -123.80000000000003) node[anchor=north west,align=left] {Continued\\ fractions};
\draw (156.56999999999996, -123.80000000000003) rectangle (159.41999999999996,-124.90000000000002);
\draw(150.86999999999998, -125.30000000000003) node[anchor=north west,align=left] {Arithmetic\\ functions;\\ related\\ numbers; inversion\\ formulas};
\draw (150.86999999999998, -125.30000000000003) rectangle (155.96999999999997,-127.90000000000002);
\draw(156.06999999999996, -125.30000000000003) node[anchor=north west,align=left] {Congruences;\\ primitive\\  roots;\\ residue systems};
\draw (156.06999999999996, -125.30000000000003) rectangle (160.41999999999996,-127.40000000000002);
\draw(150.86999999999998, -128.00000000000003) node[anchor=north west,align=left] {Radix\\ representation;\\ digital\\ problems};
\draw (150.86999999999998, -128.00000000000003) rectangle (155.21999999999997,-130.10000000000002);
\draw(155.31999999999996, -128.00000000000003) node[anchor=north west,align=left] {Other\\ number\\ representations};
\draw (155.31999999999996, -128.00000000000003) rectangle (159.66999999999996,-129.60000000000002);
\draw(150.86999999999998, -130.20000000000002) node[anchor=north west,align=left] {Power\\ residues,\\ reciprocity};
\draw (150.86999999999998, -130.20000000000002) rectangle (154.21999999999997,-131.8);
\draw(154.31999999999996, -130.20000000000002) node[anchor=north west,align=left] {Primes};
\draw (154.31999999999996, -130.20000000000002) rectangle (156.41999999999996,-130.8);
\draw(160.86999999999998, -121.60000000000002) node[anchor=north west,align=left] {\large Computational number theory};
\draw (160.86999999999998, -121.60000000000002) rectangle (171.26999999999998,-131.90000000000003);
\draw(161.86999999999998, -122.60000000000002) node[anchor=north west,align=left] {Continued\\ fraction\\ calculations\\ (number-theoretic\\ aspects)};
\draw (161.86999999999998, -122.60000000000002) rectangle (166.71999999999997,-125.20000000000002);
\draw(166.81999999999996, -122.60000000000002) node[anchor=north west,align=left] {Factorization};
\draw (166.81999999999996, -122.60000000000002) rectangle (170.66999999999996,-123.20000000000002);
\draw(166.81999999999996, -123.30000000000003) node[anchor=north west,align=left] {Analytic\\ computations};
\draw (166.81999999999996, -123.30000000000003) rectangle (170.41999999999996,-124.40000000000002);
\draw(166.81999999999996, -124.50000000000003) node[anchor=north west,align=left] {Primality};
\draw (166.81999999999996, -124.50000000000003) rectangle (169.66999999999996,-125.10000000000002);
\draw(161.86999999999998, -125.30000000000003) node[anchor=north west,align=left] {Number-theoretic\\ algorithms;\\ complexity};
\draw (161.86999999999998, -125.30000000000003) rectangle (166.46999999999997,-126.90000000000002);
\draw(166.56999999999996, -125.30000000000003) node[anchor=north west,align=left] {Evaluation\\  of\\ number-theoretic\\ constants};
\draw (166.56999999999996, -125.30000000000003) rectangle (171.16999999999996,-127.40000000000002);
\draw(161.86999999999998, -127.50000000000003) node[anchor=north west,align=left] {Algebraic\\  number\\ theory\\ computations};
\draw (161.86999999999998, -127.50000000000003) rectangle (165.46999999999997,-129.60000000000002);
\draw(165.56999999999996, -127.50000000000003) node[anchor=north west,align=left] {Computer\\ solution of\\ Diophantine\\ equations};
\draw (165.56999999999996, -127.50000000000003) rectangle (168.91999999999996,-129.60000000000002);
\draw(161.86999999999998, -129.70000000000002) node[anchor=north west,align=left] {Calculation\\ of integer\\ sequences};
\draw (161.86999999999998, -129.70000000000002) rectangle (165.21999999999997,-131.3);
\draw(165.31999999999996, -129.70000000000002) node[anchor=north west,align=left] {Values of\\ arithmetic\\ functions;\\ tables};
\draw (165.31999999999996, -129.70000000000002) rectangle (168.41999999999996,-131.8);
\draw(171.36999999999998, -121.60000000000002) node[anchor=north west,align=left] {\large Diophantine equations};
\draw (171.36999999999998, -121.60000000000002) rectangle (180.51999999999998,-138.70000000000002);
\draw(172.36999999999998, -122.60000000000002) node[anchor=north west,align=left] {Higher\\ degree equations;\\ Fermat’s\\ equation};
\draw (172.36999999999998, -122.60000000000002) rectangle (177.21999999999997,-124.70000000000002);
\draw(177.31999999999996, -122.60000000000002) node[anchor=north west,align=left] {Rational\\ numbers\\ as sums of\\ fractions};
\draw (177.31999999999996, -122.60000000000002) rectangle (180.41999999999996,-124.70000000000002);
\draw(172.36999999999998, -124.80000000000003) node[anchor=north west,align=left] {Counting\\ solutions\\ of Diophantine\\ equations};
\draw (172.36999999999998, -124.80000000000003) rectangle (176.46999999999997,-126.90000000000002);
\draw(176.56999999999996, -124.80000000000003) node[anchor=north west,align=left] {\(p\)-adic and\\  power\\ series fields};
\draw (176.56999999999996, -124.80000000000003) rectangle (180.41999999999996,-126.40000000000002);
\draw(172.36999999999998, -127.00000000000003) node[anchor=north west,align=left] {Multiplicative\\ and\\ norm form\\ equations};
\draw (172.36999999999998, -127.00000000000003) rectangle (176.46999999999997,-129.10000000000002);
\draw(176.56999999999996, -127.00000000000003) node[anchor=north west,align=left] {Quadratic\\ and bilinear\\ Diophantine\\ equations};
\draw (176.56999999999996, -127.00000000000003) rectangle (180.16999999999996,-129.10000000000002);
\draw(172.36999999999998, -129.20000000000002) node[anchor=north west,align=left] {Diophantine\\ equations\\  in\\ many variables};
\draw (172.36999999999998, -129.20000000000002) rectangle (176.46999999999997,-131.3);
\draw(176.56999999999996, -129.20000000000002) node[anchor=north west,align=left] {Diophantine\\ inequalities};
\draw (176.56999999999996, -129.20000000000002) rectangle (180.16999999999996,-130.3);
\draw(172.36999999999998, -131.40000000000003) node[anchor=north west,align=left] {Representation\\ problems};
\draw (172.36999999999998, -131.40000000000003) rectangle (176.46999999999997,-132.50000000000003);
\draw(176.56999999999996, -131.40000000000003) node[anchor=north west,align=left] {Linear\\ Diophantine\\ equations};
\draw (176.56999999999996, -131.40000000000003) rectangle (179.91999999999996,-133.00000000000003);
\draw(172.36999999999998, -133.10000000000002) node[anchor=north west,align=left] {Cubic and\\  quartic\\ Diophantine\\ equations};
\draw (172.36999999999998, -133.10000000000002) rectangle (175.71999999999997,-135.20000000000002);
\draw(175.81999999999996, -133.10000000000002) node[anchor=north west,align=left] {Thue-Mahler\\ equations};
\draw (175.81999999999996, -133.10000000000002) rectangle (179.16999999999996,-134.20000000000002);
\draw(172.36999999999998, -135.3) node[anchor=north west,align=left] {Exponential\\ Diophantine\\ equations};
\draw (172.36999999999998, -135.3) rectangle (175.71999999999997,-136.9);
\draw(175.81999999999996, -135.3) node[anchor=north west,align=left] {Congruences\\ in many\\ variables};
\draw (175.81999999999996, -135.3) rectangle (179.16999999999996,-136.9);
\draw(172.36999999999998, -137.00000000000003) node[anchor=north west,align=left] {The\\ Frobenius\\ problem};
\draw (172.36999999999998, -137.00000000000003) rectangle (175.21999999999997,-138.60000000000002);
\draw(187.67, -1) node[anchor=north west,align=left] {\LARGE Group theory and generalizations};
\draw (187.67, -1) rectangle (233.92,-58.80000000000001);
\draw(188.67, -2) node[anchor=north west,align=left] {\large Groupoids (i.e. small categories in which all morphisms are isomorphisms)};
\draw (188.67, -2) rectangle (211.89999999999998,-5.7);
\draw(189.67, -3) node[anchor=north west,align=left] {Groupoids (i.e.\\ small categories\\  in which\\ all morphisms\\ are isomorphisms)};
\draw (189.67, -3) rectangle (194.51999999999998,-5.6);
\draw(212.0, -2) node[anchor=north west,align=left] {\large Structure and classification of infinite or finite groups};
\draw (212.0, -2) rectangle (233.15,-11.1);
\draw(213.0, -3) node[anchor=north west,align=left] {Free products of\\ groups, free products\\ with amalgamation,\\ Higman-Neumann-Neumann\\ extensions,\\ and generalizations};
\draw (213.0, -3) rectangle (219.1,-6.1);
\draw(219.2, -3) node[anchor=north west,align=left] {Chains and\\ lattices of\\ subgroups, subnormal\\ subgroups};
\draw (219.2, -3) rectangle (224.79999999999998,-5.1);
\draw(224.9, -3) node[anchor=north west,align=left] {Extensions,\\ wreath products,\\ and other\\ compositions\\ of groups};
\draw (224.9, -3) rectangle (229.5,-5.6);
\draw(229.6, -3) node[anchor=north west,align=left] {Groups with\\ a \(BN\)-pair;\\ buildings};
\draw (229.6, -3) rectangle (232.95,-4.6);
\draw(213.0, -6.2) node[anchor=north west,align=left] {Residual\\ properties and\\ generalizations;\\  residually\\ finite groups};
\draw (213.0, -6.2) rectangle (217.6,-8.8);
\draw(217.7, -6.2) node[anchor=north west,align=left] {Automorphisms\\  of\\ infinite groups};
\draw (217.7, -6.2) rectangle (222.04999999999998,-7.800000000000001);
\draw(222.15, -6.2) node[anchor=north west,align=left] {Quasivarieties\\ and\\ varieties\\ of groups};
\draw (222.15, -6.2) rectangle (226.25,-8.3);
\draw(226.35, -6.2) node[anchor=north west,align=left] {Free\\ nonabelian\\ groups};
\draw (226.35, -6.2) rectangle (229.45,-7.800000000000001);
\draw(229.55, -6.2) node[anchor=north west,align=left] {Local\\ properties\\ of groups};
\draw (229.55, -6.2) rectangle (232.65,-7.800000000000001);
\draw(213.0, -8.9) node[anchor=north west,align=left] {General\\ structure\\ theorems\\ for groups};
\draw (213.0, -8.9) rectangle (216.1,-11.0);
\draw(216.2, -8.9) node[anchor=north west,align=left] {Conjugacy\\  classes\\ for groups};
\draw (216.2, -8.9) rectangle (219.29999999999998,-10.5);
\draw(219.4, -8.9) node[anchor=north west,align=left] {Subgroup\\ theorems;\\ subgroup\\ growth};
\draw (219.4, -8.9) rectangle (222.25,-11.0);
\draw(222.35, -8.9) node[anchor=north west,align=left] {Limits,\\ profinite\\ groups};
\draw (222.35, -8.9) rectangle (225.2,-10.5);
\draw(225.3, -8.9) node[anchor=north west,align=left] {Maximal\\ subgroups};
\draw (225.3, -8.9) rectangle (228.15,-10.0);
\draw(228.25, -8.9) node[anchor=north west,align=left] {Groups\\ acting\\ on trees};
\draw (228.25, -8.9) rectangle (230.85,-10.5);
\draw(230.95, -8.9) node[anchor=north west,align=left] {Simple\\ groups};
\draw (230.95, -8.9) rectangle (233.04999999999998,-10.0);
\draw(188.67, -5.8) node[anchor=north west,align=left] {\large Connections of group theory with homological algebra and category theory};
\draw (188.67, -5.8) rectangle (211.58999999999997,-9.0);
\draw(189.67, -6.8) node[anchor=north west,align=left] {Homological\\ methods\\  in\\ group theory};
\draw (189.67, -6.8) rectangle (193.26999999999998,-8.9);
\draw(193.36999999999998, -6.8) node[anchor=north west,align=left] {Cohomology\\ of groups};
\draw (193.36999999999998, -6.8) rectangle (196.46999999999997,-7.9);
\draw(196.57, -6.8) node[anchor=north west,align=left] {Category\\  of\\ groups};
\draw (196.57, -6.8) rectangle (199.17,-8.4);
\draw(188.67, -9.1) node[anchor=north west,align=left] {\large Computational methods\\  for problems\\ pertaining to group theory};
\draw (188.67, -9.1) rectangle (197.32999999999998,-10.7);
\draw(188.67, -11.2) node[anchor=north west,align=left] {\large Special aspects of infinite or finite groups};
\draw (188.67, -11.2) rectangle (205.51999999999998,-31.8);
\draw(189.67, -12.2) node[anchor=north west,align=left] {Word problems,\\ other decision\\ problems, connections\\  with logic\\ and automata\\ (group-theoretic aspects)};
\draw (189.67, -12.2) rectangle (196.51999999999998,-15.299999999999999);
\draw(196.61999999999998, -12.2) node[anchor=north west,align=left] {Cancellation\\ theory of\\ groups; application\\  of van\\ Kampen diagrams};
\draw (196.61999999999998, -12.2) rectangle (201.96999999999997,-14.799999999999999);
\draw(202.07, -12.2) node[anchor=north west,align=left] {Groups of\\  finite\\ Morley rank};
\draw (202.07, -12.2) rectangle (205.42,-13.799999999999999);
\draw(189.67, -15.399999999999999) node[anchor=north west,align=left] {Generators,\\ relations,\\ and presentations\\ of groups};
\draw (189.67, -15.399999999999999) rectangle (194.51999999999998,-17.5);
\draw(194.61999999999998, -15.399999999999999) node[anchor=north west,align=left] {Representations\\  of groups\\ as automorphism\\  groups of\\ algebraic systems};
\draw (194.61999999999998, -15.399999999999999) rectangle (199.46999999999997,-18.0);
\draw(199.57, -15.399999999999999) node[anchor=north west,align=left] {Algebraic\\ geometry over\\ groups; equations\\ over groups};
\draw (199.57, -15.399999999999999) rectangle (204.42,-17.5);
\draw(189.67, -18.1) node[anchor=north west,align=left] {Generalizations\\ of solvable\\  and\\ nilpotent groups};
\draw (189.67, -18.1) rectangle (194.26999999999998,-20.200000000000003);
\draw(194.36999999999998, -18.1) node[anchor=north west,align=left] {Fundamental\\ groups and their\\ automorphisms\\ (group-theoretic\\ aspects)};
\draw (194.36999999999998, -18.1) rectangle (198.96999999999997,-20.700000000000003);
\draw(199.07, -18.1) node[anchor=north west,align=left] {Reflection\\ and Coxeter\\  groups\\ (group-theoretic\\ aspects)};
\draw (199.07, -18.1) rectangle (203.67,-20.700000000000003);
\draw(189.67, -20.8) node[anchor=north west,align=left] {Ordered\\ groups\\ (group-theoretic\\ aspects)};
\draw (189.67, -20.8) rectangle (194.26999999999998,-22.900000000000002);
\draw(194.36999999999998, -20.8) node[anchor=north west,align=left] {Derived series,\\  central\\ series, and\\ generalizations\\ for groups};
\draw (194.36999999999998, -20.8) rectangle (198.71999999999997,-23.400000000000002);
\draw(198.82, -20.8) node[anchor=north west,align=left] {Other classes\\  of groups\\ defined by\\ subgroup chains};
\draw (198.82, -20.8) rectangle (203.17,-22.900000000000002);
\draw(189.67, -23.5) node[anchor=north west,align=left] {FC-groups\\  and\\ their\\ generalizations};
\draw (189.67, -23.5) rectangle (194.01999999999998,-25.6);
\draw(194.11999999999998, -23.5) node[anchor=north west,align=left] {Solvable\\ groups,\\ supersolvable\\ groups};
\draw (194.11999999999998, -23.5) rectangle (197.96999999999997,-25.6);
\draw(198.07, -23.5) node[anchor=north west,align=left] {Periodic\\ groups;\\ locally\\ finite groups};
\draw (198.07, -23.5) rectangle (201.92,-25.6);
\draw(202.01999999999998, -23.5) node[anchor=north west,align=left] {Commutator\\ calculus};
\draw (202.01999999999998, -23.5) rectangle (205.11999999999998,-24.6);
\draw(189.67, -25.7) node[anchor=north west,align=left] {Hyperbolic\\ groups and\\ nonpositively\\ curved groups};
\draw (189.67, -25.7) rectangle (193.51999999999998,-27.8);
\draw(193.61999999999998, -25.7) node[anchor=north west,align=left] {Automorphism\\ groups\\ of groups};
\draw (193.61999999999998, -25.7) rectangle (197.21999999999997,-27.3);
\draw(197.32, -25.7) node[anchor=north west,align=left] {Braid\\ groups;\\ Artin groups};
\draw (197.32, -25.7) rectangle (200.92,-27.3);
\draw(201.01999999999998, -25.7) node[anchor=north west,align=left] {Other groups\\  related\\ to topology\\ or analysis};
\draw (201.01999999999998, -25.7) rectangle (204.61999999999998,-27.8);
\draw(189.67, -27.9) node[anchor=north west,align=left] {Formations\\ of groups,\\ Fitting\\ classes};
\draw (189.67, -27.9) rectangle (192.76999999999998,-30.0);
\draw(192.86999999999998, -27.9) node[anchor=north west,align=left] {Associated\\  Lie\\ structures\\ for groups};
\draw (192.86999999999998, -27.9) rectangle (195.96999999999997,-30.0);
\draw(196.07, -27.9) node[anchor=north west,align=left] {Engel\\ conditions};
\draw (196.07, -27.9) rectangle (199.17,-29.0);
\draw(199.26999999999998, -27.9) node[anchor=north west,align=left] {Asymptotic\\ properties\\ of groups};
\draw (199.26999999999998, -27.9) rectangle (202.36999999999998,-29.5);
\draw(202.47, -27.9) node[anchor=north west,align=left] {Nilpotent\\ groups};
\draw (202.47, -27.9) rectangle (205.32,-29.0);
\draw(189.67, -30.099999999999998) node[anchor=north west,align=left] {Geometric\\ group\\ theory};
\draw (189.67, -30.099999999999998) rectangle (192.51999999999998,-31.7);
\draw(205.62, -11.2) node[anchor=north west,align=left] {\large Linear algebraic groups and related topics};
\draw (205.62, -11.2) rectangle (221.47,-22.5);
\draw(206.62, -12.2) node[anchor=north west,align=left] {Quantum groups\\  (quantized\\ function algebras)\\  and their\\ representations};
\draw (206.62, -12.2) rectangle (211.72,-14.799999999999999);
\draw(211.82, -12.2) node[anchor=north west,align=left] {Linear algebraic\\  groups\\ over adèles\\ and other\\ rings and schemes};
\draw (211.82, -12.2) rectangle (216.67,-14.799999999999999);
\draw(216.77, -12.2) node[anchor=north west,align=left] {Representation\\ theory for\\  linear\\ algebraic groups};
\draw (216.77, -12.2) rectangle (221.37,-14.299999999999999);
\draw(206.62, -14.899999999999999) node[anchor=north west,align=left] {Structure\\ theory for\\ linear algebraic\\ groups};
\draw (206.62, -14.899999999999999) rectangle (211.22,-17.0);
\draw(211.32, -14.899999999999999) node[anchor=north west,align=left] {Cohomology\\ theory for\\ linear\\ algebraic groups};
\draw (211.32, -14.899999999999999) rectangle (215.92,-17.0);
\draw(216.02, -14.899999999999999) node[anchor=north west,align=left] {Linear\\ algebraic groups\\  over\\ arbitrary fields};
\draw (216.02, -14.899999999999999) rectangle (220.62,-17.0);
\draw(206.62, -17.1) node[anchor=north west,align=left] {Linear algebraic\\ groups over\\ the reals, the\\  complexes,\\ the quaternions};
\draw (206.62, -17.1) rectangle (211.22,-19.700000000000003);
\draw(211.32, -17.1) node[anchor=north west,align=left] {Linear algebraic\\  groups\\ over local\\ fields and\\ their integers};
\draw (211.32, -17.1) rectangle (215.92,-19.700000000000003);
\draw(216.02, -17.1) node[anchor=north west,align=left] {Linear algebraic\\  groups\\ over global\\ fields and\\ their integers};
\draw (216.02, -17.1) rectangle (220.62,-19.700000000000003);
\draw(206.62, -19.8) node[anchor=north west,align=left] {Linear\\ algebraic\\ groups over\\ finite fields};
\draw (206.62, -19.8) rectangle (210.47,-21.900000000000002);
\draw(210.57, -19.8) node[anchor=north west,align=left] {Applications\\ of linear\\  algebraic\\ groups to\\ the sciences};
\draw (210.57, -19.8) rectangle (214.17,-22.400000000000002);
\draw(214.27, -19.8) node[anchor=north west,align=left] {Exceptional\\ groups};
\draw (214.27, -19.8) rectangle (217.62,-20.900000000000002);
\draw(214.27, -21.0) node[anchor=north west,align=left] {Kac-Moody\\ groups};
\draw (214.27, -21.0) rectangle (217.12,-22.1);
\draw(217.72, -19.8) node[anchor=north west,align=left] {Schur and\\ \(q\)-Schur\\ algebras};
\draw (217.72, -19.8) rectangle (220.57,-21.400000000000002);
\draw(205.62, -22.6) node[anchor=north west,align=left] {\large Probabilistic methods in group theory};
\draw (205.62, -22.6) rectangle (217.69,-25.8);
\draw(206.62, -23.6) node[anchor=north west,align=left] {Probabilistic\\ methods\\  in\\ group theory};
\draw (206.62, -23.6) rectangle (210.47,-25.700000000000003);
\draw(205.62, -25.900000000000002) node[anchor=north west,align=left] {\large Foundations};
\draw (205.62, -25.900000000000002) rectangle (214.52,-30.800000000000004);
\draw(206.62, -26.900000000000002) node[anchor=north west,align=left] {Metamathematical\\ considerations\\ in group theory};
\draw (206.62, -26.900000000000002) rectangle (211.22,-28.500000000000004);
\draw(206.62, -28.6) node[anchor=north west,align=left] {Axiomatics\\ and elementary\\ properties\\ of groups};
\draw (206.62, -28.6) rectangle (210.72,-30.700000000000003);
\draw(210.82, -28.6) node[anchor=north west,align=left] {Applications\\  of\\ logic to\\ group theory};
\draw (210.82, -28.6) rectangle (214.42,-30.700000000000003);
\draw(221.57, -11.2) node[anchor=north west,align=left] {\large Abstract finite groups};
\draw (221.57, -11.2) rectangle (233.47,-26.9);
\draw(222.57, -12.2) node[anchor=north west,align=left] {Finite solvable\\ groups, theory\\ of formations,\\ Schunck classes,\\  Fitting\\ classes, \(\pi\)-length, ranks};
\draw (222.57, -12.2) rectangle (229.17,-15.299999999999999);
\draw(229.26999999999998, -12.2) node[anchor=north west,align=left] {Finite simple\\  groups\\ and their\\ classification};
\draw (229.26999999999998, -12.2) rectangle (233.36999999999998,-14.299999999999999);
\draw(222.57, -15.399999999999999) node[anchor=north west,align=left] {Arithmetic and\\ combinatorial\\  problems\\ involving abstract\\ finite groups};
\draw (222.57, -15.399999999999999) rectangle (227.67,-18.0);
\draw(227.76999999999998, -15.399999999999999) node[anchor=north west,align=left] {Simple\\ groups: sporadic\\ groups};
\draw (227.76999999999998, -15.399999999999999) rectangle (232.36999999999998,-17.0);
\draw(222.57, -18.1) node[anchor=north west,align=left] {Sylow\\ subgroups, Sylow\\ properties,\\ \(\pi\)-groups,\\ \(\pi\)-structure};
\draw (222.57, -18.1) rectangle (227.17,-20.700000000000003);
\draw(227.26999999999998, -18.1) node[anchor=north west,align=left] {Simple groups:\\ alternating\\ groups\\ and groups\\ of Lie type};
\draw (227.26999999999998, -18.1) rectangle (231.36999999999998,-20.700000000000003);
\draw(222.57, -20.8) node[anchor=north west,align=left] {Special\\ subgroups\\ (Frattini,\\ Fitting, etc.)};
\draw (222.57, -20.8) rectangle (226.67,-22.900000000000002);
\draw(226.76999999999998, -20.8) node[anchor=north west,align=left] {Subnormal\\ subgroups of\\  abstract\\ finite groups};
\draw (226.76999999999998, -20.8) rectangle (230.61999999999998,-22.900000000000002);
\draw(222.57, -23.0) node[anchor=north west,align=left] {Products of\\  subgroups\\ of abstract\\ finite groups};
\draw (222.57, -23.0) rectangle (226.42,-25.1);
\draw(226.51999999999998, -23.0) node[anchor=north west,align=left] {Automorphisms\\  of\\ abstract\\ finite groups};
\draw (226.51999999999998, -23.0) rectangle (230.36999999999998,-25.1);
\draw(230.47, -23.0) node[anchor=north west,align=left] {Finite\\ nilpotent\\ groups,\\ \(p\)-groups};
\draw (230.47, -23.0) rectangle (233.32,-25.1);
\draw(222.57, -25.2) node[anchor=north west,align=left] {Series and\\  lattices\\ of subgroups};
\draw (222.57, -25.2) rectangle (226.17,-26.8);
\draw(188.67, -31.900000000000002) node[anchor=north west,align=left] {\large Representation theory of groups};
\draw (188.67, -31.900000000000002) rectangle (202.32,-49.300000000000004);
\draw(189.67, -32.900000000000006) node[anchor=north west,align=left] {Group rings of\\ infinite groups\\  and their\\ modules\\ (group-theoretic aspects)};
\draw (189.67, -32.900000000000006) rectangle (196.51999999999998,-35.50000000000001);
\draw(196.61999999999998, -32.900000000000006) node[anchor=north west,align=left] {Representations\\  of\\ infinite symmetric\\ groups};
\draw (196.61999999999998, -32.900000000000006) rectangle (201.71999999999997,-35.00000000000001);
\draw(189.67, -35.6) node[anchor=north west,align=left] {Group rings of\\ finite groups\\  and their\\ modules (group-theoretic\\ aspects)};
\draw (189.67, -35.6) rectangle (196.26999999999998,-38.2);
\draw(196.36999999999998, -35.6) node[anchor=north west,align=left] {Applications of\\ group representations\\ to physics\\  and other\\ areas of science};
\draw (196.36999999999998, -35.6) rectangle (202.21999999999997,-38.2);
\draw(189.67, -38.300000000000004) node[anchor=north west,align=left] {Representations\\  of\\ finite\\ symmetric groups};
\draw (189.67, -38.300000000000004) rectangle (194.26999999999998,-40.400000000000006);
\draw(194.36999999999998, -38.300000000000004) node[anchor=north west,align=left] {Hecke\\ algebras and\\ their\\ representations};
\draw (194.36999999999998, -38.300000000000004) rectangle (198.71999999999997,-40.400000000000006);
\draw(189.67, -40.5) node[anchor=north west,align=left] {Integral\\ representations\\  of\\ finite groups};
\draw (189.67, -40.5) rectangle (194.01999999999998,-42.6);
\draw(194.11999999999998, -40.5) node[anchor=north west,align=left] {\(p\)-adic\\ representations\\  of\\ finite groups};
\draw (194.11999999999998, -40.5) rectangle (198.46999999999997,-42.6);
\draw(189.67, -42.7) node[anchor=north west,align=left] {Integral\\ representations\\  of\\ infinite groups};
\draw (189.67, -42.7) rectangle (194.01999999999998,-44.800000000000004);
\draw(194.11999999999998, -42.7) node[anchor=north west,align=left] {Ordinary\\ representations\\ and\\ characters};
\draw (194.11999999999998, -42.7) rectangle (198.46999999999997,-44.800000000000004);
\draw(189.67, -44.900000000000006) node[anchor=north west,align=left] {Modular\\ representations\\ and\\ characters};
\draw (189.67, -44.900000000000006) rectangle (194.01999999999998,-47.00000000000001);
\draw(194.11999999999998, -44.900000000000006) node[anchor=north west,align=left] {Projective\\ representations\\ and\\ multipliers};
\draw (194.11999999999998, -44.900000000000006) rectangle (198.46999999999997,-47.00000000000001);
\draw(189.67, -47.1) node[anchor=north west,align=left] {Representations\\  of\\ finite groups\\ of Lie type};
\draw (189.67, -47.1) rectangle (194.01999999999998,-49.2);
\draw(194.11999999999998, -47.1) node[anchor=north west,align=left] {Representations\\ of sporadic\\ groups};
\draw (194.11999999999998, -47.1) rectangle (198.46999999999997,-48.7);
\draw(202.42, -31.900000000000002) node[anchor=north west,align=left] {\large Permutation groups};
\draw (202.42, -31.900000000000002) rectangle (213.82,-43.400000000000006);
\draw(203.42, -32.900000000000006) node[anchor=north west,align=left] {Finite automorphism\\  groups of\\ algebraic,\\ geometric, or\\ combinatorial structures};
\draw (203.42, -32.900000000000006) rectangle (210.01999999999998,-35.50000000000001);
\draw(210.11999999999998, -32.900000000000006) node[anchor=north west,align=left] {Infinite\\ automorphism\\ groups};
\draw (210.11999999999998, -32.900000000000006) rectangle (213.71999999999997,-34.50000000000001);
\draw(203.42, -35.6) node[anchor=north west,align=left] {General\\ theory for\\ finite permutation\\ groups};
\draw (203.42, -35.6) rectangle (208.51999999999998,-37.7);
\draw(208.61999999999998, -35.6) node[anchor=north west,align=left] {General\\ theory for\\ infinite\\ permutation groups};
\draw (208.61999999999998, -35.6) rectangle (213.71999999999997,-37.7);
\draw(203.42, -37.800000000000004) node[anchor=north west,align=left] {Characterization\\ theorems\\ for permutation\\ groups};
\draw (203.42, -37.800000000000004) rectangle (208.01999999999998,-39.900000000000006);
\draw(208.11999999999998, -37.800000000000004) node[anchor=north west,align=left] {Multiply\\ transitive\\ infinite groups};
\draw (208.11999999999998, -37.800000000000004) rectangle (212.46999999999997,-39.400000000000006);
\draw(203.42, -40.0) node[anchor=north west,align=left] {Subgroups\\ of symmetric\\ groups};
\draw (203.42, -40.0) rectangle (207.01999999999998,-41.6);
\draw(207.11999999999998, -40.0) node[anchor=north west,align=left] {Multiply\\ transitive\\ finite\\ groups};
\draw (207.11999999999998, -40.0) rectangle (210.21999999999997,-42.1);
\draw(210.32, -40.0) node[anchor=north west,align=left] {Primitive\\ groups};
\draw (210.32, -40.0) rectangle (213.17,-41.1);
\draw(203.42, -42.2) node[anchor=north west,align=left] {Symmetric\\ groups};
\draw (203.42, -42.2) rectangle (206.26999999999998,-43.300000000000004);
\draw(202.42, -43.50000000000001) node[anchor=north west,align=left] {\large History of\\ group theory};
\draw (202.42, -43.50000000000001) rectangle (206.73999999999998,-44.60000000000001);
\draw(213.92, -31.900000000000002) node[anchor=north west,align=left] {\large Semigroups};
\draw (213.92, -31.900000000000002) rectangle (224.57,-49.5);
\draw(214.92, -32.900000000000006) node[anchor=north west,align=left] {Semigroup\\ rings,\\ multiplicative\\ semigroups of rings};
\draw (214.92, -32.900000000000006) rectangle (220.26999999999998,-35.00000000000001);
\draw(220.36999999999998, -32.900000000000006) node[anchor=north west,align=left] {Commutative\\ semigroups};
\draw (220.36999999999998, -32.900000000000006) rectangle (223.71999999999997,-34.00000000000001);
\draw(214.92, -35.1) node[anchor=north west,align=left] {Representation\\  of\\ semigroups;\\ actions of\\ semigroups on sets};
\draw (214.92, -35.1) rectangle (220.01999999999998,-37.7);
\draw(220.11999999999998, -35.1) node[anchor=north west,align=left] {Varieties\\ and\\ pseudovarieties\\ of semigroups};
\draw (220.11999999999998, -35.1) rectangle (224.46999999999997,-37.2);
\draw(214.92, -37.800000000000004) node[anchor=north west,align=left] {Semigroups\\ in automata\\ theory,\\ linguistics, etc.};
\draw (214.92, -37.800000000000004) rectangle (219.76999999999998,-39.900000000000006);
\draw(219.86999999999998, -37.800000000000004) node[anchor=north west,align=left] {Free semigroups,\\ generators and\\ relations,\\ word problems};
\draw (219.86999999999998, -37.800000000000004) rectangle (224.46999999999997,-39.900000000000006);
\draw(214.92, -40.0) node[anchor=north west,align=left] {Semigroups\\ of\\ transformations,\\ relations,\\ partitions, etc.};
\draw (214.92, -40.0) rectangle (219.51999999999998,-42.6);
\draw(219.61999999999998, -40.0) node[anchor=north west,align=left] {Connections of\\  semigroups\\ with homological\\  algebra and\\ category theory};
\draw (219.61999999999998, -40.0) rectangle (224.21999999999997,-42.6);
\draw(214.92, -42.7) node[anchor=north west,align=left] {Generalizations\\ of\\ semigroups};
\draw (214.92, -42.7) rectangle (219.26999999999998,-44.300000000000004);
\draw(219.36999999999998, -42.7) node[anchor=north west,align=left] {General\\ structure\\ theory for\\ semigroups};
\draw (219.36999999999998, -42.7) rectangle (222.46999999999997,-44.800000000000004);
\draw(214.92, -44.900000000000006) node[anchor=north west,align=left] {Radical\\ theory for\\ semigroups};
\draw (214.92, -44.900000000000006) rectangle (218.01999999999998,-46.50000000000001);
\draw(218.11999999999998, -44.900000000000006) node[anchor=north west,align=left] {Ideal\\ theory for\\ semigroups};
\draw (218.11999999999998, -44.900000000000006) rectangle (221.21999999999997,-46.50000000000001);
\draw(221.32, -44.900000000000006) node[anchor=north west,align=left] {Arithmetic\\ theory of\\ semigroups};
\draw (221.32, -44.900000000000006) rectangle (224.42,-46.50000000000001);
\draw(214.92, -46.6) node[anchor=north west,align=left] {Mappings\\  of\\ semigroups};
\draw (214.92, -46.6) rectangle (218.01999999999998,-48.2);
\draw(218.11999999999998, -46.6) node[anchor=north west,align=left] {Regular\\ semigroups};
\draw (218.11999999999998, -46.6) rectangle (221.21999999999997,-47.7);
\draw(221.32, -46.6) node[anchor=north west,align=left] {Inverse\\ semigroups};
\draw (221.32, -46.6) rectangle (224.42,-47.7);
\draw(214.92, -48.3) node[anchor=north west,align=left] {Orthodox\\ semigroups};
\draw (214.92, -48.3) rectangle (218.01999999999998,-49.4);
\draw(218.11999999999998, -48.3) node[anchor=north west,align=left] {Algebraic\\ monoids};
\draw (218.11999999999998, -48.3) rectangle (220.96999999999997,-49.4);
\draw(224.67, -31.900000000000002) node[anchor=north west,align=left] {\large Abelian groups};
\draw (224.67, -31.900000000000002) rectangle (233.82,-46.6);
\draw(225.67, -32.900000000000006) node[anchor=north west,align=left] {Direct sums,\\ direct products,\\  etc. for\\ abelian groups};
\draw (225.67, -32.900000000000006) rectangle (230.26999999999998,-35.00000000000001);
\draw(230.36999999999998, -32.900000000000006) node[anchor=north west,align=left] {Subgroups\\ of abelian\\ groups};
\draw (230.36999999999998, -32.900000000000006) rectangle (233.46999999999997,-34.50000000000001);
\draw(225.67, -35.1) node[anchor=north west,align=left] {Torsion groups,\\  primary\\ groups and\\ generalized\\ primary groups};
\draw (225.67, -35.1) rectangle (230.01999999999998,-37.7);
\draw(230.11999999999998, -35.1) node[anchor=north west,align=left] {Torsion-free\\ groups,\\ finite rank};
\draw (230.11999999999998, -35.1) rectangle (233.71999999999997,-36.7);
\draw(225.67, -37.800000000000004) node[anchor=north west,align=left] {Automorphisms,\\ homomorphisms,\\ endomorphisms,\\ etc. for\\ abelian groups};
\draw (225.67, -37.800000000000004) rectangle (229.76999999999998,-40.400000000000006);
\draw(229.86999999999998, -37.800000000000004) node[anchor=north west,align=left] {Torsion-free\\ groups,\\ infinite rank};
\draw (229.86999999999998, -37.800000000000004) rectangle (233.71999999999997,-39.400000000000006);
\draw(225.67, -40.5) node[anchor=north west,align=left] {Homological\\  and\\ categorical\\ methods for\\ abelian groups};
\draw (225.67, -40.5) rectangle (229.76999999999998,-43.1);
\draw(229.86999999999998, -40.5) node[anchor=north west,align=left] {Extensions\\ of abelian\\ groups};
\draw (229.86999999999998, -40.5) rectangle (232.96999999999997,-42.1);
\draw(225.67, -43.2) node[anchor=north west,align=left] {Topological\\ methods\\  for\\ abelian groups};
\draw (225.67, -43.2) rectangle (229.76999999999998,-45.300000000000004);
\draw(229.86999999999998, -43.2) node[anchor=north west,align=left] {Finite\\ abelian\\ groups};
\draw (229.86999999999998, -43.2) rectangle (232.21999999999997,-44.800000000000004);
\draw(225.67, -45.400000000000006) node[anchor=north west,align=left] {Mixed\\ groups};
\draw (225.67, -45.400000000000006) rectangle (227.76999999999998,-46.50000000000001);
\draw(188.67, -49.60000000000001) node[anchor=north west,align=left] {\large Other generalizations of groups};
\draw (188.67, -49.60000000000001) rectangle (199.01999999999998,-55.70000000000001);
\draw(189.67, -50.60000000000001) node[anchor=north west,align=left] {Sets with a\\  single\\ binary operation\\ (groupoids)};
\draw (189.67, -50.60000000000001) rectangle (194.26999999999998,-52.70000000000001);
\draw(194.36999999999998, -50.60000000000001) node[anchor=north west,align=left] {Ternary systems\\  (heaps,\\ semiheaps,\\ heapoids, etc.)};
\draw (194.36999999999998, -50.60000000000001) rectangle (198.71999999999997,-52.70000000000001);
\draw(189.67, -52.80000000000001) node[anchor=north west,align=left] {Loops,\\ quasigroups};
\draw (189.67, -52.80000000000001) rectangle (193.01999999999998,-53.90000000000001);
\draw(193.11999999999998, -52.80000000000001) node[anchor=north west,align=left] {Hypergroups};
\draw (193.11999999999998, -52.80000000000001) rectangle (196.46999999999997,-53.40000000000001);
\draw(196.57, -52.80000000000001) node[anchor=north west,align=left] {\(n\)-ary\\ systems\\ \((n\ge~3)\)};
\draw (196.57, -52.80000000000001) rectangle (198.92,-54.40000000000001);
\draw(189.67, -54.50000000000001) node[anchor=north west,align=left] {Fuzzy\\ groups};
\draw (189.67, -54.50000000000001) rectangle (191.76999999999998,-55.60000000000001);
\draw(199.11999999999998, -49.60000000000001) node[anchor=north west,align=left] {\large Other groups of matrices};
\draw (199.11999999999998, -49.60000000000001) rectangle (209.01999999999998,-58.70000000000001);
\draw(200.11999999999998, -50.60000000000001) node[anchor=north west,align=left] {Unimodular\\ groups, congruence\\ subgroups\\ (group-theoretic\\ aspects)};
\draw (200.11999999999998, -50.60000000000001) rectangle (205.21999999999997,-53.20000000000001);
\draw(205.31999999999996, -50.60000000000001) node[anchor=north west,align=left] {Other matrix\\  groups\\ over fields};
\draw (205.31999999999996, -50.60000000000001) rectangle (208.91999999999996,-52.20000000000001);
\draw(200.11999999999998, -53.30000000000001) node[anchor=north west,align=left] {Fuchsian groups\\  and their\\ generalizations\\ (group-theoretic\\ aspects)};
\draw (200.11999999999998, -53.30000000000001) rectangle (204.71999999999997,-55.90000000000001);
\draw(204.81999999999996, -53.30000000000001) node[anchor=north west,align=left] {Other\\ matrix groups\\  over\\ finite fields};
\draw (204.81999999999996, -53.30000000000001) rectangle (208.66999999999996,-55.40000000000001);
\draw(200.11999999999998, -56.00000000000001) node[anchor=north west,align=left] {Other geometric\\ groups,\\ including\\ crystallographic\\ groups};
\draw (200.11999999999998, -56.00000000000001) rectangle (204.71999999999997,-58.60000000000001);
\draw(204.81999999999996, -56.00000000000001) node[anchor=north west,align=left] {Other matrix\\  groups\\ over rings};
\draw (204.81999999999996, -56.00000000000001) rectangle (208.41999999999996,-57.60000000000001);
\draw(187.67, -58.90000000000001) node[anchor=north west,align=left] {\LARGE Category theory; homological algebra};
\draw (187.67, -58.90000000000001) rectangle (230.07,-119.70000000000002);
\draw(188.67, -59.90000000000001) node[anchor=north west,align=left] {\large Homological algebra in category theory, derived categories and functors};
\draw (188.67, -59.90000000000001) rectangle (215.42,-69.00000000000001);
\draw(189.67, -60.90000000000001) node[anchor=north west,align=left] {Relative homological\\  algebra,\\ projective classes\\ (category-theoretic\\ aspects)};
\draw (189.67, -60.90000000000001) rectangle (195.26999999999998,-63.500000000000014);
\draw(195.36999999999998, -60.90000000000001) node[anchor=north west,align=left] {Projectives\\ and injectives\\ (category-theoretic\\ aspects)};
\draw (195.36999999999998, -60.90000000000001) rectangle (200.71999999999997,-63.000000000000014);
\draw(200.82, -60.90000000000001) node[anchor=north west,align=left] {Resolutions;\\  derived\\ functors\\ (category-theoretic\\ aspects)};
\draw (200.82, -60.90000000000001) rectangle (206.17,-63.500000000000014);
\draw(206.26999999999998, -60.90000000000001) node[anchor=north west,align=left] {Ext and Tor,\\ generalizations,\\ Künneth formula\\ (category-theoretic\\ aspects)};
\draw (206.26999999999998, -60.90000000000001) rectangle (211.61999999999998,-63.500000000000014);
\draw(211.72, -60.90000000000001) node[anchor=north west,align=left] {Derived\\ categories,\\ triangulated\\ categories};
\draw (211.72, -60.90000000000001) rectangle (215.32,-63.000000000000014);
\draw(189.67, -63.600000000000016) node[anchor=north west,align=left] {Homological\\ dimension\\ (category-theoretic\\ aspects)};
\draw (189.67, -63.600000000000016) rectangle (195.01999999999998,-65.70000000000002);
\draw(195.11999999999998, -63.600000000000016) node[anchor=north west,align=left] {Chain complexes\\ (category-theoretic\\ aspects),\\ dg categories};
\draw (195.11999999999998, -63.600000000000016) rectangle (200.46999999999997,-65.70000000000002);
\draw(200.57, -63.600000000000016) node[anchor=north west,align=left] {Nonabelian\\ homological\\  algebra\\ (category-theoretic\\ aspects)};
\draw (200.57, -63.600000000000016) rectangle (205.92,-66.20000000000002);
\draw(206.01999999999998, -63.600000000000016) node[anchor=north west,align=left] {Other\\ (co)homology\\ theories\\ (category-theoretic\\ aspects)};
\draw (206.01999999999998, -63.600000000000016) rectangle (211.36999999999998,-66.20000000000002);
\draw(211.47, -63.600000000000016) node[anchor=north west,align=left] {Stable\\ module\\ categories};
\draw (211.47, -63.600000000000016) rectangle (214.57,-65.20000000000002);
\draw(189.67, -66.30000000000001) node[anchor=north west,align=left] {2-groups,\\ crossed\\ modules, crossed\\ complexes};
\draw (189.67, -66.30000000000001) rectangle (194.26999999999998,-68.4);
\draw(194.36999999999998, -66.30000000000001) node[anchor=north west,align=left] {Spectral\\ sequences,\\ hypercohomology};
\draw (194.36999999999998, -66.30000000000001) rectangle (198.71999999999997,-67.9);
\draw(198.82, -66.30000000000001) node[anchor=north west,align=left] {\(A_{\infty}\)-categories,\\ relations\\  with\\ homological\\ mirror symmetry};
\draw (198.82, -66.30000000000001) rectangle (203.17,-68.9);
\draw(203.26999999999998, -66.30000000000001) node[anchor=north west,align=left] {Simplicial\\ modules and\\  Dold-Kan\\ correspondence};
\draw (203.26999999999998, -66.30000000000001) rectangle (207.36999999999998,-68.4);
\draw(207.46999999999997, -66.30000000000001) node[anchor=north west,align=left] {Graph\\ complexes\\ and graph\\ homology};
\draw (207.46999999999997, -66.30000000000001) rectangle (210.31999999999996,-68.4);
\draw(215.51999999999998, -59.90000000000001) node[anchor=north west,align=left] {\large Categories in geometry and topology};
\draw (215.51999999999998, -59.90000000000001) rectangle (229.67,-73.4);
\draw(216.51999999999998, -60.90000000000001) node[anchor=north west,align=left] {Presheaves and\\ sheaves, stacks,\\  descent\\ conditions\\ (category-theoretic aspects)};
\draw (216.51999999999998, -60.90000000000001) rectangle (224.11999999999998,-63.500000000000014);
\draw(224.21999999999997, -60.90000000000001) node[anchor=north west,align=left] {Abstract\\ manifolds and fiber\\  bundles\\ (category-theoretic\\ aspects)};
\draw (224.21999999999997, -60.90000000000001) rectangle (229.56999999999996,-63.500000000000014);
\draw(216.51999999999998, -63.600000000000016) node[anchor=north west,align=left] {Synthetic\\ differential geometry,\\  tangent\\ categories, differential\\ categories};
\draw (216.51999999999998, -63.600000000000016) rectangle (223.11999999999998,-66.20000000000002);
\draw(223.21999999999997, -63.600000000000016) node[anchor=north west,align=left] {Algebraic\\ \(K\)-theory and\\ \(L\)-theory\\ (category-theoretic\\ aspects)};
\draw (223.21999999999997, -63.600000000000016) rectangle (228.56999999999996,-66.20000000000002);
\draw(216.51999999999998, -66.30000000000001) node[anchor=north west,align=left] {Grothendieck\\  groups\\ (category-theoretic\\ aspects)};
\draw (216.51999999999998, -66.30000000000001) rectangle (221.86999999999998,-68.4);
\draw(221.96999999999997, -66.30000000000001) node[anchor=north west,align=left] {Grothendieck\\ topologies\\  and\\ Grothendieck topoi};
\draw (221.96999999999997, -66.30000000000001) rectangle (227.06999999999996,-68.4);
\draw(216.51999999999998, -68.50000000000001) node[anchor=north west,align=left] {Frames and\\ locales, pointfree\\ topology,\\ Stone duality};
\draw (216.51999999999998, -68.50000000000001) rectangle (221.61999999999998,-70.60000000000001);
\draw(221.71999999999997, -68.50000000000001) node[anchor=north west,align=left] {Categories\\ of topological\\  spaces\\ and continuous\\ mappings};
\draw (221.71999999999997, -68.50000000000001) rectangle (225.81999999999996,-71.10000000000001);
\draw(225.92, -68.50000000000001) node[anchor=north west,align=left] {Local\\ categories\\ and functors};
\draw (225.92, -68.50000000000001) rectangle (229.51999999999998,-70.10000000000001);
\draw(225.92, -70.20000000000002) node[anchor=north west,align=left] {Quantales};
\draw (225.92, -70.20000000000002) rectangle (228.76999999999998,-70.80000000000001);
\draw(216.51999999999998, -71.20000000000002) node[anchor=north west,align=left] {Goodwillie\\  calculus\\ and functor\\ calculus};
\draw (216.51999999999998, -71.20000000000002) rectangle (219.86999999999998,-73.30000000000001);
\draw(188.67, -69.10000000000001) node[anchor=north west,align=left] {\large Computational methods\\ for problems pertaining\\ to category theory};
\draw (188.67, -69.10000000000001) rectangle (196.39999999999998,-70.7);
\draw(188.67, -70.80000000000001) node[anchor=north west,align=left] {\large History of\\ category theory};
\draw (188.67, -70.80000000000001) rectangle (193.92,-71.9);
\draw(188.67, -73.50000000000001) node[anchor=north west,align=left] {\large General theory of categories and functors};
\draw (188.67, -73.50000000000001) rectangle (204.26999999999998,-89.50000000000001);
\draw(189.67, -74.50000000000001) node[anchor=north west,align=left] {Factorization\\ systems, substructures,\\  quotient\\ structures,\\ congruences, amalgams};
\draw (189.67, -74.50000000000001) rectangle (196.01999999999998,-77.10000000000001);
\draw(196.11999999999998, -74.50000000000001) node[anchor=north west,align=left] {Limits and colimits\\ (products, sums,\\ directed limits,\\ pushouts, fiber\\  products,\\ equalizers, kernels,\\ ends and coends, etc.)};
\draw (196.11999999999998, -74.50000000000001) rectangle (202.21999999999997,-78.10000000000001);
\draw(189.67, -78.20000000000002) node[anchor=north west,align=left] {Categories\\ admitting limits\\ (complete categories),\\  functors\\ preserving limits,\\ completions};
\draw (189.67, -78.20000000000002) rectangle (195.76999999999998,-81.30000000000001);
\draw(195.86999999999998, -78.20000000000002) node[anchor=north west,align=left] {Special\\ properties of\\ functors (faithful,\\ full, etc.)};
\draw (195.86999999999998, -78.20000000000002) rectangle (201.21999999999997,-80.30000000000001);
\draw(189.67, -81.40000000000002) node[anchor=north west,align=left] {Adjoint functors\\  (universal\\ constructions,\\ reflective\\ subcategories, Kan\\ extensions, etc.)};
\draw (189.67, -81.40000000000002) rectangle (194.76999999999998,-84.50000000000001);
\draw(194.86999999999998, -81.40000000000002) node[anchor=north west,align=left] {Foundations,\\ relations to\\ logic and\\ deductive systems};
\draw (194.86999999999998, -81.40000000000002) rectangle (199.71999999999997,-83.50000000000001);
\draw(199.82, -81.40000000000002) node[anchor=north west,align=left] {Definitions\\  and\\ generalizations\\ in theory of\\ categories};
\draw (199.82, -81.40000000000002) rectangle (204.17,-84.00000000000001);
\draw(189.67, -84.60000000000002) node[anchor=north west,align=left] {Epimorphisms,\\ monomorphisms,\\ special classes\\ of morphisms,\\ null morphisms};
\draw (189.67, -84.60000000000002) rectangle (194.01999999999998,-87.20000000000002);
\draw(194.11999999999998, -84.60000000000002) node[anchor=north west,align=left] {Graphs,\\ diagram\\ schemes,\\ precategories};
\draw (194.11999999999998, -84.60000000000002) rectangle (197.96999999999997,-86.70000000000002);
\draw(198.07, -84.60000000000002) node[anchor=north west,align=left] {Functor\\ categories,\\  comma\\ categories};
\draw (198.07, -84.60000000000002) rectangle (201.42,-86.70000000000002);
\draw(189.67, -87.30000000000001) node[anchor=north west,align=left] {Natural\\ morphisms,\\ dinatural\\ morphisms};
\draw (189.67, -87.30000000000001) rectangle (192.76999999999998,-89.4);
\draw(192.86999999999998, -87.30000000000001) node[anchor=north west,align=left] {Graded\\ categories\\ (general)};
\draw (192.86999999999998, -87.30000000000001) rectangle (195.96999999999997,-88.9);
\draw(204.36999999999998, -73.50000000000001) node[anchor=north west,align=left] {\large Higher categories and homotopical algebra};
\draw (204.36999999999998, -73.50000000000001) rectangle (219.71999999999997,-85.30000000000001);
\draw(205.36999999999998, -74.50000000000001) node[anchor=north west,align=left] {Localizations\\  (e.g.,\\ simplicial localization,\\ Bousfield\\ localization)};
\draw (205.36999999999998, -74.50000000000001) rectangle (211.96999999999997,-77.10000000000001);
\draw(212.06999999999996, -74.50000000000001) node[anchor=north west,align=left] {Tricategories,\\  weak\\ \(n\)-categories,\\ coherence,\\ semi-strictification};
\draw (212.06999999999996, -74.50000000000001) rectangle (217.66999999999996,-77.10000000000001);
\draw(205.36999999999998, -77.20000000000002) node[anchor=north west,align=left] {Categories of\\ fibrations,\\ relations to\\ \(K\)-theory, relations\\ to type theory};
\draw (205.36999999999998, -77.20000000000002) rectangle (210.71999999999997,-79.80000000000001);
\draw(210.81999999999996, -77.20000000000002) node[anchor=north west,align=left] {\((\infty,1)\)-categories\\ (quasi-categories,\\  Segal\\ spaces, etc.);\\  \(\infty\)-topoi,\\ stable \(\infty\)-categories};
\draw (210.81999999999996, -77.20000000000002) rectangle (216.16999999999996,-80.30000000000001);
\draw(216.26999999999998, -77.20000000000002) node[anchor=north west,align=left] {Simplicial\\ sets,\\ simplicial\\ objects};
\draw (216.26999999999998, -77.20000000000002) rectangle (219.36999999999998,-79.30000000000001);
\draw(205.36999999999998, -80.40000000000002) node[anchor=north west,align=left] {Strict\\ omega-categories,\\ computads,\\ polygraphs};
\draw (205.36999999999998, -80.40000000000002) rectangle (210.21999999999997,-82.50000000000001);
\draw(210.31999999999996, -80.40000000000002) node[anchor=north west,align=left] {Categorification};
\draw (210.31999999999996, -80.40000000000002) rectangle (214.91999999999996,-81.00000000000001);
\draw(210.31999999999996, -81.10000000000001) node[anchor=north west,align=left] {2-dimensional\\ monad theory};
\draw (210.31999999999996, -81.10000000000001) rectangle (214.16999999999996,-82.2);
\draw(215.01999999999998, -80.40000000000002) node[anchor=north west,align=left] {\((\infty,~n)\)-categories\\  and\\ \((\infty,\infty)\)-categories};
\draw (215.01999999999998, -80.40000000000002) rectangle (219.61999999999998,-82.00000000000001);
\draw(205.36999999999998, -82.60000000000002) node[anchor=north west,align=left] {2-categories,\\ bicategories,\\ double\\ categories};
\draw (205.36999999999998, -82.60000000000002) rectangle (209.21999999999997,-84.70000000000002);
\draw(209.31999999999996, -82.60000000000002) node[anchor=north west,align=left] {Homotopical\\  algebra,\\ Quillen model\\ categories,\\ derivators};
\draw (209.31999999999996, -82.60000000000002) rectangle (213.16999999999996,-85.20000000000002);
\draw(213.26999999999998, -82.60000000000002) node[anchor=north west,align=left] {\(\infty\)-operads\\ and higher\\ algebra};
\draw (213.26999999999998, -82.60000000000002) rectangle (216.36999999999998,-84.20000000000002);
\draw(219.82, -73.50000000000001) node[anchor=north west,align=left] {\large Categories and theories};
\draw (219.82, -73.50000000000001) rectangle (229.97,-85.30000000000001);
\draw(220.82, -74.50000000000001) node[anchor=north west,align=left] {Monads (= standard\\  construction,\\ triple or triad),\\ algebras for monads,\\ homology and derived\\ functors for monads};
\draw (220.82, -74.50000000000001) rectangle (226.42,-77.60000000000001);
\draw(226.51999999999998, -74.50000000000001) node[anchor=north west,align=left] {Accessible\\ and locally\\ presentable\\ categories};
\draw (226.51999999999998, -74.50000000000001) rectangle (229.86999999999998,-76.60000000000001);
\draw(220.82, -77.70000000000002) node[anchor=north west,align=left] {Theories\\ (e.g., algebraic\\ theories),\\  structure,\\ and semantics};
\draw (220.82, -77.70000000000002) rectangle (225.42,-80.30000000000001);
\draw(225.51999999999998, -77.70000000000002) node[anchor=north west,align=left] {Eilenberg-Moore\\ and Kleisli\\ constructions\\ for monads};
\draw (225.51999999999998, -77.70000000000002) rectangle (229.86999999999998,-79.80000000000001);
\draw(220.82, -80.40000000000002) node[anchor=north west,align=left] {Sketches\\ and\\ generalizations};
\draw (220.82, -80.40000000000002) rectangle (225.17,-82.00000000000001);
\draw(225.26999999999998, -80.40000000000002) node[anchor=north west,align=left] {Structured\\  objects\\ in a category\\  (group\\ objects, etc.)};
\draw (225.26999999999998, -80.40000000000002) rectangle (229.36999999999998,-83.00000000000001);
\draw(220.82, -83.10000000000002) node[anchor=north west,align=left] {Categorical\\ semantics\\ of formal\\ languages};
\draw (220.82, -83.10000000000002) rectangle (224.17,-85.20000000000002);
\draw(224.26999999999998, -83.10000000000002) node[anchor=north west,align=left] {Equational\\ categories};
\draw (224.26999999999998, -83.10000000000002) rectangle (227.36999999999998,-84.20000000000002);
\draw(188.67, -89.60000000000002) node[anchor=north west,align=left] {\large Monoidal categories and operads};
\draw (188.67, -89.60000000000002) rectangle (201.01999999999998,-109.20000000000002);
\draw(189.67, -90.60000000000002) node[anchor=north west,align=left] {Polycategories/dioperads,\\ properads, PROPs,\\ cyclic operads,\\ modular operads};
\draw (189.67, -90.60000000000002) rectangle (196.51999999999998,-92.70000000000002);
\draw(196.61999999999998, -90.60000000000002) node[anchor=north west,align=left] {Bimonoidal,\\ skew-monoidal,\\ duoidal\\ categories};
\draw (196.61999999999998, -90.60000000000002) rectangle (200.71999999999997,-92.70000000000002);
\draw(189.67, -92.80000000000003) node[anchor=north west,align=left] {Braided\\ monoidal categories\\ and ribbon\\ categories};
\draw (189.67, -92.80000000000003) rectangle (195.01999999999998,-94.90000000000002);
\draw(195.11999999999998, -92.80000000000003) node[anchor=north west,align=left] {Fusion\\ categories, modular\\  tensor\\ categories,\\ modular functors};
\draw (195.11999999999998, -92.80000000000003) rectangle (200.46999999999997,-95.40000000000002);
\draw(189.67, -95.50000000000003) node[anchor=north west,align=left] {Dagger\\ categories,\\ categorical quantum\\ mechanics};
\draw (189.67, -95.50000000000003) rectangle (195.01999999999998,-97.60000000000002);
\draw(195.11999999999998, -95.50000000000003) node[anchor=north west,align=left] {Monoidal\\ categories,\\ symmetric monoidal\\ categories};
\draw (195.11999999999998, -95.50000000000003) rectangle (200.21999999999997,-97.60000000000002);
\draw(189.67, -97.70000000000002) node[anchor=north west,align=left] {Traced monoidal\\ categories,\\  compact\\ closed categories,\\ star-autonomous\\ categories};
\draw (189.67, -97.70000000000002) rectangle (194.76999999999998,-100.80000000000001);
\draw(194.86999999999998, -97.70000000000002) node[anchor=north west,align=left] {Categories\\ of networks\\  and\\ processes,\\ compositionality};
\draw (194.86999999999998, -97.70000000000002) rectangle (199.46999999999997,-100.30000000000001);
\draw(189.67, -100.90000000000002) node[anchor=north west,align=left] {Non-symmetric\\  operads,\\ multicategories,\\  generalized\\ multicategories};
\draw (189.67, -100.90000000000002) rectangle (194.26999999999998,-103.50000000000001);
\draw(194.36999999999998, -100.90000000000002) node[anchor=north west,align=left] {Algebraic\\ operads,\\ cooperads, and\\ Koszul duality};
\draw (194.36999999999998, -100.90000000000002) rectangle (198.46999999999997,-103.00000000000001);
\draw(189.67, -103.60000000000002) node[anchor=north west,align=left] {Species,\\ Hopf monoids,\\ operads in\\ combinatorics};
\draw (189.67, -103.60000000000002) rectangle (193.51999999999998,-105.70000000000002);
\draw(193.61999999999998, -103.60000000000002) node[anchor=north west,align=left] {String\\ diagrams and\\ graphical\\ calculi};
\draw (193.61999999999998, -103.60000000000002) rectangle (197.21999999999997,-105.70000000000002);
\draw(197.32, -103.60000000000002) node[anchor=north west,align=left] {Categorical\\ aspects\\  of\\ linear logic};
\draw (197.32, -103.60000000000002) rectangle (200.92,-105.70000000000002);
\draw(189.67, -105.80000000000003) node[anchor=north west,align=left] {Topological\\  and\\ simplicial\\ operads};
\draw (189.67, -105.80000000000003) rectangle (193.01999999999998,-107.90000000000002);
\draw(193.11999999999998, -105.80000000000003) node[anchor=north west,align=left] {Tannakian\\ categories};
\draw (193.11999999999998, -105.80000000000003) rectangle (196.21999999999997,-106.90000000000002);
\draw(196.32, -105.80000000000003) node[anchor=north west,align=left] {Operads\\ (general)};
\draw (196.32, -105.80000000000003) rectangle (199.17,-106.90000000000002);
\draw(189.67, -108.00000000000003) node[anchor=north west,align=left] {Globular\\ operads};
\draw (189.67, -108.00000000000003) rectangle (192.26999999999998,-109.10000000000002);
\draw(201.11999999999998, -89.60000000000002) node[anchor=north west,align=left] {\large Categorical structures};
\draw (201.11999999999998, -89.60000000000002) rectangle (212.01999999999998,-100.90000000000002);
\draw(202.11999999999998, -90.60000000000002) node[anchor=north west,align=left] {Proarrow equipments,\\  Yoneda\\ structures, KZ\\ doctrines (lax\\ idempotent monads)};
\draw (202.11999999999998, -90.60000000000002) rectangle (207.71999999999997,-93.20000000000002);
\draw(207.81999999999996, -90.60000000000002) node[anchor=north west,align=left] {Enriched\\ categories\\ (over closed\\ or monoidal\\ categories)};
\draw (207.81999999999996, -90.60000000000002) rectangle (211.41999999999996,-93.20000000000002);
\draw(202.11999999999998, -93.30000000000003) node[anchor=north west,align=left] {Actions of a\\  monoidal\\ category,\\ tensorial strength};
\draw (202.11999999999998, -93.30000000000003) rectangle (207.21999999999997,-95.40000000000002);
\draw(207.31999999999996, -93.30000000000003) node[anchor=north west,align=left] {Profunctors\\  (=\\ correspondences,\\ distributors,\\ modules)};
\draw (207.31999999999996, -93.30000000000003) rectangle (211.91999999999996,-95.90000000000002);
\draw(202.11999999999998, -96.00000000000003) node[anchor=north west,align=left] {Closed categories\\  (closed\\ monoidal and\\ Cartesian closed\\ categories, etc.)};
\draw (202.11999999999998, -96.00000000000003) rectangle (206.96999999999997,-98.60000000000002);
\draw(207.06999999999996, -96.00000000000003) node[anchor=north west,align=left] {Fibered\\ categories};
\draw (207.06999999999996, -96.00000000000003) rectangle (210.16999999999996,-97.10000000000002);
\draw(202.11999999999998, -98.70000000000002) node[anchor=north west,align=left] {Internal\\ categories\\  and\\ groupoids};
\draw (202.11999999999998, -98.70000000000002) rectangle (205.21999999999997,-100.80000000000001);
\draw(205.31999999999996, -98.70000000000002) node[anchor=north west,align=left] {Formal\\ category\\ theory};
\draw (205.31999999999996, -98.70000000000002) rectangle (207.91999999999996,-100.30000000000001);
\draw(212.11999999999998, -89.60000000000002) node[anchor=north west,align=left] {\large Categorical algebra};
\draw (212.11999999999998, -89.60000000000002) rectangle (222.26999999999998,-101.60000000000002);
\draw(213.11999999999998, -90.60000000000002) node[anchor=north west,align=left] {Protomodular\\ categories,\\ semi-abelian\\ categories,\\ Mal’tsev categories};
\draw (213.11999999999998, -90.60000000000002) rectangle (218.46999999999997,-93.20000000000002);
\draw(218.56999999999996, -90.60000000000002) node[anchor=north west,align=left] {Preadditive,\\ additive\\ categories};
\draw (218.56999999999996, -90.60000000000002) rectangle (222.16999999999996,-92.20000000000002);
\draw(213.11999999999998, -93.30000000000003) node[anchor=north west,align=left] {Definable\\ subcategories\\  and\\ connections with\\ model theory};
\draw (213.11999999999998, -93.30000000000003) rectangle (217.71999999999997,-95.90000000000002);
\draw(217.81999999999996, -93.30000000000003) node[anchor=north west,align=left] {Localization\\ of categories,\\ calculus\\ of fractions};
\draw (217.81999999999996, -93.30000000000003) rectangle (221.91999999999996,-95.40000000000002);
\draw(213.11999999999998, -96.00000000000003) node[anchor=north west,align=left] {Abelian\\ categories,\\ Grothendieck\\ categories};
\draw (213.11999999999998, -96.00000000000003) rectangle (216.71999999999997,-98.10000000000002);
\draw(216.81999999999996, -96.00000000000003) node[anchor=north west,align=left] {Regular\\ categories,\\ Barr-exact\\ categories};
\draw (216.81999999999996, -96.00000000000003) rectangle (220.16999999999996,-98.10000000000002);
\draw(213.11999999999998, -98.20000000000002) node[anchor=north west,align=left] {Categorical\\ embedding\\ theorems};
\draw (213.11999999999998, -98.20000000000002) rectangle (216.46999999999997,-99.80000000000001);
\draw(216.56999999999996, -98.20000000000002) node[anchor=north west,align=left] {Categorical\\ Galois\\ theory};
\draw (216.56999999999996, -98.20000000000002) rectangle (219.91999999999996,-99.80000000000001);
\draw(213.11999999999998, -99.90000000000002) node[anchor=north west,align=left] {Torsion\\ theories,\\ radicals};
\draw (213.11999999999998, -99.90000000000002) rectangle (215.96999999999997,-101.50000000000001);
\draw(188.67, -109.30000000000001) node[anchor=north west,align=left] {\large Special categories};
\draw (188.67, -109.30000000000001) rectangle (198.57,-119.60000000000001);
\draw(189.67, -110.30000000000001) node[anchor=north west,align=left] {Categories\\ of sets,\\ characterizations};
\draw (189.67, -110.30000000000001) rectangle (194.51999999999998,-111.9);
\draw(194.61999999999998, -110.30000000000001) node[anchor=north west,align=left] {Extensive,\\ distributive,\\ and adhesive\\ categories};
\draw (194.61999999999998, -110.30000000000001) rectangle (198.46999999999997,-112.4);
\draw(189.67, -112.50000000000001) node[anchor=north west,align=left] {Categories\\ of spans/cospans,\\ relations, or\\ partial maps};
\draw (189.67, -112.50000000000001) rectangle (194.51999999999998,-114.60000000000001);
\draw(194.61999999999998, -112.50000000000001) node[anchor=north west,align=left] {Categories\\ of machines,\\ automata};
\draw (194.61999999999998, -112.50000000000001) rectangle (198.21999999999997,-114.10000000000001);
\draw(189.67, -114.70000000000002) node[anchor=north west,align=left] {Preorders,\\ orders, domains\\ and lattices\\  (viewed\\ as categories)};
\draw (189.67, -114.70000000000002) rectangle (194.01999999999998,-117.30000000000001);
\draw(194.11999999999998, -114.70000000000002) node[anchor=north west,align=left] {Groupoids,\\ semigroupoids,\\ semigroups,\\ groups (viewed\\ as categories)};
\draw (194.11999999999998, -114.70000000000002) rectangle (198.21999999999997,-117.30000000000001);
\draw(189.67, -117.4) node[anchor=north west,align=left] {Embedding\\ theorems,\\ universal\\ categories};
\draw (189.67, -117.4) rectangle (192.76999999999998,-119.5);
\draw(192.86999999999998, -117.4) node[anchor=north west,align=left] {Topoi};
\draw (192.86999999999998, -117.4) rectangle (194.71999999999997,-118.0);
\draw(234.01999999999998, -1) node[anchor=north west,align=left] {\LARGE Measure and integration};
\draw (234.01999999999998, -1) rectangle (275.56,-31.1);
\draw(235.01999999999998, -2) node[anchor=north west,align=left] {\large Set functions, measures and integrals with values in abstract spaces};
\draw (235.01999999999998, -2) rectangle (256.7,-6.199999999999999);
\draw(236.01999999999998, -3) node[anchor=north west,align=left] {Set-valued set\\ functions and\\ measures; integration\\ of set-valued\\  functions;\\ measurable selections};
\draw (236.01999999999998, -3) rectangle (241.86999999999998,-6.1);
\draw(241.96999999999997, -3) node[anchor=north west,align=left] {Group- or\\ semigroup-valued\\  set\\ functions, measures\\ and integrals};
\draw (241.96999999999997, -3) rectangle (247.31999999999996,-5.6);
\draw(247.42, -3) node[anchor=north west,align=left] {Vector-valued\\ set functions,\\ measures\\ and integrals};
\draw (247.42, -3) rectangle (251.51999999999998,-5.1);
\draw(251.61999999999998, -3) node[anchor=north west,align=left] {Set functions,\\ measures and\\  integrals\\ with values in\\ ordered spaces};
\draw (251.61999999999998, -3) rectangle (255.71999999999997,-5.6);
\draw(256.79999999999995, -2) node[anchor=north west,align=left] {\large Miscellaneous topics in measure theory};
\draw (256.79999999999995, -2) rectangle (269.17999999999995,-5.2);
\draw(257.79999999999995, -3) node[anchor=north west,align=left] {Other\\ connections\\ with logic\\ and set theory};
\draw (257.79999999999995, -3) rectangle (261.9,-5.1);
\draw(261.99999999999994, -3) node[anchor=north west,align=left] {Nonstandard\\ measure\\ theory};
\draw (261.99999999999994, -3) rectangle (265.34999999999997,-4.6);
\draw(265.44999999999993, -3) node[anchor=north west,align=left] {Fuzzy\\ measure\\ theory};
\draw (265.44999999999993, -3) rectangle (267.79999999999995,-4.6);
\draw(269.28, -2) node[anchor=north west,align=left] {\large History of measure\\ and integration};
\draw (269.28, -2) rectangle (275.46,-3.1);
\draw(235.01999999999998, -6.299999999999999) node[anchor=north west,align=left] {\large Set functions and measures on spaces with additional structure};
\draw (235.01999999999998, -6.299999999999999) rectangle (254.86999999999998,-13.7);
\draw(236.01999999999998, -7.299999999999999) node[anchor=north west,align=left] {Set functions and\\ measures and integrals\\  in\\ infinite-dimensional spaces\\ (Wiener measure,\\ Gaussian measure, etc.)};
\draw (236.01999999999998, -7.299999999999999) rectangle (243.36999999999998,-10.399999999999999);
\draw(243.46999999999997, -7.299999999999999) node[anchor=north west,align=left] {Integration theory\\  via linear\\ functionals (Radon\\ measures, Daniell\\ integrals, etc.),\\ representing set\\ functions and measures};
\draw (243.46999999999997, -7.299999999999999) rectangle (249.56999999999996,-10.899999999999999);
\draw(249.67, -7.299999999999999) node[anchor=north west,align=left] {Set functions and\\  measures on\\ topological groups\\  or semigroups,\\ Haar measures,\\ invariant measures};
\draw (249.67, -7.299999999999999) rectangle (254.76999999999998,-10.399999999999999);
\draw(236.01999999999998, -11.0) node[anchor=north west,align=left] {Set functions\\ and measures on\\ topological\\ spaces (regularity\\ of measures, etc.)};
\draw (236.01999999999998, -11.0) rectangle (241.11999999999998,-13.6);
\draw(254.96999999999997, -6.299999999999999) node[anchor=north west,align=left] {\large Measure-theoretic ergodic theory};
\draw (254.96999999999997, -6.299999999999999) rectangle (266.36999999999995,-12.2);
\draw(255.96999999999997, -7.299999999999999) node[anchor=north west,align=left] {Measure-preserving\\ transformations};
\draw (255.96999999999997, -7.299999999999999) rectangle (261.07,-8.399999999999999);
\draw(261.16999999999996, -7.299999999999999) node[anchor=north west,align=left] {One-parameter\\ continuous\\ families of\\ measure-preserving\\ transformations};
\draw (261.16999999999996, -7.299999999999999) rectangle (266.27,-9.899999999999999);
\draw(255.96999999999997, -10.0) node[anchor=north west,align=left] {General groups\\  of\\ measure-preserving\\ transformations};
\draw (255.96999999999997, -10.0) rectangle (261.07,-12.1);
\draw(261.16999999999996, -10.0) node[anchor=north west,align=left] {Entropy\\ and other\\ invariants};
\draw (261.16999999999996, -10.0) rectangle (264.27,-11.6);
\draw(266.46999999999997, -6.299999999999999) node[anchor=north west,align=left] {\large Computational methods for\\ problems pertaining to\\ measure and integration};
\draw (266.46999999999997, -6.299999999999999) rectangle (274.82,-7.899999999999999);
\draw(235.01999999999998, -13.799999999999999) node[anchor=north west,align=left] {\large Classical measure theory};
\draw (235.01999999999998, -13.799999999999999) rectangle (245.92,-31.0);
\draw(236.01999999999998, -14.799999999999999) node[anchor=north west,align=left] {Measurable and\\ nonmeasurable\\ functions, sequences\\ of measurable\\  functions,\\ modes of convergence};
\draw (236.01999999999998, -14.799999999999999) rectangle (241.61999999999998,-17.9);
\draw(241.71999999999997, -14.799999999999999) node[anchor=north west,align=left] {Real- or\\ complex-valued\\ set\\ functions};
\draw (241.71999999999997, -14.799999999999999) rectangle (245.81999999999996,-16.9);
\draw(241.71999999999997, -17.0) node[anchor=north west,align=left] {Fractals};
\draw (241.71999999999997, -17.0) rectangle (244.31999999999996,-17.6);
\draw(236.01999999999998, -18.0) node[anchor=north west,align=left] {Classes of sets\\ (Borel fields,\\ \(\sigma\)-rings, etc.),\\  measurable\\ sets, Suslin sets,\\ analytic sets};
\draw (236.01999999999998, -18.0) rectangle (241.11999999999998,-21.1);
\draw(241.21999999999997, -18.0) node[anchor=north west,align=left] {Measures\\ on Boolean\\ rings,\\ measure algebras};
\draw (241.21999999999997, -18.0) rectangle (245.81999999999996,-20.1);
\draw(236.01999999999998, -21.2) node[anchor=north west,align=left] {Abstract\\ differentiation\\ theory,\\ differentiation of\\ set functions};
\draw (236.01999999999998, -21.2) rectangle (241.11999999999998,-23.8);
\draw(241.21999999999997, -21.2) node[anchor=north west,align=left] {Contents,\\ measures,\\ outer measures,\\ capacities};
\draw (241.21999999999997, -21.2) rectangle (245.56999999999996,-23.3);
\draw(236.01999999999998, -23.9) node[anchor=north west,align=left] {Integration\\  and\\ disintegration\\ of measures};
\draw (236.01999999999998, -23.9) rectangle (240.11999999999998,-26.0);
\draw(240.21999999999997, -23.9) node[anchor=north west,align=left] {Length, area,\\ volume, other\\  geometric\\ measure theory};
\draw (240.21999999999997, -23.9) rectangle (244.31999999999996,-26.0);
\draw(236.01999999999998, -26.1) node[anchor=north west,align=left] {Integration\\ with respect\\ to measures\\  and other\\ set functions};
\draw (236.01999999999998, -26.1) rectangle (239.86999999999998,-28.700000000000003);
\draw(239.96999999999997, -26.1) node[anchor=north west,align=left] {Measures\\ and integrals\\ in product\\ spaces};
\draw (239.96999999999997, -26.1) rectangle (243.81999999999996,-28.200000000000003);
\draw(236.01999999999998, -28.799999999999997) node[anchor=north west,align=left] {Spaces of\\ measures,\\ convergence\\ of measures};
\draw (236.01999999999998, -28.799999999999997) rectangle (239.36999999999998,-30.9);
\draw(239.46999999999997, -28.799999999999997) node[anchor=north west,align=left] {Hausdorff\\ and packing\\ measures};
\draw (239.46999999999997, -28.799999999999997) rectangle (242.81999999999996,-30.4);
\draw(242.92, -28.799999999999997) node[anchor=north west,align=left] {Lifting\\ theory};
\draw (242.92, -28.799999999999997) rectangle (245.26999999999998,-29.9);
\draw(234.01999999999998, -31.200000000000003) node[anchor=north west,align=left] {\LARGE Algebraic geometry};
\draw (234.01999999999998, -31.200000000000003) rectangle (274.62,-118.6);
\draw(235.01999999999998, -32.2) node[anchor=north west,align=left] {\large Arithmetic problems in algebraic geometry; Diophantine geometry};
\draw (235.01999999999998, -32.2) rectangle (258.82,-41.300000000000004);
\draw(236.01999999999998, -33.2) node[anchor=north west,align=left] {Zeta functions\\ and related questions\\ in algebraic\\ geometry (e.g.,\\ Birch-Swinnerton-Dyer\\ conjecture)};
\draw (236.01999999999998, -33.2) rectangle (241.86999999999998,-36.300000000000004);
\draw(241.96999999999997, -33.2) node[anchor=north west,align=left] {Hasse principle,\\  weak and\\ strong approximation,\\ Brauer-Manin\\ obstruction};
\draw (241.96999999999997, -33.2) rectangle (247.81999999999996,-35.800000000000004);
\draw(247.92, -33.2) node[anchor=north west,align=left] {Universal profinite\\ groups (relationship\\  to moduli\\ spaces, projective\\ and moduli towers,\\ Galois theory)};
\draw (247.92, -33.2) rectangle (253.51999999999998,-36.300000000000004);
\draw(253.61999999999998, -33.2) node[anchor=north west,align=left] {Positive\\ characteristic\\  ground\\ fields in\\ algebraic geometry};
\draw (253.61999999999998, -33.2) rectangle (258.71999999999997,-35.800000000000004);
\draw(236.01999999999998, -36.400000000000006) node[anchor=north west,align=left] {Other\\ nonalgebraically\\ closed ground\\ fields in algebraic\\ geometry};
\draw (236.01999999999998, -36.400000000000006) rectangle (241.36999999999998,-39.00000000000001);
\draw(241.46999999999997, -36.400000000000006) node[anchor=north west,align=left] {Applications\\ to coding\\ theory and\\ cryptography of\\ arithmetic geometry};
\draw (241.46999999999997, -36.400000000000006) rectangle (246.81999999999996,-39.00000000000001);
\draw(246.92, -36.400000000000006) node[anchor=north west,align=left] {Arithmetic\\ varieties\\ and schemes;\\  Arakelov\\ theory; heights};
\draw (246.92, -36.400000000000006) rectangle (251.26999999999998,-39.00000000000001);
\draw(251.36999999999998, -36.400000000000006) node[anchor=north west,align=left] {Perfectoid\\ spaces and\\  mixed\\ characteristic};
\draw (251.36999999999998, -36.400000000000006) rectangle (255.46999999999997,-38.50000000000001);
\draw(255.57, -36.400000000000006) node[anchor=north west,align=left] {Rational\\ points};
\draw (255.57, -36.400000000000006) rectangle (258.17,-37.50000000000001);
\draw(236.01999999999998, -39.1) node[anchor=north west,align=left] {Finite\\ ground fields\\ in algebraic\\ geometry};
\draw (236.01999999999998, -39.1) rectangle (239.86999999999998,-41.2);
\draw(239.96999999999997, -39.1) node[anchor=north west,align=left] {Global\\ ground fields\\ in algebraic\\ geometry};
\draw (239.96999999999997, -39.1) rectangle (243.81999999999996,-41.2);
\draw(243.92, -39.1) node[anchor=north west,align=left] {Local ground\\  fields\\ in algebraic\\ geometry};
\draw (243.92, -39.1) rectangle (247.51999999999998,-41.2);
\draw(247.61999999999998, -39.1) node[anchor=north west,align=left] {Modular\\ and Shimura\\ varieties};
\draw (247.61999999999998, -39.1) rectangle (250.96999999999997,-40.7);
\draw(251.07, -39.1) node[anchor=north west,align=left] {Rigid\\ analytic\\ geometry};
\draw (251.07, -39.1) rectangle (253.67,-40.7);
\draw(258.91999999999996, -32.2) node[anchor=north west,align=left] {\large (Co)homology theory in algebraic geometry};
\draw (258.91999999999996, -32.2) rectangle (274.52,-46.2);
\draw(259.91999999999996, -33.2) node[anchor=north west,align=left] {Other algebro-geometric\\ (co)homologies\\  (e.g.,\\ intersection, equivariant,\\ Lawson, Deligne\\ (co)homologies)};
\draw (259.91999999999996, -33.2) rectangle (267.02,-36.300000000000004);
\draw(267.11999999999995, -33.2) node[anchor=north west,align=left] {Differentials\\ and other special\\  sheaves;\\ D-modules;\\ Bernstein-Sato\\ ideals and polynomials};
\draw (267.11999999999995, -33.2) rectangle (273.21999999999997,-36.300000000000004);
\draw(259.91999999999996, -36.400000000000006) node[anchor=north west,align=left] {Derived categories\\ of sheaves,\\ dg categories,\\  and related\\ constructions in\\ algebraic geometry};
\draw (259.91999999999996, -36.400000000000006) rectangle (265.02,-39.50000000000001);
\draw(265.11999999999995, -36.400000000000006) node[anchor=north west,align=left] {Homotopy\\ theory and\\ fundamental groups\\ in algebraic\\ geometry};
\draw (265.11999999999995, -36.400000000000006) rectangle (270.21999999999997,-39.00000000000001);
\draw(270.31999999999994, -36.400000000000006) node[anchor=north west,align=left] {Étale and\\ other\\ Grothendieck\\ topologies and\\ (co)homologies};
\draw (270.31999999999994, -36.400000000000006) rectangle (274.41999999999996,-39.00000000000001);
\draw(259.91999999999996, -39.6) node[anchor=north west,align=left] {Classical\\ real and complex\\ (co)homology\\ in algebraic\\ geometry};
\draw (259.91999999999996, -39.6) rectangle (264.52,-42.2);
\draw(264.61999999999995, -39.6) node[anchor=north west,align=left] {Motivic\\ cohomology;\\ motivic\\ homotopy theory};
\draw (264.61999999999995, -39.6) rectangle (268.96999999999997,-41.7);
\draw(269.06999999999994, -39.6) node[anchor=north west,align=left] {de Rham\\ cohomology\\ and algebraic\\ geometry};
\draw (269.06999999999994, -39.6) rectangle (272.91999999999996,-41.7);
\draw(259.91999999999996, -42.300000000000004) node[anchor=north west,align=left] {Sheaves\\ in algebraic\\ geometry};
\draw (259.91999999999996, -42.300000000000004) rectangle (263.52,-43.900000000000006);
\draw(263.61999999999995, -42.300000000000004) node[anchor=north west,align=left] {Vanishing\\ theorems\\ in algebraic\\ geometry};
\draw (263.61999999999995, -42.300000000000004) rectangle (267.21999999999997,-44.400000000000006);
\draw(267.31999999999994, -42.300000000000004) node[anchor=north west,align=left] {Topological\\ properties\\ in algebraic\\ geometry};
\draw (267.31999999999994, -42.300000000000004) rectangle (270.91999999999996,-44.400000000000006);
\draw(271.02, -42.300000000000004) node[anchor=north west,align=left] {\(p\)-adic\\ cohomology,\\ crystalline\\ cohomology};
\draw (271.02, -42.300000000000004) rectangle (274.37,-44.400000000000006);
\draw(259.91999999999996, -44.5) node[anchor=north west,align=left] {Multiplier\\ ideals};
\draw (259.91999999999996, -44.5) rectangle (263.02,-45.6);
\draw(263.11999999999995, -44.5) node[anchor=north west,align=left] {Brauer\\ groups of\\ schemes};
\draw (263.11999999999995, -44.5) rectangle (265.96999999999997,-46.1);
\draw(235.01999999999998, -41.400000000000006) node[anchor=north west,align=left] {\large History of\\ algebraic geometry};
\draw (235.01999999999998, -41.400000000000006) rectangle (241.2,-42.50000000000001);
\draw(235.01999999999998, -46.300000000000004) node[anchor=north west,align=left] {\large Projective and enumerative algebraic geometry};
\draw (235.01999999999998, -46.300000000000004) rectangle (251.86999999999998,-55.900000000000006);
\draw(236.01999999999998, -47.300000000000004) node[anchor=north west,align=left] {Gromov-Witten\\ invariants, quantum\\ cohomology, Gopakumar-Vafa\\ invariants,\\  Donaldson-Thomas\\ invariants\\ (algebro-geometric aspects)};
\draw (236.01999999999998, -47.300000000000004) rectangle (243.36999999999998,-50.900000000000006);
\draw(243.46999999999997, -47.300000000000004) node[anchor=north west,align=left] {Enumerative\\ problems\\ (combinatorial\\ problems) in\\ algebraic geometry};
\draw (243.46999999999997, -47.300000000000004) rectangle (248.56999999999996,-49.900000000000006);
\draw(248.67, -47.300000000000004) node[anchor=north west,align=left] {Varieties\\  of\\ low degree};
\draw (248.67, -47.300000000000004) rectangle (251.76999999999998,-48.900000000000006);
\draw(236.01999999999998, -51.00000000000001) node[anchor=north west,align=left] {Secant\\ varieties, tensor\\  rank,\\ varieties of\\ sums of powers};
\draw (236.01999999999998, -51.00000000000001) rectangle (240.86999999999998,-53.60000000000001);
\draw(240.96999999999997, -51.00000000000001) node[anchor=north west,align=left] {Configurations\\  and\\ arrangements of\\ linear subspaces};
\draw (240.96999999999997, -51.00000000000001) rectangle (245.56999999999996,-53.10000000000001);
\draw(245.67, -51.00000000000001) node[anchor=north west,align=left] {Projective\\ techniques\\ in algebraic\\ geometry};
\draw (245.67, -51.00000000000001) rectangle (249.26999999999998,-53.10000000000001);
\draw(236.01999999999998, -53.7) node[anchor=north west,align=left] {Adjunction\\ problems};
\draw (236.01999999999998, -53.7) rectangle (239.11999999999998,-54.800000000000004);
\draw(239.21999999999997, -53.7) node[anchor=north west,align=left] {Classical\\ problems,\\ Schubert\\ calculus};
\draw (239.21999999999997, -53.7) rectangle (242.06999999999996,-55.800000000000004);
\draw(251.96999999999997, -46.300000000000004) node[anchor=north west,align=left] {\large Families, fibrations in algebraic geometry};
\draw (251.96999999999997, -46.300000000000004) rectangle (268.07,-59.10000000000001);
\draw(252.96999999999997, -47.300000000000004) node[anchor=north west,align=left] {Applications of\\ vector bundles and\\ moduli spaces in\\  mathematical\\ physics (twistor\\ theory, instantons,\\ quantum field theory)};
\draw (252.96999999999997, -47.300000000000004) rectangle (258.82,-50.900000000000006);
\draw(258.91999999999996, -47.300000000000004) node[anchor=north west,align=left] {Structure\\ of families\\ (Picard-Lefschetz,\\ monodromy, etc.)};
\draw (258.91999999999996, -47.300000000000004) rectangle (264.02,-49.400000000000006);
\draw(264.11999999999995, -47.300000000000004) node[anchor=north west,align=left] {Fine and\\ coarse\\ moduli spaces};
\draw (264.11999999999995, -47.300000000000004) rectangle (267.96999999999997,-48.900000000000006);
\draw(252.96999999999997, -51.00000000000001) node[anchor=north west,align=left] {Fibrations,\\ degenerations\\  in\\ algebraic geometry};
\draw (252.96999999999997, -51.00000000000001) rectangle (258.07,-53.10000000000001);
\draw(258.16999999999996, -51.00000000000001) node[anchor=north west,align=left] {Variation\\ of Hodge\\ structures\\ (algebro-geometric\\ aspects)};
\draw (258.16999999999996, -51.00000000000001) rectangle (263.27,-53.60000000000001);
\draw(263.36999999999995, -51.00000000000001) node[anchor=north west,align=left] {Algebraic\\ moduli problems,\\ moduli of\\ vector bundles};
\draw (263.36999999999995, -51.00000000000001) rectangle (267.96999999999997,-53.10000000000001);
\draw(252.96999999999997, -53.7) node[anchor=north west,align=left] {Formal methods\\ and deformations\\  in\\ algebraic geometry};
\draw (252.96999999999997, -53.7) rectangle (258.07,-55.800000000000004);
\draw(258.16999999999996, -53.7) node[anchor=north west,align=left] {Geometric\\ Langlands\\ program\\ (algebro-geometric\\ aspects)};
\draw (258.16999999999996, -53.7) rectangle (263.27,-56.300000000000004);
\draw(263.36999999999995, -53.7) node[anchor=north west,align=left] {Stacks\\ and moduli\\ problems};
\draw (263.36999999999995, -53.7) rectangle (266.46999999999997,-55.300000000000004);
\draw(252.96999999999997, -56.400000000000006) node[anchor=north west,align=left] {Arithmetic ground\\ fields (finite,\\ local, global)\\ and families\\ or fibrations};
\draw (252.96999999999997, -56.400000000000006) rectangle (257.82,-59.00000000000001);
\draw(235.01999999999998, -59.2) node[anchor=north west,align=left] {\large Surfaces and higher-dimensional varieties};
\draw (235.01999999999998, -59.2) rectangle (250.61999999999998,-74.9);
\draw(236.01999999999998, -60.2) node[anchor=north west,align=left] {Arithmetic\\ ground fields\\ for surfaces\\ or higher-dimensional\\ varieties};
\draw (236.01999999999998, -60.2) rectangle (241.86999999999998,-62.800000000000004);
\draw(241.96999999999997, -60.2) node[anchor=north west,align=left] {Topology of\\ surfaces (Donaldson\\ polynomials,\\ Seiberg-Witten\\ invariants)};
\draw (241.96999999999997, -60.2) rectangle (247.31999999999996,-62.800000000000004);
\draw(247.42, -60.2) node[anchor=north west,align=left] {Surfaces\\ of general\\ type};
\draw (247.42, -60.2) rectangle (250.51999999999998,-61.800000000000004);
\draw(247.42, -61.900000000000006) node[anchor=north west,align=left] {\(3\)-folds};
\draw (247.42, -61.900000000000006) rectangle (249.76999999999998,-62.50000000000001);
\draw(236.01999999999998, -62.900000000000006) node[anchor=north west,align=left] {Moduli,\\ classification:\\ analytic theory;\\  relations\\ with modular forms};
\draw (236.01999999999998, -62.900000000000006) rectangle (241.11999999999998,-65.5);
\draw(241.21999999999997, -62.900000000000006) node[anchor=north west,align=left] {Singularities\\ of surfaces\\  or\\ higher-dimensional\\ varieties};
\draw (241.21999999999997, -62.900000000000006) rectangle (246.31999999999996,-65.5);
\draw(246.42, -62.900000000000006) node[anchor=north west,align=left] {Hypersurfaces\\  and\\ algebraic\\ geometry};
\draw (246.42, -62.900000000000006) rectangle (250.26999999999998,-65.0);
\draw(236.01999999999998, -65.60000000000001) node[anchor=north west,align=left] {Elliptic\\ surfaces, elliptic\\ or Calabi-Yau\\ fibrations};
\draw (236.01999999999998, -65.60000000000001) rectangle (241.11999999999998,-67.7);
\draw(241.21999999999997, -65.60000000000001) node[anchor=north west,align=left] {Calabi-Yau\\ manifolds\\ (algebro-geometric\\ aspects)};
\draw (241.21999999999997, -65.60000000000001) rectangle (246.31999999999996,-67.7);
\draw(246.42, -65.60000000000001) node[anchor=north west,align=left] {Relationships\\ with physics};
\draw (246.42, -65.60000000000001) rectangle (250.26999999999998,-66.7);
\draw(246.42, -66.8) node[anchor=north west,align=left] {\(4\)-folds};
\draw (246.42, -66.8) rectangle (248.76999999999998,-67.39999999999999);
\draw(236.01999999999998, -67.80000000000001) node[anchor=north west,align=left] {Mirror\\ symmetry\\ (algebro-geometric\\ aspects)};
\draw (236.01999999999998, -67.80000000000001) rectangle (241.11999999999998,-69.9);
\draw(241.21999999999997, -67.80000000000001) node[anchor=north west,align=left] {Automorphisms\\ of surfaces\\  and\\ higher-dimensional\\ varieties};
\draw (241.21999999999997, -67.80000000000001) rectangle (246.31999999999996,-70.4);
\draw(246.42, -67.80000000000001) node[anchor=north west,align=left] {\(K3\) surfaces\\ and Enriques\\ surfaces};
\draw (246.42, -67.80000000000001) rectangle (250.01999999999998,-69.4);
\draw(236.01999999999998, -70.5) node[anchor=north west,align=left] {Vector bundles\\ on surfaces and\\ higher-dimensional\\ varieties,\\ and their moduli};
\draw (236.01999999999998, -70.5) rectangle (241.11999999999998,-73.1);
\draw(241.21999999999997, -70.5) node[anchor=north west,align=left] {Families,\\ moduli,\\ classification:\\ algebraic theory};
\draw (241.21999999999997, -70.5) rectangle (245.81999999999996,-72.6);
\draw(245.92, -70.5) node[anchor=north west,align=left] {Holomorphic\\ symplectic\\ varieties,\\ hyper-Kähler\\ varieties};
\draw (245.92, -70.5) rectangle (249.51999999999998,-73.1);
\draw(236.01999999999998, -73.2) node[anchor=north west,align=left] {Rational\\ and ruled\\ surfaces};
\draw (236.01999999999998, -73.2) rectangle (238.86999999999998,-74.8);
\draw(238.96999999999997, -73.2) node[anchor=north west,align=left] {Fano\\ varieties};
\draw (238.96999999999997, -73.2) rectangle (241.81999999999996,-74.3);
\draw(241.92, -73.2) node[anchor=north west,align=left] {Special\\ surfaces};
\draw (241.92, -73.2) rectangle (244.51999999999998,-74.3);
\draw(244.61999999999998, -73.2) node[anchor=north west,align=left] {\(n\)-folds\\ (\(n~>~4\))};
\draw (244.61999999999998, -73.2) rectangle (247.21999999999997,-74.3);
\draw(250.71999999999997, -59.2) node[anchor=north west,align=left] {\large Computational aspects in algebraic geometry};
\draw (250.71999999999997, -59.2) rectangle (266.32,-67.3);
\draw(251.71999999999997, -60.2) node[anchor=north west,align=left] {Geometric\\ aspects of\\ numerical algebraic\\ geometry};
\draw (251.71999999999997, -60.2) rectangle (257.07,-62.300000000000004);
\draw(257.16999999999996, -60.2) node[anchor=north west,align=left] {Computational\\  aspects of\\ higher-dimensional\\ varieties};
\draw (257.16999999999996, -60.2) rectangle (262.27,-62.300000000000004);
\draw(262.36999999999995, -60.2) node[anchor=north west,align=left] {Computational\\ aspects\\ of algebraic\\ curves};
\draw (262.36999999999995, -60.2) rectangle (266.21999999999997,-62.300000000000004);
\draw(251.71999999999997, -62.400000000000006) node[anchor=north west,align=left] {Effectivity,\\ complexity and\\ computational\\ aspects of\\ algebraic geometry};
\draw (251.71999999999997, -62.400000000000006) rectangle (256.82,-65.0);
\draw(256.91999999999996, -62.400000000000006) node[anchor=north west,align=left] {Computational\\ aspects\\ of algebraic\\ surfaces};
\draw (256.91999999999996, -62.400000000000006) rectangle (260.77,-64.5);
\draw(260.86999999999995, -62.400000000000006) node[anchor=north west,align=left] {Computational\\  algebraic\\ geometry over\\ arithmetic\\ ground fields};
\draw (260.86999999999995, -62.400000000000006) rectangle (264.71999999999997,-65.0);
\draw(251.71999999999997, -65.10000000000001) node[anchor=north west,align=left] {Computational\\ real\\ algebraic\\ geometry};
\draw (251.71999999999997, -65.10000000000001) rectangle (255.56999999999996,-67.2);
\draw(250.71999999999997, -67.4) node[anchor=north west,align=left] {\large Real algebraic and real-analytic geometry};
\draw (250.71999999999997, -67.4) rectangle (264.03,-72.80000000000001);
\draw(251.71999999999997, -68.4) node[anchor=north west,align=left] {Semialgebraic\\ sets\\ and related\\ spaces};
\draw (251.71999999999997, -68.4) rectangle (255.56999999999996,-70.5);
\draw(255.66999999999996, -68.4) node[anchor=north west,align=left] {Real-analytic\\  and\\ semi-analytic\\ sets};
\draw (255.66999999999996, -68.4) rectangle (259.52,-70.5);
\draw(259.61999999999995, -68.4) node[anchor=north west,align=left] {Nash\\ functions and\\ manifolds};
\draw (259.61999999999995, -68.4) rectangle (263.46999999999997,-70.0);
\draw(251.71999999999997, -70.60000000000001) node[anchor=north west,align=left] {Real\\ algebraic\\ sets};
\draw (251.71999999999997, -70.60000000000001) rectangle (254.56999999999996,-72.2);
\draw(254.66999999999996, -70.60000000000001) node[anchor=north west,align=left] {Topology\\ of real\\ algebraic\\ varieties};
\draw (254.66999999999996, -70.60000000000001) rectangle (257.52,-72.7);
\draw(266.41999999999996, -59.2) node[anchor=north west,align=left] {\large Tropical geometry};
\draw (266.41999999999996, -59.2) rectangle (274.31999999999994,-69.5);
\draw(267.41999999999996, -60.2) node[anchor=north west,align=left] {Foundations\\ of tropical\\  geometry\\ and relations\\ with algebra};
\draw (267.41999999999996, -60.2) rectangle (271.27,-62.800000000000004);
\draw(267.41999999999996, -62.900000000000006) node[anchor=north west,align=left] {Combinatorial\\ aspects\\ of tropical\\ varieties};
\draw (267.41999999999996, -62.900000000000006) rectangle (271.27,-65.0);
\draw(267.41999999999996, -65.10000000000001) node[anchor=north west,align=left] {Applications\\  of\\ tropical\\ geometry};
\draw (267.41999999999996, -65.10000000000001) rectangle (271.02,-67.2);
\draw(267.41999999999996, -67.30000000000001) node[anchor=north west,align=left] {Geometric\\  aspects\\ of tropical\\ varieties};
\draw (267.41999999999996, -67.30000000000001) rectangle (270.77,-69.4);
\draw(270.86999999999995, -67.30000000000001) node[anchor=north west,align=left] {Arithmetic\\  aspects\\ of tropical\\ varieties};
\draw (270.86999999999995, -67.30000000000001) rectangle (274.21999999999997,-69.4);
\draw(235.01999999999998, -75.0) node[anchor=north west,align=left] {\large Local theory in algebraic geometry};
\draw (235.01999999999998, -75.0) rectangle (247.86999999999998,-83.1);
\draw(236.01999999999998, -76.0) node[anchor=north west,align=left] {Local deformation\\ theory,\\  Artin\\ approximation, etc.};
\draw (236.01999999999998, -76.0) rectangle (241.36999999999998,-78.1);
\draw(241.46999999999997, -76.0) node[anchor=north west,align=left] {Local structure\\ of morphisms\\ in algebraic\\  geometry:\\ étale, flat, etc.};
\draw (241.46999999999997, -76.0) rectangle (246.31999999999996,-78.6);
\draw(236.01999999999998, -78.7) node[anchor=north west,align=left] {Singularities\\ in\\ algebraic\\ geometry};
\draw (236.01999999999998, -78.7) rectangle (239.86999999999998,-80.8);
\draw(239.96999999999997, -78.7) node[anchor=north west,align=left] {Deformations\\  of\\ singularities};
\draw (239.96999999999997, -78.7) rectangle (243.81999999999996,-80.3);
\draw(243.92, -78.7) node[anchor=north west,align=left] {Infinitesimal\\ methods\\ in algebraic\\ geometry};
\draw (243.92, -78.7) rectangle (247.76999999999998,-80.8);
\draw(236.01999999999998, -80.9) node[anchor=north west,align=left] {Local\\ cohomology\\ and algebraic\\ geometry};
\draw (236.01999999999998, -80.9) rectangle (239.86999999999998,-83.0);
\draw(239.96999999999997, -80.9) node[anchor=north west,align=left] {Formal\\ neighborhoods\\ in algebraic\\ geometry};
\draw (239.96999999999997, -80.9) rectangle (243.81999999999996,-83.0);
\draw(247.96999999999997, -75.0) node[anchor=north west,align=left] {\large Foundations of algebraic geometry};
\draw (247.96999999999997, -75.0) rectangle (260.36999999999995,-84.6);
\draw(248.96999999999997, -76.0) node[anchor=north west,align=left] {Fundamental constructions\\ in algebraic\\ geometry involving\\ higher and derived\\ categories (homotopical\\  algebraic\\ geometry, derived\\ algebraic geometry, etc.)};
\draw (248.96999999999997, -76.0) rectangle (255.81999999999996,-80.1);
\draw(255.91999999999996, -76.0) node[anchor=north west,align=left] {Generalizations\\ (algebraic\\ spaces, stacks)};
\draw (255.91999999999996, -76.0) rectangle (260.27,-77.6);
\draw(255.91999999999996, -77.7) node[anchor=north west,align=left] {Noncommutative\\ algebraic\\ geometry};
\draw (255.91999999999996, -77.7) rectangle (260.02,-79.3);
\draw(248.96999999999997, -80.2) node[anchor=north west,align=left] {Elementary\\ questions\\ in algebraic\\ geometry};
\draw (248.96999999999997, -80.2) rectangle (252.56999999999996,-82.3);
\draw(252.66999999999996, -80.2) node[anchor=north west,align=left] {Relevant\\ commutative\\ algebra};
\draw (252.66999999999996, -80.2) rectangle (256.02,-81.8);
\draw(256.11999999999995, -80.2) node[anchor=north west,align=left] {Logarithmic\\ algebraic\\ geometry,\\ log schemes};
\draw (256.11999999999995, -80.2) rectangle (259.46999999999997,-82.3);
\draw(248.96999999999997, -82.4) node[anchor=north west,align=left] {Geometry\\ over the\\ field with\\ one element};
\draw (248.96999999999997, -82.4) rectangle (252.31999999999996,-84.5);
\draw(252.41999999999996, -82.4) node[anchor=north west,align=left] {Varieties\\  and\\ morphisms};
\draw (252.41999999999996, -82.4) rectangle (255.26999999999995,-84.0);
\draw(255.36999999999998, -82.4) node[anchor=north west,align=left] {Schemes\\  and\\ morphisms};
\draw (255.36999999999998, -82.4) rectangle (258.21999999999997,-84.0);
\draw(260.46999999999997, -75.0) node[anchor=north west,align=left] {\large Curves in algebraic geometry};
\draw (260.46999999999997, -75.0) rectangle (272.32,-92.6);
\draw(261.46999999999997, -76.0) node[anchor=north west,align=left] {Special\\ divisors on\\ curves (gonality,\\ Brill-Noether theory)};
\draw (261.46999999999997, -76.0) rectangle (267.32,-78.1);
\draw(267.41999999999996, -76.0) node[anchor=north west,align=left] {Theta functions\\  and\\ curves; Schottky\\ problem};
\draw (267.41999999999996, -76.0) rectangle (272.02,-78.1);
\draw(261.46999999999997, -78.2) node[anchor=north west,align=left] {Riemann\\ surfaces; Weierstrass\\ points;\\ gap sequences};
\draw (261.46999999999997, -78.2) rectangle (267.32,-80.3);
\draw(267.41999999999996, -78.2) node[anchor=north west,align=left] {Special\\ algebraic curves\\  and curves\\ of low genus};
\draw (267.41999999999996, -78.2) rectangle (272.02,-80.3);
\draw(261.46999999999997, -80.4) node[anchor=north west,align=left] {Algebraic\\ functions and\\ function\\ fields in algebraic\\ geometry};
\draw (261.46999999999997, -80.4) rectangle (266.82,-83.0);
\draw(266.91999999999996, -80.4) node[anchor=north west,align=left] {Relationships\\  between\\ algebraic curves\\  and\\ integrable systems};
\draw (266.91999999999996, -80.4) rectangle (272.02,-83.0);
\draw(261.46999999999997, -83.1) node[anchor=north west,align=left] {Relationships\\  between\\ algebraic curves\\ and physics};
\draw (261.46999999999997, -83.1) rectangle (266.07,-85.19999999999999);
\draw(266.16999999999996, -83.1) node[anchor=north west,align=left] {Singularities\\  of\\ curves,\\ local rings};
\draw (266.16999999999996, -83.1) rectangle (270.02,-85.19999999999999);
\draw(261.46999999999997, -85.3) node[anchor=north west,align=left] {Automorphisms\\ of curves};
\draw (261.46999999999997, -85.3) rectangle (265.32,-86.39999999999999);
\draw(265.41999999999996, -85.3) node[anchor=north west,align=left] {Vector\\ bundles on\\ curves and\\ their moduli};
\draw (265.41999999999996, -85.3) rectangle (269.02,-87.39999999999999);
\draw(269.11999999999995, -85.3) node[anchor=north west,align=left] {Families,\\  moduli\\ of curves\\ (analytic)};
\draw (269.11999999999995, -85.3) rectangle (272.21999999999997,-87.39999999999999);
\draw(261.46999999999997, -87.5) node[anchor=north west,align=left] {Families,\\  moduli\\ of curves\\ (algebraic)};
\draw (261.46999999999997, -87.5) rectangle (264.82,-89.6);
\draw(264.91999999999996, -87.5) node[anchor=north west,align=left] {Coverings\\ of curves,\\ fundamental\\ group};
\draw (264.91999999999996, -87.5) rectangle (268.27,-89.6);
\draw(268.36999999999995, -87.5) node[anchor=north west,align=left] {Arithmetic\\ ground\\ fields\\ for curves};
\draw (268.36999999999995, -87.5) rectangle (271.46999999999997,-89.6);
\draw(261.46999999999997, -89.7) node[anchor=north west,align=left] {Jacobians,\\  Prym\\ varieties};
\draw (261.46999999999997, -89.7) rectangle (264.57,-91.3);
\draw(264.66999999999996, -89.7) node[anchor=north west,align=left] {Plane\\ and space\\ curves};
\draw (264.66999999999996, -89.7) rectangle (267.52,-91.3);
\draw(267.61999999999995, -89.7) node[anchor=north west,align=left] {Dessins\\ d’enfants\\ theory};
\draw (267.61999999999995, -89.7) rectangle (270.46999999999997,-91.3);
\draw(261.46999999999997, -91.4) node[anchor=north west,align=left] {Elliptic\\ curves};
\draw (261.46999999999997, -91.4) rectangle (264.07,-92.5);
\draw(247.96999999999997, -84.7) node[anchor=north west,align=left] {\large Affine geometry};
\draw (247.96999999999997, -84.7) rectangle (258.11999999999995,-91.8);
\draw(248.96999999999997, -85.7) node[anchor=north west,align=left] {Affine spaces\\ (automorphisms,\\ embeddings,\\  exotic\\ structures,\\ cancellation problem)};
\draw (248.96999999999997, -85.7) rectangle (254.81999999999996,-88.8);
\draw(254.91999999999996, -85.7) node[anchor=north west,align=left] {Group\\ actions on\\  affine\\ varieties};
\draw (254.91999999999996, -85.7) rectangle (258.02,-87.8);
\draw(248.96999999999997, -88.9) node[anchor=north west,align=left] {Classification\\ of affine\\ varieties};
\draw (248.96999999999997, -88.9) rectangle (253.06999999999996,-90.5);
\draw(253.16999999999996, -88.9) node[anchor=north west,align=left] {Affine\\ fibrations};
\draw (253.16999999999996, -88.9) rectangle (256.27,-90.0);
\draw(248.96999999999997, -90.60000000000001) node[anchor=north west,align=left] {Jacobian\\ problem};
\draw (248.96999999999997, -90.60000000000001) rectangle (251.56999999999996,-91.7);
\draw(235.01999999999998, -92.7) node[anchor=north west,align=left] {\large Cycles and subschemes};
\draw (235.01999999999998, -92.7) rectangle (246.42,-105.2);
\draw(236.01999999999998, -93.7) node[anchor=north west,align=left] {Intersection\\ theory, characteristic\\  classes,\\ intersection\\ multiplicities in\\ algebraic geometry};
\draw (236.01999999999998, -93.7) rectangle (242.11999999999998,-96.8);
\draw(242.21999999999997, -93.7) node[anchor=north west,align=left] {(Equivariant)\\  Chow\\ groups and\\ rings; motives};
\draw (242.21999999999997, -93.7) rectangle (246.31999999999996,-95.8);
\draw(236.01999999999998, -96.9) node[anchor=north west,align=left] {Transcendental\\  methods,\\ Hodge theory\\ (algebro-geometric\\ aspects)};
\draw (236.01999999999998, -96.9) rectangle (241.11999999999998,-99.5);
\draw(241.21999999999997, -96.9) node[anchor=north west,align=left] {Applications\\ of methods of\\  algebraic\\ \(K\)-theory in\\ algebraic geometry};
\draw (241.21999999999997, -96.9) rectangle (246.31999999999996,-99.5);
\draw(236.01999999999998, -99.60000000000001) node[anchor=north west,align=left] {Parametrization\\ (Chow\\ and Hilbert\\ schemes)};
\draw (236.01999999999998, -99.60000000000001) rectangle (240.36999999999998,-101.7);
\draw(240.46999999999997, -99.60000000000001) node[anchor=north west,align=left] {Divisors,\\ linear systems,\\ invertible\\ sheaves};
\draw (240.46999999999997, -99.60000000000001) rectangle (244.81999999999996,-101.7);
\draw(236.01999999999998, -101.80000000000001) node[anchor=north west,align=left] {Pencils,\\ nets, webs\\ in algebraic\\ geometry};
\draw (236.01999999999998, -101.80000000000001) rectangle (239.61999999999998,-103.9);
\draw(239.71999999999997, -101.80000000000001) node[anchor=north west,align=left] {Riemann-Roch\\ theorems};
\draw (239.71999999999997, -101.80000000000001) rectangle (243.31999999999996,-102.9);
\draw(243.42, -101.80000000000001) node[anchor=north west,align=left] {Algebraic\\ cycles};
\draw (243.42, -101.80000000000001) rectangle (246.26999999999998,-102.9);
\draw(236.01999999999998, -104.0) node[anchor=north west,align=left] {Torelli\\ problem};
\draw (236.01999999999998, -104.0) rectangle (238.36999999999998,-105.1);
\draw(238.46999999999997, -104.0) node[anchor=north west,align=left] {Picard\\ groups};
\draw (238.46999999999997, -104.0) rectangle (240.56999999999996,-105.1);
\draw(246.51999999999998, -92.7) node[anchor=north west,align=left] {\large Special varieties};
\draw (246.51999999999998, -92.7) rectangle (257.91999999999996,-105.9);
\draw(247.51999999999998, -93.7) node[anchor=north west,align=left] {Varieties defined\\  by ring\\ conditions (factorial,\\ Cohen-Macaulay,\\ seminormal)};
\draw (247.51999999999998, -93.7) rectangle (253.61999999999998,-96.3);
\draw(253.71999999999997, -93.7) node[anchor=north west,align=left] {Grassmannians,\\  Schubert\\ varieties,\\ flag manifolds};
\draw (253.71999999999997, -93.7) rectangle (257.82,-95.8);
\draw(247.51999999999998, -96.4) node[anchor=north west,align=left] {Compactifications;\\ symmetric\\ and spherical\\ varieties};
\draw (247.51999999999998, -96.4) rectangle (252.61999999999998,-98.5);
\draw(252.71999999999997, -96.4) node[anchor=north west,align=left] {Toric varieties,\\  Newton\\ polyhedra,\\ Okounkov bodies};
\draw (252.71999999999997, -96.4) rectangle (257.32,-98.5);
\draw(247.51999999999998, -98.60000000000001) node[anchor=north west,align=left] {Low codimension\\ problems\\ in algebraic\\ geometry};
\draw (247.51999999999998, -98.60000000000001) rectangle (251.86999999999998,-100.7);
\draw(251.96999999999997, -98.60000000000001) node[anchor=north west,align=left] {Homogeneous\\ spaces\\  and\\ generalizations};
\draw (251.96999999999997, -98.60000000000001) rectangle (256.32,-100.7);
\draw(247.51999999999998, -100.80000000000001) node[anchor=north west,align=left] {Supervarieties};
\draw (247.51999999999998, -100.80000000000001) rectangle (251.61999999999998,-101.4);
\draw(251.71999999999997, -100.80000000000001) node[anchor=north west,align=left] {Complete\\ intersections};
\draw (251.71999999999997, -100.80000000000001) rectangle (255.56999999999996,-101.9);
\draw(247.51999999999998, -102.0) node[anchor=north west,align=left] {Determinantal\\ varieties};
\draw (247.51999999999998, -102.0) rectangle (251.36999999999998,-103.1);
\draw(251.46999999999997, -102.0) node[anchor=north west,align=left] {Rational\\  and\\ unirational\\ varieties};
\draw (251.46999999999997, -102.0) rectangle (254.81999999999996,-104.1);
\draw(247.51999999999998, -103.2) node[anchor=north west,align=left] {Linkage};
\draw (247.51999999999998, -103.2) rectangle (249.86999999999998,-103.8);
\draw(254.92, -102.0) node[anchor=north west,align=left] {Character\\ varieties};
\draw (254.92, -102.0) rectangle (257.77,-103.1);
\draw(247.51999999999998, -104.2) node[anchor=north west,align=left] {Rationally\\ connected\\ varieties};
\draw (247.51999999999998, -104.2) rectangle (250.61999999999998,-105.8);
\draw(258.02, -92.7) node[anchor=north west,align=left] {\large Abelian varieties and schemes};
\draw (258.02, -92.7) rectangle (269.16999999999996,-103.0);
\draw(259.02, -93.7) node[anchor=north west,align=left] {Analytic theory\\  of abelian\\ varieties; abelian\\  integrals\\ and differentials};
\draw (259.02, -93.7) rectangle (264.12,-96.3);
\draw(264.21999999999997, -93.7) node[anchor=north west,align=left] {Algebraic\\ moduli of abelian\\  varieties,\\ classification};
\draw (264.21999999999997, -93.7) rectangle (269.07,-95.8);
\draw(259.02, -96.4) node[anchor=north west,align=left] {Picard\\ schemes, higher\\ Jacobians};
\draw (259.02, -96.4) rectangle (263.37,-98.0);
\draw(263.46999999999997, -96.4) node[anchor=north west,align=left] {Complex\\ multiplication\\ and abelian\\ varieties};
\draw (263.46999999999997, -96.4) rectangle (267.57,-98.5);
\draw(259.02, -98.60000000000001) node[anchor=north west,align=left] {Arithmetic\\ ground fields\\ for abelian\\ varieties};
\draw (259.02, -98.60000000000001) rectangle (262.87,-100.7);
\draw(262.96999999999997, -98.60000000000001) node[anchor=north west,align=left] {Theta\\ functions and\\ abelian\\ varieties};
\draw (262.96999999999997, -98.60000000000001) rectangle (266.82,-100.7);
\draw(259.02, -100.80000000000001) node[anchor=north west,align=left] {Subvarieties\\  of\\ abelian\\ varieties};
\draw (259.02, -100.80000000000001) rectangle (262.62,-102.9);
\draw(262.71999999999997, -100.80000000000001) node[anchor=north west,align=left] {Algebraic\\  theory\\ of abelian\\ varieties};
\draw (262.71999999999997, -100.80000000000001) rectangle (265.82,-102.9);
\draw(265.91999999999996, -100.80000000000001) node[anchor=north west,align=left] {Isogeny};
\draw (265.91999999999996, -100.80000000000001) rectangle (268.27,-101.4);
\draw(235.01999999999998, -106.0) node[anchor=north west,align=left] {\large Birational geometry};
\draw (235.01999999999998, -106.0) rectangle (244.67,-118.5);
\draw(236.01999999999998, -107.0) node[anchor=north west,align=left] {Global theory\\ and resolution\\ of singularities\\ (algebro-geometric\\ aspects)};
\draw (236.01999999999998, -107.0) rectangle (241.11999999999998,-109.6);
\draw(241.21999999999997, -107.0) node[anchor=north west,align=left] {Arcs and\\ motivic\\ integration};
\draw (241.21999999999997, -107.0) rectangle (244.56999999999996,-108.6);
\draw(236.01999999999998, -109.7) node[anchor=north west,align=left] {Rational\\ and\\ birational maps};
\draw (236.01999999999998, -109.7) rectangle (240.36999999999998,-111.3);
\draw(240.46999999999997, -109.7) node[anchor=north west,align=left] {McKay\\ correspondence};
\draw (240.46999999999997, -109.7) rectangle (244.56999999999996,-110.8);
\draw(236.01999999999998, -111.4) node[anchor=north west,align=left] {Birational\\ automorphisms,\\ Cremona\\  group and\\ generalizations};
\draw (236.01999999999998, -111.4) rectangle (240.36999999999998,-114.0);
\draw(240.46999999999997, -111.4) node[anchor=north west,align=left] {Minimal model\\  program\\ (Mori theory,\\ extremal rays)};
\draw (240.46999999999997, -111.4) rectangle (244.56999999999996,-113.5);
\draw(236.01999999999998, -114.1) node[anchor=north west,align=left] {Rationality\\ questions\\ in algebraic\\ geometry};
\draw (236.01999999999998, -114.1) rectangle (239.61999999999998,-116.19999999999999);
\draw(239.71999999999997, -114.1) node[anchor=north west,align=left] {Coverings\\ in algebraic\\ geometry};
\draw (239.71999999999997, -114.1) rectangle (243.31999999999996,-115.69999999999999);
\draw(236.01999999999998, -116.3) node[anchor=north west,align=left] {Ramification\\ problems\\ in algebraic\\ geometry};
\draw (236.01999999999998, -116.3) rectangle (239.61999999999998,-118.39999999999999);
\draw(239.71999999999997, -116.3) node[anchor=north west,align=left] {Embeddings\\ in algebraic\\ geometry};
\draw (239.71999999999997, -116.3) rectangle (243.31999999999996,-117.89999999999999);
\draw(244.76999999999998, -106.0) node[anchor=north west,align=left] {\large Algebraic groups};
\draw (244.76999999999998, -106.0) rectangle (254.42,-115.3);
\draw(245.76999999999998, -107.0) node[anchor=north west,align=left] {Classical\\ groups\\ (algebro-geometric\\ aspects)};
\draw (245.76999999999998, -107.0) rectangle (250.86999999999998,-109.1);
\draw(250.96999999999997, -107.0) node[anchor=north west,align=left] {Formal\\ groups,\\ \(p\)-divisible\\ groups};
\draw (250.96999999999997, -107.0) rectangle (254.31999999999996,-109.1);
\draw(245.76999999999998, -109.2) node[anchor=north west,align=left] {Affine algebraic\\ groups,\\ hyperalgebra\\ constructions};
\draw (245.76999999999998, -109.2) rectangle (250.36999999999998,-111.3);
\draw(250.46999999999997, -109.2) node[anchor=north west,align=left] {Group actions\\ on varieties\\ or schemes\\ (quotients)};
\draw (250.46999999999997, -109.2) rectangle (254.31999999999996,-111.3);
\draw(245.76999999999998, -111.4) node[anchor=north west,align=left] {Other\\ algebraic groups\\ (geometric\\ aspects)};
\draw (245.76999999999998, -111.4) rectangle (250.36999999999998,-113.5);
\draw(250.46999999999997, -111.4) node[anchor=north west,align=left] {Group\\ varieties};
\draw (250.46999999999997, -111.4) rectangle (253.31999999999996,-112.5);
\draw(245.76999999999998, -113.6) node[anchor=north west,align=left] {Geometric\\ invariant\\ theory};
\draw (245.76999999999998, -113.6) rectangle (248.61999999999998,-115.19999999999999);
\draw(248.71999999999997, -113.6) node[anchor=north west,align=left] {Group\\ schemes};
\draw (248.71999999999997, -113.6) rectangle (251.06999999999996,-114.69999999999999);
\draw(275.66, -1) node[anchor=north west,align=left] {\LARGE Commutative algebra};
\draw (275.66, -1) rectangle (315.97,-49.1);
\draw(276.66, -2) node[anchor=north west,align=left] {\large Chain conditions, finiteness conditions in commutative ring theory};
\draw (276.66, -2) rectangle (297.72,-6.199999999999999);
\draw(277.66, -3) node[anchor=north west,align=left] {Commutative\\ rings and modules\\  of finite\\ generation or\\ presentation;\\ number of generators};
\draw (277.66, -3) rectangle (283.26000000000005,-6.1);
\draw(283.36, -3) node[anchor=north west,align=left] {Commutative\\ Artinian rings\\ and modules,\\ finite-dimensional\\ algebras};
\draw (283.36, -3) rectangle (288.46000000000004,-5.6);
\draw(288.56, -3) node[anchor=north west,align=left] {Commutative\\ Noetherian\\  rings\\ and modules};
\draw (288.56, -3) rectangle (291.91,-5.1);
\draw(297.82000000000005, -2) node[anchor=north west,align=left] {\large Theory of modules and ideals in commutative rings};
\draw (297.82000000000005, -2) rectangle (315.87000000000006,-12.8);
\draw(298.82000000000005, -3) node[anchor=north west,align=left] {Structure,\\ classification\\ theorems for modules\\ and ideals in\\ commutative rings};
\draw (298.82000000000005, -3) rectangle (304.4200000000001,-5.6);
\draw(304.52000000000004, -3) node[anchor=north west,align=left] {Dimension\\ theory, depth,\\ related commutative\\  rings\\ (catenary, etc.)};
\draw (304.52000000000004, -3) rectangle (309.87000000000006,-5.6);
\draw(309.97, -3) node[anchor=north west,align=left] {Projective\\ and free\\ modules and\\ ideals in\\ commutative rings};
\draw (309.97, -3) rectangle (314.82000000000005,-5.6);
\draw(298.82000000000005, -5.7) node[anchor=north west,align=left] {Injective\\ and flat\\ modules and\\ ideals in\\ commutative rings};
\draw (298.82000000000005, -5.7) rectangle (303.6700000000001,-8.3);
\draw(303.77000000000004, -5.7) node[anchor=north west,align=left] {Torsion\\ modules and\\ ideals in\\ commutative rings};
\draw (303.77000000000004, -5.7) rectangle (308.62000000000006,-7.800000000000001);
\draw(308.72, -5.7) node[anchor=north west,align=left] {Other special\\  types of\\ modules and\\ ideals in\\ commutative rings};
\draw (308.72, -5.7) rectangle (313.57000000000005,-8.3);
\draw(313.6700000000001, -5.7) node[anchor=north west,align=left] {Class\\ groups};
\draw (313.6700000000001, -5.7) rectangle (315.7700000000001,-6.800000000000001);
\draw(298.82000000000005, -8.4) node[anchor=north west,align=left] {Linkage,\\ complete\\ intersections\\ and determinantal\\ ideals};
\draw (298.82000000000005, -8.4) rectangle (303.6700000000001,-11.0);
\draw(303.77000000000004, -8.4) node[anchor=north west,align=left] {Module\\ categories\\ and\\ commutative rings};
\draw (303.77000000000004, -8.4) rectangle (308.62000000000006,-10.5);
\draw(308.72, -8.4) node[anchor=north west,align=left] {Theory of modules\\ and ideals\\ in commutative\\ rings described\\ by combinatorial\\ properties};
\draw (308.72, -8.4) rectangle (313.57000000000005,-11.5);
\draw(298.82000000000005, -11.600000000000001) node[anchor=north west,align=left] {Cohen-Macaulay\\ modules};
\draw (298.82000000000005, -11.600000000000001) rectangle (302.9200000000001,-12.700000000000001);
\draw(276.66, -6.299999999999999) node[anchor=north west,align=left] {\large Computational aspects and applications of commutative rings};
\draw (276.66, -6.299999999999999) rectangle (297.21000000000004,-12.7);
\draw(277.66, -7.299999999999999) node[anchor=north west,align=left] {Applications of\\  commutative\\ algebra (e.g., to\\  statistics,\\ control theory,\\ optimization, etc.)};
\draw (277.66, -7.299999999999999) rectangle (283.01000000000005,-10.399999999999999);
\draw(283.11, -7.299999999999999) node[anchor=north west,align=left] {Gröbner bases;\\ other bases for\\ ideals and modules\\  (e.g., Janet\\ and border bases)};
\draw (283.11, -7.299999999999999) rectangle (288.21000000000004,-9.899999999999999);
\draw(288.31, -7.299999999999999) node[anchor=north west,align=left] {Polynomials,\\ factorization\\  in\\ commutative rings};
\draw (288.31, -7.299999999999999) rectangle (293.16,-9.399999999999999);
\draw(293.26000000000005, -7.299999999999999) node[anchor=north west,align=left] {Computational\\ homological\\ algebra};
\draw (293.26000000000005, -7.299999999999999) rectangle (297.11000000000007,-8.899999999999999);
\draw(277.66, -10.5) node[anchor=north west,align=left] {Solving\\ polynomial\\ systems;\\ resultants};
\draw (277.66, -10.5) rectangle (280.76000000000005,-12.6);
\draw(276.66, -12.9) node[anchor=north west,align=left] {\large Arithmetic rings and other special commutative rings};
\draw (276.66, -12.9) rectangle (296.46000000000004,-23.200000000000003);
\draw(277.66, -13.9) node[anchor=north west,align=left] {Commutative rings\\  defined by\\ monomial ideals;\\ Stanley-Reisner\\  face rings;\\ simplicial complexes};
\draw (277.66, -13.9) rectangle (283.26000000000005,-17.0);
\draw(283.36, -13.9) node[anchor=north west,align=left] {Commutative rings\\  defined by\\ factorization\\ properties (e.g.,\\ atomic, factorial,\\ half-factorial)};
\draw (283.36, -13.9) rectangle (288.46000000000004,-17.0);
\draw(288.56, -13.9) node[anchor=north west,align=left] {Polynomial\\ rings and\\ ideals; rings\\ of integer-valued\\ polynomials};
\draw (288.56, -13.9) rectangle (293.41,-16.5);
\draw(293.51000000000005, -13.9) node[anchor=north west,align=left] {Principal\\ ideal\\ rings};
\draw (293.51000000000005, -13.9) rectangle (296.36000000000007,-15.5);
\draw(277.66, -17.1) node[anchor=north west,align=left] {Commutative\\ rings defined\\ by binomial\\  ideals,\\ toric rings, etc.};
\draw (277.66, -17.1) rectangle (282.51000000000005,-19.700000000000003);
\draw(282.61, -17.1) node[anchor=north west,align=left] {Other\\ commutative rings\\  defined\\ by combinatorial\\ properties};
\draw (282.61, -17.1) rectangle (287.46000000000004,-19.700000000000003);
\draw(287.56, -17.1) node[anchor=north west,align=left] {Dedekind,\\ Prüfer, Krull\\ and Mori rings\\ and their\\ generalizations};
\draw (287.56, -17.1) rectangle (291.91,-19.700000000000003);
\draw(292.01000000000005, -17.1) node[anchor=north west,align=left] {Euclidean\\  rings\\ and\\ generalizations};
\draw (292.01000000000005, -17.1) rectangle (296.36000000000007,-19.200000000000003);
\draw(277.66, -19.8) node[anchor=north west,align=left] {Rings with\\ straightening\\  laws,\\ Hodge algebras};
\draw (277.66, -19.8) rectangle (281.76000000000005,-21.900000000000002);
\draw(281.86, -19.8) node[anchor=north west,align=left] {Witt vectors\\  and\\ related rings};
\draw (281.86, -19.8) rectangle (285.71000000000004,-21.400000000000002);
\draw(285.81, -19.8) node[anchor=north west,align=left] {Formal\\ power\\ series rings};
\draw (285.81, -19.8) rectangle (289.41,-21.400000000000002);
\draw(289.51000000000005, -19.8) node[anchor=north west,align=left] {Seminormal\\ rings};
\draw (289.51000000000005, -19.8) rectangle (292.61000000000007,-20.900000000000002);
\draw(292.71000000000004, -19.8) node[anchor=north west,align=left] {Valuation\\ rings};
\draw (292.71000000000004, -19.8) rectangle (295.56000000000006,-20.900000000000002);
\draw(277.66, -22.0) node[anchor=north west,align=left] {Excellent\\ rings};
\draw (277.66, -22.0) rectangle (280.51000000000005,-23.1);
\draw(280.61, -22.0) node[anchor=north west,align=left] {Cluster\\ algebras};
\draw (280.61, -22.0) rectangle (283.21000000000004,-23.1);
\draw(296.56000000000006, -12.9) node[anchor=north west,align=left] {\large Commutative ring extensions and related topics};
\draw (296.56000000000006, -12.9) rectangle (314.11000000000007,-22.0);
\draw(297.56000000000006, -13.9) node[anchor=north west,align=left] {Integral\\ closure of\\ commutative rings\\ and ideals};
\draw (297.56000000000006, -13.9) rectangle (302.4100000000001,-16.0);
\draw(302.51000000000005, -13.9) node[anchor=north west,align=left] {Rings of\\ fractions and\\ localization\\  for\\ commutative rings};
\draw (302.51000000000005, -13.9) rectangle (307.36000000000007,-16.5);
\draw(307.46000000000004, -13.9) node[anchor=north west,align=left] {Galois theory\\  and\\ commutative ring\\ extensions};
\draw (307.46000000000004, -13.9) rectangle (312.06000000000006,-16.0);
\draw(297.56000000000006, -16.6) node[anchor=north west,align=left] {Étale and\\ flat extensions;\\ Henselization;\\  Artin\\ approximation};
\draw (297.56000000000006, -16.6) rectangle (302.1600000000001,-19.200000000000003);
\draw(302.26000000000005, -16.6) node[anchor=north west,align=left] {Extension\\  theory\\ of commutative\\ rings};
\draw (302.26000000000005, -16.6) rectangle (306.36000000000007,-18.700000000000003);
\draw(306.46000000000004, -16.6) node[anchor=north west,align=left] {Morphisms\\ of commutative\\ rings};
\draw (306.46000000000004, -16.6) rectangle (310.56000000000006,-18.200000000000003);
\draw(310.6600000000001, -16.6) node[anchor=north west,align=left] {Polynomials\\ over\\ commutative\\ rings};
\draw (310.6600000000001, -16.6) rectangle (314.0100000000001,-18.700000000000003);
\draw(297.56000000000006, -19.3) node[anchor=north west,align=left] {Integral\\ dependence in\\ commutative\\ rings; going\\ up, going down};
\draw (297.56000000000006, -19.3) rectangle (301.6600000000001,-21.900000000000002);
\draw(301.76000000000005, -19.3) node[anchor=north west,align=left] {Completion\\ of commutative\\ rings};
\draw (301.76000000000005, -19.3) rectangle (305.86000000000007,-20.900000000000002);
\draw(296.56000000000006, -22.1) node[anchor=north west,align=left] {\large History of\\ commutative algebra};
\draw (296.56000000000006, -22.1) rectangle (303.05000000000007,-23.200000000000003);
\draw(276.66, -23.3) node[anchor=north west,align=left] {\large Homological methods in commutative ring theory};
\draw (276.66, -23.3) rectangle (293.76000000000005,-35.6);
\draw(277.66, -24.3) node[anchor=north west,align=left] {Homological\\ conjectures\\ (intersection\\ theorems) in\\ commutative ring theory};
\draw (277.66, -24.3) rectangle (284.01000000000005,-26.900000000000002);
\draw(284.11, -24.3) node[anchor=north west,align=left] {(Co)homology of\\ commutative rings\\ and algebras (e.g.,\\  Hochschild,\\ André-Quillen, cyclic,\\ dihedral, etc.)};
\draw (284.11, -24.3) rectangle (290.21000000000004,-27.400000000000002);
\draw(290.31, -24.3) node[anchor=north west,align=left] {Torsion\\ theory for\\ commutative\\ rings};
\draw (290.31, -24.3) rectangle (293.66,-26.400000000000002);
\draw(277.66, -27.5) node[anchor=north west,align=left] {Syzygies,\\ resolutions,\\ complexes\\  and\\ commutative rings};
\draw (277.66, -27.5) rectangle (282.51000000000005,-30.1);
\draw(282.61, -27.5) node[anchor=north west,align=left] {Homological\\ dimension\\  and\\ commutative rings};
\draw (282.61, -27.5) rectangle (287.46000000000004,-29.6);
\draw(287.56, -27.5) node[anchor=north west,align=left] {Homological\\ functors on\\ modules of\\ commutative rings\\ (Tor, Ext, etc.)};
\draw (287.56, -27.5) rectangle (292.41,-30.1);
\draw(277.66, -30.200000000000003) node[anchor=north west,align=left] {Deformations\\ and infinitesimal\\  methods\\ in commutative\\ ring theory};
\draw (277.66, -30.200000000000003) rectangle (282.51000000000005,-32.800000000000004);
\draw(282.61, -30.200000000000003) node[anchor=north west,align=left] {Grothendieck\\  groups,\\ \(K\)-theory and\\ commutative rings};
\draw (282.61, -30.200000000000003) rectangle (287.46000000000004,-32.300000000000004);
\draw(287.56, -30.200000000000003) node[anchor=north west,align=left] {Derived\\ categories\\ and commutative\\ rings};
\draw (287.56, -30.200000000000003) rectangle (291.91,-32.300000000000004);
\draw(277.66, -32.900000000000006) node[anchor=north west,align=left] {Hilbert-Samuel\\  and\\ Hilbert-Kunz\\ functions;\\ Poincaré series};
\draw (277.66, -32.900000000000006) rectangle (282.01000000000005,-35.50000000000001);
\draw(282.11, -32.900000000000006) node[anchor=north west,align=left] {Local\\ cohomology\\ and commutative\\ rings};
\draw (282.11, -32.900000000000006) rectangle (286.46000000000004,-35.00000000000001);
\draw(293.86, -23.3) node[anchor=north west,align=left] {\large Applications of logic to commutative algebra};
\draw (293.86, -23.3) rectangle (308.1,-26.5);
\draw(294.86, -24.3) node[anchor=north west,align=left] {Applications\\  of logic\\ to commutative\\ algebra};
\draw (294.86, -24.3) rectangle (298.96000000000004,-26.400000000000002);
\draw(293.86, -26.6) node[anchor=north west,align=left] {\large Topological rings and modules};
\draw (293.86, -26.6) rectangle (304.91,-31.0);
\draw(294.86, -27.6) node[anchor=north west,align=left] {Global\\ topological\\ rings};
\draw (294.86, -27.6) rectangle (298.21000000000004,-29.200000000000003);
\draw(298.31, -27.6) node[anchor=north west,align=left] {Analytical\\ algebras\\ and rings};
\draw (298.31, -27.6) rectangle (301.41,-29.200000000000003);
\draw(301.51, -27.6) node[anchor=north west,align=left] {Complete\\  rings,\\ completion};
\draw (301.51, -27.6) rectangle (304.61,-29.200000000000003);
\draw(294.86, -29.3) node[anchor=north west,align=left] {Henselian\\ rings};
\draw (294.86, -29.3) rectangle (297.71000000000004,-30.400000000000002);
\draw(297.81, -29.3) node[anchor=north west,align=left] {Ordered\\ rings};
\draw (297.81, -29.3) rectangle (300.16,-30.400000000000002);
\draw(300.26, -29.3) node[anchor=north west,align=left] {Real\\ algebra};
\draw (300.26, -29.3) rectangle (302.61,-30.400000000000002);
\draw(302.71000000000004, -29.3) node[anchor=north west,align=left] {Power\\ series\\ rings};
\draw (302.71000000000004, -29.3) rectangle (304.81000000000006,-30.900000000000002);
\draw(293.86, -31.1) node[anchor=north west,align=left] {\large Finite commutative rings};
\draw (293.86, -31.1) rectangle (301.90000000000003,-34.300000000000004);
\draw(294.86, -32.1) node[anchor=north west,align=left] {Structure\\ of finite\\ commutative\\ rings};
\draw (294.86, -32.1) rectangle (298.21000000000004,-34.2);
\draw(298.31, -32.1) node[anchor=north west,align=left] {Polynomials\\ and finite\\ commutative\\ rings};
\draw (298.31, -32.1) rectangle (301.66,-34.2);
\draw(308.20000000000005, -23.3) node[anchor=north west,align=left] {\large Differential algebra};
\draw (308.20000000000005, -23.3) rectangle (315.00000000000006,-30.9);
\draw(309.20000000000005, -24.3) node[anchor=north west,align=left] {Modules\\ of\\ differentials};
\draw (309.20000000000005, -24.3) rectangle (313.05000000000007,-25.900000000000002);
\draw(309.20000000000005, -26.0) node[anchor=north west,align=left] {Commutative\\ rings of\\ differential\\ operators and\\ their modules};
\draw (309.20000000000005, -26.0) rectangle (313.05000000000007,-28.6);
\draw(309.20000000000005, -28.700000000000003) node[anchor=north west,align=left] {Derivations\\  and\\ commutative\\ rings};
\draw (309.20000000000005, -28.700000000000003) rectangle (312.55000000000007,-30.800000000000004);
\draw(276.66, -35.7) node[anchor=north west,align=left] {\large General commutative ring theory};
\draw (276.66, -35.7) rectangle (288.56,-49.0);
\draw(277.66, -36.7) node[anchor=north west,align=left] {General commutative\\ ring theory and\\  combinatorics\\ (zero-divisor graphs,\\ annihilating-ideal\\ graphs, etc.)};
\draw (277.66, -36.7) rectangle (283.51000000000005,-39.800000000000004);
\draw(283.61, -36.7) node[anchor=north west,align=left] {Ideals and\\ multiplicative\\  ideal\\ theory in\\ commutative rings};
\draw (283.61, -36.7) rectangle (288.46000000000004,-39.300000000000004);
\draw(277.66, -39.900000000000006) node[anchor=north west,align=left] {Divisibility\\ and factorizations\\  in\\ commutative rings};
\draw (277.66, -39.900000000000006) rectangle (282.76000000000005,-42.00000000000001);
\draw(282.86, -39.900000000000006) node[anchor=north west,align=left] {Associated graded\\  rings of\\ ideals (Rees ring,\\  form ring),\\ analytic spread\\ and related topics};
\draw (282.86, -39.900000000000006) rectangle (287.96000000000004,-43.00000000000001);
\draw(277.66, -43.1) node[anchor=north west,align=left] {Characteristic \(p\)\\ methods (Frobenius\\ endomorphism)\\  and reduction\\ to characteristic\\ \(p\); tight closure};
\draw (277.66, -43.1) rectangle (282.76000000000005,-46.2);
\draw(282.86, -43.1) node[anchor=north west,align=left] {Valuations\\ and their\\ generalizations\\  for\\ commutative rings};
\draw (282.86, -43.1) rectangle (287.71000000000004,-45.7);
\draw(277.66, -46.300000000000004) node[anchor=north west,align=left] {Actions of\\ groups on\\ commutative\\ rings;\\ invariant theory};
\draw (277.66, -46.300000000000004) rectangle (282.26000000000005,-48.900000000000006);
\draw(282.36, -46.300000000000004) node[anchor=north west,align=left] {Graded\\ rings};
\draw (282.36, -46.300000000000004) rectangle (284.46000000000004,-47.400000000000006);
\draw(288.66, -35.7) node[anchor=north west,align=left] {\large Local rings and semilocal rings};
\draw (288.66, -35.7) rectangle (298.87,-40.6);
\draw(289.66, -36.7) node[anchor=north west,align=left] {Special types\\ (Cohen-Macaulay,\\ Gorenstein,\\ Buchsbaum, etc.)};
\draw (289.66, -36.7) rectangle (294.26000000000005,-38.800000000000004);
\draw(294.36, -36.7) node[anchor=north west,align=left] {Multiplicity\\ theory\\  and\\ related topics};
\draw (294.36, -36.7) rectangle (298.46000000000004,-38.800000000000004);
\draw(289.66, -38.900000000000006) node[anchor=north west,align=left] {Regular\\ local\\ rings};
\draw (289.66, -38.900000000000006) rectangle (292.01000000000005,-40.50000000000001);
\draw(288.66, -40.7) node[anchor=north west,align=left] {\large Integral domains};
\draw (288.66, -40.7) rectangle (294.22,-42.900000000000006);
\draw(289.66, -41.7) node[anchor=north west,align=left] {Integral\\ domains};
\draw (289.66, -41.7) rectangle (292.26000000000005,-42.800000000000004);
\draw(275.66, -49.2) node[anchor=north west,align=left] {\LARGE Topological groups, Lie groups};
\draw (275.66, -49.2) rectangle (306.64000000000004,-85.80000000000001);
\draw(276.66, -50.2) node[anchor=north west,align=left] {\large Topological and differentiable algebraic systems};
\draw (276.66, -50.2) rectangle (292.51000000000005,-58.800000000000004);
\draw(277.66, -51.2) node[anchor=north west,align=left] {Topological\\ groupoids\\ (including\\ differentiable and\\ Lie groupoids)};
\draw (277.66, -51.2) rectangle (282.76000000000005,-53.800000000000004);
\draw(282.86, -51.2) node[anchor=north west,align=left] {Other topological\\ algebraic\\  systems\\ and their\\ representations};
\draw (282.86, -51.2) rectangle (287.71000000000004,-53.800000000000004);
\draw(287.81, -51.2) node[anchor=north west,align=left] {Topological\\ semilattices,\\  lattices\\ and applications};
\draw (287.81, -51.2) rectangle (292.41,-53.300000000000004);
\draw(277.66, -53.900000000000006) node[anchor=north west,align=left] {Representations\\ of general\\ topological\\ groups and\\ semigroups};
\draw (277.66, -53.900000000000006) rectangle (282.01000000000005,-56.50000000000001);
\draw(282.11, -53.900000000000006) node[anchor=north west,align=left] {Structure\\ of general\\ topological\\ groups};
\draw (282.11, -53.900000000000006) rectangle (285.46000000000004,-56.00000000000001);
\draw(285.56, -53.900000000000006) node[anchor=north west,align=left] {Analysis\\ on general\\ topological\\ groups};
\draw (285.56, -53.900000000000006) rectangle (288.91,-56.00000000000001);
\draw(289.01000000000005, -53.900000000000006) node[anchor=north west,align=left] {Structure\\  of\\ topological\\ semigroups};
\draw (289.01000000000005, -53.900000000000006) rectangle (292.36000000000007,-56.00000000000001);
\draw(277.66, -56.6) node[anchor=north west,align=left] {Analysis\\  on\\ topological\\ semigroups};
\draw (277.66, -56.6) rectangle (281.01000000000005,-58.7);
\draw(292.61, -50.2) node[anchor=north west,align=left] {\large Locally compact abelian groups (LCA groups)};
\draw (292.61, -50.2) rectangle (306.54,-53.400000000000006);
\draw(293.61, -51.2) node[anchor=north west,align=left] {General\\ properties and\\  structure\\ of LCA groups};
\draw (293.61, -51.2) rectangle (297.71000000000004,-53.300000000000004);
\draw(297.81, -51.2) node[anchor=north west,align=left] {Structure\\ of group\\ algebras of\\ LCA groups};
\draw (297.81, -51.2) rectangle (301.16,-53.300000000000004);
\draw(292.61, -53.5) node[anchor=north west,align=left] {\large Noncompact transformation groups};
\draw (292.61, -53.5) rectangle (303.13,-58.4);
\draw(293.61, -54.5) node[anchor=north west,align=left] {General\\ theory of group\\  and\\ pseudogroup actions};
\draw (293.61, -54.5) rectangle (298.96000000000004,-56.6);
\draw(299.06, -54.5) node[anchor=north west,align=left] {Groups as\\ automorphisms\\ of other\\ structures};
\draw (299.06, -54.5) rectangle (302.91,-56.6);
\draw(293.61, -56.7) node[anchor=north west,align=left] {Homogeneous\\ spaces};
\draw (293.61, -56.7) rectangle (296.96000000000004,-57.800000000000004);
\draw(297.06, -56.7) node[anchor=north west,align=left] {Measurable\\ group\\ actions};
\draw (297.06, -56.7) rectangle (300.16,-58.300000000000004);
\draw(276.66, -58.900000000000006) node[anchor=north west,align=left] {\large Locally compact groups and their algebras};
\draw (276.66, -58.900000000000006) rectangle (291.51000000000005,-70.2);
\draw(277.66, -59.900000000000006) node[anchor=north west,align=left] {Kazhdan’s\\ property (T), the\\  Haagerup\\ property, and\\ generalizations};
\draw (277.66, -59.900000000000006) rectangle (282.51000000000005,-62.50000000000001);
\draw(282.61, -59.900000000000006) node[anchor=north west,align=left] {Unitary\\ representations\\ of locally\\ compact groups};
\draw (282.61, -59.900000000000006) rectangle (286.96000000000004,-62.00000000000001);
\draw(287.06, -59.900000000000006) node[anchor=north west,align=left] {Other\\ representations\\ of locally\\ compact groups};
\draw (287.06, -59.900000000000006) rectangle (291.41,-62.00000000000001);
\draw(277.66, -62.60000000000001) node[anchor=north west,align=left] {Group\\ algebras of\\ locally compact\\ groups};
\draw (277.66, -62.60000000000001) rectangle (282.01000000000005,-64.7);
\draw(282.11, -62.60000000000001) node[anchor=north west,align=left] {Representations\\ of group\\ algebras};
\draw (282.11, -62.60000000000001) rectangle (286.46000000000004,-64.2);
\draw(286.56, -62.60000000000001) node[anchor=north west,align=left] {\(C^*\)-algebras\\ and \(W^*\)-algebras\\ in relation\\  to group\\ representations};
\draw (286.56, -62.60000000000001) rectangle (290.91,-65.2);
\draw(277.66, -65.30000000000001) node[anchor=north west,align=left] {Induced\\ representations\\ for locally\\ compact groups};
\draw (277.66, -65.30000000000001) rectangle (282.01000000000005,-67.4);
\draw(282.11, -65.30000000000001) node[anchor=north west,align=left] {General\\ properties and\\  structure\\ of locally\\ compact groups};
\draw (282.11, -65.30000000000001) rectangle (286.21000000000004,-67.9);
\draw(286.31, -65.30000000000001) node[anchor=north west,align=left] {Duality\\ theorems for\\  locally\\ compact groups};
\draw (286.31, -65.30000000000001) rectangle (290.41,-67.4);
\draw(277.66, -68.0) node[anchor=north west,align=left] {Automorphism\\ groups of\\  locally\\ compact groups};
\draw (277.66, -68.0) rectangle (281.76000000000005,-70.1);
\draw(281.86, -68.0) node[anchor=north west,align=left] {Rigidity\\ in locally\\ compact\\ groups};
\draw (281.86, -68.0) rectangle (284.96000000000004,-70.1);
\draw(285.06, -68.0) node[anchor=north west,align=left] {Ergodic\\ theory\\ on groups};
\draw (285.06, -68.0) rectangle (287.91,-69.6);
\draw(291.61, -58.900000000000006) node[anchor=north west,align=left] {\large Lie groups};
\draw (291.61, -58.900000000000006) rectangle (304.51,-85.7);
\draw(292.61, -59.900000000000006) node[anchor=north west,align=left] {Geometric\\ Langlands\\ program:\\ representation-theoretic\\ aspects};
\draw (292.61, -59.900000000000006) rectangle (299.21000000000004,-62.50000000000001);
\draw(299.31, -59.900000000000006) node[anchor=north west,align=left] {General\\ properties and\\ structure of other\\ Lie groups};
\draw (299.31, -59.900000000000006) rectangle (304.41,-62.00000000000001);
\draw(292.61, -62.60000000000001) node[anchor=north west,align=left] {Representations\\ of nilpotent and\\  solvable Lie\\ groups (special\\ orbital integrals,\\  non-type I\\ representations, etc.)};
\draw (292.61, -62.60000000000001) rectangle (298.71000000000004,-66.2);
\draw(298.81, -62.60000000000001) node[anchor=north west,align=left] {Infinite-dimensional\\  Lie\\ groups and their\\ Lie algebras:\\ general properties};
\draw (298.81, -62.60000000000001) rectangle (304.41,-65.2);
\draw(292.61, -66.30000000000001) node[anchor=north west,align=left] {Representations\\  of Lie and\\ real algebraic\\ groups: algebraic\\  methods\\ (Verma modules, etc.)};
\draw (292.61, -66.30000000000001) rectangle (298.46000000000004,-69.4);
\draw(298.56, -66.30000000000001) node[anchor=north west,align=left] {Analysis on\\ and representations\\  of\\ infinite-dimensional\\ Lie groups};
\draw (298.56, -66.30000000000001) rectangle (304.16,-68.9);
\draw(292.61, -69.5) node[anchor=north west,align=left] {Representations\\ of Lie and\\ linear algebraic\\  groups\\ over local fields};
\draw (292.61, -69.5) rectangle (297.46000000000004,-72.1);
\draw(297.56, -69.5) node[anchor=north west,align=left] {Representations\\  of Lie and\\ linear algebraic\\  groups over\\ real fields:\\ analytic methods};
\draw (297.56, -69.5) rectangle (302.16,-72.6);
\draw(292.61, -72.7) node[anchor=north west,align=left] {Representations\\  of Lie and\\ linear algebraic\\  groups over\\ global fields\\ and adèle rings};
\draw (292.61, -72.7) rectangle (297.21000000000004,-75.8);
\draw(297.31, -72.7) node[anchor=north west,align=left] {Applications\\ of Lie groups\\ to the sciences;\\ explicit\\ representations};
\draw (297.31, -72.7) rectangle (301.91,-75.3);
\draw(292.61, -75.9) node[anchor=north west,align=left] {General\\ properties and\\ structure of\\ real Lie groups};
\draw (292.61, -75.9) rectangle (296.96000000000004,-78.0);
\draw(297.06, -75.9) node[anchor=north west,align=left] {Semisimple\\ Lie groups\\ and their\\ representations};
\draw (297.06, -75.9) rectangle (301.41,-78.0);
\draw(301.51, -75.9) node[anchor=north west,align=left] {Local Lie\\ groups};
\draw (301.51, -75.9) rectangle (304.36,-77.0);
\draw(292.61, -78.10000000000001) node[anchor=north west,align=left] {Loop groups\\ and related\\ constructions,\\ group-theoretic\\ treatment};
\draw (292.61, -78.10000000000001) rectangle (296.96000000000004,-80.7);
\draw(297.06, -78.10000000000001) node[anchor=north west,align=left] {Structure and\\ representation\\  of the\\ Lorentz group};
\draw (297.06, -78.10000000000001) rectangle (301.16,-80.2);
\draw(301.26, -78.10000000000001) node[anchor=north west,align=left] {Nilpotent\\  and\\ solvable\\ Lie groups};
\draw (301.26, -78.10000000000001) rectangle (304.36,-80.2);
\draw(292.61, -80.80000000000001) node[anchor=north west,align=left] {General\\ properties\\ and structure\\ of complex\\ Lie groups};
\draw (292.61, -80.80000000000001) rectangle (296.46000000000004,-83.4);
\draw(296.56, -80.80000000000001) node[anchor=north west,align=left] {Discrete\\ subgroups\\ of Lie groups};
\draw (296.56, -80.80000000000001) rectangle (300.41,-82.4);
\draw(300.51, -80.80000000000001) node[anchor=north west,align=left] {Continuous\\ cohomologyof\\ Lie groups};
\draw (300.51, -80.80000000000001) rectangle (304.11,-82.4);
\draw(292.61, -83.5) node[anchor=north west,align=left] {Analysis\\ on real\\ and complex\\ Lie groups};
\draw (292.61, -83.5) rectangle (295.96000000000004,-85.6);
\draw(296.06, -83.5) node[anchor=north west,align=left] {Lie\\ algebras of\\ Lie groups};
\draw (296.06, -83.5) rectangle (299.41,-85.1);
\draw(299.51, -83.5) node[anchor=north west,align=left] {Analysis\\ on \(p\)-adic\\ Lie groups};
\draw (299.51, -83.5) rectangle (302.61,-85.1);
\draw(276.66, -70.30000000000001) node[anchor=north west,align=left] {\large Computational methods\\ for problems pertaining\\ to topological groups};
\draw (276.66, -70.30000000000001) rectangle (284.39000000000004,-71.9);
\draw(276.66, -72.0) node[anchor=north west,align=left] {\large History of\\ topological groups};
\draw (276.66, -72.0) rectangle (282.84000000000003,-73.1);
\draw(276.66, -73.2) node[anchor=north west,align=left] {\large Compact groups};
\draw (276.66, -73.2) rectangle (281.6,-75.4);
\draw(277.66, -74.2) node[anchor=north west,align=left] {Compact\\ groups};
\draw (277.66, -74.2) rectangle (280.01000000000005,-75.3);
\draw(275.66, -85.9) node[anchor=north west,align=left] {\LARGE Nonassociative rings and algebras};
\draw (275.66, -85.9) rectangle (305.06,-137.3);
\draw(276.66, -86.9) node[anchor=north west,align=left] {\large Jordan algebras (algebras, triples and pairs)};
\draw (276.66, -86.9) rectangle (293.71000000000004,-96.7);
\draw(277.66, -87.9) node[anchor=north west,align=left] {Jordan\\ structures associated\\  with\\ other structures};
\draw (277.66, -87.9) rectangle (283.51000000000005,-90.0);
\draw(283.61, -87.9) node[anchor=north west,align=left] {Finite-dimensional\\ structures of\\ Jordan algebras};
\draw (283.61, -87.9) rectangle (288.71000000000004,-89.5);
\draw(288.81, -87.9) node[anchor=north west,align=left] {Associated\\ groups,\\ automorphisms of\\ Jordan algebras};
\draw (288.81, -87.9) rectangle (293.41,-90.0);
\draw(277.66, -90.10000000000001) node[anchor=north west,align=left] {Associated\\ manifolds\\  of\\ Jordan algebras};
\draw (277.66, -90.10000000000001) rectangle (282.01000000000005,-92.2);
\draw(282.11, -90.10000000000001) node[anchor=north west,align=left] {Idempotents,\\ Peirce\\ decompositions};
\draw (282.11, -90.10000000000001) rectangle (286.21000000000004,-91.7);
\draw(286.31, -90.10000000000001) node[anchor=north west,align=left] {Jordan\\ structures on\\ Banach spaces\\ and algebras};
\draw (286.31, -90.10000000000001) rectangle (290.16,-92.2);
\draw(290.26000000000005, -90.10000000000001) node[anchor=north west,align=left] {Exceptional\\ Jordan\\ structures};
\draw (290.26000000000005, -90.10000000000001) rectangle (293.61000000000007,-91.7);
\draw(277.66, -92.30000000000001) node[anchor=north west,align=left] {Applications\\  of Jordan\\ algebras to\\ physics, etc.};
\draw (277.66, -92.30000000000001) rectangle (281.51000000000005,-94.4);
\draw(281.61, -92.30000000000001) node[anchor=north west,align=left] {Identities\\ and free\\  Jordan\\ structures};
\draw (281.61, -92.30000000000001) rectangle (284.71000000000004,-94.4);
\draw(284.81, -92.30000000000001) node[anchor=north west,align=left] {Structure\\  theory\\ for Jordan\\ algebras};
\draw (284.81, -92.30000000000001) rectangle (287.91,-94.4);
\draw(288.01000000000005, -92.30000000000001) node[anchor=north west,align=left] {Simple,\\ semisimple\\ Jordan\\ algebras};
\draw (288.01000000000005, -92.30000000000001) rectangle (291.11000000000007,-94.4);
\draw(277.66, -94.5) node[anchor=north west,align=left] {Associated\\ geometries\\ of Jordan\\ algebras};
\draw (277.66, -94.5) rectangle (280.76000000000005,-96.6);
\draw(280.86, -94.5) node[anchor=north west,align=left] {Division\\ algebras\\ and Jordan\\ algebras};
\draw (280.86, -94.5) rectangle (283.96000000000004,-96.6);
\draw(284.06, -94.5) node[anchor=north west,align=left] {Super\\ structures};
\draw (284.06, -94.5) rectangle (287.16,-95.6);
\draw(287.26000000000005, -94.5) node[anchor=north west,align=left] {Radicals\\ in Jordan\\ algebras};
\draw (287.26000000000005, -94.5) rectangle (290.11000000000007,-96.1);
\draw(293.81, -86.9) node[anchor=north west,align=left] {\large General nonassociative rings};
\draw (293.81, -86.9) rectangle (304.96,-105.0);
\draw(294.81, -87.9) node[anchor=north west,align=left] {General theory\\  of\\ nonassociative rings\\ and algebras};
\draw (294.81, -87.9) rectangle (300.41,-90.0);
\draw(300.51, -87.9) node[anchor=north west,align=left] {Radical theory\\ (nonassociative\\ rings\\ and algebras)};
\draw (300.51, -87.9) rectangle (304.86,-90.0);
\draw(294.81, -90.10000000000001) node[anchor=north west,align=left] {Automorphisms,\\ derivations,\\ other operators\\ (nonassociative\\ rings and algebras)};
\draw (294.81, -90.10000000000001) rectangle (300.16,-92.7);
\draw(300.26, -90.10000000000001) node[anchor=north west,align=left] {Nonassociative\\  algebras\\ satisfying\\ other identities};
\draw (300.26, -90.10000000000001) rectangle (304.86,-92.2);
\draw(294.81, -92.80000000000001) node[anchor=north west,align=left] {Power-associative\\ rings};
\draw (294.81, -92.80000000000001) rectangle (299.66,-93.9);
\draw(299.76, -92.80000000000001) node[anchor=north west,align=left] {Gröbner-Shirshov\\  bases\\ in nonassociative\\ algebras};
\draw (299.76, -92.80000000000001) rectangle (304.61,-94.9);
\draw(294.81, -94.0) node[anchor=north west,align=left] {Superalgebras};
\draw (294.81, -94.0) rectangle (298.66,-94.6);
\draw(294.81, -95.0) node[anchor=north west,align=left] {Quadratic\\ algebras\\ (but not\\ quadratic\\ Jordan algebras)};
\draw (294.81, -95.0) rectangle (299.41,-97.6);
\draw(299.51, -95.0) node[anchor=north west,align=left] {Noncommutative\\ Jordan\\ algebras};
\draw (299.51, -95.0) rectangle (303.61,-96.6);
\draw(294.81, -97.7) node[anchor=north west,align=left] {Nonassociative\\ division\\ algebras};
\draw (294.81, -97.7) rectangle (298.91,-99.3);
\draw(299.01, -97.7) node[anchor=north west,align=left] {Free\\ nonassociative\\ algebras};
\draw (299.01, -97.7) rectangle (303.11,-99.3);
\draw(294.81, -99.4) node[anchor=north west,align=left] {Structure\\ theory for\\ nonassociative\\ algebras};
\draw (294.81, -99.4) rectangle (298.91,-101.5);
\draw(299.01, -99.4) node[anchor=north west,align=left] {Ternary\\ compositions};
\draw (299.01, -99.4) rectangle (302.61,-100.5);
\draw(294.81, -101.60000000000001) node[anchor=north west,align=left] {Other\\ \(n\)-ary\\ compositions\\ \((n~\ge~3)\)};
\draw (294.81, -101.60000000000001) rectangle (298.41,-103.7);
\draw(298.51, -101.60000000000001) node[anchor=north west,align=left] {Composition\\ algebras};
\draw (298.51, -101.60000000000001) rectangle (301.86,-102.7);
\draw(301.96, -101.60000000000001) node[anchor=north west,align=left] {Flexible\\ algebras};
\draw (301.96, -101.60000000000001) rectangle (304.56,-102.7);
\draw(294.81, -103.80000000000001) node[anchor=north west,align=left] {Leibniz\\ algebras};
\draw (294.81, -103.80000000000001) rectangle (297.41,-104.9);
\draw(297.51, -103.80000000000001) node[anchor=north west,align=left] {Valued\\ algebras};
\draw (297.51, -103.80000000000001) rectangle (300.11,-104.9);
\draw(276.66, -96.80000000000001) node[anchor=north west,align=left] {\large Other nonassociative rings and algebras};
\draw (276.66, -96.80000000000001) rectangle (290.26000000000005,-101.70000000000002);
\draw(277.66, -97.80000000000001) node[anchor=north west,align=left] {\((\gamma,~\delta)\)-rings,\\ including\\ \((1,-1)\)-rings};
\draw (277.66, -97.80000000000001) rectangle (282.51000000000005,-99.4);
\draw(282.61, -97.80000000000001) node[anchor=north west,align=left] {Lie-admissible\\ algebras};
\draw (282.61, -97.80000000000001) rectangle (286.71000000000004,-98.9);
\draw(286.81, -97.80000000000001) node[anchor=north west,align=left] {Alternative\\ rings};
\draw (286.81, -97.80000000000001) rectangle (290.16,-98.9);
\draw(277.66, -99.50000000000001) node[anchor=north west,align=left] {Right\\ alternative\\ rings};
\draw (277.66, -99.50000000000001) rectangle (281.01000000000005,-101.10000000000001);
\draw(281.11, -99.50000000000001) node[anchor=north west,align=left] {(non-Lie)\\  Hom\\ algebras\\ and topics};
\draw (281.11, -99.50000000000001) rectangle (284.21000000000004,-101.60000000000001);
\draw(284.31, -99.50000000000001) node[anchor=north west,align=left] {Mal’tsev\\ rings and\\ algebras};
\draw (284.31, -99.50000000000001) rectangle (287.16,-101.10000000000001);
\draw(287.26000000000005, -99.50000000000001) node[anchor=north west,align=left] {Genetic\\ algebras};
\draw (287.26000000000005, -99.50000000000001) rectangle (289.86000000000007,-100.60000000000001);
\draw(276.66, -101.80000000000001) node[anchor=north west,align=left] {\large Computational methods\\  for problems\\ pertaining to nonassociative\\ rings and algebras};
\draw (276.66, -101.80000000000001) rectangle (285.94,-103.9);
\draw(276.66, -105.10000000000001) node[anchor=north west,align=left] {\large Lie algebras and Lie superalgebras};
\draw (276.66, -105.10000000000001) rectangle (289.76000000000005,-137.2);
\draw(277.66, -106.10000000000001) node[anchor=north west,align=left] {Lie (super)algebras\\  associated\\ with other structures\\ (associative,\\ Jordan, etc.)};
\draw (277.66, -106.10000000000001) rectangle (283.51000000000005,-108.7);
\draw(283.61, -106.10000000000001) node[anchor=north west,align=left] {Kac-Moody\\ (super)algebras;\\ extended affine Lie\\  algebras;\\ toroidal Lie algebras};
\draw (283.61, -106.10000000000001) rectangle (289.46000000000004,-108.7);
\draw(277.66, -108.80000000000001) node[anchor=north west,align=left] {Quantum groups\\  (quantized\\ enveloping algebras)\\ and related\\ deformations};
\draw (277.66, -108.80000000000001) rectangle (283.26000000000005,-111.4);
\draw(283.36, -108.80000000000001) node[anchor=north west,align=left] {Infinite-dimensional\\  Lie\\ (super)algebras};
\draw (283.36, -108.80000000000001) rectangle (288.96000000000004,-110.4);
\draw(277.66, -111.50000000000001) node[anchor=north west,align=left] {Automorphisms,\\ derivations,\\ other operators\\ for Lie algebras\\ and super algebras};
\draw (277.66, -111.50000000000001) rectangle (282.76000000000005,-114.10000000000001);
\draw(282.86, -111.50000000000001) node[anchor=north west,align=left] {Applications\\ of Lie algebras\\  and\\ superalgebras to\\ integrable systems};
\draw (282.86, -111.50000000000001) rectangle (287.96000000000004,-114.10000000000001);
\draw(277.66, -114.20000000000002) node[anchor=north west,align=left] {Applications\\ of Lie\\ (super)algebras to\\ physics, etc.};
\draw (277.66, -114.20000000000002) rectangle (282.76000000000005,-116.30000000000001);
\draw(282.86, -114.20000000000002) node[anchor=north west,align=left] {Vertex operators;\\  vertex\\ operator algebras\\ and related\\ structures};
\draw (282.86, -114.20000000000002) rectangle (287.71000000000004,-116.80000000000001);
\draw(277.66, -116.9) node[anchor=north west,align=left] {Representations\\  of Lie\\ algebras and Lie\\ superalgebras,\\  algebraic\\ theory (weights)};
\draw (277.66, -116.9) rectangle (282.26000000000005,-120.0);
\draw(282.36, -116.9) node[anchor=north west,align=left] {Lie algebras\\  of\\ linear\\ algebraic groups};
\draw (282.36, -116.9) rectangle (286.96000000000004,-119.0);
\draw(287.06, -116.9) node[anchor=north west,align=left] {Poisson\\ algebras};
\draw (287.06, -116.9) rectangle (289.66,-118.0);
\draw(277.66, -120.10000000000001) node[anchor=north west,align=left] {Lie algebras\\ of vector\\ fields and\\  related\\ (super) algebras};
\draw (277.66, -120.10000000000001) rectangle (282.26000000000005,-122.7);
\draw(282.36, -120.10000000000001) node[anchor=north west,align=left] {Identities,\\  free\\ Lie\\ (super)algebras};
\draw (282.36, -120.10000000000001) rectangle (286.71000000000004,-122.2);
\draw(286.81, -120.10000000000001) node[anchor=north west,align=left] {Coadjoint\\  orbits;\\ nilpotent\\ varieties};
\draw (286.81, -120.10000000000001) rectangle (289.66,-122.2);
\draw(277.66, -122.80000000000001) node[anchor=north west,align=left] {Representations\\ of Lie algebras\\  and Lie\\ superalgebras,\\ analytic theory};
\draw (277.66, -122.80000000000001) rectangle (282.01000000000005,-125.4);
\draw(282.11, -122.80000000000001) node[anchor=north west,align=left] {Simple,\\ semisimple,\\ reductive\\ (super)algebras};
\draw (282.11, -122.80000000000001) rectangle (286.46000000000004,-124.9);
\draw(286.56, -122.80000000000001) node[anchor=north west,align=left] {Root\\ systems};
\draw (286.56, -122.80000000000001) rectangle (288.91,-123.9);
\draw(277.66, -125.5) node[anchor=north west,align=left] {Exceptional\\ (super)algebras};
\draw (277.66, -125.5) rectangle (282.01000000000005,-126.6);
\draw(282.11, -125.5) node[anchor=north west,align=left] {Solvable,\\ nilpotent\\ (super)algebras};
\draw (282.11, -125.5) rectangle (286.46000000000004,-127.1);
\draw(277.66, -127.2) node[anchor=north west,align=left] {Universal\\ enveloping\\ (super)algebras};
\draw (277.66, -127.2) rectangle (282.01000000000005,-128.8);
\draw(282.11, -127.2) node[anchor=north west,align=left] {Yang-Baxter\\  equations\\ and Rota-Baxter\\ operators};
\draw (282.11, -127.2) rectangle (286.46000000000004,-129.3);
\draw(277.66, -129.4) node[anchor=north west,align=left] {Modular\\ Lie\\ (super)algebras};
\draw (277.66, -129.4) rectangle (282.01000000000005,-131.0);
\draw(282.11, -129.4) node[anchor=north west,align=left] {Homological\\  methods\\ in Lie\\ (super)algebras};
\draw (282.11, -129.4) rectangle (286.46000000000004,-131.5);
\draw(277.66, -131.6) node[anchor=north west,align=left] {Cohomology\\ of Lie\\ (super)algebras};
\draw (277.66, -131.6) rectangle (282.01000000000005,-133.2);
\draw(282.11, -131.6) node[anchor=north west,align=left] {Lie bialgebras;\\  Lie\\ coalgebras};
\draw (282.11, -131.6) rectangle (286.46000000000004,-133.2);
\draw(277.66, -133.3) node[anchor=north west,align=left] {Graded\\ Lie\\ (super)algebras};
\draw (277.66, -133.3) rectangle (282.01000000000005,-134.9);
\draw(282.11, -133.3) node[anchor=north west,align=left] {Color Lie\\ (super)algebras};
\draw (282.11, -133.3) rectangle (286.46000000000004,-134.4);
\draw(277.66, -135.0) node[anchor=north west,align=left] {Structure\\ theory for Lie\\ algebras and\\ superalgebras};
\draw (277.66, -135.0) rectangle (281.76000000000005,-137.1);
\draw(281.86, -135.0) node[anchor=north west,align=left] {Hom-Lie\\ and related\\ algebras};
\draw (281.86, -135.0) rectangle (285.21000000000004,-136.6);
\draw(285.31, -135.0) node[anchor=north west,align=left] {Virasoro\\ and related\\ algebras};
\draw (285.31, -135.0) rectangle (288.66,-136.6);
\draw(289.86, -105.10000000000001) node[anchor=north west,align=left] {\large History of\\ nonassociative\\ rings and algebras};
\draw (289.86, -105.10000000000001) rectangle (296.04,-106.7);
\draw(316.07000000000005, -1) node[anchor=north west,align=left] {\LARGE Mathematical logic and foundations};
\draw (316.07000000000005, -1) rectangle (345.47,-79.5);
\draw(317.07000000000005, -2) node[anchor=north west,align=left] {\large Proof theory and constructive mathematics};
\draw (317.07000000000005, -2) rectangle (332.6700000000001,-15.0);
\draw(318.07000000000005, -3) node[anchor=north west,align=left] {Provability\\ logics and related\\  algebras\\ (e.g., diagonalizable\\ algebras)};
\draw (318.07000000000005, -3) rectangle (323.9200000000001,-5.6);
\draw(324.02000000000004, -3) node[anchor=north west,align=left] {Proof-theoretic\\  aspects of\\ linear logic and\\  other\\ substructural logics};
\draw (324.02000000000004, -3) rectangle (329.62000000000006,-5.6);
\draw(329.72, -3) node[anchor=north west,align=left] {Structure\\ of\\ proofs};
\draw (329.72, -3) rectangle (332.57000000000005,-4.6);
\draw(318.07000000000005, -5.7) node[anchor=north west,align=left] {Proof theory,\\  general\\ (including\\ proof-theoretic\\ semantics)};
\draw (318.07000000000005, -5.7) rectangle (322.4200000000001,-8.3);
\draw(322.52000000000004, -5.7) node[anchor=north west,align=left] {Cut-elimination\\  and\\ normal-form\\ theorems};
\draw (322.52000000000004, -5.7) rectangle (326.87000000000006,-7.800000000000001);
\draw(326.97, -5.7) node[anchor=north west,align=left] {Relative\\ consistency\\  and\\ interpretations};
\draw (326.97, -5.7) rectangle (331.32000000000005,-7.800000000000001);
\draw(318.07000000000005, -8.4) node[anchor=north west,align=left] {Metamathematics\\ of\\ constructive\\ systems};
\draw (318.07000000000005, -8.4) rectangle (322.4200000000001,-10.5);
\draw(322.52000000000004, -8.4) node[anchor=north west,align=left] {Gödel\\ numberings and\\ issues of\\ incompleteness};
\draw (322.52000000000004, -8.4) rectangle (326.62000000000006,-10.5);
\draw(326.72, -8.4) node[anchor=north west,align=left] {Intuitionistic\\ mathematics};
\draw (326.72, -8.4) rectangle (330.82000000000005,-9.5);
\draw(318.07000000000005, -10.600000000000001) node[anchor=north west,align=left] {Second- and\\ higher-order\\  arithmetic\\ and fragments};
\draw (318.07000000000005, -10.600000000000001) rectangle (321.9200000000001,-12.700000000000001);
\draw(322.02000000000004, -10.600000000000001) node[anchor=north west,align=left] {Constructive\\ and\\ recursive\\ analysis};
\draw (322.02000000000004, -10.600000000000001) rectangle (325.62000000000006,-12.700000000000001);
\draw(325.72, -10.600000000000001) node[anchor=north west,align=left] {Other\\ constructive\\ mathematics};
\draw (325.72, -10.600000000000001) rectangle (329.32000000000005,-12.200000000000001);
\draw(329.4200000000001, -10.600000000000001) node[anchor=north west,align=left] {Complexity\\ of proofs};
\draw (329.4200000000001, -10.600000000000001) rectangle (332.5200000000001,-11.700000000000001);
\draw(318.07000000000005, -12.8) node[anchor=north west,align=left] {Functionals\\ in proof\\ theory};
\draw (318.07000000000005, -12.8) rectangle (321.4200000000001,-14.4);
\draw(321.52000000000004, -12.8) node[anchor=north west,align=left] {Recursive\\ ordinals\\ and ordinal\\ notations};
\draw (321.52000000000004, -12.8) rectangle (324.87000000000006,-14.9);
\draw(324.97, -12.8) node[anchor=north west,align=left] {First-order\\ arithmetic\\  and\\ fragments};
\draw (324.97, -12.8) rectangle (328.32000000000005,-14.9);
\draw(332.77000000000004, -2) node[anchor=north west,align=left] {\large Computability and recursion theory};
\draw (332.77000000000004, -2) rectangle (345.37000000000006,-26.3);
\draw(333.77000000000004, -3) node[anchor=north west,align=left] {Computability\\ and recursion\\  theory on\\ ordinals,\\ admissible sets, etc.};
\draw (333.77000000000004, -3) rectangle (339.62000000000006,-5.6);
\draw(339.72, -3) node[anchor=north west,align=left] {Complexity of\\ computation\\ (including implicit\\ computational\\ complexity)};
\draw (339.72, -3) rectangle (345.07000000000005,-5.6);
\draw(333.77000000000004, -5.7) node[anchor=north west,align=left] {Computation\\  over the\\ reals,\\ computable analysis};
\draw (333.77000000000004, -5.7) rectangle (339.12000000000006,-7.800000000000001);
\draw(339.22, -5.7) node[anchor=north west,align=left] {Other degrees\\ and reducibilities\\  in\\ computability and\\ recursion theory};
\draw (339.22, -5.7) rectangle (344.32000000000005,-8.3);
\draw(333.77000000000004, -8.4) node[anchor=north west,align=left] {Recursive\\ equivalence\\ types of\\ sets and\\ structures, isols};
\draw (333.77000000000004, -8.4) rectangle (338.62000000000006,-11.0);
\draw(338.72, -8.4) node[anchor=north west,align=left] {Word problems,\\ etc. in\\ computability\\  and\\ recursion theory};
\draw (338.72, -8.4) rectangle (343.32000000000005,-11.0);
\draw(333.77000000000004, -11.100000000000001) node[anchor=north west,align=left] {Hierarchies\\ of computability\\  and\\ definability};
\draw (333.77000000000004, -11.100000000000001) rectangle (338.37000000000006,-13.200000000000001);
\draw(338.47, -11.100000000000001) node[anchor=north west,align=left] {Abstract and\\ axiomatic\\ computability\\  and\\ recursion theory};
\draw (338.47, -11.100000000000001) rectangle (343.07000000000005,-13.700000000000001);
\draw(333.77000000000004, -13.8) node[anchor=north west,align=left] {Applications\\ of computability\\  and\\ recursion theory};
\draw (333.77000000000004, -13.8) rectangle (338.37000000000006,-15.9);
\draw(338.47, -13.8) node[anchor=north west,align=left] {Automata and\\ formal grammars\\ in connection\\ with logical\\ questions};
\draw (338.47, -13.8) rectangle (342.82000000000005,-16.400000000000002);
\draw(333.77000000000004, -16.5) node[anchor=north west,align=left] {Recursively\\ (computably)\\ enumerable sets\\ and degrees};
\draw (333.77000000000004, -16.5) rectangle (338.12000000000006,-18.6);
\draw(338.22, -16.5) node[anchor=north west,align=left] {Undecidability\\  and\\ degrees of sets\\ of sentences};
\draw (338.22, -16.5) rectangle (342.57000000000005,-18.6);
\draw(333.77000000000004, -18.7) node[anchor=north west,align=left] {Thue and\\ Post\\ systems, etc.};
\draw (333.77000000000004, -18.7) rectangle (337.62000000000006,-20.3);
\draw(337.72, -18.7) node[anchor=north west,align=left] {Recursive\\ functions and\\ relations,\\ subrecursive\\ hierarchies};
\draw (337.72, -18.7) rectangle (341.57000000000005,-21.3);
\draw(341.67, -18.7) node[anchor=north west,align=left] {Other Turing\\  degree\\ structures};
\draw (341.67, -18.7) rectangle (345.27000000000004,-20.3);
\draw(333.77000000000004, -21.4) node[anchor=north west,align=left] {Theory of\\ numerations,\\ effectively\\ presented\\ structures};
\draw (333.77000000000004, -21.4) rectangle (337.37000000000006,-24.0);
\draw(337.47, -21.4) node[anchor=north west,align=left] {Inductive\\ definability};
\draw (337.47, -21.4) rectangle (341.07000000000005,-22.5);
\draw(341.17, -21.4) node[anchor=north west,align=left] {Turing\\ machines\\ and related\\ notions};
\draw (341.17, -21.4) rectangle (344.52000000000004,-23.5);
\draw(333.77000000000004, -24.099999999999998) node[anchor=north west,align=left] {Algorithmic\\ randomness\\  and\\ dimension};
\draw (333.77000000000004, -24.099999999999998) rectangle (337.12000000000006,-26.2);
\draw(337.22, -24.099999999999998) node[anchor=north west,align=left] {Higher-type\\ and set\\ recursion\\ theory};
\draw (337.22, -24.099999999999998) rectangle (340.57000000000005,-26.2);
\draw(317.07000000000005, -15.1) node[anchor=north west,align=left] {\large Philosophical aspects of logic and foundations};
\draw (317.07000000000005, -15.1) rectangle (331.93000000000006,-18.8);
\draw(318.07000000000005, -16.1) node[anchor=north west,align=left] {Philosophical\\  and\\ critical aspects\\  of logic\\ and foundations};
\draw (318.07000000000005, -16.1) rectangle (322.6700000000001,-18.700000000000003);
\draw(322.77000000000004, -16.1) node[anchor=north west,align=left] {Logic in\\  the\\ philosophy\\ of science};
\draw (322.77000000000004, -16.1) rectangle (325.87000000000006,-18.200000000000003);
\draw(317.07000000000005, -18.900000000000002) node[anchor=north west,align=left] {\large Computational methods\\  for problems\\ pertaining to mathematical\\ logic and foundations};
\draw (317.07000000000005, -18.900000000000002) rectangle (325.7300000000001,-21.000000000000004);
\draw(317.07000000000005, -21.1) node[anchor=north west,align=left] {\large History of\\ mathematical logic\\ and foundations};
\draw (317.07000000000005, -21.1) rectangle (323.25000000000006,-22.700000000000003);
\draw(317.07000000000005, -26.400000000000002) node[anchor=north west,align=left] {\large General logic};
\draw (317.07000000000005, -26.400000000000002) rectangle (329.22,-46.5);
\draw(318.07000000000005, -27.400000000000002) node[anchor=north west,align=left] {Substructural\\ logics (including\\ relevance, entailment,\\ linear logic,\\ Lambek calculus,\\ BCK and BCI logics)};
\draw (318.07000000000005, -27.400000000000002) rectangle (324.1700000000001,-30.500000000000004);
\draw(324.27000000000004, -27.400000000000002) node[anchor=north west,align=left] {Probability\\  and\\ inductive logic};
\draw (324.27000000000004, -27.400000000000002) rectangle (328.62000000000006,-29.000000000000004);
\draw(324.27000000000004, -29.1) node[anchor=north west,align=left] {Paraconsistent\\ logics};
\draw (324.27000000000004, -29.1) rectangle (328.37000000000006,-30.200000000000003);
\draw(318.07000000000005, -30.6) node[anchor=north west,align=left] {Subsystems\\ of classical\\ logic (including\\ intuitionistic logic)};
\draw (318.07000000000005, -30.6) rectangle (323.9200000000001,-32.7);
\draw(324.02000000000004, -30.6) node[anchor=north west,align=left] {Foundations\\ of classical\\ theories\\ (including reverse\\ mathematics)};
\draw (324.02000000000004, -30.6) rectangle (329.12000000000006,-33.2);
\draw(318.07000000000005, -33.300000000000004) node[anchor=north west,align=left] {Logics of\\ knowledge and\\  belief\\ (including\\ belief change)};
\draw (318.07000000000005, -33.300000000000004) rectangle (322.1700000000001,-35.900000000000006);
\draw(322.27000000000004, -33.300000000000004) node[anchor=north west,align=left] {Classical\\ propositional\\ logic};
\draw (322.27000000000004, -33.300000000000004) rectangle (326.12000000000006,-34.900000000000006);
\draw(326.22, -33.300000000000004) node[anchor=north west,align=left] {Abstract\\ deductive\\ systems};
\draw (326.22, -33.300000000000004) rectangle (329.07000000000005,-34.900000000000006);
\draw(318.07000000000005, -36.0) node[anchor=north west,align=left] {Mechanization\\ of proofs\\ and logical\\ operations};
\draw (318.07000000000005, -36.0) rectangle (321.9200000000001,-38.1);
\draw(322.02000000000004, -36.0) node[anchor=north west,align=left] {Higher-order\\ logic};
\draw (322.02000000000004, -36.0) rectangle (325.62000000000006,-37.1);
\draw(325.72, -36.0) node[anchor=north west,align=left] {Classical\\ first-order\\ logic};
\draw (325.72, -36.0) rectangle (329.07000000000005,-37.6);
\draw(318.07000000000005, -38.2) node[anchor=north west,align=left] {Decidability\\ of theories\\ and sets\\ of sentences};
\draw (318.07000000000005, -38.2) rectangle (321.6700000000001,-40.300000000000004);
\draw(321.77000000000004, -38.2) node[anchor=north west,align=left] {Fuzzy logic;\\  logic\\ of vagueness};
\draw (321.77000000000004, -38.2) rectangle (325.37000000000006,-39.800000000000004);
\draw(325.47, -38.2) node[anchor=north west,align=left] {Intermediate\\ logics};
\draw (325.47, -38.2) rectangle (329.07000000000005,-39.300000000000004);
\draw(318.07000000000005, -40.400000000000006) node[anchor=north west,align=left] {Other\\ nonclassical\\ logic};
\draw (318.07000000000005, -40.400000000000006) rectangle (321.6700000000001,-42.00000000000001);
\draw(321.77000000000004, -40.400000000000006) node[anchor=north west,align=left] {Other\\ applications\\ of logic};
\draw (321.77000000000004, -40.400000000000006) rectangle (325.37000000000006,-42.00000000000001);
\draw(325.47, -40.400000000000006) node[anchor=north west,align=left] {Combinatory\\  logic\\ and lambda\\ calculus};
\draw (325.47, -40.400000000000006) rectangle (328.82000000000005,-42.50000000000001);
\draw(318.07000000000005, -42.6) node[anchor=north west,align=left] {Modal logic\\ (including\\ the logic\\ of norms)};
\draw (318.07000000000005, -42.6) rectangle (321.4200000000001,-44.7);
\draw(321.52000000000004, -42.6) node[anchor=north west,align=left] {Many-valued\\ logic};
\draw (321.52000000000004, -42.6) rectangle (324.87000000000006,-43.7);
\draw(324.97, -42.6) node[anchor=north west,align=left] {Logic of\\ natural\\ languages};
\draw (324.97, -42.6) rectangle (327.82000000000005,-44.2);
\draw(318.07000000000005, -44.8) node[anchor=north west,align=left] {Temporal\\ logic};
\draw (318.07000000000005, -44.8) rectangle (320.6700000000001,-45.9);
\draw(320.77000000000004, -44.8) node[anchor=north west,align=left] {Combined\\ logics};
\draw (320.77000000000004, -44.8) rectangle (323.37000000000006,-45.9);
\draw(323.47, -44.8) node[anchor=north west,align=left] {Logic in\\ computer\\ science};
\draw (323.47, -44.8) rectangle (326.07000000000005,-46.4);
\draw(326.1700000000001, -44.8) node[anchor=north west,align=left] {Type\\ theory};
\draw (326.1700000000001, -44.8) rectangle (328.2700000000001,-45.9);
\draw(329.32000000000005, -26.400000000000002) node[anchor=north west,align=left] {\large Set theory};
\draw (329.32000000000005, -26.400000000000002) rectangle (341.1700000000001,-46.0);
\draw(330.32000000000005, -27.400000000000002) node[anchor=north west,align=left] {Other classical\\  set theory\\ (including functions,\\  relations,\\ and set algebra)};
\draw (330.32000000000005, -27.400000000000002) rectangle (336.1700000000001,-30.000000000000004);
\draw(336.27000000000004, -27.400000000000002) node[anchor=north west,align=left] {Cardinal\\ characteristics\\ of the\\ continuum};
\draw (336.27000000000004, -27.400000000000002) rectangle (340.62000000000006,-29.500000000000004);
\draw(330.32000000000005, -30.1) node[anchor=north west,align=left] {Other aspects\\ of forcing\\  and\\ Boolean-valued models};
\draw (330.32000000000005, -30.1) rectangle (336.1700000000001,-32.2);
\draw(336.27000000000004, -30.1) node[anchor=north west,align=left] {Continuum\\ hypothesis\\  and\\ Martin’s axiom};
\draw (336.27000000000004, -30.1) rectangle (340.37000000000006,-32.2);
\draw(330.32000000000005, -32.300000000000004) node[anchor=north west,align=left] {Inner models,\\ including\\ constructibility,\\ ordinal definability,\\ and core models};
\draw (330.32000000000005, -32.300000000000004) rectangle (336.1700000000001,-34.900000000000006);
\draw(336.27000000000004, -32.300000000000004) node[anchor=north west,align=left] {Generic\\ absoluteness\\  and\\ forcing axioms};
\draw (336.27000000000004, -32.300000000000004) rectangle (340.37000000000006,-34.400000000000006);
\draw(330.32000000000005, -35.0) node[anchor=north west,align=left] {Ordered sets\\ and their\\ cofinalities;\\ pcf theory};
\draw (330.32000000000005, -35.0) rectangle (334.1700000000001,-37.1);
\draw(334.27000000000004, -35.0) node[anchor=north west,align=left] {Other\\ combinatorial\\ set theory};
\draw (334.27000000000004, -35.0) rectangle (338.12000000000006,-36.6);
\draw(338.22, -35.0) node[anchor=north west,align=left] {Partition\\ relations};
\draw (338.22, -35.0) rectangle (341.07000000000005,-36.1);
\draw(330.32000000000005, -37.2) node[anchor=north west,align=left] {Axiomatics of\\ classical set\\  theory and\\ its fragments};
\draw (330.32000000000005, -37.2) rectangle (334.1700000000001,-39.300000000000004);
\draw(334.27000000000004, -37.2) node[anchor=north west,align=left] {Other notions\\  of\\ set-theoretic\\ definability};
\draw (334.27000000000004, -37.2) rectangle (338.12000000000006,-39.300000000000004);
\draw(338.22, -37.2) node[anchor=north west,align=left] {Large\\ cardinals};
\draw (338.22, -37.2) rectangle (341.07000000000005,-38.300000000000004);
\draw(330.32000000000005, -39.400000000000006) node[anchor=north west,align=left] {Other\\ set-theoretic\\ hypotheses\\ and axioms};
\draw (330.32000000000005, -39.400000000000006) rectangle (334.1700000000001,-41.50000000000001);
\draw(334.27000000000004, -39.400000000000006) node[anchor=north west,align=left] {Ordinal\\ and cardinal\\ numbers};
\draw (334.27000000000004, -39.400000000000006) rectangle (337.87000000000006,-41.00000000000001);
\draw(337.97, -39.400000000000006) node[anchor=north west,align=left] {Theory\\ of fuzzy\\ sets, etc.};
\draw (337.97, -39.400000000000006) rectangle (341.07000000000005,-41.00000000000001);
\draw(330.32000000000005, -41.6) node[anchor=north west,align=left] {Axiom of\\ choice and\\ related\\ propositions};
\draw (330.32000000000005, -41.6) rectangle (333.9200000000001,-43.7);
\draw(334.02000000000004, -41.6) node[anchor=north west,align=left] {Consistency\\  and\\ independence\\ results};
\draw (334.02000000000004, -41.6) rectangle (337.62000000000006,-43.7);
\draw(337.72, -41.6) node[anchor=north west,align=left] {Descriptive\\ set\\ theory};
\draw (337.72, -41.6) rectangle (341.07000000000005,-43.2);
\draw(330.32000000000005, -43.8) node[anchor=north west,align=left] {Nonclassical\\  and\\ second-order\\ set theories};
\draw (330.32000000000005, -43.8) rectangle (333.9200000000001,-45.9);
\draw(334.02000000000004, -43.8) node[anchor=north west,align=left] {Applications\\  of\\ set theory};
\draw (334.02000000000004, -43.8) rectangle (337.62000000000006,-45.4);
\draw(337.72, -43.8) node[anchor=north west,align=left] {Determinacy\\ principles};
\draw (337.72, -43.8) rectangle (341.07000000000005,-44.9);
\draw(317.07000000000005, -46.60000000000001) node[anchor=north west,align=left] {\large Model theory};
\draw (317.07000000000005, -46.60000000000001) rectangle (327.22,-79.4);
\draw(318.07000000000005, -47.60000000000001) node[anchor=north west,align=left] {Equational\\ classes, universal\\  algebra\\ in model theory};
\draw (318.07000000000005, -47.60000000000001) rectangle (323.1700000000001,-49.70000000000001);
\draw(323.27000000000004, -47.60000000000001) node[anchor=north west,align=left] {Basic\\ properties of\\ first-order\\ languages and\\ structures};
\draw (323.27000000000004, -47.60000000000001) rectangle (327.12000000000006,-50.20000000000001);
\draw(318.07000000000005, -50.30000000000001) node[anchor=north west,align=left] {Computable\\ structure theory,\\ computable\\ model theory};
\draw (318.07000000000005, -50.30000000000001) rectangle (322.9200000000001,-52.40000000000001);
\draw(323.02000000000004, -50.30000000000001) node[anchor=north west,align=left] {Model theory\\ of denumerable\\ and separable\\ structures};
\draw (323.02000000000004, -50.30000000000001) rectangle (327.12000000000006,-52.40000000000001);
\draw(318.07000000000005, -52.50000000000001) node[anchor=north west,align=left] {Continuous\\ model theory,\\  model\\ theory of\\ metric structures};
\draw (318.07000000000005, -52.50000000000001) rectangle (322.9200000000001,-55.10000000000001);
\draw(323.02000000000004, -52.50000000000001) node[anchor=north west,align=left] {Interpolation,\\ preservation,\\ definability};
\draw (323.02000000000004, -52.50000000000001) rectangle (327.12000000000006,-54.10000000000001);
\draw(318.07000000000005, -55.20000000000001) node[anchor=north west,align=left] {Quantifier\\ elimination,\\ model\\ completeness and\\ related topics};
\draw (318.07000000000005, -55.20000000000001) rectangle (322.6700000000001,-57.80000000000001);
\draw(322.77000000000004, -55.20000000000001) node[anchor=north west,align=left] {Model-theoretic\\ forcing};
\draw (322.77000000000004, -55.20000000000001) rectangle (327.12000000000006,-56.30000000000001);
\draw(322.77000000000004, -56.400000000000006) node[anchor=north west,align=left] {Model-theoretic\\ algebra};
\draw (322.77000000000004, -56.400000000000006) rectangle (327.12000000000006,-57.50000000000001);
\draw(318.07000000000005, -57.900000000000006) node[anchor=north west,align=left] {Classification\\  theory,\\ stability and\\ related concepts\\ in model theory};
\draw (318.07000000000005, -57.900000000000006) rectangle (322.6700000000001,-60.50000000000001);
\draw(322.77000000000004, -57.900000000000006) node[anchor=north west,align=left] {Abstract\\ elementary\\ classes and\\ related topics};
\draw (322.77000000000004, -57.900000000000006) rectangle (326.87000000000006,-60.00000000000001);
\draw(318.07000000000005, -60.60000000000001) node[anchor=north west,align=left] {Nonclassical\\  models\\ (Boolean-valued,\\ sheaf, etc.)};
\draw (318.07000000000005, -60.60000000000001) rectangle (322.6700000000001,-62.70000000000001);
\draw(322.77000000000004, -60.60000000000001) node[anchor=north west,align=left] {Ultraproducts\\  and\\ related\\ constructions};
\draw (322.77000000000004, -60.60000000000001) rectangle (326.62000000000006,-62.70000000000001);
\draw(318.07000000000005, -62.80000000000001) node[anchor=north west,align=left] {Other\\ model\\ constructions};
\draw (318.07000000000005, -62.80000000000001) rectangle (321.9200000000001,-64.4);
\draw(322.02000000000004, -62.80000000000001) node[anchor=north west,align=left] {Set-theoretic\\ model theory};
\draw (322.02000000000004, -62.80000000000001) rectangle (325.87000000000006,-63.90000000000001);
\draw(318.07000000000005, -64.5) node[anchor=north west,align=left] {Models\\ of arithmetic\\ and\\ set theory};
\draw (318.07000000000005, -64.5) rectangle (321.9200000000001,-66.6);
\draw(322.02000000000004, -64.5) node[anchor=north west,align=left] {Logic with\\  extra\\ quantifiers\\ and operators};
\draw (322.02000000000004, -64.5) rectangle (325.87000000000006,-66.6);
\draw(318.07000000000005, -66.7) node[anchor=north west,align=left] {Categoricity\\  and\\ completeness\\ of theories};
\draw (318.07000000000005, -66.7) rectangle (321.6700000000001,-68.8);
\draw(321.77000000000004, -66.7) node[anchor=north west,align=left] {Models with\\  special\\ properties\\ (saturated,\\ rigid, etc.)};
\draw (321.77000000000004, -66.7) rectangle (325.37000000000006,-69.3);
\draw(318.07000000000005, -69.4) node[anchor=north west,align=left] {Model theory\\  of ordered\\ structures;\\ o-minimality};
\draw (318.07000000000005, -69.4) rectangle (321.6700000000001,-71.5);
\draw(321.77000000000004, -69.4) node[anchor=north west,align=left] {Models of\\  other\\ mathematical\\ theories};
\draw (321.77000000000004, -69.4) rectangle (325.37000000000006,-71.5);
\draw(318.07000000000005, -71.60000000000001) node[anchor=north west,align=left] {Other\\ classical\\ first-order\\ model theory};
\draw (318.07000000000005, -71.60000000000001) rectangle (321.6700000000001,-73.7);
\draw(321.77000000000004, -71.60000000000001) node[anchor=north west,align=left] {Second-\\ and\\ higher-order\\ model theory};
\draw (321.77000000000004, -71.60000000000001) rectangle (325.37000000000006,-73.7);
\draw(318.07000000000005, -73.80000000000001) node[anchor=north west,align=left] {Applications\\  of\\ model theory};
\draw (318.07000000000005, -73.80000000000001) rectangle (321.6700000000001,-75.4);
\draw(321.77000000000004, -73.80000000000001) node[anchor=north west,align=left] {Model\\ theory of\\ finite\\ structures};
\draw (321.77000000000004, -73.80000000000001) rectangle (324.87000000000006,-75.9);
\draw(318.07000000000005, -76.0) node[anchor=north west,align=left] {Properties\\ of classes\\ of models};
\draw (318.07000000000005, -76.0) rectangle (321.1700000000001,-77.6);
\draw(321.27000000000004, -76.0) node[anchor=north west,align=left] {Logic on\\ admissible\\ sets};
\draw (321.27000000000004, -76.0) rectangle (324.37000000000006,-77.6);
\draw(324.47, -76.0) node[anchor=north west,align=left] {Abstract\\ model\\ theory};
\draw (324.47, -76.0) rectangle (327.07000000000005,-77.6);
\draw(318.07000000000005, -77.7) node[anchor=north west,align=left] {Other\\ infinitary\\ logic};
\draw (318.07000000000005, -77.7) rectangle (321.1700000000001,-79.3);
\draw(327.32000000000005, -46.60000000000001) node[anchor=north west,align=left] {\large Algebraic logic};
\draw (327.32000000000005, -46.60000000000001) rectangle (336.72,-56.400000000000006);
\draw(328.32000000000005, -47.60000000000001) node[anchor=north west,align=left] {Cylindric and\\  polyadic\\ algebras;\\ relation algebras};
\draw (328.32000000000005, -47.60000000000001) rectangle (333.1700000000001,-49.70000000000001);
\draw(333.27000000000004, -47.60000000000001) node[anchor=north west,align=left] {Categorical\\ logic,\\ topoi};
\draw (333.27000000000004, -47.60000000000001) rectangle (336.62000000000006,-49.20000000000001);
\draw(328.32000000000005, -49.80000000000001) node[anchor=north west,align=left] {Other\\ algebras related\\ to logic};
\draw (328.32000000000005, -49.80000000000001) rectangle (332.9200000000001,-51.40000000000001);
\draw(333.02000000000004, -49.80000000000001) node[anchor=north west,align=left] {Logical\\ aspects\\ of Boolean\\ algebras};
\draw (333.02000000000004, -49.80000000000001) rectangle (336.12000000000006,-51.90000000000001);
\draw(328.32000000000005, -52.00000000000001) node[anchor=north west,align=left] {Logical aspects\\ of lattices\\ and related\\ structures};
\draw (328.32000000000005, -52.00000000000001) rectangle (332.6700000000001,-54.10000000000001);
\draw(332.77000000000004, -52.00000000000001) node[anchor=north west,align=left] {Abstract\\ algebraic\\ logic};
\draw (332.77000000000004, -52.00000000000001) rectangle (335.62000000000006,-53.60000000000001);
\draw(328.32000000000005, -54.20000000000001) node[anchor=north west,align=left] {Logical aspects\\  of\\ Łukasiewicz and\\ Post algebras};
\draw (328.32000000000005, -54.20000000000001) rectangle (332.6700000000001,-56.30000000000001);
\draw(332.77000000000004, -54.20000000000001) node[anchor=north west,align=left] {Quantum\\ logic};
\draw (332.77000000000004, -54.20000000000001) rectangle (335.12000000000006,-55.30000000000001);
\draw(327.32000000000005, -56.50000000000001) node[anchor=north west,align=left] {\large Nonstandard models};
\draw (327.32000000000005, -56.50000000000001) rectangle (335.72,-62.400000000000006);
\draw(328.32000000000005, -57.50000000000001) node[anchor=north west,align=left] {Other applications\\  of\\ nonstandard models\\ (economics,\\ physics, etc.)};
\draw (328.32000000000005, -57.50000000000001) rectangle (333.4200000000001,-60.10000000000001);
\draw(328.32000000000005, -60.20000000000001) node[anchor=north west,align=left] {Nonstandard\\  models\\ of arithmetic};
\draw (328.32000000000005, -60.20000000000001) rectangle (332.1700000000001,-61.80000000000001);
\draw(332.27000000000004, -60.20000000000001) node[anchor=north west,align=left] {Nonstandard\\ models\\  in\\ mathematics};
\draw (332.27000000000004, -60.20000000000001) rectangle (335.62000000000006,-62.30000000000001);
\draw(316.07000000000005, -79.6) node[anchor=north west,align=left] {\LARGE Real functions};
\draw (316.07000000000005, -79.6) rectangle (344.40000000000003,-116.19999999999999);
\draw(317.07000000000005, -80.6) node[anchor=north west,align=left] {\large Polynomials, rational functions in real analysis};
\draw (317.07000000000005, -80.6) rectangle (332.55000000000007,-83.8);
\draw(318.07000000000005, -81.6) node[anchor=north west,align=left] {Real\\ polynomials:\\ analytic\\ properties, etc.};
\draw (318.07000000000005, -81.6) rectangle (322.6700000000001,-83.69999999999999);
\draw(322.77000000000004, -81.6) node[anchor=north west,align=left] {Real\\ polynomials:\\ location\\ of zeros};
\draw (322.77000000000004, -81.6) rectangle (326.37000000000006,-83.69999999999999);
\draw(326.47, -81.6) node[anchor=north west,align=left] {Real\\ rational\\ functions};
\draw (326.47, -81.6) rectangle (329.32000000000005,-83.19999999999999);
\draw(332.65000000000003, -80.6) node[anchor=north west,align=left] {\large Functions of several variables};
\draw (332.65000000000003, -80.6) rectangle (344.3,-95.1);
\draw(333.65000000000003, -81.6) node[anchor=north west,align=left] {Implicit function\\  theorems,\\ Jacobians,\\ transformations with\\ several variables};
\draw (333.65000000000003, -81.6) rectangle (339.25000000000006,-84.19999999999999);
\draw(339.35, -81.6) node[anchor=north west,align=left] {Integral formulas\\  of real\\ functions of\\ several variables\\ (Stokes, Gauss,\\ Green, etc.)};
\draw (339.35, -81.6) rectangle (344.20000000000005,-84.69999999999999);
\draw(333.65000000000003, -84.8) node[anchor=north west,align=left] {Absolutely\\ continuous real\\ functions of\\ several variables,\\  functions\\ of bounded variation};
\draw (333.65000000000003, -84.8) rectangle (339.25000000000006,-87.89999999999999);
\draw(339.35, -84.8) node[anchor=north west,align=left] {Continuity\\  and\\ differentiation\\ questions};
\draw (339.35, -84.8) rectangle (343.70000000000005,-86.89999999999999);
\draw(333.65000000000003, -88.0) node[anchor=north west,align=left] {Integration of\\ real functions\\  of several\\ variables: length,\\ area, volume};
\draw (333.65000000000003, -88.0) rectangle (338.75000000000006,-90.6);
\draw(338.85, -88.0) node[anchor=north west,align=left] {Special properties\\ of functions\\  of several\\ variables, Hölder\\ conditions, etc.};
\draw (338.85, -88.0) rectangle (343.95000000000005,-90.6);
\draw(333.65000000000003, -90.69999999999999) node[anchor=north west,align=left] {Convexity of\\ real functions\\  of several\\ variables,\\ generalizations};
\draw (333.65000000000003, -90.69999999999999) rectangle (338.00000000000006,-93.29999999999998);
\draw(338.1, -90.69999999999999) node[anchor=north west,align=left] {Representation\\  and\\ superposition\\ of functions};
\draw (338.1, -90.69999999999999) rectangle (342.20000000000005,-92.79999999999998);
\draw(333.65000000000003, -93.39999999999999) node[anchor=north west,align=left] {Calculus\\ of vector\\ functions};
\draw (333.65000000000003, -93.39999999999999) rectangle (336.50000000000006,-94.99999999999999);
\draw(317.07000000000005, -83.89999999999999) node[anchor=north west,align=left] {\large Miscellaneous topics in real functions};
\draw (317.07000000000005, -83.89999999999999) rectangle (331.4200000000001,-93.19999999999999);
\draw(318.07000000000005, -84.89999999999999) node[anchor=north west,align=left] {Calculus of\\ functions on\\ infinite-dimensional\\ spaces};
\draw (318.07000000000005, -84.89999999999999) rectangle (323.6700000000001,-86.99999999999999);
\draw(323.77000000000004, -84.89999999999999) node[anchor=north west,align=left] {Calculus of\\ functions taking\\  values in\\ infinite-dimensional\\ spaces};
\draw (323.77000000000004, -84.89999999999999) rectangle (329.37000000000006,-87.49999999999999);
\draw(329.47, -84.89999999999999) node[anchor=north west,align=left] {Means};
\draw (329.47, -84.89999999999999) rectangle (331.32000000000005,-85.49999999999999);
\draw(318.07000000000005, -87.6) node[anchor=north west,align=left] {Non-Archimedean\\ analysis};
\draw (318.07000000000005, -87.6) rectangle (322.4200000000001,-88.69999999999999);
\draw(322.52000000000004, -87.6) node[anchor=north west,align=left] {\(C^\infty\)-functions,\\ quasi-analytic\\ functions};
\draw (322.52000000000004, -87.6) rectangle (326.62000000000006,-89.19999999999999);
\draw(326.72, -87.6) node[anchor=north west,align=left] {Real analysis\\  on time\\ scales or\\ measure chains};
\draw (326.72, -87.6) rectangle (330.82000000000005,-89.69999999999999);
\draw(318.07000000000005, -89.8) node[anchor=north west,align=left] {Real-analytic\\ functions};
\draw (318.07000000000005, -89.8) rectangle (321.9200000000001,-90.89999999999999);
\draw(322.02000000000004, -89.8) node[anchor=north west,align=left] {Constructive\\ real\\ analysis};
\draw (322.02000000000004, -89.8) rectangle (325.62000000000006,-91.39999999999999);
\draw(325.72, -89.8) node[anchor=north west,align=left] {Nonstandard\\ analysis};
\draw (325.72, -89.8) rectangle (329.07000000000005,-90.89999999999999);
\draw(318.07000000000005, -91.49999999999999) node[anchor=north west,align=left] {Set-valued\\ functions};
\draw (318.07000000000005, -91.49999999999999) rectangle (321.1700000000001,-92.59999999999998);
\draw(321.27000000000004, -91.49999999999999) node[anchor=north west,align=left] {Fuzzy\\ real\\ analysis};
\draw (321.27000000000004, -91.49999999999999) rectangle (323.87000000000006,-93.09999999999998);
\draw(317.07000000000005, -93.3) node[anchor=north west,align=left] {\large Computational methods\\ for problems pertaining\\ to real functions};
\draw (317.07000000000005, -93.3) rectangle (324.80000000000007,-94.89999999999999);
\draw(317.07000000000005, -95.19999999999999) node[anchor=north west,align=left] {\large Functions of one variable};
\draw (317.07000000000005, -95.19999999999999) rectangle (330.9200000000001,-116.1);
\draw(318.07000000000005, -96.19999999999999) node[anchor=north west,align=left] {Continuity and\\ related questions\\ (modulus of continuity,\\ semicontinuity,\\  discontinuities,\\ etc.) for real\\ functions in one variable};
\draw (318.07000000000005, -96.19999999999999) rectangle (324.9200000000001,-99.79999999999998);
\draw(325.02000000000004, -96.19999999999999) node[anchor=north west,align=left] {Foundations:\\ limits and\\ generalizations,\\ elementary\\ topology of the line};
\draw (325.02000000000004, -96.19999999999999) rectangle (330.62000000000006,-98.79999999999998);
\draw(325.02000000000004, -98.89999999999999) node[anchor=north west,align=left] {Antidifferentiation};
\draw (325.02000000000004, -98.89999999999999) rectangle (330.37000000000006,-99.49999999999999);
\draw(318.07000000000005, -99.89999999999999) node[anchor=north west,align=left] {Nondifferentiability\\ (nondifferentiable\\  functions,\\ points of\\ nondifferentiability),\\ discontinuous derivatives};
\draw (318.07000000000005, -99.89999999999999) rectangle (324.9200000000001,-102.99999999999999);
\draw(325.02000000000004, -99.89999999999999) node[anchor=north west,align=left] {Singular functions,\\  Cantor\\ functions, functions\\  with other\\ special properties};
\draw (325.02000000000004, -99.89999999999999) rectangle (330.62000000000006,-102.49999999999999);
\draw(318.07000000000005, -103.1) node[anchor=north west,align=left] {Differentiation\\ (real functions of\\ one variable): general\\ theory, generalized\\ derivatives,\\ mean value theorems};
\draw (318.07000000000005, -103.1) rectangle (324.1700000000001,-106.19999999999999);
\draw(324.27000000000004, -103.1) node[anchor=north west,align=left] {Classification\\ of real functions;\\  Baire\\ classification of\\ sets and functions};
\draw (324.27000000000004, -103.1) rectangle (329.37000000000006,-105.69999999999999);
\draw(318.07000000000005, -106.29999999999998) node[anchor=north west,align=left] {Rate of growth\\ of functions,\\  orders of\\ infinity, slowly\\ varying functions};
\draw (318.07000000000005, -106.29999999999998) rectangle (322.9200000000001,-108.89999999999998);
\draw(323.02000000000004, -106.29999999999998) node[anchor=north west,align=left] {Denjoy and\\ Perron integrals,\\ other special\\ integrals};
\draw (323.02000000000004, -106.29999999999998) rectangle (327.87000000000006,-108.39999999999998);
\draw(327.97, -106.29999999999998) node[anchor=north west,align=left] {Lipschitz\\ (Hölder)\\ classes};
\draw (327.97, -106.29999999999998) rectangle (330.82000000000005,-107.89999999999998);
\draw(318.07000000000005, -108.99999999999999) node[anchor=north west,align=left] {Functions\\ of bounded\\ variation,\\ generalizations};
\draw (318.07000000000005, -108.99999999999999) rectangle (322.4200000000001,-111.09999999999998);
\draw(322.52000000000004, -108.99999999999999) node[anchor=north west,align=left] {Absolutely\\ continuous real\\ functions in\\ one variable};
\draw (322.52000000000004, -108.99999999999999) rectangle (326.87000000000006,-111.09999999999998);
\draw(326.97, -108.99999999999999) node[anchor=north west,align=left] {Integrals of\\  Riemann,\\ Stieltjes and\\ Lebesgue type};
\draw (326.97, -108.99999999999999) rectangle (330.82000000000005,-111.09999999999998);
\draw(318.07000000000005, -111.19999999999999) node[anchor=north west,align=left] {Monotonic\\ functions,\\ generalizations};
\draw (318.07000000000005, -111.19999999999999) rectangle (322.4200000000001,-112.79999999999998);
\draw(322.52000000000004, -111.19999999999999) node[anchor=north west,align=left] {Convexity of\\ real functions\\  in one\\ variable,\\ generalizations};
\draw (322.52000000000004, -111.19999999999999) rectangle (326.87000000000006,-113.79999999999998);
\draw(326.97, -111.19999999999999) node[anchor=north west,align=left] {One-variable\\ calculus};
\draw (326.97, -111.19999999999999) rectangle (330.57000000000005,-112.29999999999998);
\draw(326.97, -112.39999999999999) node[anchor=north west,align=left] {Elementary\\ functions};
\draw (326.97, -112.39999999999999) rectangle (330.07000000000005,-113.49999999999999);
\draw(318.07000000000005, -113.89999999999999) node[anchor=north west,align=left] {Iteration\\ of real\\ functions in\\ one variable};
\draw (318.07000000000005, -113.89999999999999) rectangle (321.6700000000001,-115.99999999999999);
\draw(321.77000000000004, -113.89999999999999) node[anchor=north west,align=left] {Fractional\\ derivatives\\  and\\ integrals};
\draw (321.77000000000004, -113.89999999999999) rectangle (325.12000000000006,-115.99999999999999);
\draw(331.02000000000004, -95.19999999999999) node[anchor=north west,align=left] {\large Inequalities in real analysis};
\draw (331.02000000000004, -95.19999999999999) rectangle (341.17,-102.79999999999998);
\draw(332.02000000000004, -96.19999999999999) node[anchor=north west,align=left] {Inequalities\\ involving\\ derivatives and\\ differential and\\ integral operators};
\draw (332.02000000000004, -96.19999999999999) rectangle (337.12000000000006,-98.79999999999998);
\draw(337.22, -96.19999999999999) node[anchor=north west,align=left] {Inequalities\\  for\\ trigonometric\\ functions and\\ polynomials};
\draw (337.22, -96.19999999999999) rectangle (341.07000000000005,-98.79999999999998);
\draw(332.02000000000004, -98.89999999999999) node[anchor=north west,align=left] {Inequalities\\ for sums,\\  series\\ and integrals};
\draw (332.02000000000004, -98.89999999999999) rectangle (335.87000000000006,-100.99999999999999);
\draw(335.97, -98.89999999999999) node[anchor=north west,align=left] {Inequalities\\ involving\\ other types\\ of functions};
\draw (335.97, -98.89999999999999) rectangle (339.57000000000005,-100.99999999999999);
\draw(332.02000000000004, -101.1) node[anchor=north west,align=left] {Other\\ analytical\\ inequalities};
\draw (332.02000000000004, -101.1) rectangle (335.62000000000006,-102.69999999999999);
\draw(331.02000000000004, -102.89999999999999) node[anchor=north west,align=left] {\large History of\\ real functions};
\draw (331.02000000000004, -102.89999999999999) rectangle (335.96000000000004,-103.99999999999999);
\draw(316.07000000000005, -116.3) node[anchor=north west,align=left] {\LARGE Potential theory};
\draw (316.07000000000005, -116.3) rectangle (342.77000000000004,-147.4);
\draw(317.07000000000005, -117.3) node[anchor=north west,align=left] {\large Potential theory on fractals and metric spaces};
\draw (317.07000000000005, -117.3) rectangle (331.93000000000006,-120.5);
\draw(318.07000000000005, -118.3) node[anchor=north west,align=left] {Potential\\ theory on\\ fractals and\\ metric spaces};
\draw (318.07000000000005, -118.3) rectangle (321.9200000000001,-120.39999999999999);
\draw(332.03000000000003, -117.3) node[anchor=north west,align=left] {\large Axiomatic potential theory};
\draw (332.03000000000003, -117.3) rectangle (340.69000000000005,-120.0);
\draw(333.03000000000003, -118.3) node[anchor=north west,align=left] {Axiomatic\\ potential\\ theory};
\draw (333.03000000000003, -118.3) rectangle (335.88000000000005,-119.89999999999999);
\draw(317.07000000000005, -120.6) node[anchor=north west,align=left] {\large Generalizations of potential theory};
\draw (317.07000000000005, -120.6) rectangle (330.1700000000001,-129.7);
\draw(318.07000000000005, -121.6) node[anchor=north west,align=left] {Pluriharmonic\\  and\\ plurisubharmonic\\ functions};
\draw (318.07000000000005, -121.6) rectangle (322.6700000000001,-123.69999999999999);
\draw(322.77000000000004, -121.6) node[anchor=north west,align=left] {Harmonic,\\ subharmonic,\\ superharmonic\\  functions\\ on other spaces};
\draw (322.77000000000004, -121.6) rectangle (327.12000000000006,-124.19999999999999);
\draw(327.22, -121.6) node[anchor=north west,align=left] {Discrete\\ potential\\ theory};
\draw (327.22, -121.6) rectangle (330.07000000000005,-123.19999999999999);
\draw(318.07000000000005, -124.3) node[anchor=north west,align=left] {Fine potential\\  theory;\\ fine properties\\  of sets\\ and functions};
\draw (318.07000000000005, -124.3) rectangle (322.4200000000001,-126.89999999999999);
\draw(322.52000000000004, -124.3) node[anchor=north west,align=left] {Other\\ generalizations\\ (nonlinear\\ potential\\ theory, etc.)};
\draw (322.52000000000004, -124.3) rectangle (326.87000000000006,-126.89999999999999);
\draw(326.97, -124.3) node[anchor=north west,align=left] {Dirichlet\\ forms};
\draw (326.97, -124.3) rectangle (329.82000000000005,-125.39999999999999);
\draw(318.07000000000005, -127.0) node[anchor=north west,align=left] {Potential\\ theory on\\ Riemannian\\ manifolds and\\ other spaces};
\draw (318.07000000000005, -127.0) rectangle (321.9200000000001,-129.6);
\draw(322.02000000000004, -127.0) node[anchor=north west,align=left] {Potentials\\  and\\ capacities on\\ other spaces};
\draw (322.02000000000004, -127.0) rectangle (325.87000000000006,-129.1);
\draw(325.97, -127.0) node[anchor=north west,align=left] {Martin\\ boundary\\ theory};
\draw (325.97, -127.0) rectangle (328.57000000000005,-128.6);
\draw(330.27000000000004, -120.6) node[anchor=north west,align=left] {\large Two-dimensional potential theory};
\draw (330.27000000000004, -120.6) rectangle (342.67,-133.9);
\draw(331.27000000000004, -121.6) node[anchor=north west,align=left] {Connections of\\ harmonic functions\\  with\\ differential equations\\ in two dimensions};
\draw (331.27000000000004, -121.6) rectangle (337.37000000000006,-124.19999999999999);
\draw(337.47, -121.6) node[anchor=north west,align=left] {Potentials and\\ capacity, harmonic\\ measure, extremal\\  length and\\ related notions\\ in two dimensions};
\draw (337.47, -121.6) rectangle (342.57000000000005,-124.69999999999999);
\draw(331.27000000000004, -124.8) node[anchor=north west,align=left] {Boundary value\\ and inverse\\ problems for harmonic\\  functions\\ in two dimensions};
\draw (331.27000000000004, -124.8) rectangle (337.12000000000006,-127.39999999999999);
\draw(337.22, -124.8) node[anchor=north west,align=left] {Integral\\ representations,\\ integral operators,\\ integral equations\\ methods in\\ two dimensions};
\draw (337.22, -124.8) rectangle (342.57000000000005,-127.89999999999999);
\draw(331.27000000000004, -128.0) node[anchor=north west,align=left] {Biharmonic,\\ polyharmonic\\ functions and\\ equations, Poisson’s\\  equation\\ in two dimensions};
\draw (331.27000000000004, -128.0) rectangle (336.87000000000006,-131.1);
\draw(336.97, -128.0) node[anchor=north west,align=left] {Boundary behavior\\  (theorems\\ of Fatou type,\\  etc.) of\\ harmonic functions\\ in two dimensions};
\draw (336.97, -128.0) rectangle (342.07000000000005,-131.1);
\draw(331.27000000000004, -131.2) node[anchor=north west,align=left] {Harmonic,\\ subharmonic,\\ superharmonic\\  functions\\ in two dimensions};
\draw (331.27000000000004, -131.2) rectangle (336.12000000000006,-133.79999999999998);
\draw(317.07000000000005, -129.8) node[anchor=north west,align=left] {\large Computational methods\\ for problems pertaining\\ to potential theory};
\draw (317.07000000000005, -129.8) rectangle (324.80000000000007,-131.4);
\draw(317.07000000000005, -131.5) node[anchor=north west,align=left] {\large History of\\ potential theory};
\draw (317.07000000000005, -131.5) rectangle (322.63000000000005,-132.6);
\draw(317.07000000000005, -134.0) node[anchor=north west,align=left] {\large Higher-dimensional potential theory};
\draw (317.07000000000005, -134.0) rectangle (329.47,-147.3);
\draw(318.07000000000005, -135.0) node[anchor=north west,align=left] {Boundary value\\ and inverse\\ problems for harmonic\\  functions\\ in higher dimensions};
\draw (318.07000000000005, -135.0) rectangle (323.9200000000001,-137.6);
\draw(324.02000000000004, -135.0) node[anchor=north west,align=left] {Integral\\ representations,\\ integral operators,\\ integral equations\\ methods in\\ higher dimensions};
\draw (324.02000000000004, -135.0) rectangle (329.37000000000006,-138.1);
\draw(318.07000000000005, -138.2) node[anchor=north west,align=left] {Potentials and\\ capacities,\\  extremal\\ length and related\\ notions in\\ higher dimensions};
\draw (318.07000000000005, -138.2) rectangle (323.1700000000001,-141.29999999999998);
\draw(323.27000000000004, -138.2) node[anchor=north west,align=left] {Boundary\\ behavior of\\ harmonic functions\\ in higher\\ dimensions};
\draw (323.27000000000004, -138.2) rectangle (328.37000000000006,-140.79999999999998);
\draw(318.07000000000005, -141.4) node[anchor=north west,align=left] {Harmonic,\\ subharmonic,\\ superharmonic\\ functions in\\ higher dimensions};
\draw (318.07000000000005, -141.4) rectangle (322.9200000000001,-144.0);
\draw(323.02000000000004, -141.4) node[anchor=north west,align=left] {Biharmonic and\\  polyharmonic\\ equations and\\ functions in\\ higher dimensions};
\draw (323.02000000000004, -141.4) rectangle (327.87000000000006,-144.0);
\draw(318.07000000000005, -144.1) node[anchor=north west,align=left] {Connections\\ of harmonic\\ functions with\\ differential\\ equations in\\ higher dimensions};
\draw (318.07000000000005, -144.1) rectangle (322.9200000000001,-147.2);
\draw(345.57000000000005, -1) node[anchor=north west,align=left] {\LARGE Special functions};
\draw (345.57000000000005, -1) rectangle (371.5400000000001,-47.300000000000004);
\draw(346.57000000000005, -2) node[anchor=north west,align=left] {\large Computational aspects of special functions};
\draw (346.57000000000005, -2) rectangle (360.19000000000005,-6.199999999999999);
\draw(347.57000000000005, -3) node[anchor=north west,align=left] {Numerical\\ approximation\\  and\\ evaluation of\\ special functions};
\draw (347.57000000000005, -3) rectangle (352.4200000000001,-5.6);
\draw(352.52000000000004, -3) node[anchor=north west,align=left] {Symbolic\\ computation of\\ special functions\\ (Gosper and\\  Zeilberger\\ algorithms, etc.)};
\draw (352.52000000000004, -3) rectangle (357.37000000000006,-6.1);
\draw(360.2900000000001, -2) node[anchor=north west,align=left] {\large Elementary classical functions};
\draw (360.2900000000001, -2) rectangle (371.44000000000005,-8.4);
\draw(361.2900000000001, -3) node[anchor=north west,align=left] {Incomplete beta\\  and gamma\\ functions (error\\ functions, probability\\ integral,\\ Fresnel integrals)};
\draw (361.2900000000001, -3) rectangle (367.3900000000001,-6.1);
\draw(367.49000000000007, -3) node[anchor=north west,align=left] {Exponential\\  and\\ trigonometric\\ functions};
\draw (367.49000000000007, -3) rectangle (371.3400000000001,-5.1);
\draw(361.2900000000001, -6.2) node[anchor=north west,align=left] {Gamma,\\ beta and\\ polygamma\\ functions};
\draw (361.2900000000001, -6.2) rectangle (364.1400000000001,-8.3);
\draw(364.24000000000007, -6.2) node[anchor=north west,align=left] {Higher\\ logarithm\\ functions};
\draw (364.24000000000007, -6.2) rectangle (367.0900000000001,-7.800000000000001);
\draw(346.57000000000005, -6.299999999999999) node[anchor=north west,align=left] {\large History of\\ special functions};
\draw (346.57000000000005, -6.299999999999999) rectangle (352.44000000000005,-7.399999999999999);
\draw(346.57000000000005, -8.5) node[anchor=north west,align=left] {\large Hypergeometric functions};
\draw (346.57000000000005, -8.5) rectangle (360.1700000000001,-28.400000000000002);
\draw(347.57000000000005, -9.5) node[anchor=north west,align=left] {Orthogonal\\ polynomials and functions\\  in several\\ variables expressible\\  in terms\\ of special functions\\ in one variable};
\draw (347.57000000000005, -9.5) rectangle (354.4200000000001,-13.1);
\draw(354.52000000000004, -9.5) node[anchor=north west,align=left] {Hypergeometric\\ integrals and\\ functions defined\\ by them (\(E\), \(G\), \(H\)\\ and \(I\) functions)};
\draw (354.52000000000004, -9.5) rectangle (359.37000000000006,-12.1);
\draw(347.57000000000005, -13.2) node[anchor=north west,align=left] {Other\\ hypergeometric functions\\  and\\ integrals in several\\ variables};
\draw (347.57000000000005, -13.2) rectangle (354.1700000000001,-15.799999999999999);
\draw(354.27000000000004, -13.2) node[anchor=north west,align=left] {Bessel and\\ Airy functions,\\  cylinder\\ functions, \({}_0F_1\)};
\draw (354.27000000000004, -13.2) rectangle (358.62000000000006,-15.299999999999999);
\draw(347.57000000000005, -15.9) node[anchor=north west,align=left] {Orthogonal polynomials\\ and functions\\ of hypergeometric\\ type (Jacobi,\\ Laguerre, Hermite,\\ Askey scheme, etc.)};
\draw (347.57000000000005, -15.9) rectangle (353.6700000000001,-19.0);
\draw(353.77000000000004, -15.9) node[anchor=north west,align=left] {Orthogonal\\ polynomials and\\  functions\\ associated with\\ root systems};
\draw (353.77000000000004, -15.9) rectangle (358.12000000000006,-18.5);
\draw(347.57000000000005, -19.1) node[anchor=north west,align=left] {Hypergeometric\\  functions\\ associated with\\ root systems};
\draw (347.57000000000005, -19.1) rectangle (351.9200000000001,-21.200000000000003);
\draw(352.02000000000004, -19.1) node[anchor=north west,align=left] {Connections of\\ hypergeometric\\  functions\\ with groups and\\ algebras, and\\ related topics};
\draw (352.02000000000004, -19.1) rectangle (356.37000000000006,-22.200000000000003);
\draw(356.47, -19.1) node[anchor=north west,align=left] {Appell,\\ Horn and\\ Lauricella\\ functions};
\draw (356.47, -19.1) rectangle (359.57000000000005,-21.200000000000003);
\draw(347.57000000000005, -22.3) node[anchor=north west,align=left] {Classical\\ hypergeometric\\ functions, \({}_2F_1\)};
\draw (347.57000000000005, -22.3) rectangle (351.6700000000001,-23.900000000000002);
\draw(351.77000000000004, -22.3) node[anchor=north west,align=left] {Confluent\\ hypergeometric\\ functions,\\ Whittaker\\ functions, \({}_1F_1\)};
\draw (351.77000000000004, -22.3) rectangle (355.87000000000006,-24.900000000000002);
\draw(355.97, -22.3) node[anchor=north west,align=left] {Generalized\\ hypergeometric\\ series, \({}_pF_q\)};
\draw (355.97, -22.3) rectangle (360.07000000000005,-23.900000000000002);
\draw(347.57000000000005, -25.0) node[anchor=north west,align=left] {Elliptic\\ integrals as\\ hypergeometric\\ functions};
\draw (347.57000000000005, -25.0) rectangle (351.6700000000001,-27.1);
\draw(351.77000000000004, -25.0) node[anchor=north west,align=left] {Applications\\  of\\ hypergeometric\\ functions};
\draw (351.77000000000004, -25.0) rectangle (355.87000000000006,-27.1);
\draw(355.97, -25.0) node[anchor=north west,align=left] {Other special\\ orthogonal\\ polynomials\\ and functions};
\draw (355.97, -25.0) rectangle (359.82000000000005,-27.1);
\draw(347.57000000000005, -27.2) node[anchor=north west,align=left] {Spherical\\ harmonics};
\draw (347.57000000000005, -27.2) rectangle (350.4200000000001,-28.3);
\draw(360.27000000000004, -8.5) node[anchor=north west,align=left] {\large Other special functions};
\draw (360.27000000000004, -8.5) rectangle (370.17,-19.3);
\draw(361.27000000000004, -9.5) node[anchor=north west,align=left] {Other functions\\  coming from\\ differential,\\ difference and\\ integral equations};
\draw (361.27000000000004, -9.5) rectangle (366.37000000000006,-12.1);
\draw(366.47, -9.5) node[anchor=north west,align=left] {Other\\ wave\\ functions};
\draw (366.47, -9.5) rectangle (369.32000000000005,-11.1);
\draw(361.27000000000004, -12.2) node[anchor=north west,align=left] {Special\\ functions in\\ characteristic\\ \(p\) (gamma\\ functions, etc.)};
\draw (361.27000000000004, -12.2) rectangle (365.87000000000006,-14.799999999999999);
\draw(365.97, -12.2) node[anchor=north west,align=left] {Elliptic\\ functions\\ and integrals};
\draw (365.97, -12.2) rectangle (369.82000000000005,-13.799999999999999);
\draw(361.27000000000004, -14.9) node[anchor=north west,align=left] {Lamé, Mathieu,\\  and\\ spheroidal wave\\ functions};
\draw (361.27000000000004, -14.9) rectangle (365.62000000000006,-17.0);
\draw(365.72, -14.9) node[anchor=north west,align=left] {Mittag-Leffler\\ functions\\  and\\ generalizations};
\draw (365.72, -14.9) rectangle (370.07000000000005,-17.0);
\draw(361.27000000000004, -17.1) node[anchor=north west,align=left] {Other functions\\ defined\\ by series\\ and integrals};
\draw (361.27000000000004, -17.1) rectangle (365.62000000000006,-19.200000000000003);
\draw(365.72, -17.1) node[anchor=north west,align=left] {Painlevé-type\\ functions};
\draw (365.72, -17.1) rectangle (369.57000000000005,-18.200000000000003);
\draw(346.57000000000005, -28.500000000000004) node[anchor=north west,align=left] {\large Basic hypergeometric functions};
\draw (346.57000000000005, -28.500000000000004) rectangle (359.72,-47.2);
\draw(347.57000000000005, -29.500000000000004) node[anchor=north west,align=left] {Orthogonal polynomials\\ and functions\\ in several variables\\ expressible in\\  terms of basic\\ hypergeometric functions\\ in one variable};
\draw (347.57000000000005, -29.500000000000004) rectangle (354.1700000000001,-33.1);
\draw(354.27000000000004, -29.500000000000004) node[anchor=north west,align=left] {Basic orthogonal\\  polynomials\\ and functions\\ (Askey-Wilson\\ polynomials, etc.)};
\draw (354.27000000000004, -29.500000000000004) rectangle (359.37000000000006,-32.1);
\draw(347.57000000000005, -33.2) node[anchor=north west,align=left] {Connections of basic\\  hypergeometric\\ functions with quantum\\ groups, Chevalley\\  groups, \(p\)-adic\\ groups, Hecke algebras,\\ and related topics};
\draw (347.57000000000005, -33.2) rectangle (353.9200000000001,-36.800000000000004);
\draw(354.02000000000004, -33.2) node[anchor=north west,align=left] {Basic orthogonal\\ polynomials and\\  functions\\ associated with root\\ systems (Macdonald\\ polynomials, etc.)};
\draw (354.02000000000004, -33.2) rectangle (359.62000000000006,-36.300000000000004);
\draw(347.57000000000005, -36.900000000000006) node[anchor=north west,align=left] {\(q\)-gamma\\ functions, \(q\)-beta\\  functions\\ and integrals};
\draw (347.57000000000005, -36.900000000000006) rectangle (352.4200000000001,-39.00000000000001);
\draw(352.52000000000004, -36.900000000000006) node[anchor=north west,align=left] {Basic\\ hypergeometric\\ functions\\ associated\\ with root systems};
\draw (352.52000000000004, -36.900000000000006) rectangle (357.37000000000006,-39.50000000000001);
\draw(347.57000000000005, -39.60000000000001) node[anchor=north west,align=left] {Other basic\\ hypergeometric\\ functions and\\ integrals in\\ several variables};
\draw (347.57000000000005, -39.60000000000001) rectangle (352.4200000000001,-42.20000000000001);
\draw(352.52000000000004, -39.60000000000001) node[anchor=north west,align=left] {Basic\\ hypergeometric\\ functions in one\\ variable, \({}_r\phi_s\)};
\draw (352.52000000000004, -39.60000000000001) rectangle (357.12000000000006,-41.70000000000001);
\draw(347.57000000000005, -42.300000000000004) node[anchor=north west,align=left] {Basic\\ hypergeometric\\ integrals and\\ functions\\ defined by them};
\draw (347.57000000000005, -42.300000000000004) rectangle (351.9200000000001,-44.900000000000006);
\draw(352.02000000000004, -42.300000000000004) node[anchor=north west,align=left] {Applications\\  of basic\\ hypergeometric\\ functions};
\draw (352.02000000000004, -42.300000000000004) rectangle (356.12000000000006,-44.400000000000006);
\draw(347.57000000000005, -45.0) node[anchor=north west,align=left] {Bibasic\\ functions\\ and multiple\\ bases};
\draw (347.57000000000005, -45.0) rectangle (351.1700000000001,-47.1);
\draw(345.57000000000005, -47.400000000000006) node[anchor=north west,align=left] {\LARGE Order, lattices, ordered algebraic structures};
\draw (345.57000000000005, -47.400000000000006) rectangle (371.52000000000004,-81.30000000000001);
\draw(346.57000000000005, -48.400000000000006) node[anchor=north west,align=left] {\large Modular lattices, complemented lattices};
\draw (346.57000000000005, -48.400000000000006) rectangle (361.1700000000001,-54.300000000000004);
\draw(347.57000000000005, -49.400000000000006) node[anchor=north west,align=left] {Complemented\\  lattices,\\ orthocomplemented\\ lattices\\ and posets};
\draw (347.57000000000005, -49.400000000000006) rectangle (352.4200000000001,-52.00000000000001);
\draw(352.52000000000004, -49.400000000000006) node[anchor=north west,align=left] {Complemented\\ modular lattices,\\ continuous\\ geometries};
\draw (352.52000000000004, -49.400000000000006) rectangle (357.37000000000006,-51.50000000000001);
\draw(357.47, -49.400000000000006) node[anchor=north west,align=left] {Modular\\ lattices,\\ Desarguesian\\ lattices};
\draw (357.47, -49.400000000000006) rectangle (361.07000000000005,-51.50000000000001);
\draw(347.57000000000005, -52.10000000000001) node[anchor=north west,align=left] {Semimodular\\ lattices,\\ geometric\\ lattices};
\draw (347.57000000000005, -52.10000000000001) rectangle (350.9200000000001,-54.20000000000001);
\draw(361.27000000000004, -48.400000000000006) node[anchor=north west,align=left] {\large Ordered structures};
\draw (361.27000000000004, -48.400000000000006) rectangle (371.42,-57.50000000000001);
\draw(362.27000000000004, -49.400000000000006) node[anchor=north west,align=left] {Ordered\\ topological\\ structures (aspects\\ of ordered\\ structures)};
\draw (362.27000000000004, -49.400000000000006) rectangle (367.62000000000006,-52.00000000000001);
\draw(367.72, -49.400000000000006) node[anchor=north west,align=left] {Ordered\\ semigroups\\ and monoids};
\draw (367.72, -49.400000000000006) rectangle (371.07000000000005,-51.00000000000001);
\draw(367.72, -51.10000000000001) node[anchor=north west,align=left] {Quantales};
\draw (367.72, -51.10000000000001) rectangle (370.57000000000005,-51.70000000000001);
\draw(362.27000000000004, -52.10000000000001) node[anchor=north west,align=left] {Ordered\\ rings, algebras,\\ modules};
\draw (362.27000000000004, -52.10000000000001) rectangle (366.87000000000006,-53.70000000000001);
\draw(366.97, -52.10000000000001) node[anchor=north west,align=left] {Ordered\\ abelian groups,\\  Riesz\\ groups, ordered\\ linear spaces};
\draw (366.97, -52.10000000000001) rectangle (371.32000000000005,-54.70000000000001);
\draw(362.27000000000004, -54.800000000000004) node[anchor=north west,align=left] {BCK-algebras,\\ BCI-algebras\\ (aspects\\ of ordered\\ structures)};
\draw (362.27000000000004, -54.800000000000004) rectangle (366.12000000000006,-57.400000000000006);
\draw(366.22, -54.800000000000004) node[anchor=north west,align=left] {Noether\\ lattices};
\draw (366.22, -54.800000000000004) rectangle (368.82000000000005,-55.900000000000006);
\draw(366.22, -56.00000000000001) node[anchor=north west,align=left] {Ordered\\ groups};
\draw (366.22, -56.00000000000001) rectangle (368.57000000000005,-57.10000000000001);
\draw(346.57000000000005, -54.400000000000006) node[anchor=north west,align=left] {\large Computational methods\\ for problems pertaining\\ to ordered structures};
\draw (346.57000000000005, -54.400000000000006) rectangle (354.30000000000007,-56.00000000000001);
\draw(346.57000000000005, -56.10000000000001) node[anchor=north west,align=left] {\large History of\\ ordered structures};
\draw (346.57000000000005, -56.10000000000001) rectangle (352.75000000000006,-57.20000000000001);
\draw(346.57000000000005, -57.60000000000001) node[anchor=north west,align=left] {\large Distributive lattices};
\draw (346.57000000000005, -57.60000000000001) rectangle (357.72,-70.10000000000001);
\draw(347.57000000000005, -58.60000000000001) node[anchor=north west,align=left] {De Morgan\\ algebras, Łukasiewicz\\ algebras\\ (lattice-theoretic\\ aspects)};
\draw (347.57000000000005, -58.60000000000001) rectangle (353.4200000000001,-61.20000000000001);
\draw(353.52000000000004, -58.60000000000001) node[anchor=north west,align=left] {Complete\\ distributivity};
\draw (353.52000000000004, -58.60000000000001) rectangle (357.62000000000006,-59.70000000000001);
\draw(353.52000000000004, -59.80000000000001) node[anchor=north west,align=left] {MV-algebras};
\draw (353.52000000000004, -59.80000000000001) rectangle (356.87000000000006,-60.40000000000001);
\draw(347.57000000000005, -61.30000000000001) node[anchor=north west,align=left] {Pseudocomplemented\\ lattices};
\draw (347.57000000000005, -61.30000000000001) rectangle (352.6700000000001,-62.40000000000001);
\draw(352.77000000000004, -61.30000000000001) node[anchor=north west,align=left] {Structure and\\ representation\\  theory\\ of distributive\\ lattices};
\draw (352.77000000000004, -61.30000000000001) rectangle (357.12000000000006,-63.90000000000001);
\draw(347.57000000000005, -62.50000000000001) node[anchor=north west,align=left] {Frames,\\ locales};
\draw (347.57000000000005, -62.50000000000001) rectangle (349.9200000000001,-63.60000000000001);
\draw(347.57000000000005, -64.00000000000001) node[anchor=north west,align=left] {Heyting\\ algebras\\ (lattice-theoretic\\ aspects)};
\draw (347.57000000000005, -64.00000000000001) rectangle (352.6700000000001,-66.10000000000001);
\draw(352.77000000000004, -64.00000000000001) node[anchor=north west,align=left] {Other\\ generalizations\\ of distributive\\ lattices};
\draw (352.77000000000004, -64.00000000000001) rectangle (357.12000000000006,-66.10000000000001);
\draw(347.57000000000005, -66.20000000000002) node[anchor=north west,align=left] {Post algebras\\ (lattice-theoretic\\ aspects)};
\draw (347.57000000000005, -66.20000000000002) rectangle (352.6700000000001,-67.80000000000001);
\draw(352.77000000000004, -66.20000000000002) node[anchor=north west,align=left] {Fuzzy lattices\\  (soft\\ algebras) and\\ related topics};
\draw (352.77000000000004, -66.20000000000002) rectangle (356.87000000000006,-68.30000000000001);
\draw(347.57000000000005, -68.4) node[anchor=north west,align=left] {Lattices\\ and\\ duality};
\draw (347.57000000000005, -68.4) rectangle (350.1700000000001,-70.0);
\draw(357.82000000000005, -57.60000000000001) node[anchor=north west,align=left] {\large Ordered sets};
\draw (357.82000000000005, -57.60000000000001) rectangle (368.47,-65.2);
\draw(358.82000000000005, -58.60000000000001) node[anchor=north west,align=left] {Galois\\ correspondences, closure\\  operators\\ (in relation\\ to ordered sets)};
\draw (358.82000000000005, -58.60000000000001) rectangle (365.4200000000001,-61.20000000000001);
\draw(365.52000000000004, -58.60000000000001) node[anchor=north west,align=left] {Algebraic\\  aspects\\ of posets};
\draw (365.52000000000004, -58.60000000000001) rectangle (368.37000000000006,-60.20000000000001);
\draw(358.82000000000005, -61.30000000000001) node[anchor=north west,align=left] {Generalizations\\  of\\ ordered sets};
\draw (358.82000000000005, -61.30000000000001) rectangle (363.1700000000001,-62.90000000000001);
\draw(363.27000000000004, -61.30000000000001) node[anchor=north west,align=left] {Combinatorics\\  of\\ partially\\ ordered sets};
\draw (363.27000000000004, -61.30000000000001) rectangle (367.12000000000006,-63.40000000000001);
\draw(358.82000000000005, -63.50000000000001) node[anchor=north west,align=left] {Semilattices};
\draw (358.82000000000005, -63.50000000000001) rectangle (362.4200000000001,-64.10000000000001);
\draw(362.52000000000004, -63.50000000000001) node[anchor=north west,align=left] {Partial\\ orders,\\ general};
\draw (362.52000000000004, -63.50000000000001) rectangle (364.87000000000006,-65.10000000000001);
\draw(364.97, -63.50000000000001) node[anchor=north west,align=left] {Total\\ orders};
\draw (364.97, -63.50000000000001) rectangle (367.07000000000005,-64.60000000000001);
\draw(346.57000000000005, -70.2) node[anchor=north west,align=left] {\large Boolean algebras (Boolean rings)};
\draw (346.57000000000005, -70.2) rectangle (357.1700000000001,-78.3);
\draw(347.57000000000005, -71.2) node[anchor=north west,align=left] {Stone spaces\\ (Boolean spaces)\\ and related\\ structures};
\draw (347.57000000000005, -71.2) rectangle (352.1700000000001,-73.3);
\draw(352.27000000000004, -71.2) node[anchor=north west,align=left] {Boolean algebras\\ with additional\\  operations\\ (diagonalizable\\ algebras, etc.)};
\draw (352.27000000000004, -71.2) rectangle (356.87000000000006,-73.8);
\draw(347.57000000000005, -73.9) node[anchor=north west,align=left] {Generalizations\\ of Boolean\\ algebras};
\draw (347.57000000000005, -73.9) rectangle (351.9200000000001,-75.5);
\draw(352.02000000000004, -73.9) node[anchor=north west,align=left] {Ring-theoretic\\ properties\\ of Boolean\\ algebras};
\draw (352.02000000000004, -73.9) rectangle (356.12000000000006,-76.0);
\draw(347.57000000000005, -76.10000000000001) node[anchor=north west,align=left] {Chain\\ conditions,\\ complete\\ algebras};
\draw (347.57000000000005, -76.10000000000001) rectangle (350.9200000000001,-78.2);
\draw(351.02000000000004, -76.10000000000001) node[anchor=north west,align=left] {Structure\\  theory\\ of Boolean\\ algebras};
\draw (351.02000000000004, -76.10000000000001) rectangle (354.12000000000006,-78.2);
\draw(354.22, -76.10000000000001) node[anchor=north west,align=left] {Boolean\\ functions};
\draw (354.22, -76.10000000000001) rectangle (357.07000000000005,-77.2);
\draw(357.27000000000004, -70.2) node[anchor=north west,align=left] {\large Lattices};
\draw (357.27000000000004, -70.2) rectangle (365.92,-81.2);
\draw(358.27000000000004, -71.2) node[anchor=north west,align=left] {Generalizations\\ of\\ lattices};
\draw (358.27000000000004, -71.2) rectangle (362.62000000000006,-72.8);
\draw(362.72, -71.2) node[anchor=north west,align=left] {Lattice\\ ideals,\\ congruence\\ relations};
\draw (362.72, -71.2) rectangle (365.82000000000005,-73.3);
\draw(358.27000000000004, -73.4) node[anchor=north west,align=left] {Representation\\ theory of\\ lattices};
\draw (358.27000000000004, -73.4) rectangle (362.37000000000006,-75.0);
\draw(362.47, -73.4) node[anchor=north west,align=left] {Structure\\  theory\\ of lattices};
\draw (362.47, -73.4) rectangle (365.82000000000005,-75.0);
\draw(358.27000000000004, -75.10000000000001) node[anchor=north west,align=left] {Free lattices,\\ projective\\  lattices,\\ word problems};
\draw (358.27000000000004, -75.10000000000001) rectangle (362.37000000000006,-77.2);
\draw(362.47, -75.10000000000001) node[anchor=north west,align=left] {Complete\\ lattices,\\ completions};
\draw (362.47, -75.10000000000001) rectangle (365.82000000000005,-76.7);
\draw(358.27000000000004, -77.3) node[anchor=north west,align=left] {Continuous\\ lattices and\\  posets,\\ applications};
\draw (358.27000000000004, -77.3) rectangle (361.87000000000006,-79.39999999999999);
\draw(361.97, -77.3) node[anchor=north west,align=left] {Topological\\ lattices};
\draw (361.97, -77.3) rectangle (365.32000000000005,-78.39999999999999);
\draw(358.27000000000004, -79.5) node[anchor=north west,align=left] {Varieties\\  of\\ lattices};
\draw (358.27000000000004, -79.5) rectangle (361.12000000000006,-81.1);
\draw(345.57000000000005, -81.4) node[anchor=north west,align=left] {\LARGE K-Theory};
\draw (345.57000000000005, -81.4) rectangle (371.2900000000001,-115.0);
\draw(346.57000000000005, -82.4) node[anchor=north west,align=left] {\large Miscellaneous applications of \(K\)-theory};
\draw (346.57000000000005, -82.4) rectangle (360.19000000000005,-85.10000000000001);
\draw(347.57000000000005, -83.4) node[anchor=north west,align=left] {Miscellaneous\\ applications\\ of \(K\)-theory};
\draw (347.57000000000005, -83.4) rectangle (351.4200000000001,-85.0);
\draw(360.2900000000001, -82.4) node[anchor=north west,align=left] {\large Grothendieck groups and \(K_0\)};
\draw (360.2900000000001, -82.4) rectangle (371.19000000000005,-87.80000000000001);
\draw(361.2900000000001, -83.4) node[anchor=north west,align=left] {Frobenius\\ induction,\\  Burnside\\ and\\ representation rings};
\draw (361.2900000000001, -83.4) rectangle (366.8900000000001,-86.0);
\draw(366.99000000000007, -83.4) node[anchor=north west,align=left] {Stability\\ for projective\\ modules};
\draw (366.99000000000007, -83.4) rectangle (371.0900000000001,-85.0);
\draw(361.2900000000001, -86.10000000000001) node[anchor=north west,align=left] {Efficient\\ generation\\ of modules};
\draw (361.2900000000001, -86.10000000000001) rectangle (364.3900000000001,-87.7);
\draw(364.49000000000007, -86.10000000000001) node[anchor=north west,align=left] {\(K_0\) of group\\  rings\\ and orders};
\draw (364.49000000000007, -86.10000000000001) rectangle (367.5900000000001,-87.7);
\draw(367.69000000000005, -86.10000000000001) node[anchor=north west,align=left] {\(K_0\) of other\\ rings};
\draw (367.69000000000005, -86.10000000000001) rectangle (370.7900000000001,-87.2);
\draw(346.57000000000005, -85.2) node[anchor=north west,align=left] {\large Computational methods\\  for problems\\ pertaining to \(K\)-theory};
\draw (346.57000000000005, -85.2) rectangle (353.99000000000007,-86.8);
\draw(346.57000000000005, -87.9) node[anchor=north west,align=left] {\large Higher algebraic \(K\)-theory};
\draw (346.57000000000005, -87.9) rectangle (358.47,-96.7);
\draw(347.57000000000005, -88.9) node[anchor=north west,align=left] {Karoubi-Villamayor-Gersten\\ \(K\)-theory};
\draw (347.57000000000005, -88.9) rectangle (354.6700000000001,-90.0);
\draw(354.77000000000004, -88.9) node[anchor=north west,align=left] {Computations\\ of higher\\ \(K\)-theory\\ of rings};
\draw (354.77000000000004, -88.9) rectangle (358.37000000000006,-91.0);
\draw(347.57000000000005, -91.10000000000001) node[anchor=north west,align=left] {\(Q\)- and\\ plus-constructions};
\draw (347.57000000000005, -91.10000000000001) rectangle (352.6700000000001,-92.2);
\draw(352.77000000000004, -91.10000000000001) node[anchor=north west,align=left] {Higher\\ symbols, Milnor\\ \(K\)-theory};
\draw (352.77000000000004, -91.10000000000001) rectangle (357.12000000000006,-92.7);
\draw(347.57000000000005, -92.80000000000001) node[anchor=north west,align=left] {\(K\)-theory and\\ homology;\\ cyclic homology\\ and cohomology};
\draw (347.57000000000005, -92.80000000000001) rectangle (351.9200000000001,-94.9);
\draw(352.02000000000004, -92.80000000000001) node[anchor=north west,align=left] {Symmetric\\ monoidal\\ categories};
\draw (352.02000000000004, -92.80000000000001) rectangle (355.12000000000006,-94.4);
\draw(355.22, -92.80000000000001) node[anchor=north west,align=left] {Negative\\ \(K\)-theory,\\ NK and Nil};
\draw (355.22, -92.80000000000001) rectangle (358.32000000000005,-94.4);
\draw(347.57000000000005, -95.0) node[anchor=north west,align=left] {Algebraic\\ \(K\)-theory\\ of spaces};
\draw (347.57000000000005, -95.0) rectangle (350.4200000000001,-96.6);
\draw(358.57000000000005, -87.9) node[anchor=north west,align=left] {\large \(K\)-theory and operator algebras};
\draw (358.57000000000005, -87.9) rectangle (369.9200000000001,-91.80000000000001);
\draw(359.57000000000005, -88.9) node[anchor=north west,align=left] {\(K_0\) as an\\ ordered\\ group, traces};
\draw (359.57000000000005, -88.9) rectangle (363.4200000000001,-90.5);
\draw(363.52000000000004, -88.9) node[anchor=north west,align=left] {Ext and\\ \(K\)-homology};
\draw (363.52000000000004, -88.9) rectangle (366.62000000000006,-90.0);
\draw(366.72, -88.9) node[anchor=north west,align=left] {Kasparov\\  theory\\ (\(KK\)-theory)};
\draw (366.72, -88.9) rectangle (369.82000000000005,-90.5);
\draw(359.57000000000005, -90.60000000000001) node[anchor=north west,align=left] {Index\\ theory};
\draw (359.57000000000005, -90.60000000000001) rectangle (361.6700000000001,-91.7);
\draw(358.57000000000005, -91.9) node[anchor=north west,align=left] {\large Whitehead groups and \(K_1\)};
\draw (358.57000000000005, -91.9) rectangle (369.1700000000001,-96.30000000000001);
\draw(359.57000000000005, -92.9) node[anchor=north west,align=left] {Stable\\ range\\ conditions};
\draw (359.57000000000005, -92.9) rectangle (362.6700000000001,-94.5);
\draw(362.77000000000004, -92.9) node[anchor=north west,align=left] {Stability\\ for linear\\ groups};
\draw (362.77000000000004, -92.9) rectangle (365.87000000000006,-94.5);
\draw(365.97, -92.9) node[anchor=north west,align=left] {\(K_1\) of group\\  rings\\ and orders};
\draw (365.97, -92.9) rectangle (369.07000000000005,-94.5);
\draw(359.57000000000005, -94.60000000000001) node[anchor=north west,align=left] {Congruence\\ subgroup\\ problems};
\draw (359.57000000000005, -94.60000000000001) rectangle (362.6700000000001,-96.2);
\draw(346.57000000000005, -96.80000000000001) node[anchor=north west,align=left] {\large Obstructions from topology};
\draw (346.57000000000005, -96.80000000000001) rectangle (357.47,-102.20000000000002);
\draw(347.57000000000005, -97.80000000000001) node[anchor=north west,align=left] {Obstructions\\  to group\\ actions\\ (\(K\)-theoretic aspects)};
\draw (347.57000000000005, -97.80000000000001) rectangle (353.4200000000001,-99.9);
\draw(353.52000000000004, -97.80000000000001) node[anchor=north west,align=left] {Whitehead\\ (and related)\\ torsion};
\draw (353.52000000000004, -97.80000000000001) rectangle (357.37000000000006,-99.4);
\draw(347.57000000000005, -100.00000000000001) node[anchor=north west,align=left] {Finiteness\\ and other\\ obstructions in \(K_0\)};
\draw (347.57000000000005, -100.00000000000001) rectangle (352.4200000000001,-101.60000000000001);
\draw(352.52000000000004, -100.00000000000001) node[anchor=north west,align=left] {Surgery\\ obstructions\\ (\(K\)-theoretic\\ aspects)};
\draw (352.52000000000004, -100.00000000000001) rectangle (356.12000000000006,-102.10000000000001);
\draw(357.57000000000005, -96.80000000000001) node[anchor=north west,align=left] {\large \(K\)-theory in number theory};
\draw (357.57000000000005, -96.80000000000001) rectangle (368.22,-102.70000000000002);
\draw(358.57000000000005, -97.80000000000001) node[anchor=north west,align=left] {Étale cohomology,\\  higher\\ regulators, zeta and\\  \(L\)-functions\\ (\(K\)-theoretic aspects)};
\draw (358.57000000000005, -97.80000000000001) rectangle (364.4200000000001,-100.4);
\draw(364.52000000000004, -97.80000000000001) node[anchor=north west,align=left] {Symbols and\\ arithmetic\\ (\(K\)-theoretic\\ aspects)};
\draw (364.52000000000004, -97.80000000000001) rectangle (368.12000000000006,-99.9);
\draw(358.57000000000005, -100.50000000000001) node[anchor=north west,align=left] {Generalized\\ class field\\ theory (\(K\)-theoretic\\ aspects)};
\draw (358.57000000000005, -100.50000000000001) rectangle (363.9200000000001,-102.60000000000001);
\draw(346.57000000000005, -102.80000000000001) node[anchor=north west,align=left] {\large Steinberg groups and \(K_2\)};
\draw (346.57000000000005, -102.80000000000001) rectangle (355.85,-107.70000000000002);
\draw(347.57000000000005, -103.80000000000001) node[anchor=north west,align=left] {Symbols,\\ presentations\\  and\\ stability of \(K_2\)};
\draw (347.57000000000005, -103.80000000000001) rectangle (351.6700000000001,-105.9);
\draw(351.77000000000004, -103.80000000000001) node[anchor=north west,align=left] {Central\\ extensions\\ and Schur\\ multipliers};
\draw (351.77000000000004, -103.80000000000001) rectangle (355.12000000000006,-105.9);
\draw(347.57000000000005, -106.00000000000001) node[anchor=north west,align=left] {\(K_2\) and\\ the Brauer\\ group};
\draw (347.57000000000005, -106.00000000000001) rectangle (350.6700000000001,-107.60000000000001);
\draw(350.77000000000004, -106.00000000000001) node[anchor=north west,align=left] {Excision\\ for \(K_2\)};
\draw (350.77000000000004, -106.00000000000001) rectangle (353.37000000000006,-107.10000000000001);
\draw(355.95000000000005, -102.80000000000001) node[anchor=north west,align=left] {\large \(K\)-theory of forms};
\draw (355.95000000000005, -102.80000000000001) rectangle (365.1,-107.70000000000002);
\draw(356.95000000000005, -103.80000000000001) node[anchor=north west,align=left] {Hermitian\\ \(K\)-theory, relations\\  with\\ \(K\)-theory of rings};
\draw (356.95000000000005, -103.80000000000001) rectangle (362.30000000000007,-105.9);
\draw(362.40000000000003, -103.80000000000001) node[anchor=north west,align=left] {Witt\\ groups\\ of rings};
\draw (362.40000000000003, -103.80000000000001) rectangle (365.00000000000006,-105.4);
\draw(356.95000000000005, -106.00000000000001) node[anchor=north west,align=left] {Stability\\ for quadratic\\ modules};
\draw (356.95000000000005, -106.00000000000001) rectangle (360.80000000000007,-107.60000000000001);
\draw(360.90000000000003, -106.00000000000001) node[anchor=north west,align=left] {\(L\)-theory\\  of\\ group rings};
\draw (360.90000000000003, -106.00000000000001) rectangle (364.25000000000006,-107.60000000000001);
\draw(365.20000000000005, -102.80000000000001) node[anchor=north west,align=left] {\large History\\ of \(K\)-theory};
\draw (365.20000000000005, -102.80000000000001) rectangle (369.21000000000004,-103.9);
\draw(346.57000000000005, -107.80000000000001) node[anchor=north west,align=left] {\large Topological \(K\)-theory};
\draw (346.57000000000005, -107.80000000000001) rectangle (355.72,-114.9);
\draw(347.57000000000005, -108.80000000000001) node[anchor=north west,align=left] {\(J\)-homomorphism,\\ Adams\\ operations};
\draw (347.57000000000005, -108.80000000000001) rectangle (351.9200000000001,-110.4);
\draw(352.02000000000004, -108.80000000000001) node[anchor=north west,align=left] {Riemann-Roch\\ theorems,\\ Chern\\ characters};
\draw (352.02000000000004, -108.80000000000001) rectangle (355.62000000000006,-110.9);
\draw(347.57000000000005, -111.00000000000001) node[anchor=north west,align=left] {Geometric\\ applications\\ of topological\\ \(K\)-theory};
\draw (347.57000000000005, -111.00000000000001) rectangle (351.6700000000001,-113.10000000000001);
\draw(351.77000000000004, -111.00000000000001) node[anchor=north west,align=left] {Twisted\\ \(K\)-theory;\\ differential\\ \(K\)-theory};
\draw (351.77000000000004, -111.00000000000001) rectangle (355.37000000000006,-113.10000000000001);
\draw(347.57000000000005, -113.20000000000002) node[anchor=north west,align=left] {Equivariant\\ \(K\)-theory};
\draw (347.57000000000005, -113.20000000000002) rectangle (350.9200000000001,-114.30000000000001);
\draw(351.02000000000004, -113.20000000000002) node[anchor=north west,align=left] {Connective\\ \(K\)-theory,\\ cobordism};
\draw (351.02000000000004, -113.20000000000002) rectangle (354.12000000000006,-114.80000000000001);
\draw(355.82000000000005, -107.80000000000001) node[anchor=north west,align=left] {\large \(K\)-theory in geometry};
\draw (355.82000000000005, -107.80000000000001) rectangle (364.72,-113.70000000000002);
\draw(356.82000000000005, -108.80000000000001) node[anchor=north west,align=left] {Algebraic\\ cycles and\\ motivic cohomology\\ (\(K\)-theoretic\\ aspects)};
\draw (356.82000000000005, -108.80000000000001) rectangle (361.9200000000001,-111.4);
\draw(362.02000000000004, -108.80000000000001) node[anchor=north west,align=left] {\(K\)-theory\\  of\\ schemes};
\draw (362.02000000000004, -108.80000000000001) rectangle (364.62000000000006,-110.4);
\draw(356.82000000000005, -111.50000000000001) node[anchor=north west,align=left] {Relations\\ of \(K\)-theory\\ with cohomology\\ theories};
\draw (356.82000000000005, -111.50000000000001) rectangle (361.1700000000001,-113.60000000000001);
\draw(371.6400000000001, -1) node[anchor=north west,align=left] {\LARGE Combinatorics};
\draw (371.6400000000001, -1) rectangle (394.8400000000001,-49.800000000000004);
\draw(372.6400000000001, -2) node[anchor=north west,align=left] {\large Graph theory};
\draw (372.6400000000001, -2) rectangle (385.2400000000001,-36.1);
\draw(373.6400000000001, -3) node[anchor=north west,align=left] {Isomorphism\\ problems in graph\\ theory (reconstruction\\ conjecture,\\  etc.) and\\ homomorphisms (subgraph\\ embedding, etc.)};
\draw (373.6400000000001, -3) rectangle (379.9900000000001,-6.6);
\draw(380.0900000000001, -3) node[anchor=north west,align=left] {Graphs and\\ abstract\\ algebra (groups,\\  rings,\\ fields, etc.)};
\draw (380.0900000000001, -3) rectangle (384.6900000000001,-5.6);
\draw(380.0900000000001, -5.7) node[anchor=north west,align=left] {Connectivity};
\draw (380.0900000000001, -5.7) rectangle (383.6900000000001,-6.3);
\draw(373.6400000000001, -6.7) node[anchor=north west,align=left] {Graph representations\\ (geometric\\  and\\ intersection\\ representations, etc.)};
\draw (373.6400000000001, -6.7) rectangle (379.7400000000001,-9.3);
\draw(379.8400000000001, -6.7) node[anchor=north west,align=left] {Games on\\ graphs\\ (graph-theoretic\\ aspects)};
\draw (379.8400000000001, -6.7) rectangle (384.4400000000001,-8.8);
\draw(373.6400000000001, -9.4) node[anchor=north west,align=left] {Edge subsets with\\ special properties\\ (factorization,\\  matching,\\ partitioning, covering\\ and packing, etc.)};
\draw (373.6400000000001, -9.4) rectangle (379.7400000000001,-12.5);
\draw(379.8400000000001, -9.4) node[anchor=north west,align=left] {Structural\\ characterization\\ of families\\ of graphs};
\draw (379.8400000000001, -9.4) rectangle (384.4400000000001,-11.5);
\draw(379.8400000000001, -11.6) node[anchor=north west,align=left] {Hypergraphs};
\draw (379.8400000000001, -11.6) rectangle (383.1900000000001,-12.2);
\draw(373.6400000000001, -12.600000000000001) node[anchor=north west,align=left] {Vertex subsets\\  with special\\ properties\\ (dominating sets,\\  independent\\ sets, cliques, etc.)};
\draw (373.6400000000001, -12.600000000000001) rectangle (379.2400000000001,-15.700000000000001);
\draw(379.3400000000001, -12.600000000000001) node[anchor=north west,align=left] {Graph\\ operations (line\\  graphs,\\ products, etc.)};
\draw (379.3400000000001, -12.600000000000001) rectangle (383.9400000000001,-14.700000000000001);
\draw(379.3400000000001, -14.8) node[anchor=north west,align=left] {Trees};
\draw (379.3400000000001, -14.8) rectangle (381.1900000000001,-15.4);
\draw(373.6400000000001, -15.8) node[anchor=north west,align=left] {Graph labelling\\ (graceful\\  graphs,\\ bandwidth, etc.)};
\draw (373.6400000000001, -15.8) rectangle (378.2400000000001,-17.900000000000002);
\draw(378.3400000000001, -15.8) node[anchor=north west,align=left] {Random\\ graphs\\ (graph-theoretic\\ aspects)};
\draw (378.3400000000001, -15.8) rectangle (382.9400000000001,-17.900000000000002);
\draw(383.0400000000001, -15.8) node[anchor=north west,align=left] {Graph\\ minors};
\draw (383.0400000000001, -15.8) rectangle (385.1400000000001,-16.900000000000002);
\draw(373.6400000000001, -18.0) node[anchor=north west,align=left] {Small world\\ graphs, complex\\ networks\\ (graph-theoretic\\ aspects)};
\draw (373.6400000000001, -18.0) rectangle (378.2400000000001,-20.6);
\draw(378.3400000000001, -18.0) node[anchor=north west,align=left] {Graph\\ algorithms\\ (graph-theoretic\\ aspects)};
\draw (378.3400000000001, -18.0) rectangle (382.9400000000001,-20.1);
\draw(373.6400000000001, -20.7) node[anchor=north west,align=left] {Graphical\\ indices (Wiener\\ index, Zagreb\\ index, Randić\\ index, etc.)};
\draw (373.6400000000001, -20.7) rectangle (377.9900000000001,-23.3);
\draw(378.0900000000001, -20.7) node[anchor=north west,align=left] {Planar graphs;\\ geometric\\ and topological\\ aspects\\ of graph theory};
\draw (378.0900000000001, -20.7) rectangle (382.4400000000001,-23.3);
\draw(382.5400000000001, -20.7) node[anchor=north west,align=left] {Distance\\  in\\ graphs};
\draw (382.5400000000001, -20.7) rectangle (385.1400000000001,-22.3);
\draw(373.6400000000001, -23.4) node[anchor=north west,align=left] {Eulerian\\ and Hamiltonian\\ graphs};
\draw (373.6400000000001, -23.4) rectangle (377.9900000000001,-25.0);
\draw(378.0900000000001, -23.4) node[anchor=north west,align=left] {Graphs and\\ linear algebra\\ (matrices,\\ eigenvalues,\\ etc.)};
\draw (378.0900000000001, -23.4) rectangle (382.1900000000001,-26.0);
\draw(382.2900000000001, -23.4) node[anchor=north west,align=left] {Flows\\ in graphs};
\draw (382.2900000000001, -23.4) rectangle (385.1400000000001,-24.5);
\draw(382.2900000000001, -24.6) node[anchor=north west,align=left] {Paths and\\ cycles};
\draw (382.2900000000001, -24.6) rectangle (385.1400000000001,-25.700000000000003);
\draw(373.6400000000001, -26.099999999999998) node[anchor=north west,align=left] {Graph designs\\  and\\ isomorphic\\ decomposition};
\draw (373.6400000000001, -26.099999999999998) rectangle (377.4900000000001,-28.2);
\draw(377.5900000000001, -26.099999999999998) node[anchor=north west,align=left] {Fractional\\  graph\\ theory, fuzzy\\ graph theory};
\draw (377.5900000000001, -26.099999999999998) rectangle (381.4400000000001,-28.2);
\draw(381.5400000000001, -26.099999999999998) node[anchor=north west,align=left] {Signed\\ and weighted\\ graphs};
\draw (381.5400000000001, -26.099999999999998) rectangle (385.1400000000001,-27.7);
\draw(373.6400000000001, -28.299999999999997) node[anchor=north west,align=left] {Applications\\  of\\ graph theory};
\draw (373.6400000000001, -28.299999999999997) rectangle (377.2400000000001,-29.9);
\draw(377.3400000000001, -28.299999999999997) node[anchor=north west,align=left] {Coloring\\ of graphs\\  and\\ hypergraphs};
\draw (377.3400000000001, -28.299999999999997) rectangle (380.6900000000001,-30.4);
\draw(380.7900000000001, -28.299999999999997) node[anchor=north west,align=left] {Directed\\ graphs\\ (digraphs),\\ tournaments};
\draw (380.7900000000001, -28.299999999999997) rectangle (384.1400000000001,-30.4);
\draw(373.6400000000001, -30.499999999999996) node[anchor=north west,align=left] {Enumeration\\ in graph\\ theory};
\draw (373.6400000000001, -30.499999999999996) rectangle (376.9900000000001,-32.099999999999994);
\draw(377.0900000000001, -30.499999999999996) node[anchor=north west,align=left] {Graph\\ polynomials};
\draw (377.0900000000001, -30.499999999999996) rectangle (380.4400000000001,-31.599999999999998);
\draw(380.5400000000001, -30.499999999999996) node[anchor=north west,align=left] {Density\\ (toughness,\\ etc.)};
\draw (380.5400000000001, -30.499999999999996) rectangle (383.8900000000001,-32.099999999999994);
\draw(373.6400000000001, -32.199999999999996) node[anchor=north west,align=left] {Generalized\\ Ramsey\\ theory};
\draw (373.6400000000001, -32.199999999999996) rectangle (376.9900000000001,-33.8);
\draw(377.0900000000001, -32.199999999999996) node[anchor=north west,align=left] {Random\\ walks\\ on graphs};
\draw (377.0900000000001, -32.199999999999996) rectangle (379.9400000000001,-33.8);
\draw(380.0400000000001, -32.199999999999996) node[anchor=north west,align=left] {Extremal\\ problems\\ in graph\\ theory};
\draw (380.0400000000001, -32.199999999999996) rectangle (382.6400000000001,-34.3);
\draw(382.7400000000001, -32.199999999999996) node[anchor=north west,align=left] {Vertex\\ degrees};
\draw (382.7400000000001, -32.199999999999996) rectangle (385.09000000000015,-33.3);
\draw(373.6400000000001, -34.4) node[anchor=north west,align=left] {Expander\\ graphs};
\draw (373.6400000000001, -34.4) rectangle (376.2400000000001,-35.5);
\draw(376.3400000000001, -34.4) node[anchor=north west,align=left] {Infinite\\ graphs};
\draw (376.3400000000001, -34.4) rectangle (378.9400000000001,-35.5);
\draw(379.0400000000001, -34.4) node[anchor=north west,align=left] {Chemical\\ graph\\ theory};
\draw (379.0400000000001, -34.4) rectangle (381.6400000000001,-36.0);
\draw(381.7400000000001, -34.4) node[anchor=north west,align=left] {Perfect\\ graphs};
\draw (381.7400000000001, -34.4) rectangle (384.09000000000015,-35.5);
\draw(385.3400000000001, -2) node[anchor=north west,align=left] {\large Enumerative combinatorics};
\draw (385.3400000000001, -2) rectangle (394.74000000000007,-13.0);
\draw(386.3400000000001, -3) node[anchor=north west,align=left] {Exact enumeration\\ problems,\\ generating\\ functions};
\draw (386.3400000000001, -3) rectangle (391.1900000000001,-5.1);
\draw(391.2900000000001, -3) node[anchor=north west,align=left] {Asymptotic\\ enumeration};
\draw (391.2900000000001, -3) rectangle (394.6400000000001,-4.1);
\draw(386.3400000000001, -5.2) node[anchor=north west,align=left] {\(q\)-calculus\\  and\\ related topics};
\draw (386.3400000000001, -5.2) rectangle (390.4400000000001,-6.800000000000001);
\draw(390.5400000000001, -5.2) node[anchor=north west,align=left] {Permutations,\\ words,\\ matrices};
\draw (390.5400000000001, -5.2) rectangle (394.3900000000001,-6.800000000000001);
\draw(386.3400000000001, -6.9) node[anchor=north west,align=left] {Factorials,\\  binomial\\ coefficients,\\ combinatorial\\ functions};
\draw (386.3400000000001, -6.9) rectangle (390.1900000000001,-9.5);
\draw(390.2900000000001, -6.9) node[anchor=north west,align=left] {Combinatorial\\  aspects\\ of partitions\\ of integers};
\draw (390.2900000000001, -6.9) rectangle (394.1400000000001,-9.0);
\draw(386.3400000000001, -9.600000000000001) node[anchor=north west,align=left] {Combinatorial\\ identities,\\  bijective\\ combinatorics};
\draw (386.3400000000001, -9.600000000000001) rectangle (390.1900000000001,-11.700000000000001);
\draw(390.2900000000001, -9.600000000000001) node[anchor=north west,align=left] {Combinatorial\\ inequalities};
\draw (390.2900000000001, -9.600000000000001) rectangle (394.1400000000001,-10.700000000000001);
\draw(386.3400000000001, -11.8) node[anchor=north west,align=left] {Partitions\\ of sets};
\draw (386.3400000000001, -11.8) rectangle (389.4400000000001,-12.9);
\draw(389.5400000000001, -11.8) node[anchor=north west,align=left] {Umbral\\ calculus};
\draw (389.5400000000001, -11.8) rectangle (392.1400000000001,-12.9);
\draw(385.3400000000001, -13.1) node[anchor=north west,align=left] {\large Extremal combinatorics};
\draw (385.3400000000001, -13.1) rectangle (394.74000000000007,-19.5);
\draw(386.3400000000001, -14.1) node[anchor=north west,align=left] {Probabilistic\\ methods in extremal\\ combinatorics,\\  including\\ polynomial methods\\ (combinatorial\\ Nullstellensatz, etc.)};
\draw (386.3400000000001, -14.1) rectangle (392.4400000000001,-17.7);
\draw(392.5400000000001, -14.1) node[anchor=north west,align=left] {Ramsey\\ theory};
\draw (392.5400000000001, -14.1) rectangle (394.6400000000001,-15.2);
\draw(386.3400000000001, -17.8) node[anchor=north west,align=left] {Transversal\\ (matching)\\ theory};
\draw (386.3400000000001, -17.8) rectangle (389.6900000000001,-19.400000000000002);
\draw(389.7900000000001, -17.8) node[anchor=north west,align=left] {Extremal\\ set\\ theory};
\draw (389.7900000000001, -17.8) rectangle (392.3900000000001,-19.400000000000002);
\draw(385.3400000000001, -19.6) node[anchor=north west,align=left] {\large Algebraic combinatorics};
\draw (385.3400000000001, -19.6) rectangle (394.74000000000007,-31.6);
\draw(386.3400000000001, -20.6) node[anchor=north west,align=left] {Combinatorial\\  aspects\\ of representation\\ theory};
\draw (386.3400000000001, -20.6) rectangle (391.1900000000001,-22.700000000000003);
\draw(386.3400000000001, -22.8) node[anchor=north west,align=left] {Symmetric\\ functions\\  and\\ generalizations};
\draw (386.3400000000001, -22.8) rectangle (390.6900000000001,-24.900000000000002);
\draw(390.7900000000001, -22.8) node[anchor=north west,align=left] {Combinatorial\\ aspects\\ of algebraic\\ geometry};
\draw (390.7900000000001, -22.8) rectangle (394.6400000000001,-24.900000000000002);
\draw(386.3400000000001, -25.0) node[anchor=north west,align=left] {Association\\  schemes,\\ strongly\\ regular graphs};
\draw (386.3400000000001, -25.0) rectangle (390.4400000000001,-27.1);
\draw(390.5400000000001, -25.0) node[anchor=north west,align=left] {Combinatorial\\  aspects\\ of commutative\\ algebra};
\draw (390.5400000000001, -25.0) rectangle (394.6400000000001,-27.1);
\draw(386.3400000000001, -27.200000000000003) node[anchor=north west,align=left] {Combinatorial\\  aspects\\ of groups\\ and algebras};
\draw (386.3400000000001, -27.200000000000003) rectangle (390.1900000000001,-29.300000000000004);
\draw(390.2900000000001, -27.200000000000003) node[anchor=north west,align=left] {Group actions\\  on\\ combinatorial\\ structures};
\draw (390.2900000000001, -27.200000000000003) rectangle (394.1400000000001,-29.300000000000004);
\draw(386.3400000000001, -29.400000000000002) node[anchor=north west,align=left] {Combinatorial\\  aspects\\ of simplicial\\ complexes};
\draw (386.3400000000001, -29.400000000000002) rectangle (390.1900000000001,-31.500000000000004);
\draw(385.3400000000001, -31.700000000000003) node[anchor=north west,align=left] {\large Computational methods\\  for problems\\ pertaining to combinatorics};
\draw (385.3400000000001, -31.700000000000003) rectangle (394.31000000000006,-33.300000000000004);
\draw(385.3400000000001, -33.4) node[anchor=north west,align=left] {\large History of\\ combinatorics};
\draw (385.3400000000001, -33.4) rectangle (389.9700000000001,-34.5);
\draw(372.6400000000001, -36.2) node[anchor=north west,align=left] {\large Designs and configurations};
\draw (372.6400000000001, -36.2) rectangle (383.2900000000001,-49.7);
\draw(373.6400000000001, -37.2) node[anchor=north west,align=left] {Combinatorial\\  aspects of\\ difference sets\\ (number-theoretic,\\ group-theoretic, etc.)};
\draw (373.6400000000001, -37.2) rectangle (379.7400000000001,-39.800000000000004);
\draw(379.8400000000001, -37.2) node[anchor=north west,align=left] {Polyominoes};
\draw (379.8400000000001, -37.2) rectangle (383.1900000000001,-37.800000000000004);
\draw(379.8400000000001, -37.900000000000006) node[anchor=north west,align=left] {Triple\\ systems};
\draw (379.8400000000001, -37.900000000000006) rectangle (382.1900000000001,-39.00000000000001);
\draw(373.6400000000001, -39.900000000000006) node[anchor=north west,align=left] {Combinatorial\\ aspects of\\  matrices\\ (incidence,\\ Hadamard, etc.)};
\draw (373.6400000000001, -39.900000000000006) rectangle (377.9900000000001,-42.50000000000001);
\draw(378.0900000000001, -39.900000000000006) node[anchor=north west,align=left] {Combinatorial\\  aspects\\ of tessellation\\  and\\ tiling problems};
\draw (378.0900000000001, -39.900000000000006) rectangle (382.4400000000001,-42.50000000000001);
\draw(373.6400000000001, -42.6) node[anchor=north west,align=left] {Other\\ designs,\\ configurations};
\draw (373.6400000000001, -42.6) rectangle (377.7400000000001,-44.2);
\draw(377.8400000000001, -42.6) node[anchor=north west,align=left] {Combinatorial\\ aspects\\  of\\ block designs};
\draw (377.8400000000001, -42.6) rectangle (381.6900000000001,-44.7);
\draw(373.6400000000001, -44.800000000000004) node[anchor=north west,align=left] {Orthogonal\\ arrays, Latin\\ squares,\\ Room squares};
\draw (373.6400000000001, -44.800000000000004) rectangle (377.4900000000001,-46.900000000000006);
\draw(377.5900000000001, -44.800000000000004) node[anchor=north west,align=left] {Combinatorial\\ aspects\\ of finite\\ geometries};
\draw (377.5900000000001, -44.800000000000004) rectangle (381.4400000000001,-46.900000000000006);
\draw(373.6400000000001, -47.0) node[anchor=north west,align=left] {Combinatorial\\  aspects\\ of matroids\\ and geometric\\ lattices};
\draw (373.6400000000001, -47.0) rectangle (377.4900000000001,-49.6);
\draw(377.5900000000001, -47.0) node[anchor=north west,align=left] {Combinatorial\\  aspects\\ of packing\\ and covering};
\draw (377.5900000000001, -47.0) rectangle (381.4400000000001,-49.1);
\draw(371.6400000000001, -49.900000000000006) node[anchor=north west,align=left] {\LARGE Linear and multilinear algebra; matrix theory};
\draw (371.6400000000001, -49.900000000000006) rectangle (391.6800000000001,-96.4);
\draw(372.6400000000001, -50.900000000000006) node[anchor=north west,align=left] {\large Basic linear algebra};
\draw (372.6400000000001, -50.900000000000006) rectangle (383.7900000000001,-83.7);
\draw(373.6400000000001, -51.900000000000006) node[anchor=north west,align=left] {Theory of\\ matrix inversion\\  and\\ generalized inverses};
\draw (373.6400000000001, -51.900000000000006) rectangle (379.2400000000001,-54.00000000000001);
\draw(379.3400000000001, -51.900000000000006) node[anchor=north west,align=left] {Vector and\\ tensor\\ algebra, theory\\ of invariants};
\draw (379.3400000000001, -51.900000000000006) rectangle (383.6900000000001,-54.00000000000001);
\draw(373.6400000000001, -54.10000000000001) node[anchor=north west,align=left] {Norms of matrices,\\ numerical\\  range,\\ applications of\\ functional analysis\\ to matrix theory};
\draw (373.6400000000001, -54.10000000000001) rectangle (378.9900000000001,-57.20000000000001);
\draw(379.0900000000001, -54.10000000000001) node[anchor=north west,align=left] {Linear\\ transformations,\\ semilinear\\ transformations};
\draw (379.0900000000001, -54.10000000000001) rectangle (383.6900000000001,-56.20000000000001);
\draw(373.6400000000001, -57.300000000000004) node[anchor=north west,align=left] {Matrix exponential\\  and\\ similar functions\\ of matrices};
\draw (373.6400000000001, -57.300000000000004) rectangle (378.7400000000001,-59.400000000000006);
\draw(378.8400000000001, -57.300000000000004) node[anchor=north west,align=left] {Vector spaces,\\  linear\\ dependence,\\ rank, lineability};
\draw (378.8400000000001, -57.300000000000004) rectangle (383.6900000000001,-59.400000000000006);
\draw(373.6400000000001, -59.50000000000001) node[anchor=north west,align=left] {Linear\\ equations\\ (linear algebraic\\ aspects)};
\draw (373.6400000000001, -59.50000000000001) rectangle (378.4900000000001,-61.60000000000001);
\draw(378.5900000000001, -59.50000000000001) node[anchor=north west,align=left] {Determinants,\\ permanents,\\  traces,\\ other special\\ matrix functions};
\draw (378.5900000000001, -59.50000000000001) rectangle (383.1900000000001,-62.10000000000001);
\draw(373.6400000000001, -62.2) node[anchor=north west,align=left] {Diagonalization,\\ Jordan forms};
\draw (373.6400000000001, -62.2) rectangle (378.2400000000001,-63.300000000000004);
\draw(378.3400000000001, -62.2) node[anchor=north west,align=left] {Inequalities\\  involving\\ eigenvalues\\ and eigenvectors};
\draw (378.3400000000001, -62.2) rectangle (382.9400000000001,-64.3);
\draw(373.6400000000001, -64.4) node[anchor=north west,align=left] {Canonical\\ forms,\\ reductions,\\ classification};
\draw (373.6400000000001, -64.4) rectangle (377.7400000000001,-66.5);
\draw(377.8400000000001, -64.4) node[anchor=north west,align=left] {Matrices over\\ function rings\\  in one or\\ more variables};
\draw (377.8400000000001, -64.4) rectangle (381.9400000000001,-66.5);
\draw(373.6400000000001, -66.60000000000001) node[anchor=north west,align=left] {Factorization\\ of\\ matrices};
\draw (373.6400000000001, -66.60000000000001) rectangle (377.4900000000001,-68.2);
\draw(377.5900000000001, -66.60000000000001) node[anchor=north west,align=left] {Matrix\\ equations and\\ identities};
\draw (377.5900000000001, -66.60000000000001) rectangle (381.4400000000001,-68.2);
\draw(373.6400000000001, -68.30000000000001) node[anchor=north west,align=left] {Commutativity\\ of\\ matrices};
\draw (373.6400000000001, -68.30000000000001) rectangle (377.4900000000001,-69.9);
\draw(377.5900000000001, -68.30000000000001) node[anchor=north west,align=left] {Miscellaneous\\ inequalities\\ involving\\ matrices};
\draw (377.5900000000001, -68.30000000000001) rectangle (381.4400000000001,-70.4);
\draw(373.6400000000001, -70.5) node[anchor=north west,align=left] {Applications\\ of Clifford\\ algebras to\\ physics, etc.};
\draw (373.6400000000001, -70.5) rectangle (377.4900000000001,-72.6);
\draw(377.5900000000001, -70.5) node[anchor=north west,align=left] {Applications\\  of\\ generalized\\ inverses};
\draw (377.5900000000001, -70.5) rectangle (381.1900000000001,-72.6);
\draw(381.2900000000001, -70.5) node[anchor=north west,align=left] {Matrix\\ pencils};
\draw (381.2900000000001, -70.5) rectangle (383.6400000000001,-71.6);
\draw(373.6400000000001, -72.7) node[anchor=north west,align=left] {Conditioning\\  of\\ matrices};
\draw (373.6400000000001, -72.7) rectangle (377.2400000000001,-74.3);
\draw(377.3400000000001, -72.7) node[anchor=north west,align=left] {Eigenvalues,\\  singular\\ values, and\\ eigenvectors};
\draw (377.3400000000001, -72.7) rectangle (380.9400000000001,-74.8);
\draw(373.6400000000001, -74.9) node[anchor=north west,align=left] {Linear\\ inequalities\\ of matrices};
\draw (373.6400000000001, -74.9) rectangle (377.2400000000001,-76.5);
\draw(377.3400000000001, -74.9) node[anchor=north west,align=left] {Quadratic\\ and bilinear\\ forms, inner\\ products};
\draw (377.3400000000001, -74.9) rectangle (380.9400000000001,-77.0);
\draw(373.6400000000001, -77.1) node[anchor=north west,align=left] {Algebraic\\  systems\\ of matrices};
\draw (373.6400000000001, -77.1) rectangle (376.9900000000001,-78.69999999999999);
\draw(377.0900000000001, -77.1) node[anchor=north west,align=left] {Multilinear\\ algebra,\\ tensor\\ calculus};
\draw (377.0900000000001, -77.1) rectangle (380.4400000000001,-79.19999999999999);
\draw(380.5400000000001, -77.1) node[anchor=north west,align=left] {Other\\ algebras\\ built from\\ modules};
\draw (380.5400000000001, -77.1) rectangle (383.6400000000001,-79.19999999999999);
\draw(373.6400000000001, -79.3) node[anchor=north west,align=left] {Max-plus\\ and related\\ algebras};
\draw (373.6400000000001, -79.3) rectangle (376.9900000000001,-80.89999999999999);
\draw(377.0900000000001, -79.3) node[anchor=north west,align=left] {Matrix\\ completion\\ problems};
\draw (377.0900000000001, -79.3) rectangle (380.1900000000001,-80.89999999999999);
\draw(380.2900000000001, -79.3) node[anchor=north west,align=left] {Inverse\\ problems\\ in linear\\ algebra};
\draw (380.2900000000001, -79.3) rectangle (383.1400000000001,-81.39999999999999);
\draw(373.6400000000001, -81.5) node[anchor=north west,align=left] {Clifford\\ algebras,\\ spinors};
\draw (373.6400000000001, -81.5) rectangle (376.4900000000001,-83.1);
\draw(376.5900000000001, -81.5) node[anchor=north west,align=left] {Exterior\\ algebra,\\ Grassmann\\ algebras};
\draw (376.5900000000001, -81.5) rectangle (379.4400000000001,-83.6);
\draw(379.5400000000001, -81.5) node[anchor=north west,align=left] {Linear\\ preserver\\ problems};
\draw (379.5400000000001, -81.5) rectangle (382.3900000000001,-83.1);
\draw(383.8900000000001, -50.900000000000006) node[anchor=north west,align=left] {\large History of\\ linear algebra};
\draw (383.8900000000001, -50.900000000000006) rectangle (388.8300000000001,-52.00000000000001);
\draw(372.6400000000001, -83.80000000000001) node[anchor=north west,align=left] {\large Special matrices};
\draw (372.6400000000001, -83.80000000000001) rectangle (383.7900000000001,-96.30000000000001);
\draw(373.6400000000001, -84.80000000000001) node[anchor=north west,align=left] {Positive\\ matrices and\\ their\\ generalizations; cones\\ of matrices};
\draw (373.6400000000001, -84.80000000000001) rectangle (379.7400000000001,-87.4);
\draw(379.8400000000001, -84.80000000000001) node[anchor=north west,align=left] {Boolean\\ and Hadamard\\ matrices};
\draw (379.8400000000001, -84.80000000000001) rectangle (383.4400000000001,-86.4);
\draw(373.6400000000001, -87.50000000000001) node[anchor=north west,align=left] {Matrices over\\  special\\ rings (quaternions,\\ finite\\ fields, etc.)};
\draw (373.6400000000001, -87.50000000000001) rectangle (378.9900000000001,-90.10000000000001);
\draw(379.0900000000001, -87.50000000000001) node[anchor=north west,align=left] {Hermitian,\\ skew-Hermitian,\\  and\\ related matrices};
\draw (379.0900000000001, -87.50000000000001) rectangle (383.6900000000001,-89.60000000000001);
\draw(373.6400000000001, -90.20000000000002) node[anchor=north west,align=left] {Toeplitz,\\  Cauchy,\\ and related\\ matrices};
\draw (373.6400000000001, -90.20000000000002) rectangle (376.9900000000001,-92.30000000000001);
\draw(377.0900000000001, -90.20000000000002) node[anchor=north west,align=left] {Orthogonal\\ matrices};
\draw (377.0900000000001, -90.20000000000002) rectangle (380.1900000000001,-91.30000000000001);
\draw(380.2900000000001, -90.20000000000002) node[anchor=north west,align=left] {Stochastic\\ matrices};
\draw (380.2900000000001, -90.20000000000002) rectangle (383.3900000000001,-91.30000000000001);
\draw(373.6400000000001, -92.4) node[anchor=north west,align=left] {Random\\ matrices\\ (algebraic\\ aspects)};
\draw (373.6400000000001, -92.4) rectangle (376.7400000000001,-94.5);
\draw(376.8400000000001, -92.4) node[anchor=north west,align=left] {Fuzzy\\ matrices};
\draw (376.8400000000001, -92.4) rectangle (379.4400000000001,-93.5);
\draw(379.5400000000001, -92.4) node[anchor=north west,align=left] {Matrix\\  Lie\\ algebras};
\draw (379.5400000000001, -92.4) rectangle (382.1400000000001,-94.0);
\draw(373.6400000000001, -94.60000000000001) node[anchor=north west,align=left] {Sign\\ pattern\\ matrices};
\draw (373.6400000000001, -94.60000000000001) rectangle (376.2400000000001,-96.2);
\draw(376.3400000000001, -94.60000000000001) node[anchor=north west,align=left] {Matrices\\  of\\ integers};
\draw (376.3400000000001, -94.60000000000001) rectangle (378.9400000000001,-96.2);
\draw(371.6400000000001, -96.5) node[anchor=north west,align=left] {\LARGE General algebraic systems};
\draw (371.6400000000001, -96.5) rectangle (391.1400000000001,-120.2);
\draw(372.6400000000001, -97.5) node[anchor=north west,align=left] {\large Varieties};
\draw (372.6400000000001, -97.5) rectangle (382.7900000000001,-105.1);
\draw(373.6400000000001, -98.5) node[anchor=north west,align=left] {Products,\\ amalgamated products,\\  and other\\ kinds of limits\\ and colimits};
\draw (373.6400000000001, -98.5) rectangle (379.4900000000001,-101.1);
\draw(379.5900000000001, -98.5) node[anchor=north west,align=left] {Equational\\  logic,\\ Mal’tsev\\ conditions};
\draw (379.5900000000001, -98.5) rectangle (382.6900000000001,-100.6);
\draw(373.6400000000001, -101.2) node[anchor=north west,align=left] {Congruence\\ modularity,\\ congruence\\ distributivity};
\draw (373.6400000000001, -101.2) rectangle (377.7400000000001,-103.3);
\draw(377.8400000000001, -101.2) node[anchor=north west,align=left] {Subdirect\\ products and\\  subdirect\\ irreducibility};
\draw (377.8400000000001, -101.2) rectangle (381.9400000000001,-103.3);
\draw(373.6400000000001, -103.4) node[anchor=north west,align=left] {Injectives,\\ projectives};
\draw (373.6400000000001, -103.4) rectangle (376.9900000000001,-104.5);
\draw(377.0900000000001, -103.4) node[anchor=north west,align=left] {Lattices\\  of\\ varieties};
\draw (377.0900000000001, -103.4) rectangle (379.9400000000001,-105.0);
\draw(380.0400000000001, -103.4) node[anchor=north west,align=left] {Free\\ algebras};
\draw (380.0400000000001, -103.4) rectangle (382.6400000000001,-104.5);
\draw(382.8900000000001, -97.5) node[anchor=north west,align=left] {\large History of general\\ algebraic systems};
\draw (382.8900000000001, -97.5) rectangle (389.0700000000001,-98.6);
\draw(372.6400000000001, -105.2) node[anchor=north west,align=left] {\large Algebraic structures};
\draw (372.6400000000001, -105.2) rectangle (381.7900000000001,-120.1);
\draw(373.6400000000001, -106.2) node[anchor=north west,align=left] {Applications\\ of universal\\ algebra in\\ computer science};
\draw (373.6400000000001, -106.2) rectangle (378.2400000000001,-108.3);
\draw(378.3400000000001, -106.2) node[anchor=north west,align=left] {Relational\\ systems,\\ laws of\\ composition};
\draw (378.3400000000001, -106.2) rectangle (381.6900000000001,-108.3);
\draw(373.6400000000001, -108.4) node[anchor=north west,align=left] {Operations and\\ polynomials\\ in algebraic\\  structures,\\ primal algebras};
\draw (373.6400000000001, -108.4) rectangle (377.9900000000001,-111.0);
\draw(378.0900000000001, -108.4) node[anchor=north west,align=left] {Subalgebras,\\ congruence\\ relations};
\draw (378.0900000000001, -108.4) rectangle (381.6900000000001,-110.0);
\draw(373.6400000000001, -111.10000000000001) node[anchor=north west,align=left] {Automorphisms\\  and\\ endomorphisms\\ of algebraic\\ structures};
\draw (373.6400000000001, -111.10000000000001) rectangle (377.4900000000001,-113.7);
\draw(377.5900000000001, -111.10000000000001) node[anchor=north west,align=left] {Word problems\\ (aspects\\ of algebraic\\ structures)};
\draw (377.5900000000001, -111.10000000000001) rectangle (381.4400000000001,-113.2);
\draw(373.6400000000001, -113.80000000000001) node[anchor=north west,align=left] {Heterogeneous\\ algebras};
\draw (373.6400000000001, -113.80000000000001) rectangle (377.4900000000001,-114.9);
\draw(377.5900000000001, -113.80000000000001) node[anchor=north west,align=left] {Equational\\ compactness};
\draw (377.5900000000001, -113.80000000000001) rectangle (380.9400000000001,-114.9);
\draw(373.6400000000001, -115.0) node[anchor=north west,align=left] {Structure\\ theory of\\ algebraic\\ structures};
\draw (373.6400000000001, -115.0) rectangle (376.7400000000001,-117.1);
\draw(376.8400000000001, -115.0) node[anchor=north west,align=left] {Infinitary\\ algebras};
\draw (376.8400000000001, -115.0) rectangle (379.9400000000001,-116.1);
\draw(373.6400000000001, -117.2) node[anchor=north west,align=left] {Fuzzy\\ algebraic\\ structures};
\draw (373.6400000000001, -117.2) rectangle (376.7400000000001,-118.8);
\draw(376.8400000000001, -117.2) node[anchor=north west,align=left] {Partial\\ algebras};
\draw (376.8400000000001, -117.2) rectangle (379.4400000000001,-118.3);
\draw(373.6400000000001, -118.9) node[anchor=north west,align=left] {Unary\\ algebras};
\draw (373.6400000000001, -118.9) rectangle (376.2400000000001,-120.0);
\draw(376.3400000000001, -118.9) node[anchor=north west,align=left] {Finitary\\ algebras};
\draw (376.3400000000001, -118.9) rectangle (378.9400000000001,-120.0);
\draw(381.8900000000001, -105.2) node[anchor=north west,align=left] {\large Other classes of algebras};
\draw (381.8900000000001, -105.2) rectangle (391.0400000000001,-110.10000000000001);
\draw(382.8900000000001, -106.2) node[anchor=north west,align=left] {Quasivarieties};
\draw (382.8900000000001, -106.2) rectangle (386.9900000000001,-106.8);
\draw(387.0900000000001, -106.2) node[anchor=north west,align=left] {Natural\\ dualities for\\  classes\\ of algebras};
\draw (387.0900000000001, -106.2) rectangle (390.9400000000001,-108.3);
\draw(382.8900000000001, -108.4) node[anchor=north west,align=left] {Categories\\  of\\ algebras};
\draw (382.8900000000001, -108.4) rectangle (385.9900000000001,-110.0);
\draw(386.0900000000001, -108.4) node[anchor=north west,align=left] {Axiomatic\\ model\\ classes};
\draw (386.0900000000001, -108.4) rectangle (388.9400000000001,-110.0);
\draw(381.8900000000001, -110.2) node[anchor=north west,align=left] {\large Computational methods for\\ problems pertaining to\\ general algebraic systems};
\draw (381.8900000000001, -110.2) rectangle (390.2400000000001,-111.8);
\draw(371.6400000000001, -120.3) node[anchor=north west,align=left] {\LARGE History and biography};
\draw (371.6400000000001, -120.3) rectangle (388.3400000000001,-144.2);
\draw(372.6400000000001, -121.3) node[anchor=north west,align=left] {\large History of mathematics and mathematicians};
\draw (372.6400000000001, -121.3) rectangle (388.2400000000001,-144.1);
\draw(373.6400000000001, -122.3) node[anchor=north west,align=left] {History of\\ mathematics of the\\ indigenous\\ cultures of Africa,\\ Asia, and Oceania};
\draw (373.6400000000001, -122.3) rectangle (378.9900000000001,-124.89999999999999);
\draw(379.0900000000001, -122.3) node[anchor=north west,align=left] {History of\\ mathematics of the\\ indigenous\\ cultures of Europe\\ (pre-Greek, etc.)};
\draw (379.0900000000001, -122.3) rectangle (384.1900000000001,-124.89999999999999);
\draw(384.2900000000001, -122.3) node[anchor=north west,align=left] {History of\\ mathematics\\ in the Golden\\ Age of Islam};
\draw (384.2900000000001, -122.3) rectangle (388.1400000000001,-124.39999999999999);
\draw(373.6400000000001, -125.0) node[anchor=north west,align=left] {Ethnomathematics,\\ general};
\draw (373.6400000000001, -125.0) rectangle (378.4900000000001,-126.1);
\draw(378.5900000000001, -125.0) node[anchor=north west,align=left] {History of\\ mathematics\\ in late antiquity\\  and\\ medieval Europe};
\draw (378.5900000000001, -125.0) rectangle (383.4400000000001,-127.6);
\draw(373.6400000000001, -126.2) node[anchor=north west,align=left] {Historiography};
\draw (373.6400000000001, -126.2) rectangle (377.7400000000001,-126.8);
\draw(383.5400000000001, -125.0) node[anchor=north west,align=left] {History of\\ mathematics at\\ institutions\\ and academies\\ (non-university)};
\draw (383.5400000000001, -125.0) rectangle (388.1400000000001,-127.6);
\draw(373.6400000000001, -127.7) node[anchor=north west,align=left] {History of\\ mathematics\\ in Paleolithic\\  and\\ Neolithic times};
\draw (373.6400000000001, -127.7) rectangle (377.9900000000001,-130.3);
\draw(378.0900000000001, -127.7) node[anchor=north west,align=left] {History of\\ mathematics\\ in Ancient\\ Greece and Rome};
\draw (378.0900000000001, -127.7) rectangle (382.4400000000001,-129.8);
\draw(382.5400000000001, -127.7) node[anchor=north west,align=left] {History of\\ mathematics in\\ the 15th and\\ 16th centuries,\\ Renaissance};
\draw (382.5400000000001, -127.7) rectangle (386.8900000000001,-130.3);
\draw(373.6400000000001, -130.4) node[anchor=north west,align=left] {Collected or\\ selected works;\\ reprintings\\ or translations\\ of classics};
\draw (373.6400000000001, -130.4) rectangle (377.9900000000001,-133.0);
\draw(378.0900000000001, -130.4) node[anchor=north west,align=left] {History of\\ mathematics of\\ the indigenous\\ cultures of\\ the Americas};
\draw (378.0900000000001, -130.4) rectangle (382.1900000000001,-133.0);
\draw(382.2900000000001, -130.4) node[anchor=north west,align=left] {History\\ of mathematics\\  in\\ Ancient Egypt};
\draw (382.2900000000001, -130.4) rectangle (386.3900000000001,-132.5);
\draw(373.6400000000001, -133.1) node[anchor=north west,align=left] {History of\\ mathematics\\  in\\ Southeast Asia};
\draw (373.6400000000001, -133.1) rectangle (377.7400000000001,-135.2);
\draw(377.8400000000001, -133.1) node[anchor=north west,align=left] {Biographies,\\ obituaries,\\ personalia,\\ bibliographies};
\draw (377.8400000000001, -133.1) rectangle (381.9400000000001,-135.2);
\draw(382.0400000000001, -133.1) node[anchor=north west,align=left] {Sociology\\ (and\\ profession) of\\ mathematics};
\draw (382.0400000000001, -133.1) rectangle (386.1400000000001,-135.2);
\draw(373.6400000000001, -135.3) node[anchor=north west,align=left] {Bibliographic\\ studies};
\draw (373.6400000000001, -135.3) rectangle (377.4900000000001,-136.4);
\draw(377.5900000000001, -135.3) node[anchor=north west,align=left] {General\\ histories,\\ source books};
\draw (377.5900000000001, -135.3) rectangle (381.1900000000001,-136.9);
\draw(381.2900000000001, -135.3) node[anchor=north west,align=left] {History of\\ mathematics\\  in the\\ 17th century};
\draw (381.2900000000001, -135.3) rectangle (384.8900000000001,-137.4);
\draw(373.6400000000001, -137.5) node[anchor=north west,align=left] {History of\\ mathematics\\  in the\\ 18th century};
\draw (373.6400000000001, -137.5) rectangle (377.2400000000001,-139.6);
\draw(377.3400000000001, -137.5) node[anchor=north west,align=left] {History of\\ mathematics\\  in the\\ 19th century};
\draw (377.3400000000001, -137.5) rectangle (380.9400000000001,-139.6);
\draw(381.0400000000001, -137.5) node[anchor=north west,align=left] {History of\\ mathematics\\  in the\\ 20th century};
\draw (381.0400000000001, -137.5) rectangle (384.6400000000001,-139.6);
\draw(384.7400000000001, -137.5) node[anchor=north west,align=left] {History of\\ mathematics\\ in Ancient\\ Babylon};
\draw (384.7400000000001, -137.5) rectangle (388.09000000000015,-139.6);
\draw(373.6400000000001, -139.7) node[anchor=north west,align=left] {History of\\ mathematics\\  in the\\ 21st century};
\draw (373.6400000000001, -139.7) rectangle (377.2400000000001,-141.79999999999998);
\draw(377.3400000000001, -139.7) node[anchor=north west,align=left] {Development\\  of\\ contemporary\\ mathematics};
\draw (377.3400000000001, -139.7) rectangle (380.9400000000001,-141.79999999999998);
\draw(381.0400000000001, -139.7) node[anchor=north west,align=left] {Future\\ perspectives\\  in\\ mathematics};
\draw (381.0400000000001, -139.7) rectangle (384.6400000000001,-141.79999999999998);
\draw(384.7400000000001, -139.7) node[anchor=north west,align=left] {History of\\ mathematics\\ in China};
\draw (384.7400000000001, -139.7) rectangle (388.09000000000015,-141.29999999999998);
\draw(373.6400000000001, -141.9) node[anchor=north west,align=left] {History of\\ mathematics\\ at specific\\ universities};
\draw (373.6400000000001, -141.9) rectangle (377.2400000000001,-144.0);
\draw(377.3400000000001, -141.9) node[anchor=north west,align=left] {History of\\ mathematics\\ in Japan};
\draw (377.3400000000001, -141.9) rectangle (380.6900000000001,-143.5);
\draw(380.7900000000001, -141.9) node[anchor=north west,align=left] {History of\\ mathematics\\ in India};
\draw (380.7900000000001, -141.9) rectangle (384.1400000000001,-143.5);
\draw(384.2400000000001, -141.9) node[anchor=north west,align=left] {Schools\\  of\\ mathematics};
\draw (384.2400000000001, -141.9) rectangle (387.59000000000015,-143.5);
\end{tikzpicture}

\end{document}