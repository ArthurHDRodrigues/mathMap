\documentclass[12pt]{article}
\usepackage[utf8]{inputenc}
\usepackage{pgf,tikz,pgfplots}
\pgfplotsset{compat=1.15}
\usepackage{mathrsfs}
\usetikzlibrary{arrows}
\usepackage{fontspec}
\setmainfont[Renderer=ICU,Mapping=tex-text]{Cousine}
\usepackage{amssymb}
\usepackage[paperwidth=494.97000000000014cm,paperheight=150.0cm,left=0.1cm,right=0.1cm,top=0.1cm,bottom=0.1cm]{geometry}
\begin{document}\begin{tikzpicture}[line cap=round,line join=round,>=triangle 45,x=1cm,y=1cm]
\clip(0, 0)rectangle(490.97000000000014, -146.0);

\draw(0, 0) node[anchor=north west,align=left] {\Huge math};
\draw (0, 0) rectangle (490.97000000000014,-146.0);
\draw(1, -1) node[anchor=north west,align=left] {\LARGE Partial differential equations};
\draw (1, -1) rectangle (76.86999999999999,-52.4);
\draw(2, -2) node[anchor=north west,align=left] {\large General higher-order partial differential equations and systems of higher-order partial differential equations};
\draw (2, -2) rectangle (42.6,-8.4);
\draw(3, -3) node[anchor=north west,align=left] {Initial value\\ problems \\ for linear \\ higher-order PDEs};
\draw (3, -3) rectangle (7.85,-5.1);
\draw(7.949999999999999, -3) node[anchor=north west,align=left] {Boundary value\\ problems for\\ linear \\ higher-order PDEs};
\draw (7.949999999999999, -3) rectangle (12.799999999999999,-5.1);
\draw(12.899999999999999, -3) node[anchor=north west,align=left] {Initial-boundary\\ value \\ problems for\\ linear \\ higher-order PDEs};
\draw (12.899999999999999, -3) rectangle (17.75,-5.6);
\draw(17.849999999999998, -3) node[anchor=north west,align=left] {Initial \\ value problems\\ for \\ nonlinear \\ higher-order PDEs};
\draw (17.849999999999998, -3) rectangle (22.699999999999996,-5.6);
\draw(22.799999999999997, -3) node[anchor=north west,align=left] {Boundary \\ value problems\\ for nonlinear\\ higher-order PDEs};
\draw (22.799999999999997, -3) rectangle (27.65,-5.6);
\draw(27.749999999999996, -3) node[anchor=north west,align=left] {Initial-boundary\\ value \\ problems for\\ nonlinear \\ higher-order PDEs};
\draw (27.749999999999996, -3) rectangle (32.599999999999994,-5.6);
\draw(32.699999999999996, -3) node[anchor=north west,align=left] {Systems\\ of linear\\ higher-order PDEs};
\draw (32.699999999999996, -3) rectangle (37.55,-5.1);
\draw(37.65, -3) node[anchor=north west,align=left] {Initial value\\ problems \\ for systems \\ of linear \\ higher-order PDEs};
\draw (37.65, -3) rectangle (42.5,-5.6);
\draw(3, -5.7) node[anchor=north west,align=left] {Boundary value\\ problems\\ for systems\\ of linear \\ higher-order PDEs};
\draw (3, -5.7) rectangle (7.85,-8.3);
\draw(7.949999999999999, -5.7) node[anchor=north west,align=left] {Initial-boundary\\ value problems\\ for systems\\ of linear \\ higher-order PDEs};
\draw (7.949999999999999, -5.7) rectangle (12.799999999999999,-8.3);
\draw(12.899999999999999, -5.7) node[anchor=north west,align=left] {Systems \\ of nonlinear\\ higher-order PDEs};
\draw (12.899999999999999, -5.7) rectangle (17.75,-7.800000000000001);
\draw(17.849999999999998, -5.7) node[anchor=north west,align=left] {Initial value\\ problems for\\ systems of \\ nonlinear \\ higher-order PDEs};
\draw (17.849999999999998, -5.7) rectangle (22.699999999999996,-8.3);
\draw(22.799999999999997, -5.7) node[anchor=north west,align=left] {Boundary value\\ problems for\\ systems of\\ nonlinear \\ higher-order PDEs};
\draw (22.799999999999997, -5.7) rectangle (27.65,-8.3);
\draw(27.749999999999996, -5.7) node[anchor=north west,align=left] {Initial-boundary\\ value problems\\ for systems\\ of nonlinear\\ higher-order PDEs};
\draw (27.749999999999996, -5.7) rectangle (32.599999999999994,-8.3);
\draw(32.699999999999996, -5.7) node[anchor=north west,align=left] {Linear \\ higher-order\\ PDEs};
\draw (32.699999999999996, -5.7) rectangle (36.3,-7.300000000000001);
\draw(36.4, -5.7) node[anchor=north west,align=left] {Nonlinear\\ higher-order\\ PDEs};
\draw (36.4, -5.7) rectangle (40.0,-7.300000000000001);
\draw(42.7, -2) node[anchor=north west,align=left] {\large Partial differential equations and systems of partial differential equations with constant coefficients};
\draw (42.7, -2) rectangle (75.23,-6.199999999999999);
\draw(43.7, -3) node[anchor=north west,align=left] {Convexity \\ properties of \\ solutions to PDEs\\ and systems of\\ PDEs with \\ constant coefficients};
\draw (43.7, -3) rectangle (49.550000000000004,-6.1);
\draw(49.650000000000006, -3) node[anchor=north west,align=left] {Initial value \\ problems for PDEs\\ and systems \\ of PDEs with \\ constant coefficients};
\draw (49.650000000000006, -3) rectangle (55.50000000000001,-5.6);
\draw(55.6, -3) node[anchor=north west,align=left] {General theory\\ of PDEs and\\ systems of \\ PDEs with \\ constant coefficients};
\draw (55.6, -3) rectangle (61.45,-5.6);
\draw(61.550000000000004, -3) node[anchor=north west,align=left] {Fundamental \\ solutions to PDEs\\ and systems of\\ PDEs with constant\\ coefficients};
\draw (61.550000000000004, -3) rectangle (66.65,-5.6);
\draw(42.7, -6.299999999999999) node[anchor=north west,align=left] {\large History of partial \\ differential equations};
\draw (42.7, -6.299999999999999) rectangle (50.120000000000005,-7.399999999999999);
\draw(2, -8.5) node[anchor=north west,align=left] {\large General first-order partial differential equations and systems of first-order partial differential equations};
\draw (2, -8.5) rectangle (41.6,-14.9);
\draw(3, -9.5) node[anchor=north west,align=left] {Hamilton-Jacobiequations};
\draw (3, -9.5) rectangle (9.6,-11.1);
\draw(9.7, -9.5) node[anchor=north west,align=left] {Initial value\\ problems \\ for linear \\ first-order PDEs};
\draw (9.7, -9.5) rectangle (14.299999999999999,-11.6);
\draw(14.399999999999999, -9.5) node[anchor=north west,align=left] {Boundary value\\ problems\\ for linear \\ first-order PDEs};
\draw (14.399999999999999, -9.5) rectangle (19.0,-11.6);
\draw(19.1, -9.5) node[anchor=north west,align=left] {Initial-boundary\\ value\\ problems \\ for linear \\ first-order PDEs};
\draw (19.1, -9.5) rectangle (23.700000000000003,-12.1);
\draw(23.8, -9.5) node[anchor=north west,align=left] {Initial value\\ problems for\\ nonlinear \\ first-order PDEs};
\draw (23.8, -9.5) rectangle (28.4,-11.6);
\draw(28.5, -9.5) node[anchor=north west,align=left] {Boundary \\ value problems\\ for \\ nonlinear \\ first-order PDEs};
\draw (28.5, -9.5) rectangle (33.1,-12.1);
\draw(33.2, -9.5) node[anchor=north west,align=left] {Initial-boundary\\ value \\ problems for\\ nonlinear \\ first-order PDEs};
\draw (33.2, -9.5) rectangle (37.800000000000004,-12.1);
\draw(37.9, -9.5) node[anchor=north west,align=left] {Systems \\ of nonlinear\\ first-order\\ PDEs};
\draw (37.9, -9.5) rectangle (41.5,-11.6);
\draw(3, -12.2) node[anchor=north west,align=left] {Initial value\\ problems \\ for systems \\ of linear \\ first-order PDEs};
\draw (3, -12.2) rectangle (7.6,-14.799999999999999);
\draw(7.699999999999999, -12.2) node[anchor=north west,align=left] {Boundary value\\ problems\\ for systems\\ of linear \\ first-order PDEs};
\draw (7.699999999999999, -12.2) rectangle (12.299999999999999,-14.799999999999999);
\draw(12.399999999999999, -12.2) node[anchor=north west,align=left] {Initial-boundary\\ value problems\\ for systems\\ of linear \\ first-order PDEs};
\draw (12.399999999999999, -12.2) rectangle (17.0,-14.799999999999999);
\draw(17.099999999999998, -12.2) node[anchor=north west,align=left] {Initial value\\ problems \\ for systems \\ of nonlinear\\ first-order PDEs};
\draw (17.099999999999998, -12.2) rectangle (21.699999999999996,-14.799999999999999);
\draw(21.799999999999997, -12.2) node[anchor=north west,align=left] {Boundary value\\ problems for\\ systems of\\ nonlinear \\ first-order PDEs};
\draw (21.799999999999997, -12.2) rectangle (26.4,-14.799999999999999);
\draw(26.499999999999996, -12.2) node[anchor=north west,align=left] {Initial-boundary\\ value problems\\ for systems\\ of nonlinear\\ first-order PDEs};
\draw (26.499999999999996, -12.2) rectangle (31.099999999999994,-14.799999999999999);
\draw(31.199999999999996, -12.2) node[anchor=north west,align=left] {Linear\\ first-order\\ PDEs};
\draw (31.199999999999996, -12.2) rectangle (34.55,-13.799999999999999);
\draw(34.65, -12.2) node[anchor=north west,align=left] {Nonlinear\\ first-order\\ PDEs};
\draw (34.65, -12.2) rectangle (38.0,-13.799999999999999);
\draw(38.1, -12.2) node[anchor=north west,align=left] {Systems\\ of linear\\ first-order\\ PDEs};
\draw (38.1, -12.2) rectangle (41.45,-14.299999999999999);
\draw(41.7, -8.5) node[anchor=north west,align=left] {\large Qualitative properties of solutions to partial differential equations};
\draw (41.7, -8.5) rectangle (67.7,-21.5);
\draw(42.7, -9.5) node[anchor=north west,align=left] {Oscillation, \\ zeros of solutions,\\ mean value \\ theorems, etc. \\ in context of PDEs};
\draw (42.7, -9.5) rectangle (48.050000000000004,-12.1);
\draw(48.150000000000006, -9.5) node[anchor=north west,align=left] {Critical points\\ of functionals\\ in context of\\ PDEs (e.g., \\ energy functionals)};
\draw (48.150000000000006, -9.5) rectangle (53.50000000000001,-12.1);
\draw(53.6, -9.5) node[anchor=north west,align=left] {Homogenization\\ in context of\\ PDEs; PDEs in\\ media with \\ periodic structure};
\draw (53.6, -9.5) rectangle (58.7,-12.1);
\draw(58.800000000000004, -9.5) node[anchor=north west,align=left] {Dependence of \\ solutions to PDEs\\ on initial \\ and/or boundary \\ data and/or on \\ parameters of PDEs};
\draw (58.800000000000004, -9.5) rectangle (63.900000000000006,-12.6);
\draw(64.0, -9.5) node[anchor=north west,align=left] {Bifurcations\\ in context\\ of PDEs};
\draw (64.0, -9.5) rectangle (67.6,-11.1);
\draw(64.0, -11.2) node[anchor=north west,align=left] {Attractors};
\draw (64.0, -11.2) rectangle (67.1,-12.299999999999999);
\draw(42.7, -12.7) node[anchor=north west,align=left] {Almost and \\ pseudo-almost\\ periodic \\ solutions to PDEs};
\draw (42.7, -12.7) rectangle (47.550000000000004,-14.799999999999999);
\draw(47.650000000000006, -12.7) node[anchor=north west,align=left] {Liouville \\ theorems and \\ Phragmén-Lindelöf\\ theorems in\\ context of PDEs};
\draw (47.650000000000006, -12.7) rectangle (52.50000000000001,-15.299999999999999);
\draw(52.6, -12.7) node[anchor=north west,align=left] {Continuation\\ and prolongation\\ of \\ solutions to PDEs};
\draw (52.6, -12.7) rectangle (57.45,-14.799999999999999);
\draw(57.55, -12.7) node[anchor=north west,align=left] {Smoothness\\ and regularity\\ of \\ solutions to PDEs};
\draw (57.55, -12.7) rectangle (62.4,-14.799999999999999);
\draw(62.5, -12.7) node[anchor=north west,align=left] {Symmetries,\\ invariants,\\ etc. in \\ context of PDEs};
\draw (62.5, -12.7) rectangle (66.85,-14.799999999999999);
\draw(42.7, -15.4) node[anchor=north west,align=left] {Perturbations\\ in \\ context of PDEs};
\draw (42.7, -15.4) rectangle (47.050000000000004,-17.0);
\draw(47.150000000000006, -15.4) node[anchor=north west,align=left] {Singular \\ perturbations\\ in context\\ of PDEs};
\draw (47.150000000000006, -15.4) rectangle (51.00000000000001,-17.5);
\draw(51.1, -15.4) node[anchor=north west,align=left] {Asymptotic\\ behavior\\ of solutions\\ to PDEs};
\draw (51.1, -15.4) rectangle (54.7,-17.5);
\draw(54.800000000000004, -15.4) node[anchor=north west,align=left] {Critical\\ exponents\\ in context\\ of PDEs};
\draw (54.800000000000004, -15.4) rectangle (57.900000000000006,-17.5);
\draw(58.0, -15.4) node[anchor=north west,align=left] {Resonance\\ in context\\ of PDEs};
\draw (58.0, -15.4) rectangle (61.1,-17.0);
\draw(61.2, -15.4) node[anchor=north west,align=left] {Stability\\ in context\\ of PDEs};
\draw (61.2, -15.4) rectangle (64.3,-17.0);
\draw(64.4, -15.4) node[anchor=north west,align=left] {Pattern \\ formations\\ in context\\ of PDEs};
\draw (64.4, -15.4) rectangle (67.5,-17.5);
\draw(42.7, -17.6) node[anchor=north west,align=left] {Blow-up\\ in context\\ of PDEs};
\draw (42.7, -17.6) rectangle (45.800000000000004,-19.200000000000003);
\draw(45.900000000000006, -17.6) node[anchor=north west,align=left] {A priori\\ estimates\\ in context\\ of PDEs};
\draw (45.900000000000006, -17.6) rectangle (49.00000000000001,-19.700000000000003);
\draw(49.1, -17.6) node[anchor=north west,align=left] {Maximum \\ principles\\ in context\\ of PDEs};
\draw (49.1, -17.6) rectangle (52.2,-19.700000000000003);
\draw(52.300000000000004, -17.6) node[anchor=north west,align=left] {Comparison\\ principles\\ in context\\ of PDEs};
\draw (52.300000000000004, -17.6) rectangle (55.400000000000006,-19.700000000000003);
\draw(55.5, -17.6) node[anchor=north west,align=left] {Axially\\ symmetric\\ solutions\\ to PDEs};
\draw (55.5, -17.6) rectangle (58.35,-19.700000000000003);
\draw(58.45, -17.6) node[anchor=north west,align=left] {Entire \\ solutions\\ to PDEs};
\draw (58.45, -17.6) rectangle (61.300000000000004,-19.200000000000003);
\draw(61.400000000000006, -17.6) node[anchor=north west,align=left] {Positive\\ solutions\\ to PDEs};
\draw (61.400000000000006, -17.6) rectangle (64.25,-19.200000000000003);
\draw(64.35, -17.6) node[anchor=north west,align=left] {Periodic\\ solutions\\ to PDEs};
\draw (64.35, -17.6) rectangle (67.19999999999999,-19.200000000000003);
\draw(42.7, -19.8) node[anchor=north west,align=left] {Inertial\\ manifolds};
\draw (42.7, -19.8) rectangle (45.550000000000004,-21.400000000000002);
\draw(2, -15.0) node[anchor=north west,align=left] {\large Representations of solutions to partial differential equations};
\draw (2, -15.0) rectangle (25.450000000000003,-19.9);
\draw(3, -16.0) node[anchor=north west,align=left] {Integral \\ representations\\ of solutions\\ to PDEs};
\draw (3, -16.0) rectangle (7.35,-18.1);
\draw(7.449999999999999, -16.0) node[anchor=north west,align=left] {Trigonometric\\ solutions\\ to PDEs};
\draw (7.449999999999999, -16.0) rectangle (11.299999999999999,-17.6);
\draw(11.4, -16.0) node[anchor=north west,align=left] {Self-similar\\ solutions\\ to PDEs};
\draw (11.4, -16.0) rectangle (15.0,-17.6);
\draw(15.1, -16.0) node[anchor=north west,align=left] {Asymptotic\\ expansions\\ of solutions\\ to PDEs};
\draw (15.1, -16.0) rectangle (18.7,-18.1);
\draw(18.8, -16.0) node[anchor=north west,align=left] {Solutions\\ to PDEs in\\ closed form};
\draw (18.8, -16.0) rectangle (22.150000000000002,-17.6);
\draw(22.25, -16.0) node[anchor=north west,align=left] {Polynomial\\ solutions\\ to PDEs};
\draw (22.25, -16.0) rectangle (25.35,-17.6);
\draw(3, -18.2) node[anchor=north west,align=left] {Traveling\\ wave \\ solutions};
\draw (3, -18.2) rectangle (5.85,-19.8);
\draw(5.95, -18.2) node[anchor=north west,align=left] {Soliton\\ solutions};
\draw (5.95, -18.2) rectangle (8.8,-19.8);
\draw(8.9, -18.2) node[anchor=north west,align=left] {Series \\ solutions\\ to PDEs};
\draw (8.9, -18.2) rectangle (11.75,-19.8);
\draw(67.8, -8.5) node[anchor=north west,align=left] {\large Close-to-elliptic equations};
\draw (67.8, -8.5) rectangle (76.77,-12.9);
\draw(68.8, -9.5) node[anchor=north west,align=left] {Quasiellipticequations};
\draw (68.8, -9.5) rectangle (74.89999999999999,-11.1);
\draw(68.8, -11.2) node[anchor=north west,align=left] {Hypoelliptic\\ equations};
\draw (68.8, -11.2) rectangle (72.39999999999999,-12.799999999999999);
\draw(72.5, -11.2) node[anchor=north west,align=left] {Subelliptic\\ equations};
\draw (72.5, -11.2) rectangle (75.85,-12.799999999999999);
\draw(2, -21.6) node[anchor=north west,align=left] {\large General topics in partial differential equations};
\draw (2, -21.6) rectangle (20.35,-36.1);
\draw(3, -22.6) node[anchor=north west,align=left] {Inequalities \\ applied to PDEs \\ involving derivatives,\\ differential\\ and integral \\ operators, or integrals};
\draw (3, -22.6) rectangle (9.35,-25.700000000000003);
\draw(9.45, -22.6) node[anchor=north west,align=left] {Cauchy-Kovalevskaya\\ theorems};
\draw (9.45, -22.6) rectangle (14.799999999999999,-24.200000000000003);
\draw(14.899999999999999, -22.6) node[anchor=north west,align=left] {Microlocal \\ methods and methods\\ of sheaf \\ theory and \\ homological algebra\\ applied to PDEs};
\draw (14.899999999999999, -22.6) rectangle (20.25,-25.700000000000003);
\draw(3, -25.8) node[anchor=north west,align=left] {Existence problems\\ for PDEs: \\ global existence,\\ local existence,\\ non-existence};
\draw (3, -25.8) rectangle (8.1,-28.400000000000002);
\draw(8.2, -25.8) node[anchor=north west,align=left] {Geometric \\ theory, \\ characteristics, \\ transformations \\ in context of PDEs};
\draw (8.2, -25.8) rectangle (13.299999999999999,-28.400000000000002);
\draw(13.399999999999999, -25.8) node[anchor=north west,align=left] {Uniqueness \\ problems for \\ PDEs: global \\ uniqueness, local\\ uniqueness,\\ non-uniqueness};
\draw (13.399999999999999, -25.8) rectangle (18.25,-28.900000000000002);
\draw(3, -29.0) node[anchor=north west,align=left] {Topological \\ and monotonicity\\ methods \\ applied to PDEs};
\draw (3, -29.0) rectangle (7.6,-31.1);
\draw(7.699999999999999, -29.0) node[anchor=north west,align=left] {Transform \\ methods (e.g.,\\ integral \\ transforms) \\ applied to PDEs};
\draw (7.699999999999999, -29.0) rectangle (12.049999999999999,-31.6);
\draw(12.149999999999999, -29.0) node[anchor=north west,align=left] {Methods of\\ ordinary \\ differential\\ equations \\ applied to PDEs};
\draw (12.149999999999999, -29.0) rectangle (16.5,-31.6);
\draw(16.599999999999998, -29.0) node[anchor=north west,align=left] {Parametrices\\ in context\\ of PDEs};
\draw (16.599999999999998, -29.0) rectangle (20.2,-30.6);
\draw(3, -31.700000000000003) node[anchor=north west,align=left] {Other \\ special methods\\ applied\\ to PDEs};
\draw (3, -31.700000000000003) rectangle (7.35,-33.800000000000004);
\draw(7.449999999999999, -31.700000000000003) node[anchor=north west,align=left] {Theoretical\\ approximation\\ in context\\ of PDEs};
\draw (7.449999999999999, -31.700000000000003) rectangle (11.299999999999999,-33.800000000000004);
\draw(11.4, -31.700000000000003) node[anchor=north west,align=left] {Fundamental\\ solutions\\ to PDEs};
\draw (11.4, -31.700000000000003) rectangle (14.75,-33.300000000000004);
\draw(14.85, -31.700000000000003) node[anchor=north west,align=left] {Variational\\ methods\\ applied\\ to PDEs};
\draw (14.85, -31.700000000000003) rectangle (18.2,-33.800000000000004);
\draw(3, -33.900000000000006) node[anchor=north west,align=left] {Analyticity\\ in context\\ of PDEs};
\draw (3, -33.900000000000006) rectangle (6.35,-35.50000000000001);
\draw(6.449999999999999, -33.900000000000006) node[anchor=north west,align=left] {Singularity\\ in context\\ of PDEs};
\draw (6.449999999999999, -33.900000000000006) rectangle (9.799999999999999,-35.50000000000001);
\draw(9.899999999999999, -33.900000000000006) node[anchor=north west,align=left] {Wave front\\ sets\\ in context\\ of PDEs};
\draw (9.899999999999999, -33.900000000000006) rectangle (12.999999999999998,-36.00000000000001);
\draw(13.1, -33.900000000000006) node[anchor=north west,align=left] {Classical\\ solutions\\ to PDEs};
\draw (13.1, -33.900000000000006) rectangle (15.95,-35.50000000000001);
\draw(20.450000000000003, -21.6) node[anchor=north west,align=left] {\large Generalized solutions to partial differential equations};
\draw (20.450000000000003, -21.6) rectangle (38.10000000000001,-24.3);
\draw(21.450000000000003, -22.6) node[anchor=north west,align=left] {Weak \\ solutions\\ to PDEs};
\draw (21.450000000000003, -22.6) rectangle (24.300000000000004,-24.200000000000003);
\draw(24.400000000000002, -22.6) node[anchor=north west,align=left] {Strong \\ solutions\\ to PDEs};
\draw (24.400000000000002, -22.6) rectangle (27.250000000000004,-24.200000000000003);
\draw(27.35, -22.6) node[anchor=north west,align=left] {Viscosity\\ solutions\\ to PDEs};
\draw (27.35, -22.6) rectangle (30.200000000000003,-24.200000000000003);
\draw(20.450000000000003, -24.400000000000002) node[anchor=north west,align=left] {\large Hyperbolic equations and hyperbolic systems};
\draw (20.450000000000003, -24.400000000000002) rectangle (36.55,-32.0);
\draw(21.450000000000003, -25.400000000000002) node[anchor=north west,align=left] {Initial value\\ problems\\ for first-order\\ hyperbolic equations};
\draw (21.450000000000003, -25.400000000000002) rectangle (27.050000000000004,-28.000000000000004);
\draw(27.150000000000002, -25.400000000000002) node[anchor=north west,align=left] {Initial-boundary\\ value \\ problems for \\ first-order \\ hyperbolic equations};
\draw (27.150000000000002, -25.400000000000002) rectangle (32.75,-28.000000000000004);
\draw(32.85, -25.400000000000002) node[anchor=north west,align=left] {Second-order\\ hyperbolic\\ equations};
\draw (32.85, -25.400000000000002) rectangle (36.45,-27.500000000000004);
\draw(21.450000000000003, -28.1) node[anchor=north west,align=left] {Initial-boundary\\ value \\ problems for \\ second-order \\ hyperbolic equations};
\draw (21.450000000000003, -28.1) rectangle (27.050000000000004,-30.700000000000003);
\draw(27.150000000000002, -28.1) node[anchor=north west,align=left] {Initial value\\ problems \\ for second-order\\ hyperbolic\\ equations};
\draw (27.150000000000002, -28.1) rectangle (31.75,-30.700000000000003);
\draw(31.85, -28.1) node[anchor=north west,align=left] {First-order\\ hyperbolic\\ equations};
\draw (31.85, -28.1) rectangle (35.2,-29.700000000000003);
\draw(21.450000000000003, -30.800000000000004) node[anchor=north west,align=left] {Wave \\ equation};
\draw (21.450000000000003, -30.800000000000004) rectangle (24.050000000000004,-31.900000000000006);
\draw(38.20000000000001, -21.6) node[anchor=north west,align=left] {\large Parabolic equations and parabolic systems};
\draw (38.20000000000001, -21.6) rectangle (53.80000000000001,-49.6);
\draw(39.20000000000001, -22.6) node[anchor=north west,align=left] {Unilateral problems\\ for nonlinear\\ parabolic \\ equations and variational\\ inequalities\\ with nonlinear\\ parabolic operators};
\draw (39.20000000000001, -22.6) rectangle (46.05000000000001,-26.200000000000003);
\draw(46.150000000000006, -22.6) node[anchor=north west,align=left] {Unilateral problems\\ for linear parabolic\\ equations and\\ variational \\ inequalities with linear\\ parabolic operators};
\draw (46.150000000000006, -22.6) rectangle (52.75000000000001,-25.700000000000003);
\draw(39.20000000000001, -26.3) node[anchor=north west,align=left] {Initial value\\ problems\\ for \\ second-order \\ parabolic equations};
\draw (39.20000000000001, -26.3) rectangle (44.55000000000001,-28.900000000000002);
\draw(44.650000000000006, -26.3) node[anchor=north west,align=left] {Initial-boundary\\ value \\ problems for \\ second-order \\ parabolic equations};
\draw (44.650000000000006, -26.3) rectangle (50.00000000000001,-28.900000000000002);
\draw(50.10000000000001, -26.3) node[anchor=north west,align=left] {Second-order\\ parabolic\\ equations};
\draw (50.10000000000001, -26.3) rectangle (53.70000000000001,-27.900000000000002);
\draw(39.20000000000001, -29.0) node[anchor=north west,align=left] {Initial value\\ problems\\ for \\ higher-order \\ parabolic equations};
\draw (39.20000000000001, -29.0) rectangle (44.55000000000001,-31.6);
\draw(44.650000000000006, -29.0) node[anchor=north west,align=left] {Initial-boundary\\ value \\ problems for \\ higher-order \\ parabolic equations};
\draw (44.650000000000006, -29.0) rectangle (50.00000000000001,-31.6);
\draw(50.10000000000001, -29.0) node[anchor=north west,align=left] {Higher-order\\ parabolic\\ equations};
\draw (50.10000000000001, -29.0) rectangle (53.70000000000001,-30.6);
\draw(39.20000000000001, -31.700000000000003) node[anchor=north west,align=left] {Nonlinear initial,\\ boundary and\\ initial-boundary\\ value problems\\ for linear\\ parabolic equations};
\draw (39.20000000000001, -31.700000000000003) rectangle (44.55000000000001,-34.800000000000004);
\draw(44.650000000000006, -31.700000000000003) node[anchor=north west,align=left] {Nonlinear initial,\\ boundary and\\ initial-boundary\\ value problems\\ for nonlinear \\ parabolic equations};
\draw (44.650000000000006, -31.700000000000003) rectangle (50.00000000000001,-34.800000000000004);
\draw(50.10000000000001, -31.700000000000003) node[anchor=north west,align=left] {Second-order\\ parabolic\\ systems};
\draw (50.10000000000001, -31.700000000000003) rectangle (53.70000000000001,-33.300000000000004);
\draw(50.10000000000001, -33.400000000000006) node[anchor=north west,align=left] {Heat \\ equation};
\draw (50.10000000000001, -33.400000000000006) rectangle (52.70000000000001,-34.50000000000001);
\draw(39.20000000000001, -34.900000000000006) node[anchor=north west,align=left] {Unilateral problems\\ for parabolic\\ systems and systems\\ of variational\\ inequalities with\\ parabolic operators};
\draw (39.20000000000001, -34.900000000000006) rectangle (44.55000000000001,-38.00000000000001);
\draw(44.650000000000006, -34.900000000000006) node[anchor=north west,align=left] {Semilinear \\ parabolic equations\\ with Laplacian,\\ bi-Laplacian\\ or poly-Laplacian};
\draw (44.650000000000006, -34.900000000000006) rectangle (50.00000000000001,-37.50000000000001);
\draw(50.10000000000001, -34.900000000000006) node[anchor=north west,align=left] {Higher-order\\ parabolic\\ systems};
\draw (50.10000000000001, -34.900000000000006) rectangle (53.70000000000001,-36.50000000000001);
\draw(39.20000000000001, -38.1) node[anchor=north west,align=left] {Reaction-diffusion\\ equations};
\draw (39.20000000000001, -38.1) rectangle (44.30000000000001,-39.7);
\draw(44.400000000000006, -38.1) node[anchor=north west,align=left] {Initial value\\ problems\\ for \\ second-order \\ parabolic systems};
\draw (44.400000000000006, -38.1) rectangle (49.25000000000001,-40.7);
\draw(49.35000000000001, -38.1) node[anchor=north west,align=left] {Quasilinear \\ parabolic \\ equations with \\ \(p\)-Laplacian};
\draw (49.35000000000001, -38.1) rectangle (53.70000000000001,-40.2);
\draw(39.20000000000001, -40.8) node[anchor=north west,align=left] {Initial value\\ problems\\ for \\ higher-order \\ parabolic systems};
\draw (39.20000000000001, -40.8) rectangle (44.05000000000001,-43.4);
\draw(44.150000000000006, -40.8) node[anchor=north west,align=left] {Initial-boundary\\ value \\ problems for \\ second-order \\ parabolic systems};
\draw (44.150000000000006, -40.8) rectangle (49.00000000000001,-43.4);
\draw(49.10000000000001, -40.8) node[anchor=north west,align=left] {Ultraparabolic\\ equations,\\ pseudoparabolic \\ equations, etc.};
\draw (49.10000000000001, -40.8) rectangle (53.70000000000001,-43.4);
\draw(39.20000000000001, -43.5) node[anchor=north west,align=left] {Initial-boundary\\ value \\ problems for \\ higher-order \\ parabolic systems};
\draw (39.20000000000001, -43.5) rectangle (44.05000000000001,-46.1);
\draw(44.150000000000006, -43.5) node[anchor=north west,align=left] {Quasilinear\\ parabolic \\ equations with\\ mean curvature\\ operator};
\draw (44.150000000000006, -43.5) rectangle (48.25000000000001,-46.1);
\draw(48.35000000000001, -43.5) node[anchor=north west,align=left] {Parabolic\\ Monge-Ampère\\ equations};
\draw (48.35000000000001, -43.5) rectangle (51.95000000000001,-45.1);
\draw(39.20000000000001, -46.2) node[anchor=north west,align=left] {Quasilinear\\ parabolic\\ equations};
\draw (39.20000000000001, -46.2) rectangle (42.55000000000001,-47.800000000000004);
\draw(42.650000000000006, -46.2) node[anchor=north west,align=left] {Semilinear\\ parabolic\\ equations};
\draw (42.650000000000006, -46.2) rectangle (45.75000000000001,-47.800000000000004);
\draw(45.85000000000001, -46.2) node[anchor=north west,align=left] {Degenerate\\ parabolic\\ equations};
\draw (45.85000000000001, -46.2) rectangle (48.95000000000001,-47.800000000000004);
\draw(49.05000000000001, -46.2) node[anchor=north west,align=left] {Nonlinear\\ parabolic\\ equations};
\draw (49.05000000000001, -46.2) rectangle (51.90000000000001,-47.800000000000004);
\draw(39.20000000000001, -47.9) node[anchor=north west,align=left] {Singular\\ parabolic\\ equations};
\draw (39.20000000000001, -47.9) rectangle (42.05000000000001,-49.5);
\draw(42.15000000000001, -47.9) node[anchor=north west,align=left] {Abstract\\ parabolic\\ equations};
\draw (42.15000000000001, -47.9) rectangle (45.000000000000014,-49.5);
\draw(45.10000000000001, -47.9) node[anchor=north west,align=left] {Heat\\ kernel};
\draw (45.10000000000001, -47.9) rectangle (47.20000000000001,-49.0);
\draw(53.900000000000006, -21.6) node[anchor=north west,align=left] {\large Elliptic equations and elliptic systems};
\draw (53.900000000000006, -21.6) rectangle (68.75,-52.3);
\draw(54.900000000000006, -22.6) node[anchor=north west,align=left] {Unilateral problems\\ for linear elliptic\\ equations and\\ variational \\ inequalities with linear\\ elliptic operators};
\draw (54.900000000000006, -22.6) rectangle (61.50000000000001,-25.700000000000003);
\draw(61.60000000000001, -22.6) node[anchor=north west,align=left] {Unilateral problems\\ for nonlinear\\ elliptic equations\\ and variational\\ inequalities\\ with nonlinear\\ elliptic operators};
\draw (61.60000000000001, -22.6) rectangle (66.95,-26.200000000000003);
\draw(54.900000000000006, -26.3) node[anchor=north west,align=left] {Unilateral problems\\ for elliptic \\ systems and systems\\ of variational\\ inequalities with\\ elliptic operators};
\draw (54.900000000000006, -26.3) rectangle (60.25000000000001,-29.400000000000002);
\draw(60.35000000000001, -26.3) node[anchor=north west,align=left] {Boundary \\ value problems\\ for \\ second-order \\ elliptic equations};
\draw (60.35000000000001, -26.3) rectangle (65.45,-28.900000000000002);
\draw(65.55000000000001, -26.3) node[anchor=north west,align=left] {Semilinear\\ elliptic\\ equations};
\draw (65.55000000000001, -26.3) rectangle (68.65,-27.900000000000002);
\draw(54.900000000000006, -29.5) node[anchor=north west,align=left] {Boundary \\ value problems\\ for \\ higher-order \\ elliptic equations};
\draw (54.900000000000006, -29.5) rectangle (60.00000000000001,-32.1);
\draw(60.10000000000001, -29.5) node[anchor=north west,align=left] {Nonlinear \\ boundary value\\ problems for\\ linear \\ elliptic equations};
\draw (60.10000000000001, -29.5) rectangle (65.2,-32.1);
\draw(65.30000000000001, -29.5) node[anchor=north west,align=left] {Schrödinger\\ operator,\\ Schrödinger\\ equation};
\draw (65.30000000000001, -29.5) rectangle (68.65,-31.6);
\draw(54.900000000000006, -32.2) node[anchor=north west,align=left] {Nonlinear \\ boundary value \\ problems for \\ nonlinear \\ elliptic equations};
\draw (54.900000000000006, -32.2) rectangle (60.00000000000001,-34.800000000000004);
\draw(60.10000000000001, -32.2) node[anchor=north west,align=left] {Semilinear \\ elliptic equations\\ with Laplacian,\\ bi-Laplacian \\ or poly-Laplacian};
\draw (60.10000000000001, -32.2) rectangle (65.2,-34.800000000000004);
\draw(65.30000000000001, -32.2) node[anchor=north west,align=left] {First-order\\ elliptic\\ systems};
\draw (65.30000000000001, -32.2) rectangle (68.65,-33.800000000000004);
\draw(54.900000000000006, -34.900000000000006) node[anchor=north west,align=left] {Quasilinear\\ elliptic \\ equations \\ with mean \\ curvature operator};
\draw (54.900000000000006, -34.900000000000006) rectangle (60.00000000000001,-37.50000000000001);
\draw(60.10000000000001, -34.900000000000006) node[anchor=north west,align=left] {Elliptic \\ equations \\ with \\ infinity-Laplacian};
\draw (60.10000000000001, -34.900000000000006) rectangle (65.2,-37.00000000000001);
\draw(65.30000000000001, -34.900000000000006) node[anchor=north west,align=left] {Quasilinear\\ elliptic\\ equations};
\draw (65.30000000000001, -34.900000000000006) rectangle (68.65,-36.50000000000001);
\draw(54.900000000000006, -37.6) node[anchor=north west,align=left] {Laplace operator,\\ Helmholtz \\ equation (reduced\\ wave equation),\\ Poisson equation};
\draw (54.900000000000006, -37.6) rectangle (59.75000000000001,-40.2);
\draw(59.85000000000001, -37.6) node[anchor=north west,align=left] {Boundary \\ value problems\\ for \\ first-order \\ elliptic systems};
\draw (59.85000000000001, -37.6) rectangle (64.45,-40.2);
\draw(64.55000000000001, -37.6) node[anchor=north west,align=left] {Green’s \\ functions for\\ elliptic\\ equations};
\draw (64.55000000000001, -37.6) rectangle (68.4,-39.7);
\draw(54.900000000000006, -40.3) node[anchor=north west,align=left] {Boundary \\ value problems\\ for \\ second-order \\ elliptic systems};
\draw (54.900000000000006, -40.3) rectangle (59.50000000000001,-42.9);
\draw(59.60000000000001, -40.3) node[anchor=north west,align=left] {Boundary \\ value problems\\ for \\ higher-order \\ elliptic systems};
\draw (59.60000000000001, -40.3) rectangle (64.2,-42.9);
\draw(64.30000000000001, -40.3) node[anchor=north west,align=left] {Quasilinear\\ elliptic \\ equations with\\ \(p\)-Laplacian};
\draw (64.30000000000001, -40.3) rectangle (68.65,-42.4);
\draw(54.900000000000006, -43.0) node[anchor=north west,align=left] {Boundary values\\ of solutions\\ to elliptic \\ equations and \\ elliptic systems};
\draw (54.900000000000006, -43.0) rectangle (59.50000000000001,-45.6);
\draw(59.60000000000001, -43.0) node[anchor=north west,align=left] {Second-order\\ elliptic\\ equations};
\draw (59.60000000000001, -43.0) rectangle (63.20000000000001,-44.6);
\draw(63.300000000000004, -43.0) node[anchor=north west,align=left] {Variational\\ methods for\\ second-order\\ elliptic\\ equations};
\draw (63.300000000000004, -43.0) rectangle (66.9,-45.6);
\draw(54.900000000000006, -45.7) node[anchor=north west,align=left] {Higher-order\\ elliptic\\ equations};
\draw (54.900000000000006, -45.7) rectangle (58.50000000000001,-47.300000000000004);
\draw(58.60000000000001, -45.7) node[anchor=north west,align=left] {Variational\\ methods for\\ higher-order\\ elliptic\\ equations};
\draw (58.60000000000001, -45.7) rectangle (62.20000000000001,-48.300000000000004);
\draw(62.300000000000004, -45.7) node[anchor=north west,align=left] {Second-order\\ elliptic\\ systems};
\draw (62.300000000000004, -45.7) rectangle (65.9,-47.300000000000004);
\draw(54.900000000000006, -48.4) node[anchor=north west,align=left] {Higher-order\\ elliptic\\ systems};
\draw (54.900000000000006, -48.4) rectangle (58.50000000000001,-50.0);
\draw(58.60000000000001, -48.4) node[anchor=north west,align=left] {Variational\\ methods\\ for elliptic\\ systems};
\draw (58.60000000000001, -48.4) rectangle (62.20000000000001,-50.5);
\draw(62.300000000000004, -48.4) node[anchor=north west,align=left] {Monge-Ampère\\ equations};
\draw (62.300000000000004, -48.4) rectangle (65.9,-50.0);
\draw(54.900000000000006, -50.599999999999994) node[anchor=north west,align=left] {Degenerate\\ elliptic\\ equations};
\draw (54.900000000000006, -50.599999999999994) rectangle (58.00000000000001,-52.199999999999996);
\draw(58.10000000000001, -50.599999999999994) node[anchor=north west,align=left] {Nonlinear\\ elliptic\\ equations};
\draw (58.10000000000001, -50.599999999999994) rectangle (60.95000000000001,-52.199999999999996);
\draw(61.050000000000004, -50.599999999999994) node[anchor=north west,align=left] {Singular\\ elliptic\\ equations};
\draw (61.050000000000004, -50.599999999999994) rectangle (63.900000000000006,-52.199999999999996);
\draw(76.96999999999998, -1) node[anchor=north west,align=left] {\LARGE Functions of a complex variable};
\draw (76.96999999999998, -1) rectangle (139.61999999999998,-43.0);
\draw(77.96999999999998, -2) node[anchor=north west,align=left] {\large Entire and meromorphic functions of one complex variable, and related topics};
\draw (77.96999999999998, -2) rectangle (105.96999999999998,-9.4);
\draw(78.96999999999998, -3) node[anchor=north west,align=left] {Functional equations\\ in the complex\\ plane, iteration\\ and composition\\ of analytic\\ functions of one\\ complex variable};
\draw (78.96999999999998, -3) rectangle (84.56999999999998,-6.6);
\draw(84.66999999999999, -3) node[anchor=north west,align=left] {Representations\\ of entire\\ functions of\\ one complex \\ variable by \\ series and integrals};
\draw (84.66999999999999, -3) rectangle (90.26999999999998,-6.1);
\draw(90.36999999999998, -3) node[anchor=north west,align=left] {Special classes\\ of entire functions\\ of one complex\\ variable and\\ growth estimates};
\draw (90.36999999999998, -3) rectangle (95.71999999999997,-5.6);
\draw(95.82, -3) node[anchor=north west,align=left] {Value distribution\\ of meromorphic\\ functions\\ of one complex\\ variable, \\ Nevanlinna theory};
\draw (95.82, -3) rectangle (100.91999999999999,-6.1);
\draw(101.01999999999998, -3) node[anchor=north west,align=left] {Cluster \\ sets, prime\\ ends, \\ boundary behavior};
\draw (101.01999999999998, -3) rectangle (105.86999999999998,-5.1);
\draw(78.96999999999998, -6.7) node[anchor=north west,align=left] {Entire \\ functions of one\\ complex \\ variable, \\ general theory};
\draw (78.96999999999998, -6.7) rectangle (83.56999999999998,-9.3);
\draw(83.66999999999999, -6.7) node[anchor=north west,align=left] {Normal \\ functions of one\\ complex \\ variable, \\ normal families};
\draw (83.66999999999999, -6.7) rectangle (88.26999999999998,-9.3);
\draw(88.36999999999998, -6.7) node[anchor=north west,align=left] {Quasi-analytic\\ and other \\ classes of \\ functions of one\\ complex variable};
\draw (88.36999999999998, -6.7) rectangle (92.96999999999997,-9.3);
\draw(93.06999999999998, -6.7) node[anchor=north west,align=left] {Meromorphic\\ functions of\\ one complex\\ variable, \\ general theory};
\draw (93.06999999999998, -6.7) rectangle (97.16999999999997,-9.3);
\draw(106.07, -2) node[anchor=north west,align=left] {\large Series expansions of functions of one complex variable};
\draw (106.07, -2) rectangle (125.86999999999999,-8.4);
\draw(107.07, -3) node[anchor=north west,align=left] {Boundary behavior\\ of power \\ series in one complex\\ variable; \\ over-convergence};
\draw (107.07, -3) rectangle (112.91999999999999,-5.6);
\draw(113.02, -3) node[anchor=north west,align=left] {Completeness \\ problems, closure\\ of a system of\\ functions of \\ one complex variable};
\draw (113.02, -3) rectangle (118.61999999999999,-5.6);
\draw(118.72, -3) node[anchor=north west,align=left] {Dirichlet series,\\ exponential\\ series and other\\ series in one\\ complex variable};
\draw (118.72, -3) rectangle (123.57,-5.6);
\draw(107.07, -5.7) node[anchor=north west,align=left] {Power series\\ (including\\ lacunary \\ series) in one\\ complex variable};
\draw (107.07, -5.7) rectangle (111.66999999999999,-8.3);
\draw(111.77, -5.7) node[anchor=north west,align=left] {Random power\\ series\\ in one \\ complex variable};
\draw (111.77, -5.7) rectangle (116.36999999999999,-7.800000000000001);
\draw(116.47, -5.7) node[anchor=north west,align=left] {Analytic \\ continuation\\ of functions\\ of one \\ complex variable};
\draw (116.47, -5.7) rectangle (121.07,-8.3);
\draw(121.16999999999999, -5.7) node[anchor=north west,align=left] {Continued \\ fractions; \\ complex-analytic\\ aspects};
\draw (121.16999999999999, -5.7) rectangle (125.76999999999998,-7.800000000000001);
\draw(77.96999999999998, -9.5) node[anchor=north west,align=left] {\large Spaces and algebras of analytic functions of one complex variable};
\draw (77.96999999999998, -9.5) rectangle (100.46999999999998,-14.9);
\draw(78.96999999999998, -10.5) node[anchor=north west,align=left] {Spaces of \\ bounded analytic\\ functions\\ of one complex\\ variable};
\draw (78.96999999999998, -10.5) rectangle (83.56999999999998,-13.1);
\draw(83.66999999999999, -10.5) node[anchor=north west,align=left] {Algebras \\ of analytic\\ functions\\ of one \\ complex variable};
\draw (83.66999999999999, -10.5) rectangle (88.26999999999998,-13.1);
\draw(88.36999999999998, -10.5) node[anchor=north west,align=left] {de \\ Branges-Rovnyak\\ spaces};
\draw (88.36999999999998, -10.5) rectangle (92.71999999999997,-12.1);
\draw(92.81999999999998, -10.5) node[anchor=north west,align=left] {Besov spaces\\ and \\ \(Q_p\)-spaces};
\draw (92.81999999999998, -10.5) rectangle (96.91999999999997,-12.1);
\draw(97.01999999999998, -10.5) node[anchor=north west,align=left] {Nevanlinna\\ spaces\\ and Smirnov\\ spaces};
\draw (97.01999999999998, -10.5) rectangle (100.36999999999998,-12.6);
\draw(78.96999999999998, -13.2) node[anchor=north west,align=left] {Bergman \\ spaces and \\ Fock spaces};
\draw (78.96999999999998, -13.2) rectangle (82.31999999999998,-14.799999999999999);
\draw(82.41999999999999, -13.2) node[anchor=north west,align=left] {BMO-spaces};
\draw (82.41999999999999, -13.2) rectangle (85.51999999999998,-14.299999999999999);
\draw(85.61999999999999, -13.2) node[anchor=north west,align=left] {Corona\\ theorems};
\draw (85.61999999999999, -13.2) rectangle (88.21999999999998,-14.299999999999999);
\draw(88.31999999999998, -13.2) node[anchor=north west,align=left] {Zygmund\\ spaces};
\draw (88.31999999999998, -13.2) rectangle (90.66999999999997,-14.299999999999999);
\draw(90.76999999999998, -13.2) node[anchor=north west,align=left] {Hardy\\ spaces};
\draw (90.76999999999998, -13.2) rectangle (92.86999999999998,-14.299999999999999);
\draw(92.96999999999998, -13.2) node[anchor=north west,align=left] {Bloch\\ spaces};
\draw (92.96999999999998, -13.2) rectangle (95.06999999999998,-14.299999999999999);
\draw(100.57, -9.5) node[anchor=north west,align=left] {\large Miscellaneous topics of analysis in the complex plane};
\draw (100.57, -9.5) rectangle (120.36999999999999,-15.9);
\draw(101.57, -10.5) node[anchor=north west,align=left] {Integration, \\ integrals of Cauchy \\ type, integral \\ representations of \\ analytic functions\\ in the complex plane};
\draw (101.57, -10.5) rectangle (107.16999999999999,-13.6);
\draw(107.27, -10.5) node[anchor=north west,align=left] {Moment problems\\ and \\ interpolation \\ problems in the\\ complex plane};
\draw (107.27, -10.5) rectangle (111.61999999999999,-13.1);
\draw(111.72, -10.5) node[anchor=north west,align=left] {Asymptotic\\ representations\\ in the\\ complex plane};
\draw (111.72, -10.5) rectangle (116.07,-12.6);
\draw(116.16999999999999, -10.5) node[anchor=north west,align=left] {Boundary \\ value problems\\ in the \\ complex plane};
\draw (116.16999999999999, -10.5) rectangle (120.26999999999998,-12.6);
\draw(101.57, -13.7) node[anchor=north west,align=left] {Approximation\\ in\\ the \\ complex plane};
\draw (101.57, -13.7) rectangle (105.41999999999999,-15.799999999999999);
\draw(120.46999999999998, -9.5) node[anchor=north west,align=left] {\large History of \\ functions of a \\ complex variable};
\draw (120.46999999999998, -9.5) rectangle (126.02999999999999,-11.1);
\draw(120.46999999999998, -11.2) node[anchor=north west,align=left] {\large Universal holomorphic functions of one complex variable};
\draw (120.46999999999998, -11.2) rectangle (139.51999999999998,-14.399999999999999);
\draw(121.46999999999998, -12.2) node[anchor=north west,align=left] {Universal \\ Taylor series\\ in one \\ complex variable};
\draw (121.46999999999998, -12.2) rectangle (126.06999999999998,-14.299999999999999);
\draw(126.16999999999999, -12.2) node[anchor=north west,align=left] {Universal \\ Dirichlet series\\ in one \\ complex variable};
\draw (126.16999999999999, -12.2) rectangle (130.76999999999998,-14.299999999999999);
\draw(130.86999999999998, -12.2) node[anchor=north west,align=left] {Universal\\ functions\\ of one \\ complex variable};
\draw (130.86999999999998, -12.2) rectangle (135.46999999999997,-14.299999999999999);
\draw(135.57, -12.2) node[anchor=north west,align=left] {Compositional\\ universality};
\draw (135.57, -12.2) rectangle (139.42,-13.799999999999999);
\draw(77.96999999999998, -16.0) node[anchor=north west,align=left] {\large General properties of functions of one complex variable};
\draw (77.96999999999998, -16.0) rectangle (95.61999999999999,-19.7);
\draw(78.96999999999998, -17.0) node[anchor=north west,align=left] {Monogenic \\ and polygenic\\ functions\\ of one \\ complex variable};
\draw (78.96999999999998, -17.0) rectangle (83.56999999999998,-19.6);
\draw(83.66999999999999, -17.0) node[anchor=north west,align=left] {Inequalities\\ in\\ the complex\\ plane};
\draw (83.66999999999999, -17.0) rectangle (87.26999999999998,-19.1);
\draw(95.71999999999998, -16.0) node[anchor=north west,align=left] {\large Geometric function theory};
\draw (95.71999999999998, -16.0) rectangle (109.11999999999998,-37.900000000000006);
\draw(96.71999999999998, -17.0) node[anchor=north west,align=left] {Schwarz-Christoffel-type\\ mappings};
\draw (96.71999999999998, -17.0) rectangle (103.31999999999998,-19.1);
\draw(103.41999999999999, -17.0) node[anchor=north west,align=left] {Maximum principle,\\ Schwarz’s \\ lemma, Lindelöf \\ principle, analogues\\ and generalizations;\\ subordination};
\draw (103.41999999999999, -17.0) rectangle (109.01999999999998,-20.1);
\draw(96.71999999999998, -20.2) node[anchor=north west,align=left] {Extremal problems\\ for conformal\\ and \\ quasiconformal mappings,\\ other methods};
\draw (96.71999999999998, -20.2) rectangle (103.31999999999998,-22.8);
\draw(103.41999999999999, -20.2) node[anchor=north west,align=left] {Extremal problems\\ for conformal\\ and quasiconformal\\ mappings, \\ variational methods};
\draw (103.41999999999999, -20.2) rectangle (108.76999999999998,-22.8);
\draw(96.71999999999998, -22.9) node[anchor=north west,align=left] {Zeros of polynomials,\\ rational functions,\\ and other analytic\\ functions of one\\ complex variable\\ (e.g., zeros of \\ functions with bounded\\ Dirichlet integral)};
\draw (96.71999999999998, -22.9) rectangle (102.81999999999998,-27.0);
\draw(102.91999999999999, -22.9) node[anchor=north west,align=left] {Special classes \\ of univalent and \\ multivalent functions\\ of one complex\\ variable (starlike,\\ convex, bounded\\ rotation, etc.)};
\draw (102.91999999999999, -22.9) rectangle (108.76999999999998,-26.5);
\draw(96.71999999999998, -27.1) node[anchor=north west,align=left] {Coefficient \\ problems for \\ univalent and \\ multivalent functions\\ of one \\ complex variable};
\draw (96.71999999999998, -27.1) rectangle (102.56999999999998,-30.200000000000003);
\draw(102.66999999999999, -27.1) node[anchor=north west,align=left] {Quasiconformal\\ mappings in\\ \(\mathbb{R}^n\),\\ other \\ generalizations};
\draw (102.66999999999999, -27.1) rectangle (107.51999999999998,-29.700000000000003);
\draw(96.71999999999998, -30.3) node[anchor=north west,align=left] {Polynomials\\ and rational\\ functions\\ of one \\ complex variable};
\draw (96.71999999999998, -30.3) rectangle (101.31999999999998,-32.9);
\draw(101.41999999999999, -30.3) node[anchor=north west,align=left] {Kernel \\ functions in one\\ complex \\ variable and\\ applications};
\draw (101.41999999999999, -30.3) rectangle (106.01999999999998,-32.9);
\draw(106.11999999999998, -30.3) node[anchor=north west,align=left] {General \\ theory of\\ conformal\\ mappings};
\draw (106.11999999999998, -30.3) rectangle (108.96999999999997,-32.4);
\draw(96.71999999999998, -33.0) node[anchor=north west,align=left] {General theory\\ of univalent and\\ multivalent \\ functions of one\\ complex variable};
\draw (96.71999999999998, -33.0) rectangle (101.31999999999998,-35.6);
\draw(101.41999999999999, -33.0) node[anchor=north west,align=left] {Covering \\ theorems in \\ conformal \\ mapping theory};
\draw (101.41999999999999, -33.0) rectangle (105.51999999999998,-35.1);
\draw(105.61999999999998, -33.0) node[anchor=north west,align=left] {Conformal\\ mappings\\ of special\\ domains};
\draw (105.61999999999998, -33.0) rectangle (108.71999999999997,-35.1);
\draw(96.71999999999998, -35.7) node[anchor=north west,align=left] {Quasiconformal\\ mappings\\ in the \\ complex plane};
\draw (96.71999999999998, -35.7) rectangle (100.81999999999998,-37.800000000000004);
\draw(100.91999999999999, -35.7) node[anchor=north west,align=left] {Capacity and\\ harmonic \\ measure in the\\ complex plane};
\draw (100.91999999999999, -35.7) rectangle (105.01999999999998,-37.800000000000004);
\draw(77.96999999999998, -19.8) node[anchor=north west,align=left] {\large Riemann surfaces};
\draw (77.96999999999998, -19.8) rectangle (89.61999999999998,-32.3);
\draw(78.96999999999998, -20.8) node[anchor=north west,align=left] {Conformal \\ metrics \\ (hyperbolic, Poincaré,\\ distance\\ functions)};
\draw (78.96999999999998, -20.8) rectangle (85.06999999999998,-23.400000000000002);
\draw(85.16999999999999, -20.8) node[anchor=north west,align=left] {Ideal \\ boundary theory\\ for Riemann\\ surfaces};
\draw (85.16999999999999, -20.8) rectangle (89.51999999999998,-22.900000000000002);
\draw(78.96999999999998, -23.5) node[anchor=north west,align=left] {Fuchsian groups\\ and automorphic\\ functions (aspects\\ of compact \\ Riemann surfaces \\ and uniformization)};
\draw (78.96999999999998, -23.5) rectangle (84.31999999999998,-26.6);
\draw(84.41999999999999, -23.5) node[anchor=north west,align=left] {Kleinian groups\\ (aspects of\\ compact Riemann\\ surfaces and\\ uniformization)};
\draw (84.41999999999999, -23.5) rectangle (88.76999999999998,-26.1);
\draw(78.96999999999998, -26.700000000000003) node[anchor=north west,align=left] {Compact \\ Riemann \\ surfaces and \\ uniformization};
\draw (78.96999999999998, -26.700000000000003) rectangle (83.06999999999998,-28.800000000000004);
\draw(83.16999999999999, -26.700000000000003) node[anchor=north west,align=left] {Classification\\ theory\\ of Riemann\\ surfaces};
\draw (83.16999999999999, -26.700000000000003) rectangle (87.26999999999998,-28.800000000000004);
\draw(78.96999999999998, -28.900000000000002) node[anchor=north west,align=left] {Differentials\\ on\\ Riemann\\ surfaces};
\draw (78.96999999999998, -28.900000000000002) rectangle (82.81999999999998,-31.000000000000004);
\draw(82.91999999999999, -28.900000000000002) node[anchor=north west,align=left] {Teichmüller\\ theory\\ for Riemann\\ surfaces};
\draw (82.91999999999999, -28.900000000000002) rectangle (86.26999999999998,-31.000000000000004);
\draw(86.36999999999999, -28.900000000000002) node[anchor=north west,align=left] {Harmonic\\ functions\\ on Riemann\\ surfaces};
\draw (86.36999999999999, -28.900000000000002) rectangle (89.46999999999998,-31.000000000000004);
\draw(78.96999999999998, -31.1) node[anchor=north west,align=left] {Klein \\ surfaces};
\draw (78.96999999999998, -31.1) rectangle (81.56999999999998,-32.2);
\draw(77.96999999999998, -32.400000000000006) node[anchor=north west,align=left] {\large Computational methods for\\ problems pertaining to \\ functions of a complex variable};
\draw (77.96999999999998, -32.400000000000006) rectangle (88.17999999999998,-34.00000000000001);
\draw(109.21999999999998, -16.0) node[anchor=north west,align=left] {\large Generalized function theory};
\draw (109.21999999999998, -16.0) rectangle (120.61999999999998,-25.6);
\draw(110.21999999999998, -17.0) node[anchor=north west,align=left] {Finely \\ holomorphic functions\\ and \\ topological \\ function theory};
\draw (110.21999999999998, -17.0) rectangle (116.06999999999998,-19.6);
\draw(116.16999999999999, -17.0) node[anchor=north west,align=left] {Non-Archimedean\\ function theory};
\draw (116.16999999999999, -17.0) rectangle (120.51999999999998,-18.6);
\draw(110.21999999999998, -19.7) node[anchor=north west,align=left] {Generalizations\\ of Bers and \\ Vekua type \\ (pseudoanalytic, \\ \(p\)-analytic, etc.)};
\draw (110.21999999999998, -19.7) rectangle (116.06999999999998,-22.3);
\draw(116.16999999999999, -19.7) node[anchor=north west,align=left] {Functions of\\ hypercomplex\\ variables\\ and generalized\\ variables};
\draw (116.16999999999999, -19.7) rectangle (120.51999999999998,-22.3);
\draw(110.21999999999998, -22.4) node[anchor=north west,align=left] {Other \\ generalizations of \\ analytic functions\\ (including\\ abstract-valued\\ functions)};
\draw (110.21999999999998, -22.4) rectangle (115.56999999999998,-25.5);
\draw(115.66999999999999, -22.4) node[anchor=north west,align=left] {Discrete\\ analytic\\ functions};
\draw (115.66999999999999, -22.4) rectangle (118.51999999999998,-24.0);
\draw(77.96999999999998, -34.1) node[anchor=north west,align=left] {\large Analysis on metric spaces};
\draw (77.96999999999998, -34.1) rectangle (87.11999999999998,-39.0);
\draw(78.96999999999998, -35.1) node[anchor=north west,align=left] {Quasiconformal\\ mappings\\ in \\ metric spaces};
\draw (78.96999999999998, -35.1) rectangle (83.06999999999998,-37.2);
\draw(83.16999999999999, -35.1) node[anchor=north west,align=left] {Geometric\\ embeddings\\ of \\ metric spaces};
\draw (83.16999999999999, -35.1) rectangle (87.01999999999998,-37.2);
\draw(78.96999999999998, -37.300000000000004) node[anchor=north west,align=left] {Inequalities\\ in \\ metric spaces};
\draw (78.96999999999998, -37.300000000000004) rectangle (82.81999999999998,-38.900000000000006);
\draw(77.96999999999998, -38.0) node[anchor=north west,align=left] {\large Function theory on the disc};
\draw (77.96999999999998, -38.0) rectangle (86.93999999999998,-42.9);
\draw(78.96999999999998, -39.0) node[anchor=north west,align=left] {Singular inner\\ functions\\ of one complex\\ variable};
\draw (78.96999999999998, -39.0) rectangle (83.06999999999998,-41.1);
\draw(83.16999999999999, -39.0) node[anchor=north west,align=left] {Inner \\ functions of\\ one complex\\ variable};
\draw (83.16999999999999, -39.0) rectangle (86.76999999999998,-41.1);
\draw(78.96999999999998, -41.2) node[anchor=north west,align=left] {Blaschke\\ products};
\draw (78.96999999999998, -41.2) rectangle (81.56999999999998,-42.800000000000004);
\draw(1, -52.5) node[anchor=north west,align=left] {\LARGE Several complex variables and analytic spaces};
\draw (1, -52.5) rectangle (65.14999999999999,-116.0);
\draw(2, -53.5) node[anchor=north west,align=left] {\large Non-Archimedean analysis (should also be assigned at least one other classification number from Section 32-XX describing the type of problem)};
\draw (2, -53.5) rectangle (46.31,-58.2);
\draw(3, -54.5) node[anchor=north west,align=left] {Non-Archimedean \\ analysis (should also be\\ assigned at least \\ one other classification\\ number from Section\\ 32-XX describing\\ the type of problem)};
\draw (3, -54.5) rectangle (9.6,-58.1);
\draw(46.410000000000004, -53.5) node[anchor=north west,align=left] {\large Geometric convexity in several complex variables};
\draw (46.410000000000004, -53.5) rectangle (62.760000000000005,-59.4);
\draw(47.410000000000004, -54.5) node[anchor=north west,align=left] {Analytical \\ consequences of \\ geometric convexity\\ (vanishing\\ theorems, etc.)};
\draw (47.410000000000004, -54.5) rectangle (52.760000000000005,-57.1);
\draw(52.86, -54.5) node[anchor=north west,align=left] {Other notions\\ of convexity\\ in relation\\ to several \\ complex variables};
\draw (52.86, -54.5) rectangle (57.71,-57.1);
\draw(57.81, -54.5) node[anchor=north west,align=left] {Invariant \\ metrics and \\ pseudodistances \\ in several \\ complex variables};
\draw (57.81, -54.5) rectangle (62.660000000000004,-57.1);
\draw(47.410000000000004, -57.2) node[anchor=north west,align=left] {\(q\)-convexity,\\ \(q\)-concavity};
\draw (47.410000000000004, -57.2) rectangle (52.010000000000005,-58.800000000000004);
\draw(52.11, -57.2) node[anchor=north west,align=left] {Finite-type\\ conditions \\ for the boundary\\ of a domain};
\draw (52.11, -57.2) rectangle (56.71,-59.300000000000004);
\draw(56.81, -57.2) node[anchor=north west,align=left] {Topological\\ consequences\\ of geometric\\ convexity};
\draw (56.81, -57.2) rectangle (60.410000000000004,-59.300000000000004);
\draw(2, -59.5) node[anchor=north west,align=left] {\large Holomorphic functions of several complex variables};
\draw (2, -59.5) rectangle (21.85,-84.6);
\draw(3, -60.5) node[anchor=north west,align=left] {Other generalizations\\ of function theory\\ of one complex \\ variable (should also be\\ assigned at least \\ one classification \\ number from Section 30-XX)};
\draw (3, -60.5) rectangle (10.1,-64.1);
\draw(10.2, -60.5) node[anchor=north west,align=left] {Other spaces of \\ holomorphic functions of\\ several complex \\ variables (e.g., bounded\\ mean oscillation \\ (BMOA), vanishing mean\\ oscillation (VMOA))};
\draw (10.2, -60.5) rectangle (16.799999999999997,-64.1);
\draw(16.9, -60.5) node[anchor=north west,align=left] {Hyperfunctions};
\draw (16.9, -60.5) rectangle (21.0,-61.6);
\draw(16.9, -61.7) node[anchor=north west,align=left] {Holomorphic\\ functions of\\ several \\ complex variables};
\draw (16.9, -61.7) rectangle (21.75,-63.800000000000004);
\draw(3, -64.2) node[anchor=north west,align=left] {Integral \\ representations, \\ constructed kernels\\ (e.g., \\ Cauchy, Fantappiè-type\\ kernels)};
\draw (3, -64.2) rectangle (9.1,-67.3);
\draw(9.2, -64.2) node[anchor=north west,align=left] {Normal families \\ of holomorphic \\ functions, mappings\\ of several complex\\ variables, and\\ related topics \\ (taut manifolds etc.)};
\draw (9.2, -64.2) rectangle (15.049999999999999,-67.8);
\draw(15.149999999999999, -64.2) node[anchor=north west,align=left] {Functional analysis\\ techniques \\ applied to functions\\ of several \\ complex variables};
\draw (15.149999999999999, -64.2) rectangle (20.75,-66.8);
\draw(3, -67.9) node[anchor=north west,align=left] {\(H^p\)-spaces,\\ Nevanlinna \\ spaces of functions\\ in several \\ complex variables};
\draw (3, -67.9) rectangle (8.35,-70.5);
\draw(8.45, -67.9) node[anchor=north west,align=left] {Banach algebra\\ techniques applied\\ to functions\\ of several \\ complex variables};
\draw (8.45, -67.9) rectangle (13.549999999999999,-70.5);
\draw(13.649999999999999, -67.9) node[anchor=north west,align=left] {Power series,\\ series of\\ functions of\\ several \\ complex variables};
\draw (13.649999999999999, -67.9) rectangle (18.5,-70.5);
\draw(3, -70.6) node[anchor=north west,align=left] {Polynomials\\ and rational\\ functions \\ of several \\ complex variables};
\draw (3, -70.6) rectangle (7.85,-73.19999999999999);
\draw(7.949999999999999, -70.6) node[anchor=north west,align=left] {Special \\ families of \\ functions of\\ several \\ complex variables};
\draw (7.949999999999999, -70.6) rectangle (12.799999999999999,-73.19999999999999);
\draw(12.899999999999999, -70.6) node[anchor=north west,align=left] {Bloch functions,\\ normal\\ functions of\\ several \\ complex variables};
\draw (12.899999999999999, -70.6) rectangle (17.75,-73.19999999999999);
\draw(3, -73.3) node[anchor=north west,align=left] {Meromorphic\\ functions of\\ several \\ complex variables};
\draw (3, -73.3) rectangle (7.85,-75.39999999999999);
\draw(7.949999999999999, -73.3) node[anchor=north west,align=left] {Nevanlinna \\ theory; growth \\ estimates; other\\ inequalities\\ of several \\ complex variables};
\draw (7.949999999999999, -73.3) rectangle (12.799999999999999,-76.39999999999999);
\draw(12.899999999999999, -73.3) node[anchor=north west,align=left] {Algebras of\\ holomorphic\\ functions of\\ several \\ complex variables};
\draw (12.899999999999999, -73.3) rectangle (17.75,-75.89999999999999);
\draw(3, -76.5) node[anchor=north west,align=left] {Boundary behavior\\ of holomorphic\\ functions\\ of several \\ complex variables};
\draw (3, -76.5) rectangle (7.85,-79.1);
\draw(7.949999999999999, -76.5) node[anchor=north west,align=left] {Zero sets of\\ holomorphic\\ functions \\ of several \\ complex variables};
\draw (7.949999999999999, -76.5) rectangle (12.799999999999999,-79.1);
\draw(12.899999999999999, -76.5) node[anchor=north west,align=left] {Integral \\ representations;\\ canonical \\ kernels (Szegő,\\ Bergman, etc.)};
\draw (12.899999999999999, -76.5) rectangle (17.5,-79.1);
\draw(17.599999999999998, -76.5) node[anchor=north west,align=left] {Residues\\ for several\\ complex\\ variables};
\draw (17.599999999999998, -76.5) rectangle (20.95,-78.6);
\draw(3, -79.2) node[anchor=north west,align=left] {Multifunctions\\ of \\ several complex\\ variables};
\draw (3, -79.2) rectangle (7.35,-81.3);
\draw(7.449999999999999, -79.2) node[anchor=north west,align=left] {Entire \\ functions of \\ several complex\\ variables};
\draw (7.449999999999999, -79.2) rectangle (11.799999999999999,-81.3);
\draw(11.899999999999999, -79.2) node[anchor=north west,align=left] {Bergman \\ spaces of \\ functions in \\ several complex\\ variables};
\draw (11.899999999999999, -79.2) rectangle (16.25,-81.8);
\draw(16.349999999999998, -79.2) node[anchor=north west,align=left] {Harmonic \\ analysis of \\ several complex\\ variables};
\draw (16.349999999999998, -79.2) rectangle (20.699999999999996,-81.3);
\draw(3, -81.9) node[anchor=north west,align=left] {Singular \\ integrals of \\ functions in \\ several complex\\ variables};
\draw (3, -81.9) rectangle (7.35,-84.5);
\draw(21.950000000000003, -59.5) node[anchor=north west,align=left] {\large Differential operators in several variables};
\draw (21.950000000000003, -59.5) rectangle (37.1,-68.1);
\draw(22.950000000000003, -60.5) node[anchor=north west,align=left] {\(\overline\partial_b\)\\ and\\ \(\overline\partial_b\)-Neumann\\ operators};
\draw (22.950000000000003, -60.5) rectangle (31.300000000000004,-63.1);
\draw(31.400000000000002, -60.5) node[anchor=north west,align=left] {Other partial \\ differential \\ equations of complex\\ analysis in\\ several variables};
\draw (31.400000000000002, -60.5) rectangle (37.0,-63.1);
\draw(22.950000000000003, -63.2) node[anchor=north west,align=left] {\(\overline\partial\)\\ and\\ \(\overline\partial\)-Neumann\\ operators};
\draw (22.950000000000003, -63.2) rectangle (30.800000000000004,-65.8);
\draw(30.900000000000002, -63.2) node[anchor=north west,align=left] {Pseudodifferential\\ operators in \\ several complex\\ variables};
\draw (30.900000000000002, -63.2) rectangle (36.0,-65.8);
\draw(22.950000000000003, -65.9) node[anchor=north west,align=left] {Complex \\ Monge-Ampère\\ operators};
\draw (22.950000000000003, -65.9) rectangle (26.550000000000004,-67.5);
\draw(26.650000000000002, -65.9) node[anchor=north west,align=left] {Heat kernels\\ in several\\ complex\\ variables};
\draw (26.650000000000002, -65.9) rectangle (30.250000000000004,-68.0);
\draw(21.950000000000003, -68.2) node[anchor=north west,align=left] {\large Holomorphic mappings and correspondences};
\draw (21.950000000000003, -68.2) rectangle (36.55,-81.5);
\draw(22.950000000000003, -69.2) node[anchor=north west,align=left] {Holomorphic \\ mappings, (holomorphic)\\ embeddings \\ and related questions\\ in several\\ complex variables};
\draw (22.950000000000003, -69.2) rectangle (29.300000000000004,-72.3);
\draw(29.400000000000002, -69.2) node[anchor=north west,align=left] {Proper \\ holomorphic mappings,\\ finiteness\\ theorems};
\draw (29.400000000000002, -69.2) rectangle (35.25,-71.3);
\draw(22.950000000000003, -72.4) node[anchor=north west,align=left] {Iteration of \\ holomorphic maps, \\ fixed points of \\ holomorphic maps\\ and related \\ problems for several\\ complex variables};
\draw (22.950000000000003, -72.4) rectangle (28.550000000000004,-76.0);
\draw(28.650000000000002, -72.4) node[anchor=north west,align=left] {Picard-type \\ theorems and \\ generalizations \\ for several \\ complex variables};
\draw (28.650000000000002, -72.4) rectangle (33.5,-75.0);
\draw(22.950000000000003, -76.10000000000001) node[anchor=north west,align=left] {Value \\ distribution \\ theory in higher\\ dimensions};
\draw (22.950000000000003, -76.10000000000001) rectangle (27.550000000000004,-78.2);
\draw(27.650000000000002, -76.10000000000001) node[anchor=north west,align=left] {Meromorphic\\ mappings in\\ several complex\\ variables};
\draw (27.650000000000002, -76.10000000000001) rectangle (32.0,-78.2);
\draw(32.1, -76.10000000000001) node[anchor=north west,align=left] {Boundary \\ uniqueness of \\ mappings in \\ several complex\\ variables};
\draw (32.1, -76.10000000000001) rectangle (36.45,-78.7);
\draw(22.950000000000003, -78.80000000000001) node[anchor=north west,align=left] {Boundary \\ regularity of \\ mappings in \\ several complex\\ variables};
\draw (22.950000000000003, -78.80000000000001) rectangle (27.300000000000004,-81.4);
\draw(21.950000000000003, -81.6) node[anchor=north west,align=left] {\large Computational methods \\ for problems pertaining\\ to several complex \\ variables and analytic spaces};
\draw (21.950000000000003, -81.6) rectangle (31.540000000000003,-83.69999999999999);
\draw(37.2, -59.5) node[anchor=north west,align=left] {\large Complex spaces with a group of automorphisms};
\draw (37.2, -59.5) rectangle (51.6,-68.6);
\draw(38.2, -60.5) node[anchor=north west,align=left] {Hermitian symmetric\\ spaces, bounded\\ symmetric \\ domains, Jordan \\ algebras (complex-analytic\\ aspects)};
\draw (38.2, -60.5) rectangle (45.300000000000004,-63.6);
\draw(45.400000000000006, -60.5) node[anchor=north west,align=left] {Complex vector\\ fields, \\ holomorphic \\ foliations, \\ \(\mathbb{C}\)-actions};
\draw (45.400000000000006, -60.5) rectangle (51.50000000000001,-63.1);
\draw(38.2, -63.7) node[anchor=north west,align=left] {Automorphism\\ groups of \\ \(\mathbb{C}^n\)\\ and affine\\ manifolds};
\draw (38.2, -63.7) rectangle (42.800000000000004,-66.3);
\draw(42.900000000000006, -63.7) node[anchor=north west,align=left] {Complex Lie\\ groups, group\\ actions on\\ complex spaces};
\draw (42.900000000000006, -63.7) rectangle (47.00000000000001,-65.8);
\draw(47.1, -63.7) node[anchor=north west,align=left] {Automorphism\\ groups\\ of other \\ complex spaces};
\draw (47.1, -63.7) rectangle (51.2,-65.8);
\draw(38.2, -66.4) node[anchor=north west,align=left] {Homogeneous\\ complex\\ manifolds};
\draw (38.2, -66.4) rectangle (41.550000000000004,-68.0);
\draw(41.650000000000006, -66.4) node[anchor=north west,align=left] {Almost \\ homogeneous\\ manifolds\\ and spaces};
\draw (41.650000000000006, -66.4) rectangle (45.00000000000001,-68.5);
\draw(21.950000000000003, -83.80000000000001) node[anchor=north west,align=left] {\large Generalizations of analytic spaces};
\draw (21.950000000000003, -83.80000000000001) rectangle (34.6,-89.70000000000002);
\draw(22.950000000000003, -84.80000000000001) node[anchor=north west,align=left] {Differentiable\\ functions \\ on analytic \\ spaces, \\ differentiable spaces};
\draw (22.950000000000003, -84.80000000000001) rectangle (28.800000000000004,-87.4);
\draw(28.900000000000002, -84.80000000000001) node[anchor=north west,align=left] {Holomorphic\\ maps with \\ infinite-dimensional\\ arguments\\ or values};
\draw (28.900000000000002, -84.80000000000001) rectangle (34.5,-87.4);
\draw(22.950000000000003, -87.50000000000001) node[anchor=north west,align=left] {Formal and\\ graded \\ complex spaces};
\draw (22.950000000000003, -87.50000000000001) rectangle (27.050000000000004,-89.10000000000001);
\draw(27.150000000000002, -87.50000000000001) node[anchor=north west,align=left] {Banach \\ analytic \\ manifolds \\ and spaces};
\draw (27.150000000000002, -87.50000000000001) rectangle (30.250000000000004,-89.60000000000001);
\draw(51.7, -59.5) node[anchor=north west,align=left] {\large Deformations of analytic structures};
\draw (51.7, -59.5) rectangle (65.05,-71.3);
\draw(52.7, -60.5) node[anchor=north west,align=left] {Moduli and \\ deformations for \\ ordinary differential\\ equations (e.g.,\\ Knizhnik-Zamolodchikov\\ equation)};
\draw (52.7, -60.5) rectangle (58.800000000000004,-63.6);
\draw(58.900000000000006, -60.5) node[anchor=north west,align=left] {Moduli of Riemann\\ surfaces, \\ Teichmüller theory\\ (complex-analytic\\ aspects in\\ several variables)};
\draw (58.900000000000006, -60.5) rectangle (64.0,-63.6);
\draw(52.7, -63.7) node[anchor=north west,align=left] {Complex-analytic\\ moduli problems};
\draw (52.7, -63.7) rectangle (57.300000000000004,-65.3);
\draw(57.400000000000006, -63.7) node[anchor=north west,align=left] {Period matrices,\\ variation\\ of Hodge\\ structure; \\ degenerations};
\draw (57.400000000000006, -63.7) rectangle (62.00000000000001,-66.3);
\draw(52.7, -66.4) node[anchor=north west,align=left] {Applications\\ of deformations\\ of analytic\\ structures\\ to the sciences};
\draw (52.7, -66.4) rectangle (57.050000000000004,-69.0);
\draw(57.150000000000006, -66.4) node[anchor=north west,align=left] {Deformations\\ of \\ fiber bundles};
\draw (57.150000000000006, -66.4) rectangle (61.00000000000001,-68.0);
\draw(61.1, -66.4) node[anchor=north west,align=left] {Deformations\\ of \\ submanifolds \\ and subspaces};
\draw (61.1, -66.4) rectangle (64.95,-68.5);
\draw(52.7, -69.1) node[anchor=north west,align=left] {Deformations\\ of\\ complex\\ structures};
\draw (52.7, -69.1) rectangle (56.300000000000004,-71.19999999999999);
\draw(56.400000000000006, -69.1) node[anchor=north west,align=left] {Deformations\\ of special\\ (e.g., CR)\\ structures};
\draw (56.400000000000006, -69.1) rectangle (60.00000000000001,-71.19999999999999);
\draw(21.950000000000003, -89.80000000000001) node[anchor=north west,align=left] {\large History of several\\ complex variables\\ and analytic spaces};
\draw (21.950000000000003, -89.80000000000001) rectangle (28.44,-91.4);
\draw(2, -84.7) node[anchor=north west,align=left] {\large Complex singularities};
\draw (2, -84.7) rectangle (16.099999999999998,-102.9);
\draw(3, -85.7) node[anchor=north west,align=left] {Monodromy; \\ relations with \\ differential equations\\ and \(D\)-modules\\ (complex-analytic aspects)};
\draw (3, -85.7) rectangle (10.1,-88.8);
\draw(10.2, -85.7) node[anchor=north west,align=left] {Mixed Hodge \\ theory of \\ singular varieties\\ (complex-analytic\\ aspects)};
\draw (10.2, -85.7) rectangle (15.299999999999999,-88.3);
\draw(3, -88.9) node[anchor=north west,align=left] {Topological aspects\\ of complex \\ singularities: Lefschetz\\ theorems, \\ topological \\ classification, invariants};
\draw (3, -88.9) rectangle (10.1,-92.0);
\draw(10.2, -88.9) node[anchor=north west,align=left] {Singularities\\ of \\ holomorphic vector\\ fields \\ and foliations};
\draw (10.2, -88.9) rectangle (15.299999999999999,-91.5);
\draw(3, -92.10000000000001) node[anchor=north west,align=left] {Stratifications;\\ constructible\\ sheaves; \\ intersection cohomology\\ (complex-analytic\\ aspects)};
\draw (3, -92.10000000000001) rectangle (9.35,-95.2);
\draw(9.45, -92.10000000000001) node[anchor=north west,align=left] {Modifications;\\ resolution\\ of singularities\\ (complex-analytic\\ aspects)};
\draw (9.45, -92.10000000000001) rectangle (14.299999999999999,-94.7);
\draw(3, -95.30000000000001) node[anchor=north west,align=left] {Equisingularity\\ (topological \\ and analytic)};
\draw (3, -95.30000000000001) rectangle (7.35,-97.4);
\draw(7.449999999999999, -95.30000000000001) node[anchor=north west,align=left] {Relations\\ with \\ arrangements of\\ hyperplanes};
\draw (7.449999999999999, -95.30000000000001) rectangle (11.799999999999999,-97.4);
\draw(11.899999999999999, -95.30000000000001) node[anchor=north west,align=left] {Global theory\\ of complex\\ singularities;\\ cohomological\\ properties};
\draw (11.899999999999999, -95.30000000000001) rectangle (15.999999999999998,-97.9);
\draw(3, -98.0) node[anchor=north west,align=left] {Deformations\\ of complex\\ singularities;\\ vanishing\\ cycles};
\draw (3, -98.0) rectangle (7.1,-100.6);
\draw(7.199999999999999, -98.0) node[anchor=north west,align=left] {Milnor \\ fibration; \\ relations with\\ knot theory};
\draw (7.199999999999999, -98.0) rectangle (11.299999999999999,-100.1);
\draw(11.399999999999999, -98.0) node[anchor=north west,align=left] {Local \\ complex \\ singularities};
\draw (11.399999999999999, -98.0) rectangle (15.249999999999998,-99.6);
\draw(3, -100.7) node[anchor=north west,align=left] {Complex \\ surface and \\ hypersurface\\ singularities};
\draw (3, -100.7) rectangle (6.85,-102.8);
\draw(6.949999999999999, -100.7) node[anchor=north west,align=left] {Other \\ operations on\\ complex \\ singularities};
\draw (6.949999999999999, -100.7) rectangle (10.799999999999999,-102.8);
\draw(10.9, -100.7) node[anchor=north west,align=left] {Invariants\\ of \\ analytic \\ local rings};
\draw (10.9, -100.7) rectangle (14.25,-102.8);
\draw(16.199999999999996, -84.7) node[anchor=north west,align=left] {\large Complex manifolds};
\draw (16.199999999999996, -84.7) rectangle (28.849999999999994,-99.9);
\draw(17.199999999999996, -85.7) node[anchor=north west,align=left] {Kähler-Einsteinmanifolds};
\draw (17.199999999999996, -85.7) rectangle (23.799999999999997,-87.3);
\draw(23.899999999999995, -85.7) node[anchor=north west,align=left] {Calabi-Yau\\ theory \\ (complex-analytic\\ aspects)};
\draw (23.899999999999995, -85.7) rectangle (28.749999999999993,-87.8);
\draw(17.199999999999996, -87.9) node[anchor=north west,align=left] {Special domains\\ (Reinhardt, \\ Hartogs, circular, \\ tube, etc.) in \\ \(\mathbb{C}^n\) \\ and complex manifolds};
\draw (17.199999999999996, -87.9) rectangle (23.049999999999997,-91.0);
\draw(23.149999999999995, -87.9) node[anchor=north west,align=left] {Pseudoholomorphic\\ curves};
\draw (23.149999999999995, -87.9) rectangle (27.999999999999993,-89.5);
\draw(23.149999999999995, -89.60000000000001) node[anchor=north west,align=left] {Kähler \\ manifolds};
\draw (23.149999999999995, -89.60000000000001) rectangle (25.999999999999996,-90.7);
\draw(17.199999999999996, -91.10000000000001) node[anchor=north west,align=left] {Complex \\ manifolds as \\ subdomains of \\ Euclidean space};
\draw (17.199999999999996, -91.10000000000001) rectangle (21.549999999999997,-93.2);
\draw(21.649999999999995, -91.10000000000001) node[anchor=north west,align=left] {Uniformization\\ of complex\\ manifolds};
\draw (21.649999999999995, -91.10000000000001) rectangle (25.749999999999993,-93.2);
\draw(25.849999999999994, -91.10000000000001) node[anchor=north west,align=left] {Negative\\ curvature\\ complex\\ manifolds};
\draw (25.849999999999994, -91.10000000000001) rectangle (28.699999999999996,-93.2);
\draw(17.199999999999996, -93.30000000000001) node[anchor=north west,align=left] {Classification\\ theorems\\ for complex\\ manifolds};
\draw (17.199999999999996, -93.30000000000001) rectangle (21.299999999999997,-95.4);
\draw(21.399999999999995, -93.30000000000001) node[anchor=north west,align=left] {Hyperbolic\\ and Kobayashi\\ hyperbolic\\ manifolds};
\draw (21.399999999999995, -93.30000000000001) rectangle (25.249999999999996,-95.4);
\draw(25.349999999999994, -93.30000000000001) node[anchor=north west,align=left] {Notions of\\ stability\\ for complex\\ manifolds};
\draw (25.349999999999994, -93.30000000000001) rectangle (28.699999999999996,-95.4);
\draw(17.199999999999996, -95.5) node[anchor=north west,align=left] {Oka principle\\ and Oka\\ manifolds};
\draw (17.199999999999996, -95.5) rectangle (21.049999999999997,-97.1);
\draw(21.149999999999995, -95.5) node[anchor=north west,align=left] {Embedding\\ theorems \\ for complex\\ manifolds};
\draw (21.149999999999995, -95.5) rectangle (24.499999999999996,-97.6);
\draw(24.599999999999994, -95.5) node[anchor=north west,align=left] {Topological\\ aspects\\ of complex\\ manifolds};
\draw (24.599999999999994, -95.5) rectangle (27.949999999999996,-97.6);
\draw(17.199999999999996, -97.7) node[anchor=north west,align=left] {Positive\\ curvature\\ complex\\ manifolds};
\draw (17.199999999999996, -97.7) rectangle (20.049999999999997,-99.8);
\draw(20.149999999999995, -97.7) node[anchor=north west,align=left] {Stein \\ manifolds};
\draw (20.149999999999995, -97.7) rectangle (22.999999999999996,-98.8);
\draw(23.099999999999994, -97.7) node[anchor=north west,align=left] {Almost \\ complex \\ manifolds};
\draw (23.099999999999994, -97.7) rectangle (25.949999999999996,-99.3);
\draw(28.949999999999996, -84.7) node[anchor=north west,align=left] {\large Holomorphic convexity};
\draw (28.949999999999996, -84.7) rectangle (41.349999999999994,-94.8);
\draw(29.949999999999996, -85.7) node[anchor=north west,align=left] {Holomorphic, \\ polynomial and rational\\ approximation, and\\ interpolation in\\ several complex \\ variables; Runge pairs};
\draw (29.949999999999996, -85.7) rectangle (36.3,-88.8);
\draw(36.39999999999999, -85.7) node[anchor=north west,align=left] {Holomorphically\\ convex\\ complex \\ spaces, reduction\\ theory};
\draw (36.39999999999999, -85.7) rectangle (41.24999999999999,-88.3);
\draw(29.949999999999996, -88.9) node[anchor=north west,align=left] {Polynomial \\ convexity, rational\\ convexity, \\ meromorphic convexity\\ in several\\ complex variables};
\draw (29.949999999999996, -88.9) rectangle (35.8,-92.0);
\draw(35.89999999999999, -88.9) node[anchor=north west,align=left] {Stein \\ spaces, Stein\\ manifolds};
\draw (35.89999999999999, -88.9) rectangle (39.74999999999999,-90.5);
\draw(35.89999999999999, -90.60000000000001) node[anchor=north west,align=left] {The Levi\\ problem};
\draw (35.89999999999999, -90.60000000000001) rectangle (38.49999999999999,-91.7);
\draw(29.949999999999996, -92.10000000000001) node[anchor=north west,align=left] {Global boundary\\ behavior of \\ holomorphic functions\\ of several\\ complex variables};
\draw (29.949999999999996, -92.10000000000001) rectangle (35.8,-94.7);
\draw(41.449999999999996, -84.7) node[anchor=north west,align=left] {\large Local analytic geometry};
\draw (41.449999999999996, -84.7) rectangle (52.849999999999994,-93.3);
\draw(42.449999999999996, -85.7) node[anchor=north west,align=left] {Triangulation and\\ topological \\ properties of \\ semi-analytic \\ andsubanalytic sets, \\ and related questions};
\draw (42.449999999999996, -85.7) rectangle (48.3,-88.8);
\draw(48.39999999999999, -85.7) node[anchor=north west,align=left] {Germs of \\ analytic sets,\\ local \\ parametrization};
\draw (48.39999999999999, -85.7) rectangle (52.74999999999999,-87.8);
\draw(42.449999999999996, -88.9) node[anchor=north west,align=left] {Analytic \\ algebras and\\ generalizations,\\ preparation theorems};
\draw (42.449999999999996, -88.9) rectangle (48.05,-91.5);
\draw(48.14999999999999, -88.9) node[anchor=north west,align=left] {Semi-analytic\\ sets, \\ subanalytic \\ sets, and \\ generalizations};
\draw (48.14999999999999, -88.9) rectangle (52.49999999999999,-91.5);
\draw(42.449999999999996, -91.60000000000001) node[anchor=north west,align=left] {Analytic \\ subsets of\\ affine space};
\draw (42.449999999999996, -91.60000000000001) rectangle (46.05,-93.2);
\draw(52.949999999999996, -84.7) node[anchor=north west,align=left] {\large Analytic spaces};
\draw (52.949999999999996, -84.7) rectangle (63.099999999999994,-103.30000000000001);
\draw(53.949999999999996, -85.7) node[anchor=north west,align=left] {The Levi \\ problem in complex\\ spaces;\\ generalizations};
\draw (53.949999999999996, -85.7) rectangle (59.05,-87.8);
\draw(59.14999999999999, -85.7) node[anchor=north west,align=left] {Real-analytic\\ manifolds,\\ real-analytic\\ spaces};
\draw (59.14999999999999, -85.7) rectangle (62.99999999999999,-87.8);
\draw(53.949999999999996, -87.9) node[anchor=north west,align=left] {Applications\\ of analytic \\ spaces to physics\\ and other\\ areas of science};
\draw (53.949999999999996, -87.9) rectangle (58.8,-90.5);
\draw(58.89999999999999, -87.9) node[anchor=north west,align=left] {Real-analytic\\ sets,\\ complex \\ Nash functions};
\draw (58.89999999999999, -87.9) rectangle (62.99999999999999,-90.0);
\draw(53.949999999999996, -90.60000000000001) node[anchor=north west,align=left] {Integration\\ on analytic\\ sets and \\ spaces, currents};
\draw (53.949999999999996, -90.60000000000001) rectangle (58.55,-92.7);
\draw(58.64999999999999, -90.60000000000001) node[anchor=north west,align=left] {Local \\ cohomology\\ of \\ analytic spaces};
\draw (58.64999999999999, -90.60000000000001) rectangle (62.99999999999999,-92.7);
\draw(53.949999999999996, -92.80000000000001) node[anchor=north west,align=left] {Analytic\\ sheaves \\ and cohomology\\ groups};
\draw (53.949999999999996, -92.80000000000001) rectangle (58.05,-94.9);
\draw(58.14999999999999, -92.80000000000001) node[anchor=north west,align=left] {Sheaves of \\ differential \\ operators and \\ their modules,\\ \(D\)-modules};
\draw (58.14999999999999, -92.80000000000001) rectangle (62.24999999999999,-95.4);
\draw(53.949999999999996, -95.5) node[anchor=north west,align=left] {Embedding\\ of \\ real-analytic\\ manifolds};
\draw (53.949999999999996, -95.5) rectangle (57.8,-97.6);
\draw(57.89999999999999, -95.5) node[anchor=north west,align=left] {Complex\\ supergeometry};
\draw (57.89999999999999, -95.5) rectangle (61.74999999999999,-97.1);
\draw(53.949999999999996, -97.7) node[anchor=north west,align=left] {Analytic\\ subsets\\ and \\ submanifolds};
\draw (53.949999999999996, -97.7) rectangle (57.55,-99.8);
\draw(57.64999999999999, -97.7) node[anchor=north west,align=left] {Duality \\ theorems \\ for analytic\\ spaces};
\draw (57.64999999999999, -97.7) rectangle (61.24999999999999,-99.8);
\draw(53.949999999999996, -99.9) node[anchor=north west,align=left] {Topology\\ of analytic\\ spaces};
\draw (53.949999999999996, -99.9) rectangle (57.3,-101.5);
\draw(57.39999999999999, -99.9) node[anchor=north west,align=left] {Embedding\\ of analytic\\ spaces};
\draw (57.39999999999999, -99.9) rectangle (60.74999999999999,-101.5);
\draw(53.949999999999996, -101.6) node[anchor=north west,align=left] {Normal\\ analytic\\ spaces};
\draw (53.949999999999996, -101.6) rectangle (56.55,-103.19999999999999);
\draw(56.65, -101.6) node[anchor=north west,align=left] {Complex\\ spaces};
\draw (56.65, -101.6) rectangle (59.0,-102.69999999999999);
\draw(41.449999999999996, -93.4) node[anchor=north west,align=left] {\large Pseudoconvex domains};
\draw (41.449999999999996, -93.4) rectangle (51.599999999999994,-100.5);
\draw(42.449999999999996, -94.4) node[anchor=north west,align=left] {Geometric and\\ analytic \\ invariants on \\ weakly pseudoconvex\\ boundaries};
\draw (42.449999999999996, -94.4) rectangle (47.8,-97.0);
\draw(47.89999999999999, -94.4) node[anchor=north west,align=left] {Strongly\\ pseudoconvex\\ domains};
\draw (47.89999999999999, -94.4) rectangle (51.49999999999999,-96.0);
\draw(42.449999999999996, -97.10000000000001) node[anchor=north west,align=left] {Finite-typedomains};
\draw (42.449999999999996, -97.10000000000001) rectangle (47.55,-98.7);
\draw(47.64999999999999, -97.10000000000001) node[anchor=north west,align=left] {Domains\\ of \\ holomorphy};
\draw (47.64999999999999, -97.10000000000001) rectangle (50.74999999999999,-98.7);
\draw(42.449999999999996, -98.80000000000001) node[anchor=north west,align=left] {Exhaustion\\ functions};
\draw (42.449999999999996, -98.80000000000001) rectangle (45.55,-100.4);
\draw(45.64999999999999, -98.80000000000001) node[anchor=north west,align=left] {Peak \\ functions};
\draw (45.64999999999999, -98.80000000000001) rectangle (48.49999999999999,-99.9);
\draw(48.599999999999994, -98.80000000000001) node[anchor=north west,align=left] {Worm \\ domains};
\draw (48.599999999999994, -98.80000000000001) rectangle (50.949999999999996,-99.9);
\draw(2, -103.4) node[anchor=north west,align=left] {\large Compact analytic spaces};
\draw (2, -103.4) rectangle (11.899999999999999,-113.7);
\draw(3, -104.4) node[anchor=north west,align=left] {Transcendental\\ methods of \\ algebraic geometry\\ (complex-analytic\\ aspects)};
\draw (3, -104.4) rectangle (8.1,-107.0);
\draw(8.2, -104.4) node[anchor=north west,align=left] {Applications\\ of compact\\ analytic\\ spaces to \\ the sciences};
\draw (8.2, -104.4) rectangle (11.799999999999999,-107.0);
\draw(3, -107.10000000000001) node[anchor=north west,align=left] {Compact \\ Kähler manifolds:\\ generalizations, \\ classification};
\draw (3, -107.10000000000001) rectangle (7.85,-109.7);
\draw(7.949999999999999, -107.10000000000001) node[anchor=north west,align=left] {Compact \\ complex \\ \(3\)-folds};
\draw (7.949999999999999, -107.10000000000001) rectangle (11.299999999999999,-108.7);
\draw(3, -109.80000000000001) node[anchor=north west,align=left] {Compactification\\ of \\ analytic spaces};
\draw (3, -109.80000000000001) rectangle (7.6,-111.9);
\draw(7.699999999999999, -109.80000000000001) node[anchor=north west,align=left] {Compact \\ complex \\ \(n\)-folds};
\draw (7.699999999999999, -109.80000000000001) rectangle (11.049999999999999,-111.4);
\draw(3, -112.0) node[anchor=north west,align=left] {Algebraic\\ dependence\\ theorems};
\draw (3, -112.0) rectangle (6.1,-113.6);
\draw(6.199999999999999, -112.0) node[anchor=north west,align=left] {Compact\\ complex\\ surfaces};
\draw (6.199999999999999, -112.0) rectangle (8.799999999999999,-113.6);
\draw(11.999999999999998, -103.4) node[anchor=north west,align=left] {\large Pluripotential theory};
\draw (11.999999999999998, -103.4) rectangle (21.4,-115.9);
\draw(12.999999999999998, -104.4) node[anchor=north west,align=left] {Plurisubharmonic\\ functions\\ and \\ generalizations};
\draw (12.999999999999998, -104.4) rectangle (17.599999999999998,-106.5);
\draw(17.699999999999996, -104.4) node[anchor=north west,align=left] {Currents};
\draw (17.699999999999996, -104.4) rectangle (20.299999999999997,-105.5);
\draw(12.999999999999998, -106.60000000000001) node[anchor=north west,align=left] {Plurisubharmonic\\ exhaustion\\ functions};
\draw (12.999999999999998, -106.60000000000001) rectangle (17.599999999999998,-108.7);
\draw(17.699999999999996, -106.60000000000001) node[anchor=north west,align=left] {Lelong\\ numbers};
\draw (17.699999999999996, -106.60000000000001) rectangle (20.049999999999997,-107.7);
\draw(12.999999999999998, -108.80000000000001) node[anchor=north west,align=left] {Plurisubharmonic\\ extremal\\ functions, \\ pluricomplex \\ Green functions};
\draw (12.999999999999998, -108.80000000000001) rectangle (17.599999999999998,-111.4);
\draw(12.999999999999998, -111.5) node[anchor=north west,align=left] {Capacity\\ theory\\ and \\ generalizations};
\draw (12.999999999999998, -111.5) rectangle (17.349999999999998,-113.6);
\draw(12.999999999999998, -113.7) node[anchor=north west,align=left] {General \\ pluripotential\\ theory};
\draw (12.999999999999998, -113.7) rectangle (17.099999999999998,-115.3);
\draw(17.199999999999996, -113.7) node[anchor=north west,align=left] {Removable\\ sets in\\ pluripotential\\ theory};
\draw (17.199999999999996, -113.7) rectangle (21.299999999999997,-115.8);
\draw(21.5, -103.4) node[anchor=north west,align=left] {\large Automorphic functions};
\draw (21.5, -103.4) rectangle (30.65,-109.30000000000001);
\draw(22.5, -104.4) node[anchor=north west,align=left] {General theory\\ of automorphic\\ functions\\ of several \\ complex variables};
\draw (22.5, -104.4) rectangle (27.35,-107.0);
\draw(22.5, -107.10000000000001) node[anchor=north west,align=left] {Automorphic\\ forms in \\ several complex\\ variables};
\draw (22.5, -107.10000000000001) rectangle (26.85,-109.2);
\draw(26.95, -107.10000000000001) node[anchor=north west,align=left] {Automorphic\\ functions\\ in symmetric\\ domains};
\draw (26.95, -107.10000000000001) rectangle (30.55,-109.2);
\draw(30.75, -103.4) node[anchor=north west,align=left] {\large CR manifolds};
\draw (30.75, -103.4) rectangle (39.9,-113.2);
\draw(31.75, -104.4) node[anchor=north west,align=left] {Extension of\\ functions and\\ other analytic\\ objects \\ from CR manifolds};
\draw (31.75, -104.4) rectangle (36.6,-107.0);
\draw(36.7, -104.4) node[anchor=north west,align=left] {Embeddings\\ of CR\\ manifolds};
\draw (36.7, -104.4) rectangle (39.800000000000004,-106.0);
\draw(31.75, -107.10000000000001) node[anchor=north west,align=left] {CR structures,\\ CR operators,\\ and \\ generalizations};
\draw (31.75, -107.10000000000001) rectangle (36.1,-109.2);
\draw(36.2, -107.10000000000001) node[anchor=north west,align=left] {CR manifolds\\ as \\ boundaries\\ of domains};
\draw (36.2, -107.10000000000001) rectangle (39.800000000000004,-109.2);
\draw(31.75, -109.30000000000001) node[anchor=north west,align=left] {Finite-type\\ conditions\\ on \\ CR manifolds};
\draw (31.75, -109.30000000000001) rectangle (35.35,-111.4);
\draw(35.45, -109.30000000000001) node[anchor=north west,align=left] {Real \\ submanifolds\\ in complex\\ manifolds};
\draw (35.45, -109.30000000000001) rectangle (39.050000000000004,-111.4);
\draw(31.75, -111.5) node[anchor=north west,align=left] {CR \\ functions};
\draw (31.75, -111.5) rectangle (34.6,-112.6);
\draw(34.7, -111.5) node[anchor=north west,align=left] {Analysis\\ on CR \\ manifolds};
\draw (34.7, -111.5) rectangle (37.550000000000004,-113.1);
\draw(40.0, -103.4) node[anchor=north west,align=left] {\large Analytic continuation};
\draw (40.0, -103.4) rectangle (48.9,-110.5);
\draw(41.0, -104.4) node[anchor=north west,align=left] {Removable \\ singularities in\\ several complex\\ variables};
\draw (41.0, -104.4) rectangle (45.6,-106.5);
\draw(45.7, -104.4) node[anchor=north west,align=left] {Domains\\ of \\ holomorphy};
\draw (45.7, -104.4) rectangle (48.800000000000004,-106.0);
\draw(41.0, -106.60000000000001) node[anchor=north west,align=left] {Continuation\\ of analytic\\ objects in\\ several complex\\ variables};
\draw (41.0, -106.60000000000001) rectangle (45.35,-109.2);
\draw(45.45, -106.60000000000001) node[anchor=north west,align=left] {Envelopes\\ of \\ holomorphy};
\draw (45.45, -106.60000000000001) rectangle (48.550000000000004,-108.2);
\draw(41.0, -109.30000000000001) node[anchor=north west,align=left] {Riemann\\ domains};
\draw (41.0, -109.30000000000001) rectangle (43.35,-110.4);
\draw(49.0, -103.4) node[anchor=north west,align=left] {\large Holomorphic fiber spaces};
\draw (49.0, -103.4) rectangle (57.9,-115.2);
\draw(50.0, -104.4) node[anchor=north west,align=left] {Twistor theory,\\ double \\ fibrations \\ (complex-analytic\\ aspects)};
\draw (50.0, -104.4) rectangle (54.85,-107.0);
\draw(54.95, -104.4) node[anchor=north west,align=left] {Bundle \\ convexity};
\draw (54.95, -104.4) rectangle (57.800000000000004,-105.5);
\draw(50.0, -107.10000000000001) node[anchor=north west,align=left] {Holomorphic\\ bundles\\ and \\ generalizations};
\draw (50.0, -107.10000000000001) rectangle (54.35,-109.2);
\draw(54.45, -107.10000000000001) node[anchor=north west,align=left] {Vanishing\\ theorems};
\draw (54.45, -107.10000000000001) rectangle (57.300000000000004,-108.7);
\draw(50.0, -109.30000000000001) node[anchor=north west,align=left] {Sheaves and \\ cohomology of\\ sections of \\ holomorphic \\ vector bundles,\\ general results};
\draw (50.0, -109.30000000000001) rectangle (54.35,-112.4);
\draw(50.0, -112.5) node[anchor=north west,align=left] {Applications\\ of holomorphic\\ fiber\\ spaces to\\ the sciences};
\draw (50.0, -112.5) rectangle (54.1,-115.1);
\draw(65.24999999999999, -52.5) node[anchor=north west,align=left] {\LARGE Ordinary differential equations};
\draw (65.24999999999999, -52.5) rectangle (125.35,-133.2);
\draw(66.24999999999999, -53.5) node[anchor=north west,align=left] {\large Functional-differential equations (including equations with delayed, advanced or state-dependent argument)};
\draw (66.24999999999999, -53.5) rectangle (105.94999999999999,-74.9);
\draw(67.24999999999999, -54.5) node[anchor=north west,align=left] {Functional-differentialinclusions};
\draw (67.24999999999999, -54.5) rectangle (76.09999999999998,-56.6);
\draw(76.19999999999999, -54.5) node[anchor=north west,align=left] {Functional-differentialequations\\ with fractional\\ derivatives};
\draw (76.19999999999999, -54.5) rectangle (84.79999999999998,-57.1);
\draw(84.89999999999998, -54.5) node[anchor=north west,align=left] {Functional-differentialequations\\ with state-dependent\\ arguments};
\draw (84.89999999999998, -54.5) rectangle (93.49999999999997,-57.1);
\draw(93.59999999999998, -54.5) node[anchor=north west,align=left] {Transformation\\ and reduction \\ of functional-differential\\ equations and systems,\\ normal forms};
\draw (93.59999999999998, -54.5) rectangle (100.69999999999997,-57.6);
\draw(67.24999999999999, -57.7) node[anchor=north west,align=left] {Symmetries,\\ invariants\\ of \\ functional-differential\\ equations};
\draw (67.24999999999999, -57.7) rectangle (73.59999999999998,-60.300000000000004);
\draw(73.69999999999999, -57.7) node[anchor=north west,align=left] {General theory\\ of \\ functional-differential\\ equations};
\draw (73.69999999999999, -57.7) rectangle (80.04999999999998,-59.800000000000004);
\draw(80.14999999999998, -57.7) node[anchor=north west,align=left] {Linear \\ functional-differential\\ equations};
\draw (80.14999999999998, -57.7) rectangle (86.49999999999997,-59.800000000000004);
\draw(86.6, -57.7) node[anchor=north west,align=left] {Theoretical \\ approximation of \\ solutions to \\ functional-differential\\ equations};
\draw (86.6, -57.7) rectangle (92.94999999999999,-60.300000000000004);
\draw(93.04999999999998, -57.7) node[anchor=north west,align=left] {Spectral theory\\ of \\ functional-differential\\ operators};
\draw (93.04999999999998, -57.7) rectangle (99.39999999999998,-59.800000000000004);
\draw(99.49999999999999, -57.7) node[anchor=north west,align=left] {Boundary value\\ problems\\ for \\ functional-differential\\ equations};
\draw (99.49999999999999, -57.7) rectangle (105.84999999999998,-60.300000000000004);
\draw(67.24999999999999, -60.4) node[anchor=north west,align=left] {Oscillation\\ theory of\\ functional-differential\\ equations};
\draw (67.24999999999999, -60.4) rectangle (73.59999999999998,-63.0);
\draw(73.69999999999999, -60.4) node[anchor=north west,align=left] {Growth, \\ boundedness, \\ comparison of \\ solutions to \\ functional-differential\\ equations};
\draw (73.69999999999999, -60.4) rectangle (80.04999999999998,-63.5);
\draw(80.14999999999998, -60.4) node[anchor=north west,align=left] {Periodic \\ solutions to\\ functional-differential\\ equations};
\draw (80.14999999999998, -60.4) rectangle (86.49999999999997,-63.0);
\draw(86.6, -60.4) node[anchor=north west,align=left] {Almost and \\ pseudo-almost periodic\\ solutions to \\ functional-differential\\ equations};
\draw (86.6, -60.4) rectangle (92.94999999999999,-63.0);
\draw(93.04999999999998, -60.4) node[anchor=north west,align=left] {Heteroclinic \\ and homoclinic\\ orbits of \\ functional-differential\\ equations};
\draw (93.04999999999998, -60.4) rectangle (99.39999999999998,-63.0);
\draw(99.49999999999999, -60.4) node[anchor=north west,align=left] {Bifurcation\\ theory of\\ functional-differential\\ equations};
\draw (99.49999999999999, -60.4) rectangle (105.84999999999998,-63.0);
\draw(67.24999999999999, -63.6) node[anchor=north west,align=left] {Invariant \\ manifolds of\\ functional-differential\\ equations};
\draw (67.24999999999999, -63.6) rectangle (73.59999999999998,-66.2);
\draw(73.69999999999999, -63.6) node[anchor=north west,align=left] {Stability \\ theory of \\ functional-differential\\ equations};
\draw (73.69999999999999, -63.6) rectangle (80.04999999999998,-65.7);
\draw(80.14999999999998, -63.6) node[anchor=north west,align=left] {Stationary\\ solutions \\ of \\ functional-differential\\ equations};
\draw (80.14999999999998, -63.6) rectangle (86.49999999999997,-66.2);
\draw(86.6, -63.6) node[anchor=north west,align=left] {Complex (chaotic)\\ behavior of\\ solutions to \\ functional-differential\\ equations};
\draw (86.6, -63.6) rectangle (92.94999999999999,-66.2);
\draw(93.04999999999998, -63.6) node[anchor=north west,align=left] {Synchronization\\ of \\ functional-differential\\ equations};
\draw (93.04999999999998, -63.6) rectangle (99.39999999999998,-65.7);
\draw(99.49999999999999, -63.6) node[anchor=north west,align=left] {Asymptotic\\ theory of\\ functional-differential\\ equations};
\draw (99.49999999999999, -63.6) rectangle (105.84999999999998,-66.2);
\draw(67.24999999999999, -66.3) node[anchor=north west,align=left] {Singular \\ perturbations\\ of \\ functional-differential\\ equations};
\draw (67.24999999999999, -66.3) rectangle (73.59999999999998,-68.89999999999999);
\draw(73.69999999999999, -66.3) node[anchor=north west,align=left] {Perturbations\\ of \\ functional-differential\\ equations};
\draw (73.69999999999999, -66.3) rectangle (80.04999999999998,-68.39999999999999);
\draw(80.14999999999998, -66.3) node[anchor=north west,align=left] {Inverse \\ problems for\\ functional-differential\\ equations};
\draw (80.14999999999998, -66.3) rectangle (86.49999999999997,-68.89999999999999);
\draw(86.6, -66.3) node[anchor=north west,align=left] {Functional-differential\\ equations in \\ abstract spaces};
\draw (86.6, -66.3) rectangle (92.94999999999999,-68.39999999999999);
\draw(93.04999999999998, -66.3) node[anchor=north west,align=left] {Lattice \\ functional-differential\\ equations};
\draw (93.04999999999998, -66.3) rectangle (99.39999999999998,-68.39999999999999);
\draw(99.49999999999999, -66.3) node[anchor=north west,align=left] {Implicit \\ functional-differential\\ equations};
\draw (99.49999999999999, -66.3) rectangle (105.84999999999998,-68.39999999999999);
\draw(67.24999999999999, -69.0) node[anchor=north west,align=left] {Averaging \\ for \\ functional-differential\\ equations};
\draw (67.24999999999999, -69.0) rectangle (73.59999999999998,-71.1);
\draw(73.69999999999999, -69.0) node[anchor=north west,align=left] {Hybrid systems\\ of \\ functional-differential\\ equations};
\draw (73.69999999999999, -69.0) rectangle (80.04999999999998,-71.1);
\draw(80.14999999999998, -69.0) node[anchor=north west,align=left] {Control \\ problems for\\ functional-differential\\ equations};
\draw (80.14999999999998, -69.0) rectangle (86.49999999999997,-71.6);
\draw(86.6, -69.0) node[anchor=north west,align=left] {Fuzzy \\ functional-differential\\ equations};
\draw (86.6, -69.0) rectangle (92.94999999999999,-71.1);
\draw(93.04999999999998, -69.0) node[anchor=north west,align=left] {Functional-differential\\ inequalities};
\draw (93.04999999999998, -69.0) rectangle (99.39999999999998,-71.1);
\draw(99.49999999999999, -69.0) node[anchor=north west,align=left] {Discontinuous\\ functional-differential\\ equations};
\draw (99.49999999999999, -69.0) rectangle (105.84999999999998,-71.1);
\draw(67.24999999999999, -71.7) node[anchor=north west,align=left] {Neutral \\ functional-differential\\ equations};
\draw (67.24999999999999, -71.7) rectangle (73.59999999999998,-73.8);
\draw(73.69999999999999, -71.7) node[anchor=north west,align=left] {Functional-differential\\ equations\\ in the \\ complex domain};
\draw (73.69999999999999, -71.7) rectangle (80.04999999999998,-74.3);
\draw(80.14999999999998, -71.7) node[anchor=north west,align=left] {Functional-differential\\ equations on \\ time scales or\\ measure chains};
\draw (80.14999999999998, -71.7) rectangle (86.49999999999997,-74.3);
\draw(86.6, -71.7) node[anchor=north west,align=left] {Functional-differential\\ equations\\ with impulses};
\draw (86.6, -71.7) rectangle (92.94999999999999,-73.8);
\draw(93.04999999999998, -71.7) node[anchor=north west,align=left] {Stochastic\\ functional-differential\\ equations};
\draw (93.04999999999998, -71.7) rectangle (99.39999999999998,-73.8);
\draw(99.49999999999999, -71.7) node[anchor=north west,align=left] {Qualitative \\ investigation and\\ simulation of \\ models involving\\ functional-differential\\ equations};
\draw (99.49999999999999, -71.7) rectangle (105.84999999999998,-74.8);
\draw(106.04999999999998, -53.5) node[anchor=north west,align=left] {\large General theory for ordinary differential equations};
\draw (106.04999999999998, -53.5) rectangle (125.14999999999998,-72.2);
\draw(107.04999999999998, -54.5) node[anchor=north west,align=left] {Analytical theory\\ of ordinary \\ differential \\ equations: series, \\ transformations, \\ transforms, \\ operational calculus, etc.};
\draw (107.04999999999998, -54.5) rectangle (114.14999999999998,-58.1);
\draw(114.24999999999999, -54.5) node[anchor=north west,align=left] {Initial value problems,\\ existence, \\ uniqueness, continuous\\ dependence and \\ continuation of solutions\\ to ordinary \\ differential equations};
\draw (114.24999999999999, -54.5) rectangle (121.09999999999998,-58.1);
\draw(121.19999999999999, -54.5) node[anchor=north west,align=left] {Discontinuous\\ ordinary\\ differential\\ equations};
\draw (121.19999999999999, -54.5) rectangle (125.04999999999998,-56.6);
\draw(107.04999999999998, -58.2) node[anchor=north west,align=left] {Generalized \\ ordinary differential\\ equations \\ (measure-differential\\ equations, \\ set-valued differential\\ equations, etc.)};
\draw (107.04999999999998, -58.2) rectangle (113.39999999999998,-61.800000000000004);
\draw(113.49999999999999, -58.2) node[anchor=north west,align=left] {Fractional \\ ordinary \\ differential equations\\ and \\ fractional differential\\ inclusions};
\draw (113.49999999999999, -58.2) rectangle (119.84999999999998,-61.300000000000004);
\draw(119.94999999999999, -58.2) node[anchor=north west,align=left] {Inverse \\ problems involving\\ ordinary\\ differential\\ equations};
\draw (119.94999999999999, -58.2) rectangle (125.04999999999998,-60.800000000000004);
\draw(107.04999999999998, -61.9) node[anchor=north west,align=left] {Implicit ordinary\\ differential\\ equations,\\ differential-algebraic\\ equations};
\draw (107.04999999999998, -61.9) rectangle (113.14999999999998,-64.5);
\draw(113.24999999999999, -61.9) node[anchor=north west,align=left] {Theoretical \\ approximation of\\ solutions to\\ ordinary \\ differential equations};
\draw (113.24999999999999, -61.9) rectangle (119.34999999999998,-64.5);
\draw(119.44999999999999, -61.9) node[anchor=north west,align=left] {Geometric \\ methods in ordinary\\ differential\\ equations};
\draw (119.44999999999999, -61.9) rectangle (124.79999999999998,-64.0);
\draw(107.04999999999998, -64.6) node[anchor=north west,align=left] {Explicit \\ solutions, first \\ integrals of \\ ordinary differential\\ equations};
\draw (107.04999999999998, -64.6) rectangle (112.89999999999998,-67.19999999999999);
\draw(112.99999999999999, -64.6) node[anchor=north west,align=left] {Nonlinear \\ ordinary differential\\ equations\\ and systems,\\ general theory};
\draw (112.99999999999999, -64.6) rectangle (118.84999999999998,-67.19999999999999);
\draw(118.94999999999999, -64.6) node[anchor=north west,align=left] {Differential \\ inequalities \\ involving functions\\ of a single\\ real variable};
\draw (118.94999999999999, -64.6) rectangle (124.29999999999998,-67.19999999999999);
\draw(107.04999999999998, -67.3) node[anchor=north west,align=left] {Linear \\ ordinary \\ differential \\ equations and \\ systems, general};
\draw (107.04999999999998, -67.3) rectangle (111.64999999999998,-69.89999999999999);
\draw(111.74999999999999, -67.3) node[anchor=north west,align=left] {Ordinary \\ differential \\ equations of \\ infinite order};
\draw (111.74999999999999, -67.3) rectangle (115.84999999999998,-69.39999999999999);
\draw(115.94999999999999, -67.3) node[anchor=north west,align=left] {Hybrid systems\\ of ordinary\\ differential\\ equations};
\draw (115.94999999999999, -67.3) rectangle (120.04999999999998,-69.39999999999999);
\draw(120.14999999999998, -67.3) node[anchor=north west,align=left] {Ordinary \\ differential\\ equations \\ with impulses};
\draw (120.14999999999998, -67.3) rectangle (123.99999999999997,-69.39999999999999);
\draw(107.04999999999998, -70.0) node[anchor=north west,align=left] {Fuzzy \\ ordinary \\ differential\\ equations};
\draw (107.04999999999998, -70.0) rectangle (110.64999999999998,-72.1);
\draw(110.74999999999999, -70.0) node[anchor=north west,align=left] {Ordinary\\ lattice \\ differential\\ equations};
\draw (110.74999999999999, -70.0) rectangle (114.34999999999998,-72.1);
\draw(114.44999999999999, -70.0) node[anchor=north west,align=left] {Ordinary \\ differential\\ inclusions};
\draw (114.44999999999999, -70.0) rectangle (118.04999999999998,-71.6);
\draw(106.04999999999998, -72.30000000000001) node[anchor=north west,align=left] {\large History of \\ ordinary differential\\ equations};
\draw (106.04999999999998, -72.30000000000001) rectangle (113.15999999999998,-73.9);
\draw(66.24999999999999, -75.0) node[anchor=north west,align=left] {\large Ordinary differential equations in the complex domain};
\draw (66.24999999999999, -75.0) rectangle (86.35,-96.2);
\draw(67.24999999999999, -76.0) node[anchor=north west,align=left] {Singular perturbation\\ problems for \\ ordinary differential\\ equations in the \\ complex domain (complex\\ WKB, turning \\ points, steepest descent)};
\draw (67.24999999999999, -76.0) rectangle (74.09999999999998,-79.6);
\draw(74.19999999999999, -76.0) node[anchor=north west,align=left] {Algebraic aspects \\ (differential-algebraic,\\ hypertranscendence,\\ group-theoretical)\\ of ordinary \\ differential equations\\ in the complex domain};
\draw (74.19999999999999, -76.0) rectangle (80.79999999999998,-79.6);
\draw(80.89999999999998, -76.0) node[anchor=north west,align=left] {Oscillation, \\ growth of solutions\\ to ordinary\\ differential\\ equations in\\ the complex domain};
\draw (80.89999999999998, -76.0) rectangle (86.24999999999997,-79.1);
\draw(67.24999999999999, -79.7) node[anchor=north west,align=left] {Painlevé and other\\ special ordinary\\ differential equations\\ in the complex\\ domain; classification,\\ hierarchies};
\draw (67.24999999999999, -79.7) rectangle (73.59999999999998,-82.8);
\draw(73.69999999999999, -79.7) node[anchor=north west,align=left] {Singularities, \\ monodromy and local \\ behavior of solutions\\ to ordinary \\ differential equations\\ in the complex \\ domain, normal forms};
\draw (73.69999999999999, -79.7) rectangle (79.79999999999998,-83.3);
\draw(79.89999999999998, -79.7) node[anchor=north west,align=left] {Entire and \\ meromorphic solutions\\ to ordinary\\ differential\\ equations in\\ the complex domain};
\draw (79.89999999999998, -79.7) rectangle (85.74999999999997,-82.8);
\draw(67.24999999999999, -83.4) node[anchor=north west,align=left] {Formal solutions\\ and transform \\ techniques for \\ ordinary differential\\ equations in\\ the complex domain};
\draw (67.24999999999999, -83.4) rectangle (73.09999999999998,-86.5);
\draw(73.19999999999999, -83.4) node[anchor=north west,align=left] {Stokes phenomena\\ and connection \\ problems (linear and\\ nonlinear) for\\ ordinary differential\\ equations in\\ the complex domain};
\draw (73.19999999999999, -83.4) rectangle (79.04999999999998,-87.0);
\draw(79.14999999999998, -83.4) node[anchor=north west,align=left] {Inverse problems \\ (Riemann-Hilbert, \\ inverse differential\\ Galois, etc.) for\\ ordinary differential\\ equations in\\ the complex domain};
\draw (79.14999999999998, -83.4) rectangle (84.99999999999997,-87.0);
\draw(67.24999999999999, -87.1) node[anchor=north west,align=left] {Topological \\ structure of \\ trajectories of \\ ordinary differential\\ equations in\\ the complex domain};
\draw (67.24999999999999, -87.1) rectangle (73.09999999999998,-90.19999999999999);
\draw(73.19999999999999, -87.1) node[anchor=north west,align=left] {Ordinary \\ differential \\ equations on complex\\ manifolds};
\draw (73.19999999999999, -87.1) rectangle (78.79999999999998,-89.19999999999999);
\draw(78.89999999999998, -87.1) node[anchor=north west,align=left] {Nonlinear ordinary\\ differential\\ equations and\\ systems in the\\ complex domain};
\draw (78.89999999999998, -87.1) rectangle (83.99999999999997,-89.69999999999999);
\draw(67.24999999999999, -90.3) node[anchor=north west,align=left] {Spectral theory\\ for ordinary\\ differential\\ operators in \\ the complex domain};
\draw (67.24999999999999, -90.3) rectangle (72.34999999999998,-92.89999999999999);
\draw(72.44999999999999, -90.3) node[anchor=north west,align=left] {Asymptotics and\\ summation methods\\ for ordinary\\ differential\\ equations in the\\ complex domain};
\draw (72.44999999999999, -90.3) rectangle (77.29999999999998,-93.39999999999999);
\draw(77.39999999999998, -90.3) node[anchor=north west,align=left] {Isomonodromic\\ deformations \\ for ordinary \\ differential \\ equations in the\\ complex domain};
\draw (77.39999999999998, -90.3) rectangle (81.99999999999997,-93.39999999999999);
\draw(67.24999999999999, -93.5) node[anchor=north west,align=left] {Linear ordinary\\ differential\\ equations and\\ systems in the\\ complex domain};
\draw (67.24999999999999, -93.5) rectangle (71.59999999999998,-96.1);
\draw(86.44999999999999, -75.0) node[anchor=north west,align=left] {\large Control problems including ordinary differential equations};
\draw (86.44999999999999, -75.0) rectangle (106.5,-78.7);
\draw(87.44999999999999, -76.0) node[anchor=north west,align=left] {Chaos control\\ for problems\\ involving \\ ordinary \\ differential equations};
\draw (87.44999999999999, -76.0) rectangle (93.54999999999998,-78.6);
\draw(93.64999999999999, -76.0) node[anchor=north west,align=left] {Control \\ problems involving\\ ordinary\\ differential\\ equations};
\draw (93.64999999999999, -76.0) rectangle (98.74999999999999,-78.6);
\draw(98.85, -76.0) node[anchor=north west,align=left] {Stabilization\\ of solutions\\ to ordinary\\ differential\\ equations};
\draw (98.85, -76.0) rectangle (102.69999999999999,-78.6);
\draw(102.79999999999998, -76.0) node[anchor=north west,align=left] {Bifurcation\\ control\\ of ordinary\\ differential\\ equations};
\draw (102.79999999999998, -76.0) rectangle (106.39999999999998,-78.6);
\draw(86.44999999999999, -78.80000000000001) node[anchor=north west,align=left] {\large Stability theory for ordinary differential equations};
\draw (86.44999999999999, -78.80000000000001) rectangle (106.04999999999998,-91.10000000000001);
\draw(87.44999999999999, -79.80000000000001) node[anchor=north west,align=left] {Structural \\ stability and analogous\\ concepts \\ of solutions to\\ ordinary \\ differential equations};
\draw (87.44999999999999, -79.80000000000001) rectangle (93.79999999999998,-82.9);
\draw(93.89999999999999, -79.80000000000001) node[anchor=north west,align=left] {Asymptotic \\ properties of \\ solutions to \\ ordinary \\ differential equations};
\draw (93.89999999999999, -79.80000000000001) rectangle (99.99999999999999,-82.4);
\draw(100.1, -79.80000000000001) node[anchor=north west,align=left] {Synchronization\\ of solutions\\ to \\ ordinary differential\\ equations};
\draw (100.1, -79.80000000000001) rectangle (105.94999999999999,-82.4);
\draw(87.44999999999999, -83.00000000000001) node[anchor=north west,align=left] {Characteristic\\ and Lyapunov\\ exponents of \\ ordinary \\ differential equations};
\draw (87.44999999999999, -83.00000000000001) rectangle (93.54999999999998,-85.60000000000001);
\draw(93.64999999999999, -83.00000000000001) node[anchor=north west,align=left] {Dichotomy, \\ trichotomy of \\ solutions to \\ ordinary \\ differential equations};
\draw (93.64999999999999, -83.00000000000001) rectangle (99.74999999999999,-85.60000000000001);
\draw(99.85, -83.00000000000001) node[anchor=north west,align=left] {Global stability\\ of \\ solutions to \\ ordinary \\ differential equations};
\draw (99.85, -83.00000000000001) rectangle (105.94999999999999,-85.60000000000001);
\draw(87.44999999999999, -85.70000000000002) node[anchor=north west,align=left] {Stability of \\ manifolds of \\ solutions to \\ ordinary differential\\ equations};
\draw (87.44999999999999, -85.70000000000002) rectangle (93.29999999999998,-88.30000000000001);
\draw(93.39999999999999, -85.70000000000002) node[anchor=north west,align=left] {Perturbations\\ of ordinary\\ differential\\ equations};
\draw (93.39999999999999, -85.70000000000002) rectangle (97.24999999999999,-87.80000000000001);
\draw(97.35, -85.70000000000002) node[anchor=north west,align=left] {Singular \\ perturbations\\ of ordinary\\ differential\\ equations};
\draw (97.35, -85.70000000000002) rectangle (101.19999999999999,-88.30000000000001);
\draw(101.29999999999998, -85.70000000000002) node[anchor=north west,align=left] {Stability \\ of solutions\\ to ordinary\\ differential\\ equations};
\draw (101.29999999999998, -85.70000000000002) rectangle (104.89999999999998,-88.30000000000001);
\draw(87.44999999999999, -88.4) node[anchor=north west,align=left] {Attractors\\ of solutions\\ to ordinary\\ differential\\ equations};
\draw (87.44999999999999, -88.4) rectangle (91.04999999999998,-91.0);
\draw(86.44999999999999, -91.20000000000002) node[anchor=north west,align=left] {\large Ordinary differential equations and systems with randomness};
\draw (86.44999999999999, -91.20000000000002) rectangle (105.33999999999999,-94.90000000000002);
\draw(87.44999999999999, -92.20000000000002) node[anchor=north west,align=left] {Bifurcation of \\ solutions to \\ ordinary differential\\ equations \\ involving randomness};
\draw (87.44999999999999, -92.20000000000002) rectangle (93.29999999999998,-94.80000000000001);
\draw(93.39999999999999, -92.20000000000002) node[anchor=north west,align=left] {Resonance phenomena\\ for ordinary\\ differential\\ equations \\ involving randomness};
\draw (93.39999999999999, -92.20000000000002) rectangle (98.99999999999999,-94.80000000000001);
\draw(99.1, -92.20000000000002) node[anchor=north west,align=left] {Ordinary \\ differential\\ equations \\ and systems\\ with randomness};
\draw (99.1, -92.20000000000002) rectangle (103.44999999999999,-94.80000000000001);
\draw(106.6, -75.0) node[anchor=north west,align=left] {\large Dynamic equations on time scales or measure chains};
\draw (106.6, -75.0) rectangle (122.69999999999999,-78.2);
\draw(107.6, -76.0) node[anchor=north west,align=left] {Dynamic \\ equations on time\\ scales or\\ measure chains};
\draw (107.6, -76.0) rectangle (112.44999999999999,-78.1);
\draw(66.24999999999999, -96.30000000000001) node[anchor=north west,align=left] {\large Boundary value problems for ordinary differential equations};
\draw (66.24999999999999, -96.30000000000001) rectangle (85.85,-115.00000000000001);
\draw(67.24999999999999, -97.30000000000001) node[anchor=north west,align=left] {Linear boundary\\ value problems \\ for ordinary \\ differential equations\\ with nonlinear\\ dependence on\\ the spectral parameter};
\draw (67.24999999999999, -97.30000000000001) rectangle (73.34999999999998,-100.9);
\draw(73.44999999999999, -97.30000000000001) node[anchor=north west,align=left] {Parameter dependent\\ boundary \\ value problems \\ for ordinary \\ differential equations};
\draw (73.44999999999999, -97.30000000000001) rectangle (79.54999999999998,-99.9);
\draw(79.64999999999998, -97.30000000000001) node[anchor=north west,align=left] {Boundary \\ eigenvalue \\ problems for \\ ordinary \\ differential equations};
\draw (79.64999999999998, -97.30000000000001) rectangle (85.74999999999997,-99.9);
\draw(67.24999999999999, -101.00000000000001) node[anchor=north west,align=left] {Nonlocal and\\ multipoint \\ boundary value\\ problems for\\ ordinary \\ differential equations};
\draw (67.24999999999999, -101.00000000000001) rectangle (73.34999999999998,-104.10000000000001);
\draw(73.44999999999999, -101.00000000000001) node[anchor=north west,align=left] {Nonlinear \\ boundary value \\ problems for \\ ordinary \\ differential equations};
\draw (73.44999999999999, -101.00000000000001) rectangle (79.54999999999998,-103.60000000000001);
\draw(79.64999999999998, -101.00000000000001) node[anchor=north west,align=left] {Singular nonlinear\\ boundary \\ value problems for\\ ordinary \\ differential equations};
\draw (79.64999999999998, -101.00000000000001) rectangle (85.74999999999997,-103.60000000000001);
\draw(67.24999999999999, -104.20000000000002) node[anchor=north west,align=left] {Positive solutions\\ to nonlinear\\ boundary value\\ problems for\\ ordinary \\ differential equations};
\draw (67.24999999999999, -104.20000000000002) rectangle (73.34999999999998,-107.30000000000001);
\draw(73.44999999999999, -104.20000000000002) node[anchor=north west,align=left] {Boundary value\\ problems with\\ impulses for\\ ordinary \\ differential equations};
\draw (73.44999999999999, -104.20000000000002) rectangle (79.54999999999998,-106.80000000000001);
\draw(79.64999999999998, -104.20000000000002) node[anchor=north west,align=left] {Boundary value\\ problems on\\ infinite \\ intervals for \\ ordinary \\ differential equations};
\draw (79.64999999999998, -104.20000000000002) rectangle (85.74999999999997,-107.30000000000001);
\draw(67.24999999999999, -107.4) node[anchor=north west,align=left] {Boundary value\\ problems on\\ graphs and \\ networks for \\ ordinary \\ differential equations};
\draw (67.24999999999999, -107.4) rectangle (73.34999999999998,-110.5);
\draw(73.44999999999999, -107.4) node[anchor=north west,align=left] {Applications\\ of boundary \\ value problems\\ involving \\ ordinary \\ differential equations};
\draw (73.44999999999999, -107.4) rectangle (79.54999999999998,-110.5);
\draw(79.64999999999998, -107.4) node[anchor=north west,align=left] {Linear boundary\\ value \\ problems for \\ ordinary differential\\ equations};
\draw (79.64999999999998, -107.4) rectangle (85.49999999999997,-110.0);
\draw(67.24999999999999, -110.60000000000001) node[anchor=north west,align=left] {Weyl theory and\\ its generalizations\\ for \\ ordinary differential\\ equations};
\draw (67.24999999999999, -110.60000000000001) rectangle (73.09999999999998,-113.2);
\draw(73.19999999999999, -110.60000000000001) node[anchor=north west,align=left] {Green’s functions\\ for \\ ordinary differential\\ equations};
\draw (73.19999999999999, -110.60000000000001) rectangle (79.04999999999998,-112.7);
\draw(79.14999999999998, -110.60000000000001) node[anchor=north west,align=left] {Special ordinary\\ differential\\ equations\\ (Mathieu, \\ Hill, Bessel, etc.)};
\draw (79.14999999999998, -110.60000000000001) rectangle (84.49999999999997,-113.2);
\draw(67.24999999999999, -113.30000000000001) node[anchor=north west,align=left] {Sturm-Liouville\\ theory};
\draw (67.24999999999999, -113.30000000000001) rectangle (71.59999999999998,-114.9);
\draw(85.94999999999999, -96.30000000000001) node[anchor=north west,align=left] {\large Qualitative theory for ordinary differential equations};
\draw (85.94999999999999, -96.30000000000001) rectangle (105.54999999999998,-119.20000000000002);
\draw(86.94999999999999, -97.30000000000001) node[anchor=north west,align=left] {Ordinary differential\\ equations and \\ connections with real \\ algebraic geometry \\ (fewnomials, desingularization,\\ zeros of \\ abelian integrals, etc.)};
\draw (86.94999999999999, -97.30000000000001) rectangle (95.29999999999998,-100.9);
\draw(95.39999999999999, -97.30000000000001) node[anchor=north west,align=left] {Theory of limit cycles\\ of polynomial and \\ analytic vector fields\\ (existence, uniqueness,\\ bounds, Hilbert’s\\ 16th problem and \\ ramifications) for ordinary\\ differential equations};
\draw (95.39999999999999, -97.30000000000001) rectangle (102.74999999999999,-101.4);
\draw(86.94999999999999, -101.50000000000001) node[anchor=north west,align=left] {Topological structure\\ of integral\\ curves, singular\\ points, limit \\ cycles of ordinary \\ differential equations};
\draw (86.94999999999999, -101.50000000000001) rectangle (93.04999999999998,-104.60000000000001);
\draw(93.14999999999999, -101.50000000000001) node[anchor=north west,align=left] {Oscillation theory,\\ zeros, \\ disconjugacy and \\ comparison theory for\\ ordinary \\ differential equations};
\draw (93.14999999999999, -101.50000000000001) rectangle (99.24999999999999,-104.60000000000001);
\draw(99.35, -101.50000000000001) node[anchor=north west,align=left] {Nonlinear \\ oscillations and\\ coupled \\ oscillators for \\ ordinary \\ differential equations};
\draw (99.35, -101.50000000000001) rectangle (105.44999999999999,-104.60000000000001);
\draw(86.94999999999999, -104.70000000000002) node[anchor=north west,align=left] {Almost and \\ pseudo-almost periodic\\ solutions \\ to ordinary \\ differential equations};
\draw (86.94999999999999, -104.70000000000002) rectangle (93.04999999999998,-107.30000000000001);
\draw(93.14999999999999, -104.70000000000002) node[anchor=north west,align=left] {Homoclinic and\\ heteroclinic\\ solutions to \\ ordinary \\ differential equations};
\draw (93.14999999999999, -104.70000000000002) rectangle (99.24999999999999,-107.30000000000001);
\draw(99.35, -104.70000000000002) node[anchor=north west,align=left] {Equivalence and\\ asymptotic \\ equivalence of\\ ordinary \\ differential equations};
\draw (99.35, -104.70000000000002) rectangle (105.44999999999999,-107.30000000000001);
\draw(86.94999999999999, -107.4) node[anchor=north west,align=left] {Growth and \\ boundedness of \\ solutions to \\ ordinary differential\\ equations};
\draw (86.94999999999999, -107.4) rectangle (92.79999999999998,-110.0);
\draw(92.89999999999999, -107.4) node[anchor=north west,align=left] {Transformation\\ and reduction\\ of ordinary \\ differential \\ equations and \\ systems, normal forms};
\draw (92.89999999999999, -107.4) rectangle (98.74999999999999,-110.5);
\draw(98.85, -107.4) node[anchor=north west,align=left] {Periodic \\ solutions to \\ ordinary differential\\ equations};
\draw (98.85, -107.4) rectangle (104.69999999999999,-109.5);
\draw(86.94999999999999, -110.60000000000001) node[anchor=north west,align=left] {Complex behavior\\ and chaotic\\ systems of \\ ordinary differential\\ equations};
\draw (86.94999999999999, -110.60000000000001) rectangle (92.79999999999998,-113.2);
\draw(92.89999999999999, -110.60000000000001) node[anchor=north west,align=left] {Averaging \\ method for ordinary\\ differential\\ equations};
\draw (92.89999999999999, -110.60000000000001) rectangle (98.24999999999999,-112.7);
\draw(98.35, -110.60000000000001) node[anchor=north west,align=left] {Monotone \\ systems involving\\ ordinary\\ differential\\ equations};
\draw (98.35, -110.60000000000001) rectangle (103.19999999999999,-113.2);
\draw(86.94999999999999, -113.30000000000001) node[anchor=north west,align=left] {Qualitative \\ investigation\\ and simulation\\ of ordinary\\ differential\\ equation models};
\draw (86.94999999999999, -113.30000000000001) rectangle (91.29999999999998,-116.4);
\draw(91.39999999999999, -113.30000000000001) node[anchor=north west,align=left] {Multifrequency\\ systems\\ of ordinary\\ differential\\ equations};
\draw (91.39999999999999, -113.30000000000001) rectangle (95.49999999999999,-115.9);
\draw(95.6, -113.30000000000001) node[anchor=north west,align=left] {Invariant \\ manifolds for\\ ordinary\\ differential\\ equations};
\draw (95.6, -113.30000000000001) rectangle (99.44999999999999,-115.9);
\draw(99.54999999999998, -113.30000000000001) node[anchor=north west,align=left] {Symmetries,\\ invariants\\ of ordinary\\ differential\\ equations};
\draw (99.54999999999998, -113.30000000000001) rectangle (103.14999999999998,-115.9);
\draw(86.94999999999999, -116.5) node[anchor=north west,align=left] {Bifurcation\\ theory \\ for ordinary\\ differential\\ equations};
\draw (86.94999999999999, -116.5) rectangle (90.54999999999998,-119.1);
\draw(90.64999999999999, -116.5) node[anchor=north west,align=left] {Relaxation\\ oscillations\\ for ordinary\\ differential\\ equations};
\draw (90.64999999999999, -116.5) rectangle (94.24999999999999,-119.1);
\draw(94.35, -116.5) node[anchor=north west,align=left] {Ordinary \\ differential\\ equations \\ and systems\\ on manifolds};
\draw (94.35, -116.5) rectangle (97.94999999999999,-119.1);
\draw(98.04999999999998, -116.5) node[anchor=north west,align=left] {Hysteresis\\ for ordinary\\ differential\\ equations};
\draw (98.04999999999998, -116.5) rectangle (101.64999999999998,-118.6);
\draw(66.24999999999999, -115.10000000000002) node[anchor=north west,align=left] {\large Differential equations in abstract spaces};
\draw (66.24999999999999, -115.10000000000002) rectangle (79.55999999999999,-118.30000000000003);
\draw(67.24999999999999, -116.10000000000002) node[anchor=north west,align=left] {Linear \\ differential \\ equations in \\ abstract spaces};
\draw (67.24999999999999, -116.10000000000002) rectangle (71.59999999999998,-118.20000000000002);
\draw(71.69999999999999, -116.10000000000002) node[anchor=north west,align=left] {Nonlinear \\ differential \\ equations in\\ abstract spaces};
\draw (71.69999999999999, -116.10000000000002) rectangle (76.04999999999998,-118.20000000000002);
\draw(76.14999999999998, -116.10000000000002) node[anchor=north west,align=left] {Evolution\\ inclusions};
\draw (76.14999999999998, -116.10000000000002) rectangle (79.24999999999997,-117.70000000000002);
\draw(105.64999999999999, -96.30000000000001) node[anchor=north west,align=left] {\large Asymptotic theory for ordinary differential equations};
\draw (105.64999999999999, -96.30000000000001) rectangle (125.25,-105.9);
\draw(106.64999999999999, -97.30000000000001) node[anchor=north west,align=left] {Asymptotic \\ expansions of \\ solutions to \\ ordinary \\ differential equations};
\draw (106.64999999999999, -97.30000000000001) rectangle (112.74999999999999,-99.9);
\draw(112.85, -97.30000000000001) node[anchor=north west,align=left] {Perturbations,\\ asymptotics \\ of solutions to\\ ordinary \\ differential equations};
\draw (112.85, -97.30000000000001) rectangle (118.94999999999999,-99.9);
\draw(119.04999999999998, -97.30000000000001) node[anchor=north west,align=left] {Singular \\ perturbations, general\\ theory for\\ ordinary \\ differential equations};
\draw (119.04999999999998, -97.30000000000001) rectangle (125.14999999999998,-99.9);
\draw(106.64999999999999, -100.00000000000001) node[anchor=north west,align=left] {Singular \\ perturbations, turning\\ point theory,\\ WKB methods for\\ ordinary \\ differential equations};
\draw (106.64999999999999, -100.00000000000001) rectangle (112.74999999999999,-103.10000000000001);
\draw(112.85, -100.00000000000001) node[anchor=north west,align=left] {Canard solutions\\ to \\ ordinary differential\\ equations};
\draw (112.85, -100.00000000000001) rectangle (118.69999999999999,-102.10000000000001);
\draw(118.79999999999998, -100.00000000000001) node[anchor=north west,align=left] {Methods of \\ nonstandard \\ analysis for \\ ordinary differential\\ equations};
\draw (118.79999999999998, -100.00000000000001) rectangle (124.64999999999998,-102.60000000000001);
\draw(106.64999999999999, -103.20000000000002) node[anchor=north west,align=left] {Multiple \\ scale methods\\ for ordinary\\ differential\\ equations};
\draw (106.64999999999999, -103.20000000000002) rectangle (110.49999999999999,-105.80000000000001);
\draw(66.24999999999999, -119.30000000000001) node[anchor=north west,align=left] {\large Ordinary differential operators};
\draw (66.24999999999999, -119.30000000000001) rectangle (79.64999999999998,-133.10000000000002);
\draw(67.24999999999999, -120.30000000000001) node[anchor=north west,align=left] {Eigenfunctions, \\ eigenfunction \\ expansions, completeness\\ of eigenfunctions\\ of ordinary\\ differential operators};
\draw (67.24999999999999, -120.30000000000001) rectangle (73.84999999999998,-123.4);
\draw(73.94999999999999, -120.30000000000001) node[anchor=north west,align=left] {General \\ spectral theory\\ of ordinary\\ differential\\ operators};
\draw (73.94999999999999, -120.30000000000001) rectangle (78.29999999999998,-122.9);
\draw(67.24999999999999, -123.50000000000001) node[anchor=north west,align=left] {Asymptotic distribution\\ of eigenvalues,\\ asymptotic \\ theory of eigenfunctions\\ for ordinary \\ differential operators};
\draw (67.24999999999999, -123.50000000000001) rectangle (73.84999999999998,-126.60000000000001);
\draw(73.94999999999999, -123.50000000000001) node[anchor=north west,align=left] {Nonlinear\\ ordinary \\ differential\\ operators};
\draw (73.94999999999999, -123.50000000000001) rectangle (77.54999999999998,-125.60000000000001);
\draw(67.24999999999999, -126.70000000000002) node[anchor=north west,align=left] {Numerical approximation\\ of eigenvalues\\ and of other \\ parts of the spectrum\\ of ordinary \\ differential operators};
\draw (67.24999999999999, -126.70000000000002) rectangle (73.59999999999998,-129.8);
\draw(73.69999999999999, -126.70000000000002) node[anchor=north west,align=left] {Particular \\ ordinary differential\\ operators\\ (Dirac, \\ one-dimensional \\ Schrödinger, etc.)};
\draw (73.69999999999999, -126.70000000000002) rectangle (79.54999999999998,-129.8);
\draw(67.24999999999999, -129.9) node[anchor=north west,align=left] {Eigenvalues, \\ estimation of \\ eigenvalues, upper and\\ lower bounds \\ of ordinary \\ differential operators};
\draw (67.24999999999999, -129.9) rectangle (73.34999999999998,-133.0);
\draw(73.44999999999999, -129.9) node[anchor=north west,align=left] {Scattering theory,\\ inverse \\ scattering involving\\ ordinary \\ differential operators};
\draw (73.44999999999999, -129.9) rectangle (79.54999999999998,-132.5);
\draw(139.72, -1) node[anchor=north west,align=left] {\LARGE Associative rings and algebras};
\draw (139.72, -1) rectangle (198.12,-48.1);
\draw(140.72, -2) node[anchor=north west,align=left] {\large Chain conditions, growth conditions, and other forms of finiteness for associative rings and algebras};
\draw (140.72, -2) rectangle (174.17000000000002,-8.9);
\draw(141.72, -3) node[anchor=north west,align=left] {Chain conditions\\ on \\ annihilators and \\ summands: \\ Goldie-type conditions};
\draw (141.72, -3) rectangle (147.82,-5.6);
\draw(147.92, -3) node[anchor=north west,align=left] {Noetherian \\ rings and \\ modules (associative\\ rings\\ and algebras)};
\draw (147.92, -3) rectangle (153.51999999999998,-5.6);
\draw(153.62, -3) node[anchor=north west,align=left] {Chain conditions\\ on other classes\\ of submodules,\\ ideals, subrings,\\ etc.; coherence\\ (associative\\ rings and algebras)};
\draw (153.62, -3) rectangle (158.97,-6.6);
\draw(159.07, -3) node[anchor=north west,align=left] {Finite rings\\ and \\ finite-dimensional\\ associative\\ algebras};
\draw (159.07, -3) rectangle (164.17,-5.6);
\draw(164.27, -3) node[anchor=north west,align=left] {Artinian \\ rings and \\ modules \\ (associative rings\\ and algebras)};
\draw (164.27, -3) rectangle (169.37,-5.6);
\draw(169.47, -3) node[anchor=north west,align=left] {Localization\\ and \\ associative \\ Noetherian rings};
\draw (169.47, -3) rectangle (174.07,-5.1);
\draw(141.72, -6.7) node[anchor=north west,align=left] {Growth \\ rate, \\ Gelfand-Kirillov\\ dimension};
\draw (141.72, -6.7) rectangle (146.32,-8.8);
\draw(174.27, -2) node[anchor=north west,align=left] {\large Associative rings and algebras with additional structure};
\draw (174.27, -2) rectangle (195.02,-11.1);
\draw(175.27, -3) node[anchor=north west,align=left] {Actions of \\ groups and \\ semigroups; invariant\\ theory \\ (associative \\ rings and algebras)};
\draw (175.27, -3) rectangle (181.12,-6.1);
\draw(181.22, -3) node[anchor=north west,align=left] {Valuations, \\ completions, formal \\ power series and \\ related constructions\\ (associative \\ rings and algebras)};
\draw (181.22, -3) rectangle (187.07,-6.1);
\draw(187.17000000000002, -3) node[anchor=north west,align=left] {Filtered \\ associative \\ rings; filtrational\\ and \\ graded techniques};
\draw (187.17000000000002, -3) rectangle (192.52,-5.6);
\draw(175.27, -6.2) node[anchor=north west,align=left] {Rings with \\ involution; Lie,\\ Jordan and other\\ nonassociative\\ structures};
\draw (175.27, -6.2) rectangle (179.87,-8.8);
\draw(179.97, -6.2) node[anchor=north west,align=left] {Automorphisms\\ and \\ endomorphisms};
\draw (179.97, -6.2) rectangle (183.82,-7.800000000000001);
\draw(183.92000000000002, -6.2) node[anchor=north west,align=left] {Graded rings\\ and modules\\ (associative\\ rings\\ and algebras)};
\draw (183.92000000000002, -6.2) rectangle (187.77,-8.8);
\draw(187.87, -6.2) node[anchor=north west,align=left] {Derivations,\\ actions\\ of Lie\\ algebras};
\draw (187.87, -6.2) rectangle (191.47,-8.3);
\draw(191.57000000000002, -6.2) node[anchor=north west,align=left] {“Super” \\ (or “skew”)\\ structure};
\draw (191.57000000000002, -6.2) rectangle (194.92000000000002,-7.800000000000001);
\draw(175.27, -8.9) node[anchor=north west,align=left] {Topological\\ and ordered\\ rings\\ and modules};
\draw (175.27, -8.9) rectangle (178.62,-11.0);
\draw(140.72, -9.0) node[anchor=north west,align=left] {\large History of associative\\ rings and algebras};
\draw (140.72, -9.0) rectangle (148.14,-10.1);
\draw(140.72, -11.2) node[anchor=north west,align=left] {\large Associative rings and algebras arising under various constructions};
\draw (140.72, -11.2) rectangle (165.52,-22.0);
\draw(141.72, -12.2) node[anchor=north west,align=left] {Rings arising\\ from \\ noncommutative algebraic\\ geometry};
\draw (141.72, -12.2) rectangle (148.32,-14.299999999999999);
\draw(148.42, -12.2) node[anchor=north west,align=left] {Associative rings\\ determined by \\ universal properties \\ (free algebras, \\ coproducts, adjunction\\ of inverses, etc.)};
\draw (148.42, -12.2) rectangle (154.51999999999998,-15.299999999999999);
\draw(154.62, -12.2) node[anchor=north west,align=left] {Associative \\ rings of \\ functions, subdirect\\ products,\\ sheaves of rings};
\draw (154.62, -12.2) rectangle (160.22,-14.799999999999999);
\draw(160.32, -12.2) node[anchor=north west,align=left] {Finite generation,\\ finite \\ presentability,\\ normal forms \\ (diamond lemma,\\ term-rewriting)};
\draw (160.32, -12.2) rectangle (165.42,-15.299999999999999);
\draw(141.72, -15.399999999999999) node[anchor=north west,align=left] {Rings of \\ differential \\ operators \\ (associative \\ algebraic aspects)};
\draw (141.72, -15.399999999999999) rectangle (146.82,-18.0);
\draw(146.92, -15.399999999999999) node[anchor=north west,align=left] {Torsion theories;\\ radicals on\\ module categories\\ (associative\\ algebraic aspects)};
\draw (146.92, -15.399999999999999) rectangle (152.01999999999998,-18.0);
\draw(152.12, -15.399999999999999) node[anchor=north west,align=left] {Twisted and\\ skew group\\ rings, \\ crossed products};
\draw (152.12, -15.399999999999999) rectangle (156.72,-17.5);
\draw(156.82, -15.399999999999999) node[anchor=north west,align=left] {Ordinary \\ and skew \\ polynomial \\ rings and \\ semigroup rings};
\draw (156.82, -15.399999999999999) rectangle (161.17,-18.0);
\draw(161.26999999999998, -15.399999999999999) node[anchor=north west,align=left] {Extensions\\ of associative\\ rings\\ by ideals};
\draw (161.26999999999998, -15.399999999999999) rectangle (165.36999999999998,-17.5);
\draw(141.72, -18.1) node[anchor=north west,align=left] {Associative\\ rings of \\ fractions and \\ localizations};
\draw (141.72, -18.1) rectangle (145.82,-20.200000000000003);
\draw(145.92, -18.1) node[anchor=north west,align=left] {Centralizing\\ and \\ normalizing\\ extensions};
\draw (145.92, -18.1) rectangle (149.51999999999998,-20.200000000000003);
\draw(149.62, -18.1) node[anchor=north west,align=left] {Universal \\ enveloping \\ algebras of\\ Lie algebras};
\draw (149.62, -18.1) rectangle (153.22,-20.200000000000003);
\draw(153.32, -18.1) node[anchor=north west,align=left] {Smash \\ products of\\ general \\ Hopf actions};
\draw (153.32, -18.1) rectangle (156.92,-20.200000000000003);
\draw(157.02, -18.1) node[anchor=north west,align=left] {Endomorphism\\ rings;\\ matrix rings};
\draw (157.02, -18.1) rectangle (160.62,-19.700000000000003);
\draw(160.72, -18.1) node[anchor=north west,align=left] {Deformations\\ of\\ associative\\ rings};
\draw (160.72, -18.1) rectangle (164.32,-20.200000000000003);
\draw(141.72, -20.3) node[anchor=north west,align=left] {Quadratic\\ and Koszul\\ algebras};
\draw (141.72, -20.3) rectangle (144.82,-21.900000000000002);
\draw(144.92, -20.3) node[anchor=north west,align=left] {Leavitt\\ path \\ algebras};
\draw (144.92, -20.3) rectangle (147.51999999999998,-21.900000000000002);
\draw(147.62, -20.3) node[anchor=north west,align=left] {Group\\ rings};
\draw (147.62, -20.3) rectangle (149.47,-21.400000000000002);
\draw(165.62, -11.2) node[anchor=north west,align=left] {\large Radicals and radical properties of associative rings};
\draw (165.62, -11.2) rectangle (184.92000000000002,-14.899999999999999);
\draw(166.62, -12.2) node[anchor=north west,align=left] {Jacobson\\ radical,\\ quasimultiplication};
\draw (166.62, -12.2) rectangle (171.97,-14.299999999999999);
\draw(172.07, -12.2) node[anchor=north west,align=left] {Nil and \\ nilpotent \\ radicals, sets,\\ ideals, \\ associative rings};
\draw (172.07, -12.2) rectangle (176.92,-14.799999999999999);
\draw(177.02, -12.2) node[anchor=north west,align=left] {General \\ radicals \\ and associative\\ rings};
\draw (177.02, -12.2) rectangle (181.37,-14.299999999999999);
\draw(181.47, -12.2) node[anchor=north west,align=left] {Prime and\\ semiprime\\ associative\\ rings};
\draw (181.47, -12.2) rectangle (184.82,-14.299999999999999);
\draw(165.62, -14.999999999999998) node[anchor=north west,align=left] {\large Representation theory of associative rings and algebras};
\draw (165.62, -14.999999999999998) rectangle (183.97,-20.9);
\draw(166.62, -15.999999999999998) node[anchor=north west,align=left] {Auslander-Reiten\\ sequences (almost\\ split sequences)\\ and \\ Auslander-Reiten quivers};
\draw (166.62, -15.999999999999998) rectangle (173.22,-18.599999999999998);
\draw(173.32, -15.999999999999998) node[anchor=north west,align=left] {Representation\\ type (finite,\\ tame, wild, \\ etc.) of associative\\ algebras};
\draw (173.32, -15.999999999999998) rectangle (178.92,-18.599999999999998);
\draw(179.02, -15.999999999999998) node[anchor=north west,align=left] {Representations\\ of orders,\\ lattices, \\ algebras over \\ commutative rings};
\draw (179.02, -15.999999999999998) rectangle (183.87,-18.599999999999998);
\draw(166.62, -18.7) node[anchor=north west,align=left] {Representations\\ of \\ associative \\ Artinian rings};
\draw (166.62, -18.7) rectangle (170.97,-20.8);
\draw(171.07, -18.7) node[anchor=north west,align=left] {Representations\\ of quivers\\ and partially\\ ordered sets};
\draw (171.07, -18.7) rectangle (175.42,-20.8);
\draw(175.52, -18.7) node[anchor=north west,align=left] {Cohen-Macaulay\\ modules\\ in associative\\ algebras};
\draw (175.52, -18.7) rectangle (179.62,-20.8);
\draw(185.02, -11.2) node[anchor=north west,align=left] {\large Associative algebras and orders};
\draw (185.02, -11.2) rectangle (196.67000000000002,-16.6);
\draw(186.02, -12.2) node[anchor=north west,align=left] {Separable \\ algebras (e.g., \\ quaternion algebras,\\ Azumaya \\ algebras, etc.)};
\draw (186.02, -12.2) rectangle (191.62,-14.799999999999999);
\draw(191.72, -12.2) node[anchor=north west,align=left] {Commutativeorders};
\draw (191.72, -12.2) rectangle (196.57,-13.799999999999999);
\draw(186.02, -14.899999999999999) node[anchor=north west,align=left] {Orders in\\ separable\\ algebras};
\draw (186.02, -14.899999999999999) rectangle (188.87,-16.5);
\draw(188.97, -14.899999999999999) node[anchor=north west,align=left] {Lattices\\ over\\ orders};
\draw (188.97, -14.899999999999999) rectangle (191.57,-16.5);
\draw(140.72, -22.1) node[anchor=north west,align=left] {\large Modules, bimodules and ideals in associative algebras};
\draw (140.72, -22.1) rectangle (159.82,-32.7);
\draw(141.72, -23.1) node[anchor=north west,align=left] {Structure and \\ classification for \\ modules, bimodules and\\ ideals (except \\ as in 16Gxx), direct\\ sum decomposition\\ and cancellation\\ in associative algebras)};
\draw (141.72, -23.1) rectangle (148.32,-27.200000000000003);
\draw(148.42, -23.1) node[anchor=north west,align=left] {Infinite-dimensional\\ simple \\ rings (except\\ as in 16Kxx)};
\draw (148.42, -23.1) rectangle (154.01999999999998,-25.700000000000003);
\draw(154.12, -23.1) node[anchor=north west,align=left] {Free, projective,\\ and flat\\ modules and\\ ideals in \\ associative algebras};
\draw (154.12, -23.1) rectangle (159.72,-25.700000000000003);
\draw(141.72, -27.3) node[anchor=north west,align=left] {Simple and \\ semisimple \\ modules, primitive\\ rings and \\ ideals in \\ associative algebras};
\draw (141.72, -27.3) rectangle (147.32,-30.400000000000002);
\draw(147.42, -27.3) node[anchor=north west,align=left] {Other classes\\ of modules\\ and ideals\\ in \\ associative algebras};
\draw (147.42, -27.3) rectangle (153.01999999999998,-29.900000000000002);
\draw(153.12, -27.3) node[anchor=north west,align=left] {Injective \\ modules, \\ self-injective \\ associative rings};
\draw (153.12, -27.3) rectangle (157.97,-29.400000000000002);
\draw(141.72, -30.5) node[anchor=north west,align=left] {General \\ module theory\\ in associative\\ algebras};
\draw (141.72, -30.5) rectangle (145.82,-32.6);
\draw(145.92, -30.5) node[anchor=north west,align=left] {Module \\ categories in\\ associative\\ algebras};
\draw (145.92, -30.5) rectangle (149.76999999999998,-32.6);
\draw(149.87, -30.5) node[anchor=north west,align=left] {Bimodules\\ in \\ associative\\ algebras};
\draw (149.87, -30.5) rectangle (153.22,-32.6);
\draw(153.32, -30.5) node[anchor=north west,align=left] {Ideals in\\ associative\\ algebras};
\draw (153.32, -30.5) rectangle (156.67,-32.1);
\draw(159.92000000000002, -22.1) node[anchor=north west,align=left] {\large Homological methods in associative algebras};
\draw (159.92000000000002, -22.1) rectangle (176.27,-34.900000000000006);
\draw(160.92000000000002, -23.1) node[anchor=north west,align=left] {Homological conditions\\ on associative\\ rings (generalizations\\ of regular,\\ Gorenstein, \\ Cohen-Macaulay rings, etc.)};
\draw (160.92000000000002, -23.1) rectangle (168.27,-26.200000000000003);
\draw(168.37, -23.1) node[anchor=north west,align=left] {Homological \\ functors on modules\\ (Tor, Ext,\\ etc.) in \\ associative algebras};
\draw (168.37, -23.1) rectangle (173.97,-25.700000000000003);
\draw(160.92000000000002, -26.3) node[anchor=north west,align=left] {(Co)homology \\ of rings and \\ associative \\ algebras (e.g., \\ Hochschild, cyclic,\\ dihedral, etc.)};
\draw (160.92000000000002, -26.3) rectangle (166.27,-29.400000000000002);
\draw(166.37, -26.3) node[anchor=north west,align=left] {Grothendieck\\ groups,\\ \(K\)-theory, etc.};
\draw (166.37, -26.3) rectangle (171.47,-28.400000000000002);
\draw(171.57000000000002, -26.3) node[anchor=north west,align=left] {Derived \\ categories \\ and associative\\ algebras};
\draw (171.57000000000002, -26.3) rectangle (175.92000000000002,-28.400000000000002);
\draw(160.92000000000002, -29.5) node[anchor=north west,align=left] {Differential \\ graded algebras \\ and applications\\ (associative \\ algebraic aspects)};
\draw (160.92000000000002, -29.5) rectangle (166.02,-32.1);
\draw(166.12, -29.5) node[anchor=north west,align=left] {von Neumann \\ regular rings and\\ generalizations\\ (associative \\ algebraic aspects)};
\draw (166.12, -29.5) rectangle (171.22,-32.1);
\draw(171.32000000000002, -29.5) node[anchor=north west,align=left] {Semihereditary\\ and hereditary\\ rings, free ideal\\ rings, Sylvester\\ rings, etc.};
\draw (171.32000000000002, -29.5) rectangle (176.17000000000002,-32.1);
\draw(160.92000000000002, -32.2) node[anchor=north west,align=left] {Syzygies, \\ resolutions,\\ complexes\\ in associative\\ algebras};
\draw (160.92000000000002, -32.2) rectangle (165.02,-34.800000000000004);
\draw(165.12, -32.2) node[anchor=north west,align=left] {Homological\\ dimension\\ in associative\\ algebras};
\draw (165.12, -32.2) rectangle (169.22,-34.300000000000004);
\draw(176.37, -22.1) node[anchor=north west,align=left] {\large Hopf algebras, quantum groups and related topics};
\draw (176.37, -22.1) rectangle (192.67000000000002,-27.5);
\draw(177.37, -23.1) node[anchor=north west,align=left] {Ring-theoretic\\ aspects\\ of quantum\\ groups};
\draw (177.37, -23.1) rectangle (181.47,-25.200000000000003);
\draw(181.57, -23.1) node[anchor=north west,align=left] {Connections\\ of Hopf \\ algebras with\\ combinatorics};
\draw (181.57, -23.1) rectangle (185.42,-25.200000000000003);
\draw(185.52, -23.1) node[anchor=north west,align=left] {Hopf \\ algebras and\\ their \\ applications};
\draw (185.52, -23.1) rectangle (189.12,-25.200000000000003);
\draw(189.22, -23.1) node[anchor=north west,align=left] {Yang-Baxter\\ equations};
\draw (189.22, -23.1) rectangle (192.57,-24.700000000000003);
\draw(177.37, -25.3) node[anchor=north west,align=left] {Bialgebras};
\draw (177.37, -25.3) rectangle (180.47,-26.400000000000002);
\draw(180.57, -25.3) node[anchor=north west,align=left] {Coalgebras\\ and \\ comodules;\\ corings};
\draw (180.57, -25.3) rectangle (183.67,-27.400000000000002);
\draw(176.37, -27.6) node[anchor=north west,align=left] {\large Division rings and semisimple Artin rings};
\draw (176.37, -27.6) rectangle (191.47,-30.8);
\draw(177.37, -28.6) node[anchor=north west,align=left] {Infinite-dimensional\\ and general\\ division rings};
\draw (177.37, -28.6) rectangle (182.97,-30.700000000000003);
\draw(183.07, -28.6) node[anchor=north west,align=left] {Finite-dimensional\\ division\\ rings};
\draw (183.07, -28.6) rectangle (188.17,-30.700000000000003);
\draw(188.27, -28.6) node[anchor=north west,align=left] {Brauer \\ groups \\ (algebraic\\ aspects)};
\draw (188.27, -28.6) rectangle (191.37,-30.700000000000003);
\draw(176.37, -30.900000000000006) node[anchor=north west,align=left] {\large Computational aspects of associative rings};
\draw (176.37, -30.900000000000006) rectangle (189.99,-34.60000000000001);
\draw(177.37, -31.900000000000006) node[anchor=north west,align=left] {Computational\\ aspects\\ of associative\\ rings \\ (general theory)};
\draw (177.37, -31.900000000000006) rectangle (181.97,-34.50000000000001);
\draw(182.07, -31.900000000000006) node[anchor=north west,align=left] {Gröbner-Shirshov\\ bases};
\draw (182.07, -31.900000000000006) rectangle (186.67,-33.50000000000001);
\draw(192.77, -22.1) node[anchor=north west,align=left] {\large Generalizations};
\draw (192.77, -22.1) rectangle (198.02,-28.400000000000002);
\draw(193.77, -23.1) node[anchor=north west,align=left] {Hyperrings};
\draw (193.77, -23.1) rectangle (196.87,-24.200000000000003);
\draw(193.77, -24.3) node[anchor=north west,align=left] {Near-rings};
\draw (193.77, -24.3) rectangle (196.87,-25.400000000000002);
\draw(193.77, -25.5) node[anchor=north west,align=left] {\(\Gamma\)\\ and fuzzy\\ structures};
\draw (193.77, -25.5) rectangle (196.87,-27.1);
\draw(193.77, -27.200000000000003) node[anchor=north west,align=left] {Semirings};
\draw (193.77, -27.200000000000003) rectangle (196.62,-28.300000000000004);
\draw(140.72, -35.0) node[anchor=north west,align=left] {\large Rings with polynomial identity};
\draw (140.72, -35.0) rectangle (152.12,-44.1);
\draw(141.72, -36.0) node[anchor=north west,align=left] {Other kinds of\\ identities \\ (generalized \\ polynomial, rational,\\ involution)};
\draw (141.72, -36.0) rectangle (147.57,-38.6);
\draw(147.67, -36.0) node[anchor=north west,align=left] {Functional\\ identities\\ (associative\\ rings\\ and algebras)};
\draw (147.67, -36.0) rectangle (151.51999999999998,-38.6);
\draw(141.72, -38.7) node[anchor=north west,align=left] {\(T\)-ideals,\\ identities,\\ varieties of\\ associative \\ rings and algebras};
\draw (141.72, -38.7) rectangle (146.82,-41.300000000000004);
\draw(146.92, -38.7) node[anchor=north west,align=left] {Trace rings \\ and invariant\\ theory \\ (associative rings\\ and algebras)};
\draw (146.92, -38.7) rectangle (152.01999999999998,-41.300000000000004);
\draw(141.72, -41.4) node[anchor=north west,align=left] {Semiprime p.i.\\ rings, rings\\ embeddable in\\ matrices over\\ commutative rings};
\draw (141.72, -41.4) rectangle (146.57,-44.0);
\draw(146.67, -41.4) node[anchor=north west,align=left] {Identities \\ other than \\ those of matrices\\ over \\ commutative rings};
\draw (146.67, -41.4) rectangle (151.51999999999998,-44.0);
\draw(152.22, -35.0) node[anchor=north west,align=left] {\large Conditions on elements};
\draw (152.22, -35.0) rectangle (163.12,-48.0);
\draw(153.22, -36.0) node[anchor=north west,align=left] {Integral \\ domains (associative\\ rings\\ and algebras)};
\draw (153.22, -36.0) rectangle (158.82,-38.1);
\draw(158.92, -36.0) node[anchor=north west,align=left] {Ore rings, \\ multiplicative\\ sets, Ore\\ localization};
\draw (158.92, -36.0) rectangle (163.01999999999998,-38.1);
\draw(153.22, -38.2) node[anchor=north west,align=left] {Idempotent \\ elements \\ (associative rings\\ and algebras)};
\draw (153.22, -38.2) rectangle (158.32,-40.300000000000004);
\draw(158.42, -38.2) node[anchor=north west,align=left] {Divisibility,\\ noncommutative\\ UFDs};
\draw (158.42, -38.2) rectangle (162.51999999999998,-40.300000000000004);
\draw(153.22, -40.4) node[anchor=north west,align=left] {Center, normalizer\\ (invariant\\ elements) \\ (associative rings\\ and algebras)};
\draw (153.22, -40.4) rectangle (158.32,-43.0);
\draw(158.42, -40.4) node[anchor=north west,align=left] {Units, groups\\ of units\\ (associative\\ rings and\\ algebras)};
\draw (158.42, -40.4) rectangle (162.26999999999998,-43.0);
\draw(153.22, -43.1) node[anchor=north west,align=left] {Generalizations\\ of commutativity\\ (associative rings\\ and algebras)};
\draw (153.22, -43.1) rectangle (158.32,-45.7);
\draw(153.22, -45.8) node[anchor=north west,align=left] {Generalized \\ inverses \\ (associative rings\\ and algebras)};
\draw (153.22, -45.8) rectangle (158.32,-47.9);
\draw(163.22, -35.0) node[anchor=north west,align=left] {\large Local rings and generalizations};
\draw (163.22, -35.0) rectangle (173.43,-38.7);
\draw(164.22, -36.0) node[anchor=north west,align=left] {Quasi-Frobenius\\ rings};
\draw (164.22, -36.0) rectangle (168.57,-37.6);
\draw(168.67, -36.0) node[anchor=north west,align=left] {Noncommutative\\ local \\ and semilocal\\ rings, \\ perfect rings};
\draw (168.67, -36.0) rectangle (172.76999999999998,-38.6);
\draw(163.22, -38.800000000000004) node[anchor=north west,align=left] {\large General and miscellaneous};
\draw (163.22, -38.800000000000004) rectangle (171.57,-45.2);
\draw(164.22, -39.800000000000004) node[anchor=north west,align=left] {Category-theoretic\\ methods\\ and results\\ in associative\\ algebras \\ (except as in 16D90)};
\draw (164.22, -39.800000000000004) rectangle (169.82,-42.900000000000006);
\draw(164.22, -43.00000000000001) node[anchor=north west,align=left] {Applications\\ of logic\\ in associative\\ algebras};
\draw (164.22, -43.00000000000001) rectangle (168.32,-45.10000000000001);
\draw(198.22, -1) node[anchor=north west,align=left] {\LARGE Number theory};
\draw (198.22, -1) rectangle (246.02,-109.30000000000001);
\draw(199.22, -2) node[anchor=north west,align=left] {\large Probabilistic theory: distribution modulo \(1\); metric theory of algorithms};
\draw (199.22, -2) rectangle (225.87,-8.9);
\draw(200.22, -3) node[anchor=north west,align=left] {Normal numbers,\\ radix expansions,\\ Pisot \\ numbers, Salem \\ numbers, good \\ lattice points, etc.};
\draw (200.22, -3) rectangle (205.82,-6.1);
\draw(205.92, -3) node[anchor=north west,align=left] {Metric theory\\ of other \\ algorithms and \\ expansions; \\ measure and \\ Hausdorff dimension};
\draw (205.92, -3) rectangle (211.26999999999998,-6.1);
\draw(211.37, -3) node[anchor=north west,align=left] {Harmonic analysis\\ and almost\\ periodicity \\ in probabilistic\\ number theory};
\draw (211.37, -3) rectangle (216.22,-5.6);
\draw(216.32, -3) node[anchor=north west,align=left] {Well-distributed\\ sequences\\ and \\ other variations};
\draw (216.32, -3) rectangle (220.92,-5.1);
\draw(221.02, -3) node[anchor=north west,align=left] {Irregularities\\ of \\ distribution,\\ discrepancy};
\draw (221.02, -3) rectangle (225.12,-5.1);
\draw(200.22, -6.2) node[anchor=north west,align=left] {Diophantine\\ approximation\\ in \\ probabilistic \\ number theory};
\draw (200.22, -6.2) rectangle (204.32,-8.8);
\draw(204.42, -6.2) node[anchor=north west,align=left] {Pseudo-random\\ numbers;\\ Monte \\ Carlo methods};
\draw (204.42, -6.2) rectangle (208.26999999999998,-8.3);
\draw(208.37, -6.2) node[anchor=north west,align=left] {Arithmetic \\ functions in\\ probabilistic\\ number theory};
\draw (208.37, -6.2) rectangle (212.22,-8.3);
\draw(212.32, -6.2) node[anchor=north west,align=left] {General \\ theory of \\ distribution\\ modulo \(1\)};
\draw (212.32, -6.2) rectangle (215.92,-8.3);
\draw(216.02, -6.2) node[anchor=north west,align=left] {Continuous,\\ \(p\)-adic\\ and abstract\\ analogues};
\draw (216.02, -6.2) rectangle (219.62,-8.3);
\draw(219.72, -6.2) node[anchor=north west,align=left] {Metric \\ theory of \\ continued\\ fractions};
\draw (219.72, -6.2) rectangle (222.82,-8.3);
\draw(222.92, -6.2) node[anchor=north west,align=left] {Special\\ sequences};
\draw (222.92, -6.2) rectangle (225.76999999999998,-7.800000000000001);
\draw(225.97, -2) node[anchor=north west,align=left] {\large Arithmetic algebraic geometry (Diophantine geometry)};
\draw (225.97, -2) rectangle (245.77,-13.3);
\draw(226.97, -3) node[anchor=north west,align=left] {\(L\)-functions\\ of varieties\\ over global\\ fields; \\ Birch-Swinnerton-Dyer\\ conjecture};
\draw (226.97, -3) rectangle (232.82,-6.1);
\draw(232.92, -3) node[anchor=north west,align=left] {Drinfel’d \\ modules; \\ higher-dimensional\\ motives, etc.};
\draw (232.92, -3) rectangle (238.01999999999998,-5.1);
\draw(238.12, -3) node[anchor=north west,align=left] {Arithmetic \\ aspects of dessins\\ d’enfants,\\ Belyĭ theory};
\draw (238.12, -3) rectangle (243.22,-5.1);
\draw(243.32, -3) node[anchor=north west,align=left] {Heights};
\draw (243.32, -3) rectangle (245.67,-4.1);
\draw(226.97, -6.2) node[anchor=north west,align=left] {Complex \\ multiplication\\ and \\ moduli of \\ abelian varieties};
\draw (226.97, -6.2) rectangle (231.82,-8.8);
\draw(231.92, -6.2) node[anchor=north west,align=left] {Arithmetic \\ aspects of \\ modular and \\ Shimura varieties};
\draw (231.92, -6.2) rectangle (236.76999999999998,-8.3);
\draw(236.87, -6.2) node[anchor=north west,align=left] {Curves of \\ arbitrary genus\\ or genus \\ \(\ne 1\) over\\ global fields};
\draw (236.87, -6.2) rectangle (241.22,-8.8);
\draw(241.32, -6.2) node[anchor=north west,align=left] {Polylogarithms\\ and \\ relations with\\ \(K\)-theory};
\draw (241.32, -6.2) rectangle (245.42,-8.3);
\draw(226.97, -8.9) node[anchor=north west,align=left] {Elliptic\\ curves\\ over \\ global fields};
\draw (226.97, -8.9) rectangle (230.82,-11.0);
\draw(230.92, -8.9) node[anchor=north west,align=left] {Elliptic\\ and \\ modular units};
\draw (230.92, -8.9) rectangle (234.76999999999998,-10.5);
\draw(234.87, -8.9) node[anchor=north west,align=left] {Varieties\\ over \\ global fields};
\draw (234.87, -8.9) rectangle (238.72,-10.5);
\draw(238.82, -8.9) node[anchor=north west,align=left] {Abelian \\ varieties\\ of dimension\\ \(> 1\)};
\draw (238.82, -8.9) rectangle (242.42,-11.0);
\draw(242.52, -8.9) node[anchor=north west,align=left] {Elliptic\\ curves\\ over local\\ fields};
\draw (242.52, -8.9) rectangle (245.62,-11.0);
\draw(226.97, -11.100000000000001) node[anchor=north west,align=left] {Curves \\ over finite\\ and \\ local fields};
\draw (226.97, -11.100000000000001) rectangle (230.57,-13.200000000000001);
\draw(230.67, -11.100000000000001) node[anchor=north west,align=left] {Varieties\\ over \\ finite and \\ local fields};
\draw (230.67, -11.100000000000001) rectangle (234.26999999999998,-13.200000000000001);
\draw(234.37, -11.100000000000001) node[anchor=north west,align=left] {Geometric\\ class \\ field theory};
\draw (234.37, -11.100000000000001) rectangle (237.97,-12.700000000000001);
\draw(238.07, -11.100000000000001) node[anchor=north west,align=left] {Arithmetic\\ mirror\\ symmetry};
\draw (238.07, -11.100000000000001) rectangle (241.17,-12.700000000000001);
\draw(199.22, -9.0) node[anchor=north west,align=left] {\large Miscellaneous applications of number theory};
\draw (199.22, -9.0) rectangle (213.15,-12.2);
\draw(200.22, -10.0) node[anchor=north west,align=left] {Miscellaneous\\ applications of\\ number theory};
\draw (200.22, -10.0) rectangle (204.57,-12.1);
\draw(199.22, -13.4) node[anchor=north west,align=left] {\large Algebraic number theory: local and \(p\)-adic fields};
\draw (199.22, -13.4) rectangle (220.27,-22.0);
\draw(200.22, -14.4) node[anchor=north west,align=left] {Non-Archimedeandynamical\\ systems};
\draw (200.22, -14.4) rectangle (206.82,-16.5);
\draw(206.92, -14.4) node[anchor=north west,align=left] {Other analytic\\ theory (analogues\\ of beta and\\ gamma functions,\\ \(p\)-adic \\ integration, etc.)};
\draw (206.92, -14.4) rectangle (212.01999999999998,-17.5);
\draw(212.12, -14.4) node[anchor=north west,align=left] {Langlands-Weil\\ conjectures,\\ nonabelian class\\ field theory};
\draw (212.12, -14.4) rectangle (216.72,-17.0);
\draw(216.82, -14.4) node[anchor=north west,align=left] {Galois\\ theory};
\draw (216.82, -14.4) rectangle (218.92,-15.5);
\draw(216.82, -15.600000000000001) node[anchor=north west,align=left] {Polynomials};
\draw (216.82, -15.600000000000001) rectangle (220.17,-16.700000000000003);
\draw(200.22, -17.6) node[anchor=north west,align=left] {Integral\\ representations};
\draw (200.22, -17.6) rectangle (204.57,-19.200000000000003);
\draw(204.67, -17.6) node[anchor=north west,align=left] {Zeta \\ functions \\ and \\ \(L\)-functions};
\draw (204.67, -17.6) rectangle (209.01999999999998,-19.700000000000003);
\draw(209.12, -17.6) node[anchor=north west,align=left] {Algebras and\\ orders, \\ and their \\ zeta functions};
\draw (209.12, -17.6) rectangle (213.22,-19.700000000000003);
\draw(213.32, -17.6) node[anchor=north west,align=left] {Prehomogeneous\\ vector spaces};
\draw (213.32, -17.6) rectangle (217.42,-19.200000000000003);
\draw(200.22, -19.8) node[anchor=north west,align=left] {Class field\\ theory; \\ \(p\)-adic \\ formal groups};
\draw (200.22, -19.8) rectangle (204.07,-21.900000000000002);
\draw(204.17, -19.8) node[anchor=north west,align=left] {Ramification\\ and\\ extension\\ theory};
\draw (204.17, -19.8) rectangle (207.76999999999998,-21.900000000000002);
\draw(207.87, -19.8) node[anchor=north west,align=left] {\(K\)-theory\\ of \\ local fields};
\draw (207.87, -19.8) rectangle (211.47,-21.400000000000002);
\draw(211.57, -19.8) node[anchor=north west,align=left] {Other \\ nonanalytic\\ theory};
\draw (211.57, -19.8) rectangle (214.92,-21.400000000000002);
\draw(215.02, -19.8) node[anchor=north west,align=left] {Galois\\ cohomology};
\draw (215.02, -19.8) rectangle (218.12,-21.400000000000002);
\draw(220.37, -13.4) node[anchor=north west,align=left] {\large Finite fields and commutative rings (number-theoretic aspects)};
\draw (220.37, -13.4) rectangle (241.42000000000002,-19.8);
\draw(221.37, -14.4) node[anchor=north west,align=left] {Structure \\ theory for finite\\ fields and\\ commutative \\ rings \\ (number-theoretic aspects)};
\draw (221.37, -14.4) rectangle (228.47,-17.5);
\draw(228.57, -14.4) node[anchor=north west,align=left] {Algebraic \\ coding theory;\\ cryptography\\ (number-theoretic\\ aspects)};
\draw (228.57, -14.4) rectangle (233.42,-17.0);
\draw(233.52, -14.4) node[anchor=north west,align=left] {Polynomials\\ over \\ finite fields};
\draw (233.52, -14.4) rectangle (237.37,-16.0);
\draw(237.47, -14.4) node[anchor=north west,align=left] {Arithmetic\\ theory of \\ polynomial \\ rings over \\ finite fields};
\draw (237.47, -14.4) rectangle (241.32,-17.0);
\draw(221.37, -17.6) node[anchor=north west,align=left] {Exponential\\ sums};
\draw (221.37, -17.6) rectangle (224.72,-18.700000000000003);
\draw(224.82, -17.6) node[anchor=north west,align=left] {Finite \\ upper \\ half-planes};
\draw (224.82, -17.6) rectangle (228.17,-19.200000000000003);
\draw(228.27, -17.6) node[anchor=north west,align=left] {Other \\ character\\ sums and\\ Gauss sums};
\draw (228.27, -17.6) rectangle (231.37,-19.700000000000003);
\draw(231.47, -17.6) node[anchor=north west,align=left] {Cyclotomy};
\draw (231.47, -17.6) rectangle (234.32,-18.700000000000003);
\draw(220.37, -19.900000000000002) node[anchor=north west,align=left] {\large History of \\ number theory};
\draw (220.37, -19.900000000000002) rectangle (225.0,-21.000000000000004);
\draw(199.22, -22.1) node[anchor=north west,align=left] {\large Diophantine approximation, transcendental number theory};
\draw (199.22, -22.1) rectangle (219.97,-35.1);
\draw(200.22, -23.1) node[anchor=north west,align=left] {Transcendence\\ theory\\ of elliptic\\ and \\ abelian functions};
\draw (200.22, -23.1) rectangle (205.07,-25.700000000000003);
\draw(205.17, -23.1) node[anchor=north west,align=left] {Transcendence\\ theory \\ of other \\ special functions};
\draw (205.17, -23.1) rectangle (210.01999999999998,-25.200000000000003);
\draw(210.12, -23.1) node[anchor=north west,align=left] {Small \\ fractional parts\\ of polynomials\\ and \\ generalizations};
\draw (210.12, -23.1) rectangle (214.72,-25.700000000000003);
\draw(214.82, -23.1) node[anchor=north west,align=left] {Number-theoretic\\ analogues\\ of methods \\ in Nevanlinna\\ theory (work\\ of Vojta et al.)};
\draw (214.82, -23.1) rectangle (219.42,-26.200000000000003);
\draw(200.22, -26.3) node[anchor=north west,align=left] {Markov and\\ Lagrange \\ spectra and \\ generalizations};
\draw (200.22, -26.3) rectangle (204.57,-28.400000000000002);
\draw(204.67, -26.3) node[anchor=north west,align=left] {Approximation\\ in \\ non-Archimedean\\ valuations};
\draw (204.67, -26.3) rectangle (209.01999999999998,-28.400000000000002);
\draw(209.12, -26.3) node[anchor=north west,align=left] {Continued\\ fractions\\ and \\ generalizations};
\draw (209.12, -26.3) rectangle (213.47,-28.400000000000002);
\draw(213.57, -26.3) node[anchor=north west,align=left] {Homogeneous\\ approximation \\ to one number};
\draw (213.57, -26.3) rectangle (217.67,-28.400000000000002);
\draw(217.76999999999998, -26.3) node[anchor=north west,align=left] {Metric\\ theory};
\draw (217.76999999999998, -26.3) rectangle (219.86999999999998,-27.400000000000002);
\draw(200.22, -28.5) node[anchor=north west,align=left] {Simultaneous\\ homogeneous \\ approximation,\\ linear forms};
\draw (200.22, -28.5) rectangle (204.32,-30.6);
\draw(204.42, -28.5) node[anchor=north west,align=left] {Irrationality;\\ linear\\ independence\\ over a field};
\draw (204.42, -28.5) rectangle (208.51999999999998,-30.6);
\draw(208.62, -28.5) node[anchor=north west,align=left] {Linear forms\\ in \\ logarithms; \\ Baker’s method};
\draw (208.62, -28.5) rectangle (212.72,-30.6);
\draw(212.82, -28.5) node[anchor=north west,align=left] {Approximation\\ by \\ numbers from\\ a fixed field};
\draw (212.82, -28.5) rectangle (216.67,-30.6);
\draw(216.77, -28.5) node[anchor=north west,align=left] {Results\\ involving\\ abelian\\ varieties};
\draw (216.77, -28.5) rectangle (219.62,-30.6);
\draw(200.22, -30.700000000000003) node[anchor=north west,align=left] {Inhomogeneous\\ linear forms};
\draw (200.22, -30.700000000000003) rectangle (204.07,-32.300000000000004);
\draw(204.17, -30.700000000000003) node[anchor=north west,align=left] {Approximation\\ to\\ algebraic\\ numbers};
\draw (204.17, -30.700000000000003) rectangle (208.01999999999998,-32.800000000000004);
\draw(208.12, -30.700000000000003) node[anchor=north west,align=left] {Transcendence\\ (general\\ theory)};
\draw (208.12, -30.700000000000003) rectangle (211.97,-32.300000000000004);
\draw(212.07, -30.700000000000003) node[anchor=north west,align=left] {Measures of\\ irrationality\\ and of\\ transcendence};
\draw (212.07, -30.700000000000003) rectangle (215.92,-32.800000000000004);
\draw(216.02, -30.700000000000003) node[anchor=north west,align=left] {Algebraic\\ independence;\\ Gel’fond’s\\ method};
\draw (216.02, -30.700000000000003) rectangle (219.87,-32.800000000000004);
\draw(200.22, -32.900000000000006) node[anchor=north west,align=left] {Transcendence\\ theory of\\ Drinfel’d and\\ \(t\)-modules};
\draw (200.22, -32.900000000000006) rectangle (204.07,-35.00000000000001);
\draw(204.17, -32.900000000000006) node[anchor=north west,align=left] {Diophantine\\ inequalities};
\draw (204.17, -32.900000000000006) rectangle (207.76999999999998,-34.50000000000001);
\draw(207.87, -32.900000000000006) node[anchor=north west,align=left] {Distribution\\ modulo one};
\draw (207.87, -32.900000000000006) rectangle (211.47,-34.50000000000001);
\draw(211.57, -32.900000000000006) node[anchor=north west,align=left] {Schmidt \\ Subspace \\ Theorem and \\ applications};
\draw (211.57, -32.900000000000006) rectangle (215.17,-35.00000000000001);
\draw(220.07, -22.1) node[anchor=north west,align=left] {\large Discontinuous groups and automorphic forms};
\draw (220.07, -22.1) rectangle (236.17,-51.1);
\draw(221.07, -23.1) node[anchor=north west,align=left] {Automorphic forms on\\ \(\mbox{GL}(2)\); \\ Hilbert and Hilbert-Siegel\\ modular groups\\ and their modular \\ and automorphic forms;\\ Hilbert modular surfaces};
\draw (221.07, -23.1) rectangle (228.17,-26.700000000000003);
\draw(228.26999999999998, -23.1) node[anchor=north west,align=left] {Representation-theoretic\\ methods;\\ automorphic\\ representations\\ over local and\\ global fields};
\draw (228.26999999999998, -23.1) rectangle (234.86999999999998,-26.200000000000003);
\draw(221.07, -26.8) node[anchor=north west,align=left] {Special values \\ of automorphic \\ \(L\)-series, periods\\ of automorphic\\ forms, cohomology,\\ modular symbols};
\draw (221.07, -26.8) rectangle (226.92,-29.900000000000002);
\draw(227.01999999999998, -26.8) node[anchor=north west,align=left] {Langlands \\ \(L\)-functions;\\ one variable\\ Dirichlet \\ series and \\ functional equations};
\draw (227.01999999999998, -26.8) rectangle (232.61999999999998,-29.900000000000002);
\draw(232.72, -26.8) node[anchor=north west,align=left] {Modular\\ and \\ automorphic\\ functions};
\draw (232.72, -26.8) rectangle (236.07,-28.900000000000002);
\draw(221.07, -30.0) node[anchor=north west,align=left] {Other groups \\ and their modular\\ and automorphic\\ forms \\ (several variables)};
\draw (221.07, -30.0) rectangle (226.42,-32.6);
\draw(226.51999999999998, -30.0) node[anchor=north west,align=left] {Hecke-Petersson\\ operators,\\ differential\\ operators \\ (several variables)};
\draw (226.51999999999998, -30.0) rectangle (231.86999999999998,-32.6);
\draw(231.97, -30.0) node[anchor=north west,align=left] {Relations \\ with algebraic\\ geometry\\ and topology};
\draw (231.97, -30.0) rectangle (236.07,-32.1);
\draw(221.07, -32.7) node[anchor=north west,align=left] {Dirichlet series\\ in several complex\\ variables \\ associated to \\ automorphic forms; \\ Weyl group multiple\\ Dirichlet series};
\draw (221.07, -32.7) rectangle (226.42,-36.300000000000004);
\draw(226.51999999999998, -32.7) node[anchor=north west,align=left] {Siegel modular\\ groups; Siegel\\ and Hilbert-Siegel\\ modular and\\ automorphic forms};
\draw (226.51999999999998, -32.7) rectangle (231.61999999999998,-35.300000000000004);
\draw(231.72, -32.7) node[anchor=north west,align=left] {Holomorphic\\ modular \\ forms of \\ integral weight};
\draw (231.72, -32.7) rectangle (236.07,-34.800000000000004);
\draw(221.07, -36.400000000000006) node[anchor=north west,align=left] {Structure of\\ modular groups\\ and \\ generalizations; \\ arithmetic groups};
\draw (221.07, -36.400000000000006) rectangle (225.92,-39.00000000000001);
\draw(226.01999999999998, -36.400000000000006) node[anchor=north west,align=left] {Automorphic\\ forms and \\ their relations\\ with \\ perfectoid spaces};
\draw (226.01999999999998, -36.400000000000006) rectangle (230.86999999999998,-39.00000000000001);
\draw(230.97, -36.400000000000006) node[anchor=north west,align=left] {Modular \\ correspondences,\\ etc.};
\draw (230.97, -36.400000000000006) rectangle (235.57,-38.00000000000001);
\draw(221.07, -39.1) node[anchor=north west,align=left] {Relationship\\ to Lie algebras\\ and finite\\ simple groups};
\draw (221.07, -39.1) rectangle (225.42,-41.2);
\draw(225.51999999999998, -39.1) node[anchor=north west,align=left] {Hecke-Petersson\\ operators,\\ differential\\ operators\\ (one variable)};
\draw (225.51999999999998, -39.1) rectangle (229.86999999999998,-41.7);
\draw(229.97, -39.1) node[anchor=north west,align=left] {Theta series;\\ Weil \\ representation;\\ theta \\ correspondences};
\draw (229.97, -39.1) rectangle (234.32,-41.7);
\draw(221.07, -41.8) node[anchor=north west,align=left] {Congruences \\ for modular and\\ \(p\)-adic\\ modular forms};
\draw (221.07, -41.8) rectangle (225.42,-43.9);
\draw(225.51999999999998, -41.8) node[anchor=north west,align=left] {Galois\\ representations};
\draw (225.51999999999998, -41.8) rectangle (229.86999999999998,-43.4);
\draw(229.97, -41.8) node[anchor=north west,align=left] {Fourier \\ coefficients\\ of automorphic\\ forms};
\draw (229.97, -41.8) rectangle (234.07,-43.9);
\draw(221.07, -44.0) node[anchor=north west,align=left] {Forms of \\ half-integer\\ weight; \\ nonholomorphic\\ modular forms};
\draw (221.07, -44.0) rectangle (225.17,-46.6);
\draw(225.26999999999998, -44.0) node[anchor=north west,align=left] {Dedekind\\ eta function,\\ Dedekind sums};
\draw (225.26999999999998, -44.0) rectangle (229.11999999999998,-46.1);
\draw(229.22, -44.0) node[anchor=north west,align=left] {Modular forms\\ associated\\ to Drinfel’d\\ modules};
\draw (229.22, -44.0) rectangle (233.07,-46.1);
\draw(233.17, -44.0) node[anchor=north west,align=left] {Jacobi\\ forms};
\draw (233.17, -44.0) rectangle (235.26999999999998,-45.1);
\draw(221.07, -46.7) node[anchor=north west,align=left] {Spectral \\ theory; trace\\ formulas \\ (e.g., that\\ of Selberg)};
\draw (221.07, -46.7) rectangle (224.92,-49.300000000000004);
\draw(225.01999999999998, -46.7) node[anchor=north west,align=left] {Cohomology\\ of arithmetic\\ groups};
\draw (225.01999999999998, -46.7) rectangle (228.86999999999998,-48.300000000000004);
\draw(228.97, -46.7) node[anchor=north west,align=left] {Automorphic\\ forms, \\ one variable};
\draw (228.97, -46.7) rectangle (232.57,-48.300000000000004);
\draw(221.07, -49.4) node[anchor=north west,align=left] {\(p\)-adic\\ theory, \\ local fields};
\draw (221.07, -49.4) rectangle (224.67,-51.0);
\draw(199.22, -35.2) node[anchor=north west,align=left] {\large Zeta and \(L\)-functions: analytic theory};
\draw (199.22, -35.2) rectangle (214.57,-47.5);
\draw(200.22, -36.2) node[anchor=north west,align=left] {Selberg zeta functions\\ and regularized \\ determinants; applications\\ to spectral theory,\\ Dirichlet series, \\ Eisenstein series, etc.\\ (explicit formulas)};
\draw (200.22, -36.2) rectangle (207.32,-39.800000000000004);
\draw(207.42, -36.2) node[anchor=north west,align=left] {Nonreal zeros \\ of \(\zeta (s)\)\\ and \(L(s,  \chi)\); Riemann \\ and other hypotheses};
\draw (207.42, -36.2) rectangle (213.01999999999998,-38.800000000000004);
\draw(200.22, -39.900000000000006) node[anchor=north west,align=left] {Zeta and \\ \(L\)-functions\\ in characteristic\\ \(p\)};
\draw (200.22, -39.900000000000006) rectangle (205.07,-42.00000000000001);
\draw(205.17, -39.900000000000006) node[anchor=north west,align=left] {Multiple \\ Dirichlet series\\ and zeta \\ functions and \\ multizeta values};
\draw (205.17, -39.900000000000006) rectangle (209.76999999999998,-42.50000000000001);
\draw(209.87, -39.900000000000006) node[anchor=north west,align=left] {Other \\ Dirichlet series\\ and zeta\\ functions};
\draw (209.87, -39.900000000000006) rectangle (214.47,-42.00000000000001);
\draw(200.22, -42.6) node[anchor=north west,align=left] {\(\zeta  (s)\) and \\ \(L(s, \chi)\)};
\draw (200.22, -42.6) rectangle (204.32,-44.2);
\draw(204.42, -42.6) node[anchor=north west,align=left] {Real zeros\\ of \(L(s,  \chi)\); \\ results on \\ \(L(1, \chi)\)};
\draw (204.42, -42.6) rectangle (208.51999999999998,-45.2);
\draw(208.62, -42.6) node[anchor=north west,align=left] {Hurwitz and\\ Lerch \\ zeta functions};
\draw (208.62, -42.6) rectangle (212.72,-44.2);
\draw(200.22, -45.300000000000004) node[anchor=north west,align=left] {Relations\\ with \\ noncommutative\\ geometry};
\draw (200.22, -45.300000000000004) rectangle (204.32,-47.400000000000006);
\draw(204.42, -45.300000000000004) node[anchor=north west,align=left] {Relations\\ with random\\ matrices};
\draw (204.42, -45.300000000000004) rectangle (207.76999999999998,-46.900000000000006);
\draw(207.87, -45.300000000000004) node[anchor=north west,align=left] {Tauberian\\ theorems};
\draw (207.87, -45.300000000000004) rectangle (210.72,-46.900000000000006);
\draw(199.22, -47.6) node[anchor=north west,align=left] {\large Polynomials and matrices};
\draw (199.22, -47.6) rectangle (208.12,-50.800000000000004);
\draw(200.22, -48.6) node[anchor=north west,align=left] {Polynomials\\ in \\ number theory};
\draw (200.22, -48.6) rectangle (204.07,-50.2);
\draw(204.17, -48.6) node[anchor=north west,align=left] {Matrices,\\ determinants\\ in \\ number theory};
\draw (204.17, -48.6) rectangle (208.01999999999998,-50.7);
\draw(236.27, -22.1) node[anchor=north west,align=left] {\large Sequences and sets};
\draw (236.27, -22.1) rectangle (245.92000000000002,-38.5);
\draw(237.27, -23.1) node[anchor=north west,align=left] {Arithmetic \\ combinatorics;\\ higher \\ degree uniformity};
\draw (237.27, -23.1) rectangle (242.12,-25.200000000000003);
\draw(242.22, -23.1) node[anchor=north west,align=left] {Arithmetic\\ progressions};
\draw (242.22, -23.1) rectangle (245.82,-24.700000000000003);
\draw(237.27, -25.3) node[anchor=north west,align=left] {Farey sequences;\\ the \\ sequences \(1^k, 2^k, \dots\)};
\draw (237.27, -25.3) rectangle (241.87,-27.400000000000002);
\draw(241.97, -25.3) node[anchor=north west,align=left] {Other \\ combinatorial\\ number\\ theory};
\draw (241.97, -25.3) rectangle (245.82,-27.400000000000002);
\draw(237.27, -27.5) node[anchor=north west,align=left] {Binomial \\ coefficients; \\ factorials; \\ \(q\)-identities};
\draw (237.27, -27.5) rectangle (241.87,-29.6);
\draw(241.97, -27.5) node[anchor=north west,align=left] {Recurrences};
\draw (241.97, -27.5) rectangle (245.32,-28.6);
\draw(237.27, -29.700000000000003) node[anchor=north west,align=left] {Fibonacci and\\ Lucas numbers\\ and \\ polynomials and\\ generalizations};
\draw (237.27, -29.700000000000003) rectangle (241.62,-32.300000000000004);
\draw(241.72, -29.700000000000003) node[anchor=north west,align=left] {Representation\\ functions};
\draw (241.72, -29.700000000000003) rectangle (245.82,-31.300000000000004);
\draw(237.27, -32.400000000000006) node[anchor=north west,align=left] {Bernoulli\\ and Euler\\ numbers and\\ polynomials};
\draw (237.27, -32.400000000000006) rectangle (240.62,-34.50000000000001);
\draw(240.72, -32.400000000000006) node[anchor=north west,align=left] {Special\\ sequences\\ and \\ polynomials};
\draw (240.72, -32.400000000000006) rectangle (244.07,-34.50000000000001);
\draw(237.27, -34.6) node[anchor=north west,align=left] {Additive\\ bases,\\ including\\ sumsets};
\draw (237.27, -34.6) rectangle (240.12,-36.7);
\draw(240.22, -34.6) node[anchor=north west,align=left] {Sequences\\ (mod\\ \(m\))};
\draw (240.22, -34.6) rectangle (243.07,-36.2);
\draw(243.17000000000002, -34.6) node[anchor=north west,align=left] {Density,\\ gaps,\\ topology};
\draw (243.17000000000002, -34.6) rectangle (245.77,-36.2);
\draw(237.27, -36.8) node[anchor=north west,align=left] {Automata\\ sequences};
\draw (237.27, -36.8) rectangle (240.12,-38.4);
\draw(240.22, -36.8) node[anchor=north west,align=left] {Bell and\\ Stirling\\ numbers};
\draw (240.22, -36.8) rectangle (242.82,-38.4);
\draw(199.22, -50.900000000000006) node[anchor=north west,align=left] {\large Exponential sums and character sums};
\draw (199.22, -50.900000000000006) rectangle (212.32,-58.50000000000001);
\draw(200.22, -51.900000000000006) node[anchor=north west,align=left] {Estimates\\ on \\ exponential sums};
\draw (200.22, -51.900000000000006) rectangle (204.82,-53.50000000000001);
\draw(204.92, -51.900000000000006) node[anchor=north west,align=left] {Gauss and\\ Kloosterman\\ sums; \\ generalizations};
\draw (204.92, -51.900000000000006) rectangle (209.26999999999998,-54.00000000000001);
\draw(209.37, -51.900000000000006) node[anchor=north west,align=left] {Sums over\\ primes};
\draw (209.37, -51.900000000000006) rectangle (212.22,-53.00000000000001);
\draw(200.22, -54.10000000000001) node[anchor=north west,align=left] {Jacobsthal\\ and Brewer\\ sums; other\\ complete \\ character sums};
\draw (200.22, -54.10000000000001) rectangle (204.32,-56.70000000000001);
\draw(204.42, -54.10000000000001) node[anchor=north west,align=left] {Trigonometric\\ and \\ exponential\\ sums, general};
\draw (204.42, -54.10000000000001) rectangle (208.26999999999998,-56.20000000000001);
\draw(208.37, -54.10000000000001) node[anchor=north west,align=left] {Estimates\\ on character\\ sums};
\draw (208.37, -54.10000000000001) rectangle (211.97,-55.70000000000001);
\draw(200.22, -56.800000000000004) node[anchor=north west,align=left] {Sums over\\ arbitrary\\ intervals};
\draw (200.22, -56.800000000000004) rectangle (203.07,-58.400000000000006);
\draw(203.17, -56.800000000000004) node[anchor=north west,align=left] {Weyl\\ sums};
\draw (203.17, -56.800000000000004) rectangle (204.76999999999998,-57.900000000000006);
\draw(199.22, -51.2) node[anchor=north west,align=left] {\large Algebraic number theory: global fields};
\draw (199.22, -51.2) rectangle (213.82,-73.0);
\draw(200.22, -52.2) node[anchor=north west,align=left] {PV-numbers and\\ generalizations;\\ other special\\ algebraic \\ numbers; Mahler measure};
\draw (200.22, -52.2) rectangle (206.57,-54.800000000000004);
\draw(206.67, -52.2) node[anchor=north west,align=left] {Integral \\ representations related\\ to algebraic \\ numbers; Galois \\ module structure \\ of rings of integers};
\draw (206.67, -52.2) rectangle (213.01999999999998,-55.300000000000004);
\draw(200.22, -55.400000000000006) node[anchor=north west,align=left] {Zeta functions\\ and \\ \(L\)-functions of \\ function fields};
\draw (200.22, -55.400000000000006) rectangle (205.57,-57.50000000000001);
\draw(205.67, -55.400000000000006) node[anchor=north west,align=left] {Polynomials\\ (irreducibility,\\ etc.)};
\draw (205.67, -55.400000000000006) rectangle (210.26999999999998,-57.50000000000001);
\draw(210.37, -55.400000000000006) node[anchor=north west,align=left] {Other \\ abelian and\\ metabelian\\ extensions};
\draw (210.37, -55.400000000000006) rectangle (213.72,-57.50000000000001);
\draw(200.22, -57.6) node[anchor=north west,align=left] {Langlands-Weil\\ conjectures,\\ nonabelian class\\ field theory};
\draw (200.22, -57.6) rectangle (204.82,-60.2);
\draw(204.92, -57.6) node[anchor=north west,align=left] {Zeta functions\\ and \\ \(L\)-functions \\ of number fields};
\draw (204.92, -57.6) rectangle (209.51999999999998,-59.7);
\draw(209.62, -57.6) node[anchor=north west,align=left] {Algebraic \\ numbers; rings\\ of algebraic\\ integers};
\draw (209.62, -57.6) rectangle (213.72,-59.7);
\draw(200.22, -60.300000000000004) node[anchor=north west,align=left] {Other algebras\\ and orders,\\ and their\\ zeta and \\ \(L\)-functions};
\draw (200.22, -60.300000000000004) rectangle (204.57,-62.900000000000006);
\draw(204.67, -60.300000000000004) node[anchor=north west,align=left] {Arithmetic\\ theory of \\ algebraic \\ function fields};
\draw (204.67, -60.300000000000004) rectangle (209.01999999999998,-62.400000000000006);
\draw(209.12, -60.300000000000004) node[anchor=north west,align=left] {Cyclotomic \\ function fields\\ (class groups,\\ Bernoulli\\ objects, etc.)};
\draw (209.12, -60.300000000000004) rectangle (213.47,-62.900000000000006);
\draw(200.22, -63.0) node[anchor=north west,align=left] {Class \\ numbers, class\\ groups, \\ discriminants};
\draw (200.22, -63.0) rectangle (204.32,-65.1);
\draw(204.42, -63.0) node[anchor=north west,align=left] {Quaternion and\\ other division\\ algebras:\\ arithmetic,\\ zeta functions};
\draw (204.42, -63.0) rectangle (208.51999999999998,-65.6);
\draw(208.62, -63.0) node[anchor=north west,align=left] {Units \\ and \\ factorization};
\draw (208.62, -63.0) rectangle (212.47,-64.6);
\draw(200.22, -65.7) node[anchor=north west,align=left] {Class groups\\ and \\ Picard groups\\ of orders};
\draw (200.22, -65.7) rectangle (204.07,-67.8);
\draw(204.17, -65.7) node[anchor=north west,align=left] {\(K\)-theory\\ of \\ global fields};
\draw (204.17, -65.7) rectangle (208.01999999999998,-67.3);
\draw(208.12, -65.7) node[anchor=north west,align=left] {Distribution\\ of \\ prime ideals};
\draw (208.12, -65.7) rectangle (211.72,-67.3);
\draw(200.22, -67.9) node[anchor=north west,align=left] {Quadratic\\ extensions};
\draw (200.22, -67.9) rectangle (203.32,-69.5);
\draw(203.42, -67.9) node[anchor=north west,align=left] {Cubic and\\ quartic\\ extensions};
\draw (203.42, -67.9) rectangle (206.51999999999998,-69.5);
\draw(206.62, -67.9) node[anchor=north west,align=left] {Cyclotomic\\ extensions};
\draw (206.62, -67.9) rectangle (209.72,-69.5);
\draw(209.82, -67.9) node[anchor=north west,align=left] {Galois\\ cohomology};
\draw (209.82, -67.9) rectangle (212.92,-69.5);
\draw(200.22, -69.6) node[anchor=north west,align=left] {Adèle \\ rings and\\ groups};
\draw (200.22, -69.6) rectangle (203.07,-71.19999999999999);
\draw(203.17, -69.6) node[anchor=north west,align=left] {Density\\ theorems};
\draw (203.17, -69.6) rectangle (205.76999999999998,-70.69999999999999);
\draw(205.87, -69.6) node[anchor=north west,align=left] {Other \\ analytic\\ theory};
\draw (205.87, -69.6) rectangle (208.47,-71.19999999999999);
\draw(208.57, -69.6) node[anchor=north west,align=left] {Iwasawa\\ theory};
\draw (208.57, -69.6) rectangle (210.92,-70.69999999999999);
\draw(211.02, -69.6) node[anchor=north west,align=left] {Totally\\ real\\ fields};
\draw (211.02, -69.6) rectangle (213.37,-71.19999999999999);
\draw(200.22, -71.3) node[anchor=north west,align=left] {Other\\ number\\ fields};
\draw (200.22, -71.3) rectangle (202.32,-72.89999999999999);
\draw(202.42, -71.3) node[anchor=north west,align=left] {Galois\\ theory};
\draw (202.42, -71.3) rectangle (204.51999999999998,-72.39999999999999);
\draw(204.62, -71.3) node[anchor=north west,align=left] {Class\\ field\\ theory};
\draw (204.62, -71.3) rectangle (206.72,-72.89999999999999);
\draw(213.92, -51.2) node[anchor=north west,align=left] {\large Multiplicative number theory};
\draw (213.92, -51.2) rectangle (226.82,-68.4);
\draw(214.92, -52.2) node[anchor=north west,align=left] {Distribution \\ functions \\ associated with \\ additive and \\ positive multiplicative\\ functions};
\draw (214.92, -52.2) rectangle (221.26999999999998,-55.300000000000004);
\draw(221.36999999999998, -52.2) node[anchor=north west,align=left] {Primes represented\\ by \\ polynomials; other \\ multiplicative\\ structures of\\ polynomial values};
\draw (221.36999999999998, -52.2) rectangle (226.71999999999997,-55.300000000000004);
\draw(214.92, -55.400000000000006) node[anchor=north west,align=left] {Other results \\ on the distribution\\ of values \\ or the characterization\\ of \\ arithmetic functions};
\draw (214.92, -55.400000000000006) rectangle (221.26999999999998,-58.50000000000001);
\draw(221.36999999999998, -55.400000000000006) node[anchor=north west,align=left] {Asymptotic \\ results on \\ counting functions\\ for algebraic\\ and topological\\ structures};
\draw (221.36999999999998, -55.400000000000006) rectangle (226.46999999999997,-58.50000000000001);
\draw(214.92, -58.6) node[anchor=north west,align=left] {Applications \\ of automorphic\\ functions and\\ forms to \\ multiplicative problems};
\draw (214.92, -58.6) rectangle (221.26999999999998,-61.2);
\draw(221.36999999999998, -58.6) node[anchor=north west,align=left] {Distribution\\ of integers\\ in special\\ residue classes};
\draw (221.36999999999998, -58.6) rectangle (225.71999999999997,-60.7);
\draw(214.92, -61.300000000000004) node[anchor=north west,align=left] {Distribution\\ of integers \\ with specified\\ multiplicative\\ constraints};
\draw (214.92, -61.300000000000004) rectangle (219.01999999999998,-63.900000000000006);
\draw(219.11999999999998, -61.300000000000004) node[anchor=north west,align=left] {Applications\\ of \\ sieve methods};
\draw (219.11999999999998, -61.300000000000004) rectangle (222.96999999999997,-62.900000000000006);
\draw(223.07, -61.300000000000004) node[anchor=north west,align=left] {Distribution\\ of primes};
\draw (223.07, -61.300000000000004) rectangle (226.67,-62.900000000000006);
\draw(214.92, -64.0) node[anchor=north west,align=left] {Asymptotic\\ results \\ on arithmetic\\ functions};
\draw (214.92, -64.0) rectangle (218.76999999999998,-66.1);
\draw(218.86999999999998, -64.0) node[anchor=north west,align=left] {Generalized\\ primes \\ and integers};
\draw (218.86999999999998, -64.0) rectangle (222.46999999999997,-65.6);
\draw(222.57, -64.0) node[anchor=north west,align=left] {Primes in\\ congruence\\ classes};
\draw (222.57, -64.0) rectangle (225.67,-65.6);
\draw(214.92, -66.2) node[anchor=north west,align=left] {Rate of \\ growth of\\ arithmetic\\ functions};
\draw (214.92, -66.2) rectangle (218.01999999999998,-68.3);
\draw(218.11999999999998, -66.2) node[anchor=north west,align=left] {Turán\\ theory};
\draw (218.11999999999998, -66.2) rectangle (220.21999999999997,-67.3);
\draw(220.32, -66.2) node[anchor=north west,align=left] {Sieves};
\draw (220.32, -66.2) rectangle (222.42,-67.3);
\draw(226.92, -51.2) node[anchor=north west,align=left] {\large Additive number theory; partitions};
\draw (226.92, -51.2) rectangle (239.76999999999998,-61.5);
\draw(227.92, -52.2) node[anchor=north west,align=left] {Partition \\ identities;\\ identities\\ of \\ Rogers-Ramanujan type};
\draw (227.92, -52.2) rectangle (233.76999999999998,-54.800000000000004);
\draw(233.86999999999998, -52.2) node[anchor=north west,align=left] {Inverse \\ problems of \\ additive number\\ theory, \\ including sumsets};
\draw (233.86999999999998, -52.2) rectangle (238.71999999999997,-54.800000000000004);
\draw(227.92, -54.900000000000006) node[anchor=north west,align=left] {Goldbach-type\\ theorems; \\ other additive\\ questions \\ involving primes};
\draw (227.92, -54.900000000000006) rectangle (232.51999999999998,-57.50000000000001);
\draw(232.61999999999998, -54.900000000000006) node[anchor=north west,align=left] {Applications\\ of the\\ Hardy-Littlewood\\ method};
\draw (232.61999999999998, -54.900000000000006) rectangle (237.21999999999997,-57.00000000000001);
\draw(227.92, -57.6) node[anchor=north west,align=left] {Partitions; \\ congruences and\\ congruential\\ restrictions};
\draw (227.92, -57.6) rectangle (232.26999999999998,-59.7);
\draw(232.36999999999998, -57.6) node[anchor=north west,align=left] {Waring’s\\ problem \\ and variants};
\draw (232.36999999999998, -57.6) rectangle (235.96999999999997,-59.2);
\draw(236.07, -57.6) node[anchor=north west,align=left] {Lattice\\ points \\ in specified\\ regions};
\draw (236.07, -57.6) rectangle (239.67,-59.7);
\draw(227.92, -59.800000000000004) node[anchor=north west,align=left] {Elementary\\ theory of\\ partitions};
\draw (227.92, -59.800000000000004) rectangle (231.01999999999998,-61.400000000000006);
\draw(231.11999999999998, -59.800000000000004) node[anchor=north west,align=left] {Analytic\\ theory of\\ partitions};
\draw (231.11999999999998, -59.800000000000004) rectangle (234.21999999999997,-61.400000000000006);
\draw(226.92, -61.6) node[anchor=north west,align=left] {\large Connections of number theory and logic};
\draw (226.92, -61.6) rectangle (239.29999999999998,-67.0);
\draw(227.92, -62.6) node[anchor=north west,align=left] {Decidability\\ (number-theoretic\\ aspects)};
\draw (227.92, -62.6) rectangle (232.76999999999998,-64.7);
\draw(232.86999999999998, -62.6) node[anchor=north west,align=left] {Ultraproducts\\ (number-theoretic\\ aspects)};
\draw (232.86999999999998, -62.6) rectangle (237.71999999999997,-64.7);
\draw(227.92, -64.8) node[anchor=north west,align=left] {Model \\ theory \\ (number-theoretic\\ aspects)};
\draw (227.92, -64.8) rectangle (232.76999999999998,-66.89999999999999);
\draw(232.86999999999998, -64.8) node[anchor=north west,align=left] {Nonstandard\\ arithmetic \\ (number-theoretic\\ aspects)};
\draw (232.86999999999998, -64.8) rectangle (237.71999999999997,-66.89999999999999);
\draw(199.22, -73.10000000000001) node[anchor=north west,align=left] {\large Forms and linear algebraic groups};
\draw (199.22, -73.10000000000001) rectangle (211.82,-91.50000000000001);
\draw(200.22, -74.10000000000001) node[anchor=north west,align=left] {Analytic theory\\ (Epstein zeta\\ functions; \\ relations with\\ automorphic \\ forms and functions)};
\draw (200.22, -74.10000000000001) rectangle (205.82,-77.2);
\draw(205.92, -74.10000000000001) node[anchor=north west,align=left] {Sums of squares\\ and representations\\ by other\\ particular\\ quadratic forms};
\draw (205.92, -74.10000000000001) rectangle (211.26999999999998,-76.7);
\draw(200.22, -77.30000000000001) node[anchor=north west,align=left] {General ternary\\ and quaternary \\ quadratic forms;\\ forms of more \\ than two variables};
\draw (200.22, -77.30000000000001) rectangle (205.32,-79.9);
\draw(205.42, -77.30000000000001) node[anchor=north west,align=left] {General \\ binary quadratic\\ forms};
\draw (205.42, -77.30000000000001) rectangle (210.01999999999998,-78.9);
\draw(200.22, -80.00000000000001) node[anchor=north west,align=left] {Galois \\ cohomology of \\ linear algebraic\\ groups};
\draw (200.22, -80.00000000000001) rectangle (204.82,-82.10000000000001);
\draw(204.92, -80.00000000000001) node[anchor=north west,align=left] {Algebraic \\ theory of \\ quadratic \\ forms; Witt \\ groups and rings};
\draw (204.92, -80.00000000000001) rectangle (209.51999999999998,-82.60000000000001);
\draw(200.22, -82.70000000000002) node[anchor=north west,align=left] {Class numbers\\ of quadratic\\ and \\ Hermitian forms};
\draw (200.22, -82.70000000000002) rectangle (204.57,-84.80000000000001);
\draw(204.67, -82.70000000000002) node[anchor=north west,align=left] {\(K\)-theory\\ of quadratic\\ and \\ Hermitian forms};
\draw (204.67, -82.70000000000002) rectangle (209.01999999999998,-84.80000000000001);
\draw(200.22, -84.9) node[anchor=north west,align=left] {Quadratic\\ forms\\ over \\ general fields};
\draw (200.22, -84.9) rectangle (204.32,-87.0);
\draw(204.42, -84.9) node[anchor=north west,align=left] {Bilinear\\ and Hermitian\\ forms};
\draw (204.42, -84.9) rectangle (208.26999999999998,-86.5);
\draw(208.37, -84.9) node[anchor=north west,align=left] {Quadratic\\ forms over\\ local rings\\ and fields};
\draw (208.37, -84.9) rectangle (211.72,-87.0);
\draw(200.22, -87.10000000000001) node[anchor=north west,align=left] {Forms of \\ degree higher\\ than two};
\draw (200.22, -87.10000000000001) rectangle (204.07,-88.7);
\draw(204.17, -87.10000000000001) node[anchor=north west,align=left] {Quadratic \\ forms over \\ global rings\\ and fields};
\draw (204.17, -87.10000000000001) rectangle (207.76999999999998,-89.2);
\draw(207.87, -87.10000000000001) node[anchor=north west,align=left] {\(p\)-adic\\ theory};
\draw (207.87, -87.10000000000001) rectangle (210.97,-88.7);
\draw(200.22, -89.30000000000001) node[anchor=north west,align=left] {Forms \\ over real\\ fields};
\draw (200.22, -89.30000000000001) rectangle (203.07,-90.9);
\draw(203.17, -89.30000000000001) node[anchor=north west,align=left] {Classical\\ groups};
\draw (203.17, -89.30000000000001) rectangle (206.01999999999998,-90.4);
\draw(206.12, -89.30000000000001) node[anchor=north west,align=left] {Quadratic\\ spaces;\\ Clifford\\ algebras};
\draw (206.12, -89.30000000000001) rectangle (208.97,-91.4);
\draw(211.92, -73.10000000000001) node[anchor=north west,align=left] {\large Elementary number theory};
\draw (211.92, -73.10000000000001) rectangle (222.82,-83.4);
\draw(212.92, -74.10000000000001) node[anchor=north west,align=left] {Multiplicative\\ structure; \\ Euclidean algorithm;\\ greatest\\ common divisors};
\draw (212.92, -74.10000000000001) rectangle (218.51999999999998,-76.7);
\draw(218.61999999999998, -74.10000000000001) node[anchor=north west,align=left] {Factorization;\\ primality};
\draw (218.61999999999998, -74.10000000000001) rectangle (222.71999999999997,-75.7);
\draw(212.92, -76.80000000000001) node[anchor=north west,align=left] {Arithmetic\\ functions;\\ related \\ numbers; inversion\\ formulas};
\draw (212.92, -76.80000000000001) rectangle (218.01999999999998,-79.4);
\draw(218.11999999999998, -76.80000000000001) node[anchor=north west,align=left] {Congruences;\\ primitive\\ roots; \\ residue systems};
\draw (218.11999999999998, -76.80000000000001) rectangle (222.46999999999997,-78.9);
\draw(212.92, -79.50000000000001) node[anchor=north west,align=left] {Radix \\ representation;\\ digital\\ problems};
\draw (212.92, -79.50000000000001) rectangle (217.26999999999998,-81.60000000000001);
\draw(217.36999999999998, -79.50000000000001) node[anchor=north west,align=left] {Other \\ number \\ representations};
\draw (217.36999999999998, -79.50000000000001) rectangle (221.71999999999997,-81.10000000000001);
\draw(212.92, -81.70000000000002) node[anchor=north west,align=left] {Power \\ residues, \\ reciprocity};
\draw (212.92, -81.70000000000002) rectangle (216.26999999999998,-83.30000000000001);
\draw(216.36999999999998, -81.70000000000002) node[anchor=north west,align=left] {Continued\\ fractions};
\draw (216.36999999999998, -81.70000000000002) rectangle (219.21999999999997,-83.30000000000001);
\draw(219.32, -81.70000000000002) node[anchor=north west,align=left] {Primes};
\draw (219.32, -81.70000000000002) rectangle (221.42,-82.80000000000001);
\draw(222.92, -73.10000000000001) node[anchor=north west,align=left] {\large Computational number theory};
\draw (222.92, -73.10000000000001) rectangle (233.32,-85.60000000000001);
\draw(223.92, -74.10000000000001) node[anchor=north west,align=left] {Continued \\ fraction \\ calculations \\ (number-theoretic\\ aspects)};
\draw (223.92, -74.10000000000001) rectangle (228.76999999999998,-76.7);
\draw(228.86999999999998, -74.10000000000001) node[anchor=north west,align=left] {Factorization};
\draw (228.86999999999998, -74.10000000000001) rectangle (232.71999999999997,-75.2);
\draw(228.86999999999998, -75.30000000000001) node[anchor=north west,align=left] {Primality};
\draw (228.86999999999998, -75.30000000000001) rectangle (231.71999999999997,-76.4);
\draw(223.92, -76.80000000000001) node[anchor=north west,align=left] {Number-theoretic\\ algorithms;\\ complexity};
\draw (223.92, -76.80000000000001) rectangle (228.51999999999998,-78.9);
\draw(228.61999999999998, -76.80000000000001) node[anchor=north west,align=left] {Evaluation\\ of \\ number-theoretic\\ constants};
\draw (228.61999999999998, -76.80000000000001) rectangle (233.21999999999997,-78.9);
\draw(223.92, -79.00000000000001) node[anchor=north west,align=left] {Analytic\\ computations};
\draw (223.92, -79.00000000000001) rectangle (227.51999999999998,-80.60000000000001);
\draw(227.61999999999998, -79.00000000000001) node[anchor=north west,align=left] {Algebraic\\ number \\ theory \\ computations};
\draw (227.61999999999998, -79.00000000000001) rectangle (231.21999999999997,-81.10000000000001);
\draw(223.92, -81.20000000000002) node[anchor=north west,align=left] {Computer \\ solution of\\ Diophantine\\ equations};
\draw (223.92, -81.20000000000002) rectangle (227.26999999999998,-83.30000000000001);
\draw(227.36999999999998, -81.20000000000002) node[anchor=north west,align=left] {Calculation\\ of integer\\ sequences};
\draw (227.36999999999998, -81.20000000000002) rectangle (230.71999999999997,-82.80000000000001);
\draw(223.92, -83.4) node[anchor=north west,align=left] {Values of\\ arithmetic\\ functions;\\ tables};
\draw (223.92, -83.4) rectangle (227.01999999999998,-85.5);
\draw(233.42, -73.10000000000001) node[anchor=north west,align=left] {\large Geometry of numbers};
\draw (233.42, -73.10000000000001) rectangle (243.32,-83.4);
\draw(234.42, -74.10000000000001) node[anchor=north west,align=left] {Lattices and\\ convex bodies\\ (number-theoretic\\ aspects)};
\draw (234.42, -74.10000000000001) rectangle (239.26999999999998,-76.2);
\draw(239.36999999999998, -74.10000000000001) node[anchor=north west,align=left] {Relations\\ with \\ coding theory};
\draw (239.36999999999998, -74.10000000000001) rectangle (243.21999999999997,-75.7);
\draw(234.42, -76.30000000000001) node[anchor=north west,align=left] {Lattice \\ packing and \\ covering \\ (number-theoretic\\ aspects)};
\draw (234.42, -76.30000000000001) rectangle (239.26999999999998,-78.9);
\draw(239.36999999999998, -76.30000000000001) node[anchor=north west,align=left] {Automorphism\\ groups\\ of lattices};
\draw (239.36999999999998, -76.30000000000001) rectangle (242.96999999999997,-77.9);
\draw(234.42, -79.00000000000001) node[anchor=north west,align=left] {Quadratic \\ forms (reduction\\ theory,\\ extreme \\ forms, etc.)};
\draw (234.42, -79.00000000000001) rectangle (239.01999999999998,-81.60000000000001);
\draw(239.11999999999998, -79.00000000000001) node[anchor=north west,align=left] {Mean value\\ and transfer\\ theorems};
\draw (239.11999999999998, -79.00000000000001) rectangle (242.71999999999997,-80.60000000000001);
\draw(234.42, -81.70000000000002) node[anchor=north west,align=left] {Nonconvex\\ bodies};
\draw (234.42, -81.70000000000002) rectangle (237.26999999999998,-82.80000000000001);
\draw(237.36999999999998, -81.70000000000002) node[anchor=north west,align=left] {Products\\ of linear\\ forms};
\draw (237.36999999999998, -81.70000000000002) rectangle (240.21999999999997,-83.30000000000001);
\draw(240.32, -81.70000000000002) node[anchor=north west,align=left] {Minima\\ of forms};
\draw (240.32, -81.70000000000002) rectangle (242.92,-82.80000000000001);
\draw(199.22, -91.60000000000001) node[anchor=north west,align=left] {\large Diophantine equations};
\draw (199.22, -91.60000000000001) rectangle (208.37,-109.20000000000002);
\draw(200.22, -92.60000000000001) node[anchor=north west,align=left] {Higher \\ degree equations;\\ Fermat’s\\ equation};
\draw (200.22, -92.60000000000001) rectangle (205.07,-94.7);
\draw(205.17, -92.60000000000001) node[anchor=north west,align=left] {Rational\\ numbers \\ as sums of\\ fractions};
\draw (205.17, -92.60000000000001) rectangle (208.26999999999998,-94.7);
\draw(200.22, -94.80000000000001) node[anchor=north west,align=left] {Counting \\ solutions \\ of Diophantine\\ equations};
\draw (200.22, -94.80000000000001) rectangle (204.32,-96.9);
\draw(204.42, -94.80000000000001) node[anchor=north west,align=left] {\(p\)-adic\\ and \\ power \\ series fields};
\draw (204.42, -94.80000000000001) rectangle (208.26999999999998,-96.9);
\draw(200.22, -97.00000000000001) node[anchor=north west,align=left] {Multiplicative\\ and\\ norm form\\ equations};
\draw (200.22, -97.00000000000001) rectangle (204.32,-99.10000000000001);
\draw(204.42, -97.00000000000001) node[anchor=north west,align=left] {Quadratic \\ and bilinear\\ Diophantine\\ equations};
\draw (204.42, -97.00000000000001) rectangle (208.01999999999998,-99.10000000000001);
\draw(200.22, -99.2) node[anchor=north west,align=left] {Diophantine\\ equations\\ in \\ many variables};
\draw (200.22, -99.2) rectangle (204.32,-101.3);
\draw(204.42, -99.2) node[anchor=north west,align=left] {Diophantine\\ inequalities};
\draw (204.42, -99.2) rectangle (208.01999999999998,-100.8);
\draw(200.22, -101.4) node[anchor=north west,align=left] {Representation\\ problems};
\draw (200.22, -101.4) rectangle (204.32,-103.0);
\draw(204.42, -101.4) node[anchor=north west,align=left] {Linear \\ Diophantine\\ equations};
\draw (204.42, -101.4) rectangle (207.76999999999998,-103.0);
\draw(200.22, -103.10000000000001) node[anchor=north west,align=left] {Cubic and\\ quartic\\ Diophantine\\ equations};
\draw (200.22, -103.10000000000001) rectangle (203.57,-105.2);
\draw(203.67, -103.10000000000001) node[anchor=north west,align=left] {Thue-Mahler\\ equations};
\draw (203.67, -103.10000000000001) rectangle (207.01999999999998,-104.7);
\draw(200.22, -105.30000000000001) node[anchor=north west,align=left] {Exponential\\ Diophantine\\ equations};
\draw (200.22, -105.30000000000001) rectangle (203.57,-107.4);
\draw(203.67, -105.30000000000001) node[anchor=north west,align=left] {Congruences\\ in many\\ variables};
\draw (203.67, -105.30000000000001) rectangle (207.01999999999998,-106.9);
\draw(200.22, -107.5) node[anchor=north west,align=left] {The \\ Frobenius\\ problem};
\draw (200.22, -107.5) rectangle (203.07,-109.1);
\draw(246.12, -1) node[anchor=north west,align=left] {\LARGE Group theory and generalizations};
\draw (246.12, -1) rectangle (291.97,-68.6);
\draw(247.12, -2) node[anchor=north west,align=left] {\large Groupoids (i.e. small categories in which all morphisms are isomorphisms)};
\draw (247.12, -2) rectangle (270.35,-5.7);
\draw(248.12, -3) node[anchor=north west,align=left] {Groupoids (i.e.\\ small categories\\ in which\\ all morphisms\\ are isomorphisms)};
\draw (248.12, -3) rectangle (252.97,-5.6);
\draw(270.45, -2) node[anchor=north west,align=left] {\large Structure and classification of infinite or finite groups};
\draw (270.45, -2) rectangle (291.75,-11.1);
\draw(271.45, -3) node[anchor=north west,align=left] {Free products of \\ groups, free products\\ with amalgamation,\\ Higman-Neumann-Neumann\\ extensions,\\ and generalizations};
\draw (271.45, -3) rectangle (277.55,-6.1);
\draw(277.65, -3) node[anchor=north west,align=left] {Chains and \\ lattices of \\ subgroups, subnormal\\ subgroups};
\draw (277.65, -3) rectangle (283.25,-5.1);
\draw(283.34999999999997, -3) node[anchor=north west,align=left] {Extensions,\\ wreath products,\\ and other\\ compositions\\ of groups};
\draw (283.34999999999997, -3) rectangle (287.95,-5.6);
\draw(288.05, -3) node[anchor=north west,align=left] {Groups \\ with a \\ \(BN\)-pair;\\ buildings};
\draw (288.05, -3) rectangle (291.65000000000003,-5.1);
\draw(271.45, -6.2) node[anchor=north west,align=left] {Residual \\ properties and \\ generalizations;\\ residually\\ finite groups};
\draw (271.45, -6.2) rectangle (276.05,-8.8);
\draw(276.15, -6.2) node[anchor=north west,align=left] {Automorphisms\\ of \\ infinite groups};
\draw (276.15, -6.2) rectangle (280.5,-7.800000000000001);
\draw(280.59999999999997, -6.2) node[anchor=north west,align=left] {Quasivarieties\\ and\\ varieties\\ of groups};
\draw (280.59999999999997, -6.2) rectangle (284.7,-8.3);
\draw(284.8, -6.2) node[anchor=north west,align=left] {Free \\ nonabelian\\ groups};
\draw (284.8, -6.2) rectangle (287.90000000000003,-7.800000000000001);
\draw(288.0, -6.2) node[anchor=north west,align=left] {Local \\ properties\\ of groups};
\draw (288.0, -6.2) rectangle (291.1,-7.800000000000001);
\draw(271.45, -8.9) node[anchor=north west,align=left] {General \\ structure\\ theorems\\ for groups};
\draw (271.45, -8.9) rectangle (274.55,-11.0);
\draw(274.65, -8.9) node[anchor=north west,align=left] {Conjugacy\\ classes\\ for groups};
\draw (274.65, -8.9) rectangle (277.75,-10.5);
\draw(277.84999999999997, -8.9) node[anchor=north west,align=left] {Subgroup\\ theorems;\\ subgroup\\ growth};
\draw (277.84999999999997, -8.9) rectangle (280.7,-11.0);
\draw(280.8, -8.9) node[anchor=north west,align=left] {Limits,\\ profinite\\ groups};
\draw (280.8, -8.9) rectangle (283.65000000000003,-10.5);
\draw(283.75, -8.9) node[anchor=north west,align=left] {Maximal\\ subgroups};
\draw (283.75, -8.9) rectangle (286.6,-10.5);
\draw(286.7, -8.9) node[anchor=north west,align=left] {Groups\\ acting\\ on trees};
\draw (286.7, -8.9) rectangle (289.3,-10.5);
\draw(289.4, -8.9) node[anchor=north west,align=left] {Simple\\ groups};
\draw (289.4, -8.9) rectangle (291.5,-10.0);
\draw(247.12, -5.8) node[anchor=north west,align=left] {\large Connections of group theory with homological algebra and category theory};
\draw (247.12, -5.8) rectangle (270.04,-9.0);
\draw(248.12, -6.8) node[anchor=north west,align=left] {Homological\\ methods\\ in \\ group theory};
\draw (248.12, -6.8) rectangle (251.72,-8.9);
\draw(251.82, -6.8) node[anchor=north west,align=left] {Cohomology\\ of groups};
\draw (251.82, -6.8) rectangle (254.92,-8.4);
\draw(255.02, -6.8) node[anchor=north west,align=left] {Category\\ of\\ groups};
\draw (255.02, -6.8) rectangle (257.62,-8.4);
\draw(247.12, -9.1) node[anchor=north west,align=left] {\large Computational methods\\ for problems \\ pertaining to group theory};
\draw (247.12, -9.1) rectangle (255.78,-10.7);
\draw(247.12, -11.2) node[anchor=north west,align=left] {\large Special aspects of infinite or finite groups};
\draw (247.12, -11.2) rectangle (263.97,-31.8);
\draw(248.12, -12.2) node[anchor=north west,align=left] {Word problems, \\ other decision \\ problems, connections\\ with logic \\ and automata \\ (group-theoretic aspects)};
\draw (248.12, -12.2) rectangle (254.97,-15.299999999999999);
\draw(255.07, -12.2) node[anchor=north west,align=left] {Cancellation\\ theory of \\ groups; application\\ of van \\ Kampen diagrams};
\draw (255.07, -12.2) rectangle (260.42,-14.799999999999999);
\draw(260.52, -12.2) node[anchor=north west,align=left] {Groups of\\ finite \\ Morley rank};
\draw (260.52, -12.2) rectangle (263.87,-13.799999999999999);
\draw(248.12, -15.399999999999999) node[anchor=north west,align=left] {Generators,\\ relations, \\ and presentations\\ of groups};
\draw (248.12, -15.399999999999999) rectangle (252.97,-17.5);
\draw(253.07, -15.399999999999999) node[anchor=north west,align=left] {Representations\\ of groups\\ as automorphism\\ groups of\\ algebraic systems};
\draw (253.07, -15.399999999999999) rectangle (257.92,-18.0);
\draw(258.02, -15.399999999999999) node[anchor=north west,align=left] {Algebraic \\ geometry over \\ groups; equations\\ over groups};
\draw (258.02, -15.399999999999999) rectangle (262.87,-17.5);
\draw(248.12, -18.1) node[anchor=north west,align=left] {Generalizations\\ of solvable\\ and \\ nilpotent groups};
\draw (248.12, -18.1) rectangle (252.72,-20.200000000000003);
\draw(252.82, -18.1) node[anchor=north west,align=left] {Fundamental \\ groups and their\\ automorphisms\\ (group-theoretic\\ aspects)};
\draw (252.82, -18.1) rectangle (257.42,-20.700000000000003);
\draw(257.52, -18.1) node[anchor=north west,align=left] {Reflection\\ and Coxeter\\ groups \\ (group-theoretic\\ aspects)};
\draw (257.52, -18.1) rectangle (262.12,-20.700000000000003);
\draw(248.12, -20.8) node[anchor=north west,align=left] {Ordered \\ groups \\ (group-theoretic\\ aspects)};
\draw (248.12, -20.8) rectangle (252.72,-22.900000000000002);
\draw(252.82, -20.8) node[anchor=north west,align=left] {Derived series,\\ central\\ series, and\\ generalizations\\ for groups};
\draw (252.82, -20.8) rectangle (257.17,-23.400000000000002);
\draw(257.27, -20.8) node[anchor=north west,align=left] {Other classes\\ of groups\\ defined by \\ subgroup chains};
\draw (257.27, -20.8) rectangle (261.62,-22.900000000000002);
\draw(248.12, -23.5) node[anchor=north west,align=left] {FC-groups\\ and \\ their \\ generalizations};
\draw (248.12, -23.5) rectangle (252.47,-25.6);
\draw(252.57, -23.5) node[anchor=north west,align=left] {Solvable\\ groups, \\ supersolvable\\ groups};
\draw (252.57, -23.5) rectangle (256.42,-25.6);
\draw(256.52, -23.5) node[anchor=north west,align=left] {Periodic\\ groups; \\ locally \\ finite groups};
\draw (256.52, -23.5) rectangle (260.37,-25.6);
\draw(260.47, -23.5) node[anchor=north west,align=left] {Associated\\ Lie \\ structures \\ for groups};
\draw (260.47, -23.5) rectangle (263.82000000000005,-25.6);
\draw(248.12, -25.7) node[anchor=north west,align=left] {Hyperbolic \\ groups and \\ nonpositively\\ curved groups};
\draw (248.12, -25.7) rectangle (251.97,-27.8);
\draw(252.07, -25.7) node[anchor=north west,align=left] {Automorphism\\ groups\\ of groups};
\draw (252.07, -25.7) rectangle (255.67,-27.3);
\draw(255.77, -25.7) node[anchor=north west,align=left] {Braid \\ groups; \\ Artin groups};
\draw (255.77, -25.7) rectangle (259.37,-27.3);
\draw(259.47, -25.7) node[anchor=north west,align=left] {Other groups\\ related \\ to topology\\ or analysis};
\draw (259.47, -25.7) rectangle (263.07000000000005,-27.8);
\draw(248.12, -27.9) node[anchor=north west,align=left] {Commutator\\ calculus};
\draw (248.12, -27.9) rectangle (251.22,-29.5);
\draw(251.32, -27.9) node[anchor=north west,align=left] {Formations\\ of groups,\\ Fitting\\ classes};
\draw (251.32, -27.9) rectangle (254.42,-30.0);
\draw(254.52, -27.9) node[anchor=north west,align=left] {Engel \\ conditions};
\draw (254.52, -27.9) rectangle (257.62,-29.0);
\draw(257.72, -27.9) node[anchor=north west,align=left] {Asymptotic\\ properties\\ of groups};
\draw (257.72, -27.9) rectangle (260.82000000000005,-29.5);
\draw(260.92, -27.9) node[anchor=north west,align=left] {Nilpotent\\ groups};
\draw (260.92, -27.9) rectangle (263.77000000000004,-29.0);
\draw(248.12, -30.099999999999998) node[anchor=north west,align=left] {Geometric\\ group\\ theory};
\draw (248.12, -30.099999999999998) rectangle (250.97,-31.7);
\draw(264.07, -11.2) node[anchor=north west,align=left] {\large Linear algebraic groups and related topics};
\draw (264.07, -11.2) rectangle (280.17,-24.2);
\draw(265.07, -12.2) node[anchor=north west,align=left] {Quantum groups\\ (quantized \\ function algebras)\\ and their \\ representations};
\draw (265.07, -12.2) rectangle (270.17,-14.799999999999999);
\draw(270.27, -12.2) node[anchor=north west,align=left] {Linear algebraic\\ groups\\ over adèles\\ and other \\ rings and schemes};
\draw (270.27, -12.2) rectangle (275.12,-14.799999999999999);
\draw(275.21999999999997, -12.2) node[anchor=north west,align=left] {Exceptionalgroups};
\draw (275.21999999999997, -12.2) rectangle (280.07,-13.799999999999999);
\draw(265.07, -14.899999999999999) node[anchor=north west,align=left] {Representation\\ theory for\\ linear \\ algebraic groups};
\draw (265.07, -14.899999999999999) rectangle (269.67,-17.0);
\draw(269.77, -14.899999999999999) node[anchor=north west,align=left] {Structure \\ theory for \\ linear algebraic\\ groups};
\draw (269.77, -14.899999999999999) rectangle (274.37,-17.0);
\draw(274.46999999999997, -14.899999999999999) node[anchor=north west,align=left] {Cohomology\\ theory for\\ linear \\ algebraic groups};
\draw (274.46999999999997, -14.899999999999999) rectangle (279.07,-17.0);
\draw(265.07, -17.1) node[anchor=north west,align=left] {Linear \\ algebraic groups\\ over \\ arbitrary fields};
\draw (265.07, -17.1) rectangle (269.67,-19.200000000000003);
\draw(269.77, -17.1) node[anchor=north west,align=left] {Linear algebraic\\ groups over\\ the reals, the\\ complexes, \\ the quaternions};
\draw (269.77, -17.1) rectangle (274.37,-19.700000000000003);
\draw(274.46999999999997, -17.1) node[anchor=north west,align=left] {Linear algebraic\\ groups\\ over local \\ fields and \\ their integers};
\draw (274.46999999999997, -17.1) rectangle (279.07,-19.700000000000003);
\draw(265.07, -19.8) node[anchor=north west,align=left] {Linear algebraic\\ groups\\ over global\\ fields and \\ their integers};
\draw (265.07, -19.8) rectangle (269.67,-22.400000000000002);
\draw(269.77, -19.8) node[anchor=north west,align=left] {Linear \\ algebraic \\ groups over \\ finite fields};
\draw (269.77, -19.8) rectangle (273.62,-21.900000000000002);
\draw(273.71999999999997, -19.8) node[anchor=north west,align=left] {Applications\\ of linear\\ algebraic\\ groups to \\ the sciences};
\draw (273.71999999999997, -19.8) rectangle (277.32,-22.400000000000002);
\draw(265.07, -22.5) node[anchor=north west,align=left] {Schur and\\ \(q\)-Schur\\ algebras};
\draw (265.07, -22.5) rectangle (268.42,-24.1);
\draw(268.52, -22.5) node[anchor=north west,align=left] {Kac-Moody\\ groups};
\draw (268.52, -22.5) rectangle (271.37,-23.6);
\draw(264.07, -24.3) node[anchor=north west,align=left] {\large Probabilistic methods in group theory};
\draw (264.07, -24.3) rectangle (276.14,-27.5);
\draw(265.07, -25.3) node[anchor=north west,align=left] {Probabilistic\\ methods\\ in \\ group theory};
\draw (265.07, -25.3) rectangle (268.92,-27.400000000000002);
\draw(264.07, -27.6) node[anchor=north west,align=left] {\large History of\\ group theory};
\draw (264.07, -27.6) rectangle (268.39,-28.700000000000003);
\draw(280.27, -11.2) node[anchor=north west,align=left] {\large Other generalizations of groups};
\draw (280.27, -11.2) rectangle (291.87,-17.299999999999997);
\draw(281.27, -12.2) node[anchor=north west,align=left] {Sets with a\\ single \\ binary operation\\ (groupoids)};
\draw (281.27, -12.2) rectangle (285.87,-14.299999999999999);
\draw(285.96999999999997, -12.2) node[anchor=north west,align=left] {Ternary systems\\ (heaps,\\ semiheaps, \\ heapoids, etc.)};
\draw (285.96999999999997, -12.2) rectangle (290.32,-14.299999999999999);
\draw(281.27, -14.399999999999999) node[anchor=north west,align=left] {\(n\)-ary\\ systems \\ \((n\ge 3)\)};
\draw (281.27, -14.399999999999999) rectangle (284.87,-15.999999999999998);
\draw(284.96999999999997, -14.399999999999999) node[anchor=north west,align=left] {Loops,\\ quasigroups};
\draw (284.96999999999997, -14.399999999999999) rectangle (288.32,-15.999999999999998);
\draw(288.41999999999996, -14.399999999999999) node[anchor=north west,align=left] {Hypergroups};
\draw (288.41999999999996, -14.399999999999999) rectangle (291.77,-15.499999999999998);
\draw(281.27, -16.1) node[anchor=north west,align=left] {Fuzzy\\ groups};
\draw (281.27, -16.1) rectangle (283.37,-17.200000000000003);
\draw(247.12, -31.900000000000002) node[anchor=north west,align=left] {\large Representation theory of groups};
\draw (247.12, -31.900000000000002) rectangle (260.77,-49.300000000000004);
\draw(248.12, -32.900000000000006) node[anchor=north west,align=left] {Group rings of\\ infinite groups\\ and their \\ modules \\ (group-theoretic aspects)};
\draw (248.12, -32.900000000000006) rectangle (254.97,-35.50000000000001);
\draw(255.07, -32.900000000000006) node[anchor=north west,align=left] {Representations\\ of \\ infinite symmetric\\ groups};
\draw (255.07, -32.900000000000006) rectangle (260.17,-35.00000000000001);
\draw(248.12, -35.6) node[anchor=north west,align=left] {Group rings of\\ finite groups\\ and their \\ modules (group-theoretic\\ aspects)};
\draw (248.12, -35.6) rectangle (254.72,-38.2);
\draw(254.82, -35.6) node[anchor=north west,align=left] {Applications of\\ group representations\\ to physics\\ and other \\ areas of science};
\draw (254.82, -35.6) rectangle (260.67,-38.2);
\draw(248.12, -38.300000000000004) node[anchor=north west,align=left] {Representationsof\\ sporadic\\ groups};
\draw (248.12, -38.300000000000004) rectangle (252.97,-40.400000000000006);
\draw(253.07, -38.300000000000004) node[anchor=north west,align=left] {Representations\\ of \\ finite \\ symmetric groups};
\draw (253.07, -38.300000000000004) rectangle (257.67,-40.400000000000006);
\draw(248.12, -40.5) node[anchor=north west,align=left] {Hecke \\ algebras and \\ their \\ representations};
\draw (248.12, -40.5) rectangle (252.47,-42.6);
\draw(252.57, -40.5) node[anchor=north west,align=left] {Integral \\ representations\\ of \\ finite groups};
\draw (252.57, -40.5) rectangle (256.92,-42.6);
\draw(248.12, -42.7) node[anchor=north west,align=left] {\(p\)-adic\\ representations\\ of\\ finite groups};
\draw (248.12, -42.7) rectangle (252.47,-44.800000000000004);
\draw(252.57, -42.7) node[anchor=north west,align=left] {Integral \\ representations\\ of \\ infinite groups};
\draw (252.57, -42.7) rectangle (256.92,-44.800000000000004);
\draw(248.12, -44.900000000000006) node[anchor=north west,align=left] {Ordinary\\ representations\\ and\\ characters};
\draw (248.12, -44.900000000000006) rectangle (252.47,-47.00000000000001);
\draw(252.57, -44.900000000000006) node[anchor=north west,align=left] {Modular \\ representations\\ and\\ characters};
\draw (252.57, -44.900000000000006) rectangle (256.92,-47.00000000000001);
\draw(248.12, -47.1) node[anchor=north west,align=left] {Projective\\ representations\\ and\\ multipliers};
\draw (248.12, -47.1) rectangle (252.47,-49.2);
\draw(252.57, -47.1) node[anchor=north west,align=left] {Representations\\ of \\ finite groups\\ of Lie type};
\draw (252.57, -47.1) rectangle (256.92,-49.2);
\draw(260.87, -31.900000000000002) node[anchor=north west,align=left] {\large Permutation groups};
\draw (260.87, -31.900000000000002) rectangle (272.27,-43.400000000000006);
\draw(261.87, -32.900000000000006) node[anchor=north west,align=left] {Finite automorphism\\ groups of\\ algebraic, \\ geometric, or \\ combinatorial structures};
\draw (261.87, -32.900000000000006) rectangle (268.47,-35.50000000000001);
\draw(268.57, -32.900000000000006) node[anchor=north west,align=left] {Infinite\\ automorphism\\ groups};
\draw (268.57, -32.900000000000006) rectangle (272.17,-34.50000000000001);
\draw(261.87, -35.6) node[anchor=north west,align=left] {General \\ theory for \\ finite permutation\\ groups};
\draw (261.87, -35.6) rectangle (266.97,-37.7);
\draw(267.07, -35.6) node[anchor=north west,align=left] {General \\ theory for \\ infinite \\ permutation groups};
\draw (267.07, -35.6) rectangle (272.17,-37.7);
\draw(261.87, -37.800000000000004) node[anchor=north west,align=left] {Characterization\\ theorems\\ for permutation\\ groups};
\draw (261.87, -37.800000000000004) rectangle (266.47,-39.900000000000006);
\draw(266.57, -37.800000000000004) node[anchor=north west,align=left] {Multiply\\ transitive\\ infinite groups};
\draw (266.57, -37.800000000000004) rectangle (270.92,-39.900000000000006);
\draw(261.87, -40.0) node[anchor=north west,align=left] {Subgroups\\ of symmetric\\ groups};
\draw (261.87, -40.0) rectangle (265.47,-41.6);
\draw(265.57, -40.0) node[anchor=north west,align=left] {Multiply\\ transitive\\ finite\\ groups};
\draw (265.57, -40.0) rectangle (268.67,-42.1);
\draw(268.77, -40.0) node[anchor=north west,align=left] {Primitive\\ groups};
\draw (268.77, -40.0) rectangle (271.62,-41.1);
\draw(261.87, -42.2) node[anchor=north west,align=left] {Symmetric\\ groups};
\draw (261.87, -42.2) rectangle (264.72,-43.300000000000004);
\draw(260.87, -43.50000000000001) node[anchor=north west,align=left] {\large Foundations};
\draw (260.87, -43.50000000000001) rectangle (269.77,-48.900000000000006);
\draw(261.87, -44.50000000000001) node[anchor=north west,align=left] {Metamathematical\\ considerations \\ in group theory};
\draw (261.87, -44.50000000000001) rectangle (266.47,-46.60000000000001);
\draw(261.87, -46.70000000000001) node[anchor=north west,align=left] {Axiomatics\\ and elementary\\ properties\\ of groups};
\draw (261.87, -46.70000000000001) rectangle (265.97,-48.80000000000001);
\draw(266.07, -46.70000000000001) node[anchor=north west,align=left] {Applications\\ of \\ logic to \\ group theory};
\draw (266.07, -46.70000000000001) rectangle (269.67,-48.80000000000001);
\draw(272.37, -31.900000000000002) node[anchor=north west,align=left] {\large Abstract finite groups};
\draw (272.37, -31.900000000000002) rectangle (283.52,-48.10000000000001);
\draw(273.37, -32.900000000000006) node[anchor=north west,align=left] {Finite solvable\\ groups, theory \\ of formations, \\ Schunck classes, \\ Fitting classes,\\ \(\pi\)-length, ranks};
\draw (273.37, -32.900000000000006) rectangle (279.22,-36.00000000000001);
\draw(279.32, -32.900000000000006) node[anchor=north west,align=left] {Finite simple\\ groups \\ and their \\ classification};
\draw (279.32, -32.900000000000006) rectangle (283.42,-35.00000000000001);
\draw(273.37, -36.1) node[anchor=north west,align=left] {Arithmetic and\\ combinatorial\\ problems \\ involving abstract\\ finite groups};
\draw (273.37, -36.1) rectangle (278.47,-38.7);
\draw(278.57, -36.1) node[anchor=north west,align=left] {Sylow subgroups,\\ Sylow \\ properties, \\ \(\pi\)-groups, \\ \(\pi\)-structure};
\draw (278.57, -36.1) rectangle (283.42,-38.7);
\draw(273.37, -38.800000000000004) node[anchor=north west,align=left] {Simple \\ groups: sporadic\\ groups};
\draw (273.37, -38.800000000000004) rectangle (277.97,-40.400000000000006);
\draw(278.07, -38.800000000000004) node[anchor=north west,align=left] {Simple groups:\\ alternating\\ groups\\ and groups\\ of Lie type};
\draw (278.07, -38.800000000000004) rectangle (282.17,-41.400000000000006);
\draw(273.37, -41.5) node[anchor=north west,align=left] {Special \\ subgroups \\ (Frattini, \\ Fitting, etc.)};
\draw (273.37, -41.5) rectangle (277.47,-43.6);
\draw(277.57, -41.5) node[anchor=north west,align=left] {Subnormal \\ subgroups of\\ abstract \\ finite groups};
\draw (277.57, -41.5) rectangle (281.42,-43.6);
\draw(273.37, -43.7) node[anchor=north west,align=left] {Products of\\ subgroups\\ of abstract\\ finite groups};
\draw (273.37, -43.7) rectangle (277.22,-45.800000000000004);
\draw(277.32, -43.7) node[anchor=north west,align=left] {Automorphisms\\ of \\ abstract \\ finite groups};
\draw (277.32, -43.7) rectangle (281.17,-45.800000000000004);
\draw(273.37, -45.900000000000006) node[anchor=north west,align=left] {Finite \\ nilpotent \\ groups, \\ \(p\)-groups};
\draw (273.37, -45.900000000000006) rectangle (276.97,-48.00000000000001);
\draw(277.07, -45.900000000000006) node[anchor=north west,align=left] {Series and\\ lattices\\ of subgroups};
\draw (277.07, -45.900000000000006) rectangle (280.67,-47.50000000000001);
\draw(260.87, -49.00000000000001) node[anchor=north west,align=left] {\large Other groups of matrices};
\draw (260.87, -49.00000000000001) rectangle (270.77,-58.10000000000001);
\draw(261.87, -50.00000000000001) node[anchor=north west,align=left] {Unimodular \\ groups, congruence\\ subgroups\\ (group-theoretic\\ aspects)};
\draw (261.87, -50.00000000000001) rectangle (266.97,-52.60000000000001);
\draw(267.07, -50.00000000000001) node[anchor=north west,align=left] {Other matrix\\ groups\\ over fields};
\draw (267.07, -50.00000000000001) rectangle (270.67,-51.60000000000001);
\draw(261.87, -52.70000000000001) node[anchor=north west,align=left] {Fuchsian groups\\ and their\\ generalizations\\ (group-theoretic\\ aspects)};
\draw (261.87, -52.70000000000001) rectangle (266.47,-55.30000000000001);
\draw(266.57, -52.70000000000001) node[anchor=north west,align=left] {Other \\ matrix groups\\ over \\ finite fields};
\draw (266.57, -52.70000000000001) rectangle (270.42,-54.80000000000001);
\draw(261.87, -55.400000000000006) node[anchor=north west,align=left] {Other geometric\\ groups,\\ including\\ crystallographic\\ groups};
\draw (261.87, -55.400000000000006) rectangle (266.47,-58.00000000000001);
\draw(266.57, -55.400000000000006) node[anchor=north west,align=left] {Other matrix\\ groups\\ over rings};
\draw (266.57, -55.400000000000006) rectangle (270.17,-57.00000000000001);
\draw(247.12, -49.400000000000006) node[anchor=north west,align=left] {\large Semigroups};
\draw (247.12, -49.400000000000006) rectangle (257.77,-68.5);
\draw(248.12, -50.400000000000006) node[anchor=north west,align=left] {Semigroup \\ rings, \\ multiplicative \\ semigroups of rings};
\draw (248.12, -50.400000000000006) rectangle (253.47,-52.50000000000001);
\draw(253.57, -50.400000000000006) node[anchor=north west,align=left] {Ideal \\ theory for \\ semigroups};
\draw (253.57, -50.400000000000006) rectangle (256.92,-52.00000000000001);
\draw(248.12, -52.60000000000001) node[anchor=north west,align=left] {Representation\\ of \\ semigroups; \\ actions of \\ semigroups on sets};
\draw (248.12, -52.60000000000001) rectangle (253.22,-55.20000000000001);
\draw(253.32, -52.60000000000001) node[anchor=north west,align=left] {Varieties\\ and \\ pseudovarieties\\ of semigroups};
\draw (253.32, -52.60000000000001) rectangle (257.67,-54.70000000000001);
\draw(248.12, -55.300000000000004) node[anchor=north west,align=left] {Semigroups\\ of \\ transformations, \\ relations, \\ partitions, etc.};
\draw (248.12, -55.300000000000004) rectangle (252.97,-57.900000000000006);
\draw(253.07, -55.300000000000004) node[anchor=north west,align=left] {Free semigroups,\\ generators and \\ relations, \\ word problems};
\draw (253.07, -55.300000000000004) rectangle (257.67,-57.900000000000006);
\draw(248.12, -58.00000000000001) node[anchor=north west,align=left] {Semigroups \\ in automata \\ theory, \\ linguistics, etc.};
\draw (248.12, -58.00000000000001) rectangle (252.97,-60.10000000000001);
\draw(253.07, -58.00000000000001) node[anchor=north west,align=left] {Algebraicmonoids};
\draw (253.07, -58.00000000000001) rectangle (257.67,-59.60000000000001);
\draw(248.12, -60.2) node[anchor=north west,align=left] {Connections of\\ semigroups \\ with homological\\ algebra and \\ category theory};
\draw (248.12, -60.2) rectangle (252.72,-62.800000000000004);
\draw(252.82, -60.2) node[anchor=north west,align=left] {Generalizations\\ of\\ semigroups};
\draw (252.82, -60.2) rectangle (257.17,-61.800000000000004);
\draw(248.12, -62.900000000000006) node[anchor=north west,align=left] {Commutative\\ semigroups};
\draw (248.12, -62.900000000000006) rectangle (251.47,-64.5);
\draw(251.57, -62.900000000000006) node[anchor=north west,align=left] {General \\ structure\\ theory for\\ semigroups};
\draw (251.57, -62.900000000000006) rectangle (254.67,-65.0);
\draw(248.12, -65.10000000000001) node[anchor=north west,align=left] {Radical \\ theory for\\ semigroups};
\draw (248.12, -65.10000000000001) rectangle (251.22,-66.7);
\draw(251.32, -65.10000000000001) node[anchor=north west,align=left] {Arithmetic\\ theory of\\ semigroups};
\draw (251.32, -65.10000000000001) rectangle (254.42,-66.7);
\draw(254.52, -65.10000000000001) node[anchor=north west,align=left] {Mappings\\ of \\ semigroups};
\draw (254.52, -65.10000000000001) rectangle (257.62,-66.7);
\draw(248.12, -66.80000000000001) node[anchor=north west,align=left] {Regular\\ semigroups};
\draw (248.12, -66.80000000000001) rectangle (251.22,-68.4);
\draw(251.32, -66.80000000000001) node[anchor=north west,align=left] {Inverse\\ semigroups};
\draw (251.32, -66.80000000000001) rectangle (254.42,-68.4);
\draw(254.52, -66.80000000000001) node[anchor=north west,align=left] {Orthodox\\ semigroups};
\draw (254.52, -66.80000000000001) rectangle (257.62,-68.4);
\draw(257.87, -49.400000000000006) node[anchor=north west,align=left] {\large Abelian groups};
\draw (257.87, -49.400000000000006) rectangle (267.02,-64.10000000000001);
\draw(258.87, -50.400000000000006) node[anchor=north west,align=left] {Direct sums,\\ direct products,\\ etc. for\\ abelian groups};
\draw (258.87, -50.400000000000006) rectangle (263.47,-52.50000000000001);
\draw(263.57, -50.400000000000006) node[anchor=north west,align=left] {Subgroups\\ of abelian\\ groups};
\draw (263.57, -50.400000000000006) rectangle (266.67,-52.00000000000001);
\draw(258.87, -52.60000000000001) node[anchor=north west,align=left] {Torsion groups,\\ primary\\ groups and \\ generalized \\ primary groups};
\draw (258.87, -52.60000000000001) rectangle (263.22,-55.20000000000001);
\draw(263.32, -52.60000000000001) node[anchor=north west,align=left] {Torsion-free\\ groups,\\ finite rank};
\draw (263.32, -52.60000000000001) rectangle (266.92,-54.20000000000001);
\draw(258.87, -55.300000000000004) node[anchor=north west,align=left] {Automorphisms,\\ homomorphisms,\\ endomorphisms,\\ etc. for\\ abelian groups};
\draw (258.87, -55.300000000000004) rectangle (262.97,-57.900000000000006);
\draw(263.07, -55.300000000000004) node[anchor=north west,align=left] {Torsion-free\\ groups, \\ infinite rank};
\draw (263.07, -55.300000000000004) rectangle (266.92,-57.400000000000006);
\draw(258.87, -58.00000000000001) node[anchor=north west,align=left] {Homological\\ and \\ categorical \\ methods for\\ abelian groups};
\draw (258.87, -58.00000000000001) rectangle (262.97,-60.60000000000001);
\draw(263.07, -58.00000000000001) node[anchor=north west,align=left] {Extensions\\ of abelian\\ groups};
\draw (263.07, -58.00000000000001) rectangle (266.17,-59.60000000000001);
\draw(258.87, -60.7) node[anchor=north west,align=left] {Topological\\ methods\\ for \\ abelian groups};
\draw (258.87, -60.7) rectangle (262.97,-62.800000000000004);
\draw(263.07, -60.7) node[anchor=north west,align=left] {Finite\\ abelian\\ groups};
\draw (263.07, -60.7) rectangle (265.42,-62.300000000000004);
\draw(258.87, -62.900000000000006) node[anchor=north west,align=left] {Mixed\\ groups};
\draw (258.87, -62.900000000000006) rectangle (260.97,-64.0);
\draw(246.12, -68.69999999999999) node[anchor=north west,align=left] {\LARGE Measure and integration};
\draw (246.12, -68.69999999999999) rectangle (289.83,-100.0);
\draw(247.12, -69.69999999999999) node[anchor=north west,align=left] {\large Set functions, measures and integrals with values in abstract spaces};
\draw (247.12, -69.69999999999999) rectangle (268.8,-73.89999999999999);
\draw(248.12, -70.69999999999999) node[anchor=north west,align=left] {Set-valued set \\ functions and \\ measures; integration\\ of set-valued\\ functions; \\ measurable selections};
\draw (248.12, -70.69999999999999) rectangle (253.97,-73.79999999999998);
\draw(254.07, -70.69999999999999) node[anchor=north west,align=left] {Group- or \\ semigroup-valued\\ set \\ functions, measures\\ and integrals};
\draw (254.07, -70.69999999999999) rectangle (259.42,-73.29999999999998);
\draw(259.52, -70.69999999999999) node[anchor=north west,align=left] {Vector-valued\\ set functions,\\ measures\\ and integrals};
\draw (259.52, -70.69999999999999) rectangle (263.62,-72.79999999999998);
\draw(263.72, -70.69999999999999) node[anchor=north west,align=left] {Set functions,\\ measures and\\ integrals \\ with values in\\ ordered spaces};
\draw (263.72, -70.69999999999999) rectangle (267.82000000000005,-73.29999999999998);
\draw(268.9, -69.69999999999999) node[anchor=north west,align=left] {\large Miscellaneous topics in measure theory};
\draw (268.9, -69.69999999999999) rectangle (281.28,-72.89999999999999);
\draw(269.9, -70.69999999999999) node[anchor=north west,align=left] {Other \\ connections \\ with logic \\ and set theory};
\draw (269.9, -70.69999999999999) rectangle (274.0,-72.79999999999998);
\draw(274.09999999999997, -70.69999999999999) node[anchor=north west,align=left] {Nonstandard\\ measure\\ theory};
\draw (274.09999999999997, -70.69999999999999) rectangle (277.45,-72.29999999999998);
\draw(277.54999999999995, -70.69999999999999) node[anchor=north west,align=left] {Fuzzy\\ measure\\ theory};
\draw (277.54999999999995, -70.69999999999999) rectangle (279.9,-72.29999999999998);
\draw(281.38, -69.69999999999999) node[anchor=north west,align=left] {\large History of measure\\ and integration};
\draw (281.38, -69.69999999999999) rectangle (287.56,-70.79999999999998);
\draw(281.38, -70.89999999999999) node[anchor=north west,align=left] {\large Computational methods for\\ problems pertaining to\\ measure and integration};
\draw (281.38, -70.89999999999999) rectangle (289.73,-72.49999999999999);
\draw(247.12, -73.99999999999999) node[anchor=north west,align=left] {\large Set functions and measures on spaces with additional structure};
\draw (247.12, -73.99999999999999) rectangle (266.97,-81.39999999999999);
\draw(248.12, -74.99999999999999) node[anchor=north west,align=left] {Set functions and\\ measures and integrals\\ in \\ infinite-dimensional spaces\\ (Wiener measure, \\ Gaussian measure, etc.)};
\draw (248.12, -74.99999999999999) rectangle (255.47,-78.09999999999998);
\draw(255.57, -74.99999999999999) node[anchor=north west,align=left] {Integration theory\\ via linear \\ functionals (Radon \\ measures, Daniell \\ integrals, etc.),\\ representing set\\ functions and measures};
\draw (255.57, -74.99999999999999) rectangle (261.67,-78.59999999999998);
\draw(261.77, -74.99999999999999) node[anchor=north west,align=left] {Set functions and\\ measures on \\ topological groups\\ or semigroups,\\ Haar measures, \\ invariant measures};
\draw (261.77, -74.99999999999999) rectangle (266.87,-78.09999999999998);
\draw(248.12, -78.69999999999999) node[anchor=north west,align=left] {Set functions \\ and measures on\\ topological \\ spaces (regularity\\ of measures, etc.)};
\draw (248.12, -78.69999999999999) rectangle (253.22,-81.29999999999998);
\draw(267.07, -73.99999999999999) node[anchor=north west,align=left] {\large Measure-theoretic ergodic theory};
\draw (267.07, -73.99999999999999) rectangle (278.71999999999997,-79.89999999999999);
\draw(268.07, -74.99999999999999) node[anchor=north west,align=left] {General groups\\ of \\ measure-preserving \\ transformations};
\draw (268.07, -74.99999999999999) rectangle (273.42,-77.09999999999998);
\draw(273.52, -74.99999999999999) node[anchor=north west,align=left] {Measure-preserving\\ transformations};
\draw (273.52, -74.99999999999999) rectangle (278.62,-77.09999999999998);
\draw(268.07, -77.19999999999999) node[anchor=north west,align=left] {One-parameter\\ continuous \\ families of \\ measure-preserving\\ transformations};
\draw (268.07, -77.19999999999999) rectangle (273.17,-79.79999999999998);
\draw(273.27, -77.19999999999999) node[anchor=north west,align=left] {Entropy \\ and other\\ invariants};
\draw (273.27, -77.19999999999999) rectangle (276.37,-78.79999999999998);
\draw(247.12, -81.49999999999999) node[anchor=north west,align=left] {\large Classical measure theory};
\draw (247.12, -81.49999999999999) rectangle (258.27,-99.89999999999999);
\draw(248.12, -82.49999999999999) node[anchor=north west,align=left] {Measurable and\\ nonmeasurable \\ functions, sequences\\ of measurable\\ functions,\\ modes of convergence};
\draw (248.12, -82.49999999999999) rectangle (253.72,-85.59999999999998);
\draw(253.82, -82.49999999999999) node[anchor=north west,align=left] {Contents, \\ measures, \\ outer measures,\\ capacities};
\draw (253.82, -82.49999999999999) rectangle (258.17,-84.59999999999998);
\draw(248.12, -85.69999999999999) node[anchor=north west,align=left] {Classes of sets\\ (Borel fields, \\ \(\sigma\)-rings,\\ etc.), measurable\\ sets, Suslin\\ sets, analytic sets};
\draw (248.12, -85.69999999999999) rectangle (253.47,-88.79999999999998);
\draw(253.57, -85.69999999999999) node[anchor=north west,align=left] {Measures \\ on Boolean\\ rings, \\ measure algebras};
\draw (253.57, -85.69999999999999) rectangle (258.17,-87.79999999999998);
\draw(248.12, -88.89999999999999) node[anchor=north west,align=left] {Abstract \\ differentiation \\ theory, \\ differentiation of\\ set functions};
\draw (248.12, -88.89999999999999) rectangle (253.22,-91.49999999999999);
\draw(253.32, -88.89999999999999) node[anchor=north west,align=left] {Real- or\\ complex-valued\\ set\\ functions};
\draw (253.32, -88.89999999999999) rectangle (257.42,-90.99999999999999);
\draw(248.12, -91.6) node[anchor=north west,align=left] {Integration\\ and \\ disintegration\\ of measures};
\draw (248.12, -91.6) rectangle (252.22,-93.69999999999999);
\draw(252.32, -91.6) node[anchor=north west,align=left] {Length, area,\\ volume, other\\ geometric \\ measure theory};
\draw (252.32, -91.6) rectangle (256.42,-93.69999999999999);
\draw(248.12, -93.79999999999998) node[anchor=north west,align=left] {Integration\\ with respect\\ to measures\\ and other \\ set functions};
\draw (248.12, -93.79999999999998) rectangle (251.97,-96.39999999999998);
\draw(252.07, -93.79999999999998) node[anchor=north west,align=left] {Measures \\ and integrals\\ in product\\ spaces};
\draw (252.07, -93.79999999999998) rectangle (255.92,-95.89999999999998);
\draw(248.12, -96.49999999999999) node[anchor=north west,align=left] {Spaces of\\ measures,\\ convergence\\ of measures};
\draw (248.12, -96.49999999999999) rectangle (251.47,-98.59999999999998);
\draw(251.57, -96.49999999999999) node[anchor=north west,align=left] {Hausdorff\\ and packing\\ measures};
\draw (251.57, -96.49999999999999) rectangle (254.92,-98.09999999999998);
\draw(255.02, -96.49999999999999) node[anchor=north west,align=left] {Fractals};
\draw (255.02, -96.49999999999999) rectangle (257.62,-97.59999999999998);
\draw(248.12, -98.69999999999999) node[anchor=north west,align=left] {Lifting\\ theory};
\draw (248.12, -98.69999999999999) rectangle (250.47,-99.79999999999998);
\draw(292.07000000000005, -1) node[anchor=north west,align=left] {\LARGE Category theory; homological algebra};
\draw (292.07000000000005, -1) rectangle (334.52000000000004,-62.800000000000004);
\draw(293.07000000000005, -2) node[anchor=north west,align=left] {\large Homological algebra in category theory, derived categories and functors};
\draw (293.07000000000005, -2) rectangle (319.57000000000005,-11.1);
\draw(294.07000000000005, -3) node[anchor=north west,align=left] {\(A_{\infty}\)-categories,\\ relations \\ with homological\\ mirror symmetry};
\draw (294.07000000000005, -3) rectangle (301.1700000000001,-5.6);
\draw(301.27000000000004, -3) node[anchor=north west,align=left] {Relative homological\\ algebra,\\ projective classes\\ (category-theoretic\\ aspects)};
\draw (301.27000000000004, -3) rectangle (306.87000000000006,-5.6);
\draw(306.97, -3) node[anchor=north west,align=left] {Projectives\\ and injectives\\ (category-theoretic\\ aspects)};
\draw (306.97, -3) rectangle (312.32000000000005,-5.6);
\draw(312.4200000000001, -3) node[anchor=north west,align=left] {Resolutions;\\ derived \\ functors \\ (category-theoretic\\ aspects)};
\draw (312.4200000000001, -3) rectangle (317.7700000000001,-5.6);
\draw(294.07000000000005, -5.7) node[anchor=north west,align=left] {Ext and Tor, \\ generalizations,\\ Künneth formula\\ (category-theoretic\\ aspects)};
\draw (294.07000000000005, -5.7) rectangle (299.4200000000001,-8.3);
\draw(299.52000000000004, -5.7) node[anchor=north west,align=left] {Homological\\ dimension \\ (category-theoretic\\ aspects)};
\draw (299.52000000000004, -5.7) rectangle (304.87000000000006,-7.800000000000001);
\draw(304.97, -5.7) node[anchor=north west,align=left] {Chain complexes\\ (category-theoretic\\ aspects),\\ dg categories};
\draw (304.97, -5.7) rectangle (310.32000000000005,-8.3);
\draw(310.4200000000001, -5.7) node[anchor=north west,align=left] {Nonabelian\\ homological\\ algebra \\ (category-theoretic\\ aspects)};
\draw (310.4200000000001, -5.7) rectangle (315.7700000000001,-8.3);
\draw(315.87000000000006, -5.7) node[anchor=north west,align=left] {Derived \\ categories,\\ triangulated\\ categories};
\draw (315.87000000000006, -5.7) rectangle (319.4700000000001,-7.800000000000001);
\draw(294.07000000000005, -8.4) node[anchor=north west,align=left] {Other \\ (co)homology \\ theories \\ (category-theoretic\\ aspects)};
\draw (294.07000000000005, -8.4) rectangle (299.4200000000001,-11.0);
\draw(299.52000000000004, -8.4) node[anchor=north west,align=left] {2-groups, \\ crossed \\ modules, crossed\\ complexes};
\draw (299.52000000000004, -8.4) rectangle (304.12000000000006,-10.5);
\draw(304.22, -8.4) node[anchor=north west,align=left] {Spectral\\ sequences,\\ hypercohomology};
\draw (304.22, -8.4) rectangle (308.57000000000005,-10.5);
\draw(308.6700000000001, -8.4) node[anchor=north west,align=left] {Simplicial\\ modules and\\ Dold-Kan \\ correspondence};
\draw (308.6700000000001, -8.4) rectangle (312.7700000000001,-10.5);
\draw(312.87000000000006, -8.4) node[anchor=north west,align=left] {Stable \\ module \\ categories};
\draw (312.87000000000006, -8.4) rectangle (315.9700000000001,-10.0);
\draw(316.07000000000005, -8.4) node[anchor=north west,align=left] {Graph \\ complexes\\ and graph\\ homology};
\draw (316.07000000000005, -8.4) rectangle (318.9200000000001,-10.5);
\draw(319.6700000000001, -2) node[anchor=north west,align=left] {\large Categories in geometry and topology};
\draw (319.6700000000001, -2) rectangle (333.82000000000005,-15.5);
\draw(320.6700000000001, -3) node[anchor=north west,align=left] {Presheaves and\\ sheaves, stacks,\\ descent \\ conditions \\ (category-theoretic aspects)};
\draw (320.6700000000001, -3) rectangle (328.2700000000001,-5.6);
\draw(328.37000000000006, -3) node[anchor=north west,align=left] {Abstract \\ manifolds and fiber\\ bundles \\ (category-theoretic\\ aspects)};
\draw (328.37000000000006, -3) rectangle (333.7200000000001,-5.6);
\draw(320.6700000000001, -5.7) node[anchor=north west,align=left] {Synthetic \\ differential geometry,\\ tangent \\ categories, differential\\ categories};
\draw (320.6700000000001, -5.7) rectangle (327.2700000000001,-8.3);
\draw(327.37000000000006, -5.7) node[anchor=north west,align=left] {Algebraic \\ \(K\)-theory and\\ \(L\)-theory\\ (category-theoretic\\ aspects)};
\draw (327.37000000000006, -5.7) rectangle (332.7200000000001,-8.3);
\draw(320.6700000000001, -8.4) node[anchor=north west,align=left] {Grothendieck\\ groups \\ (category-theoretic\\ aspects)};
\draw (320.6700000000001, -8.4) rectangle (326.0200000000001,-10.5);
\draw(326.12000000000006, -8.4) node[anchor=north west,align=left] {Grothendieck\\ topologies\\ and \\ Grothendieck topoi};
\draw (326.12000000000006, -8.4) rectangle (331.2200000000001,-10.5);
\draw(320.6700000000001, -10.600000000000001) node[anchor=north west,align=left] {Frames and \\ locales, pointfree\\ topology,\\ Stone duality};
\draw (320.6700000000001, -10.600000000000001) rectangle (325.7700000000001,-12.700000000000001);
\draw(325.87000000000006, -10.600000000000001) node[anchor=north west,align=left] {Categories\\ of topological\\ spaces\\ and continuous\\ mappings};
\draw (325.87000000000006, -10.600000000000001) rectangle (329.9700000000001,-13.200000000000001);
\draw(330.07000000000005, -10.600000000000001) node[anchor=north west,align=left] {Local \\ categories \\ and functors};
\draw (330.07000000000005, -10.600000000000001) rectangle (333.6700000000001,-12.200000000000001);
\draw(320.6700000000001, -13.3) node[anchor=north west,align=left] {Goodwillie\\ calculus\\ and functor\\ calculus};
\draw (320.6700000000001, -13.3) rectangle (324.0200000000001,-15.4);
\draw(324.12000000000006, -13.3) node[anchor=north west,align=left] {Quantales};
\draw (324.12000000000006, -13.3) rectangle (326.9700000000001,-14.4);
\draw(293.07000000000005, -11.2) node[anchor=north west,align=left] {\large Computational methods\\ for problems pertaining\\ to category theory};
\draw (293.07000000000005, -11.2) rectangle (300.80000000000007,-12.799999999999999);
\draw(293.07000000000005, -12.9) node[anchor=north west,align=left] {\large History of \\ category theory};
\draw (293.07000000000005, -12.9) rectangle (298.32000000000005,-14.0);
\draw(293.07000000000005, -15.6) node[anchor=north west,align=left] {\large Higher categories and homotopical algebra};
\draw (293.07000000000005, -15.6) rectangle (308.72,-30.1);
\draw(294.07000000000005, -16.6) node[anchor=north west,align=left] {\((\infty,  n)\)-categories\\ and \\ \((\infty,\infty)\)-categories};
\draw (294.07000000000005, -16.6) rectangle (302.1700000000001,-19.200000000000003);
\draw(302.27000000000004, -16.6) node[anchor=north west,align=left] {Categories of \\ fibrations, \\ relations to \\ \(K\)-theory, relations\\ to type theory};
\draw (302.27000000000004, -16.6) rectangle (308.62000000000006,-19.200000000000003);
\draw(294.07000000000005, -19.3) node[anchor=north west,align=left] {\((\infty,1)\)-categories\\ (quasi-categories,\\ Segal \\ spaces, etc.); \\ \(\infty\)-topoi, stable\\ \(\infty\)-categories};
\draw (294.07000000000005, -19.3) rectangle (300.9200000000001,-22.400000000000002);
\draw(301.02000000000004, -19.3) node[anchor=north west,align=left] {Localizations\\ (e.g., \\ simplicial localization,\\ Bousfield\\ localization)};
\draw (301.02000000000004, -19.3) rectangle (307.62000000000006,-21.900000000000002);
\draw(294.07000000000005, -22.5) node[anchor=north west,align=left] {Tricategories,\\ weak \\ \(n\)-categories, \\ coherence, \\ semi-strictification};
\draw (294.07000000000005, -22.5) rectangle (299.6700000000001,-25.1);
\draw(299.77000000000004, -22.5) node[anchor=north west,align=left] {\(\infty\)-operads\\ and higher\\ algebra};
\draw (299.77000000000004, -22.5) rectangle (304.87000000000006,-24.6);
\draw(304.97, -22.5) node[anchor=north west,align=left] {Simplicial\\ sets,\\ simplicial\\ objects};
\draw (304.97, -22.5) rectangle (308.07000000000005,-24.6);
\draw(294.07000000000005, -25.200000000000003) node[anchor=north west,align=left] {Strict \\ omega-categories,\\ computads,\\ polygraphs};
\draw (294.07000000000005, -25.200000000000003) rectangle (298.9200000000001,-27.300000000000004);
\draw(299.02000000000004, -25.200000000000003) node[anchor=north west,align=left] {Categorification};
\draw (299.02000000000004, -25.200000000000003) rectangle (303.62000000000006,-26.300000000000004);
\draw(303.72, -25.200000000000003) node[anchor=north west,align=left] {2-categories,\\ bicategories,\\ double\\ categories};
\draw (303.72, -25.200000000000003) rectangle (307.57000000000005,-27.300000000000004);
\draw(294.07000000000005, -27.4) node[anchor=north west,align=left] {2-dimensional\\ monad theory};
\draw (294.07000000000005, -27.4) rectangle (297.9200000000001,-29.0);
\draw(298.02000000000004, -27.4) node[anchor=north west,align=left] {Homotopical\\ algebra, \\ Quillen model\\ categories,\\ derivators};
\draw (298.02000000000004, -27.4) rectangle (301.87000000000006,-30.0);
\draw(308.82000000000005, -15.6) node[anchor=north west,align=left] {\large General theory of categories and functors};
\draw (308.82000000000005, -15.6) rectangle (324.1700000000001,-31.6);
\draw(309.82000000000005, -16.6) node[anchor=north west,align=left] {Factorization \\ systems, substructures,\\ quotient \\ structures, \\ congruences, amalgams};
\draw (309.82000000000005, -16.6) rectangle (316.1700000000001,-19.200000000000003);
\draw(316.27000000000004, -16.6) node[anchor=north west,align=left] {Limits and colimits\\ (products, sums,\\ directed limits,\\ pushouts, fiber\\ products, \\ equalizers, kernels, \\ ends and coends, etc.)};
\draw (316.27000000000004, -16.6) rectangle (322.37000000000006,-20.200000000000003);
\draw(309.82000000000005, -20.3) node[anchor=north west,align=left] {Categories \\ admitting limits \\ (complete categories),\\ functors\\ preserving limits,\\ completions};
\draw (309.82000000000005, -20.3) rectangle (315.9200000000001,-23.400000000000002);
\draw(316.02000000000004, -20.3) node[anchor=north west,align=left] {Special \\ properties of \\ functors (faithful,\\ full, etc.)};
\draw (316.02000000000004, -20.3) rectangle (321.37000000000006,-22.400000000000002);
\draw(309.82000000000005, -23.5) node[anchor=north west,align=left] {Adjoint functors\\ (universal \\ constructions, \\ reflective \\ subcategories, Kan \\ extensions, etc.)};
\draw (309.82000000000005, -23.5) rectangle (315.1700000000001,-26.6);
\draw(315.27000000000004, -23.5) node[anchor=north west,align=left] {Foundations,\\ relations to\\ logic and \\ deductive systems};
\draw (315.27000000000004, -23.5) rectangle (320.12000000000006,-25.6);
\draw(320.22, -23.5) node[anchor=north west,align=left] {Graphs, \\ diagram \\ schemes, \\ precategories};
\draw (320.22, -23.5) rectangle (324.07000000000005,-25.6);
\draw(309.82000000000005, -26.700000000000003) node[anchor=north west,align=left] {Definitions\\ and \\ generalizations \\ in theory of\\ categories};
\draw (309.82000000000005, -26.700000000000003) rectangle (314.4200000000001,-29.300000000000004);
\draw(314.52000000000004, -26.700000000000003) node[anchor=north west,align=left] {Epimorphisms,\\ monomorphisms,\\ special classes\\ of morphisms,\\ null morphisms};
\draw (314.52000000000004, -26.700000000000003) rectangle (318.87000000000006,-29.300000000000004);
\draw(318.97, -26.700000000000003) node[anchor=north west,align=left] {Functor \\ categories,\\ comma \\ categories};
\draw (318.97, -26.700000000000003) rectangle (322.32000000000005,-28.800000000000004);
\draw(309.82000000000005, -29.4) node[anchor=north west,align=left] {Natural \\ morphisms,\\ dinatural\\ morphisms};
\draw (309.82000000000005, -29.4) rectangle (312.9200000000001,-31.5);
\draw(313.02000000000004, -29.4) node[anchor=north west,align=left] {Graded \\ categories\\ (general)};
\draw (313.02000000000004, -29.4) rectangle (316.12000000000006,-31.0);
\draw(324.27000000000004, -15.6) node[anchor=north west,align=left] {\large Categories and theories};
\draw (324.27000000000004, -15.6) rectangle (334.42,-27.4);
\draw(325.27000000000004, -16.6) node[anchor=north west,align=left] {Monads (= standard\\ construction, \\ triple or triad), \\ algebras for monads,\\ homology and derived\\ functors for monads};
\draw (325.27000000000004, -16.6) rectangle (330.87000000000006,-19.700000000000003);
\draw(330.97, -16.6) node[anchor=north west,align=left] {Accessible\\ and locally\\ presentable\\ categories};
\draw (330.97, -16.6) rectangle (334.32000000000005,-18.700000000000003);
\draw(325.27000000000004, -19.8) node[anchor=north west,align=left] {Theories \\ (e.g., algebraic\\ theories),\\ structure,\\ and semantics};
\draw (325.27000000000004, -19.8) rectangle (329.87000000000006,-22.400000000000002);
\draw(329.97, -19.8) node[anchor=north west,align=left] {Eilenberg-Moore\\ and Kleisli\\ constructions\\ for monads};
\draw (329.97, -19.8) rectangle (334.32000000000005,-21.900000000000002);
\draw(325.27000000000004, -22.5) node[anchor=north west,align=left] {Sketches\\ and \\ generalizations};
\draw (325.27000000000004, -22.5) rectangle (329.62000000000006,-24.1);
\draw(329.72, -22.5) node[anchor=north west,align=left] {Structured\\ objects \\ in a category\\ (group\\ objects, etc.)};
\draw (329.72, -22.5) rectangle (333.82000000000005,-25.1);
\draw(325.27000000000004, -25.200000000000003) node[anchor=north west,align=left] {Categorical\\ semantics\\ of formal\\ languages};
\draw (325.27000000000004, -25.200000000000003) rectangle (328.62000000000006,-27.300000000000004);
\draw(328.72, -25.200000000000003) node[anchor=north west,align=left] {Equational\\ categories};
\draw (328.72, -25.200000000000003) rectangle (331.82000000000005,-26.800000000000004);
\draw(293.07000000000005, -31.700000000000003) node[anchor=north west,align=left] {\large Monoidal categories and operads};
\draw (293.07000000000005, -31.700000000000003) rectangle (305.47,-51.800000000000004);
\draw(294.07000000000005, -32.7) node[anchor=north west,align=left] {Polycategories/dioperads,\\ properads, PROPs, \\ cyclic operads,\\ modular operads};
\draw (294.07000000000005, -32.7) rectangle (300.9200000000001,-35.300000000000004);
\draw(301.02000000000004, -32.7) node[anchor=north west,align=left] {Algebraic \\ operads, \\ cooperads, and \\ Koszul duality};
\draw (301.02000000000004, -32.7) rectangle (305.37000000000006,-34.800000000000004);
\draw(294.07000000000005, -35.400000000000006) node[anchor=north west,align=left] {Braided \\ monoidal categories\\ and ribbon\\ categories};
\draw (294.07000000000005, -35.400000000000006) rectangle (299.4200000000001,-37.50000000000001);
\draw(299.52000000000004, -35.400000000000006) node[anchor=north west,align=left] {Fusion \\ categories, modular\\ tensor \\ categories, \\ modular functors};
\draw (299.52000000000004, -35.400000000000006) rectangle (304.87000000000006,-38.00000000000001);
\draw(294.07000000000005, -38.1) node[anchor=north west,align=left] {Dagger \\ categories, \\ categorical quantum\\ mechanics};
\draw (294.07000000000005, -38.1) rectangle (299.4200000000001,-40.2);
\draw(299.52000000000004, -38.1) node[anchor=north west,align=left] {Monoidal \\ categories, \\ symmetric monoidal\\ categories};
\draw (299.52000000000004, -38.1) rectangle (304.62000000000006,-40.2);
\draw(294.07000000000005, -40.300000000000004) node[anchor=north west,align=left] {Traced monoidal\\ categories,\\ compact \\ closed categories,\\ star-autonomous\\ categories};
\draw (294.07000000000005, -40.300000000000004) rectangle (299.1700000000001,-43.400000000000006);
\draw(299.27000000000004, -40.300000000000004) node[anchor=north west,align=left] {Categories\\ of networks\\ and \\ processes, \\ compositionality};
\draw (299.27000000000004, -40.300000000000004) rectangle (303.87000000000006,-42.900000000000006);
\draw(294.07000000000005, -43.5) node[anchor=north west,align=left] {Non-symmetric\\ operads, \\ multicategories,\\ generalized\\ multicategories};
\draw (294.07000000000005, -43.5) rectangle (298.6700000000001,-46.1);
\draw(298.77000000000004, -43.5) node[anchor=north west,align=left] {Bimonoidal,\\ skew-monoidal,\\ duoidal\\ categories};
\draw (298.77000000000004, -43.5) rectangle (302.87000000000006,-45.6);
\draw(294.07000000000005, -46.2) node[anchor=north west,align=left] {Species, \\ Hopf monoids,\\ operads in\\ combinatorics};
\draw (294.07000000000005, -46.2) rectangle (297.9200000000001,-48.300000000000004);
\draw(298.02000000000004, -46.2) node[anchor=north west,align=left] {String \\ diagrams and\\ graphical\\ calculi};
\draw (298.02000000000004, -46.2) rectangle (301.62000000000006,-48.300000000000004);
\draw(301.72, -46.2) node[anchor=north west,align=left] {Categorical\\ aspects\\ of \\ linear logic};
\draw (301.72, -46.2) rectangle (305.32000000000005,-48.300000000000004);
\draw(294.07000000000005, -48.400000000000006) node[anchor=north west,align=left] {Topological\\ and\\ simplicial\\ operads};
\draw (294.07000000000005, -48.400000000000006) rectangle (297.4200000000001,-50.50000000000001);
\draw(297.52000000000004, -48.400000000000006) node[anchor=north west,align=left] {Tannakian\\ categories};
\draw (297.52000000000004, -48.400000000000006) rectangle (300.62000000000006,-50.00000000000001);
\draw(300.72, -48.400000000000006) node[anchor=north west,align=left] {Operads\\ (general)};
\draw (300.72, -48.400000000000006) rectangle (303.57000000000005,-50.00000000000001);
\draw(294.07000000000005, -50.6) node[anchor=north west,align=left] {Globular\\ operads};
\draw (294.07000000000005, -50.6) rectangle (296.6700000000001,-51.7);
\draw(305.57000000000005, -31.700000000000003) node[anchor=north west,align=left] {\large Categorical structures};
\draw (305.57000000000005, -31.700000000000003) rectangle (316.47,-43.0);
\draw(306.57000000000005, -32.7) node[anchor=north west,align=left] {Proarrow equipments,\\ Yoneda \\ structures, KZ \\ doctrines (lax \\ idempotent monads)};
\draw (306.57000000000005, -32.7) rectangle (312.1700000000001,-35.300000000000004);
\draw(312.27000000000004, -32.7) node[anchor=north west,align=left] {Enriched \\ categories \\ (over closed\\ or monoidal\\ categories)};
\draw (312.27000000000004, -32.7) rectangle (315.87000000000006,-35.300000000000004);
\draw(306.57000000000005, -35.400000000000006) node[anchor=north west,align=left] {Actions of a\\ monoidal \\ category, \\ tensorial strength};
\draw (306.57000000000005, -35.400000000000006) rectangle (311.6700000000001,-37.50000000000001);
\draw(311.77000000000004, -35.400000000000006) node[anchor=north west,align=left] {Profunctors\\ (= \\ correspondences,\\ distributors,\\ modules)};
\draw (311.77000000000004, -35.400000000000006) rectangle (316.37000000000006,-38.00000000000001);
\draw(306.57000000000005, -38.1) node[anchor=north west,align=left] {Closed categories\\ (closed \\ monoidal and \\ Cartesian closed\\ categories, etc.)};
\draw (306.57000000000005, -38.1) rectangle (311.4200000000001,-40.7);
\draw(311.52000000000004, -38.1) node[anchor=north west,align=left] {Fibered\\ categories};
\draw (311.52000000000004, -38.1) rectangle (314.62000000000006,-39.7);
\draw(306.57000000000005, -40.800000000000004) node[anchor=north west,align=left] {Internal\\ categories\\ and\\ groupoids};
\draw (306.57000000000005, -40.800000000000004) rectangle (309.6700000000001,-42.900000000000006);
\draw(309.77000000000004, -40.800000000000004) node[anchor=north west,align=left] {Formal\\ category\\ theory};
\draw (309.77000000000004, -40.800000000000004) rectangle (312.37000000000006,-42.400000000000006);
\draw(316.57000000000005, -31.700000000000003) node[anchor=north west,align=left] {\large Categorical algebra};
\draw (316.57000000000005, -31.700000000000003) rectangle (326.72,-43.7);
\draw(317.57000000000005, -32.7) node[anchor=north west,align=left] {Protomodular\\ categories, \\ semi-abelian \\ categories, \\ Mal’tsev categories};
\draw (317.57000000000005, -32.7) rectangle (322.9200000000001,-35.300000000000004);
\draw(323.02000000000004, -32.7) node[anchor=north west,align=left] {Preadditive,\\ additive\\ categories};
\draw (323.02000000000004, -32.7) rectangle (326.62000000000006,-34.300000000000004);
\draw(317.57000000000005, -35.400000000000006) node[anchor=north west,align=left] {Definable \\ subcategories\\ and \\ connections with\\ model theory};
\draw (317.57000000000005, -35.400000000000006) rectangle (322.1700000000001,-38.00000000000001);
\draw(322.27000000000004, -35.400000000000006) node[anchor=north west,align=left] {Localization\\ of categories,\\ calculus\\ of fractions};
\draw (322.27000000000004, -35.400000000000006) rectangle (326.37000000000006,-37.50000000000001);
\draw(317.57000000000005, -38.1) node[anchor=north west,align=left] {Abelian \\ categories,\\ Grothendieck\\ categories};
\draw (317.57000000000005, -38.1) rectangle (321.1700000000001,-40.2);
\draw(321.27000000000004, -38.1) node[anchor=north west,align=left] {Regular \\ categories,\\ Barr-exact\\ categories};
\draw (321.27000000000004, -38.1) rectangle (324.62000000000006,-40.2);
\draw(317.57000000000005, -40.300000000000004) node[anchor=north west,align=left] {Categorical\\ embedding\\ theorems};
\draw (317.57000000000005, -40.300000000000004) rectangle (320.9200000000001,-41.900000000000006);
\draw(321.02000000000004, -40.300000000000004) node[anchor=north west,align=left] {Categorical\\ Galois\\ theory};
\draw (321.02000000000004, -40.300000000000004) rectangle (324.37000000000006,-41.900000000000006);
\draw(317.57000000000005, -42.0) node[anchor=north west,align=left] {Torsion\\ theories,\\ radicals};
\draw (317.57000000000005, -42.0) rectangle (320.4200000000001,-43.6);
\draw(293.07000000000005, -51.900000000000006) node[anchor=north west,align=left] {\large Special categories};
\draw (293.07000000000005, -51.900000000000006) rectangle (302.97,-62.7);
\draw(294.07000000000005, -52.900000000000006) node[anchor=north west,align=left] {Categories\\ of sets,\\ characterizations};
\draw (294.07000000000005, -52.900000000000006) rectangle (298.9200000000001,-55.00000000000001);
\draw(299.02000000000004, -52.900000000000006) node[anchor=north west,align=left] {Extensive, \\ distributive,\\ and adhesive\\ categories};
\draw (299.02000000000004, -52.900000000000006) rectangle (302.87000000000006,-55.00000000000001);
\draw(294.07000000000005, -55.10000000000001) node[anchor=north west,align=left] {Categories\\ of spans/cospans,\\ relations, or \\ partial maps};
\draw (294.07000000000005, -55.10000000000001) rectangle (298.9200000000001,-57.70000000000001);
\draw(299.02000000000004, -55.10000000000001) node[anchor=north west,align=left] {Categories\\ of machines,\\ automata};
\draw (299.02000000000004, -55.10000000000001) rectangle (302.62000000000006,-56.70000000000001);
\draw(299.02000000000004, -56.800000000000004) node[anchor=north west,align=left] {Topoi};
\draw (299.02000000000004, -56.800000000000004) rectangle (300.87000000000006,-57.400000000000006);
\draw(294.07000000000005, -57.800000000000004) node[anchor=north west,align=left] {Preorders, \\ orders, domains\\ and lattices\\ (viewed\\ as categories)};
\draw (294.07000000000005, -57.800000000000004) rectangle (298.4200000000001,-60.400000000000006);
\draw(298.52000000000004, -57.800000000000004) node[anchor=north west,align=left] {Groupoids, \\ semigroupoids,\\ semigroups, \\ groups (viewed\\ as categories)};
\draw (298.52000000000004, -57.800000000000004) rectangle (302.62000000000006,-60.400000000000006);
\draw(294.07000000000005, -60.50000000000001) node[anchor=north west,align=left] {Embedding\\ theorems,\\ universal\\ categories};
\draw (294.07000000000005, -60.50000000000001) rectangle (297.1700000000001,-62.60000000000001);
\draw(334.62000000000006, -1) node[anchor=north west,align=left] {\LARGE Algebraic geometry};
\draw (334.62000000000006, -1) rectangle (375.2200000000001,-89.4);
\draw(335.62000000000006, -2) node[anchor=north west,align=left] {\large Arithmetic problems in algebraic geometry; Diophantine geometry};
\draw (335.62000000000006, -2) rectangle (359.4200000000001,-11.1);
\draw(336.62000000000006, -3) node[anchor=north west,align=left] {Zeta functions \\ and related questions\\ in algebraic\\ geometry (e.g.,\\ Birch-Swinnerton-Dyer\\ conjecture)};
\draw (336.62000000000006, -3) rectangle (342.4700000000001,-6.1);
\draw(342.57000000000005, -3) node[anchor=north west,align=left] {Hasse principle,\\ weak and \\ strong approximation,\\ Brauer-Manin\\ obstruction};
\draw (342.57000000000005, -3) rectangle (348.4200000000001,-5.6);
\draw(348.52000000000004, -3) node[anchor=north west,align=left] {Universal profinite\\ groups (relationship\\ to moduli\\ spaces, projective\\ and moduli towers,\\ Galois theory)};
\draw (348.52000000000004, -3) rectangle (354.12000000000006,-6.1);
\draw(354.2200000000001, -3) node[anchor=north west,align=left] {Positive \\ characteristic\\ ground \\ fields in \\ algebraic geometry};
\draw (354.2200000000001, -3) rectangle (359.3200000000001,-5.6);
\draw(336.62000000000006, -6.2) node[anchor=north west,align=left] {Other \\ nonalgebraically \\ closed ground \\ fields in algebraic\\ geometry};
\draw (336.62000000000006, -6.2) rectangle (341.9700000000001,-8.8);
\draw(342.07000000000005, -6.2) node[anchor=north west,align=left] {Applications\\ to coding \\ theory and \\ cryptography of \\ arithmetic geometry};
\draw (342.07000000000005, -6.2) rectangle (347.4200000000001,-8.8);
\draw(347.52000000000004, -6.2) node[anchor=north west,align=left] {Arithmetic\\ varieties \\ and schemes;\\ Arakelov \\ theory; heights};
\draw (347.52000000000004, -6.2) rectangle (351.87000000000006,-8.8);
\draw(351.9700000000001, -6.2) node[anchor=north west,align=left] {Perfectoid\\ spaces and\\ mixed \\ characteristic};
\draw (351.9700000000001, -6.2) rectangle (356.0700000000001,-8.3);
\draw(356.1700000000001, -6.2) node[anchor=north west,align=left] {Rational\\ points};
\draw (356.1700000000001, -6.2) rectangle (358.7700000000001,-7.300000000000001);
\draw(336.62000000000006, -8.9) node[anchor=north west,align=left] {Finite \\ ground fields\\ in algebraic\\ geometry};
\draw (336.62000000000006, -8.9) rectangle (340.4700000000001,-11.0);
\draw(340.57000000000005, -8.9) node[anchor=north west,align=left] {Global \\ ground fields\\ in algebraic\\ geometry};
\draw (340.57000000000005, -8.9) rectangle (344.4200000000001,-11.0);
\draw(344.52000000000004, -8.9) node[anchor=north west,align=left] {Local ground\\ fields\\ in algebraic\\ geometry};
\draw (344.52000000000004, -8.9) rectangle (348.12000000000006,-11.0);
\draw(348.2200000000001, -8.9) node[anchor=north west,align=left] {Modular \\ and Shimura\\ varieties};
\draw (348.2200000000001, -8.9) rectangle (351.5700000000001,-10.5);
\draw(351.6700000000001, -8.9) node[anchor=north west,align=left] {Rigid \\ analytic\\ geometry};
\draw (351.6700000000001, -8.9) rectangle (354.2700000000001,-10.5);
\draw(359.5200000000001, -2) node[anchor=north west,align=left] {\large (Co)homology theory in algebraic geometry};
\draw (359.5200000000001, -2) rectangle (375.1200000000001,-16.0);
\draw(360.5200000000001, -3) node[anchor=north west,align=left] {Other algebro-geometric\\ (co)homologies\\ (e.g., \\ intersection, equivariant,\\ Lawson, Deligne\\ (co)homologies)};
\draw (360.5200000000001, -3) rectangle (367.6200000000001,-6.1);
\draw(367.7200000000001, -3) node[anchor=north west,align=left] {Differentials\\ and other special\\ sheaves; \\ D-modules; \\ Bernstein-Sato \\ ideals and polynomials};
\draw (367.7200000000001, -3) rectangle (373.8200000000001,-6.1);
\draw(360.5200000000001, -6.2) node[anchor=north west,align=left] {Derived categories\\ of sheaves,\\ dg categories,\\ and related \\ constructions in\\ algebraic geometry};
\draw (360.5200000000001, -6.2) rectangle (365.6200000000001,-9.3);
\draw(365.7200000000001, -6.2) node[anchor=north west,align=left] {Homotopy \\ theory and \\ fundamental groups\\ in algebraic\\ geometry};
\draw (365.7200000000001, -6.2) rectangle (370.8200000000001,-8.8);
\draw(370.9200000000001, -6.2) node[anchor=north west,align=left] {Étale and \\ other \\ Grothendieck \\ topologies and\\ (co)homologies};
\draw (370.9200000000001, -6.2) rectangle (375.0200000000001,-8.8);
\draw(360.5200000000001, -9.4) node[anchor=north west,align=left] {Classical \\ real and complex\\ (co)homology\\ in algebraic\\ geometry};
\draw (360.5200000000001, -9.4) rectangle (365.1200000000001,-12.0);
\draw(365.2200000000001, -9.4) node[anchor=north west,align=left] {Motivic \\ cohomology;\\ motivic \\ homotopy theory};
\draw (365.2200000000001, -9.4) rectangle (369.5700000000001,-11.5);
\draw(369.6700000000001, -9.4) node[anchor=north west,align=left] {de Rham \\ cohomology \\ and algebraic\\ geometry};
\draw (369.6700000000001, -9.4) rectangle (373.5200000000001,-11.5);
\draw(360.5200000000001, -12.100000000000001) node[anchor=north west,align=left] {Sheaves \\ in algebraic\\ geometry};
\draw (360.5200000000001, -12.100000000000001) rectangle (364.1200000000001,-13.700000000000001);
\draw(364.2200000000001, -12.100000000000001) node[anchor=north west,align=left] {Vanishing\\ theorems \\ in algebraic\\ geometry};
\draw (364.2200000000001, -12.100000000000001) rectangle (367.8200000000001,-14.200000000000001);
\draw(367.9200000000001, -12.100000000000001) node[anchor=north west,align=left] {Topological\\ properties\\ in algebraic\\ geometry};
\draw (367.9200000000001, -12.100000000000001) rectangle (371.5200000000001,-14.200000000000001);
\draw(371.6200000000001, -12.100000000000001) node[anchor=north west,align=left] {\(p\)-adic\\ cohomology,\\ crystalline\\ cohomology};
\draw (371.6200000000001, -12.100000000000001) rectangle (374.97000000000014,-14.200000000000001);
\draw(360.5200000000001, -14.3) node[anchor=north west,align=left] {Multiplier\\ ideals};
\draw (360.5200000000001, -14.3) rectangle (363.6200000000001,-15.9);
\draw(363.7200000000001, -14.3) node[anchor=north west,align=left] {Brauer \\ groups of\\ schemes};
\draw (363.7200000000001, -14.3) rectangle (366.5700000000001,-15.9);
\draw(335.62000000000006, -11.2) node[anchor=north west,align=left] {\large History of \\ algebraic geometry};
\draw (335.62000000000006, -11.2) rectangle (341.80000000000007,-12.299999999999999);
\draw(335.62000000000006, -16.1) node[anchor=north west,align=left] {\large Projective and enumerative algebraic geometry};
\draw (335.62000000000006, -16.1) rectangle (352.4700000000001,-25.700000000000003);
\draw(336.62000000000006, -17.1) node[anchor=north west,align=left] {Gromov-Witten \\ invariants, quantum \\ cohomology, Gopakumar-Vafa\\ invariants,\\ Donaldson-Thomas\\ invariants \\ (algebro-geometric aspects)};
\draw (336.62000000000006, -17.1) rectangle (343.9700000000001,-20.700000000000003);
\draw(344.07000000000005, -17.1) node[anchor=north west,align=left] {Enumerative \\ problems \\ (combinatorial \\ problems) in \\ algebraic geometry};
\draw (344.07000000000005, -17.1) rectangle (349.1700000000001,-19.700000000000003);
\draw(349.27000000000004, -17.1) node[anchor=north west,align=left] {Varieties\\ of \\ low degree};
\draw (349.27000000000004, -17.1) rectangle (352.37000000000006,-18.700000000000003);
\draw(336.62000000000006, -20.8) node[anchor=north west,align=left] {Secant \\ varieties, tensor\\ rank, \\ varieties of\\ sums of powers};
\draw (336.62000000000006, -20.8) rectangle (341.4700000000001,-23.400000000000002);
\draw(341.57000000000005, -20.8) node[anchor=north west,align=left] {Configurations\\ and \\ arrangements of \\ linear subspaces};
\draw (341.57000000000005, -20.8) rectangle (346.1700000000001,-22.900000000000002);
\draw(346.27000000000004, -20.8) node[anchor=north west,align=left] {Projective\\ techniques\\ in algebraic\\ geometry};
\draw (346.27000000000004, -20.8) rectangle (349.87000000000006,-22.900000000000002);
\draw(336.62000000000006, -23.5) node[anchor=north west,align=left] {Adjunction\\ problems};
\draw (336.62000000000006, -23.5) rectangle (339.7200000000001,-25.1);
\draw(339.82000000000005, -23.5) node[anchor=north west,align=left] {Classical\\ problems,\\ Schubert\\ calculus};
\draw (339.82000000000005, -23.5) rectangle (342.6700000000001,-25.6);
\draw(352.57000000000005, -16.1) node[anchor=north west,align=left] {\large Families, fibrations in algebraic geometry};
\draw (352.57000000000005, -16.1) rectangle (368.6700000000001,-28.900000000000002);
\draw(353.57000000000005, -17.1) node[anchor=north west,align=left] {Applications of \\ vector bundles and\\ moduli spaces in\\ mathematical \\ physics (twistor \\ theory, instantons,\\ quantum field theory)};
\draw (353.57000000000005, -17.1) rectangle (359.4200000000001,-20.700000000000003);
\draw(359.52000000000004, -17.1) node[anchor=north west,align=left] {Structure \\ of families\\ (Picard-Lefschetz,\\ monodromy, etc.)};
\draw (359.52000000000004, -17.1) rectangle (364.62000000000006,-19.700000000000003);
\draw(364.72, -17.1) node[anchor=north west,align=left] {Fine and\\ coarse \\ moduli spaces};
\draw (364.72, -17.1) rectangle (368.57000000000005,-18.700000000000003);
\draw(353.57000000000005, -20.8) node[anchor=north west,align=left] {Fibrations,\\ degenerations\\ in \\ algebraic geometry};
\draw (353.57000000000005, -20.8) rectangle (358.6700000000001,-22.900000000000002);
\draw(358.77000000000004, -20.8) node[anchor=north west,align=left] {Variation \\ of Hodge \\ structures \\ (algebro-geometric\\ aspects)};
\draw (358.77000000000004, -20.8) rectangle (363.87000000000006,-23.400000000000002);
\draw(363.97, -20.8) node[anchor=north west,align=left] {Algebraic \\ moduli problems,\\ moduli of\\ vector bundles};
\draw (363.97, -20.8) rectangle (368.57000000000005,-22.900000000000002);
\draw(353.57000000000005, -23.5) node[anchor=north west,align=left] {Formal methods\\ and deformations\\ in \\ algebraic geometry};
\draw (353.57000000000005, -23.5) rectangle (358.6700000000001,-25.6);
\draw(358.77000000000004, -23.5) node[anchor=north west,align=left] {Geometric \\ Langlands \\ program \\ (algebro-geometric\\ aspects)};
\draw (358.77000000000004, -23.5) rectangle (363.87000000000006,-26.1);
\draw(363.97, -23.5) node[anchor=north west,align=left] {Stacks \\ and moduli\\ problems};
\draw (363.97, -23.5) rectangle (367.07000000000005,-25.1);
\draw(353.57000000000005, -26.200000000000003) node[anchor=north west,align=left] {Arithmetic ground\\ fields (finite,\\ local, global)\\ and families\\ or fibrations};
\draw (353.57000000000005, -26.200000000000003) rectangle (358.4200000000001,-28.800000000000004);
\draw(335.62000000000006, -29.0) node[anchor=north west,align=left] {\large Surfaces and higher-dimensional varieties};
\draw (335.62000000000006, -29.0) rectangle (351.2200000000001,-45.900000000000006);
\draw(336.62000000000006, -30.0) node[anchor=north west,align=left] {Arithmetic \\ ground fields \\ for surfaces \\ or higher-dimensional\\ varieties};
\draw (336.62000000000006, -30.0) rectangle (342.4700000000001,-32.6);
\draw(342.57000000000005, -30.0) node[anchor=north west,align=left] {Topology of \\ surfaces (Donaldson\\ polynomials,\\ Seiberg-Witten\\ invariants)};
\draw (342.57000000000005, -30.0) rectangle (347.9200000000001,-32.6);
\draw(348.02000000000004, -30.0) node[anchor=north west,align=left] {Surfaces\\ of general\\ type};
\draw (348.02000000000004, -30.0) rectangle (351.12000000000006,-31.6);
\draw(336.62000000000006, -32.7) node[anchor=north west,align=left] {Moduli, \\ classification: \\ analytic theory;\\ relations \\ with modular forms};
\draw (336.62000000000006, -32.7) rectangle (341.7200000000001,-35.300000000000004);
\draw(341.82000000000005, -32.7) node[anchor=north west,align=left] {Singularities\\ of surfaces\\ or \\ higher-dimensional\\ varieties};
\draw (341.82000000000005, -32.7) rectangle (346.9200000000001,-35.300000000000004);
\draw(347.02000000000004, -32.7) node[anchor=north west,align=left] {Hypersurfaces\\ and\\ algebraic\\ geometry};
\draw (347.02000000000004, -32.7) rectangle (350.87000000000006,-34.800000000000004);
\draw(336.62000000000006, -35.4) node[anchor=north west,align=left] {Elliptic \\ surfaces, elliptic\\ or Calabi-Yau\\ fibrations};
\draw (336.62000000000006, -35.4) rectangle (341.7200000000001,-37.5);
\draw(341.82000000000005, -35.4) node[anchor=north west,align=left] {Calabi-Yau \\ manifolds \\ (algebro-geometric\\ aspects)};
\draw (341.82000000000005, -35.4) rectangle (346.9200000000001,-37.5);
\draw(347.02000000000004, -35.4) node[anchor=north west,align=left] {Relationships\\ with physics};
\draw (347.02000000000004, -35.4) rectangle (350.87000000000006,-37.0);
\draw(336.62000000000006, -37.6) node[anchor=north west,align=left] {Mirror \\ symmetry \\ (algebro-geometric\\ aspects)};
\draw (336.62000000000006, -37.6) rectangle (341.7200000000001,-39.7);
\draw(341.82000000000005, -37.6) node[anchor=north west,align=left] {Automorphisms\\ of surfaces\\ and \\ higher-dimensional\\ varieties};
\draw (341.82000000000005, -37.6) rectangle (346.9200000000001,-40.2);
\draw(347.02000000000004, -37.6) node[anchor=north west,align=left] {\(K3\) \\ surfaces and\\ Enriques\\ surfaces};
\draw (347.02000000000004, -37.6) rectangle (350.62000000000006,-39.7);
\draw(336.62000000000006, -40.3) node[anchor=north west,align=left] {Vector bundles\\ on surfaces and\\ higher-dimensional\\ varieties,\\ and their moduli};
\draw (336.62000000000006, -40.3) rectangle (341.7200000000001,-42.9);
\draw(341.82000000000005, -40.3) node[anchor=north west,align=left] {Families, \\ moduli, \\ classification: \\ algebraic theory};
\draw (341.82000000000005, -40.3) rectangle (346.4200000000001,-42.4);
\draw(346.52000000000004, -40.3) node[anchor=north west,align=left] {Holomorphic\\ symplectic\\ varieties,\\ hyper-Kähler\\ varieties};
\draw (346.52000000000004, -40.3) rectangle (350.12000000000006,-42.9);
\draw(336.62000000000006, -43.0) node[anchor=north west,align=left] {\(3\)-folds};
\draw (336.62000000000006, -43.0) rectangle (339.9700000000001,-44.1);
\draw(340.07000000000005, -43.0) node[anchor=north west,align=left] {\(4\)-folds};
\draw (340.07000000000005, -43.0) rectangle (343.4200000000001,-44.1);
\draw(343.52000000000004, -43.0) node[anchor=north west,align=left] {\(n\)-folds\\ (\(n>4\))};
\draw (343.52000000000004, -43.0) rectangle (346.87000000000006,-44.6);
\draw(346.9700000000001, -43.0) node[anchor=north west,align=left] {Rational\\ and ruled\\ surfaces};
\draw (346.9700000000001, -43.0) rectangle (349.8200000000001,-44.6);
\draw(336.62000000000006, -44.7) node[anchor=north west,align=left] {Fano \\ varieties};
\draw (336.62000000000006, -44.7) rectangle (339.4700000000001,-45.800000000000004);
\draw(339.57000000000005, -44.7) node[anchor=north west,align=left] {Special\\ surfaces};
\draw (339.57000000000005, -44.7) rectangle (342.1700000000001,-45.800000000000004);
\draw(351.32000000000005, -29.0) node[anchor=north west,align=left] {\large Computational aspects in algebraic geometry};
\draw (351.32000000000005, -29.0) rectangle (366.9200000000001,-37.1);
\draw(352.32000000000005, -30.0) node[anchor=north west,align=left] {Geometric \\ aspects of \\ numerical algebraic\\ geometry};
\draw (352.32000000000005, -30.0) rectangle (357.6700000000001,-32.1);
\draw(357.77000000000004, -30.0) node[anchor=north west,align=left] {Computational\\ aspects of \\ higher-dimensional\\ varieties};
\draw (357.77000000000004, -30.0) rectangle (362.87000000000006,-32.1);
\draw(362.97, -30.0) node[anchor=north west,align=left] {Computational\\ aspects\\ of algebraic\\ curves};
\draw (362.97, -30.0) rectangle (366.82000000000005,-32.1);
\draw(352.32000000000005, -32.2) node[anchor=north west,align=left] {Effectivity, \\ complexity and\\ computational\\ aspects of \\ algebraic geometry};
\draw (352.32000000000005, -32.2) rectangle (357.4200000000001,-34.800000000000004);
\draw(357.52000000000004, -32.2) node[anchor=north west,align=left] {Computational\\ aspects\\ of algebraic\\ surfaces};
\draw (357.52000000000004, -32.2) rectangle (361.37000000000006,-34.300000000000004);
\draw(361.47, -32.2) node[anchor=north west,align=left] {Computational\\ algebraic\\ geometry over\\ arithmetic\\ ground fields};
\draw (361.47, -32.2) rectangle (365.32000000000005,-34.800000000000004);
\draw(352.32000000000005, -34.9) node[anchor=north west,align=left] {Computational\\ real\\ algebraic\\ geometry};
\draw (352.32000000000005, -34.9) rectangle (356.1700000000001,-37.0);
\draw(351.32000000000005, -37.2) node[anchor=north west,align=left] {\large Real algebraic and real-analytic geometry};
\draw (351.32000000000005, -37.2) rectangle (364.63000000000005,-42.6);
\draw(352.32000000000005, -38.2) node[anchor=north west,align=left] {Semialgebraic\\ sets\\ and related\\ spaces};
\draw (352.32000000000005, -38.2) rectangle (356.1700000000001,-40.300000000000004);
\draw(356.27000000000004, -38.2) node[anchor=north west,align=left] {Real-analytic\\ and\\ semi-analytic\\ sets};
\draw (356.27000000000004, -38.2) rectangle (360.12000000000006,-40.300000000000004);
\draw(360.22, -38.2) node[anchor=north west,align=left] {Nash \\ functions and\\ manifolds};
\draw (360.22, -38.2) rectangle (364.07000000000005,-39.800000000000004);
\draw(352.32000000000005, -40.400000000000006) node[anchor=north west,align=left] {Real \\ algebraic\\ sets};
\draw (352.32000000000005, -40.400000000000006) rectangle (355.1700000000001,-42.00000000000001);
\draw(355.27000000000004, -40.400000000000006) node[anchor=north west,align=left] {Topology\\ of real \\ algebraic\\ varieties};
\draw (355.27000000000004, -40.400000000000006) rectangle (358.12000000000006,-42.50000000000001);
\draw(367.02000000000004, -29.0) node[anchor=north west,align=left] {\large Tropical geometry};
\draw (367.02000000000004, -29.0) rectangle (374.92,-39.3);
\draw(368.02000000000004, -30.0) node[anchor=north west,align=left] {Foundations\\ of tropical\\ geometry\\ and relations\\ with algebra};
\draw (368.02000000000004, -30.0) rectangle (371.87000000000006,-32.6);
\draw(368.02000000000004, -32.7) node[anchor=north west,align=left] {Combinatorial\\ aspects\\ of tropical\\ varieties};
\draw (368.02000000000004, -32.7) rectangle (371.87000000000006,-34.800000000000004);
\draw(368.02000000000004, -34.9) node[anchor=north west,align=left] {Applications\\ of\\ tropical\\ geometry};
\draw (368.02000000000004, -34.9) rectangle (371.62000000000006,-37.0);
\draw(368.02000000000004, -37.1) node[anchor=north west,align=left] {Geometric\\ aspects\\ of tropical\\ varieties};
\draw (368.02000000000004, -37.1) rectangle (371.37000000000006,-39.2);
\draw(371.47, -37.1) node[anchor=north west,align=left] {Arithmetic\\ aspects \\ of tropical\\ varieties};
\draw (371.47, -37.1) rectangle (374.82000000000005,-39.2);
\draw(335.62000000000006, -46.0) node[anchor=north west,align=left] {\large Local theory in algebraic geometry};
\draw (335.62000000000006, -46.0) rectangle (348.4700000000001,-54.1);
\draw(336.62000000000006, -47.0) node[anchor=north west,align=left] {Local deformation\\ theory,\\ Artin \\ approximation, etc.};
\draw (336.62000000000006, -47.0) rectangle (341.9700000000001,-49.1);
\draw(342.07000000000005, -47.0) node[anchor=north west,align=left] {Local structure\\ of morphisms\\ in algebraic\\ geometry:\\ étale, flat, etc.};
\draw (342.07000000000005, -47.0) rectangle (346.9200000000001,-49.6);
\draw(336.62000000000006, -49.7) node[anchor=north west,align=left] {Singularities\\ in\\ algebraic\\ geometry};
\draw (336.62000000000006, -49.7) rectangle (340.4700000000001,-51.800000000000004);
\draw(340.57000000000005, -49.7) node[anchor=north west,align=left] {Deformations\\ of \\ singularities};
\draw (340.57000000000005, -49.7) rectangle (344.4200000000001,-51.300000000000004);
\draw(344.52000000000004, -49.7) node[anchor=north west,align=left] {Infinitesimal\\ methods\\ in algebraic\\ geometry};
\draw (344.52000000000004, -49.7) rectangle (348.37000000000006,-51.800000000000004);
\draw(336.62000000000006, -51.9) node[anchor=north west,align=left] {Local \\ cohomology \\ and algebraic\\ geometry};
\draw (336.62000000000006, -51.9) rectangle (340.4700000000001,-54.0);
\draw(340.57000000000005, -51.9) node[anchor=north west,align=left] {Formal \\ neighborhoods\\ in algebraic\\ geometry};
\draw (340.57000000000005, -51.9) rectangle (344.4200000000001,-54.0);
\draw(348.57000000000005, -46.0) node[anchor=north west,align=left] {\large Foundations of algebraic geometry};
\draw (348.57000000000005, -46.0) rectangle (360.97,-55.6);
\draw(349.57000000000005, -47.0) node[anchor=north west,align=left] {Fundamental constructions\\ in algebraic\\ geometry involving\\ higher and derived\\ categories (homotopical\\ algebraic \\ geometry, derived \\ algebraic geometry, etc.)};
\draw (349.57000000000005, -47.0) rectangle (356.4200000000001,-51.1);
\draw(356.52000000000004, -47.0) node[anchor=north west,align=left] {Generalizations\\ (algebraic \\ spaces, stacks)};
\draw (356.52000000000004, -47.0) rectangle (360.87000000000006,-49.1);
\draw(356.52000000000004, -49.2) node[anchor=north west,align=left] {Relevant\\ commutative\\ algebra};
\draw (356.52000000000004, -49.2) rectangle (359.87000000000006,-50.800000000000004);
\draw(349.57000000000005, -51.2) node[anchor=north west,align=left] {Noncommutative\\ algebraic\\ geometry};
\draw (349.57000000000005, -51.2) rectangle (353.6700000000001,-53.300000000000004);
\draw(353.77000000000004, -51.2) node[anchor=north west,align=left] {Elementary\\ questions\\ in algebraic\\ geometry};
\draw (353.77000000000004, -51.2) rectangle (357.37000000000006,-53.300000000000004);
\draw(357.47, -51.2) node[anchor=north west,align=left] {Logarithmic\\ algebraic\\ geometry,\\ log schemes};
\draw (357.47, -51.2) rectangle (360.82000000000005,-53.300000000000004);
\draw(349.57000000000005, -53.4) node[anchor=north west,align=left] {Geometry \\ over the \\ field with \\ one element};
\draw (349.57000000000005, -53.4) rectangle (352.9200000000001,-55.5);
\draw(353.02000000000004, -53.4) node[anchor=north west,align=left] {Varieties\\ and\\ morphisms};
\draw (353.02000000000004, -53.4) rectangle (355.87000000000006,-55.0);
\draw(355.97, -53.4) node[anchor=north west,align=left] {Schemes\\ and \\ morphisms};
\draw (355.97, -53.4) rectangle (358.82000000000005,-55.0);
\draw(361.07000000000005, -46.0) node[anchor=north west,align=left] {\large Special varieties};
\draw (361.07000000000005, -46.0) rectangle (373.47,-58.5);
\draw(362.07000000000005, -47.0) node[anchor=north west,align=left] {Varieties defined\\ by ring \\ conditions (factorial,\\ Cohen-Macaulay,\\ seminormal)};
\draw (362.07000000000005, -47.0) rectangle (368.1700000000001,-49.6);
\draw(368.27000000000004, -47.0) node[anchor=north west,align=left] {Compactifications;\\ symmetric\\ and spherical\\ varieties};
\draw (368.27000000000004, -47.0) rectangle (373.37000000000006,-49.1);
\draw(362.07000000000005, -49.7) node[anchor=north west,align=left] {Determinantalvarieties};
\draw (362.07000000000005, -49.7) rectangle (368.1700000000001,-51.300000000000004);
\draw(368.27000000000004, -49.7) node[anchor=north west,align=left] {Toric varieties,\\ Newton\\ polyhedra, \\ Okounkov bodies};
\draw (368.27000000000004, -49.7) rectangle (372.87000000000006,-51.800000000000004);
\draw(362.07000000000005, -51.9) node[anchor=north west,align=left] {Low codimension\\ problems\\ in algebraic\\ geometry};
\draw (362.07000000000005, -51.9) rectangle (366.4200000000001,-54.0);
\draw(366.52000000000004, -51.9) node[anchor=north west,align=left] {Homogeneous\\ spaces\\ and \\ generalizations};
\draw (366.52000000000004, -51.9) rectangle (370.87000000000006,-54.0);
\draw(370.97, -51.9) node[anchor=north west,align=left] {Linkage};
\draw (370.97, -51.9) rectangle (373.32000000000005,-53.0);
\draw(362.07000000000005, -54.1) node[anchor=north west,align=left] {Grassmannians,\\ Schubert\\ varieties, \\ flag manifolds};
\draw (362.07000000000005, -54.1) rectangle (366.1700000000001,-56.2);
\draw(366.27000000000004, -54.1) node[anchor=north west,align=left] {Supervarieties};
\draw (366.27000000000004, -54.1) rectangle (370.37000000000006,-55.2);
\draw(370.47, -54.1) node[anchor=north west,align=left] {Character\\ varieties};
\draw (370.47, -54.1) rectangle (373.32000000000005,-55.7);
\draw(362.07000000000005, -56.3) node[anchor=north west,align=left] {Complete\\ intersections};
\draw (362.07000000000005, -56.3) rectangle (365.9200000000001,-57.9);
\draw(366.02000000000004, -56.3) node[anchor=north west,align=left] {Rational\\ and \\ unirational\\ varieties};
\draw (366.02000000000004, -56.3) rectangle (369.37000000000006,-58.4);
\draw(369.47, -56.3) node[anchor=north west,align=left] {Rationally\\ connected\\ varieties};
\draw (369.47, -56.3) rectangle (372.57000000000005,-57.9);
\draw(335.62000000000006, -58.6) node[anchor=north west,align=left] {\large Curves in algebraic geometry};
\draw (335.62000000000006, -58.6) rectangle (347.4700000000001,-76.7);
\draw(336.62000000000006, -59.6) node[anchor=north west,align=left] {Special \\ divisors on \\ curves (gonality,\\ Brill-Noether theory)};
\draw (336.62000000000006, -59.6) rectangle (342.4700000000001,-62.2);
\draw(342.57000000000005, -59.6) node[anchor=north west,align=left] {Theta functions\\ and \\ curves; Schottky\\ problem};
\draw (342.57000000000005, -59.6) rectangle (347.1700000000001,-61.7);
\draw(336.62000000000006, -62.300000000000004) node[anchor=north west,align=left] {Riemann \\ surfaces; Weierstrass\\ points;\\ gap sequences};
\draw (336.62000000000006, -62.300000000000004) rectangle (342.4700000000001,-64.4);
\draw(342.57000000000005, -62.300000000000004) node[anchor=north west,align=left] {Special \\ algebraic curves\\ and curves\\ of low genus};
\draw (342.57000000000005, -62.300000000000004) rectangle (347.1700000000001,-64.4);
\draw(336.62000000000006, -64.5) node[anchor=north west,align=left] {Algebraic \\ functions and\\ function \\ fields in algebraic\\ geometry};
\draw (336.62000000000006, -64.5) rectangle (341.9700000000001,-67.1);
\draw(342.07000000000005, -64.5) node[anchor=north west,align=left] {Relationships\\ between \\ algebraic curves\\ and \\ integrable systems};
\draw (342.07000000000005, -64.5) rectangle (347.1700000000001,-67.1);
\draw(336.62000000000006, -67.2) node[anchor=north west,align=left] {Relationships\\ between \\ algebraic curves\\ and physics};
\draw (336.62000000000006, -67.2) rectangle (341.2200000000001,-69.3);
\draw(341.32000000000005, -67.2) node[anchor=north west,align=left] {Automorphismsof\\ curves};
\draw (341.32000000000005, -67.2) rectangle (345.6700000000001,-68.8);
\draw(336.62000000000006, -69.4) node[anchor=north west,align=left] {Singularities\\ of \\ curves, \\ local rings};
\draw (336.62000000000006, -69.4) rectangle (340.4700000000001,-71.5);
\draw(340.57000000000005, -69.4) node[anchor=north west,align=left] {Vector \\ bundles on \\ curves and \\ their moduli};
\draw (340.57000000000005, -69.4) rectangle (344.1700000000001,-71.5);
\draw(344.27000000000004, -69.4) node[anchor=north west,align=left] {Families,\\ moduli \\ of curves\\ (analytic)};
\draw (344.27000000000004, -69.4) rectangle (347.37000000000006,-71.5);
\draw(336.62000000000006, -71.6) node[anchor=north west,align=left] {Families,\\ moduli \\ of curves\\ (algebraic)};
\draw (336.62000000000006, -71.6) rectangle (339.9700000000001,-73.69999999999999);
\draw(340.07000000000005, -71.6) node[anchor=north west,align=left] {Coverings\\ of curves,\\ fundamental\\ group};
\draw (340.07000000000005, -71.6) rectangle (343.4200000000001,-73.69999999999999);
\draw(343.52000000000004, -71.6) node[anchor=north west,align=left] {Arithmetic\\ ground\\ fields\\ for curves};
\draw (343.52000000000004, -71.6) rectangle (346.62000000000006,-73.69999999999999);
\draw(336.62000000000006, -73.8) node[anchor=north west,align=left] {Jacobians,\\ Prym\\ varieties};
\draw (336.62000000000006, -73.8) rectangle (339.7200000000001,-75.39999999999999);
\draw(339.82000000000005, -73.8) node[anchor=north west,align=left] {Plane \\ and space\\ curves};
\draw (339.82000000000005, -73.8) rectangle (342.6700000000001,-75.39999999999999);
\draw(342.77000000000004, -73.8) node[anchor=north west,align=left] {Dessins\\ d’enfants\\ theory};
\draw (342.77000000000004, -73.8) rectangle (345.62000000000006,-75.39999999999999);
\draw(336.62000000000006, -75.5) node[anchor=north west,align=left] {Elliptic\\ curves};
\draw (336.62000000000006, -75.5) rectangle (339.2200000000001,-76.6);
\draw(347.57000000000005, -58.6) node[anchor=north west,align=left] {\large Cycles and subschemes};
\draw (347.57000000000005, -58.6) rectangle (358.97,-71.1);
\draw(348.57000000000005, -59.6) node[anchor=north west,align=left] {Intersection \\ theory, characteristic\\ classes,\\ intersection \\ multiplicities in\\ algebraic geometry};
\draw (348.57000000000005, -59.6) rectangle (354.6700000000001,-62.7);
\draw(354.77000000000004, -59.6) node[anchor=north west,align=left] {(Equivariant)\\ Chow \\ groups and \\ rings; motives};
\draw (354.77000000000004, -59.6) rectangle (358.87000000000006,-61.7);
\draw(348.57000000000005, -62.800000000000004) node[anchor=north west,align=left] {Transcendental\\ methods,\\ Hodge theory\\ (algebro-geometric\\ aspects)};
\draw (348.57000000000005, -62.800000000000004) rectangle (353.6700000000001,-65.4);
\draw(353.77000000000004, -62.800000000000004) node[anchor=north west,align=left] {Applications \\ of methods of \\ algebraic \\ \(K\)-theory in \\ algebraic geometry};
\draw (353.77000000000004, -62.800000000000004) rectangle (358.87000000000006,-65.4);
\draw(348.57000000000005, -65.5) node[anchor=north west,align=left] {Parametrization\\ (Chow\\ and Hilbert\\ schemes)};
\draw (348.57000000000005, -65.5) rectangle (352.9200000000001,-67.6);
\draw(353.02000000000004, -65.5) node[anchor=north west,align=left] {Divisors, \\ linear systems,\\ invertible\\ sheaves};
\draw (353.02000000000004, -65.5) rectangle (357.37000000000006,-67.6);
\draw(348.57000000000005, -67.7) node[anchor=north west,align=left] {Pencils, \\ nets, webs\\ in algebraic\\ geometry};
\draw (348.57000000000005, -67.7) rectangle (352.1700000000001,-69.8);
\draw(352.27000000000004, -67.7) node[anchor=north west,align=left] {Riemann-Roch\\ theorems};
\draw (352.27000000000004, -67.7) rectangle (355.87000000000006,-69.3);
\draw(355.97, -67.7) node[anchor=north west,align=left] {Algebraic\\ cycles};
\draw (355.97, -67.7) rectangle (358.82000000000005,-68.8);
\draw(348.57000000000005, -69.9) node[anchor=north west,align=left] {Torelli\\ problem};
\draw (348.57000000000005, -69.9) rectangle (350.9200000000001,-71.0);
\draw(351.02000000000004, -69.9) node[anchor=north west,align=left] {Picard\\ groups};
\draw (351.02000000000004, -69.9) rectangle (353.12000000000006,-71.0);
\draw(359.07000000000005, -58.6) node[anchor=north west,align=left] {\large Abelian varieties and schemes};
\draw (359.07000000000005, -58.6) rectangle (370.22,-68.9);
\draw(360.07000000000005, -59.6) node[anchor=north west,align=left] {Analytic theory\\ of abelian \\ varieties; abelian\\ integrals\\ and differentials};
\draw (360.07000000000005, -59.6) rectangle (365.1700000000001,-62.2);
\draw(365.27000000000004, -59.6) node[anchor=north west,align=left] {Algebraic \\ moduli of abelian\\ varieties,\\ classification};
\draw (365.27000000000004, -59.6) rectangle (370.12000000000006,-61.7);
\draw(360.07000000000005, -62.300000000000004) node[anchor=north west,align=left] {Picard \\ schemes, higher\\ Jacobians};
\draw (360.07000000000005, -62.300000000000004) rectangle (364.4200000000001,-63.900000000000006);
\draw(364.52000000000004, -62.300000000000004) node[anchor=north west,align=left] {Complex \\ multiplication\\ and abelian\\ varieties};
\draw (364.52000000000004, -62.300000000000004) rectangle (368.62000000000006,-64.4);
\draw(360.07000000000005, -64.5) node[anchor=north west,align=left] {Arithmetic\\ ground fields\\ for abelian\\ varieties};
\draw (360.07000000000005, -64.5) rectangle (363.9200000000001,-66.6);
\draw(364.02000000000004, -64.5) node[anchor=north west,align=left] {Theta \\ functions and\\ abelian\\ varieties};
\draw (364.02000000000004, -64.5) rectangle (367.87000000000006,-66.6);
\draw(360.07000000000005, -66.7) node[anchor=north west,align=left] {Subvarieties\\ of\\ abelian\\ varieties};
\draw (360.07000000000005, -66.7) rectangle (363.6700000000001,-68.8);
\draw(363.77000000000004, -66.7) node[anchor=north west,align=left] {Algebraic\\ theory \\ of abelian\\ varieties};
\draw (363.77000000000004, -66.7) rectangle (366.87000000000006,-68.8);
\draw(366.97, -66.7) node[anchor=north west,align=left] {Isogeny};
\draw (366.97, -66.7) rectangle (369.32000000000005,-67.8);
\draw(335.62000000000006, -76.80000000000001) node[anchor=north west,align=left] {\large Affine geometry};
\draw (335.62000000000006, -76.80000000000001) rectangle (345.77000000000004,-84.4);
\draw(336.62000000000006, -77.80000000000001) node[anchor=north west,align=left] {Affine spaces\\ (automorphisms,\\ embeddings,\\ exotic \\ structures, \\ cancellation problem)};
\draw (336.62000000000006, -77.80000000000001) rectangle (342.4700000000001,-80.9);
\draw(342.57000000000005, -77.80000000000001) node[anchor=north west,align=left] {Group \\ actions on\\ affine\\ varieties};
\draw (342.57000000000005, -77.80000000000001) rectangle (345.6700000000001,-79.9);
\draw(336.62000000000006, -81.00000000000001) node[anchor=north west,align=left] {Classification\\ of affine\\ varieties};
\draw (336.62000000000006, -81.00000000000001) rectangle (340.7200000000001,-83.10000000000001);
\draw(340.82000000000005, -81.00000000000001) node[anchor=north west,align=left] {Affine\\ fibrations};
\draw (340.82000000000005, -81.00000000000001) rectangle (343.9200000000001,-82.60000000000001);
\draw(336.62000000000006, -83.20000000000002) node[anchor=north west,align=left] {Jacobian\\ problem};
\draw (336.62000000000006, -83.20000000000002) rectangle (339.2200000000001,-84.30000000000001);
\draw(345.87000000000006, -76.80000000000001) node[anchor=north west,align=left] {\large Birational geometry};
\draw (345.87000000000006, -76.80000000000001) rectangle (355.52000000000004,-89.30000000000001);
\draw(346.87000000000006, -77.80000000000001) node[anchor=north west,align=left] {Global theory\\ and resolution\\ of singularities\\ (algebro-geometric\\ aspects)};
\draw (346.87000000000006, -77.80000000000001) rectangle (351.9700000000001,-80.4);
\draw(352.07000000000005, -77.80000000000001) node[anchor=north west,align=left] {Arcs and\\ motivic \\ integration};
\draw (352.07000000000005, -77.80000000000001) rectangle (355.4200000000001,-79.4);
\draw(346.87000000000006, -80.50000000000001) node[anchor=north west,align=left] {Rational\\ and \\ birational maps};
\draw (346.87000000000006, -80.50000000000001) rectangle (351.2200000000001,-82.10000000000001);
\draw(351.32000000000005, -80.50000000000001) node[anchor=north west,align=left] {McKay\\ correspondence};
\draw (351.32000000000005, -80.50000000000001) rectangle (355.4200000000001,-82.10000000000001);
\draw(346.87000000000006, -82.20000000000002) node[anchor=north west,align=left] {Birational\\ automorphisms,\\ Cremona\\ group and\\ generalizations};
\draw (346.87000000000006, -82.20000000000002) rectangle (351.2200000000001,-84.80000000000001);
\draw(351.32000000000005, -82.20000000000002) node[anchor=north west,align=left] {Minimal model\\ program \\ (Mori theory,\\ extremal rays)};
\draw (351.32000000000005, -82.20000000000002) rectangle (355.4200000000001,-84.30000000000001);
\draw(346.87000000000006, -84.9) node[anchor=north west,align=left] {Rationality\\ questions\\ in algebraic\\ geometry};
\draw (346.87000000000006, -84.9) rectangle (350.4700000000001,-87.0);
\draw(350.57000000000005, -84.9) node[anchor=north west,align=left] {Coverings\\ in algebraic\\ geometry};
\draw (350.57000000000005, -84.9) rectangle (354.1700000000001,-86.5);
\draw(346.87000000000006, -87.10000000000001) node[anchor=north west,align=left] {Ramification\\ problems\\ in algebraic\\ geometry};
\draw (346.87000000000006, -87.10000000000001) rectangle (350.4700000000001,-89.2);
\draw(350.57000000000005, -87.10000000000001) node[anchor=north west,align=left] {Embeddings\\ in algebraic\\ geometry};
\draw (350.57000000000005, -87.10000000000001) rectangle (354.1700000000001,-88.7);
\draw(355.62000000000006, -76.80000000000001) node[anchor=north west,align=left] {\large Algebraic groups};
\draw (355.62000000000006, -76.80000000000001) rectangle (365.27000000000004,-86.60000000000001);
\draw(356.62000000000006, -77.80000000000001) node[anchor=north west,align=left] {Classical \\ groups \\ (algebro-geometric\\ aspects)};
\draw (356.62000000000006, -77.80000000000001) rectangle (361.7200000000001,-79.9);
\draw(361.82000000000005, -77.80000000000001) node[anchor=north west,align=left] {Group \\ varieties};
\draw (361.82000000000005, -77.80000000000001) rectangle (364.6700000000001,-78.9);
\draw(356.62000000000006, -80.00000000000001) node[anchor=north west,align=left] {Affine algebraic\\ groups,\\ hyperalgebra\\ constructions};
\draw (356.62000000000006, -80.00000000000001) rectangle (361.2200000000001,-82.10000000000001);
\draw(361.32000000000005, -80.00000000000001) node[anchor=north west,align=left] {Group actions\\ on varieties\\ or schemes\\ (quotients)};
\draw (361.32000000000005, -80.00000000000001) rectangle (365.1700000000001,-82.10000000000001);
\draw(356.62000000000006, -82.20000000000002) node[anchor=north west,align=left] {Other \\ algebraic groups\\ (geometric\\ aspects)};
\draw (356.62000000000006, -82.20000000000002) rectangle (361.2200000000001,-84.30000000000001);
\draw(361.32000000000005, -82.20000000000002) node[anchor=north west,align=left] {Geometric\\ invariant\\ theory};
\draw (361.32000000000005, -82.20000000000002) rectangle (364.1700000000001,-83.80000000000001);
\draw(356.62000000000006, -84.4) node[anchor=north west,align=left] {Formal \\ groups, \\ \(p\)-divisible\\ groups};
\draw (356.62000000000006, -84.4) rectangle (360.9700000000001,-86.5);
\draw(361.07000000000005, -84.4) node[anchor=north west,align=left] {Group\\ schemes};
\draw (361.07000000000005, -84.4) rectangle (363.4200000000001,-85.5);
\draw(334.62000000000006, -89.5) node[anchor=north west,align=left] {\LARGE Commutative algebra};
\draw (334.62000000000006, -89.5) rectangle (374.93000000000006,-138.1);
\draw(335.62000000000006, -90.5) node[anchor=north west,align=left] {\large Chain conditions, finiteness conditions in commutative ring theory};
\draw (335.62000000000006, -90.5) rectangle (356.68000000000006,-94.7);
\draw(336.62000000000006, -91.5) node[anchor=north west,align=left] {Commutative \\ rings and modules\\ of finite \\ generation or \\ presentation; \\ number of generators};
\draw (336.62000000000006, -91.5) rectangle (342.2200000000001,-94.6);
\draw(342.32000000000005, -91.5) node[anchor=north west,align=left] {Commutative \\ Artinian rings\\ and modules,\\ finite-dimensional\\ algebras};
\draw (342.32000000000005, -91.5) rectangle (347.4200000000001,-94.1);
\draw(347.52000000000004, -91.5) node[anchor=north west,align=left] {Commutative\\ Noetherian\\ rings \\ and modules};
\draw (347.52000000000004, -91.5) rectangle (350.87000000000006,-93.6);
\draw(356.7800000000001, -90.5) node[anchor=north west,align=left] {\large Theory of modules and ideals in commutative rings};
\draw (356.7800000000001, -90.5) rectangle (374.8300000000001,-101.8);
\draw(357.7800000000001, -91.5) node[anchor=north west,align=left] {Structure, \\ classification \\ theorems for modules\\ and ideals in\\ commutative rings};
\draw (357.7800000000001, -91.5) rectangle (363.3800000000001,-94.1);
\draw(363.4800000000001, -91.5) node[anchor=north west,align=left] {Dimension \\ theory, depth, \\ related commutative\\ rings \\ (catenary, etc.)};
\draw (363.4800000000001, -91.5) rectangle (368.8300000000001,-94.1);
\draw(368.93000000000006, -91.5) node[anchor=north west,align=left] {Projective\\ and free \\ modules and \\ ideals in \\ commutative rings};
\draw (368.93000000000006, -91.5) rectangle (373.7800000000001,-94.1);
\draw(357.7800000000001, -94.2) node[anchor=north west,align=left] {Injective \\ and flat \\ modules and \\ ideals in \\ commutative rings};
\draw (357.7800000000001, -94.2) rectangle (362.6300000000001,-96.8);
\draw(362.7300000000001, -94.2) node[anchor=north west,align=left] {Torsion \\ modules and \\ ideals in \\ commutative rings};
\draw (362.7300000000001, -94.2) rectangle (367.5800000000001,-96.3);
\draw(367.68000000000006, -94.2) node[anchor=north west,align=left] {Other special\\ types of \\ modules and \\ ideals in \\ commutative rings};
\draw (367.68000000000006, -94.2) rectangle (372.5300000000001,-96.8);
\draw(372.6300000000001, -94.2) node[anchor=north west,align=left] {Class\\ groups};
\draw (372.6300000000001, -94.2) rectangle (374.73000000000013,-95.3);
\draw(357.7800000000001, -96.9) node[anchor=north west,align=left] {Linkage, \\ complete \\ intersections \\ and determinantal\\ ideals};
\draw (357.7800000000001, -96.9) rectangle (362.6300000000001,-99.5);
\draw(362.7300000000001, -96.9) node[anchor=north west,align=left] {Module \\ categories\\ and \\ commutative rings};
\draw (362.7300000000001, -96.9) rectangle (367.5800000000001,-99.0);
\draw(367.68000000000006, -96.9) node[anchor=north west,align=left] {Theory of modules\\ and ideals\\ in commutative\\ rings described\\ by combinatorial\\ properties};
\draw (367.68000000000006, -96.9) rectangle (372.5300000000001,-100.0);
\draw(357.7800000000001, -100.1) node[anchor=north west,align=left] {Cohen-Macaulay\\ modules};
\draw (357.7800000000001, -100.1) rectangle (361.8800000000001,-101.69999999999999);
\draw(335.62000000000006, -94.8) node[anchor=north west,align=left] {\large Computational aspects and applications of commutative rings};
\draw (335.62000000000006, -94.8) rectangle (356.1700000000001,-101.2);
\draw(336.62000000000006, -95.8) node[anchor=north west,align=left] {Applications of\\ commutative \\ algebra (e.g., to\\ statistics, \\ control theory, \\ optimization, etc.)};
\draw (336.62000000000006, -95.8) rectangle (341.9700000000001,-98.89999999999999);
\draw(342.07000000000005, -95.8) node[anchor=north west,align=left] {Gröbner bases; \\ other bases for \\ ideals and modules\\ (e.g., Janet \\ and border bases)};
\draw (342.07000000000005, -95.8) rectangle (347.1700000000001,-98.39999999999999);
\draw(347.27000000000004, -95.8) node[anchor=north west,align=left] {Polynomials,\\ factorization\\ in \\ commutative rings};
\draw (347.27000000000004, -95.8) rectangle (352.12000000000006,-97.89999999999999);
\draw(352.2200000000001, -95.8) node[anchor=north west,align=left] {Computational\\ homological\\ algebra};
\draw (352.2200000000001, -95.8) rectangle (356.0700000000001,-97.89999999999999);
\draw(336.62000000000006, -99.0) node[anchor=north west,align=left] {Solving \\ polynomial\\ systems;\\ resultants};
\draw (336.62000000000006, -99.0) rectangle (339.7200000000001,-101.1);
\draw(335.62000000000006, -101.9) node[anchor=north west,align=left] {\large Arithmetic rings and other special commutative rings};
\draw (335.62000000000006, -101.9) rectangle (355.4200000000001,-112.2);
\draw(336.62000000000006, -102.9) node[anchor=north west,align=left] {Commutative rings\\ defined by \\ monomial ideals;\\ Stanley-Reisner\\ face rings; \\ simplicial complexes};
\draw (336.62000000000006, -102.9) rectangle (342.2200000000001,-106.0);
\draw(342.32000000000005, -102.9) node[anchor=north west,align=left] {Commutative rings\\ defined by \\ factorization \\ properties (e.g.,\\ atomic, factorial,\\ half-factorial)};
\draw (342.32000000000005, -102.9) rectangle (347.4200000000001,-106.0);
\draw(347.52000000000004, -102.9) node[anchor=north west,align=left] {Polynomial \\ rings and \\ ideals; rings \\ of integer-valued\\ polynomials};
\draw (347.52000000000004, -102.9) rectangle (352.37000000000006,-105.5);
\draw(352.4700000000001, -102.9) node[anchor=north west,align=left] {Principal\\ ideal\\ rings};
\draw (352.4700000000001, -102.9) rectangle (355.3200000000001,-104.5);
\draw(336.62000000000006, -106.10000000000001) node[anchor=north west,align=left] {Commutative\\ rings defined\\ by binomial\\ ideals, \\ toric rings, etc.};
\draw (336.62000000000006, -106.10000000000001) rectangle (341.4700000000001,-108.7);
\draw(341.57000000000005, -106.10000000000001) node[anchor=north west,align=left] {Other \\ commutative rings\\ defined \\ by combinatorial\\ properties};
\draw (341.57000000000005, -106.10000000000001) rectangle (346.4200000000001,-108.7);
\draw(346.52000000000004, -106.10000000000001) node[anchor=north west,align=left] {Dedekind, \\ Prüfer, Krull\\ and Mori rings\\ and their\\ generalizations};
\draw (346.52000000000004, -106.10000000000001) rectangle (350.87000000000006,-108.7);
\draw(350.9700000000001, -106.10000000000001) node[anchor=north west,align=left] {Euclidean\\ rings\\ and \\ generalizations};
\draw (350.9700000000001, -106.10000000000001) rectangle (355.3200000000001,-108.2);
\draw(336.62000000000006, -108.80000000000001) node[anchor=north west,align=left] {Rings with\\ straightening\\ laws, \\ Hodge algebras};
\draw (336.62000000000006, -108.80000000000001) rectangle (340.7200000000001,-110.9);
\draw(340.82000000000005, -108.80000000000001) node[anchor=north west,align=left] {Witt vectors\\ and \\ related rings};
\draw (340.82000000000005, -108.80000000000001) rectangle (344.6700000000001,-110.4);
\draw(344.77000000000004, -108.80000000000001) node[anchor=north west,align=left] {Formal \\ power \\ series rings};
\draw (344.77000000000004, -108.80000000000001) rectangle (348.37000000000006,-110.4);
\draw(348.4700000000001, -108.80000000000001) node[anchor=north west,align=left] {Seminormal\\ rings};
\draw (348.4700000000001, -108.80000000000001) rectangle (351.5700000000001,-109.9);
\draw(351.6700000000001, -108.80000000000001) node[anchor=north west,align=left] {Valuation\\ rings};
\draw (351.6700000000001, -108.80000000000001) rectangle (354.5200000000001,-109.9);
\draw(336.62000000000006, -111.0) node[anchor=north west,align=left] {Excellent\\ rings};
\draw (336.62000000000006, -111.0) rectangle (339.4700000000001,-112.1);
\draw(339.57000000000005, -111.0) node[anchor=north west,align=left] {Cluster\\ algebras};
\draw (339.57000000000005, -111.0) rectangle (342.1700000000001,-112.1);
\draw(355.5200000000001, -101.9) node[anchor=north west,align=left] {\large Commutative ring extensions and related topics};
\draw (355.5200000000001, -101.9) rectangle (373.0700000000001,-111.0);
\draw(356.5200000000001, -102.9) node[anchor=north west,align=left] {Integral \\ closure of \\ commutative rings\\ and ideals};
\draw (356.5200000000001, -102.9) rectangle (361.3700000000001,-105.0);
\draw(361.4700000000001, -102.9) node[anchor=north west,align=left] {Rings of \\ fractions and\\ localization\\ for \\ commutative rings};
\draw (361.4700000000001, -102.9) rectangle (366.3200000000001,-105.5);
\draw(366.4200000000001, -102.9) node[anchor=north west,align=left] {Galois theory\\ and \\ commutative ring\\ extensions};
\draw (366.4200000000001, -102.9) rectangle (371.0200000000001,-105.0);
\draw(356.5200000000001, -105.60000000000001) node[anchor=north west,align=left] {Étale and \\ flat extensions;\\ Henselization;\\ Artin\\ approximation};
\draw (356.5200000000001, -105.60000000000001) rectangle (361.1200000000001,-108.2);
\draw(361.2200000000001, -105.60000000000001) node[anchor=north west,align=left] {Extension\\ theory \\ of commutative\\ rings};
\draw (361.2200000000001, -105.60000000000001) rectangle (365.3200000000001,-107.7);
\draw(365.4200000000001, -105.60000000000001) node[anchor=north west,align=left] {Morphisms\\ of commutative\\ rings};
\draw (365.4200000000001, -105.60000000000001) rectangle (369.5200000000001,-107.2);
\draw(369.6200000000001, -105.60000000000001) node[anchor=north west,align=left] {Polynomials\\ over\\ commutative\\ rings};
\draw (369.6200000000001, -105.60000000000001) rectangle (372.97000000000014,-107.7);
\draw(356.5200000000001, -108.30000000000001) node[anchor=north west,align=left] {Integral \\ dependence in \\ commutative \\ rings; going\\ up, going down};
\draw (356.5200000000001, -108.30000000000001) rectangle (360.6200000000001,-110.9);
\draw(360.7200000000001, -108.30000000000001) node[anchor=north west,align=left] {Completion\\ of commutative\\ rings};
\draw (360.7200000000001, -108.30000000000001) rectangle (364.8200000000001,-109.9);
\draw(355.5200000000001, -111.1) node[anchor=north west,align=left] {\large History of \\ commutative algebra};
\draw (355.5200000000001, -111.1) rectangle (362.0100000000001,-112.19999999999999);
\draw(335.62000000000006, -112.3) node[anchor=north west,align=left] {\large Homological methods in commutative ring theory};
\draw (335.62000000000006, -112.3) rectangle (352.7200000000001,-124.6);
\draw(336.62000000000006, -113.3) node[anchor=north west,align=left] {Homological \\ conjectures \\ (intersection \\ theorems) in \\ commutative ring theory};
\draw (336.62000000000006, -113.3) rectangle (342.9700000000001,-115.89999999999999);
\draw(343.07000000000005, -113.3) node[anchor=north west,align=left] {(Co)homology of \\ commutative rings\\ and algebras (e.g.,\\ Hochschild, \\ André-Quillen, cyclic,\\ dihedral, etc.)};
\draw (343.07000000000005, -113.3) rectangle (349.1700000000001,-116.39999999999999);
\draw(349.27000000000004, -113.3) node[anchor=north west,align=left] {Torsion \\ theory for\\ commutative\\ rings};
\draw (349.27000000000004, -113.3) rectangle (352.62000000000006,-115.39999999999999);
\draw(336.62000000000006, -116.5) node[anchor=north west,align=left] {Syzygies,\\ resolutions,\\ complexes\\ and \\ commutative rings};
\draw (336.62000000000006, -116.5) rectangle (341.4700000000001,-119.1);
\draw(341.57000000000005, -116.5) node[anchor=north west,align=left] {Homological\\ dimension\\ and \\ commutative rings};
\draw (341.57000000000005, -116.5) rectangle (346.4200000000001,-118.6);
\draw(346.52000000000004, -116.5) node[anchor=north west,align=left] {Homological \\ functors on \\ modules of \\ commutative rings\\ (Tor, Ext, etc.)};
\draw (346.52000000000004, -116.5) rectangle (351.37000000000006,-119.1);
\draw(336.62000000000006, -119.2) node[anchor=north west,align=left] {Deformations\\ and infinitesimal\\ methods\\ in commutative\\ ring theory};
\draw (336.62000000000006, -119.2) rectangle (341.4700000000001,-121.8);
\draw(341.57000000000005, -119.2) node[anchor=north west,align=left] {Derived \\ categories \\ and commutative\\ rings};
\draw (341.57000000000005, -119.2) rectangle (345.9200000000001,-121.3);
\draw(346.02000000000004, -119.2) node[anchor=north west,align=left] {Grothendieck\\ groups, \\ \(K\)-theory\\ and commutative\\ rings};
\draw (346.02000000000004, -119.2) rectangle (350.37000000000006,-121.8);
\draw(336.62000000000006, -121.9) node[anchor=north west,align=left] {Hilbert-Samuel\\ and \\ Hilbert-Kunz \\ functions; \\ Poincaré series};
\draw (336.62000000000006, -121.9) rectangle (340.9700000000001,-124.5);
\draw(341.07000000000005, -121.9) node[anchor=north west,align=left] {Local \\ cohomology \\ and commutative\\ rings};
\draw (341.07000000000005, -121.9) rectangle (345.4200000000001,-124.0);
\draw(352.82000000000005, -112.3) node[anchor=north west,align=left] {\large Applications of logic to commutative algebra};
\draw (352.82000000000005, -112.3) rectangle (367.06000000000006,-115.5);
\draw(353.82000000000005, -113.3) node[anchor=north west,align=left] {Applications\\ of logic\\ to commutative\\ algebra};
\draw (353.82000000000005, -113.3) rectangle (357.9200000000001,-115.39999999999999);
\draw(352.82000000000005, -115.6) node[anchor=north west,align=left] {\large Topological rings and modules};
\draw (352.82000000000005, -115.6) rectangle (363.87000000000006,-120.0);
\draw(353.82000000000005, -116.6) node[anchor=north west,align=left] {Global \\ topological\\ rings};
\draw (353.82000000000005, -116.6) rectangle (357.1700000000001,-118.19999999999999);
\draw(357.27000000000004, -116.6) node[anchor=north west,align=left] {Analytical\\ algebras\\ and rings};
\draw (357.27000000000004, -116.6) rectangle (360.37000000000006,-118.19999999999999);
\draw(360.47, -116.6) node[anchor=north west,align=left] {Complete\\ rings,\\ completion};
\draw (360.47, -116.6) rectangle (363.57000000000005,-118.19999999999999);
\draw(353.82000000000005, -118.3) node[anchor=north west,align=left] {Henselian\\ rings};
\draw (353.82000000000005, -118.3) rectangle (356.6700000000001,-119.39999999999999);
\draw(356.77000000000004, -118.3) node[anchor=north west,align=left] {Ordered\\ rings};
\draw (356.77000000000004, -118.3) rectangle (359.12000000000006,-119.39999999999999);
\draw(359.22, -118.3) node[anchor=north west,align=left] {Real \\ algebra};
\draw (359.22, -118.3) rectangle (361.57000000000005,-119.39999999999999);
\draw(361.6700000000001, -118.3) node[anchor=north west,align=left] {Power\\ series\\ rings};
\draw (361.6700000000001, -118.3) rectangle (363.7700000000001,-119.89999999999999);
\draw(352.82000000000005, -120.1) node[anchor=north west,align=left] {\large Finite commutative rings};
\draw (352.82000000000005, -120.1) rectangle (360.86000000000007,-123.3);
\draw(353.82000000000005, -121.1) node[anchor=north west,align=left] {Structure\\ of finite\\ commutative\\ rings};
\draw (353.82000000000005, -121.1) rectangle (357.1700000000001,-123.19999999999999);
\draw(357.27000000000004, -121.1) node[anchor=north west,align=left] {Polynomials\\ and finite\\ commutative\\ rings};
\draw (357.27000000000004, -121.1) rectangle (360.62000000000006,-123.19999999999999);
\draw(367.1600000000001, -112.3) node[anchor=north west,align=left] {\large Differential algebra};
\draw (367.1600000000001, -112.3) rectangle (373.9600000000001,-119.89999999999999);
\draw(368.1600000000001, -113.3) node[anchor=north west,align=left] {Modules\\ of \\ differentials};
\draw (368.1600000000001, -113.3) rectangle (372.0100000000001,-114.89999999999999);
\draw(368.1600000000001, -115.0) node[anchor=north west,align=left] {Commutative\\ rings of \\ differential \\ operators and\\ their modules};
\draw (368.1600000000001, -115.0) rectangle (372.0100000000001,-117.6);
\draw(368.1600000000001, -117.7) node[anchor=north west,align=left] {Derivations\\ and\\ commutative\\ rings};
\draw (368.1600000000001, -117.7) rectangle (371.5100000000001,-119.8);
\draw(352.82000000000005, -123.4) node[anchor=north west,align=left] {\large Integral domains};
\draw (352.82000000000005, -123.4) rectangle (358.38000000000005,-125.60000000000001);
\draw(353.82000000000005, -124.4) node[anchor=north west,align=left] {Integral\\ domains};
\draw (353.82000000000005, -124.4) rectangle (356.4200000000001,-125.5);
\draw(335.62000000000006, -124.7) node[anchor=north west,align=left] {\large General commutative ring theory};
\draw (335.62000000000006, -124.7) rectangle (348.27000000000004,-138.0);
\draw(336.62000000000006, -125.7) node[anchor=north west,align=left] {Characteristic \\ \(p\) methods \\ (Frobenius endomorphism)\\ and reduction\\ to characteristic\\ \(p\); tight closure};
\draw (336.62000000000006, -125.7) rectangle (343.2200000000001,-128.8);
\draw(343.32000000000005, -125.7) node[anchor=north west,align=left] {Ideals and\\ multiplicative\\ ideal \\ theory in \\ commutative rings};
\draw (343.32000000000005, -125.7) rectangle (348.1700000000001,-128.3);
\draw(336.62000000000006, -128.9) node[anchor=north west,align=left] {General commutative\\ ring theory and\\ combinatorics \\ (zero-divisor graphs,\\ annihilating-ideal\\ graphs, etc.)};
\draw (336.62000000000006, -128.9) rectangle (342.4700000000001,-132.0);
\draw(342.57000000000005, -128.9) node[anchor=north west,align=left] {Divisibility\\ and factorizations\\ in \\ commutative rings};
\draw (342.57000000000005, -128.9) rectangle (347.6700000000001,-131.0);
\draw(336.62000000000006, -132.1) node[anchor=north west,align=left] {Associated graded\\ rings of \\ ideals (Rees ring,\\ form ring), \\ analytic spread \\ and related topics};
\draw (336.62000000000006, -132.1) rectangle (341.7200000000001,-135.2);
\draw(341.82000000000005, -132.1) node[anchor=north west,align=left] {Valuations\\ and their \\ generalizations\\ for \\ commutative rings};
\draw (341.82000000000005, -132.1) rectangle (346.6700000000001,-134.7);
\draw(336.62000000000006, -135.3) node[anchor=north west,align=left] {Actions of\\ groups on \\ commutative\\ rings; \\ invariant theory};
\draw (336.62000000000006, -135.3) rectangle (341.2200000000001,-137.9);
\draw(341.32000000000005, -135.3) node[anchor=north west,align=left] {Graded\\ rings};
\draw (341.32000000000005, -135.3) rectangle (343.4200000000001,-136.4);
\draw(348.37000000000006, -124.7) node[anchor=north west,align=left] {\large Local rings and semilocal rings};
\draw (348.37000000000006, -124.7) rectangle (358.58000000000004,-130.1);
\draw(349.37000000000006, -125.7) node[anchor=north west,align=left] {Special types\\ (Cohen-Macaulay, \\ Gorenstein, \\ Buchsbaum, etc.)};
\draw (349.37000000000006, -125.7) rectangle (354.2200000000001,-128.3);
\draw(354.32000000000005, -125.7) node[anchor=north west,align=left] {Multiplicity\\ theory\\ and \\ related topics};
\draw (354.32000000000005, -125.7) rectangle (358.4200000000001,-127.8);
\draw(349.37000000000006, -128.4) node[anchor=north west,align=left] {Regular\\ local\\ rings};
\draw (349.37000000000006, -128.4) rectangle (351.7200000000001,-130.0);
\draw(375.3200000000001, -1) node[anchor=north west,align=left] {\LARGE Real functions};
\draw (375.3200000000001, -1) rectangle (406.7700000000001,-39.300000000000004);
\draw(376.3200000000001, -2) node[anchor=north west,align=left] {\large Miscellaneous topics in real functions};
\draw (376.3200000000001, -2) rectangle (392.7200000000001,-11.8);
\draw(377.3200000000001, -3) node[anchor=north west,align=left] {\(C^\infty\)-functions,quasi-analytic\\ functions};
\draw (377.3200000000001, -3) rectangle (387.17000000000013,-5.1);
\draw(387.2700000000001, -3) node[anchor=north west,align=left] {Nonstandardanalysis};
\draw (387.2700000000001, -3) rectangle (392.6200000000001,-4.6);
\draw(377.3200000000001, -5.2) node[anchor=north west,align=left] {Non-Archimedeananalysis};
\draw (377.3200000000001, -5.2) rectangle (383.67000000000013,-6.800000000000001);
\draw(383.7700000000001, -5.2) node[anchor=north west,align=left] {Real-analyticfunctions};
\draw (383.7700000000001, -5.2) rectangle (389.8700000000001,-6.800000000000001);
\draw(389.9700000000001, -5.2) node[anchor=north west,align=left] {Fuzzy\\ real \\ analysis};
\draw (389.9700000000001, -5.2) rectangle (392.5700000000001,-6.800000000000001);
\draw(377.3200000000001, -6.9) node[anchor=north west,align=left] {Calculus of \\ functions on \\ infinite-dimensional\\ spaces};
\draw (377.3200000000001, -6.9) rectangle (382.92000000000013,-9.0);
\draw(383.0200000000001, -6.9) node[anchor=north west,align=left] {Calculus of \\ functions taking\\ values in\\ infinite-dimensional\\ spaces};
\draw (383.0200000000001, -6.9) rectangle (388.6200000000001,-9.5);
\draw(388.7200000000001, -6.9) node[anchor=north west,align=left] {Constructive\\ real\\ analysis};
\draw (388.7200000000001, -6.9) rectangle (392.3200000000001,-8.5);
\draw(388.7200000000001, -8.6) node[anchor=north west,align=left] {Means};
\draw (388.7200000000001, -8.6) rectangle (390.5700000000001,-9.2);
\draw(377.3200000000001, -9.600000000000001) node[anchor=north west,align=left] {Real analysis\\ on time\\ scales or \\ measure chains};
\draw (377.3200000000001, -9.600000000000001) rectangle (381.42000000000013,-11.700000000000001);
\draw(381.5200000000001, -9.600000000000001) node[anchor=north west,align=left] {Set-valued\\ functions};
\draw (381.5200000000001, -9.600000000000001) rectangle (384.6200000000001,-11.200000000000001);
\draw(392.8200000000001, -2) node[anchor=north west,align=left] {\large Functions of one variable};
\draw (392.8200000000001, -2) rectangle (406.67000000000013,-24.6);
\draw(393.8200000000001, -3) node[anchor=north west,align=left] {Continuity and \\ related questions \\ (modulus of continuity,\\ semicontinuity,\\ discontinuities,\\ etc.) for real \\ functions in one variable};
\draw (393.8200000000001, -3) rectangle (400.67000000000013,-6.6);
\draw(400.7700000000001, -3) node[anchor=north west,align=left] {Foundations: \\ limits and \\ generalizations, \\ elementary \\ topology of the line};
\draw (400.7700000000001, -3) rectangle (406.3700000000001,-5.6);
\draw(393.8200000000001, -6.7) node[anchor=north west,align=left] {Nondifferentiability\\ (nondifferentiable\\ functions, \\ points of \\ nondifferentiability), \\ discontinuous derivatives};
\draw (393.8200000000001, -6.7) rectangle (400.67000000000013,-9.8);
\draw(400.7700000000001, -6.7) node[anchor=north west,align=left] {Singular functions,\\ Cantor \\ functions, functions\\ with other \\ special properties};
\draw (400.7700000000001, -6.7) rectangle (406.3700000000001,-9.3);
\draw(393.8200000000001, -9.9) node[anchor=north west,align=left] {Differentiation \\ (real functions of \\ one variable): general\\ theory, generalized\\ derivatives,\\ mean value theorems};
\draw (393.8200000000001, -9.9) rectangle (399.92000000000013,-13.0);
\draw(400.0200000000001, -9.9) node[anchor=north west,align=left] {Antidifferentiation};
\draw (400.0200000000001, -9.9) rectangle (405.3700000000001,-11.5);
\draw(393.8200000000001, -13.100000000000001) node[anchor=north west,align=left] {Classification\\ of real functions;\\ Baire \\ classification of \\ sets and functions};
\draw (393.8200000000001, -13.100000000000001) rectangle (398.92000000000013,-15.700000000000001);
\draw(399.0200000000001, -13.100000000000001) node[anchor=north west,align=left] {Rate of growth\\ of functions,\\ orders of \\ infinity, slowly\\ varying functions};
\draw (399.0200000000001, -13.100000000000001) rectangle (403.8700000000001,-15.700000000000001);
\draw(393.8200000000001, -15.8) node[anchor=north west,align=left] {Denjoy and \\ Perron integrals,\\ other special\\ integrals};
\draw (393.8200000000001, -15.8) rectangle (398.67000000000013,-17.900000000000002);
\draw(398.7700000000001, -15.8) node[anchor=north west,align=left] {Functions \\ of bounded \\ variation, \\ generalizations};
\draw (398.7700000000001, -15.8) rectangle (403.1200000000001,-17.900000000000002);
\draw(403.2200000000001, -15.8) node[anchor=north west,align=left] {Fractional\\ derivatives\\ and\\ integrals};
\draw (403.2200000000001, -15.8) rectangle (406.5700000000001,-17.900000000000002);
\draw(393.8200000000001, -18.0) node[anchor=north west,align=left] {Absolutely \\ continuous real\\ functions in\\ one variable};
\draw (393.8200000000001, -18.0) rectangle (398.17000000000013,-20.1);
\draw(398.2700000000001, -18.0) node[anchor=north west,align=left] {Monotonic\\ functions,\\ generalizations};
\draw (398.2700000000001, -18.0) rectangle (402.6200000000001,-20.1);
\draw(402.7200000000001, -18.0) node[anchor=north west,align=left] {Integrals of\\ Riemann, \\ Stieltjes and\\ Lebesgue type};
\draw (402.7200000000001, -18.0) rectangle (406.5700000000001,-20.1);
\draw(393.8200000000001, -20.2) node[anchor=north west,align=left] {Convexity of\\ real functions\\ in one \\ variable, \\ generalizations};
\draw (393.8200000000001, -20.2) rectangle (398.17000000000013,-22.8);
\draw(398.2700000000001, -20.2) node[anchor=north west,align=left] {One-variable\\ calculus};
\draw (398.2700000000001, -20.2) rectangle (401.8700000000001,-21.8);
\draw(401.9700000000001, -20.2) node[anchor=north west,align=left] {Iteration\\ of real \\ functions in\\ one variable};
\draw (401.9700000000001, -20.2) rectangle (405.5700000000001,-22.3);
\draw(393.8200000000001, -22.9) node[anchor=north west,align=left] {Elementary\\ functions};
\draw (393.8200000000001, -22.9) rectangle (396.92000000000013,-24.5);
\draw(397.0200000000001, -22.9) node[anchor=north west,align=left] {Lipschitz\\ (Hölder)\\ classes};
\draw (397.0200000000001, -22.9) rectangle (399.8700000000001,-24.5);
\draw(376.3200000000001, -11.9) node[anchor=north west,align=left] {\large Polynomials, rational functions in real analysis};
\draw (376.3200000000001, -11.9) rectangle (391.8000000000001,-15.100000000000001);
\draw(377.3200000000001, -12.9) node[anchor=north west,align=left] {Real \\ polynomials: \\ analytic \\ properties, etc.};
\draw (377.3200000000001, -12.9) rectangle (381.92000000000013,-15.0);
\draw(382.0200000000001, -12.9) node[anchor=north west,align=left] {Real \\ polynomials:\\ location\\ of zeros};
\draw (382.0200000000001, -12.9) rectangle (385.6200000000001,-15.0);
\draw(385.7200000000001, -12.9) node[anchor=north west,align=left] {Real \\ rational\\ functions};
\draw (385.7200000000001, -12.9) rectangle (388.5700000000001,-14.5);
\draw(376.3200000000001, -15.200000000000001) node[anchor=north west,align=left] {\large Inequalities in real analysis};
\draw (376.3200000000001, -15.200000000000001) rectangle (386.7200000000001,-22.8);
\draw(377.3200000000001, -16.200000000000003) node[anchor=north west,align=left] {Inequalities \\ involving \\ derivatives and \\ differential and\\ integral operators};
\draw (377.3200000000001, -16.200000000000003) rectangle (382.42000000000013,-18.800000000000004);
\draw(382.5200000000001, -16.200000000000003) node[anchor=north west,align=left] {Inequalities\\ for \\ trigonometric \\ functions and\\ polynomials};
\draw (382.5200000000001, -16.200000000000003) rectangle (386.6200000000001,-18.800000000000004);
\draw(377.3200000000001, -18.900000000000002) node[anchor=north west,align=left] {Inequalities\\ for sums,\\ series\\ and integrals};
\draw (377.3200000000001, -18.900000000000002) rectangle (381.17000000000013,-21.000000000000004);
\draw(381.2700000000001, -18.900000000000002) node[anchor=north west,align=left] {Inequalities\\ involving\\ other types\\ of functions};
\draw (381.2700000000001, -18.900000000000002) rectangle (384.8700000000001,-21.000000000000004);
\draw(377.3200000000001, -21.1) node[anchor=north west,align=left] {Other \\ analytical \\ inequalities};
\draw (377.3200000000001, -21.1) rectangle (380.92000000000013,-22.700000000000003);
\draw(376.3200000000001, -22.900000000000002) node[anchor=north west,align=left] {\large History of \\ real functions};
\draw (376.3200000000001, -22.900000000000002) rectangle (381.2600000000001,-24.000000000000004);
\draw(376.3200000000001, -24.700000000000003) node[anchor=north west,align=left] {\large Functions of several variables};
\draw (376.3200000000001, -24.700000000000003) rectangle (387.9700000000001,-39.2);
\draw(377.3200000000001, -25.700000000000003) node[anchor=north west,align=left] {Implicit function\\ theorems, \\ Jacobians, \\ transformations with\\ several variables};
\draw (377.3200000000001, -25.700000000000003) rectangle (382.92000000000013,-28.300000000000004);
\draw(383.0200000000001, -25.700000000000003) node[anchor=north west,align=left] {Integral formulas\\ of real \\ functions of \\ several variables\\ (Stokes, Gauss,\\ Green, etc.)};
\draw (383.0200000000001, -25.700000000000003) rectangle (387.8700000000001,-28.800000000000004);
\draw(377.3200000000001, -28.900000000000002) node[anchor=north west,align=left] {Absolutely \\ continuous real \\ functions of \\ several variables,\\ functions \\ of bounded variation};
\draw (377.3200000000001, -28.900000000000002) rectangle (382.92000000000013,-32.0);
\draw(383.0200000000001, -28.900000000000002) node[anchor=north west,align=left] {Continuity\\ and \\ differentiation\\ questions};
\draw (383.0200000000001, -28.900000000000002) rectangle (387.3700000000001,-31.000000000000004);
\draw(377.3200000000001, -32.1) node[anchor=north west,align=left] {Integration of\\ real functions\\ of several \\ variables: length,\\ area, volume};
\draw (377.3200000000001, -32.1) rectangle (382.42000000000013,-34.7);
\draw(382.5200000000001, -32.1) node[anchor=north west,align=left] {Special properties\\ of functions\\ of several \\ variables, Hölder\\ conditions, etc.};
\draw (382.5200000000001, -32.1) rectangle (387.6200000000001,-34.7);
\draw(377.3200000000001, -34.800000000000004) node[anchor=north west,align=left] {Convexity of\\ real functions\\ of several\\ variables, \\ generalizations};
\draw (377.3200000000001, -34.800000000000004) rectangle (381.67000000000013,-37.400000000000006);
\draw(381.7700000000001, -34.800000000000004) node[anchor=north west,align=left] {Representation\\ and \\ superposition\\ of functions};
\draw (381.7700000000001, -34.800000000000004) rectangle (385.8700000000001,-36.900000000000006);
\draw(377.3200000000001, -37.5) node[anchor=north west,align=left] {Calculus\\ of vector\\ functions};
\draw (377.3200000000001, -37.5) rectangle (380.17000000000013,-39.1);
\draw(388.0700000000001, -24.700000000000003) node[anchor=north west,align=left] {\large Computational methods\\ for problems pertaining\\ to real functions};
\draw (388.0700000000001, -24.700000000000003) rectangle (395.8000000000001,-26.300000000000004);
\draw(375.3200000000001, -39.400000000000006) node[anchor=north west,align=left] {\LARGE Topological groups, Lie groups};
\draw (375.3200000000001, -39.400000000000006) rectangle (406.3000000000001,-76.0);
\draw(376.3200000000001, -40.400000000000006) node[anchor=north west,align=left] {\large Topological and differentiable algebraic systems};
\draw (376.3200000000001, -40.400000000000006) rectangle (392.17000000000013,-49.00000000000001);
\draw(377.3200000000001, -41.400000000000006) node[anchor=north west,align=left] {Topological \\ groupoids \\ (including \\ differentiable and\\ Lie groupoids)};
\draw (377.3200000000001, -41.400000000000006) rectangle (382.42000000000013,-44.00000000000001);
\draw(382.5200000000001, -41.400000000000006) node[anchor=north west,align=left] {Other topological\\ algebraic\\ systems\\ and their \\ representations};
\draw (382.5200000000001, -41.400000000000006) rectangle (387.3700000000001,-44.00000000000001);
\draw(387.4700000000001, -41.400000000000006) node[anchor=north west,align=left] {Topological\\ semilattices,\\ lattices \\ and applications};
\draw (387.4700000000001, -41.400000000000006) rectangle (392.0700000000001,-43.50000000000001);
\draw(377.3200000000001, -44.10000000000001) node[anchor=north west,align=left] {Representations\\ of general\\ topological\\ groups and\\ semigroups};
\draw (377.3200000000001, -44.10000000000001) rectangle (381.67000000000013,-46.70000000000001);
\draw(381.7700000000001, -44.10000000000001) node[anchor=north west,align=left] {Structure\\ of general\\ topological\\ groups};
\draw (381.7700000000001, -44.10000000000001) rectangle (385.1200000000001,-46.20000000000001);
\draw(385.2200000000001, -44.10000000000001) node[anchor=north west,align=left] {Analysis\\ on general\\ topological\\ groups};
\draw (385.2200000000001, -44.10000000000001) rectangle (388.5700000000001,-46.20000000000001);
\draw(388.67000000000013, -44.10000000000001) node[anchor=north west,align=left] {Structure\\ of \\ topological\\ semigroups};
\draw (388.67000000000013, -44.10000000000001) rectangle (392.02000000000015,-46.20000000000001);
\draw(377.3200000000001, -46.800000000000004) node[anchor=north west,align=left] {Analysis\\ on \\ topological\\ semigroups};
\draw (377.3200000000001, -46.800000000000004) rectangle (380.67000000000013,-48.900000000000006);
\draw(392.2700000000001, -40.400000000000006) node[anchor=north west,align=left] {\large Locally compact abelian groups (LCA groups)};
\draw (392.2700000000001, -40.400000000000006) rectangle (406.2000000000001,-43.60000000000001);
\draw(393.2700000000001, -41.400000000000006) node[anchor=north west,align=left] {General \\ properties and\\ structure\\ of LCA groups};
\draw (393.2700000000001, -41.400000000000006) rectangle (397.3700000000001,-43.50000000000001);
\draw(397.4700000000001, -41.400000000000006) node[anchor=north west,align=left] {Structure\\ of group \\ algebras of\\ LCA groups};
\draw (397.4700000000001, -41.400000000000006) rectangle (400.8200000000001,-43.50000000000001);
\draw(392.2700000000001, -43.7) node[anchor=north west,align=left] {\large Computational methods\\ for problems pertaining\\ to topological groups};
\draw (392.2700000000001, -43.7) rectangle (400.0000000000001,-45.300000000000004);
\draw(392.2700000000001, -45.400000000000006) node[anchor=north west,align=left] {\large History of \\ topological groups};
\draw (392.2700000000001, -45.400000000000006) rectangle (398.4500000000001,-46.50000000000001);
\draw(392.2700000000001, -46.60000000000001) node[anchor=north west,align=left] {\large Compact groups};
\draw (392.2700000000001, -46.60000000000001) rectangle (397.2100000000001,-48.80000000000001);
\draw(393.2700000000001, -47.60000000000001) node[anchor=north west,align=left] {Compact\\ groups};
\draw (393.2700000000001, -47.60000000000001) rectangle (395.6200000000001,-48.70000000000001);
\draw(376.3200000000001, -49.10000000000001) node[anchor=north west,align=left] {\large Locally compact groups and their algebras};
\draw (376.3200000000001, -49.10000000000001) rectangle (391.17000000000013,-60.400000000000006);
\draw(377.3200000000001, -50.10000000000001) node[anchor=north west,align=left] {\(C^*\)-algebras\\ and \\ \(W^*\)-algebras in \\ relation to group\\ representations};
\draw (377.3200000000001, -50.10000000000001) rectangle (382.92000000000013,-52.70000000000001);
\draw(383.0200000000001, -50.10000000000001) node[anchor=north west,align=left] {Representationsof\\ group\\ algebras};
\draw (383.0200000000001, -50.10000000000001) rectangle (387.8700000000001,-52.20000000000001);
\draw(387.9700000000001, -50.10000000000001) node[anchor=north west,align=left] {Rigidity\\ in locally\\ compact\\ groups};
\draw (387.9700000000001, -50.10000000000001) rectangle (391.0700000000001,-52.20000000000001);
\draw(377.3200000000001, -52.80000000000001) node[anchor=north west,align=left] {Kazhdan’s \\ property (T), the\\ Haagerup \\ property, and \\ generalizations};
\draw (377.3200000000001, -52.80000000000001) rectangle (382.17000000000013,-55.40000000000001);
\draw(382.2700000000001, -52.80000000000001) node[anchor=north west,align=left] {Unitary \\ representations\\ of locally \\ compact groups};
\draw (382.2700000000001, -52.80000000000001) rectangle (386.6200000000001,-54.90000000000001);
\draw(386.7200000000001, -52.80000000000001) node[anchor=north west,align=left] {Other \\ representations\\ of locally\\ compact groups};
\draw (386.7200000000001, -52.80000000000001) rectangle (391.0700000000001,-54.90000000000001);
\draw(377.3200000000001, -55.50000000000001) node[anchor=north west,align=left] {Group \\ algebras of \\ locally compact\\ groups};
\draw (377.3200000000001, -55.50000000000001) rectangle (381.67000000000013,-57.60000000000001);
\draw(381.7700000000001, -55.50000000000001) node[anchor=north west,align=left] {Induced \\ representations\\ for locally\\ compact groups};
\draw (381.7700000000001, -55.50000000000001) rectangle (386.1200000000001,-57.60000000000001);
\draw(386.2200000000001, -55.50000000000001) node[anchor=north west,align=left] {General \\ properties and\\ structure\\ of locally\\ compact groups};
\draw (386.2200000000001, -55.50000000000001) rectangle (390.3200000000001,-58.10000000000001);
\draw(377.3200000000001, -58.20000000000001) node[anchor=north west,align=left] {Duality \\ theorems for\\ locally \\ compact groups};
\draw (377.3200000000001, -58.20000000000001) rectangle (381.42000000000013,-60.30000000000001);
\draw(381.5200000000001, -58.20000000000001) node[anchor=north west,align=left] {Automorphism\\ groups of\\ locally \\ compact groups};
\draw (381.5200000000001, -58.20000000000001) rectangle (385.6200000000001,-60.30000000000001);
\draw(385.7200000000001, -58.20000000000001) node[anchor=north west,align=left] {Ergodic\\ theory \\ on groups};
\draw (385.7200000000001, -58.20000000000001) rectangle (388.5700000000001,-59.80000000000001);
\draw(391.2700000000001, -49.10000000000001) node[anchor=north west,align=left] {\large Lie groups};
\draw (391.2700000000001, -49.10000000000001) rectangle (404.1700000000001,-75.9);
\draw(392.2700000000001, -50.10000000000001) node[anchor=north west,align=left] {Geometric \\ Langlands \\ program: \\ representation-theoretic\\ aspects};
\draw (392.2700000000001, -50.10000000000001) rectangle (398.8700000000001,-52.70000000000001);
\draw(398.9700000000001, -50.10000000000001) node[anchor=north west,align=left] {General \\ properties and \\ structure of other\\ Lie groups};
\draw (398.9700000000001, -50.10000000000001) rectangle (404.0700000000001,-52.20000000000001);
\draw(392.2700000000001, -52.80000000000001) node[anchor=north west,align=left] {Representations\\ of nilpotent and\\ solvable Lie \\ groups (special \\ orbital integrals,\\ non-type I \\ representations, etc.)};
\draw (392.2700000000001, -52.80000000000001) rectangle (398.3700000000001,-56.40000000000001);
\draw(398.4700000000001, -52.80000000000001) node[anchor=north west,align=left] {Infinite-dimensional\\ Lie \\ groups and their\\ Lie algebras:\\ general properties};
\draw (398.4700000000001, -52.80000000000001) rectangle (404.0700000000001,-55.40000000000001);
\draw(392.2700000000001, -56.50000000000001) node[anchor=north west,align=left] {Representations\\ of Lie and \\ real algebraic\\ groups: algebraic\\ methods \\ (Verma modules, etc.)};
\draw (392.2700000000001, -56.50000000000001) rectangle (398.1200000000001,-59.60000000000001);
\draw(398.2200000000001, -56.50000000000001) node[anchor=north west,align=left] {Analysis on \\ and representations\\ of \\ infinite-dimensional\\ Lie groups};
\draw (398.2200000000001, -56.50000000000001) rectangle (403.8200000000001,-59.10000000000001);
\draw(392.2700000000001, -59.70000000000001) node[anchor=north west,align=left] {Representations\\ of Lie and\\ linear algebraic\\ groups \\ over local fields};
\draw (392.2700000000001, -59.70000000000001) rectangle (397.1200000000001,-62.30000000000001);
\draw(397.2200000000001, -59.70000000000001) node[anchor=north west,align=left] {Representations\\ of Lie and \\ linear algebraic\\ groups over\\ real fields: \\ analytic methods};
\draw (397.2200000000001, -59.70000000000001) rectangle (401.8200000000001,-62.80000000000001);
\draw(392.2700000000001, -62.900000000000006) node[anchor=north west,align=left] {Representations\\ of Lie and \\ linear algebraic\\ groups over\\ global fields\\ and adèle rings};
\draw (392.2700000000001, -62.900000000000006) rectangle (396.8700000000001,-66.0);
\draw(396.9700000000001, -62.900000000000006) node[anchor=north west,align=left] {Applications\\ of Lie groups\\ to the sciences;\\ explicit\\ representations};
\draw (396.9700000000001, -62.900000000000006) rectangle (401.5700000000001,-65.5);
\draw(392.2700000000001, -66.10000000000001) node[anchor=north west,align=left] {General \\ properties and \\ structure of\\ real Lie groups};
\draw (392.2700000000001, -66.10000000000001) rectangle (396.6200000000001,-68.2);
\draw(396.7200000000001, -66.10000000000001) node[anchor=north west,align=left] {Semisimple\\ Lie groups\\ and their \\ representations};
\draw (396.7200000000001, -66.10000000000001) rectangle (401.0700000000001,-68.2);
\draw(401.1700000000001, -66.10000000000001) node[anchor=north west,align=left] {Local Lie\\ groups};
\draw (401.1700000000001, -66.10000000000001) rectangle (404.0200000000001,-67.2);
\draw(392.2700000000001, -68.30000000000001) node[anchor=north west,align=left] {Loop groups\\ and related\\ constructions,\\ group-theoretic\\ treatment};
\draw (392.2700000000001, -68.30000000000001) rectangle (396.6200000000001,-70.9);
\draw(396.7200000000001, -68.30000000000001) node[anchor=north west,align=left] {Structure and\\ representation\\ of the\\ Lorentz group};
\draw (396.7200000000001, -68.30000000000001) rectangle (400.8200000000001,-70.4);
\draw(400.9200000000001, -68.30000000000001) node[anchor=north west,align=left] {Nilpotent\\ and \\ solvable \\ Lie groups};
\draw (400.9200000000001, -68.30000000000001) rectangle (404.0200000000001,-70.4);
\draw(392.2700000000001, -71.0) node[anchor=north west,align=left] {General \\ properties \\ and structure\\ of complex\\ Lie groups};
\draw (392.2700000000001, -71.0) rectangle (396.1200000000001,-73.6);
\draw(396.2200000000001, -71.0) node[anchor=north west,align=left] {Discrete \\ subgroups \\ of Lie groups};
\draw (396.2200000000001, -71.0) rectangle (400.0700000000001,-72.6);
\draw(400.1700000000001, -71.0) node[anchor=north west,align=left] {Continuous\\ cohomologyof\\ Lie groups};
\draw (400.1700000000001, -71.0) rectangle (403.7700000000001,-73.1);
\draw(392.2700000000001, -73.7) node[anchor=north west,align=left] {Analysis\\ on real \\ and complex\\ Lie groups};
\draw (392.2700000000001, -73.7) rectangle (395.6200000000001,-75.8);
\draw(395.7200000000001, -73.7) node[anchor=north west,align=left] {Analysis\\ on \\ \(p\)-adic \\ Lie groups};
\draw (395.7200000000001, -73.7) rectangle (399.0700000000001,-75.8);
\draw(399.1700000000001, -73.7) node[anchor=north west,align=left] {Lie \\ algebras of\\ Lie groups};
\draw (399.1700000000001, -73.7) rectangle (402.5200000000001,-75.3);
\draw(376.3200000000001, -60.50000000000001) node[anchor=north west,align=left] {\large Noncompact transformation groups};
\draw (376.3200000000001, -60.50000000000001) rectangle (387.7200000000001,-65.9);
\draw(377.3200000000001, -61.50000000000001) node[anchor=north west,align=left] {General \\ theory of group\\ and \\ pseudogroup actions};
\draw (377.3200000000001, -61.50000000000001) rectangle (382.67000000000013,-63.60000000000001);
\draw(382.7700000000001, -61.50000000000001) node[anchor=north west,align=left] {Homogeneousspaces};
\draw (382.7700000000001, -61.50000000000001) rectangle (387.6200000000001,-63.10000000000001);
\draw(377.3200000000001, -63.70000000000001) node[anchor=north west,align=left] {Groups as\\ automorphisms\\ of other\\ structures};
\draw (377.3200000000001, -63.70000000000001) rectangle (381.17000000000013,-65.80000000000001);
\draw(381.2700000000001, -63.70000000000001) node[anchor=north west,align=left] {Measurable\\ group\\ actions};
\draw (381.2700000000001, -63.70000000000001) rectangle (384.3700000000001,-65.30000000000001);
\draw(375.3200000000001, -76.10000000000001) node[anchor=north west,align=left] {\LARGE Nonassociative rings and algebras};
\draw (375.3200000000001, -76.10000000000001) rectangle (404.7200000000001,-130.20000000000002);
\draw(376.3200000000001, -77.10000000000001) node[anchor=north west,align=left] {\large Jordan algebras (algebras, triples and pairs)};
\draw (376.3200000000001, -77.10000000000001) rectangle (393.3700000000001,-86.9);
\draw(377.3200000000001, -78.10000000000001) node[anchor=north west,align=left] {Jordan \\ structures associated\\ with \\ other structures};
\draw (377.3200000000001, -78.10000000000001) rectangle (383.17000000000013,-80.2);
\draw(383.2700000000001, -78.10000000000001) node[anchor=north west,align=left] {Finite-dimensional\\ structures of \\ Jordan algebras};
\draw (383.2700000000001, -78.10000000000001) rectangle (388.3700000000001,-80.2);
\draw(388.4700000000001, -78.10000000000001) node[anchor=north west,align=left] {Associated \\ groups, \\ automorphisms of\\ Jordan algebras};
\draw (388.4700000000001, -78.10000000000001) rectangle (393.0700000000001,-80.2);
\draw(377.3200000000001, -80.30000000000001) node[anchor=north west,align=left] {Associated\\ manifolds\\ of \\ Jordan algebras};
\draw (377.3200000000001, -80.30000000000001) rectangle (381.67000000000013,-82.4);
\draw(381.7700000000001, -80.30000000000001) node[anchor=north west,align=left] {Idempotents,\\ Peirce \\ decompositions};
\draw (381.7700000000001, -80.30000000000001) rectangle (385.8700000000001,-82.4);
\draw(385.9700000000001, -80.30000000000001) node[anchor=north west,align=left] {Jordan \\ structures on \\ Banach spaces\\ and algebras};
\draw (385.9700000000001, -80.30000000000001) rectangle (390.0700000000001,-82.4);
\draw(390.17000000000013, -80.30000000000001) node[anchor=north west,align=left] {Identities\\ and free\\ Jordan\\ structures};
\draw (390.17000000000013, -80.30000000000001) rectangle (393.27000000000015,-82.4);
\draw(377.3200000000001, -82.50000000000001) node[anchor=north west,align=left] {Applications\\ of Jordan \\ algebras to \\ physics, etc.};
\draw (377.3200000000001, -82.50000000000001) rectangle (381.17000000000013,-84.60000000000001);
\draw(381.2700000000001, -82.50000000000001) node[anchor=north west,align=left] {Exceptional\\ Jordan\\ structures};
\draw (381.2700000000001, -82.50000000000001) rectangle (384.6200000000001,-84.10000000000001);
\draw(384.7200000000001, -82.50000000000001) node[anchor=north west,align=left] {Structure\\ theory \\ for Jordan\\ algebras};
\draw (384.7200000000001, -82.50000000000001) rectangle (387.8200000000001,-84.60000000000001);
\draw(387.92000000000013, -82.50000000000001) node[anchor=north west,align=left] {Simple,\\ semisimple\\ Jordan\\ algebras};
\draw (387.92000000000013, -82.50000000000001) rectangle (391.02000000000015,-84.60000000000001);
\draw(377.3200000000001, -84.7) node[anchor=north west,align=left] {Associated\\ geometries\\ of Jordan\\ algebras};
\draw (377.3200000000001, -84.7) rectangle (380.42000000000013,-86.8);
\draw(380.5200000000001, -84.7) node[anchor=north west,align=left] {Division\\ algebras\\ and Jordan\\ algebras};
\draw (380.5200000000001, -84.7) rectangle (383.6200000000001,-86.8);
\draw(383.7200000000001, -84.7) node[anchor=north west,align=left] {Super \\ structures};
\draw (383.7200000000001, -84.7) rectangle (386.8200000000001,-85.8);
\draw(386.92000000000013, -84.7) node[anchor=north west,align=left] {Radicals\\ in Jordan\\ algebras};
\draw (386.92000000000013, -84.7) rectangle (389.77000000000015,-86.3);
\draw(393.4700000000001, -77.10000000000001) node[anchor=north west,align=left] {\large General nonassociative rings};
\draw (393.4700000000001, -77.10000000000001) rectangle (404.62000000000006,-96.9);
\draw(394.4700000000001, -78.10000000000001) node[anchor=north west,align=left] {General theory\\ of \\ nonassociative rings\\ and algebras};
\draw (394.4700000000001, -78.10000000000001) rectangle (400.0700000000001,-80.2);
\draw(400.1700000000001, -78.10000000000001) node[anchor=north west,align=left] {Radical theory\\ (nonassociative\\ rings\\ and algebras)};
\draw (400.1700000000001, -78.10000000000001) rectangle (404.5200000000001,-80.2);
\draw(394.4700000000001, -80.30000000000001) node[anchor=north west,align=left] {Automorphisms,\\ derivations, \\ other operators \\ (nonassociative\\ rings and algebras)};
\draw (394.4700000000001, -80.30000000000001) rectangle (399.8200000000001,-82.9);
\draw(399.9200000000001, -80.30000000000001) node[anchor=north west,align=left] {Nonassociative\\ algebras\\ satisfying \\ other identities};
\draw (399.9200000000001, -80.30000000000001) rectangle (404.5200000000001,-82.4);
\draw(394.4700000000001, -83.00000000000001) node[anchor=north west,align=left] {Compositionalgebras};
\draw (394.4700000000001, -83.00000000000001) rectangle (399.8200000000001,-84.60000000000001);
\draw(399.9200000000001, -83.00000000000001) node[anchor=north west,align=left] {Quadratic\\ algebras \\ (but not \\ quadratic \\ Jordan algebras)};
\draw (399.9200000000001, -83.00000000000001) rectangle (404.5200000000001,-85.60000000000001);
\draw(394.4700000000001, -85.70000000000002) node[anchor=north west,align=left] {Power-associative\\ rings};
\draw (394.4700000000001, -85.70000000000002) rectangle (399.3200000000001,-87.30000000000001);
\draw(399.4200000000001, -85.70000000000002) node[anchor=north west,align=left] {Gröbner-Shirshov\\ bases \\ in nonassociative\\ algebras};
\draw (399.4200000000001, -85.70000000000002) rectangle (404.2700000000001,-87.80000000000001);
\draw(394.4700000000001, -87.9) node[anchor=north west,align=left] {Noncommutative\\ Jordan\\ algebras};
\draw (394.4700000000001, -87.9) rectangle (398.5700000000001,-89.5);
\draw(398.6700000000001, -87.9) node[anchor=north west,align=left] {Nonassociative\\ division\\ algebras};
\draw (398.6700000000001, -87.9) rectangle (402.7700000000001,-89.5);
\draw(394.4700000000001, -89.60000000000001) node[anchor=north west,align=left] {Free \\ nonassociative\\ algebras};
\draw (394.4700000000001, -89.60000000000001) rectangle (398.5700000000001,-91.2);
\draw(398.6700000000001, -89.60000000000001) node[anchor=north west,align=left] {Structure \\ theory for \\ nonassociative\\ algebras};
\draw (398.6700000000001, -89.60000000000001) rectangle (402.7700000000001,-91.7);
\draw(394.4700000000001, -91.80000000000001) node[anchor=north west,align=left] {Other \\ \(n\)-ary \\ compositions \\ \((n \ge 3)\)};
\draw (394.4700000000001, -91.80000000000001) rectangle (398.3200000000001,-93.9);
\draw(398.4200000000001, -91.80000000000001) node[anchor=north west,align=left] {Superalgebras};
\draw (398.4200000000001, -91.80000000000001) rectangle (402.2700000000001,-92.9);
\draw(394.4700000000001, -94.0) node[anchor=north west,align=left] {Ternary\\ compositions};
\draw (394.4700000000001, -94.0) rectangle (398.0700000000001,-95.6);
\draw(398.1700000000001, -94.0) node[anchor=north west,align=left] {Flexible\\ algebras};
\draw (398.1700000000001, -94.0) rectangle (400.7700000000001,-95.6);
\draw(400.87000000000006, -94.0) node[anchor=north west,align=left] {Leibniz\\ algebras};
\draw (400.87000000000006, -94.0) rectangle (403.4700000000001,-95.1);
\draw(394.4700000000001, -95.7) node[anchor=north west,align=left] {Valued\\ algebras};
\draw (394.4700000000001, -95.7) rectangle (397.0700000000001,-96.8);
\draw(376.3200000000001, -87.00000000000001) node[anchor=north west,align=left] {\large Other nonassociative rings and algebras};
\draw (376.3200000000001, -87.00000000000001) rectangle (389.67000000000013,-92.90000000000002);
\draw(377.3200000000001, -88.00000000000001) node[anchor=north west,align=left] {\((\gamma, \delta)\)-rings,\\ including \\ \((1,-1)\)-rings};
\draw (377.3200000000001, -88.00000000000001) rectangle (381.92000000000013,-90.60000000000001);
\draw(382.0200000000001, -88.00000000000001) node[anchor=north west,align=left] {Lie-admissible\\ algebras};
\draw (382.0200000000001, -88.00000000000001) rectangle (386.1200000000001,-89.60000000000001);
\draw(386.2200000000001, -88.00000000000001) node[anchor=north west,align=left] {Alternative\\ rings};
\draw (386.2200000000001, -88.00000000000001) rectangle (389.5700000000001,-89.60000000000001);
\draw(377.3200000000001, -90.70000000000002) node[anchor=north west,align=left] {Right \\ alternative\\ rings};
\draw (377.3200000000001, -90.70000000000002) rectangle (380.67000000000013,-92.30000000000001);
\draw(380.7700000000001, -90.70000000000002) node[anchor=north west,align=left] {(non-Lie)\\ Hom \\ algebras \\ and topics};
\draw (380.7700000000001, -90.70000000000002) rectangle (383.8700000000001,-92.80000000000001);
\draw(383.9700000000001, -90.70000000000002) node[anchor=north west,align=left] {Mal’tsev\\ rings and\\ algebras};
\draw (383.9700000000001, -90.70000000000002) rectangle (386.8200000000001,-92.30000000000001);
\draw(386.92000000000013, -90.70000000000002) node[anchor=north west,align=left] {Genetic\\ algebras};
\draw (386.92000000000013, -90.70000000000002) rectangle (389.52000000000015,-91.80000000000001);
\draw(376.3200000000001, -93.00000000000001) node[anchor=north west,align=left] {\large Computational methods\\ for problems \\ pertaining to nonassociative\\ rings and algebras};
\draw (376.3200000000001, -93.00000000000001) rectangle (385.6000000000001,-95.10000000000001);
\draw(376.3200000000001, -97.00000000000001) node[anchor=north west,align=left] {\large Lie algebras and Lie superalgebras};
\draw (376.3200000000001, -97.00000000000001) rectangle (389.42000000000013,-130.10000000000002);
\draw(377.3200000000001, -98.00000000000001) node[anchor=north west,align=left] {Lie (super)algebras\\ associated \\ with other structures\\ (associative,\\ Jordan, etc.)};
\draw (377.3200000000001, -98.00000000000001) rectangle (383.17000000000013,-100.60000000000001);
\draw(383.2700000000001, -98.00000000000001) node[anchor=north west,align=left] {Kac-Moody \\ (super)algebras; \\ extended affine Lie\\ algebras; \\ toroidal Lie algebras};
\draw (383.2700000000001, -98.00000000000001) rectangle (389.1200000000001,-100.60000000000001);
\draw(377.3200000000001, -100.70000000000002) node[anchor=north west,align=left] {Quantum groups\\ (quantized \\ enveloping algebras)\\ and related\\ deformations};
\draw (377.3200000000001, -100.70000000000002) rectangle (382.92000000000013,-103.30000000000001);
\draw(383.0200000000001, -100.70000000000002) node[anchor=north west,align=left] {Infinite-dimensional\\ Lie \\ (super)algebras};
\draw (383.0200000000001, -100.70000000000002) rectangle (388.6200000000001,-102.80000000000001);
\draw(377.3200000000001, -103.40000000000002) node[anchor=north west,align=left] {Automorphisms,\\ derivations, \\ other operators \\ for Lie algebras\\ and super algebras};
\draw (377.3200000000001, -103.40000000000002) rectangle (382.42000000000013,-106.00000000000001);
\draw(382.5200000000001, -103.40000000000002) node[anchor=north west,align=left] {Applications\\ of Lie algebras\\ and \\ superalgebras to \\ integrable systems};
\draw (382.5200000000001, -103.40000000000002) rectangle (387.6200000000001,-106.00000000000001);
\draw(377.3200000000001, -106.10000000000002) node[anchor=north west,align=left] {Applications\\ of Lie \\ (super)algebras to\\ physics, etc.};
\draw (377.3200000000001, -106.10000000000002) rectangle (382.42000000000013,-108.20000000000002);
\draw(382.5200000000001, -106.10000000000002) node[anchor=north west,align=left] {Vertex operators;\\ vertex \\ operator algebras\\ and related\\ structures};
\draw (382.5200000000001, -106.10000000000002) rectangle (387.3700000000001,-108.70000000000002);
\draw(377.3200000000001, -108.80000000000001) node[anchor=north west,align=left] {Representations\\ of Lie \\ algebras and Lie\\ superalgebras,\\ algebraic\\ theory (weights)};
\draw (377.3200000000001, -108.80000000000001) rectangle (381.92000000000013,-111.9);
\draw(382.0200000000001, -108.80000000000001) node[anchor=north west,align=left] {Lie algebras\\ of \\ linear \\ algebraic groups};
\draw (382.0200000000001, -108.80000000000001) rectangle (386.6200000000001,-110.9);
\draw(386.7200000000001, -108.80000000000001) node[anchor=north west,align=left] {Poisson\\ algebras};
\draw (386.7200000000001, -108.80000000000001) rectangle (389.3200000000001,-109.9);
\draw(377.3200000000001, -112.00000000000001) node[anchor=north west,align=left] {Lie algebras\\ of vector\\ fields and\\ related \\ (super) algebras};
\draw (377.3200000000001, -112.00000000000001) rectangle (381.92000000000013,-114.60000000000001);
\draw(382.0200000000001, -112.00000000000001) node[anchor=north west,align=left] {Identities,\\ free \\ Lie \\ (super)algebras};
\draw (382.0200000000001, -112.00000000000001) rectangle (386.3700000000001,-114.10000000000001);
\draw(386.4700000000001, -112.00000000000001) node[anchor=north west,align=left] {Coadjoint\\ orbits;\\ nilpotent\\ varieties};
\draw (386.4700000000001, -112.00000000000001) rectangle (389.3200000000001,-114.10000000000001);
\draw(377.3200000000001, -114.70000000000002) node[anchor=north west,align=left] {Representations\\ of Lie algebras\\ and Lie \\ superalgebras, \\ analytic theory};
\draw (377.3200000000001, -114.70000000000002) rectangle (381.67000000000013,-117.30000000000001);
\draw(381.7700000000001, -114.70000000000002) node[anchor=north west,align=left] {Simple, \\ semisimple, \\ reductive \\ (super)algebras};
\draw (381.7700000000001, -114.70000000000002) rectangle (386.1200000000001,-116.80000000000001);
\draw(386.2200000000001, -114.70000000000002) node[anchor=north west,align=left] {Root \\ systems};
\draw (386.2200000000001, -114.70000000000002) rectangle (388.5700000000001,-115.80000000000001);
\draw(377.3200000000001, -117.4) node[anchor=north west,align=left] {Exceptional\\ (super)algebras};
\draw (377.3200000000001, -117.4) rectangle (381.67000000000013,-119.0);
\draw(381.7700000000001, -117.4) node[anchor=north west,align=left] {Solvable,\\ nilpotent\\ (super)algebras};
\draw (381.7700000000001, -117.4) rectangle (386.1200000000001,-119.5);
\draw(377.3200000000001, -119.60000000000001) node[anchor=north west,align=left] {Universal\\ enveloping\\ (super)algebras};
\draw (377.3200000000001, -119.60000000000001) rectangle (381.67000000000013,-121.7);
\draw(381.7700000000001, -119.60000000000001) node[anchor=north west,align=left] {Yang-Baxter\\ equations\\ and Rota-Baxter\\ operators};
\draw (381.7700000000001, -119.60000000000001) rectangle (386.1200000000001,-121.7);
\draw(377.3200000000001, -121.80000000000001) node[anchor=north west,align=left] {Modular \\ Lie \\ (super)algebras};
\draw (377.3200000000001, -121.80000000000001) rectangle (381.67000000000013,-123.4);
\draw(381.7700000000001, -121.80000000000001) node[anchor=north west,align=left] {Homological\\ methods\\ in Lie \\ (super)algebras};
\draw (381.7700000000001, -121.80000000000001) rectangle (386.1200000000001,-123.9);
\draw(377.3200000000001, -124.00000000000001) node[anchor=north west,align=left] {Cohomology\\ of Lie\\ (super)algebras};
\draw (377.3200000000001, -124.00000000000001) rectangle (381.67000000000013,-126.10000000000001);
\draw(381.7700000000001, -124.00000000000001) node[anchor=north west,align=left] {Lie bialgebras;\\ Lie\\ coalgebras};
\draw (381.7700000000001, -124.00000000000001) rectangle (386.1200000000001,-125.60000000000001);
\draw(377.3200000000001, -126.20000000000002) node[anchor=north west,align=left] {Graded \\ Lie \\ (super)algebras};
\draw (377.3200000000001, -126.20000000000002) rectangle (381.67000000000013,-127.80000000000001);
\draw(381.7700000000001, -126.20000000000002) node[anchor=north west,align=left] {Color Lie\\ (super)algebras};
\draw (381.7700000000001, -126.20000000000002) rectangle (386.1200000000001,-127.80000000000001);
\draw(377.3200000000001, -127.9) node[anchor=north west,align=left] {Structure \\ theory for Lie\\ algebras and\\ superalgebras};
\draw (377.3200000000001, -127.9) rectangle (381.42000000000013,-130.0);
\draw(381.5200000000001, -127.9) node[anchor=north west,align=left] {Hom-Lie \\ and related\\ algebras};
\draw (381.5200000000001, -127.9) rectangle (384.8700000000001,-129.5);
\draw(384.9700000000001, -127.9) node[anchor=north west,align=left] {Virasoro\\ and related\\ algebras};
\draw (384.9700000000001, -127.9) rectangle (388.3200000000001,-129.5);
\draw(389.5200000000001, -97.00000000000001) node[anchor=north west,align=left] {\large History of \\ nonassociative \\ rings and algebras};
\draw (389.5200000000001, -97.00000000000001) rectangle (395.7000000000001,-98.60000000000001);
\draw(406.8700000000001, -1) node[anchor=north west,align=left] {\LARGE Mathematical logic and foundations};
\draw (406.8700000000001, -1) rectangle (436.2700000000001,-80.69999999999999);
\draw(407.8700000000001, -2) node[anchor=north west,align=left] {\large Proof theory and constructive mathematics};
\draw (407.8700000000001, -2) rectangle (423.47000000000014,-15.0);
\draw(408.8700000000001, -3) node[anchor=north west,align=left] {Provability \\ logics and related\\ algebras \\ (e.g., diagonalizable\\ algebras)};
\draw (408.8700000000001, -3) rectangle (414.72000000000014,-5.6);
\draw(414.8200000000001, -3) node[anchor=north west,align=left] {Proof-theoretic\\ aspects of \\ linear logic and\\ other \\ substructural logics};
\draw (414.8200000000001, -3) rectangle (420.42000000000013,-5.6);
\draw(420.5200000000001, -3) node[anchor=north west,align=left] {Structure\\ of\\ proofs};
\draw (420.5200000000001, -3) rectangle (423.3700000000001,-4.6);
\draw(408.8700000000001, -5.7) node[anchor=north west,align=left] {Proof theory,\\ general\\ (including\\ proof-theoretic\\ semantics)};
\draw (408.8700000000001, -5.7) rectangle (413.22000000000014,-8.3);
\draw(413.3200000000001, -5.7) node[anchor=north west,align=left] {Cut-elimination\\ and\\ normal-form\\ theorems};
\draw (413.3200000000001, -5.7) rectangle (417.67000000000013,-7.800000000000001);
\draw(417.7700000000001, -5.7) node[anchor=north west,align=left] {Relative \\ consistency\\ and \\ interpretations};
\draw (417.7700000000001, -5.7) rectangle (422.1200000000001,-7.800000000000001);
\draw(408.8700000000001, -8.4) node[anchor=north west,align=left] {Gödel \\ numberings and \\ issues of \\ incompleteness};
\draw (408.8700000000001, -8.4) rectangle (413.22000000000014,-10.5);
\draw(413.3200000000001, -8.4) node[anchor=north west,align=left] {Metamathematics\\ of\\ constructive\\ systems};
\draw (413.3200000000001, -8.4) rectangle (417.67000000000013,-10.5);
\draw(417.7700000000001, -8.4) node[anchor=north west,align=left] {Intuitionistic\\ mathematics};
\draw (417.7700000000001, -8.4) rectangle (421.8700000000001,-10.0);
\draw(408.8700000000001, -10.600000000000001) node[anchor=north west,align=left] {Second- and\\ higher-order\\ arithmetic\\ and fragments};
\draw (408.8700000000001, -10.600000000000001) rectangle (412.72000000000014,-12.700000000000001);
\draw(412.8200000000001, -10.600000000000001) node[anchor=north west,align=left] {Other \\ constructive \\ mathematics};
\draw (412.8200000000001, -10.600000000000001) rectangle (416.67000000000013,-12.200000000000001);
\draw(416.7700000000001, -10.600000000000001) node[anchor=north west,align=left] {Constructive\\ and\\ recursive\\ analysis};
\draw (416.7700000000001, -10.600000000000001) rectangle (420.3700000000001,-12.700000000000001);
\draw(408.8700000000001, -12.8) node[anchor=north west,align=left] {Functionals\\ in proof\\ theory};
\draw (408.8700000000001, -12.8) rectangle (412.22000000000014,-14.4);
\draw(412.3200000000001, -12.8) node[anchor=north west,align=left] {Recursive\\ ordinals \\ and ordinal\\ notations};
\draw (412.3200000000001, -12.8) rectangle (415.67000000000013,-14.9);
\draw(415.7700000000001, -12.8) node[anchor=north west,align=left] {First-order\\ arithmetic\\ and\\ fragments};
\draw (415.7700000000001, -12.8) rectangle (419.1200000000001,-14.9);
\draw(419.22000000000014, -12.8) node[anchor=north west,align=left] {Complexity\\ of proofs};
\draw (419.22000000000014, -12.8) rectangle (422.32000000000016,-14.4);
\draw(423.5700000000001, -2) node[anchor=north west,align=left] {\large Computability and recursion theory};
\draw (423.5700000000001, -2) rectangle (436.17000000000013,-26.3);
\draw(424.5700000000001, -3) node[anchor=north west,align=left] {Computability\\ and recursion\\ theory on \\ ordinals, \\ admissible sets, etc.};
\draw (424.5700000000001, -3) rectangle (430.42000000000013,-5.6);
\draw(430.5200000000001, -3) node[anchor=north west,align=left] {Complexity of\\ computation \\ (including implicit\\ computational\\ complexity)};
\draw (430.5200000000001, -3) rectangle (435.8700000000001,-5.6);
\draw(424.5700000000001, -5.7) node[anchor=north west,align=left] {Computation\\ over the \\ reals, \\ computable analysis};
\draw (424.5700000000001, -5.7) rectangle (429.92000000000013,-7.800000000000001);
\draw(430.0200000000001, -5.7) node[anchor=north west,align=left] {Other degrees\\ and reducibilities\\ in \\ computability and \\ recursion theory};
\draw (430.0200000000001, -5.7) rectangle (435.1200000000001,-8.3);
\draw(424.5700000000001, -8.4) node[anchor=north west,align=left] {Recursive \\ equivalence\\ types of \\ sets and \\ structures, isols};
\draw (424.5700000000001, -8.4) rectangle (429.42000000000013,-11.0);
\draw(429.5200000000001, -8.4) node[anchor=north west,align=left] {Word problems,\\ etc. in\\ computability\\ and \\ recursion theory};
\draw (429.5200000000001, -8.4) rectangle (434.1200000000001,-11.0);
\draw(424.5700000000001, -11.100000000000001) node[anchor=north west,align=left] {Hierarchies\\ of computability\\ and\\ definability};
\draw (424.5700000000001, -11.100000000000001) rectangle (429.17000000000013,-13.200000000000001);
\draw(429.2700000000001, -11.100000000000001) node[anchor=north west,align=left] {Abstract and\\ axiomatic\\ computability\\ and \\ recursion theory};
\draw (429.2700000000001, -11.100000000000001) rectangle (433.8700000000001,-13.700000000000001);
\draw(424.5700000000001, -13.8) node[anchor=north west,align=left] {Applications\\ of computability\\ and \\ recursion theory};
\draw (424.5700000000001, -13.8) rectangle (429.17000000000013,-15.9);
\draw(429.2700000000001, -13.8) node[anchor=north west,align=left] {Automata and\\ formal grammars\\ in connection\\ with logical\\ questions};
\draw (429.2700000000001, -13.8) rectangle (433.6200000000001,-16.400000000000002);
\draw(424.5700000000001, -16.5) node[anchor=north west,align=left] {Recursively \\ (computably) \\ enumerable sets\\ and degrees};
\draw (424.5700000000001, -16.5) rectangle (428.92000000000013,-18.6);
\draw(429.0200000000001, -16.5) node[anchor=north west,align=left] {Undecidability\\ and \\ degrees of sets\\ of sentences};
\draw (429.0200000000001, -16.5) rectangle (433.3700000000001,-18.6);
\draw(424.5700000000001, -18.7) node[anchor=north west,align=left] {Thue and\\ Post \\ systems, etc.};
\draw (424.5700000000001, -18.7) rectangle (428.42000000000013,-20.3);
\draw(428.5200000000001, -18.7) node[anchor=north west,align=left] {Recursive \\ functions and\\ relations,\\ subrecursive\\ hierarchies};
\draw (428.5200000000001, -18.7) rectangle (432.3700000000001,-21.3);
\draw(432.4700000000001, -18.7) node[anchor=north west,align=left] {Other Turing\\ degree\\ structures};
\draw (432.4700000000001, -18.7) rectangle (436.0700000000001,-20.3);
\draw(424.5700000000001, -21.4) node[anchor=north west,align=left] {Theory of \\ numerations,\\ effectively\\ presented\\ structures};
\draw (424.5700000000001, -21.4) rectangle (428.17000000000013,-24.0);
\draw(428.2700000000001, -21.4) node[anchor=north west,align=left] {Inductive\\ definability};
\draw (428.2700000000001, -21.4) rectangle (431.8700000000001,-23.0);
\draw(431.9700000000001, -21.4) node[anchor=north west,align=left] {Turing \\ machines\\ and related\\ notions};
\draw (431.9700000000001, -21.4) rectangle (435.3200000000001,-23.5);
\draw(424.5700000000001, -24.099999999999998) node[anchor=north west,align=left] {Algorithmic\\ randomness\\ and\\ dimension};
\draw (424.5700000000001, -24.099999999999998) rectangle (427.92000000000013,-26.2);
\draw(428.0200000000001, -24.099999999999998) node[anchor=north west,align=left] {Higher-type\\ and set\\ recursion\\ theory};
\draw (428.0200000000001, -24.099999999999998) rectangle (431.3700000000001,-26.2);
\draw(407.8700000000001, -15.1) node[anchor=north west,align=left] {\large Philosophical aspects of logic and foundations};
\draw (407.8700000000001, -15.1) rectangle (422.73000000000013,-18.8);
\draw(408.8700000000001, -16.1) node[anchor=north west,align=left] {Philosophical\\ and \\ critical aspects\\ of logic\\ and foundations};
\draw (408.8700000000001, -16.1) rectangle (413.47000000000014,-18.700000000000003);
\draw(413.5700000000001, -16.1) node[anchor=north west,align=left] {Logic in\\ the \\ philosophy\\ of science};
\draw (413.5700000000001, -16.1) rectangle (416.67000000000013,-18.200000000000003);
\draw(407.8700000000001, -18.900000000000002) node[anchor=north west,align=left] {\large Computational methods\\ for problems \\ pertaining to mathematical\\ logic and foundations};
\draw (407.8700000000001, -18.900000000000002) rectangle (416.53000000000014,-21.000000000000004);
\draw(407.8700000000001, -21.1) node[anchor=north west,align=left] {\large History of \\ mathematical logic\\ and foundations};
\draw (407.8700000000001, -21.1) rectangle (414.0500000000001,-22.700000000000003);
\draw(407.8700000000001, -26.400000000000002) node[anchor=north west,align=left] {\large General logic};
\draw (407.8700000000001, -26.400000000000002) rectangle (420.0200000000001,-47.2);
\draw(408.8700000000001, -27.400000000000002) node[anchor=north west,align=left] {Substructural \\ logics (including \\ relevance, entailment,\\ linear logic,\\ Lambek calculus,\\ BCK and BCI logics)};
\draw (408.8700000000001, -27.400000000000002) rectangle (414.97000000000014,-30.500000000000004);
\draw(415.0700000000001, -27.400000000000002) node[anchor=north west,align=left] {Probability\\ and \\ inductive logic};
\draw (415.0700000000001, -27.400000000000002) rectangle (419.42000000000013,-29.000000000000004);
\draw(415.0700000000001, -29.1) node[anchor=north west,align=left] {Temporal\\ logic};
\draw (415.0700000000001, -29.1) rectangle (417.67000000000013,-30.200000000000003);
\draw(408.8700000000001, -30.6) node[anchor=north west,align=left] {Subsystems \\ of classical\\ logic (including\\ intuitionistic logic)};
\draw (408.8700000000001, -30.6) rectangle (414.72000000000014,-33.2);
\draw(414.8200000000001, -30.6) node[anchor=north west,align=left] {Foundations \\ of classical \\ theories \\ (including reverse\\ mathematics)};
\draw (414.8200000000001, -30.6) rectangle (419.92000000000013,-33.2);
\draw(408.8700000000001, -33.300000000000004) node[anchor=north west,align=left] {Logics of \\ knowledge and\\ belief \\ (including \\ belief change)};
\draw (408.8700000000001, -33.300000000000004) rectangle (412.97000000000014,-35.900000000000006);
\draw(413.0700000000001, -33.300000000000004) node[anchor=north west,align=left] {Paraconsistent\\ logics};
\draw (413.0700000000001, -33.300000000000004) rectangle (417.17000000000013,-34.900000000000006);
\draw(417.2700000000001, -33.300000000000004) node[anchor=north west,align=left] {Combined\\ logics};
\draw (417.2700000000001, -33.300000000000004) rectangle (419.8700000000001,-34.400000000000006);
\draw(408.8700000000001, -36.0) node[anchor=north west,align=left] {Classical\\ propositional\\ logic};
\draw (408.8700000000001, -36.0) rectangle (412.72000000000014,-37.6);
\draw(412.8200000000001, -36.0) node[anchor=north west,align=left] {Mechanization\\ of proofs\\ and logical\\ operations};
\draw (412.8200000000001, -36.0) rectangle (416.67000000000013,-38.1);
\draw(416.7700000000001, -36.0) node[anchor=north west,align=left] {Abstract\\ deductive\\ systems};
\draw (416.7700000000001, -36.0) rectangle (419.6200000000001,-37.6);
\draw(408.8700000000001, -38.2) node[anchor=north west,align=left] {Higher-order\\ logic};
\draw (408.8700000000001, -38.2) rectangle (412.47000000000014,-39.800000000000004);
\draw(412.5700000000001, -38.2) node[anchor=north west,align=left] {Decidability\\ of theories\\ and sets\\ of sentences};
\draw (412.5700000000001, -38.2) rectangle (416.17000000000013,-40.300000000000004);
\draw(416.2700000000001, -38.2) node[anchor=north west,align=left] {Fuzzy logic;\\ logic \\ of vagueness};
\draw (416.2700000000001, -38.2) rectangle (419.8700000000001,-39.800000000000004);
\draw(408.8700000000001, -40.400000000000006) node[anchor=north west,align=left] {Intermediate\\ logics};
\draw (408.8700000000001, -40.400000000000006) rectangle (412.47000000000014,-42.00000000000001);
\draw(412.5700000000001, -40.400000000000006) node[anchor=north west,align=left] {Other \\ nonclassical\\ logic};
\draw (412.5700000000001, -40.400000000000006) rectangle (416.17000000000013,-42.00000000000001);
\draw(416.2700000000001, -40.400000000000006) node[anchor=north west,align=left] {Other \\ applications\\ of logic};
\draw (416.2700000000001, -40.400000000000006) rectangle (419.8700000000001,-42.00000000000001);
\draw(408.8700000000001, -42.1) node[anchor=north west,align=left] {Classical\\ first-order\\ logic};
\draw (408.8700000000001, -42.1) rectangle (412.22000000000014,-43.7);
\draw(412.3200000000001, -42.1) node[anchor=north west,align=left] {Combinatory\\ logic\\ and lambda\\ calculus};
\draw (412.3200000000001, -42.1) rectangle (415.67000000000013,-44.2);
\draw(415.7700000000001, -42.1) node[anchor=north west,align=left] {Modal logic\\ (including\\ the logic\\ of norms)};
\draw (415.7700000000001, -42.1) rectangle (419.1200000000001,-44.2);
\draw(408.8700000000001, -44.3) node[anchor=north west,align=left] {Many-valued\\ logic};
\draw (408.8700000000001, -44.3) rectangle (412.22000000000014,-45.9);
\draw(412.3200000000001, -44.3) node[anchor=north west,align=left] {Logic of\\ natural\\ languages};
\draw (412.3200000000001, -44.3) rectangle (415.17000000000013,-45.9);
\draw(415.2700000000001, -44.3) node[anchor=north west,align=left] {Logic in\\ computer\\ science};
\draw (415.2700000000001, -44.3) rectangle (417.8700000000001,-45.9);
\draw(408.8700000000001, -46.0) node[anchor=north west,align=left] {Type\\ theory};
\draw (408.8700000000001, -46.0) rectangle (410.97000000000014,-47.1);
\draw(420.1200000000001, -26.400000000000002) node[anchor=north west,align=left] {\large Set theory};
\draw (420.1200000000001, -26.400000000000002) rectangle (431.97000000000014,-46.0);
\draw(421.1200000000001, -27.400000000000002) node[anchor=north west,align=left] {Other classical\\ set theory \\ (including functions,\\ relations,\\ and set algebra)};
\draw (421.1200000000001, -27.400000000000002) rectangle (426.97000000000014,-30.000000000000004);
\draw(427.0700000000001, -27.400000000000002) node[anchor=north west,align=left] {Cardinal \\ characteristics\\ of the\\ continuum};
\draw (427.0700000000001, -27.400000000000002) rectangle (431.42000000000013,-29.500000000000004);
\draw(421.1200000000001, -30.1) node[anchor=north west,align=left] {Other aspects\\ of forcing\\ and \\ Boolean-valued models};
\draw (421.1200000000001, -30.1) rectangle (426.97000000000014,-32.2);
\draw(427.0700000000001, -30.1) node[anchor=north west,align=left] {Continuum\\ hypothesis\\ and \\ Martin’s axiom};
\draw (427.0700000000001, -30.1) rectangle (431.17000000000013,-32.2);
\draw(421.1200000000001, -32.300000000000004) node[anchor=north west,align=left] {Inner models, \\ including \\ constructibility, \\ ordinal definability,\\ and core models};
\draw (421.1200000000001, -32.300000000000004) rectangle (426.97000000000014,-34.900000000000006);
\draw(427.0700000000001, -32.300000000000004) node[anchor=north west,align=left] {Generic \\ absoluteness\\ and \\ forcing axioms};
\draw (427.0700000000001, -32.300000000000004) rectangle (431.17000000000013,-34.400000000000006);
\draw(421.1200000000001, -35.0) node[anchor=north west,align=left] {Ordered sets\\ and their\\ cofinalities;\\ pcf theory};
\draw (421.1200000000001, -35.0) rectangle (424.97000000000014,-37.1);
\draw(425.0700000000001, -35.0) node[anchor=north west,align=left] {Other \\ combinatorial\\ set theory};
\draw (425.0700000000001, -35.0) rectangle (428.92000000000013,-36.6);
\draw(429.0200000000001, -35.0) node[anchor=north west,align=left] {Partition\\ relations};
\draw (429.0200000000001, -35.0) rectangle (431.8700000000001,-36.6);
\draw(421.1200000000001, -37.2) node[anchor=north west,align=left] {Axiomatics of\\ classical set\\ theory and\\ its fragments};
\draw (421.1200000000001, -37.2) rectangle (424.97000000000014,-39.300000000000004);
\draw(425.0700000000001, -37.2) node[anchor=north west,align=left] {Other notions\\ of \\ set-theoretic\\ definability};
\draw (425.0700000000001, -37.2) rectangle (428.92000000000013,-39.300000000000004);
\draw(429.0200000000001, -37.2) node[anchor=north west,align=left] {Large \\ cardinals};
\draw (429.0200000000001, -37.2) rectangle (431.8700000000001,-38.300000000000004);
\draw(421.1200000000001, -39.400000000000006) node[anchor=north west,align=left] {Other \\ set-theoretic\\ hypotheses\\ and axioms};
\draw (421.1200000000001, -39.400000000000006) rectangle (424.97000000000014,-41.50000000000001);
\draw(425.0700000000001, -39.400000000000006) node[anchor=north west,align=left] {Ordinal \\ and cardinal\\ numbers};
\draw (425.0700000000001, -39.400000000000006) rectangle (428.67000000000013,-41.00000000000001);
\draw(428.7700000000001, -39.400000000000006) node[anchor=north west,align=left] {Theory \\ of fuzzy\\ sets, etc.};
\draw (428.7700000000001, -39.400000000000006) rectangle (431.8700000000001,-41.00000000000001);
\draw(421.1200000000001, -41.6) node[anchor=north west,align=left] {Axiom of \\ choice and\\ related \\ propositions};
\draw (421.1200000000001, -41.6) rectangle (424.72000000000014,-43.7);
\draw(424.8200000000001, -41.6) node[anchor=north west,align=left] {Consistency\\ and \\ independence\\ results};
\draw (424.8200000000001, -41.6) rectangle (428.42000000000013,-43.7);
\draw(428.5200000000001, -41.6) node[anchor=north west,align=left] {Descriptive\\ set\\ theory};
\draw (428.5200000000001, -41.6) rectangle (431.8700000000001,-43.2);
\draw(421.1200000000001, -43.8) node[anchor=north west,align=left] {Nonclassical\\ and \\ second-order\\ set theories};
\draw (421.1200000000001, -43.8) rectangle (424.72000000000014,-45.9);
\draw(424.8200000000001, -43.8) node[anchor=north west,align=left] {Applications\\ of\\ set theory};
\draw (424.8200000000001, -43.8) rectangle (428.42000000000013,-45.4);
\draw(428.5200000000001, -43.8) node[anchor=north west,align=left] {Determinacy\\ principles};
\draw (428.5200000000001, -43.8) rectangle (431.8700000000001,-45.4);
\draw(407.8700000000001, -47.300000000000004) node[anchor=north west,align=left] {\large Model theory};
\draw (407.8700000000001, -47.300000000000004) rectangle (418.0200000000001,-80.6);
\draw(408.8700000000001, -48.300000000000004) node[anchor=north west,align=left] {Equational \\ classes, universal\\ algebra \\ in model theory};
\draw (408.8700000000001, -48.300000000000004) rectangle (413.97000000000014,-50.400000000000006);
\draw(414.0700000000001, -48.300000000000004) node[anchor=north west,align=left] {Ultraproducts\\ and\\ related \\ constructions};
\draw (414.0700000000001, -48.300000000000004) rectangle (417.92000000000013,-50.400000000000006);
\draw(408.8700000000001, -50.50000000000001) node[anchor=north west,align=left] {Quantifier \\ elimination,\\ model \\ completeness and \\ related topics};
\draw (408.8700000000001, -50.50000000000001) rectangle (413.72000000000014,-53.10000000000001);
\draw(413.8200000000001, -50.50000000000001) node[anchor=north west,align=left] {Basic \\ properties of \\ first-order \\ languages and\\ structures};
\draw (413.8200000000001, -50.50000000000001) rectangle (417.92000000000013,-53.10000000000001);
\draw(408.8700000000001, -53.2) node[anchor=north west,align=left] {Computable \\ structure theory,\\ computable\\ model theory};
\draw (408.8700000000001, -53.2) rectangle (413.72000000000014,-55.300000000000004);
\draw(413.8200000000001, -53.2) node[anchor=north west,align=left] {Model theory\\ of denumerable\\ and separable\\ structures};
\draw (413.8200000000001, -53.2) rectangle (417.92000000000013,-55.300000000000004);
\draw(408.8700000000001, -55.400000000000006) node[anchor=north west,align=left] {Continuous\\ model theory,\\ model \\ theory of \\ metric structures};
\draw (408.8700000000001, -55.400000000000006) rectangle (413.72000000000014,-58.00000000000001);
\draw(413.8200000000001, -55.400000000000006) node[anchor=north west,align=left] {Interpolation,\\ preservation, \\ definability};
\draw (413.8200000000001, -55.400000000000006) rectangle (417.92000000000013,-57.50000000000001);
\draw(408.8700000000001, -58.10000000000001) node[anchor=north west,align=left] {Classification\\ theory, \\ stability and \\ related concepts\\ in model theory};
\draw (408.8700000000001, -58.10000000000001) rectangle (413.47000000000014,-60.70000000000001);
\draw(413.5700000000001, -58.10000000000001) node[anchor=north west,align=left] {Model-theoretic\\ forcing};
\draw (413.5700000000001, -58.10000000000001) rectangle (417.92000000000013,-59.70000000000001);
\draw(408.8700000000001, -60.800000000000004) node[anchor=north west,align=left] {Nonclassical\\ models \\ (Boolean-valued,\\ sheaf, etc.)};
\draw (408.8700000000001, -60.800000000000004) rectangle (413.47000000000014,-62.900000000000006);
\draw(413.5700000000001, -60.800000000000004) node[anchor=north west,align=left] {Model-theoretic\\ algebra};
\draw (413.5700000000001, -60.800000000000004) rectangle (417.92000000000013,-62.400000000000006);
\draw(408.8700000000001, -63.0) node[anchor=north west,align=left] {Abstract \\ elementary \\ classes and \\ related topics};
\draw (408.8700000000001, -63.0) rectangle (412.97000000000014,-65.1);
\draw(413.0700000000001, -63.0) node[anchor=north west,align=left] {Other \\ model \\ constructions};
\draw (413.0700000000001, -63.0) rectangle (416.92000000000013,-64.6);
\draw(408.8700000000001, -65.2) node[anchor=north west,align=left] {Set-theoretic\\ model theory};
\draw (408.8700000000001, -65.2) rectangle (412.72000000000014,-66.8);
\draw(412.8200000000001, -65.2) node[anchor=north west,align=left] {Models \\ of arithmetic\\ and\\ set theory};
\draw (412.8200000000001, -65.2) rectangle (416.67000000000013,-67.3);
\draw(408.8700000000001, -67.4) node[anchor=north west,align=left] {Logic with\\ extra \\ quantifiers \\ and operators};
\draw (408.8700000000001, -67.4) rectangle (412.72000000000014,-69.5);
\draw(412.8200000000001, -67.4) node[anchor=north west,align=left] {Second- \\ and \\ higher-order \\ model theory};
\draw (412.8200000000001, -67.4) rectangle (416.67000000000013,-69.5);
\draw(408.8700000000001, -69.6) node[anchor=north west,align=left] {Categoricity\\ and \\ completeness\\ of theories};
\draw (408.8700000000001, -69.6) rectangle (412.47000000000014,-71.69999999999999);
\draw(412.5700000000001, -69.6) node[anchor=north west,align=left] {Models with\\ special \\ properties \\ (saturated, \\ rigid, etc.)};
\draw (412.5700000000001, -69.6) rectangle (416.17000000000013,-72.19999999999999);
\draw(408.8700000000001, -72.3) node[anchor=north west,align=left] {Model theory\\ of ordered\\ structures;\\ o-minimality};
\draw (408.8700000000001, -72.3) rectangle (412.47000000000014,-74.39999999999999);
\draw(412.5700000000001, -72.3) node[anchor=north west,align=left] {Models of\\ other \\ mathematical\\ theories};
\draw (412.5700000000001, -72.3) rectangle (416.17000000000013,-74.39999999999999);
\draw(408.8700000000001, -74.5) node[anchor=north west,align=left] {Other \\ classical \\ first-order \\ model theory};
\draw (408.8700000000001, -74.5) rectangle (412.47000000000014,-76.6);
\draw(412.5700000000001, -74.5) node[anchor=north west,align=left] {Applications\\ of \\ model theory};
\draw (412.5700000000001, -74.5) rectangle (416.17000000000013,-76.1);
\draw(408.8700000000001, -76.7) node[anchor=north west,align=left] {Model \\ theory of\\ finite \\ structures};
\draw (408.8700000000001, -76.7) rectangle (411.97000000000014,-78.8);
\draw(412.0700000000001, -76.7) node[anchor=north west,align=left] {Properties\\ of classes\\ of models};
\draw (412.0700000000001, -76.7) rectangle (415.17000000000013,-78.3);
\draw(415.2700000000001, -76.7) node[anchor=north west,align=left] {Abstract\\ model\\ theory};
\draw (415.2700000000001, -76.7) rectangle (417.8700000000001,-78.3);
\draw(408.8700000000001, -78.9) node[anchor=north west,align=left] {Logic on\\ admissible\\ sets};
\draw (408.8700000000001, -78.9) rectangle (411.97000000000014,-80.5);
\draw(412.0700000000001, -78.9) node[anchor=north west,align=left] {Other \\ infinitary\\ logic};
\draw (412.0700000000001, -78.9) rectangle (415.17000000000013,-80.5);
\draw(418.1200000000001, -47.300000000000004) node[anchor=north west,align=left] {\large Algebraic logic};
\draw (418.1200000000001, -47.300000000000004) rectangle (427.5200000000001,-56.60000000000001);
\draw(419.1200000000001, -48.300000000000004) node[anchor=north west,align=left] {Cylindric and\\ polyadic \\ algebras; \\ relation algebras};
\draw (419.1200000000001, -48.300000000000004) rectangle (423.97000000000014,-50.400000000000006);
\draw(424.0700000000001, -48.300000000000004) node[anchor=north west,align=left] {Categorical\\ logic,\\ topoi};
\draw (424.0700000000001, -48.300000000000004) rectangle (427.42000000000013,-49.900000000000006);
\draw(419.1200000000001, -50.50000000000001) node[anchor=north west,align=left] {Logical aspects\\ of \\ Łukasiewicz and \\ Post algebras};
\draw (419.1200000000001, -50.50000000000001) rectangle (423.72000000000014,-52.60000000000001);
\draw(423.8200000000001, -50.50000000000001) node[anchor=north west,align=left] {Logical\\ aspects\\ of Boolean\\ algebras};
\draw (423.8200000000001, -50.50000000000001) rectangle (426.92000000000013,-52.60000000000001);
\draw(419.1200000000001, -52.7) node[anchor=north west,align=left] {Other \\ algebras related\\ to logic};
\draw (419.1200000000001, -52.7) rectangle (423.72000000000014,-54.300000000000004);
\draw(423.8200000000001, -52.7) node[anchor=north west,align=left] {Abstract\\ algebraic\\ logic};
\draw (423.8200000000001, -52.7) rectangle (426.67000000000013,-54.300000000000004);
\draw(419.1200000000001, -54.400000000000006) node[anchor=north west,align=left] {Logical aspects\\ of lattices\\ and related\\ structures};
\draw (419.1200000000001, -54.400000000000006) rectangle (423.47000000000014,-56.50000000000001);
\draw(423.5700000000001, -54.400000000000006) node[anchor=north west,align=left] {Quantum\\ logic};
\draw (423.5700000000001, -54.400000000000006) rectangle (425.92000000000013,-55.50000000000001);
\draw(418.1200000000001, -56.70000000000001) node[anchor=north west,align=left] {\large Nonstandard models};
\draw (418.1200000000001, -56.70000000000001) rectangle (426.5200000000001,-62.60000000000001);
\draw(419.1200000000001, -57.70000000000001) node[anchor=north west,align=left] {Other applications\\ of \\ nonstandard models\\ (economics,\\ physics, etc.)};
\draw (419.1200000000001, -57.70000000000001) rectangle (424.22000000000014,-60.30000000000001);
\draw(419.1200000000001, -60.40000000000001) node[anchor=north west,align=left] {Nonstandard\\ models \\ of arithmetic};
\draw (419.1200000000001, -60.40000000000001) rectangle (422.97000000000014,-62.000000000000014);
\draw(423.0700000000001, -60.40000000000001) node[anchor=north west,align=left] {Nonstandard\\ models\\ in \\ mathematics};
\draw (423.0700000000001, -60.40000000000001) rectangle (426.42000000000013,-62.500000000000014);
\draw(406.8700000000001, -80.79999999999998) node[anchor=north west,align=left] {\LARGE Special functions};
\draw (406.8700000000001, -80.79999999999998) rectangle (436.09000000000015,-119.39999999999999);
\draw(407.8700000000001, -81.79999999999998) node[anchor=north west,align=left] {\large Basic hypergeometric functions};
\draw (407.8700000000001, -81.79999999999998) rectangle (422.2700000000001,-98.29999999999998);
\draw(408.8700000000001, -82.79999999999998) node[anchor=north west,align=left] {Connections of basic\\ hypergeometric \\ functions with quantum\\ groups, Chevalley\\ groups, \(p\)-adic\\ groups, Hecke \\ algebras, and related topics};
\draw (408.8700000000001, -82.79999999999998) rectangle (416.47000000000014,-86.39999999999998);
\draw(416.5700000000001, -82.79999999999998) node[anchor=north west,align=left] {Basic orthogonal\\ polynomials and\\ functions \\ associated with root\\ systems (Macdonald\\ polynomials, etc.)};
\draw (416.5700000000001, -82.79999999999998) rectangle (422.17000000000013,-85.89999999999998);
\draw(408.8700000000001, -86.49999999999999) node[anchor=north west,align=left] {Orthogonal polynomials\\ and functions\\ in several variables\\ expressible in\\ terms of basic \\ hypergeometric functions\\ in one variable};
\draw (408.8700000000001, -86.49999999999999) rectangle (415.47000000000014,-90.09999999999998);
\draw(415.5700000000001, -86.49999999999999) node[anchor=north west,align=left] {Basic orthogonal\\ polynomials \\ and functions \\ (Askey-Wilson \\ polynomials, etc.)};
\draw (415.5700000000001, -86.49999999999999) rectangle (420.67000000000013,-89.09999999999998);
\draw(408.8700000000001, -90.19999999999999) node[anchor=north west,align=left] {Basic \\ hypergeometric \\ functions \\ associated \\ with root systems};
\draw (408.8700000000001, -90.19999999999999) rectangle (413.72000000000014,-92.79999999999998);
\draw(413.8200000000001, -90.19999999999999) node[anchor=north west,align=left] {Other basic \\ hypergeometric \\ functions and \\ integrals in \\ several variables};
\draw (413.8200000000001, -90.19999999999999) rectangle (418.67000000000013,-92.79999999999998);
\draw(408.8700000000001, -92.89999999999998) node[anchor=north west,align=left] {Basic \\ hypergeometric \\ functions in one\\ variable,\\ \({}_r\phi_s\)};
\draw (408.8700000000001, -92.89999999999998) rectangle (413.47000000000014,-95.49999999999997);
\draw(413.5700000000001, -92.89999999999998) node[anchor=north west,align=left] {Basic \\ hypergeometric \\ integrals and \\ functions \\ defined by them};
\draw (413.5700000000001, -92.89999999999998) rectangle (417.92000000000013,-95.49999999999997);
\draw(418.0200000000001, -92.89999999999998) node[anchor=north west,align=left] {Applications\\ of basic\\ hypergeometric\\ functions};
\draw (418.0200000000001, -92.89999999999998) rectangle (422.1200000000001,-94.99999999999997);
\draw(408.8700000000001, -95.59999999999998) node[anchor=north west,align=left] {\(q\)-gamma\\ functions,\\ \(q\)-beta\\ functions\\ and integrals};
\draw (408.8700000000001, -95.59999999999998) rectangle (412.72000000000014,-98.19999999999997);
\draw(412.8200000000001, -95.59999999999998) node[anchor=north west,align=left] {Bibasic \\ functions\\ and multiple\\ bases};
\draw (412.8200000000001, -95.59999999999998) rectangle (416.42000000000013,-97.69999999999997);
\draw(422.3700000000001, -81.79999999999998) node[anchor=north west,align=left] {\large Other special functions};
\draw (422.3700000000001, -81.79999999999998) rectangle (434.2700000000001,-91.59999999999998);
\draw(423.3700000000001, -82.79999999999998) node[anchor=north west,align=left] {Painlevé-typefunctions};
\draw (423.3700000000001, -82.79999999999998) rectangle (429.47000000000014,-84.39999999999998);
\draw(429.5700000000001, -82.79999999999998) node[anchor=north west,align=left] {Lamé, Mathieu,\\ and \\ spheroidal wave\\ functions};
\draw (429.5700000000001, -82.79999999999998) rectangle (433.92000000000013,-84.89999999999998);
\draw(423.3700000000001, -84.99999999999999) node[anchor=north west,align=left] {Special functions\\ in \\ characteristic \(p\)\\ (gamma \\ functions, etc.)};
\draw (423.3700000000001, -84.99999999999999) rectangle (428.97000000000014,-87.59999999999998);
\draw(429.0700000000001, -84.99999999999999) node[anchor=north west,align=left] {Other functions\\ coming from \\ differential, \\ difference and \\ integral equations};
\draw (429.0700000000001, -84.99999999999999) rectangle (434.17000000000013,-87.59999999999998);
\draw(423.3700000000001, -87.69999999999999) node[anchor=north west,align=left] {Mittag-Leffler\\ functions\\ and \\ generalizations};
\draw (423.3700000000001, -87.69999999999999) rectangle (427.72000000000014,-89.79999999999998);
\draw(427.8200000000001, -87.69999999999999) node[anchor=north west,align=left] {Other functions\\ defined\\ by series\\ and integrals};
\draw (427.8200000000001, -87.69999999999999) rectangle (432.17000000000013,-89.79999999999998);
\draw(423.3700000000001, -89.89999999999998) node[anchor=north west,align=left] {Elliptic \\ functions \\ and integrals};
\draw (423.3700000000001, -89.89999999999998) rectangle (427.22000000000014,-91.49999999999997);
\draw(427.3200000000001, -89.89999999999998) node[anchor=north west,align=left] {Other\\ wave \\ functions};
\draw (427.3200000000001, -89.89999999999998) rectangle (430.17000000000013,-91.49999999999997);
\draw(422.3700000000001, -91.69999999999999) node[anchor=north west,align=left] {\large Computational aspects of special functions};
\draw (422.3700000000001, -91.69999999999999) rectangle (435.9900000000001,-95.89999999999999);
\draw(423.3700000000001, -92.69999999999999) node[anchor=north west,align=left] {Numerical \\ approximation\\ and \\ evaluation of \\ special functions};
\draw (423.3700000000001, -92.69999999999999) rectangle (428.22000000000014,-95.29999999999998);
\draw(428.3200000000001, -92.69999999999999) node[anchor=north west,align=left] {Symbolic \\ computation of \\ special functions\\ (Gosper and\\ Zeilberger\\ algorithms, etc.)};
\draw (428.3200000000001, -92.69999999999999) rectangle (433.17000000000013,-95.79999999999998);
\draw(422.3700000000001, -95.99999999999999) node[anchor=north west,align=left] {\large History of \\ special functions};
\draw (422.3700000000001, -95.99999999999999) rectangle (428.2400000000001,-97.09999999999998);
\draw(407.8700000000001, -98.39999999999999) node[anchor=north west,align=left] {\large Hypergeometric functions};
\draw (407.8700000000001, -98.39999999999999) rectangle (421.5200000000001,-119.3);
\draw(408.8700000000001, -99.39999999999999) node[anchor=north west,align=left] {Orthogonal \\ polynomials and functions\\ in several\\ variables expressible\\ in terms \\ of special functions\\ in one variable};
\draw (408.8700000000001, -99.39999999999999) rectangle (415.72000000000014,-102.99999999999999);
\draw(415.8200000000001, -99.39999999999999) node[anchor=north west,align=left] {Confluent \\ hypergeometric \\ functions, \\ Whittaker functions,\\ \({}_1F_1\)};
\draw (415.8200000000001, -99.39999999999999) rectangle (421.42000000000013,-101.99999999999999);
\draw(408.8700000000001, -103.1) node[anchor=north west,align=left] {Other \\ hypergeometric functions\\ and \\ integrals in several\\ variables};
\draw (408.8700000000001, -103.1) rectangle (415.47000000000014,-105.69999999999999);
\draw(415.5700000000001, -103.1) node[anchor=north west,align=left] {Hypergeometric \\ integrals and \\ functions defined \\ by them (\(E\), \\ \(G\), \(H\) and\\ \(I\) functions)};
\draw (415.5700000000001, -103.1) rectangle (420.67000000000013,-106.19999999999999);
\draw(408.8700000000001, -106.3) node[anchor=north west,align=left] {Orthogonal polynomials\\ and functions\\ of hypergeometric\\ type (Jacobi,\\ Laguerre, Hermite,\\ Askey scheme, etc.)};
\draw (408.8700000000001, -106.3) rectangle (414.97000000000014,-109.39999999999999);
\draw(415.0700000000001, -106.3) node[anchor=north west,align=left] {Bessel and\\ Airy functions,\\ cylinder\\ functions,\\ \({}_0F_1\)};
\draw (415.0700000000001, -106.3) rectangle (419.42000000000013,-108.89999999999999);
\draw(408.8700000000001, -109.5) node[anchor=north west,align=left] {Orthogonal \\ polynomials and\\ functions \\ associated with\\ root systems};
\draw (408.8700000000001, -109.5) rectangle (413.22000000000014,-112.1);
\draw(413.3200000000001, -109.5) node[anchor=north west,align=left] {Hypergeometric\\ functions \\ associated with\\ root systems};
\draw (413.3200000000001, -109.5) rectangle (417.67000000000013,-111.6);
\draw(417.7700000000001, -109.5) node[anchor=north west,align=left] {Appell, \\ Horn and \\ Lauricella\\ functions};
\draw (417.7700000000001, -109.5) rectangle (420.8700000000001,-111.6);
\draw(408.8700000000001, -112.19999999999999) node[anchor=north west,align=left] {Connections of\\ hypergeometric\\ functions \\ with groups and\\ algebras, and\\ related topics};
\draw (408.8700000000001, -112.19999999999999) rectangle (413.22000000000014,-115.29999999999998);
\draw(413.3200000000001, -112.19999999999999) node[anchor=north west,align=left] {Classical \\ hypergeometric\\ functions,\\ \({}_2F_1\)};
\draw (413.3200000000001, -112.19999999999999) rectangle (417.42000000000013,-114.29999999999998);
\draw(417.5200000000001, -112.19999999999999) node[anchor=north west,align=left] {Other special\\ orthogonal\\ polynomials\\ and functions};
\draw (417.5200000000001, -112.19999999999999) rectangle (421.3700000000001,-114.29999999999998);
\draw(408.8700000000001, -115.39999999999999) node[anchor=north west,align=left] {Generalized\\ hypergeometric\\ series,\\ \({}_pF_q\)};
\draw (408.8700000000001, -115.39999999999999) rectangle (412.97000000000014,-117.49999999999999);
\draw(413.0700000000001, -115.39999999999999) node[anchor=north west,align=left] {Elliptic \\ integrals as\\ hypergeometric\\ functions};
\draw (413.0700000000001, -115.39999999999999) rectangle (417.17000000000013,-117.49999999999999);
\draw(417.2700000000001, -115.39999999999999) node[anchor=north west,align=left] {Applications\\ of \\ hypergeometric\\ functions};
\draw (417.2700000000001, -115.39999999999999) rectangle (421.3700000000001,-117.49999999999999);
\draw(408.8700000000001, -117.6) node[anchor=north west,align=left] {Spherical\\ harmonics};
\draw (408.8700000000001, -117.6) rectangle (411.72000000000014,-119.19999999999999);
\draw(421.6200000000001, -98.39999999999999) node[anchor=north west,align=left] {\large Elementary classical functions};
\draw (421.6200000000001, -98.39999999999999) rectangle (432.7700000000001,-104.8);
\draw(422.6200000000001, -99.39999999999999) node[anchor=north west,align=left] {Incomplete beta\\ and gamma \\ functions (error \\ functions, probability\\ integral,\\ Fresnel integrals)};
\draw (422.6200000000001, -99.39999999999999) rectangle (428.72000000000014,-102.49999999999999);
\draw(428.8200000000001, -99.39999999999999) node[anchor=north west,align=left] {Exponential\\ and \\ trigonometric\\ functions};
\draw (428.8200000000001, -99.39999999999999) rectangle (432.67000000000013,-101.49999999999999);
\draw(422.6200000000001, -102.6) node[anchor=north west,align=left] {Gamma, \\ beta and\\ polygamma\\ functions};
\draw (422.6200000000001, -102.6) rectangle (425.47000000000014,-104.69999999999999);
\draw(425.5700000000001, -102.6) node[anchor=north west,align=left] {Higher \\ logarithm\\ functions};
\draw (425.5700000000001, -102.6) rectangle (428.42000000000013,-104.19999999999999);
\draw(436.3700000000001, -1) node[anchor=north west,align=left] {\LARGE K-Theory};
\draw (436.3700000000001, -1) rectangle (464.8200000000001,-38.6);
\draw(437.3700000000001, -2) node[anchor=north west,align=left] {\large Higher algebraic \(K\)-theory};
\draw (437.3700000000001, -2) rectangle (452.7700000000001,-10.1);
\draw(438.3700000000001, -3) node[anchor=north west,align=left] {Karoubi-Villamayor-Gersten\(K\)-theory};
\draw (438.3700000000001, -3) rectangle (448.47000000000014,-5.1);
\draw(448.5700000000001, -3) node[anchor=north west,align=left] {\(K\)-theory\\ and homology;\\ cyclic\\ homology \\ and cohomology};
\draw (448.5700000000001, -3) rectangle (452.67000000000013,-5.6);
\draw(438.3700000000001, -5.7) node[anchor=north west,align=left] {\(Q\)- and\\ plus-constructions};
\draw (438.3700000000001, -5.7) rectangle (443.47000000000014,-7.300000000000001);
\draw(443.5700000000001, -5.7) node[anchor=north west,align=left] {Negative\\ \(K\)-theory,\\ NK and Nil};
\draw (443.5700000000001, -5.7) rectangle (447.42000000000013,-7.800000000000001);
\draw(447.5200000000001, -5.7) node[anchor=north west,align=left] {Algebraic\\ \(K\)-theory\\ of spaces};
\draw (447.5200000000001, -5.7) rectangle (451.1200000000001,-7.300000000000001);
\draw(438.3700000000001, -7.9) node[anchor=north west,align=left] {Higher \\ symbols,\\ Milnor \\ \(K\)-theory};
\draw (438.3700000000001, -7.9) rectangle (441.97000000000014,-10.0);
\draw(442.0700000000001, -7.9) node[anchor=north west,align=left] {Computations\\ of higher\\ \(K\)-theory\\ of rings};
\draw (442.0700000000001, -7.9) rectangle (445.67000000000013,-10.0);
\draw(445.7700000000001, -7.9) node[anchor=north west,align=left] {Symmetric\\ monoidal\\ categories};
\draw (445.7700000000001, -7.9) rectangle (448.8700000000001,-9.5);
\draw(452.8700000000001, -2) node[anchor=north west,align=left] {\large \(K\)-theory and operator algebras};
\draw (452.8700000000001, -2) rectangle (464.72000000000014,-6.9);
\draw(453.8700000000001, -3) node[anchor=north west,align=left] {Kasparov \\ theory \\ (\(KK\)-theory)};
\draw (453.8700000000001, -3) rectangle (458.22000000000014,-4.6);
\draw(458.3200000000001, -3) node[anchor=north west,align=left] {Ext and\\ \(K\)-homology};
\draw (458.3200000000001, -3) rectangle (462.42000000000013,-4.6);
\draw(462.5200000000001, -3) node[anchor=north west,align=left] {Index\\ theory};
\draw (462.5200000000001, -3) rectangle (464.6200000000001,-4.1);
\draw(453.8700000000001, -4.7) node[anchor=north west,align=left] {\(K_0\)\\ as an \\ ordered \\ group, traces};
\draw (453.8700000000001, -4.7) rectangle (457.72000000000014,-6.800000000000001);
\draw(452.8700000000001, -7.0) node[anchor=north west,align=left] {\large Computational methods\\ for problems \\ pertaining to \(K\)-theory};
\draw (452.8700000000001, -7.0) rectangle (461.53000000000014,-8.6);
\draw(452.8700000000001, -8.7) node[anchor=north west,align=left] {\large History of\\ \(K\)-theory};
\draw (452.8700000000001, -8.7) rectangle (457.1900000000001,-9.799999999999999);
\draw(437.3700000000001, -10.2) node[anchor=north west,align=left] {\large Miscellaneous applications of \(K\)-theory};
\draw (437.3700000000001, -10.2) rectangle (450.9900000000001,-13.399999999999999);
\draw(438.3700000000001, -11.2) node[anchor=north west,align=left] {Miscellaneous\\ applications of\\ \(K\)-theory};
\draw (438.3700000000001, -11.2) rectangle (442.72000000000014,-13.299999999999999);
\draw(451.09000000000015, -10.2) node[anchor=north west,align=left] {\large Grothendieck groups and \(K_0\)};
\draw (451.09000000000015, -10.2) rectangle (461.9900000000001,-16.1);
\draw(452.09000000000015, -11.2) node[anchor=north west,align=left] {Frobenius\\ induction,\\ Burnside\\ and \\ representation rings};
\draw (452.09000000000015, -11.2) rectangle (457.69000000000017,-13.799999999999999);
\draw(457.79000000000013, -11.2) node[anchor=north west,align=left] {Stability\\ for projective\\ modules};
\draw (457.79000000000013, -11.2) rectangle (461.89000000000016,-12.799999999999999);
\draw(452.09000000000015, -13.899999999999999) node[anchor=north west,align=left] {Efficient\\ generation\\ of modules};
\draw (452.09000000000015, -13.899999999999999) rectangle (455.19000000000017,-15.499999999999998);
\draw(455.29000000000013, -13.899999999999999) node[anchor=north west,align=left] {\(K_0\)\\ of group\\ rings \\ and orders};
\draw (455.29000000000013, -13.899999999999999) rectangle (458.39000000000016,-15.999999999999998);
\draw(458.4900000000001, -13.899999999999999) node[anchor=north west,align=left] {\(K_0\)\\ of other\\ rings};
\draw (458.4900000000001, -13.899999999999999) rectangle (461.09000000000015,-15.499999999999998);
\draw(437.3700000000001, -16.2) node[anchor=north west,align=left] {\large Whitehead groups and \(K_1\)};
\draw (437.3700000000001, -16.2) rectangle (447.97000000000014,-21.1);
\draw(438.3700000000001, -17.2) node[anchor=north west,align=left] {Stable\\ range \\ conditions};
\draw (438.3700000000001, -17.2) rectangle (441.47000000000014,-18.8);
\draw(441.5700000000001, -17.2) node[anchor=north west,align=left] {Stability\\ for linear\\ groups};
\draw (441.5700000000001, -17.2) rectangle (444.67000000000013,-18.8);
\draw(444.7700000000001, -17.2) node[anchor=north west,align=left] {\(K_1\)\\ of group\\ rings \\ and orders};
\draw (444.7700000000001, -17.2) rectangle (447.8700000000001,-19.3);
\draw(438.3700000000001, -19.4) node[anchor=north west,align=left] {Congruence\\ subgroup\\ problems};
\draw (438.3700000000001, -19.4) rectangle (441.47000000000014,-21.0);
\draw(448.0700000000001, -16.2) node[anchor=north west,align=left] {\large \(K\)-theory in number theory};
\draw (448.0700000000001, -16.2) rectangle (458.4700000000001,-23.1);
\draw(449.0700000000001, -17.2) node[anchor=north west,align=left] {Étale cohomology,\\ higher \\ regulators, zeta\\ and \(L\)-functions\\ (\(K\)-theoretic aspects)};
\draw (449.0700000000001, -17.2) rectangle (455.92000000000013,-20.3);
\draw(449.0700000000001, -20.4) node[anchor=north west,align=left] {Generalized\\ class field\\ theory \\ (\(K\)-theoretic\\ aspects)};
\draw (449.0700000000001, -20.4) rectangle (453.67000000000013,-23.0);
\draw(453.7700000000001, -20.4) node[anchor=north west,align=left] {Symbols and\\ arithmetic \\ (\(K\)-theoretic\\ aspects)};
\draw (453.7700000000001, -20.4) rectangle (458.3700000000001,-22.5);
\draw(437.3700000000001, -23.2) node[anchor=north west,align=left] {\large \(K\)-theory in geometry};
\draw (437.3700000000001, -23.2) rectangle (447.5200000000001,-29.1);
\draw(438.3700000000001, -24.2) node[anchor=north west,align=left] {Relations of\\ \(K\)-theory\\ with \\ cohomology theories};
\draw (438.3700000000001, -24.2) rectangle (443.72000000000014,-26.3);
\draw(443.8200000000001, -24.2) node[anchor=north west,align=left] {\(K\)-theory\\ of schemes};
\draw (443.8200000000001, -24.2) rectangle (447.42000000000013,-25.8);
\draw(438.3700000000001, -26.4) node[anchor=north west,align=left] {Algebraic \\ cycles and motivic\\ cohomology\\ (\(K\)-theoretic\\ aspects)};
\draw (438.3700000000001, -26.4) rectangle (443.47000000000014,-29.0);
\draw(447.6200000000001, -23.2) node[anchor=north west,align=left] {\large Obstructions from topology};
\draw (447.6200000000001, -23.2) rectangle (457.2700000000001,-29.1);
\draw(448.6200000000001, -24.2) node[anchor=north west,align=left] {Surgery \\ obstructions \\ (\(K\)-theoretic\\ aspects)};
\draw (448.6200000000001, -24.2) rectangle (453.22000000000014,-26.3);
\draw(453.3200000000001, -24.2) node[anchor=north west,align=left] {Whitehead\\ (and related)\\ torsion};
\draw (453.3200000000001, -24.2) rectangle (457.17000000000013,-25.8);
\draw(448.6200000000001, -26.4) node[anchor=north west,align=left] {Obstructions\\ to group\\ actions \\ (\(K\)-theoretic\\ aspects)};
\draw (448.6200000000001, -26.4) rectangle (453.22000000000014,-29.0);
\draw(453.3200000000001, -26.4) node[anchor=north west,align=left] {Finiteness\\ and other \\ obstructions\\ in \(K_0\)};
\draw (453.3200000000001, -26.4) rectangle (456.92000000000013,-28.5);
\draw(437.3700000000001, -29.2) node[anchor=north west,align=left] {\large Steinberg groups and \(K_2\)};
\draw (437.3700000000001, -29.2) rectangle (446.6500000000001,-34.6);
\draw(438.3700000000001, -30.2) node[anchor=north west,align=left] {Symbols, \\ presentations\\ and stability\\ of \(K_2\)};
\draw (438.3700000000001, -30.2) rectangle (442.22000000000014,-32.3);
\draw(442.3200000000001, -30.2) node[anchor=north west,align=left] {\(K_2\) \\ and the \\ Brauer group};
\draw (442.3200000000001, -30.2) rectangle (445.92000000000013,-31.8);
\draw(438.3700000000001, -32.4) node[anchor=north west,align=left] {Central \\ extensions \\ and Schur \\ multipliers};
\draw (438.3700000000001, -32.4) rectangle (441.72000000000014,-34.5);
\draw(441.8200000000001, -32.4) node[anchor=north west,align=left] {Excision\\ for\\ \(K_2\)};
\draw (441.8200000000001, -32.4) rectangle (444.42000000000013,-34.0);
\draw(446.7500000000001, -29.2) node[anchor=north west,align=left] {\large Topological \(K\)-theory};
\draw (446.7500000000001, -29.2) rectangle (455.9000000000001,-38.5);
\draw(447.7500000000001, -30.2) node[anchor=north west,align=left] {\(J\)-homomorphism,\\ Adams \\ operations};
\draw (447.7500000000001, -30.2) rectangle (453.10000000000014,-32.3);
\draw(447.7500000000001, -32.4) node[anchor=north west,align=left] {Geometric \\ applications \\ of topological\\ \(K\)-theory};
\draw (447.7500000000001, -32.4) rectangle (451.85000000000014,-34.5);
\draw(451.9500000000001, -32.4) node[anchor=north west,align=left] {Connective\\ \(K\)-theory,\\ cobordism};
\draw (451.9500000000001, -32.4) rectangle (455.8000000000001,-34.5);
\draw(447.7500000000001, -34.6) node[anchor=north west,align=left] {Twisted \\ \(K\)-theory;\\ differential\\ \(K\)-theory};
\draw (447.7500000000001, -34.6) rectangle (451.60000000000014,-36.7);
\draw(451.7000000000001, -34.6) node[anchor=north west,align=left] {Riemann-Roch\\ theorems,\\ Chern\\ characters};
\draw (451.7000000000001, -34.6) rectangle (455.3000000000001,-36.7);
\draw(447.7500000000001, -36.8) node[anchor=north west,align=left] {Equivariant\\ \(K\)-theory};
\draw (447.7500000000001, -36.8) rectangle (451.35000000000014,-38.4);
\draw(456.0000000000001, -29.2) node[anchor=north west,align=left] {\large \(K\)-theory of forms};
\draw (456.0000000000001, -29.2) rectangle (464.6500000000001,-34.6);
\draw(457.0000000000001, -30.2) node[anchor=north west,align=left] {Hermitian \\ \(K\)-theory,\\ relations \\ with \(K\)-theory\\ of rings};
\draw (457.0000000000001, -30.2) rectangle (461.85000000000014,-32.8);
\draw(461.9500000000001, -30.2) node[anchor=north west,align=left] {Witt \\ groups\\ of rings};
\draw (461.9500000000001, -30.2) rectangle (464.5500000000001,-31.8);
\draw(457.0000000000001, -32.9) node[anchor=north west,align=left] {Stability\\ for quadratic\\ modules};
\draw (457.0000000000001, -32.9) rectangle (460.85000000000014,-34.5);
\draw(460.9500000000001, -32.9) node[anchor=north west,align=left] {\(L\)-theory\\ of \\ group rings};
\draw (460.9500000000001, -32.9) rectangle (464.5500000000001,-34.5);
\draw(436.3700000000001, -38.7) node[anchor=north west,align=left] {\LARGE Field theory and polynomials};
\draw (436.3700000000001, -38.7) rectangle (464.0100000000001,-65.60000000000001);
\draw(437.3700000000001, -39.7) node[anchor=north west,align=left] {\large Connections between field theory and logic};
\draw (437.3700000000001, -39.7) rectangle (452.67000000000013,-42.900000000000006);
\draw(438.3700000000001, -40.7) node[anchor=north west,align=left] {Ultraproducts\\ and \\ field theory};
\draw (438.3700000000001, -40.7) rectangle (442.22000000000014,-42.300000000000004);
\draw(442.3200000000001, -40.7) node[anchor=north west,align=left] {Decidability\\ and \\ field theory};
\draw (442.3200000000001, -40.7) rectangle (445.92000000000013,-42.300000000000004);
\draw(446.0200000000001, -40.7) node[anchor=north west,align=left] {Nonstandard\\ arithmetic\\ and\\ field theory};
\draw (446.0200000000001, -40.7) rectangle (449.6200000000001,-42.800000000000004);
\draw(449.72000000000014, -40.7) node[anchor=north west,align=left] {Model \\ theory \\ of fields};
\draw (449.72000000000014, -40.7) rectangle (452.57000000000016,-42.300000000000004);
\draw(452.7700000000001, -39.7) node[anchor=north west,align=left] {\large Homological methods (field theory)};
\draw (452.7700000000001, -39.7) rectangle (463.9100000000001,-42.900000000000006);
\draw(453.7700000000001, -40.7) node[anchor=north west,align=left] {Cohomological\\ dimension\\ of fields};
\draw (453.7700000000001, -40.7) rectangle (457.6200000000001,-42.800000000000004);
\draw(457.7200000000001, -40.7) node[anchor=north west,align=left] {Galois\\ cohomology};
\draw (457.7200000000001, -40.7) rectangle (460.8200000000001,-42.300000000000004);
\draw(437.3700000000001, -43.0) node[anchor=north west,align=left] {\large General field theory};
\draw (437.3700000000001, -43.0) rectangle (449.5200000000001,-50.6);
\draw(438.3700000000001, -44.0) node[anchor=north west,align=left] {Polynomials \\ in general \\ fields (irreducibility,\\ etc.)};
\draw (438.3700000000001, -44.0) rectangle (444.72000000000014,-46.1);
\draw(444.8200000000001, -44.0) node[anchor=north west,align=left] {Finite \\ fields \\ (field-theoretic\\ aspects)};
\draw (444.8200000000001, -44.0) rectangle (449.42000000000013,-46.1);
\draw(438.3700000000001, -46.2) node[anchor=north west,align=left] {Hilbertian \\ fields; Hilbert’s\\ irreducibility theorem};
\draw (438.3700000000001, -46.2) rectangle (444.47000000000014,-48.300000000000004);
\draw(444.5700000000001, -46.2) node[anchor=north west,align=left] {Skew fields,\\ division\\ rings};
\draw (444.5700000000001, -46.2) rectangle (448.17000000000013,-47.800000000000004);
\draw(438.3700000000001, -48.4) node[anchor=north west,align=left] {Special \\ polynomials\\ in general\\ fields};
\draw (438.3700000000001, -48.4) rectangle (441.72000000000014,-50.5);
\draw(441.8200000000001, -48.4) node[anchor=north west,align=left] {Equations\\ in general\\ fields};
\draw (441.8200000000001, -48.4) rectangle (444.92000000000013,-50.0);
\draw(445.0200000000001, -48.4) node[anchor=north west,align=left] {Field \\ arithmetic};
\draw (445.0200000000001, -48.4) rectangle (448.1200000000001,-49.5);
\draw(449.6200000000001, -43.0) node[anchor=north west,align=left] {\large Differential and difference algebra};
\draw (449.6200000000001, -43.0) rectangle (461.72000000000014,-47.9);
\draw(450.6200000000001, -44.0) node[anchor=north west,align=left] {Differential\\ algebra};
\draw (450.6200000000001, -44.0) rectangle (454.22000000000014,-45.6);
\draw(454.3200000000001, -44.0) node[anchor=north west,align=left] {Abstract \\ differential\\ equations};
\draw (454.3200000000001, -44.0) rectangle (457.92000000000013,-45.6);
\draw(458.0200000000001, -44.0) node[anchor=north west,align=left] {\(p\)-adic\\ differential\\ equations};
\draw (458.0200000000001, -44.0) rectangle (461.6200000000001,-46.1);
\draw(450.6200000000001, -46.2) node[anchor=north west,align=left] {Difference\\ algebra};
\draw (450.6200000000001, -46.2) rectangle (453.72000000000014,-47.800000000000004);
\draw(449.6200000000001, -48.0) node[anchor=north west,align=left] {\large Computational methods\\ for problems \\ pertaining to field theory};
\draw (449.6200000000001, -48.0) rectangle (458.28000000000014,-49.6);
\draw(437.3700000000001, -50.7) node[anchor=north west,align=left] {\large Real and complex fields};
\draw (437.3700000000001, -50.7) rectangle (447.7700000000001,-57.6);
\draw(438.3700000000001, -51.7) node[anchor=north west,align=left] {Polynomials in\\ real and complex\\ fields: location\\ of zeros \\ (algebraic theorems)};
\draw (438.3700000000001, -51.7) rectangle (443.97000000000014,-54.300000000000004);
\draw(438.3700000000001, -54.400000000000006) node[anchor=north west,align=left] {Fields related\\ with sums of\\ squares (formally\\ real fields,\\ Pythagorean\\ fields, etc.)};
\draw (438.3700000000001, -54.400000000000006) rectangle (443.22000000000014,-57.50000000000001);
\draw(443.3200000000001, -54.400000000000006) node[anchor=north west,align=left] {Polynomials \\ in real and \\ complex fields:\\ factorization};
\draw (443.3200000000001, -54.400000000000006) rectangle (447.67000000000013,-56.50000000000001);
\draw(447.8700000000001, -50.7) node[anchor=north west,align=left] {\large Generalizations of fields};
\draw (447.8700000000001, -50.7) rectangle (456.22000000000014,-52.900000000000006);
\draw(448.8700000000001, -51.7) node[anchor=north west,align=left] {Near-fields};
\draw (448.8700000000001, -51.7) rectangle (452.22000000000014,-52.800000000000004);
\draw(452.3200000000001, -51.7) node[anchor=north west,align=left] {Semifields};
\draw (452.3200000000001, -51.7) rectangle (455.42000000000013,-52.800000000000004);
\draw(447.8700000000001, -53.0) node[anchor=north west,align=left] {\large History of\\ field theory};
\draw (447.8700000000001, -53.0) rectangle (452.1900000000001,-54.1);
\draw(437.3700000000001, -57.7) node[anchor=north west,align=left] {\large Field extensions};
\draw (437.3700000000001, -57.7) rectangle (445.5200000000001,-64.3);
\draw(438.3700000000001, -58.7) node[anchor=north west,align=left] {Transcendental\\ field\\ extensions};
\draw (438.3700000000001, -58.7) rectangle (442.47000000000014,-60.300000000000004);
\draw(442.5700000000001, -58.7) node[anchor=north west,align=left] {Inverse\\ Galois\\ theory};
\draw (442.5700000000001, -58.7) rectangle (444.92000000000013,-60.300000000000004);
\draw(438.3700000000001, -60.400000000000006) node[anchor=north west,align=left] {Separable\\ extensions,\\ Galois theory};
\draw (438.3700000000001, -60.400000000000006) rectangle (442.22000000000014,-62.50000000000001);
\draw(442.3200000000001, -60.400000000000006) node[anchor=north west,align=left] {Algebraic\\ field\\ extensions};
\draw (442.3200000000001, -60.400000000000006) rectangle (445.42000000000013,-62.00000000000001);
\draw(438.3700000000001, -62.6) node[anchor=north west,align=left] {Inseparable\\ field\\ extensions};
\draw (438.3700000000001, -62.6) rectangle (441.72000000000014,-64.2);
\draw(445.6200000000001, -57.7) node[anchor=north west,align=left] {\large Topological fields};
\draw (445.6200000000001, -57.7) rectangle (453.7700000000001,-65.5);
\draw(446.6200000000001, -58.7) node[anchor=north west,align=left] {Non-Archimedean\\ valued fields};
\draw (446.6200000000001, -58.7) rectangle (450.97000000000014,-60.300000000000004);
\draw(451.0700000000001, -58.7) node[anchor=north west,align=left] {Ordered\\ fields};
\draw (451.0700000000001, -58.7) rectangle (453.42000000000013,-59.800000000000004);
\draw(446.6200000000001, -60.400000000000006) node[anchor=north west,align=left] {Krasner-Tate\\ algebras};
\draw (446.6200000000001, -60.400000000000006) rectangle (450.22000000000014,-62.00000000000001);
\draw(450.3200000000001, -60.400000000000006) node[anchor=north west,align=left] {Topological\\ semifields};
\draw (450.3200000000001, -60.400000000000006) rectangle (453.67000000000013,-62.00000000000001);
\draw(446.6200000000001, -62.1) node[anchor=north west,align=left] {Formally\\ \(p\)-adic\\ fields};
\draw (446.6200000000001, -62.1) rectangle (449.72000000000014,-63.7);
\draw(449.8200000000001, -62.1) node[anchor=north west,align=left] {General\\ valuation\\ theory\\ for fields};
\draw (449.8200000000001, -62.1) rectangle (452.92000000000013,-64.2);
\draw(446.6200000000001, -64.3) node[anchor=north west,align=left] {Normed\\ fields};
\draw (446.6200000000001, -64.3) rectangle (448.72000000000014,-65.39999999999999);
\draw(448.8200000000001, -64.3) node[anchor=north west,align=left] {Valued\\ fields};
\draw (448.8200000000001, -64.3) rectangle (450.92000000000013,-65.39999999999999);
\draw(436.3700000000001, -65.7) node[anchor=north west,align=left] {\LARGE Potential theory};
\draw (436.3700000000001, -65.7) rectangle (463.0700000000001,-96.80000000000001);
\draw(437.3700000000001, -66.7) node[anchor=north west,align=left] {\large Potential theory on fractals and metric spaces};
\draw (437.3700000000001, -66.7) rectangle (452.23000000000013,-69.9);
\draw(438.3700000000001, -67.7) node[anchor=north west,align=left] {Potential \\ theory on \\ fractals and\\ metric spaces};
\draw (438.3700000000001, -67.7) rectangle (442.22000000000014,-69.8);
\draw(452.3300000000001, -66.7) node[anchor=north west,align=left] {\large Axiomatic potential theory};
\draw (452.3300000000001, -66.7) rectangle (460.9900000000001,-69.4);
\draw(453.3300000000001, -67.7) node[anchor=north west,align=left] {Axiomatic\\ potential\\ theory};
\draw (453.3300000000001, -67.7) rectangle (456.1800000000001,-69.3);
\draw(437.3700000000001, -70.0) node[anchor=north west,align=left] {\large Generalizations of potential theory};
\draw (437.3700000000001, -70.0) rectangle (450.47000000000014,-79.1);
\draw(438.3700000000001, -71.0) node[anchor=north west,align=left] {Pluriharmonic\\ and \\ plurisubharmonic\\ functions};
\draw (438.3700000000001, -71.0) rectangle (442.97000000000014,-73.1);
\draw(443.0700000000001, -71.0) node[anchor=north west,align=left] {Harmonic, \\ subharmonic, \\ superharmonic\\ functions \\ on other spaces};
\draw (443.0700000000001, -71.0) rectangle (447.42000000000013,-73.6);
\draw(447.5200000000001, -71.0) node[anchor=north west,align=left] {Discrete\\ potential\\ theory};
\draw (447.5200000000001, -71.0) rectangle (450.3700000000001,-72.6);
\draw(438.3700000000001, -73.7) node[anchor=north west,align=left] {Fine potential\\ theory; \\ fine properties\\ of sets \\ and functions};
\draw (438.3700000000001, -73.7) rectangle (442.72000000000014,-76.3);
\draw(442.8200000000001, -73.7) node[anchor=north west,align=left] {Other \\ generalizations\\ (nonlinear\\ potential \\ theory, etc.)};
\draw (442.8200000000001, -73.7) rectangle (447.17000000000013,-76.3);
\draw(447.2700000000001, -73.7) node[anchor=north west,align=left] {Dirichlet\\ forms};
\draw (447.2700000000001, -73.7) rectangle (450.1200000000001,-74.8);
\draw(438.3700000000001, -76.4) node[anchor=north west,align=left] {Potentials\\ and \\ capacities on \\ other spaces};
\draw (438.3700000000001, -76.4) rectangle (442.47000000000014,-78.5);
\draw(442.5700000000001, -76.4) node[anchor=north west,align=left] {Potential \\ theory on \\ Riemannian \\ manifolds and\\ other spaces};
\draw (442.5700000000001, -76.4) rectangle (446.42000000000013,-79.0);
\draw(446.5200000000001, -76.4) node[anchor=north west,align=left] {Martin\\ boundary\\ theory};
\draw (446.5200000000001, -76.4) rectangle (449.1200000000001,-78.0);
\draw(450.5700000000001, -70.0) node[anchor=north west,align=left] {\large Two-dimensional potential theory};
\draw (450.5700000000001, -70.0) rectangle (462.9700000000001,-83.3);
\draw(451.5700000000001, -71.0) node[anchor=north west,align=left] {Connections of\\ harmonic functions\\ with \\ differential equations\\ in two dimensions};
\draw (451.5700000000001, -71.0) rectangle (457.67000000000013,-73.6);
\draw(457.7700000000001, -71.0) node[anchor=north west,align=left] {Potentials and \\ capacity, harmonic\\ measure, extremal\\ length and \\ related notions \\ in two dimensions};
\draw (457.7700000000001, -71.0) rectangle (462.8700000000001,-74.1);
\draw(451.5700000000001, -74.2) node[anchor=north west,align=left] {Boundary value\\ and inverse \\ problems for harmonic\\ functions \\ in two dimensions};
\draw (451.5700000000001, -74.2) rectangle (457.42000000000013,-76.8);
\draw(457.5200000000001, -74.2) node[anchor=north west,align=left] {Integral \\ representations, \\ integral operators,\\ integral equations\\ methods in\\ two dimensions};
\draw (457.5200000000001, -74.2) rectangle (462.8700000000001,-77.3);
\draw(451.5700000000001, -77.4) node[anchor=north west,align=left] {Biharmonic, \\ polyharmonic \\ functions and \\ equations, Poisson’s\\ equation \\ in two dimensions};
\draw (451.5700000000001, -77.4) rectangle (457.17000000000013,-80.5);
\draw(457.2700000000001, -77.4) node[anchor=north west,align=left] {Boundary behavior\\ (theorems\\ of Fatou type,\\ etc.) of \\ harmonic functions\\ in two dimensions};
\draw (457.2700000000001, -77.4) rectangle (462.3700000000001,-80.5);
\draw(451.5700000000001, -80.6) node[anchor=north west,align=left] {Harmonic, \\ subharmonic, \\ superharmonic\\ functions \\ in two dimensions};
\draw (451.5700000000001, -80.6) rectangle (456.42000000000013,-83.19999999999999);
\draw(437.3700000000001, -79.2) node[anchor=north west,align=left] {\large Computational methods\\ for problems pertaining\\ to potential theory};
\draw (437.3700000000001, -79.2) rectangle (445.10000000000014,-80.8);
\draw(437.3700000000001, -80.9) node[anchor=north west,align=left] {\large History of \\ potential theory};
\draw (437.3700000000001, -80.9) rectangle (442.9300000000001,-82.0);
\draw(437.3700000000001, -83.4) node[anchor=north west,align=left] {\large Higher-dimensional potential theory};
\draw (437.3700000000001, -83.4) rectangle (449.7700000000001,-96.7);
\draw(438.3700000000001, -84.4) node[anchor=north west,align=left] {Boundary value\\ and inverse \\ problems for harmonic\\ functions \\ in higher dimensions};
\draw (438.3700000000001, -84.4) rectangle (444.22000000000014,-87.0);
\draw(444.3200000000001, -84.4) node[anchor=north west,align=left] {Integral \\ representations, \\ integral operators,\\ integral equations\\ methods in\\ higher dimensions};
\draw (444.3200000000001, -84.4) rectangle (449.67000000000013,-87.5);
\draw(438.3700000000001, -87.60000000000001) node[anchor=north west,align=left] {Potentials and\\ capacities,\\ extremal \\ length and related\\ notions in\\ higher dimensions};
\draw (438.3700000000001, -87.60000000000001) rectangle (443.47000000000014,-90.7);
\draw(443.5700000000001, -87.60000000000001) node[anchor=north west,align=left] {Boundary \\ behavior of \\ harmonic functions\\ in higher\\ dimensions};
\draw (443.5700000000001, -87.60000000000001) rectangle (448.67000000000013,-90.2);
\draw(438.3700000000001, -90.80000000000001) node[anchor=north west,align=left] {Harmonic, \\ subharmonic, \\ superharmonic \\ functions in \\ higher dimensions};
\draw (438.3700000000001, -90.80000000000001) rectangle (443.22000000000014,-93.4);
\draw(443.3200000000001, -90.80000000000001) node[anchor=north west,align=left] {Biharmonic and\\ polyharmonic\\ equations and\\ functions in \\ higher dimensions};
\draw (443.3200000000001, -90.80000000000001) rectangle (448.17000000000013,-93.4);
\draw(438.3700000000001, -93.5) node[anchor=north west,align=left] {Connections \\ of harmonic \\ functions with\\ differential\\ equations in\\ higher dimensions};
\draw (438.3700000000001, -93.5) rectangle (443.22000000000014,-96.6);
\draw(436.3700000000001, -96.9) node[anchor=north west,align=left] {\LARGE Combinatorics};
\draw (436.3700000000001, -96.9) rectangle (462.3200000000001,-145.9);
\draw(437.3700000000001, -97.9) node[anchor=north west,align=left] {\large Graph theory};
\draw (437.3700000000001, -97.9) rectangle (450.72000000000014,-133.7);
\draw(438.3700000000001, -98.9) node[anchor=north west,align=left] {Isomorphism \\ problems in graph \\ theory (reconstruction\\ conjecture,\\ etc.) and \\ homomorphisms (subgraph\\ embedding, etc.)};
\draw (438.3700000000001, -98.9) rectangle (444.72000000000014,-102.5);
\draw(444.8200000000001, -98.9) node[anchor=north west,align=left] {Graphs and\\ abstract \\ algebra (groups,\\ rings,\\ fields, etc.)};
\draw (444.8200000000001, -98.9) rectangle (449.42000000000013,-101.5);
\draw(444.8200000000001, -101.60000000000001) node[anchor=north west,align=left] {Trees};
\draw (444.8200000000001, -101.60000000000001) rectangle (446.67000000000013,-102.2);
\draw(438.3700000000001, -102.60000000000001) node[anchor=north west,align=left] {Graph representations\\ (geometric\\ and \\ intersection \\ representations, etc.)};
\draw (438.3700000000001, -102.60000000000001) rectangle (444.47000000000014,-105.2);
\draw(444.5700000000001, -102.60000000000001) node[anchor=north west,align=left] {Games on \\ graphs \\ (graph-theoretic\\ aspects)};
\draw (444.5700000000001, -102.60000000000001) rectangle (449.17000000000013,-104.7);
\draw(438.3700000000001, -105.30000000000001) node[anchor=north west,align=left] {Edge subsets with\\ special properties\\ (factorization,\\ matching, \\ partitioning, covering\\ and packing, etc.)};
\draw (438.3700000000001, -105.30000000000001) rectangle (444.47000000000014,-108.4);
\draw(444.5700000000001, -105.30000000000001) node[anchor=north west,align=left] {Structural \\ characterization\\ of families\\ of graphs};
\draw (444.5700000000001, -105.30000000000001) rectangle (449.17000000000013,-107.4);
\draw(438.3700000000001, -108.5) node[anchor=north west,align=left] {Vertex subsets\\ with special\\ properties \\ (dominating sets,\\ independent \\ sets, cliques, etc.)};
\draw (438.3700000000001, -108.5) rectangle (443.97000000000014,-111.6);
\draw(444.0700000000001, -108.5) node[anchor=north west,align=left] {Graph \\ operations (line\\ graphs, \\ products, etc.)};
\draw (444.0700000000001, -108.5) rectangle (448.67000000000013,-110.6);
\draw(438.3700000000001, -111.7) node[anchor=north west,align=left] {Graph labelling\\ (graceful\\ graphs, \\ bandwidth, etc.)};
\draw (438.3700000000001, -111.7) rectangle (442.97000000000014,-113.8);
\draw(443.0700000000001, -111.7) node[anchor=north west,align=left] {Random \\ graphs \\ (graph-theoretic\\ aspects)};
\draw (443.0700000000001, -111.7) rectangle (447.67000000000013,-113.8);
\draw(447.7700000000001, -111.7) node[anchor=north west,align=left] {Graph\\ minors};
\draw (447.7700000000001, -111.7) rectangle (449.8700000000001,-112.8);
\draw(438.3700000000001, -113.9) node[anchor=north west,align=left] {Small world\\ graphs, complex\\ networks\\ (graph-theoretic\\ aspects)};
\draw (438.3700000000001, -113.9) rectangle (442.97000000000014,-116.5);
\draw(443.0700000000001, -113.9) node[anchor=north west,align=left] {Graph \\ algorithms \\ (graph-theoretic\\ aspects)};
\draw (443.0700000000001, -113.9) rectangle (447.67000000000013,-116.0);
\draw(438.3700000000001, -116.60000000000001) node[anchor=north west,align=left] {Graphical \\ indices (Wiener\\ index, Zagreb\\ index, Randić\\ index, etc.)};
\draw (438.3700000000001, -116.60000000000001) rectangle (442.72000000000014,-119.2);
\draw(442.8200000000001, -116.60000000000001) node[anchor=north west,align=left] {Planar graphs;\\ geometric\\ and topological\\ aspects\\ of graph theory};
\draw (442.8200000000001, -116.60000000000001) rectangle (447.17000000000013,-119.2);
\draw(447.2700000000001, -116.60000000000001) node[anchor=north west,align=left] {Distance\\ in\\ graphs};
\draw (447.2700000000001, -116.60000000000001) rectangle (449.8700000000001,-118.2);
\draw(438.3700000000001, -119.30000000000001) node[anchor=north west,align=left] {Eulerian \\ and Hamiltonian\\ graphs};
\draw (438.3700000000001, -119.30000000000001) rectangle (442.72000000000014,-120.9);
\draw(442.8200000000001, -119.30000000000001) node[anchor=north west,align=left] {Graphs and\\ linear algebra\\ (matrices,\\ eigenvalues,\\ etc.)};
\draw (442.8200000000001, -119.30000000000001) rectangle (446.92000000000013,-121.9);
\draw(447.0200000000001, -119.30000000000001) node[anchor=north west,align=left] {Flows \\ in graphs};
\draw (447.0200000000001, -119.30000000000001) rectangle (449.8700000000001,-120.4);
\draw(447.0200000000001, -120.5) node[anchor=north west,align=left] {Connectivity};
\draw (447.0200000000001, -120.5) rectangle (450.6200000000001,-121.6);
\draw(438.3700000000001, -122.0) node[anchor=north west,align=left] {Graph designs\\ and \\ isomorphic \\ decomposition};
\draw (438.3700000000001, -122.0) rectangle (442.22000000000014,-124.1);
\draw(442.3200000000001, -122.0) node[anchor=north west,align=left] {Fractional\\ graph \\ theory, fuzzy\\ graph theory};
\draw (442.3200000000001, -122.0) rectangle (446.17000000000013,-124.1);
\draw(446.2700000000001, -122.0) node[anchor=north west,align=left] {Signed \\ and weighted\\ graphs};
\draw (446.2700000000001, -122.0) rectangle (449.8700000000001,-123.6);
\draw(438.3700000000001, -124.2) node[anchor=north west,align=left] {Applications\\ of \\ graph theory};
\draw (438.3700000000001, -124.2) rectangle (441.97000000000014,-125.8);
\draw(442.0700000000001, -124.2) node[anchor=north west,align=left] {Coloring\\ of graphs\\ and \\ hypergraphs};
\draw (442.0700000000001, -124.2) rectangle (445.42000000000013,-126.3);
\draw(445.5200000000001, -124.2) node[anchor=north west,align=left] {Directed\\ graphs \\ (digraphs),\\ tournaments};
\draw (445.5200000000001, -124.2) rectangle (448.8700000000001,-126.3);
\draw(438.3700000000001, -126.4) node[anchor=north west,align=left] {Enumeration\\ in graph\\ theory};
\draw (438.3700000000001, -126.4) rectangle (441.72000000000014,-128.0);
\draw(441.8200000000001, -126.4) node[anchor=north west,align=left] {Graph\\ polynomials};
\draw (441.8200000000001, -126.4) rectangle (445.17000000000013,-128.0);
\draw(445.2700000000001, -126.4) node[anchor=north west,align=left] {Density\\ (toughness,\\ etc.)};
\draw (445.2700000000001, -126.4) rectangle (448.6200000000001,-128.0);
\draw(438.3700000000001, -128.1) node[anchor=north west,align=left] {Generalized\\ Ramsey\\ theory};
\draw (438.3700000000001, -128.1) rectangle (441.72000000000014,-129.7);
\draw(441.8200000000001, -128.1) node[anchor=north west,align=left] {Hypergraphs};
\draw (441.8200000000001, -128.1) rectangle (445.17000000000013,-129.2);
\draw(445.2700000000001, -128.1) node[anchor=north west,align=left] {Paths and\\ cycles};
\draw (445.2700000000001, -128.1) rectangle (448.1200000000001,-129.2);
\draw(438.3700000000001, -129.8) node[anchor=north west,align=left] {Random\\ walks \\ on graphs};
\draw (438.3700000000001, -129.8) rectangle (441.22000000000014,-131.4);
\draw(441.3200000000001, -129.8) node[anchor=north west,align=left] {Extremal\\ problems\\ in graph\\ theory};
\draw (441.3200000000001, -129.8) rectangle (443.92000000000013,-131.9);
\draw(444.0200000000001, -129.8) node[anchor=north west,align=left] {Expander\\ graphs};
\draw (444.0200000000001, -129.8) rectangle (446.6200000000001,-130.9);
\draw(446.72000000000014, -129.8) node[anchor=north west,align=left] {Infinite\\ graphs};
\draw (446.72000000000014, -129.8) rectangle (449.32000000000016,-130.9);
\draw(438.3700000000001, -132.0) node[anchor=north west,align=left] {Chemical\\ graph\\ theory};
\draw (438.3700000000001, -132.0) rectangle (440.97000000000014,-133.6);
\draw(441.0700000000001, -132.0) node[anchor=north west,align=left] {Vertex\\ degrees};
\draw (441.0700000000001, -132.0) rectangle (443.42000000000013,-133.1);
\draw(443.5200000000001, -132.0) node[anchor=north west,align=left] {Perfect\\ graphs};
\draw (443.5200000000001, -132.0) rectangle (445.8700000000001,-133.1);
\draw(450.8200000000001, -97.9) node[anchor=north west,align=left] {\large Enumerative combinatorics};
\draw (450.8200000000001, -97.9) rectangle (460.2200000000001,-109.9);
\draw(451.8200000000001, -98.9) node[anchor=north west,align=left] {Exact enumeration\\ problems,\\ generating\\ functions};
\draw (451.8200000000001, -98.9) rectangle (456.67000000000013,-101.0);
\draw(456.7700000000001, -98.9) node[anchor=north west,align=left] {Asymptotic\\ enumeration};
\draw (456.7700000000001, -98.9) rectangle (460.1200000000001,-100.5);
\draw(451.8200000000001, -101.10000000000001) node[anchor=north west,align=left] {\(q\)-calculus\\ and related\\ topics};
\draw (451.8200000000001, -101.10000000000001) rectangle (455.92000000000013,-103.2);
\draw(456.0200000000001, -101.10000000000001) node[anchor=north west,align=left] {Permutations,\\ words,\\ matrices};
\draw (456.0200000000001, -101.10000000000001) rectangle (459.8700000000001,-102.7);
\draw(451.8200000000001, -103.30000000000001) node[anchor=north west,align=left] {Factorials,\\ binomial \\ coefficients,\\ combinatorial\\ functions};
\draw (451.8200000000001, -103.30000000000001) rectangle (455.67000000000013,-105.9);
\draw(455.7700000000001, -103.30000000000001) node[anchor=north west,align=left] {Combinatorial\\ aspects\\ of partitions\\ of integers};
\draw (455.7700000000001, -103.30000000000001) rectangle (459.6200000000001,-105.4);
\draw(451.8200000000001, -106.0) node[anchor=north west,align=left] {Combinatorial\\ identities,\\ bijective\\ combinatorics};
\draw (451.8200000000001, -106.0) rectangle (455.67000000000013,-108.1);
\draw(455.7700000000001, -106.0) node[anchor=north west,align=left] {Combinatorial\\ inequalities};
\draw (455.7700000000001, -106.0) rectangle (459.6200000000001,-107.6);
\draw(451.8200000000001, -108.2) node[anchor=north west,align=left] {Partitions\\ of sets};
\draw (451.8200000000001, -108.2) rectangle (454.92000000000013,-109.8);
\draw(455.0200000000001, -108.2) node[anchor=north west,align=left] {Umbral\\ calculus};
\draw (455.0200000000001, -108.2) rectangle (457.6200000000001,-109.3);
\draw(450.8200000000001, -110.0) node[anchor=north west,align=left] {\large Designs and configurations};
\draw (450.8200000000001, -110.0) rectangle (462.2200000000001,-124.0);
\draw(451.8200000000001, -111.0) node[anchor=north west,align=left] {Combinatorial\\ aspects of \\ difference sets\\ (number-theoretic,\\ group-theoretic, etc.)};
\draw (451.8200000000001, -111.0) rectangle (457.92000000000013,-114.1);
\draw(458.0200000000001, -111.0) node[anchor=north west,align=left] {Polyominoes};
\draw (458.0200000000001, -111.0) rectangle (461.3700000000001,-112.1);
\draw(458.0200000000001, -112.2) node[anchor=north west,align=left] {Other \\ designs, \\ configurations};
\draw (458.0200000000001, -112.2) rectangle (462.1200000000001,-113.8);
\draw(451.8200000000001, -114.2) node[anchor=north west,align=left] {Combinatorial\\ aspects of\\ matrices \\ (incidence, \\ Hadamard, etc.)};
\draw (451.8200000000001, -114.2) rectangle (456.17000000000013,-116.8);
\draw(456.2700000000001, -114.2) node[anchor=north west,align=left] {Combinatorial\\ aspects\\ of tessellation\\ and \\ tiling problems};
\draw (456.2700000000001, -114.2) rectangle (460.6200000000001,-116.8);
\draw(451.8200000000001, -116.9) node[anchor=north west,align=left] {Combinatorial\\ aspects\\ of \\ block designs};
\draw (451.8200000000001, -116.9) rectangle (455.67000000000013,-119.0);
\draw(455.7700000000001, -116.9) node[anchor=north west,align=left] {Orthogonal\\ arrays, Latin\\ squares,\\ Room squares};
\draw (455.7700000000001, -116.9) rectangle (459.6200000000001,-119.0);
\draw(451.8200000000001, -119.1) node[anchor=north west,align=left] {Combinatorial\\ aspects\\ of finite\\ geometries};
\draw (451.8200000000001, -119.1) rectangle (455.67000000000013,-121.19999999999999);
\draw(455.7700000000001, -119.1) node[anchor=north west,align=left] {Combinatorial\\ aspects\\ of matroids\\ and geometric\\ lattices};
\draw (455.7700000000001, -119.1) rectangle (459.6200000000001,-121.69999999999999);
\draw(451.8200000000001, -121.8) node[anchor=north west,align=left] {Combinatorial\\ aspects\\ of packing\\ and covering};
\draw (451.8200000000001, -121.8) rectangle (455.67000000000013,-123.89999999999999);
\draw(455.7700000000001, -121.8) node[anchor=north west,align=left] {Triple\\ systems};
\draw (455.7700000000001, -121.8) rectangle (458.1200000000001,-122.89999999999999);
\draw(450.8200000000001, -124.10000000000001) node[anchor=north west,align=left] {\large Extremal combinatorics};
\draw (450.8200000000001, -124.10000000000001) rectangle (461.4700000000001,-130.5);
\draw(451.8200000000001, -125.10000000000001) node[anchor=north west,align=left] {Probabilistic \\ methods in extremal\\ combinatorics,\\ including \\ polynomial methods \\ (combinatorial \\ Nullstellensatz, etc.)};
\draw (451.8200000000001, -125.10000000000001) rectangle (457.92000000000013,-128.70000000000002);
\draw(458.0200000000001, -125.10000000000001) node[anchor=north west,align=left] {Ramsey\\ theory};
\draw (458.0200000000001, -125.10000000000001) rectangle (460.1200000000001,-126.2);
\draw(458.0200000000001, -126.30000000000001) node[anchor=north west,align=left] {Transversal\\ (matching)\\ theory};
\draw (458.0200000000001, -126.30000000000001) rectangle (461.3700000000001,-127.9);
\draw(451.8200000000001, -128.8) node[anchor=north west,align=left] {Extremal\\ set\\ theory};
\draw (451.8200000000001, -128.8) rectangle (454.42000000000013,-130.4);
\draw(450.8200000000001, -130.60000000000002) node[anchor=north west,align=left] {\large Computational methods\\ for problems \\ pertaining to combinatorics};
\draw (450.8200000000001, -130.60000000000002) rectangle (459.7900000000001,-132.20000000000002);
\draw(450.8200000000001, -132.3) node[anchor=north west,align=left] {\large History of \\ combinatorics};
\draw (450.8200000000001, -132.3) rectangle (455.4500000000001,-133.4);
\draw(437.3700000000001, -133.8) node[anchor=north west,align=left] {\large Algebraic combinatorics};
\draw (437.3700000000001, -133.8) rectangle (446.7700000000001,-145.8);
\draw(438.3700000000001, -134.8) node[anchor=north west,align=left] {Combinatorial\\ aspects\\ of representation\\ theory};
\draw (438.3700000000001, -134.8) rectangle (443.22000000000014,-136.9);
\draw(438.3700000000001, -137.0) node[anchor=north west,align=left] {Symmetric\\ functions\\ and \\ generalizations};
\draw (438.3700000000001, -137.0) rectangle (442.72000000000014,-139.1);
\draw(442.8200000000001, -137.0) node[anchor=north west,align=left] {Combinatorial\\ aspects\\ of algebraic\\ geometry};
\draw (442.8200000000001, -137.0) rectangle (446.67000000000013,-139.1);
\draw(438.3700000000001, -139.20000000000002) node[anchor=north west,align=left] {Association\\ schemes, \\ strongly \\ regular graphs};
\draw (438.3700000000001, -139.20000000000002) rectangle (442.47000000000014,-141.3);
\draw(442.5700000000001, -139.20000000000002) node[anchor=north west,align=left] {Combinatorial\\ aspects\\ of commutative\\ algebra};
\draw (442.5700000000001, -139.20000000000002) rectangle (446.67000000000013,-141.3);
\draw(438.3700000000001, -141.4) node[anchor=north west,align=left] {Combinatorial\\ aspects\\ of groups \\ and algebras};
\draw (438.3700000000001, -141.4) rectangle (442.22000000000014,-143.5);
\draw(442.3200000000001, -141.4) node[anchor=north west,align=left] {Group actions\\ on \\ combinatorial\\ structures};
\draw (442.3200000000001, -141.4) rectangle (446.17000000000013,-143.5);
\draw(438.3700000000001, -143.60000000000002) node[anchor=north west,align=left] {Combinatorial\\ aspects\\ of simplicial\\ complexes};
\draw (438.3700000000001, -143.60000000000002) rectangle (442.22000000000014,-145.70000000000002);
\draw(464.92000000000013, -1) node[anchor=north west,align=left] {\LARGE Order, lattices, ordered algebraic structures};
\draw (464.92000000000013, -1) rectangle (490.8700000000001,-36.1);
\draw(465.92000000000013, -2) node[anchor=north west,align=left] {\large Modular lattices, complemented lattices};
\draw (465.92000000000013, -2) rectangle (480.52000000000015,-7.9);
\draw(466.92000000000013, -3) node[anchor=north west,align=left] {Complemented\\ lattices, \\ orthocomplemented\\ lattices\\ and posets};
\draw (466.92000000000013, -3) rectangle (471.77000000000015,-5.6);
\draw(471.8700000000001, -3) node[anchor=north west,align=left] {Complemented\\ modular lattices,\\ continuous\\ geometries};
\draw (471.8700000000001, -3) rectangle (476.72000000000014,-5.1);
\draw(476.8200000000001, -3) node[anchor=north west,align=left] {Modular \\ lattices,\\ Desarguesian\\ lattices};
\draw (476.8200000000001, -3) rectangle (480.42000000000013,-5.1);
\draw(466.92000000000013, -5.7) node[anchor=north west,align=left] {Semimodular\\ lattices,\\ geometric\\ lattices};
\draw (466.92000000000013, -5.7) rectangle (470.27000000000015,-7.800000000000001);
\draw(480.6200000000001, -2) node[anchor=north west,align=left] {\large Ordered structures};
\draw (480.6200000000001, -2) rectangle (490.7700000000001,-12.3);
\draw(481.6200000000001, -3) node[anchor=north west,align=left] {Ordered \\ topological \\ structures (aspects\\ of ordered\\ structures)};
\draw (481.6200000000001, -3) rectangle (486.97000000000014,-5.6);
\draw(487.0700000000001, -3) node[anchor=north west,align=left] {Ordered \\ semigroups \\ and monoids};
\draw (487.0700000000001, -3) rectangle (490.42000000000013,-4.6);
\draw(481.6200000000001, -5.7) node[anchor=north west,align=left] {Ordered \\ rings, algebras,\\ modules};
\draw (481.6200000000001, -5.7) rectangle (486.22000000000014,-7.300000000000001);
\draw(486.3200000000001, -5.7) node[anchor=north west,align=left] {Ordered \\ abelian groups,\\ Riesz \\ groups, ordered\\ linear spaces};
\draw (486.3200000000001, -5.7) rectangle (490.67000000000013,-8.3);
\draw(481.6200000000001, -8.4) node[anchor=north west,align=left] {BCK-algebras,\\ BCI-algebras\\ (aspects\\ of ordered\\ structures)};
\draw (481.6200000000001, -8.4) rectangle (485.47000000000014,-11.0);
\draw(485.5700000000001, -8.4) node[anchor=north west,align=left] {Quantales};
\draw (485.5700000000001, -8.4) rectangle (488.42000000000013,-9.5);
\draw(485.5700000000001, -9.6) node[anchor=north west,align=left] {Noether\\ lattices};
\draw (485.5700000000001, -9.6) rectangle (488.17000000000013,-10.7);
\draw(481.6200000000001, -11.100000000000001) node[anchor=north west,align=left] {Ordered\\ groups};
\draw (481.6200000000001, -11.100000000000001) rectangle (483.97000000000014,-12.200000000000001);
\draw(465.92000000000013, -8.0) node[anchor=north west,align=left] {\large Computational methods\\ for problems pertaining\\ to ordered structures};
\draw (465.92000000000013, -8.0) rectangle (473.65000000000015,-9.6);
\draw(465.92000000000013, -9.700000000000001) node[anchor=north west,align=left] {\large History of \\ ordered structures};
\draw (465.92000000000013, -9.700000000000001) rectangle (472.10000000000014,-10.8);
\draw(465.92000000000013, -12.4) node[anchor=north west,align=left] {\large Distributive lattices};
\draw (465.92000000000013, -12.4) rectangle (477.0700000000001,-24.9);
\draw(466.92000000000013, -13.4) node[anchor=north west,align=left] {De Morgan \\ algebras, Łukasiewicz\\ algebras\\ (lattice-theoretic\\ aspects)};
\draw (466.92000000000013, -13.4) rectangle (472.77000000000015,-16.0);
\draw(472.8700000000001, -13.4) node[anchor=north west,align=left] {Complete\\ distributivity};
\draw (472.8700000000001, -13.4) rectangle (476.97000000000014,-15.0);
\draw(466.92000000000013, -16.1) node[anchor=north west,align=left] {Pseudocomplemented\\ lattices};
\draw (466.92000000000013, -16.1) rectangle (472.02000000000015,-17.700000000000003);
\draw(472.1200000000001, -16.1) node[anchor=north west,align=left] {Structure and\\ representation\\ theory\\ of distributive\\ lattices};
\draw (472.1200000000001, -16.1) rectangle (476.47000000000014,-18.700000000000003);
\draw(466.92000000000013, -18.8) node[anchor=north west,align=left] {Heyting \\ algebras \\ (lattice-theoretic\\ aspects)};
\draw (466.92000000000013, -18.8) rectangle (472.02000000000015,-20.900000000000002);
\draw(472.1200000000001, -18.8) node[anchor=north west,align=left] {Other \\ generalizations\\ of distributive\\ lattices};
\draw (472.1200000000001, -18.8) rectangle (476.47000000000014,-20.900000000000002);
\draw(466.92000000000013, -21.0) node[anchor=north west,align=left] {Post algebras\\ (lattice-theoretic\\ aspects)};
\draw (466.92000000000013, -21.0) rectangle (472.02000000000015,-23.1);
\draw(472.1200000000001, -21.0) node[anchor=north west,align=left] {Fuzzy lattices\\ (soft \\ algebras) and \\ related topics};
\draw (472.1200000000001, -21.0) rectangle (476.22000000000014,-23.1);
\draw(466.92000000000013, -23.200000000000003) node[anchor=north west,align=left] {MV-algebras};
\draw (466.92000000000013, -23.200000000000003) rectangle (470.27000000000015,-24.300000000000004);
\draw(470.3700000000001, -23.200000000000003) node[anchor=north west,align=left] {Lattices\\ and\\ duality};
\draw (470.3700000000001, -23.200000000000003) rectangle (472.97000000000014,-24.800000000000004);
\draw(473.0700000000001, -23.200000000000003) node[anchor=north west,align=left] {Frames,\\ locales};
\draw (473.0700000000001, -23.200000000000003) rectangle (475.42000000000013,-24.300000000000004);
\draw(477.17000000000013, -12.4) node[anchor=north west,align=left] {\large Ordered sets};
\draw (477.17000000000013, -12.4) rectangle (487.8200000000001,-20.0);
\draw(478.17000000000013, -13.4) node[anchor=north west,align=left] {Galois \\ correspondences, closure\\ operators\\ (in relation \\ to ordered sets)};
\draw (478.17000000000013, -13.4) rectangle (484.77000000000015,-16.0);
\draw(484.8700000000001, -13.4) node[anchor=north west,align=left] {Algebraic\\ aspects\\ of posets};
\draw (484.8700000000001, -13.4) rectangle (487.72000000000014,-15.0);
\draw(478.17000000000013, -16.1) node[anchor=north west,align=left] {Generalizations\\ of \\ ordered sets};
\draw (478.17000000000013, -16.1) rectangle (482.52000000000015,-17.700000000000003);
\draw(482.6200000000001, -16.1) node[anchor=north west,align=left] {Combinatorics\\ of \\ partially\\ ordered sets};
\draw (482.6200000000001, -16.1) rectangle (486.47000000000014,-18.200000000000003);
\draw(478.17000000000013, -18.3) node[anchor=north west,align=left] {Semilattices};
\draw (478.17000000000013, -18.3) rectangle (481.77000000000015,-19.400000000000002);
\draw(481.8700000000001, -18.3) node[anchor=north west,align=left] {Partial\\ orders,\\ general};
\draw (481.8700000000001, -18.3) rectangle (484.22000000000014,-19.900000000000002);
\draw(484.3200000000001, -18.3) node[anchor=north west,align=left] {Total\\ orders};
\draw (484.3200000000001, -18.3) rectangle (486.42000000000013,-19.400000000000002);
\draw(465.92000000000013, -25.0) node[anchor=north west,align=left] {\large Boolean algebras (Boolean rings)};
\draw (465.92000000000013, -25.0) rectangle (476.5700000000001,-33.1);
\draw(466.92000000000013, -26.0) node[anchor=north west,align=left] {Generalizationsof\\ Boolean\\ algebras};
\draw (466.92000000000013, -26.0) rectangle (471.77000000000015,-28.1);
\draw(471.8700000000001, -26.0) node[anchor=north west,align=left] {Stone spaces\\ (Boolean spaces)\\ and related\\ structures};
\draw (471.8700000000001, -26.0) rectangle (476.47000000000014,-28.1);
\draw(466.92000000000013, -28.2) node[anchor=north west,align=left] {Boolean algebras\\ with additional\\ operations \\ (diagonalizable\\ algebras, etc.)};
\draw (466.92000000000013, -28.2) rectangle (471.52000000000015,-30.8);
\draw(471.6200000000001, -28.2) node[anchor=north west,align=left] {Ring-theoretic\\ properties\\ of Boolean\\ algebras};
\draw (471.6200000000001, -28.2) rectangle (475.72000000000014,-30.3);
\draw(466.92000000000013, -30.9) node[anchor=north west,align=left] {Chain \\ conditions,\\ complete\\ algebras};
\draw (466.92000000000013, -30.9) rectangle (470.27000000000015,-33.0);
\draw(470.3700000000001, -30.9) node[anchor=north west,align=left] {Structure\\ theory \\ of Boolean\\ algebras};
\draw (470.3700000000001, -30.9) rectangle (473.47000000000014,-33.0);
\draw(473.5700000000001, -30.9) node[anchor=north west,align=left] {Boolean\\ functions};
\draw (473.5700000000001, -30.9) rectangle (476.42000000000013,-32.5);
\draw(476.67000000000013, -25.0) node[anchor=north west,align=left] {\large Lattices};
\draw (476.67000000000013, -25.0) rectangle (486.5700000000001,-36.0);
\draw(477.67000000000013, -26.0) node[anchor=north west,align=left] {Topologicallattices};
\draw (477.67000000000013, -26.0) rectangle (483.02000000000015,-27.6);
\draw(483.1200000000001, -26.0) node[anchor=north west,align=left] {Structure\\ theory \\ of lattices};
\draw (483.1200000000001, -26.0) rectangle (486.47000000000014,-27.6);
\draw(477.67000000000013, -27.7) node[anchor=north west,align=left] {Generalizations\\ of\\ lattices};
\draw (477.67000000000013, -27.7) rectangle (482.02000000000015,-29.3);
\draw(482.1200000000001, -27.7) node[anchor=north west,align=left] {Representation\\ theory of\\ lattices};
\draw (482.1200000000001, -27.7) rectangle (486.22000000000014,-29.8);
\draw(477.67000000000013, -29.9) node[anchor=north west,align=left] {Free lattices,\\ projective\\ lattices,\\ word problems};
\draw (477.67000000000013, -29.9) rectangle (481.77000000000015,-32.0);
\draw(481.8700000000001, -29.9) node[anchor=north west,align=left] {Continuous\\ lattices and\\ posets, \\ applications};
\draw (481.8700000000001, -29.9) rectangle (485.47000000000014,-32.0);
\draw(477.67000000000013, -32.1) node[anchor=north west,align=left] {Complete \\ lattices, \\ completions};
\draw (477.67000000000013, -32.1) rectangle (481.02000000000015,-33.7);
\draw(481.1200000000001, -32.1) node[anchor=north west,align=left] {Lattice \\ ideals, \\ congruence\\ relations};
\draw (481.1200000000001, -32.1) rectangle (484.22000000000014,-34.2);
\draw(477.67000000000013, -34.3) node[anchor=north west,align=left] {Varieties\\ of \\ lattices};
\draw (477.67000000000013, -34.3) rectangle (480.52000000000015,-35.9);
\draw(464.92000000000013, -36.2) node[anchor=north west,align=left] {\LARGE Linear and multilinear algebra; matrix theory};
\draw (464.92000000000013, -36.2) rectangle (488.42000000000013,-70.1);
\draw(465.92000000000013, -37.2) node[anchor=north west,align=left] {\large Basic linear algebra};
\draw (465.92000000000013, -37.2) rectangle (477.0700000000001,-70.0);
\draw(466.92000000000013, -38.2) node[anchor=north west,align=left] {Theory of \\ matrix inversion\\ and \\ generalized inverses};
\draw (466.92000000000013, -38.2) rectangle (472.52000000000015,-40.300000000000004);
\draw(472.6200000000001, -38.2) node[anchor=north west,align=left] {Vector and\\ tensor \\ algebra, theory\\ of invariants};
\draw (472.6200000000001, -38.2) rectangle (476.97000000000014,-40.300000000000004);
\draw(466.92000000000013, -40.400000000000006) node[anchor=north west,align=left] {Norms of matrices,\\ numerical\\ range, \\ applications of \\ functional analysis\\ to matrix theory};
\draw (466.92000000000013, -40.400000000000006) rectangle (472.27000000000015,-43.50000000000001);
\draw(472.3700000000001, -40.400000000000006) node[anchor=north west,align=left] {Linear \\ transformations,\\ semilinear \\ transformations};
\draw (472.3700000000001, -40.400000000000006) rectangle (476.97000000000014,-42.50000000000001);
\draw(466.92000000000013, -43.6) node[anchor=north west,align=left] {Matrix exponential\\ and \\ similar functions\\ of matrices};
\draw (466.92000000000013, -43.6) rectangle (472.02000000000015,-45.7);
\draw(472.1200000000001, -43.6) node[anchor=north west,align=left] {Vector spaces,\\ linear \\ dependence, \\ rank, lineability};
\draw (472.1200000000001, -43.6) rectangle (476.97000000000014,-45.7);
\draw(466.92000000000013, -45.800000000000004) node[anchor=north west,align=left] {Linear \\ equations \\ (linear algebraic\\ aspects)};
\draw (466.92000000000013, -45.800000000000004) rectangle (471.77000000000015,-47.900000000000006);
\draw(471.8700000000001, -45.800000000000004) node[anchor=north west,align=left] {Determinants,\\ permanents,\\ traces, \\ other special\\ matrix functions};
\draw (471.8700000000001, -45.800000000000004) rectangle (476.47000000000014,-48.400000000000006);
\draw(466.92000000000013, -48.5) node[anchor=north west,align=left] {Diagonalization,\\ Jordan forms};
\draw (466.92000000000013, -48.5) rectangle (471.52000000000015,-50.1);
\draw(471.6200000000001, -48.5) node[anchor=north west,align=left] {Inequalities\\ involving \\ eigenvalues \\ and eigenvectors};
\draw (471.6200000000001, -48.5) rectangle (476.22000000000014,-50.6);
\draw(466.92000000000013, -50.7) node[anchor=north west,align=left] {Canonical\\ forms, \\ reductions, \\ classification};
\draw (466.92000000000013, -50.7) rectangle (471.02000000000015,-52.800000000000004);
\draw(471.1200000000001, -50.7) node[anchor=north west,align=left] {Matrices over\\ function rings\\ in one or\\ more variables};
\draw (471.1200000000001, -50.7) rectangle (475.22000000000014,-52.800000000000004);
\draw(466.92000000000013, -52.900000000000006) node[anchor=north west,align=left] {Factorization\\ of\\ matrices};
\draw (466.92000000000013, -52.900000000000006) rectangle (470.77000000000015,-54.50000000000001);
\draw(470.8700000000001, -52.900000000000006) node[anchor=north west,align=left] {Matrix \\ equations and\\ identities};
\draw (470.8700000000001, -52.900000000000006) rectangle (474.72000000000014,-54.50000000000001);
\draw(466.92000000000013, -54.6) node[anchor=north west,align=left] {Commutativity\\ of\\ matrices};
\draw (466.92000000000013, -54.6) rectangle (470.77000000000015,-56.2);
\draw(470.8700000000001, -54.6) node[anchor=north west,align=left] {Miscellaneous\\ inequalities\\ involving\\ matrices};
\draw (470.8700000000001, -54.6) rectangle (474.72000000000014,-56.7);
\draw(466.92000000000013, -56.8) node[anchor=north west,align=left] {Applications\\ of Clifford\\ algebras to\\ physics, etc.};
\draw (466.92000000000013, -56.8) rectangle (470.77000000000015,-58.9);
\draw(470.8700000000001, -56.8) node[anchor=north west,align=left] {Applications\\ of \\ generalized\\ inverses};
\draw (470.8700000000001, -56.8) rectangle (474.47000000000014,-58.9);
\draw(474.5700000000001, -56.8) node[anchor=north west,align=left] {Matrix\\ pencils};
\draw (474.5700000000001, -56.8) rectangle (476.92000000000013,-57.9);
\draw(466.92000000000013, -59.0) node[anchor=north west,align=left] {Conditioning\\ of\\ matrices};
\draw (466.92000000000013, -59.0) rectangle (470.52000000000015,-60.6);
\draw(470.6200000000001, -59.0) node[anchor=north west,align=left] {Eigenvalues,\\ singular\\ values, and\\ eigenvectors};
\draw (470.6200000000001, -59.0) rectangle (474.22000000000014,-61.1);
\draw(466.92000000000013, -61.2) node[anchor=north west,align=left] {Linear \\ inequalities\\ of matrices};
\draw (466.92000000000013, -61.2) rectangle (470.52000000000015,-62.800000000000004);
\draw(470.6200000000001, -61.2) node[anchor=north west,align=left] {Quadratic \\ and bilinear\\ forms, inner\\ products};
\draw (470.6200000000001, -61.2) rectangle (474.22000000000014,-63.300000000000004);
\draw(466.92000000000013, -63.4) node[anchor=north west,align=left] {Algebraic\\ systems\\ of matrices};
\draw (466.92000000000013, -63.4) rectangle (470.27000000000015,-65.0);
\draw(470.3700000000001, -63.4) node[anchor=north west,align=left] {Multilinear\\ algebra,\\ tensor\\ calculus};
\draw (470.3700000000001, -63.4) rectangle (473.72000000000014,-65.5);
\draw(473.8200000000001, -63.4) node[anchor=north west,align=left] {Other \\ algebras \\ built from\\ modules};
\draw (473.8200000000001, -63.4) rectangle (476.92000000000013,-65.5);
\draw(466.92000000000013, -65.6) node[anchor=north west,align=left] {Max-plus\\ and related\\ algebras};
\draw (466.92000000000013, -65.6) rectangle (470.27000000000015,-67.19999999999999);
\draw(470.3700000000001, -65.6) node[anchor=north west,align=left] {Matrix \\ completion\\ problems};
\draw (470.3700000000001, -65.6) rectangle (473.47000000000014,-67.19999999999999);
\draw(473.5700000000001, -65.6) node[anchor=north west,align=left] {Inverse\\ problems\\ in linear\\ algebra};
\draw (473.5700000000001, -65.6) rectangle (476.42000000000013,-67.69999999999999);
\draw(466.92000000000013, -67.8) node[anchor=north west,align=left] {Clifford\\ algebras,\\ spinors};
\draw (466.92000000000013, -67.8) rectangle (469.77000000000015,-69.39999999999999);
\draw(469.8700000000001, -67.8) node[anchor=north west,align=left] {Exterior\\ algebra,\\ Grassmann\\ algebras};
\draw (469.8700000000001, -67.8) rectangle (472.72000000000014,-69.89999999999999);
\draw(472.8200000000001, -67.8) node[anchor=north west,align=left] {Linear \\ preserver\\ problems};
\draw (472.8200000000001, -67.8) rectangle (475.67000000000013,-69.39999999999999);
\draw(477.17000000000013, -37.2) node[anchor=north west,align=left] {\large History of \\ linear algebra};
\draw (477.17000000000013, -37.2) rectangle (482.1100000000001,-38.300000000000004);
\draw(477.17000000000013, -38.400000000000006) node[anchor=north west,align=left] {\large Special matrices};
\draw (477.17000000000013, -38.400000000000006) rectangle (488.3200000000001,-50.900000000000006);
\draw(478.17000000000013, -39.400000000000006) node[anchor=north west,align=left] {Positive \\ matrices and \\ their \\ generalizations; cones\\ of matrices};
\draw (478.17000000000013, -39.400000000000006) rectangle (484.27000000000015,-42.00000000000001);
\draw(484.3700000000001, -39.400000000000006) node[anchor=north west,align=left] {Boolean \\ and Hadamard\\ matrices};
\draw (484.3700000000001, -39.400000000000006) rectangle (487.97000000000014,-41.00000000000001);
\draw(478.17000000000013, -42.10000000000001) node[anchor=north west,align=left] {Matrices over\\ special \\ rings (quaternions,\\ finite\\ fields, etc.)};
\draw (478.17000000000013, -42.10000000000001) rectangle (483.52000000000015,-44.70000000000001);
\draw(483.6200000000001, -42.10000000000001) node[anchor=north west,align=left] {Hermitian,\\ skew-Hermitian,\\ and \\ related matrices};
\draw (483.6200000000001, -42.10000000000001) rectangle (488.22000000000014,-44.20000000000001);
\draw(478.17000000000013, -44.800000000000004) node[anchor=north west,align=left] {Toeplitz,\\ Cauchy,\\ and related\\ matrices};
\draw (478.17000000000013, -44.800000000000004) rectangle (481.52000000000015,-46.900000000000006);
\draw(481.6200000000001, -44.800000000000004) node[anchor=north west,align=left] {Orthogonal\\ matrices};
\draw (481.6200000000001, -44.800000000000004) rectangle (484.72000000000014,-46.400000000000006);
\draw(484.8200000000001, -44.800000000000004) node[anchor=north west,align=left] {Stochastic\\ matrices};
\draw (484.8200000000001, -44.800000000000004) rectangle (487.92000000000013,-46.400000000000006);
\draw(478.17000000000013, -47.00000000000001) node[anchor=north west,align=left] {Random \\ matrices\\ (algebraic\\ aspects)};
\draw (478.17000000000013, -47.00000000000001) rectangle (481.27000000000015,-49.10000000000001);
\draw(481.3700000000001, -47.00000000000001) node[anchor=north west,align=left] {Fuzzy \\ matrices};
\draw (481.3700000000001, -47.00000000000001) rectangle (483.97000000000014,-48.10000000000001);
\draw(484.0700000000001, -47.00000000000001) node[anchor=north west,align=left] {Matrix\\ Lie \\ algebras};
\draw (484.0700000000001, -47.00000000000001) rectangle (486.67000000000013,-48.60000000000001);
\draw(478.17000000000013, -49.2) node[anchor=north west,align=left] {Sign \\ pattern \\ matrices};
\draw (478.17000000000013, -49.2) rectangle (480.77000000000015,-50.800000000000004);
\draw(480.8700000000001, -49.2) node[anchor=north west,align=left] {Matrices\\ of\\ integers};
\draw (480.8700000000001, -49.2) rectangle (483.47000000000014,-50.800000000000004);
\draw(464.92000000000013, -70.2) node[anchor=north west,align=left] {\LARGE General algebraic systems};
\draw (464.92000000000013, -70.2) rectangle (486.92000000000013,-87.2);
\draw(465.92000000000013, -71.2) node[anchor=north west,align=left] {\large Algebraic structures};
\draw (465.92000000000013, -71.2) rectangle (476.5700000000001,-85.9);
\draw(466.92000000000013, -72.2) node[anchor=north west,align=left] {Heterogeneousalgebras};
\draw (466.92000000000013, -72.2) rectangle (472.77000000000015,-73.8);
\draw(472.8700000000001, -72.2) node[anchor=north west,align=left] {Subalgebras,\\ congruence\\ relations};
\draw (472.8700000000001, -72.2) rectangle (476.47000000000014,-74.3);
\draw(466.92000000000013, -74.4) node[anchor=north west,align=left] {Applications\\ of universal\\ algebra in \\ computer science};
\draw (466.92000000000013, -74.4) rectangle (471.52000000000015,-76.5);
\draw(471.6200000000001, -74.4) node[anchor=north west,align=left] {Operations and\\ polynomials\\ in algebraic\\ structures,\\ primal algebras};
\draw (471.6200000000001, -74.4) rectangle (475.97000000000014,-77.0);
\draw(466.92000000000013, -77.10000000000001) node[anchor=north west,align=left] {Automorphisms\\ and \\ endomorphisms \\ of algebraic\\ structures};
\draw (466.92000000000013, -77.10000000000001) rectangle (471.02000000000015,-79.7);
\draw(471.1200000000001, -77.10000000000001) node[anchor=north west,align=left] {Word problems\\ (aspects\\ of algebraic\\ structures)};
\draw (471.1200000000001, -77.10000000000001) rectangle (474.97000000000014,-79.2);
\draw(466.92000000000013, -79.80000000000001) node[anchor=north west,align=left] {Relational\\ systems,\\ laws of\\ composition};
\draw (466.92000000000013, -79.80000000000001) rectangle (470.27000000000015,-81.9);
\draw(470.3700000000001, -79.80000000000001) node[anchor=north west,align=left] {Equational\\ compactness};
\draw (470.3700000000001, -79.80000000000001) rectangle (473.72000000000014,-81.4);
\draw(473.8200000000001, -79.80000000000001) node[anchor=north west,align=left] {Partial\\ algebras};
\draw (473.8200000000001, -79.80000000000001) rectangle (476.42000000000013,-80.9);
\draw(466.92000000000013, -82.0) node[anchor=north west,align=left] {Structure\\ theory of\\ algebraic\\ structures};
\draw (466.92000000000013, -82.0) rectangle (470.02000000000015,-84.1);
\draw(470.1200000000001, -82.0) node[anchor=north west,align=left] {Infinitary\\ algebras};
\draw (470.1200000000001, -82.0) rectangle (473.22000000000014,-83.6);
\draw(473.3200000000001, -82.0) node[anchor=north west,align=left] {Fuzzy \\ algebraic\\ structures};
\draw (473.3200000000001, -82.0) rectangle (476.42000000000013,-83.6);
\draw(466.92000000000013, -84.2) node[anchor=north west,align=left] {Unary \\ algebras};
\draw (466.92000000000013, -84.2) rectangle (469.52000000000015,-85.3);
\draw(469.6200000000001, -84.2) node[anchor=north west,align=left] {Finitary\\ algebras};
\draw (469.6200000000001, -84.2) rectangle (472.22000000000014,-85.8);
\draw(476.67000000000013, -71.2) node[anchor=north west,align=left] {\large Other classes of algebras};
\draw (476.67000000000013, -71.2) rectangle (485.8200000000001,-76.10000000000001);
\draw(477.67000000000013, -72.2) node[anchor=north west,align=left] {Quasivarieties};
\draw (477.67000000000013, -72.2) rectangle (481.77000000000015,-73.3);
\draw(481.8700000000001, -72.2) node[anchor=north west,align=left] {Natural \\ dualities for\\ classes\\ of algebras};
\draw (481.8700000000001, -72.2) rectangle (485.72000000000014,-74.3);
\draw(477.67000000000013, -74.4) node[anchor=north west,align=left] {Categories\\ of\\ algebras};
\draw (477.67000000000013, -74.4) rectangle (480.77000000000015,-76.0);
\draw(480.8700000000001, -74.4) node[anchor=north west,align=left] {Axiomatic\\ model\\ classes};
\draw (480.8700000000001, -74.4) rectangle (483.72000000000014,-76.0);
\draw(476.67000000000013, -76.2) node[anchor=north west,align=left] {\large Varieties};
\draw (476.67000000000013, -76.2) rectangle (486.8200000000001,-83.8);
\draw(477.67000000000013, -77.2) node[anchor=north west,align=left] {Products, \\ amalgamated products,\\ and other\\ kinds of limits\\ and colimits};
\draw (477.67000000000013, -77.2) rectangle (483.52000000000015,-79.8);
\draw(483.6200000000001, -77.2) node[anchor=north west,align=left] {Equational\\ logic,\\ Mal’tsev\\ conditions};
\draw (483.6200000000001, -77.2) rectangle (486.72000000000014,-79.3);
\draw(477.67000000000013, -79.9) node[anchor=north west,align=left] {Congruence \\ modularity, \\ congruence \\ distributivity};
\draw (477.67000000000013, -79.9) rectangle (481.77000000000015,-82.0);
\draw(481.8700000000001, -79.9) node[anchor=north west,align=left] {Subdirect \\ products and\\ subdirect\\ irreducibility};
\draw (481.8700000000001, -79.9) rectangle (485.97000000000014,-82.0);
\draw(477.67000000000013, -82.10000000000001) node[anchor=north west,align=left] {Injectives,\\ projectives};
\draw (477.67000000000013, -82.10000000000001) rectangle (481.02000000000015,-83.7);
\draw(481.1200000000001, -82.10000000000001) node[anchor=north west,align=left] {Lattices\\ of \\ varieties};
\draw (481.1200000000001, -82.10000000000001) rectangle (483.97000000000014,-83.7);
\draw(484.0700000000001, -82.10000000000001) node[anchor=north west,align=left] {Free \\ algebras};
\draw (484.0700000000001, -82.10000000000001) rectangle (486.67000000000013,-83.2);
\draw(476.67000000000013, -83.9) node[anchor=north west,align=left] {\large Computational methods for\\ problems pertaining to\\ general algebraic systems};
\draw (476.67000000000013, -83.9) rectangle (485.02000000000015,-85.5);
\draw(465.92000000000013, -86.0) node[anchor=north west,align=left] {\large History of general\\ algebraic systems};
\draw (465.92000000000013, -86.0) rectangle (472.10000000000014,-87.1);
\draw(464.92000000000013, -87.30000000000001) node[anchor=north west,align=left] {\LARGE History and biography};
\draw (464.92000000000013, -87.30000000000001) rectangle (481.6200000000001,-112.9);
\draw(465.92000000000013, -88.30000000000001) node[anchor=north west,align=left] {\large History of mathematics and mathematicians};
\draw (465.92000000000013, -88.30000000000001) rectangle (481.52000000000015,-112.80000000000001);
\draw(466.92000000000013, -89.30000000000001) node[anchor=north west,align=left] {Bibliographicstudies};
\draw (466.92000000000013, -89.30000000000001) rectangle (472.52000000000015,-90.9);
\draw(472.6200000000001, -89.30000000000001) node[anchor=north west,align=left] {History of \\ mathematics of the\\ indigenous \\ cultures of Africa,\\ Asia, and Oceania};
\draw (472.6200000000001, -89.30000000000001) rectangle (477.97000000000014,-91.9);
\draw(478.0700000000001, -89.30000000000001) node[anchor=north west,align=left] {History of\\ mathematics\\ in Ancient\\ Babylon};
\draw (478.0700000000001, -89.30000000000001) rectangle (481.42000000000013,-91.4);
\draw(466.92000000000013, -92.00000000000001) node[anchor=north west,align=left] {History of \\ mathematics of the\\ indigenous \\ cultures of Europe\\ (pre-Greek, etc.)};
\draw (466.92000000000013, -92.00000000000001) rectangle (472.02000000000015,-94.60000000000001);
\draw(472.1200000000001, -92.00000000000001) node[anchor=north west,align=left] {Ethnomathematics,\\ general};
\draw (472.1200000000001, -92.00000000000001) rectangle (476.97000000000014,-93.60000000000001);
\draw(477.0700000000001, -92.00000000000001) node[anchor=north west,align=left] {History of\\ mathematics\\ in Paleolithic\\ and \\ Neolithic times};
\draw (477.0700000000001, -92.00000000000001) rectangle (481.42000000000013,-94.60000000000001);
\draw(466.92000000000013, -94.70000000000002) node[anchor=north west,align=left] {History of \\ mathematics \\ in late antiquity\\ and \\ medieval Europe};
\draw (466.92000000000013, -94.70000000000002) rectangle (471.77000000000015,-97.30000000000001);
\draw(471.8700000000001, -94.70000000000002) node[anchor=north west,align=left] {History of \\ mathematics at\\ institutions\\ and academies\\ (non-university)};
\draw (471.8700000000001, -94.70000000000002) rectangle (476.47000000000014,-97.30000000000001);
\draw(476.5700000000001, -94.70000000000002) node[anchor=north west,align=left] {History of \\ mathematics \\ in Ancient \\ Greece and Rome};
\draw (476.5700000000001, -94.70000000000002) rectangle (480.92000000000013,-96.80000000000001);
\draw(466.92000000000013, -97.4) node[anchor=north west,align=left] {History of \\ mathematics in\\ the 15th and\\ 16th centuries,\\ Renaissance};
\draw (466.92000000000013, -97.4) rectangle (471.27000000000015,-100.0);
\draw(471.3700000000001, -97.4) node[anchor=north west,align=left] {Collected or\\ selected works;\\ reprintings\\ or translations\\ of classics};
\draw (471.3700000000001, -97.4) rectangle (475.72000000000014,-100.0);
\draw(475.8200000000001, -97.4) node[anchor=north west,align=left] {History of \\ mathematics of\\ the indigenous\\ cultures of\\ the Americas};
\draw (475.8200000000001, -97.4) rectangle (479.92000000000013,-100.0);
\draw(466.92000000000013, -100.10000000000001) node[anchor=north west,align=left] {History \\ of mathematics\\ in \\ Ancient Egypt};
\draw (466.92000000000013, -100.10000000000001) rectangle (471.02000000000015,-102.2);
\draw(471.1200000000001, -100.10000000000001) node[anchor=north west,align=left] {History of\\ mathematics\\ in \\ Southeast Asia};
\draw (471.1200000000001, -100.10000000000001) rectangle (475.22000000000014,-102.2);
\draw(475.3200000000001, -100.10000000000001) node[anchor=north west,align=left] {Biographies,\\ obituaries,\\ personalia,\\ bibliographies};
\draw (475.3200000000001, -100.10000000000001) rectangle (479.42000000000013,-102.2);
\draw(466.92000000000013, -102.30000000000001) node[anchor=north west,align=left] {Sociology\\ (and \\ profession) of\\ mathematics};
\draw (466.92000000000013, -102.30000000000001) rectangle (471.02000000000015,-104.4);
\draw(471.1200000000001, -102.30000000000001) node[anchor=north west,align=left] {Historiography};
\draw (471.1200000000001, -102.30000000000001) rectangle (475.22000000000014,-103.4);
\draw(475.3200000000001, -102.30000000000001) node[anchor=north west,align=left] {History of \\ mathematics \\ in the Golden\\ Age of Islam};
\draw (475.3200000000001, -102.30000000000001) rectangle (479.17000000000013,-104.4);
\draw(466.92000000000013, -104.50000000000001) node[anchor=north west,align=left] {General \\ histories, \\ source books};
\draw (466.92000000000013, -104.50000000000001) rectangle (470.52000000000015,-106.10000000000001);
\draw(470.6200000000001, -104.50000000000001) node[anchor=north west,align=left] {History of\\ mathematics\\ in the\\ 17th century};
\draw (470.6200000000001, -104.50000000000001) rectangle (474.22000000000014,-106.60000000000001);
\draw(474.3200000000001, -104.50000000000001) node[anchor=north west,align=left] {History of\\ mathematics\\ in the\\ 18th century};
\draw (474.3200000000001, -104.50000000000001) rectangle (477.92000000000013,-106.60000000000001);
\draw(478.02000000000015, -104.50000000000001) node[anchor=north west,align=left] {History of\\ mathematics\\ in China};
\draw (478.02000000000015, -104.50000000000001) rectangle (481.3700000000002,-106.10000000000001);
\draw(466.92000000000013, -106.70000000000002) node[anchor=north west,align=left] {History of\\ mathematics\\ in the\\ 19th century};
\draw (466.92000000000013, -106.70000000000002) rectangle (470.52000000000015,-108.80000000000001);
\draw(470.6200000000001, -106.70000000000002) node[anchor=north west,align=left] {History of\\ mathematics\\ in the\\ 20th century};
\draw (470.6200000000001, -106.70000000000002) rectangle (474.22000000000014,-108.80000000000001);
\draw(474.3200000000001, -106.70000000000002) node[anchor=north west,align=left] {History of\\ mathematics\\ in the\\ 21st century};
\draw (474.3200000000001, -106.70000000000002) rectangle (477.92000000000013,-108.80000000000001);
\draw(478.02000000000015, -106.70000000000002) node[anchor=north west,align=left] {History of\\ mathematics\\ in Japan};
\draw (478.02000000000015, -106.70000000000002) rectangle (481.3700000000002,-108.30000000000001);
\draw(466.92000000000013, -108.9) node[anchor=north west,align=left] {Development\\ of \\ contemporary\\ mathematics};
\draw (466.92000000000013, -108.9) rectangle (470.52000000000015,-111.0);
\draw(470.6200000000001, -108.9) node[anchor=north west,align=left] {Future \\ perspectives\\ in \\ mathematics};
\draw (470.6200000000001, -108.9) rectangle (474.22000000000014,-111.0);
\draw(474.3200000000001, -108.9) node[anchor=north west,align=left] {History of\\ mathematics\\ at specific\\ universities};
\draw (474.3200000000001, -108.9) rectangle (477.92000000000013,-111.0);
\draw(478.02000000000015, -108.9) node[anchor=north west,align=left] {History of\\ mathematics\\ in India};
\draw (478.02000000000015, -108.9) rectangle (481.3700000000002,-110.5);
\draw(466.92000000000013, -111.10000000000001) node[anchor=north west,align=left] {Schools\\ of \\ mathematics};
\draw (466.92000000000013, -111.10000000000001) rectangle (470.27000000000015,-112.7);
\end{tikzpicture}

\end{document}