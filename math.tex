\documentclass[12pt]{article}
\usepackage[utf8]{inputenc}
\usepackage{pgf,tikz,pgfplots}
\pgfplotsset{compat=1.15}
\usepackage{mathrsfs}
\usetikzlibrary{arrows}
\usepackage{fontspec}
\setmainfont[Renderer=ICU,Mapping=tex-text]{Cousine}
\usepackage{amssymb}
\usepackage[paperwidth=363.70000000000005cm,paperheight=148.4cm,left=0.1cm,right=0.1cm,top=0.1cm,bottom=0.1cm]{geometry}
\begin{document}\begin{tikzpicture}[line cap=round,line join=round,>=triangle 45,x=1cm,y=1cm]
\clip(0, 0)rectangle(359.70000000000005, -144.4);

\draw(0, 0) node[anchor=north west] { \large math};
\draw (0, 0) rectangle (359.70000000000005,-144.4);
\draw(1, -1) node[anchor=north west] { \large Several complex variables and analytic spaces};
\draw (1, -1) rectangle (75.25,-54.099999999999994);
\draw(2, -2) node[anchor=north west] { \large Non-Archimedean analysis (should also be assigned at least one other classification number from Section 32-XX describing the type of problem)};
\draw (2, -2) rectangle (44.9,-4.7);
\draw(3, -3) node[anchor=north west,align=left] {Non-Archimedean analysis (should also be assigned\\ at least one other classification number from\\ Section 32-XX describing the type of problem)};

\draw(45.0, -2) node[anchor=north west] { \large Holomorphic functions of several complex variables};
\draw (45.0, -2) rectangle (75.15,-15.999999999999996);
\draw(46.0, -3) node[anchor=north west,align=left] {Normal families of holomorphic functions, mappings of several\\ complex variables, and related topics (taut manifolds etc.)};

\draw(61.95, -3) node[anchor=north west,align=left] {Other generalizations of function theory of one\\ complex variable (should also be assigned at least\\ one classification number from Section 30-XX)};

\draw(46.0, -4.7) node[anchor=north west,align=left] {Other spaces of holomorphic functions of several\\ complex variables (e.g., bounded mean oscillation\\ (BMOA), vanishing mean oscillation (VMOA))};

\draw(58.95, -4.7) node[anchor=north west,align=left] {Integral representations, constructed kernels\\ (e.g., Cauchy, Fantappiè-type kernels)};

\draw(46.0, -6.4) node[anchor=north west,align=left] {Multifunctions of several complex variables};

\draw(57.45, -6.4) node[anchor=north west,align=left] {Nevanlinna theory; growth estimates; other\\ inequalities of several complex variables};

\draw(46.0, -7.6000000000000005) node[anchor=north west,align=left] {Functional analysis techniques applied \\ to functions of several complex variables};

\draw(56.95, -7.6000000000000005) node[anchor=north west,align=left] {Residues for several complex variables};

\draw(46.0, -8.8) node[anchor=north west,align=left] {\(H^p\)-spaces, Nevanlinna spaces of\\ functions in several complex variables};

\draw(56.2, -8.8) node[anchor=north west,align=left] {Boundary behavior of holomorphic \\ functions of several complex variables};

\draw(46.0, -10.0) node[anchor=north west,align=left] {Banach algebra techniques applied to\\ functions of several complex variables};

\draw(56.2, -10.0) node[anchor=north west,align=left] {Integral representations; canonical\\ kernels (Szegő, Bergman, etc.)};

\draw(65.65, -10.0) node[anchor=north west,align=left] {Polynomials and rational functions\\ of several complex variables};

\draw(46.0, -11.2) node[anchor=north west,align=left] {Zero sets of holomorphic functions\\ of several complex variables};

\draw(55.2, -11.2) node[anchor=north west,align=left] {Power series, series of functions\\ of several complex variables};

\draw(64.15, -11.2) node[anchor=north west,align=left] {Bloch functions, normal functions\\ of several complex variables};

\draw(46.0, -12.399999999999999) node[anchor=north west,align=left] {Algebras of holomorphic functions\\ of several complex variables};

\draw(54.95, -12.399999999999999) node[anchor=north west,align=left] {Singular integrals of functions\\ in several complex variables};

\draw(63.4, -12.399999999999999) node[anchor=north west,align=left] {Special families of functions\\ of several complex variables};

\draw(46.0, -13.599999999999998) node[anchor=north west,align=left] {Bergman spaces of functions\\ in several complex variables};

\draw(53.7, -13.599999999999998) node[anchor=north west,align=left] {Holomorphic functions of\\ several complex variables};

\draw(60.65, -13.599999999999998) node[anchor=north west,align=left] {Entire functions of \\ several complex variables};

\draw(67.6, -13.599999999999998) node[anchor=north west,align=left] {Meromorphic functions of\\ several complex variables};

\draw(46.0, -14.799999999999997) node[anchor=north west,align=left] {Harmonic analysis of \\ several complex variables};

\draw(52.95, -14.799999999999997) node[anchor=north west,align=left] {Hyperfunctions};

\draw(2, -4.800000000000001) node[anchor=north west] { \large Holomorphic mappings and correspondences};
\draw (2, -4.800000000000001) rectangle (31.400000000000006,-9.4);
\draw(3, -5.800000000000001) node[anchor=north west,align=left] {Iteration of holomorphic maps, fixed points of holomorphic\\ maps and related problems for several complex variables};

\draw(18.200000000000003, -5.800000000000001) node[anchor=north west,align=left] {Holomorphic mappings, (holomorphic) embeddings \\ and related questions in several complex variables};

\draw(3, -7.000000000000001) node[anchor=north west,align=left] {Picard-type theorems and generalizations\\ for several complex variables};

\draw(13.7, -7.000000000000001) node[anchor=north west,align=left] {Boundary uniqueness of mappings\\ in several complex variables};

\draw(22.15, -7.000000000000001) node[anchor=north west,align=left] {Boundary regularity of mappings\\ in several complex variables};

\draw(3, -8.200000000000001) node[anchor=north west,align=left] {Proper holomorphic mappings,\\ finiteness theorems};

\draw(10.7, -8.200000000000001) node[anchor=north west,align=left] {Meromorphic mappings in\\ several complex variables};

\draw(17.65, -8.200000000000001) node[anchor=north west,align=left] {Value distribution theory\\ in higher dimensions};

\draw(2, -9.499999999999998) node[anchor=north west] { \large Holomorphic convexity};
\draw (2, -9.499999999999998) rectangle (30.65,-13.599999999999998);
\draw(3, -10.499999999999998) node[anchor=north west,align=left] {Holomorphic, polynomial and rational approximation, and\\ interpolation in several complex variables; Runge pairs};

\draw(17.45, -10.499999999999998) node[anchor=north west,align=left] {Polynomial convexity, rational convexity, \\ meromorphic convexity in several complex variables};

\draw(3, -11.7) node[anchor=north west,align=left] {Global boundary behavior of holomorphic\\ functions of several complex variables};

\draw(13.45, -11.7) node[anchor=north west,align=left] {Holomorphically convex complex\\ spaces, reduction theory};

\draw(21.65, -11.7) node[anchor=north west,align=left] {Stein spaces, Stein manifolds};

\draw(3, -12.899999999999999) node[anchor=north west,align=left] {The Levi problem};

\draw(2, -16.099999999999994) node[anchor=north west] { \large Complex singularities};
\draw (2, -16.099999999999994) rectangle (30.65,-24.999999999999993);
\draw(3, -17.099999999999994) node[anchor=north west,align=left] {Topological aspects of complex singularities: Lefschetz\\ theorems, topological classification, invariants};

\draw(17.45, -17.099999999999994) node[anchor=north west,align=left] {Stratifications; constructible sheaves; \\ intersection cohomology (complex-analytic aspects)};

\draw(3, -18.299999999999994) node[anchor=north west,align=left] {Monodromy; relations with differential equations\\ and \(D\)-modules (complex-analytic aspects)};

\draw(15.7, -18.299999999999994) node[anchor=north west,align=left] {Milnor fibration; relations with knot theory};

\draw(3, -19.499999999999993) node[anchor=north west,align=left] {Equisingularity (topological and analytic)};

\draw(14.2, -19.499999999999993) node[anchor=north west,align=left] {Relations with arrangements of hyperplanes};

\draw(3, -20.199999999999996) node[anchor=north west,align=left] {Other operations on complex singularities};

\draw(13.95, -20.199999999999996) node[anchor=north west,align=left] {Modifications; resolution of \\ singularities (complex-analytic aspects)};

\draw(3, -21.399999999999995) node[anchor=north west,align=left] {Global theory of complex \\ singularities; cohomological properties};

\draw(13.45, -21.399999999999995) node[anchor=north west,align=left] {Mixed Hodge theory of singular \\ varieties (complex-analytic aspects)};

\draw(3, -22.599999999999994) node[anchor=north west,align=left] {Invariants of analytic local rings};

\draw(12.2, -22.599999999999994) node[anchor=north west,align=left] {Deformations of complex \\ singularities; vanishing cycles};

\draw(20.65, -22.599999999999994) node[anchor=north west,align=left] {Singularities of holomorphic\\ vector fields and foliations};

\draw(3, -23.799999999999997) node[anchor=north west,align=left] {Local complex singularities};

\draw(10.45, -23.799999999999997) node[anchor=north west,align=left] {Complex surface and \\ hypersurface singularities};

\draw(30.75, -16.099999999999994) node[anchor=north west] { \large Deformations of analytic structures};
\draw (30.75, -16.099999999999994) rectangle (57.15,-21.399999999999995);
\draw(31.75, -17.099999999999994) node[anchor=north west,align=left] {Moduli and deformations for ordinary differential\\ equations (e.g., Knizhnik-Zamolodchikov equation)};

\draw(44.7, -17.099999999999994) node[anchor=north west,align=left] {Moduli of Riemann surfaces, Teichmüller theory\\ (complex-analytic aspects in several variables)};

\draw(31.75, -18.299999999999994) node[anchor=north west,align=left] {Deformations of submanifolds and subspaces};

\draw(42.95, -18.299999999999994) node[anchor=north west,align=left] {Applications of deformations of \\ analytic structures to the sciences};

\draw(31.75, -19.499999999999993) node[anchor=north west,align=left] {Deformations of complex structures};

\draw(40.95, -19.499999999999993) node[anchor=north west,align=left] {Complex-analytic moduli problems};

\draw(31.75, -20.199999999999996) node[anchor=north west,align=left] {Period matrices, variation of\\ Hodge structure; degenerations};

\draw(39.95, -20.199999999999996) node[anchor=north west,align=left] {Deformations of fiber bundles};

\draw(47.9, -20.199999999999996) node[anchor=north west,align=left] {Deformations of special\\ (e.g., CR) structures};

\draw(30.75, -21.5) node[anchor=north west] { \large Local analytic geometry};
\draw (30.75, -21.5) rectangle (56.4,-24.9);
\draw(31.75, -22.5) node[anchor=north west,align=left] {Triangulation and topological properties of \\ semi-analytic andsubanalytic sets, and related questions};

\draw(46.45, -22.5) node[anchor=north west,align=left] {Analytic algebras and \\ generalizations, preparation theorems};

\draw(31.75, -23.7) node[anchor=north west,align=left] {Analytic subsets of affine space};

\draw(40.45, -23.7) node[anchor=north west,align=left] {Semi-analytic sets, subanalytic\\ sets, and generalizations};

\draw(48.9, -23.7) node[anchor=north west,align=left] {Germs of analytic sets,\\ local parametrization};

\draw(2, -25.099999999999994) node[anchor=north west] { \large Complex manifolds};
\draw (2, -25.099999999999994) rectangle (28.4,-32.99999999999999);
\draw(3, -26.099999999999994) node[anchor=north west,align=left] {Special domains (Reinhardt, Hartogs, circular, tube,\\ etc.) in \(\mathbb{C}^n\) and complex manifolds};

\draw(16.7, -26.099999999999994) node[anchor=north west,align=left] {Calabi-Yau theory (complex-analytic aspects)};

\draw(3, -27.299999999999994) node[anchor=north west,align=left] {Notions of stability for complex manifolds};

\draw(14.2, -27.299999999999994) node[anchor=north west,align=left] {Embedding theorems for complex manifolds};

\draw(3, -27.999999999999993) node[anchor=north west,align=left] {Topological aspects of complex manifolds};

\draw(13.7, -27.999999999999993) node[anchor=north west,align=left] {Negative curvature complex manifolds};

\draw(3, -28.699999999999996) node[anchor=north west,align=left] {Positive curvature complex manifolds};

\draw(12.7, -28.699999999999996) node[anchor=north west,align=left] {Uniformization of complex manifolds};

\draw(3, -29.399999999999995) node[anchor=north west,align=left] {Oka principle and Oka manifolds};

\draw(11.45, -29.399999999999995) node[anchor=north west,align=left] {Complex manifolds as \\ subdomains of Euclidean space};

\draw(19.4, -29.399999999999995) node[anchor=north west,align=left] {Kähler-Einstein manifolds};

\draw(3, -30.599999999999994) node[anchor=north west,align=left] {Hyperbolic and Kobayashi\\ hyperbolic manifolds};

\draw(9.7, -30.599999999999994) node[anchor=north west,align=left] {Almost complex manifolds};

\draw(16.4, -30.599999999999994) node[anchor=north west,align=left] {Pseudoholomorphic curves};

\draw(3, -31.799999999999997) node[anchor=north west,align=left] {Classification theorems\\ for complex manifolds};

\draw(9.45, -31.799999999999997) node[anchor=north west,align=left] {Kähler manifolds};

\draw(14.149999999999999, -31.799999999999997) node[anchor=north west,align=left] {Stein manifolds};

\draw(28.5, -25.099999999999994) node[anchor=north west] { \large Complex spaces with a group of automorphisms};
\draw (28.5, -25.099999999999994) rectangle (54.4,-29.699999999999996);
\draw(29.5, -26.099999999999994) node[anchor=north west,align=left] {Hermitian symmetric spaces, bounded symmetric \\ domains, Jordan algebras (complex-analytic aspects)};

\draw(42.95, -26.099999999999994) node[anchor=north west,align=left] {Automorphism groups of other complex spaces};

\draw(29.5, -27.299999999999994) node[anchor=north west,align=left] {Almost homogeneous manifolds and spaces};

\draw(39.95, -27.299999999999994) node[anchor=north west,align=left] {Automorphism groups of \\ \(\mathbb{C}^n\) and affine manifolds};

\draw(29.5, -28.499999999999993) node[anchor=north west,align=left] {Complex vector fields, holomorphic\\ foliations, \(\mathbb{C}\)-actions};

\draw(38.7, -28.499999999999993) node[anchor=north west,align=left] {Homogeneous complex manifolds};

\draw(46.65, -28.499999999999993) node[anchor=north west,align=left] {Complex Lie groups, group\\ actions on complex spaces};

\draw(28.5, -29.799999999999997) node[anchor=north west,align=left] {Computational methods for problems pertaining \\ to several complex variables and analytic spaces};

\draw(28.5, -30.999999999999996) node[anchor=north west,align=left] {History of several complex \\ variables and analytic spaces};

\draw(2, -33.099999999999994) node[anchor=north west] { \large Holomorphic fiber spaces};
\draw (2, -33.099999999999994) rectangle (24.9,-37.199999999999996);
\draw(3, -34.099999999999994) node[anchor=north west,align=left] {Sheaves and cohomology of sections of \\ holomorphic vector bundles, general results};

\draw(14.45, -34.099999999999994) node[anchor=north west,align=left] {Holomorphic bundles and generalizations};

\draw(3, -35.3) node[anchor=north west,align=left] {Twistor theory, double fibrations\\ (complex-analytic aspects)};

\draw(11.95, -35.3) node[anchor=north west,align=left] {Applications of holomorphic\\ fiber spaces to the sciences};

\draw(19.65, -35.3) node[anchor=north west,align=left] {Vanishing theorems};

\draw(3, -36.49999999999999) node[anchor=north west,align=left] {Bundle convexity};

\draw(25.0, -33.099999999999994) node[anchor=north west] { \large Differential operators in several variables};
\draw (25.0, -33.099999999999994) rectangle (47.9,-37.699999999999996);
\draw(26.0, -34.099999999999994) node[anchor=north west,align=left] {\(\overline\partial_b\) and \\ \(\overline\partial_b\)-Neumann operators};

\draw(36.95, -34.099999999999994) node[anchor=north west,align=left] {Heat kernels in several complex variables};

\draw(26.0, -35.3) node[anchor=north west,align=left] {Other partial differential equations \\ of complex analysis in several variables};

\draw(36.7, -35.3) node[anchor=north west,align=left] {\(\overline\partial\) and \\ \(\overline\partial\)-Neumann operators};

\draw(26.0, -36.49999999999999) node[anchor=north west,align=left] {Complex Monge-Ampère operators};

\draw(34.2, -36.49999999999999) node[anchor=north west,align=left] {Pseudodifferential operators\\ in several complex variables};

\draw(48.0, -33.099999999999994) node[anchor=north west] { \large Analytic spaces};
\draw (48.0, -33.099999999999994) rectangle (70.4,-40.49999999999999);
\draw(49.0, -34.099999999999994) node[anchor=north west,align=left] {Real-analytic sets, complex Nash functions};

\draw(60.2, -34.099999999999994) node[anchor=north west,align=left] {Analytic sheaves and cohomology groups};

\draw(49.0, -34.8) node[anchor=north west,align=left] {Embedding of real-analytic manifolds};

\draw(58.7, -34.8) node[anchor=north west,align=left] {Duality theorems for analytic spaces};

\draw(49.0, -35.49999999999999) node[anchor=north west,align=left] {Local cohomology of analytic spaces};

\draw(58.45, -35.49999999999999) node[anchor=north west,align=left] {Applications of analytic spaces to\\ physics and other areas of science};

\draw(49.0, -36.699999999999996) node[anchor=north west,align=left] {Analytic subsets and submanifolds};

\draw(57.95, -36.699999999999996) node[anchor=north west,align=left] {Sheaves of differential operators\\ and their modules, \(D\)-modules};

\draw(49.0, -37.89999999999999) node[anchor=north west,align=left] {Embedding of analytic spaces};

\draw(56.7, -37.89999999999999) node[anchor=north west,align=left] {Topology of analytic spaces};

\draw(49.0, -38.599999999999994) node[anchor=north west,align=left] {The Levi problem in complex\\ spaces; generalizations};

\draw(56.45, -38.599999999999994) node[anchor=north west,align=left] {Integration on analytic\\ sets and spaces, currents};

\draw(63.4, -38.599999999999994) node[anchor=north west,align=left] {Real-analytic manifolds,\\ real-analytic spaces};

\draw(49.0, -39.8) node[anchor=north west,align=left] {Normal analytic spaces};

\draw(55.2, -39.8) node[anchor=north west,align=left] {Complex supergeometry};

\draw(61.15, -39.8) node[anchor=north west,align=left] {Complex spaces};

\draw(2, -40.599999999999994) node[anchor=north west] { \large Automorphic functions};
\draw (2, -40.599999999999994) rectangle (24.4,-43.99999999999999);
\draw(3, -41.599999999999994) node[anchor=north west,align=left] {Automorphic functions in symmetric domains};

\draw(14.2, -41.599999999999994) node[anchor=north west,align=left] {General theory of automorphic \\ functions of several complex variables};

\draw(3, -42.8) node[anchor=north west,align=left] {Automorphic forms in \\ several complex variables};

\draw(24.5, -40.599999999999994) node[anchor=north west] { \large Generalizations of analytic spaces};
\draw (24.5, -40.599999999999994) rectangle (45.9,-43.99999999999999);
\draw(25.5, -41.599999999999994) node[anchor=north west,align=left] {Holomorphic maps with \\ infinite-dimensional arguments or values};

\draw(36.2, -41.599999999999994) node[anchor=north west,align=left] {Banach analytic manifolds and spaces};

\draw(25.5, -42.8) node[anchor=north west,align=left] {Differentiable functions on analytic\\ spaces, differentiable spaces};

\draw(35.2, -42.8) node[anchor=north west,align=left] {Formal and graded complex spaces};

\draw(46.0, -40.599999999999994) node[anchor=north west] { \large Pluripotential theory};
\draw (46.0, -40.599999999999994) rectangle (67.4,-45.39999999999999);
\draw(47.0, -41.599999999999994) node[anchor=north west,align=left] {Removable sets in pluripotential theory};

\draw(57.45, -41.599999999999994) node[anchor=north west,align=left] {Plurisubharmonic exhaustion functions};

\draw(47.0, -42.3) node[anchor=north west,align=left] {Plurisubharmonic extremal functions,\\ pluricomplex Green functions};

\draw(56.7, -42.3) node[anchor=north west,align=left] {Capacity theory and generalizations};

\draw(47.0, -43.49999999999999) node[anchor=north west,align=left] {General pluripotential theory};

\draw(54.95, -43.49999999999999) node[anchor=north west,align=left] {Plurisubharmonic functions\\ and generalizations};

\draw(62.15, -43.49999999999999) node[anchor=north west,align=left] {Lelong numbers};

\draw(47.0, -44.699999999999996) node[anchor=north west,align=left] {Currents};

\draw(2, -45.49999999999999) node[anchor=north west] { \large CR manifolds};
\draw (2, -45.49999999999999) rectangle (23.4,-50.29999999999999);
\draw(3, -46.49999999999999) node[anchor=north west,align=left] {Finite-type conditions on CR manifolds};

\draw(13.2, -46.49999999999999) node[anchor=north west,align=left] {Real submanifolds in complex manifolds};

\draw(3, -47.199999999999996) node[anchor=north west,align=left] {CR manifolds as boundaries of domains};

\draw(12.95, -47.199999999999996) node[anchor=north west,align=left] {Extension of functions and other\\ analytic objects from CR manifolds};

\draw(3, -48.39999999999999) node[anchor=north west,align=left] {CR structures, CR operators,\\ and generalizations};

\draw(10.7, -48.39999999999999) node[anchor=north west,align=left] {Embeddings of CR manifolds};

\draw(3, -49.599999999999994) node[anchor=north west,align=left] {Analysis on CR manifolds};

\draw(9.7, -49.599999999999994) node[anchor=north west,align=left] {CR functions};

\draw(23.5, -45.49999999999999) node[anchor=north west] { \large Geometric convexity in several complex variables};
\draw (23.5, -45.49999999999999) rectangle (44.4,-50.099999999999994);
\draw(24.5, -46.49999999999999) node[anchor=north west,align=left] {Other notions of convexity in \\ relation to several complex variables};

\draw(34.45, -46.49999999999999) node[anchor=north west,align=left] {Invariant metrics and pseudodistances\\ in several complex variables};

\draw(24.5, -47.699999999999996) node[anchor=north west,align=left] {Analytical consequences of geometric\\ convexity (vanishing theorems, etc.)};

\draw(34.2, -47.699999999999996) node[anchor=north west,align=left] {\(q\)-convexity, \(q\)-concavity};

\draw(24.5, -48.89999999999999) node[anchor=north west,align=left] {Finite-type conditions for\\ the boundary of a domain};

\draw(31.7, -48.89999999999999) node[anchor=north west,align=left] {Topological consequences\\ of geometric convexity};

\draw(44.5, -45.49999999999999) node[anchor=north west] { \large Compact analytic spaces};
\draw (44.5, -45.49999999999999) rectangle (64.4,-50.29999999999999);
\draw(45.5, -46.49999999999999) node[anchor=north west,align=left] {Compactification of analytic spaces};

\draw(54.95, -46.49999999999999) node[anchor=north west,align=left] {Transcendental methods of algebraic\\ geometry (complex-analytic aspects)};

\draw(45.5, -47.699999999999996) node[anchor=north west,align=left] {Compact Kähler manifolds: \\ generalizations, classification};

\draw(53.95, -47.699999999999996) node[anchor=north west,align=left] {Applications of compact \\ analytic spaces to the sciences};

\draw(45.5, -48.89999999999999) node[anchor=north west,align=left] {Algebraic dependence theorems};

\draw(53.45, -48.89999999999999) node[anchor=north west,align=left] {Compact complex \(3\)-folds};

\draw(45.5, -49.599999999999994) node[anchor=north west,align=left] {Compact complex \(n\)-folds};

\draw(52.95, -49.599999999999994) node[anchor=north west,align=left] {Compact complex surfaces};

\draw(2, -50.39999999999999) node[anchor=north west] { \large Pseudoconvex domains};
\draw (2, -50.39999999999999) rectangle (20.1,-53.99999999999999);
\draw(3, -51.39999999999999) node[anchor=north west,align=left] {Geometric and analytic invariants\\ on weakly pseudoconvex boundaries};

\draw(11.95, -51.39999999999999) node[anchor=north west,align=left] {Strongly pseudoconvex domains};

\draw(3, -52.599999999999994) node[anchor=north west,align=left] {Domains of holomorphy};

\draw(8.95, -52.599999999999994) node[anchor=north west,align=left] {Exhaustion functions};

\draw(14.649999999999999, -52.599999999999994) node[anchor=north west,align=left] {Finite-type domains};

\draw(3, -53.29999999999999) node[anchor=north west,align=left] {Peak functions};

\draw(7.199999999999999, -53.29999999999999) node[anchor=north west,align=left] {Worm domains};

\draw(20.200000000000003, -50.39999999999999) node[anchor=north west] { \large Analytic continuation};
\draw (20.200000000000003, -50.39999999999999) rectangle (37.1,-53.99999999999999);
\draw(21.200000000000003, -51.39999999999999) node[anchor=north west,align=left] {Continuation of analytic objects\\ in several complex variables};

\draw(29.900000000000002, -51.39999999999999) node[anchor=north west,align=left] {Removable singularities in\\ several complex variables};

\draw(21.200000000000003, -52.599999999999994) node[anchor=north west,align=left] {Envelopes of holomorphy};

\draw(27.650000000000002, -52.599999999999994) node[anchor=north west,align=left] {Domains of holomorphy};

\draw(21.200000000000003, -53.29999999999999) node[anchor=north west,align=left] {Riemann domains};

\draw(75.35, -1) node[anchor=north west] { \large Partial differential equations};
\draw (75.35, -1) rectangle (143.14999999999998,-54.49999999999999);
\draw(76.35, -2) node[anchor=north west] { \large General higher-order partial differential equations and systems of higher-order partial differential equations};
\draw (76.35, -2) rectangle (109.94999999999999,-12.600000000000001);
\draw(77.35, -3) node[anchor=north west,align=left] {Systems of nonlinear higher-order PDEs};

\draw(87.55, -3) node[anchor=north west,align=left] {Initial-boundary value problems for \\ systems of nonlinear higher-order PDEs};

\draw(77.35, -4.2) node[anchor=north west,align=left] {Systems of linear higher-order PDEs};

\draw(86.8, -4.2) node[anchor=north west,align=left] {Boundary value problems for \\ systems of linear higher-order PDEs};

\draw(77.35, -5.4) node[anchor=north west,align=left] {Initial-boundary value problems for\\ systems of linear higher-order PDEs};

\draw(86.8, -5.4) node[anchor=north west,align=left] {Boundary value problems for systems\\ of nonlinear higher-order PDEs};

\draw(77.35, -6.6000000000000005) node[anchor=north west,align=left] {Initial value problems for systems\\ of linear higher-order PDEs};

\draw(86.55, -6.6000000000000005) node[anchor=north west,align=left] {Initial value problems for systems\\ of nonlinear higher-order PDEs};

\draw(77.35, -7.800000000000001) node[anchor=north west,align=left] {Initial-boundary value problems\\ for linear higher-order PDEs};

\draw(85.8, -7.800000000000001) node[anchor=north west,align=left] {Initial-boundary value problems\\ for nonlinear higher-order PDEs};

\draw(77.35, -9.0) node[anchor=north west,align=left] {Boundary value problems for\\ linear higher-order PDEs};

\draw(84.8, -9.0) node[anchor=north west,align=left] {Nonlinear higher-order PDEs};

\draw(77.35, -10.200000000000001) node[anchor=north west,align=left] {Initial value problems for\\ nonlinear higher-order PDEs};

\draw(84.8, -10.200000000000001) node[anchor=north west,align=left] {Boundary value problems for\\ nonlinear higher-order PDEs};

\draw(77.35, -11.400000000000002) node[anchor=north west,align=left] {Initial value problems for\\ linear higher-order PDEs};

\draw(84.55, -11.400000000000002) node[anchor=north west,align=left] {Linear higher-order PDEs};

\draw(110.05, -2) node[anchor=north west] { \large General first-order partial differential equations and systems of first-order partial differential equations};
\draw (110.05, -2) rectangle (143.05,-13.3);
\draw(111.05, -3) node[anchor=north west,align=left] {Systems of nonlinear first-order PDEs};

\draw(121.0, -3) node[anchor=north west,align=left] {Initial-boundary value problems for\\ systems of nonlinear first-order PDEs};

\draw(111.05, -4.2) node[anchor=north west,align=left] {Initial-boundary value problems for\\ systems of linear first-order PDEs};

\draw(120.5, -4.2) node[anchor=north west,align=left] {Boundary value problems for systems\\ of nonlinear first-order PDEs};

\draw(111.05, -5.4) node[anchor=north west,align=left] {Systems of linear first-order PDEs};

\draw(120.25, -5.4) node[anchor=north west,align=left] {Initial value problems for \\ systems of linear first-order PDEs};

\draw(111.05, -6.6000000000000005) node[anchor=north west,align=left] {Boundary value problems for \\ systems of linear first-order PDEs};

\draw(120.25, -6.6000000000000005) node[anchor=north west,align=left] {Initial value problems for systems\\ of nonlinear first-order PDEs};

\draw(111.05, -7.800000000000001) node[anchor=north west,align=left] {Initial-boundary value problems\\ for linear first-order PDEs};

\draw(119.5, -7.800000000000001) node[anchor=north west,align=left] {Initial-boundary value problems\\ for nonlinear first-order PDEs};

\draw(111.05, -9.0) node[anchor=north west,align=left] {Boundary value problems \\ for linear first-order PDEs};

\draw(118.5, -9.0) node[anchor=north west,align=left] {Boundary value problems for\\ nonlinear first-order PDEs};

\draw(111.05, -10.200000000000001) node[anchor=north west,align=left] {Initial value problems for\\ linear first-order PDEs};

\draw(118.25, -10.200000000000001) node[anchor=north west,align=left] {Nonlinear first-order PDEs};

\draw(111.05, -11.400000000000002) node[anchor=north west,align=left] {Initial value problems for\\ nonlinear first-order PDEs};

\draw(118.25, -11.400000000000002) node[anchor=north west,align=left] {Hamilton-Jacobi equations};

\draw(111.05, -12.600000000000001) node[anchor=north west,align=left] {Linear first-order PDEs};

\draw(76.35, -13.4) node[anchor=north west] { \large Partial differential equations and systems of partial differential equations with constant coefficients};
\draw (76.35, -13.4) rectangle (107.85,-16.8);
\draw(77.35, -14.4) node[anchor=north west,align=left] {Convexity properties of solutions to PDEs and\\ systems of PDEs with constant coefficients};

\draw(89.3, -14.4) node[anchor=north west,align=left] {Initial value problems for PDEs and \\ systems of PDEs with constant coefficients};

\draw(77.35, -15.600000000000001) node[anchor=north west,align=left] {Fundamental solutions to PDEs and systems\\ of PDEs with constant coefficients};

\draw(88.3, -15.600000000000001) node[anchor=north west,align=left] {General theory of PDEs and systems\\ of PDEs with constant coefficients};

\draw(107.94999999999999, -13.4) node[anchor=north west] { \large Parabolic equations and parabolic systems};
\draw (107.94999999999999, -13.4) rectangle (139.1,-27.1);
\draw(108.94999999999999, -14.4) node[anchor=north west,align=left] {Unilateral problems for nonlinear parabolic equations and\\ variational inequalities with nonlinear parabolic operators};

\draw(124.39999999999999, -14.4) node[anchor=north west,align=left] {Unilateral problems for linear parabolic equations and\\ variational inequalities with linear parabolic operators};

\draw(108.94999999999999, -15.600000000000001) node[anchor=north west,align=left] {Unilateral problems for parabolic systems and systems\\ of variational inequalities with parabolic operators};

\draw(122.89999999999999, -15.600000000000001) node[anchor=north west,align=left] {Nonlinear initial, boundary and initial-boundary\\ value problems for linear parabolic equations};

\draw(108.94999999999999, -16.8) node[anchor=north west,align=left] {Nonlinear initial, boundary and initial-boundary\\ value problems for nonlinear parabolic equations};

\draw(121.64999999999999, -16.8) node[anchor=north west,align=left] {Semilinear parabolic equations with \\ Laplacian, bi-Laplacian or poly-Laplacian};

\draw(108.94999999999999, -18.0) node[anchor=north west,align=left] {Initial-boundary value problems for\\ second-order parabolic equations};

\draw(118.39999999999999, -18.0) node[anchor=north west,align=left] {Initial-boundary value problems for\\ higher-order parabolic equations};

\draw(127.85, -18.0) node[anchor=north west,align=left] {Initial-boundary value problems \\ for second-order parabolic systems};

\draw(108.94999999999999, -19.200000000000003) node[anchor=north west,align=left] {Initial-boundary value problems \\ for higher-order parabolic systems};

\draw(118.14999999999999, -19.200000000000003) node[anchor=north west,align=left] {Second-order parabolic equations};

\draw(126.85, -19.200000000000003) node[anchor=north west,align=left] {Initial value problems for \\ second-order parabolic equations};

\draw(108.94999999999999, -20.400000000000002) node[anchor=north west,align=left] {Higher-order parabolic equations};

\draw(117.64999999999999, -20.400000000000002) node[anchor=north west,align=left] {Initial value problems for \\ higher-order parabolic equations};

\draw(126.35, -20.400000000000002) node[anchor=north west,align=left] {Parabolic Monge-Ampère equations};

\draw(108.94999999999999, -21.6) node[anchor=north west,align=left] {Quasilinear parabolic equations};

\draw(117.39999999999999, -21.6) node[anchor=north west,align=left] {Ultraparabolic equations, \\ pseudoparabolic equations, etc.};

\draw(125.85, -21.6) node[anchor=north west,align=left] {Quasilinear parabolic equations\\ with mean curvature operator};

\draw(108.94999999999999, -22.800000000000004) node[anchor=north west,align=left] {Second-order parabolic systems};

\draw(117.14999999999999, -22.800000000000004) node[anchor=north west,align=left] {Higher-order parabolic systems};

\draw(125.35, -22.800000000000004) node[anchor=north west,align=left] {Initial value problems for \\ second-order parabolic systems};

\draw(108.94999999999999, -24.0) node[anchor=north west,align=left] {Initial value problems for \\ higher-order parabolic systems};

\draw(117.14999999999999, -24.0) node[anchor=north west,align=left] {Semilinear parabolic equations};

\draw(125.35, -24.0) node[anchor=north west,align=left] {Degenerate parabolic equations};

\draw(108.94999999999999, -25.200000000000003) node[anchor=north west,align=left] {Quasilinear parabolic \\ equations with \(p\)-Laplacian};

\draw(117.14999999999999, -25.200000000000003) node[anchor=north west,align=left] {Nonlinear parabolic equations};

\draw(125.1, -25.200000000000003) node[anchor=north west,align=left] {Reaction-diffusion equations};

\draw(108.94999999999999, -26.4) node[anchor=north west,align=left] {Singular parabolic equations};

\draw(116.64999999999999, -26.4) node[anchor=north west,align=left] {Abstract parabolic equations};

\draw(124.35, -26.4) node[anchor=north west,align=left] {Heat equation};

\draw(128.29999999999998, -26.4) node[anchor=north west,align=left] {Heat kernel};

\draw(76.35, -27.2) node[anchor=north west] { \large Elliptic equations and elliptic systems};
\draw (76.35, -27.2) rectangle (107.0,-42.099999999999994);
\draw(77.35, -28.2) node[anchor=north west,align=left] {Unilateral problems for nonlinear elliptic equations and\\ variational inequalities with nonlinear elliptic operators};

\draw(92.55, -28.2) node[anchor=north west,align=left] {Unilateral problems for linear elliptic equations and\\ variational inequalities with linear elliptic operators};

\draw(77.35, -29.4) node[anchor=north west,align=left] {Unilateral problems for elliptic systems and systems\\ of variational inequalities with elliptic operators};

\draw(91.05, -29.4) node[anchor=north west,align=left] {Schrödinger operator, Schrödinger equation};

\draw(77.35, -30.6) node[anchor=north west,align=left] {Elliptic equations with infinity-Laplacian};

\draw(88.55, -30.6) node[anchor=north west,align=left] {Laplace operator, Helmholtz equation \\ (reduced wave equation), Poisson equation};

\draw(77.35, -31.8) node[anchor=north west,align=left] {Semilinear elliptic equations with \\ Laplacian, bi-Laplacian or poly-Laplacian};

\draw(88.3, -31.8) node[anchor=north west,align=left] {Green’s functions for elliptic equations};

\draw(77.35, -33.0) node[anchor=north west,align=left] {Variational methods for elliptic systems};

\draw(88.05, -33.0) node[anchor=north west,align=left] {Boundary values of solutions to \\ elliptic equations and elliptic systems};

\draw(77.35, -34.2) node[anchor=north west,align=left] {Nonlinear boundary value problems\\ for linear elliptic equations};

\draw(86.3, -34.2) node[anchor=north west,align=left] {Nonlinear boundary value problems\\ for nonlinear elliptic equations};

\draw(95.25, -34.2) node[anchor=north west,align=left] {Second-order elliptic equations};

\draw(77.35, -35.4) node[anchor=north west,align=left] {Variational methods for \\ second-order elliptic equations};

\draw(85.8, -35.4) node[anchor=north west,align=left] {Boundary value problems for \\ second-order elliptic equations};

\draw(94.25, -35.4) node[anchor=north west,align=left] {Higher-order elliptic equations};

\draw(77.35, -36.6) node[anchor=north west,align=left] {Variational methods for \\ higher-order elliptic equations};

\draw(85.8, -36.6) node[anchor=north west,align=left] {Boundary value problems for \\ higher-order elliptic equations};

\draw(94.25, -36.6) node[anchor=north west,align=left] {Quasilinear elliptic equations};

\draw(77.35, -37.8) node[anchor=north west,align=left] {Quasilinear elliptic \\ equations with \(p\)-Laplacian};

\draw(85.55, -37.8) node[anchor=north west,align=left] {Quasilinear elliptic equations\\ with mean curvature operator};

\draw(93.75, -37.8) node[anchor=north west,align=left] {Second-order elliptic systems};

\draw(77.35, -39.0) node[anchor=north west,align=left] {Higher-order elliptic systems};

\draw(85.3, -39.0) node[anchor=north west,align=left] {Boundary value problems for\\ second-order elliptic systems};

\draw(93.25, -39.0) node[anchor=north west,align=left] {Boundary value problems for\\ higher-order elliptic systems};

\draw(77.35, -40.2) node[anchor=north west,align=left] {Semilinear elliptic equations};

\draw(85.3, -40.2) node[anchor=north west,align=left] {Degenerate elliptic equations};

\draw(93.25, -40.2) node[anchor=north west,align=left] {First-order elliptic systems};

\draw(77.35, -40.9) node[anchor=north west,align=left] {Boundary value problems for\\ first-order elliptic systems};

\draw(85.05, -40.9) node[anchor=north west,align=left] {Nonlinear elliptic equations};

\draw(92.75, -40.9) node[anchor=north west,align=left] {Singular elliptic equations};

\draw(100.19999999999999, -40.9) node[anchor=north west,align=left] {Monge-Ampère equations};

\draw(107.1, -27.2) node[anchor=north west] { \large General topics in partial differential equations};
\draw (107.1, -27.2) rectangle (133.75,-36.8);
\draw(108.1, -28.2) node[anchor=north west,align=left] {Inequalities applied to PDEs involving derivatives,\\ differential and integral operators, or integrals};

\draw(121.55, -28.2) node[anchor=north west,align=left] {Microlocal methods and methods of sheaf \\ theory and homological algebra applied to PDEs};

\draw(108.1, -29.4) node[anchor=north west,align=left] {Uniqueness problems for PDEs: global \\ uniqueness, local uniqueness, non-uniqueness};

\draw(119.8, -29.4) node[anchor=north west,align=left] {Theoretical approximation in context of PDEs};

\draw(108.1, -30.6) node[anchor=north west,align=left] {Existence problems for PDEs: global \\ existence, local existence, non-existence};

\draw(119.05, -30.6) node[anchor=north west,align=left] {Other special methods applied to PDEs};

\draw(108.1, -31.8) node[anchor=north west,align=left] {Variational methods applied to PDEs};

\draw(117.55, -31.8) node[anchor=north west,align=left] {Wave front sets in context of PDEs};

\draw(108.1, -32.5) node[anchor=north west,align=left] {Geometric theory, characteristics,\\ transformations in context of PDEs};

\draw(117.3, -32.5) node[anchor=north west,align=left] {Transform methods (e.g., integral\\ transforms) applied to PDEs};

\draw(108.1, -33.7) node[anchor=north west,align=left] {Methods of ordinary differential\\ equations applied to PDEs};

\draw(116.8, -33.7) node[anchor=north west,align=left] {Parametrices in context of PDEs};

\draw(125.25, -33.7) node[anchor=north west,align=left] {Analyticity in context of PDEs};

\draw(108.1, -34.9) node[anchor=north west,align=left] {Singularity in context of PDEs};

\draw(116.3, -34.9) node[anchor=north west,align=left] {Fundamental solutions to PDEs};

\draw(124.25, -34.9) node[anchor=north west,align=left] {Cauchy-Kovalevskaya theorems};

\draw(108.1, -35.6) node[anchor=north west,align=left] {Topological and monotonicity\\ methods applied to PDEs};

\draw(115.8, -35.6) node[anchor=north west,align=left] {Classical solutions to PDEs};

\draw(107.1, -36.9) node[anchor=north west] { \large Representations of solutions to partial differential equations};
\draw (107.1, -36.9) rectangle (128.0,-41.199999999999996);
\draw(108.1, -37.9) node[anchor=north west,align=left] {Asymptotic expansions of solutions to PDEs};

\draw(119.3, -37.9) node[anchor=north west,align=left] {Solutions to PDEs in closed form};

\draw(108.1, -38.6) node[anchor=north west,align=left] {Trigonometric solutions to PDEs};

\draw(116.55, -38.6) node[anchor=north west,align=left] {Self-similar solutions to PDEs};

\draw(108.1, -39.3) node[anchor=north west,align=left] {Polynomial solutions to PDEs};

\draw(115.8, -39.3) node[anchor=north west,align=left] {Traveling wave solutions};

\draw(108.1, -40.0) node[anchor=north west,align=left] {Series solutions to PDEs};

\draw(114.8, -40.0) node[anchor=north west,align=left] {Integral representations\\ of solutions to PDEs};

\draw(121.5, -40.0) node[anchor=north west,align=left] {Soliton solutions};

\draw(107.1, -41.3) node[anchor=north west,align=left] {History of partial differential equations};

\draw(76.35, -42.199999999999996) node[anchor=north west] { \large Qualitative properties of solutions to partial differential equations};
\draw (76.35, -42.199999999999996) rectangle (101.44999999999999,-54.39999999999999);
\draw(77.35, -43.199999999999996) node[anchor=north west,align=left] {Dependence of solutions to PDEs on initial \\ and/or boundary data and/or on parameters of PDEs};

\draw(90.3, -43.199999999999996) node[anchor=north west,align=left] {Singular perturbations in context of PDEs};

\draw(77.35, -44.4) node[anchor=north west,align=left] {Critical points of functionals in context\\ of PDEs (e.g., energy functionals)};

\draw(88.3, -44.4) node[anchor=north west,align=left] {Asymptotic behavior of solutions to PDEs};

\draw(77.35, -45.599999999999994) node[anchor=north west,align=left] {Comparison principles in context of PDEs};

\draw(88.05, -45.599999999999994) node[anchor=north west,align=left] {Liouville theorems and Phragmén-Lindelöf\\ theorems in context of PDEs};

\draw(77.35, -46.8) node[anchor=north west,align=left] {Oscillation, zeros of solutions, mean\\ value theorems, etc. in context of PDEs};

\draw(87.8, -46.8) node[anchor=north west,align=left] {Homogenization in context of PDEs; \\ PDEs in media with periodic structure};

\draw(77.35, -48.0) node[anchor=north west,align=left] {Critical exponents in context of PDEs};

\draw(87.3, -48.0) node[anchor=north west,align=left] {Pattern formations in context of PDEs};

\draw(77.35, -48.699999999999996) node[anchor=north west,align=left] {A priori estimates in context of PDEs};

\draw(87.3, -48.699999999999996) node[anchor=north west,align=left] {Maximum principles in context of PDEs};

\draw(77.35, -49.4) node[anchor=north west,align=left] {Axially symmetric solutions to PDEs};

\draw(86.8, -49.4) node[anchor=north west,align=left] {Perturbations in context of PDEs};

\draw(77.35, -50.099999999999994) node[anchor=north west,align=left] {Bifurcations in context of PDEs};

\draw(85.8, -50.099999999999994) node[anchor=north west,align=left] {Continuation and prolongation\\ of solutions to PDEs};

\draw(93.75, -50.099999999999994) node[anchor=north west,align=left] {Resonance in context of PDEs};

\draw(77.35, -51.3) node[anchor=north west,align=left] {Stability in context of PDEs};

\draw(85.05, -51.3) node[anchor=north west,align=left] {Positive solutions to PDEs};

\draw(92.25, -51.3) node[anchor=north west,align=left] {Periodic solutions to PDEs};

\draw(77.35, -52.0) node[anchor=north west,align=left] {Almost and pseudo-almost\\ periodic solutions to PDEs};

\draw(84.55, -52.0) node[anchor=north west,align=left] {Blow-up in context of PDEs};

\draw(91.75, -52.0) node[anchor=north west,align=left] {Smoothness and regularity\\ of solutions to PDEs};

\draw(77.35, -53.199999999999996) node[anchor=north west,align=left] {Entire solutions to PDEs};

\draw(84.05, -53.199999999999996) node[anchor=north west,align=left] {Symmetries, invariants,\\ etc. in context of PDEs};

\draw(90.5, -53.199999999999996) node[anchor=north west,align=left] {Inertial manifolds};

\draw(95.69999999999999, -53.199999999999996) node[anchor=north west,align=left] {Attractors};

\draw(101.55, -42.199999999999996) node[anchor=north west] { \large Hyperbolic equations and hyperbolic systems};
\draw (101.55, -42.199999999999996) rectangle (121.44999999999999,-47.49999999999999);
\draw(102.55, -43.199999999999996) node[anchor=north west,align=left] {Initial-boundary value problems for\\ first-order hyperbolic equations};

\draw(112.0, -43.199999999999996) node[anchor=north west,align=left] {Initial-boundary value problems for\\ second-order hyperbolic equations};

\draw(102.55, -44.4) node[anchor=north west,align=left] {Second-order hyperbolic equations};

\draw(111.5, -44.4) node[anchor=north west,align=left] {Initial value problems for \\ second-order hyperbolic equations};

\draw(102.55, -45.599999999999994) node[anchor=north west,align=left] {First-order hyperbolic equations};

\draw(111.25, -45.599999999999994) node[anchor=north west,align=left] {Initial value problems for \\ first-order hyperbolic equations};

\draw(102.55, -46.8) node[anchor=north west,align=left] {Wave equation};

\draw(101.55, -47.599999999999994) node[anchor=north west] { \large Generalized solutions to partial differential equations};
\draw (101.55, -47.599999999999994) rectangle (118.65,-49.99999999999999);
\draw(102.55, -48.599999999999994) node[anchor=north west,align=left] {Viscosity solutions to PDEs};

\draw(110.0, -48.599999999999994) node[anchor=north west,align=left] {Strong solutions to PDEs};

\draw(102.55, -49.3) node[anchor=north west,align=left] {Weak solutions to PDEs};

\draw(101.55, -50.099999999999994) node[anchor=north west] { \large Close-to-elliptic equations};
\draw (101.55, -50.099999999999994) rectangle (115.19999999999999,-52.49999999999999);
\draw(102.55, -51.099999999999994) node[anchor=north west,align=left] {Quasielliptic equations};

\draw(109.0, -51.099999999999994) node[anchor=north west,align=left] {Hypoelliptic equations};

\draw(102.55, -51.8) node[anchor=north west,align=left] {Subelliptic equations};

\draw(1, -54.599999999999994) node[anchor=north west] { \large Ordinary differential equations};
\draw (1, -54.599999999999994) rectangle (68.75,-105.9);
\draw(2, -55.599999999999994) node[anchor=north west] { \large General theory for ordinary differential equations};
\draw (2, -55.599999999999994) rectangle (35.900000000000006,-65.5);
\draw(3, -56.599999999999994) node[anchor=north west,align=left] {Generalized ordinary differential equations (measure-differential\\ equations, set-valued differential equations, etc.)};

\draw(19.950000000000003, -56.599999999999994) node[anchor=north west,align=left] {Analytical theory of ordinary differential equations: series,\\ transformations, transforms, operational calculus, etc.};

\draw(3, -57.8) node[anchor=north west,align=left] {Initial value problems, existence, uniqueness,\\ continuous dependence and continuation of\\ solutions to ordinary differential equations};

\draw(15.2, -57.8) node[anchor=north west,align=left] {Fractional ordinary differential equations\\ and fractional differential inclusions};

\draw(3, -59.49999999999999) node[anchor=north west,align=left] {Implicit ordinary differential equations,\\ differential-algebraic equations};

\draw(13.95, -59.49999999999999) node[anchor=north west,align=left] {Ordinary lattice differential equations};

\draw(24.4, -59.49999999999999) node[anchor=north west,align=left] {Theoretical approximation of solutions\\ to ordinary differential equations};

\draw(3, -60.699999999999996) node[anchor=north west,align=left] {Fuzzy ordinary differential equations};

\draw(12.95, -60.699999999999996) node[anchor=north west,align=left] {Nonlinear ordinary differential \\ equations and systems, general theory};

\draw(22.9, -60.699999999999996) node[anchor=north west,align=left] {Explicit solutions, first integrals\\ of ordinary differential equations};

\draw(3, -61.89999999999999) node[anchor=north west,align=left] {Differential inequalities involving\\ functions of a single real variable};

\draw(12.45, -61.89999999999999) node[anchor=north west,align=left] {Ordinary differential inclusions};

\draw(21.15, -61.89999999999999) node[anchor=north west,align=left] {Inverse problems involving \\ ordinary differential equations};

\draw(3, -63.099999999999994) node[anchor=north west,align=left] {Linear ordinary differential\\ equations and systems, general};

\draw(11.2, -63.099999999999994) node[anchor=north west,align=left] {Geometric methods in ordinary\\ differential equations};

\draw(19.15, -63.099999999999994) node[anchor=north west,align=left] {Ordinary differential \\ equations of infinite order};

\draw(26.599999999999998, -63.099999999999994) node[anchor=north west,align=left] {Hybrid systems of ordinary\\ differential equations};

\draw(3, -64.3) node[anchor=north west,align=left] {Ordinary differential\\ equations with impulses};

\draw(9.45, -64.3) node[anchor=north west,align=left] {Discontinuous ordinary\\ differential equations};

\draw(36.00000000000001, -55.599999999999994) node[anchor=north west] { \large Ordinary differential equations in the complex domain};
\draw (36.00000000000001, -55.599999999999994) rectangle (68.65,-68.19999999999999);
\draw(37.00000000000001, -56.599999999999994) node[anchor=north west,align=left] {Stokes phenomena and connection problems (linear and nonlinear)\\ for ordinary differential equations in the complex domain};

\draw(53.45000000000001, -56.599999999999994) node[anchor=north west,align=left] {Painlevé and other special ordinary differential equations\\ in the complex domain; classification, hierarchies};

\draw(37.00000000000001, -57.8) node[anchor=north west,align=left] {Formal solutions and transform techniques for \\ ordinary differential equations in the complex domain};

\draw(50.95, -57.8) node[anchor=north west,align=left] {Algebraic aspects (differential-algebraic, \\ hypertranscendence, group-theoretical) of ordinary\\ differential equations in the complex domain};

\draw(37.00000000000001, -59.49999999999999) node[anchor=north west,align=left] {Topological structure of trajectories of ordinary\\ differential equations in the complex domain};

\draw(49.95, -59.49999999999999) node[anchor=north west,align=left] {Singular perturbation problems for ordinary \\ differential equations in the complex domain \\ (complex WKB, turning points, steepest descent)};

\draw(37.00000000000001, -61.199999999999996) node[anchor=north west,align=left] {Asymptotics and summation methods for ordinary\\ differential equations in the complex domain};

\draw(49.2, -61.199999999999996) node[anchor=north west,align=left] {Singularities, monodromy and local behavior\\ of solutions to ordinary differential \\ equations in the complex domain, normal forms};

\draw(37.00000000000001, -62.89999999999999) node[anchor=north west,align=left] {Entire and meromorphic solutions to ordinary\\ differential equations in the complex domain};

\draw(48.7, -62.89999999999999) node[anchor=north west,align=left] {Oscillation, growth of solutions to ordinary\\ differential equations in the complex domain};

\draw(37.00000000000001, -64.1) node[anchor=north west,align=left] {Inverse problems (Riemann-Hilbert, inverse\\ differential Galois, etc.) for ordinary \\ differential equations in the complex domain};

\draw(48.7, -64.1) node[anchor=north west,align=left] {Isomonodromic deformations for ordinary \\ differential equations in the complex domain};

\draw(37.00000000000001, -65.8) node[anchor=north west,align=left] {Nonlinear ordinary differential equations\\ and systems in the complex domain};

\draw(47.95, -65.8) node[anchor=north west,align=left] {Spectral theory for ordinary differential\\ operators in the complex domain};

\draw(37.00000000000001, -67.0) node[anchor=north west,align=left] {Linear ordinary differential equations\\ and systems in the complex domain};

\draw(47.2, -67.0) node[anchor=north west,align=left] {Ordinary differential \\ equations on complex manifolds};

\draw(2, -68.3) node[anchor=north west] { \large Functional-differential equations (including equations with delayed, advanced or state-dependent argument)};
\draw (2, -68.3) rectangle (34.4,-88.0);
\draw(3, -69.3) node[anchor=north west,align=left] {Transformation and reduction of functional-differential\\ equations and systems, normal forms};

\draw(17.45, -69.3) node[anchor=north west,align=left] {Qualitative investigation and simulation of \\ models involving functional-differential equations};

\draw(3, -70.5) node[anchor=north west,align=left] {Growth, boundedness, comparison of solutions\\ to functional-differential equations};

\draw(14.7, -70.5) node[anchor=north west,align=left] {Stochastic functional-differential equations};

\draw(3, -71.7) node[anchor=north west,align=left] {Almost and pseudo-almost periodic solutions\\ to functional-differential equations};

\draw(14.45, -71.7) node[anchor=north west,align=left] {Implicit functional-differential equations};

\draw(3, -72.89999999999999) node[anchor=north west,align=left] {Lattice functional-differential equations};

\draw(13.95, -72.89999999999999) node[anchor=north west,align=left] {Neutral functional-differential equations};

\draw(3, -73.6) node[anchor=north west,align=left] {Linear functional-differential equations};

\draw(13.7, -73.6) node[anchor=north west,align=left] {Complex (chaotic) behavior of solutions\\ to functional-differential equations};

\draw(3, -74.8) node[anchor=north west,align=left] {Fuzzy functional-differential equations};

\draw(13.45, -74.8) node[anchor=north west,align=left] {Theoretical approximation of solutions\\ to functional-differential equations};

\draw(3, -76.0) node[anchor=north west,align=left] {Heteroclinic and homoclinic orbits\\ of functional-differential equations};

\draw(12.7, -76.0) node[anchor=north west,align=left] {Functional-differential inequalities};

\draw(3, -77.2) node[anchor=north west,align=left] {Functional-differential inclusions};

\draw(12.2, -77.2) node[anchor=north west,align=left] {Symmetries, invariants of \\ functional-differential equations};

\draw(21.15, -77.2) node[anchor=north west,align=left] {General theory of \\ functional-differential equations};

\draw(3, -78.4) node[anchor=north west,align=left] {Spectral theory of \\ functional-differential operators};

\draw(11.95, -78.4) node[anchor=north west,align=left] {Boundary value problems for \\ functional-differential equations};

\draw(20.9, -78.4) node[anchor=north west,align=left] {Oscillation theory of \\ functional-differential equations};

\draw(3, -79.6) node[anchor=north west,align=left] {Periodic solutions to \\ functional-differential equations};

\draw(11.95, -79.6) node[anchor=north west,align=left] {Bifurcation theory of \\ functional-differential equations};

\draw(20.9, -79.6) node[anchor=north west,align=left] {Invariant manifolds of \\ functional-differential equations};

\draw(3, -80.8) node[anchor=north west,align=left] {Stability theory of \\ functional-differential equations};

\draw(11.95, -80.8) node[anchor=north west,align=left] {Stationary solutions of \\ functional-differential equations};

\draw(20.9, -80.8) node[anchor=north west,align=left] {Synchronization of \\ functional-differential equations};

\draw(3, -82.0) node[anchor=north west,align=left] {Asymptotic theory of \\ functional-differential equations};

\draw(11.95, -82.0) node[anchor=north west,align=left] {Singular perturbations of \\ functional-differential equations};

\draw(20.9, -82.0) node[anchor=north west,align=left] {Perturbations of \\ functional-differential equations};

\draw(3, -83.19999999999999) node[anchor=north west,align=left] {Inverse problems for \\ functional-differential equations};

\draw(11.95, -83.19999999999999) node[anchor=north west,align=left] {Averaging for \\ functional-differential equations};

\draw(20.9, -83.19999999999999) node[anchor=north west,align=left] {Hybrid systems of \\ functional-differential equations};

\draw(3, -84.39999999999999) node[anchor=north west,align=left] {Control problems for \\ functional-differential equations};

\draw(11.95, -84.39999999999999) node[anchor=north west,align=left] {Functional-differential equations\\ with fractional derivatives};

\draw(20.9, -84.39999999999999) node[anchor=north west,align=left] {Discontinuous \\ functional-differential equations};

\draw(3, -85.6) node[anchor=north west,align=left] {Functional-differential equations\\ on time scales or measure chains};

\draw(11.95, -85.6) node[anchor=north west,align=left] {Functional-differential equations\\ with state-dependent arguments};

\draw(20.9, -85.6) node[anchor=north west,align=left] {Functional-differential \\ equations in the complex domain};

\draw(3, -86.8) node[anchor=north west,align=left] {Functional-differential \\ equations in abstract spaces};

\draw(10.7, -86.8) node[anchor=north west,align=left] {Functional-differential\\ equations with impulses};

\draw(34.5, -68.3) node[anchor=north west] { \large Qualitative theory for ordinary differential equations};
\draw (34.5, -68.3) rectangle (66.15,-81.1);
\draw(35.5, -69.3) node[anchor=north west,align=left] {Theory of limit cycles of polynomial and analytic vector\\ fields (existence, uniqueness, bounds, Hilbert’s 16th \\ problem and ramifications) for ordinary differential equations};

\draw(51.7, -69.3) node[anchor=north west,align=left] {Topological structure of integral curves, singular \\ points, limit cycles of ordinary differential equations};

\draw(35.5, -71.0) node[anchor=north west,align=left] {Oscillation theory, zeros, disconjugacy and \\ comparison theory for ordinary differential equations};

\draw(49.45, -71.0) node[anchor=north west,align=left] {Ordinary differential equations and connections\\ with real algebraic geometry (fewnomials, \\ desingularization, zeros of abelian integrals, etc.)};

\draw(35.5, -72.7) node[anchor=north west,align=left] {Transformation and reduction of ordinary \\ differential equations and systems, normal forms};

\draw(48.2, -72.7) node[anchor=north west,align=left] {Nonlinear oscillations and coupled oscillators\\ for ordinary differential equations};

\draw(35.5, -73.89999999999999) node[anchor=north west,align=left] {Almost and pseudo-almost periodic solutions\\ to ordinary differential equations};

\draw(46.95, -73.89999999999999) node[anchor=north west,align=left] {Qualitative investigation and simulation\\ of ordinary differential equation models};

\draw(35.5, -75.1) node[anchor=north west,align=left] {Equivalence and asymptotic equivalence\\ of ordinary differential equations};

\draw(45.7, -75.1) node[anchor=north west,align=left] {Homoclinic and heteroclinic solutions\\ to ordinary differential equations};

\draw(55.65, -75.1) node[anchor=north west,align=left] {Complex behavior and chaotic systems\\ of ordinary differential equations};

\draw(35.5, -76.3) node[anchor=north west,align=left] {Growth and boundedness of solutions\\ to ordinary differential equations};

\draw(44.95, -76.3) node[anchor=north west,align=left] {Monotone systems involving \\ ordinary differential equations};

\draw(53.4, -76.3) node[anchor=north west,align=left] {Symmetries, invariants of \\ ordinary differential equations};

\draw(35.5, -77.5) node[anchor=north west,align=left] {Bifurcation theory for \\ ordinary differential equations};

\draw(43.95, -77.5) node[anchor=north west,align=left] {Periodic solutions to \\ ordinary differential equations};

\draw(52.4, -77.5) node[anchor=north west,align=left] {Relaxation oscillations for \\ ordinary differential equations};

\draw(35.5, -78.69999999999999) node[anchor=north west,align=left] {Ordinary differential equations\\ and systems on manifolds};

\draw(43.95, -78.69999999999999) node[anchor=north west,align=left] {Invariant manifolds for \\ ordinary differential equations};

\draw(52.4, -78.69999999999999) node[anchor=north west,align=left] {Multifrequency systems of \\ ordinary differential equations};

\draw(35.5, -79.89999999999999) node[anchor=north west,align=left] {Averaging method for ordinary\\ differential equations};

\draw(43.45, -79.89999999999999) node[anchor=north west,align=left] {Hysteresis for ordinary\\ differential equations};

\draw(34.5, -81.19999999999999) node[anchor=north west] { \large Ordinary differential operators};
\draw (34.5, -81.19999999999999) rectangle (65.15,-86.99999999999999);
\draw(35.5, -82.19999999999999) node[anchor=north west,align=left] {Asymptotic distribution of eigenvalues, asymptotic theory\\ of eigenfunctions for ordinary differential operators};

\draw(50.45, -82.19999999999999) node[anchor=north west,align=left] {Numerical approximation of eigenvalues and of other \\ parts of the spectrum of ordinary differential operators};

\draw(35.5, -83.39999999999999) node[anchor=north west,align=left] {Eigenfunctions, eigenfunction expansions, completeness\\ of eigenfunctions of ordinary differential operators};

\draw(49.7, -83.39999999999999) node[anchor=north west,align=left] {Eigenvalues, estimation of eigenvalues, upper and\\ lower bounds of ordinary differential operators};

\draw(35.5, -84.6) node[anchor=north west,align=left] {Particular ordinary differential operators\\ (Dirac, one-dimensional Schrödinger, etc.)};

\draw(46.7, -84.6) node[anchor=north west,align=left] {Scattering theory, inverse scattering \\ involving ordinary differential operators};

\draw(35.5, -85.79999999999998) node[anchor=north west,align=left] {Nonlinear ordinary differential operators};

\draw(46.45, -85.79999999999998) node[anchor=north west,align=left] {General spectral theory of \\ ordinary differential operators};

\draw(34.5, -87.1) node[anchor=north west,align=left] {History of ordinary differential equations};

\draw(2, -88.1) node[anchor=north west] { \large Boundary value problems for ordinary differential equations};
\draw (2, -88.1) rectangle (31.150000000000006,-97.5);
\draw(3, -89.1) node[anchor=north west,align=left] {Linear boundary value problems for ordinary differential \\ equations with nonlinear dependence on the spectral parameter};

\draw(18.950000000000003, -89.1) node[anchor=north west,align=left] {Positive solutions to nonlinear boundary value\\ problems for ordinary differential equations};

\draw(3, -90.3) node[anchor=north west,align=left] {Boundary value problems on infinite \\ intervals for ordinary differential equations};

\draw(14.95, -90.3) node[anchor=north west,align=left] {Parameter dependent boundary value \\ problems for ordinary differential equations};

\draw(3, -91.5) node[anchor=north west,align=left] {Nonlocal and multipoint boundary value \\ problems for ordinary differential equations};

\draw(14.7, -91.5) node[anchor=north west,align=left] {Boundary value problems on graphs and \\ networks for ordinary differential equations};

\draw(3, -92.69999999999999) node[anchor=north west,align=left] {Singular nonlinear boundary value problems\\ for ordinary differential equations};

\draw(14.2, -92.69999999999999) node[anchor=north west,align=left] {Applications of boundary value problems\\ involving ordinary differential equations};

\draw(3, -93.89999999999999) node[anchor=north west,align=left] {Special ordinary differential \\ equations (Mathieu, Hill, Bessel, etc.)};

\draw(13.45, -93.89999999999999) node[anchor=north west,align=left] {Boundary value problems with impulses\\ for ordinary differential equations};

\draw(3, -95.1) node[anchor=north west,align=left] {Nonlinear boundary value problems\\ for ordinary differential equations};

\draw(12.45, -95.1) node[anchor=north west,align=left] {Weyl theory and its generalizations\\ for ordinary differential equations};

\draw(21.9, -95.1) node[anchor=north west,align=left] {Linear boundary value problems for\\ ordinary differential equations};

\draw(3, -96.3) node[anchor=north west,align=left] {Boundary eigenvalue problems for\\ ordinary differential equations};

\draw(11.7, -96.3) node[anchor=north west,align=left] {Green’s functions for \\ ordinary differential equations};

\draw(20.15, -96.3) node[anchor=north west,align=left] {Sturm-Liouville theory};

\draw(31.250000000000007, -88.1) node[anchor=north west] { \large Asymptotic theory for ordinary differential equations};
\draw (31.250000000000007, -88.1) rectangle (55.150000000000006,-93.89999999999999);
\draw(32.25000000000001, -89.1) node[anchor=north west,align=left] {Singular perturbations, turning point theory,\\ WKB methods for ordinary differential equations};

\draw(44.7, -89.1) node[anchor=north west,align=left] {Perturbations, asymptotics of solutions\\ to ordinary differential equations};

\draw(32.25000000000001, -90.3) node[anchor=north west,align=left] {Singular perturbations, general theory\\ for ordinary differential equations};

\draw(42.45, -90.3) node[anchor=north west,align=left] {Methods of nonstandard analysis \\ for ordinary differential equations};

\draw(32.25000000000001, -91.5) node[anchor=north west,align=left] {Asymptotic expansions of solutions\\ to ordinary differential equations};

\draw(41.45, -91.5) node[anchor=north west,align=left] {Multiple scale methods for \\ ordinary differential equations};

\draw(32.25000000000001, -92.69999999999999) node[anchor=north west,align=left] {Canard solutions to ordinary\\ differential equations};

\draw(31.250000000000007, -94.0) node[anchor=north west] { \large Ordinary differential equations and systems with randomness};
\draw (31.250000000000007, -94.0) rectangle (55.150000000000006,-97.4);
\draw(32.25000000000001, -95.0) node[anchor=north west,align=left] {Bifurcation of solutions to ordinary \\ differential equations involving randomness};

\draw(43.7, -95.0) node[anchor=north west,align=left] {Resonance phenomena for ordinary \\ differential equations involving randomness};

\draw(32.25000000000001, -96.2) node[anchor=north west,align=left] {Ordinary differential equations\\ and systems with randomness};

\draw(2, -97.6) node[anchor=north west] { \large Stability theory for ordinary differential equations};
\draw (2, -97.6) rectangle (25.4,-105.8);
\draw(3, -98.6) node[anchor=north west,align=left] {Structural stability and analogous concepts \\ of solutions to ordinary differential equations};

\draw(15.45, -98.6) node[anchor=north west,align=left] {Characteristic and Lyapunov exponents\\ of ordinary differential equations};

\draw(3, -99.8) node[anchor=north west,align=left] {Stability of manifolds of solutions\\ to ordinary differential equations};

\draw(12.45, -99.8) node[anchor=north west,align=left] {Asymptotic properties of solutions\\ to ordinary differential equations};

\draw(3, -101.0) node[anchor=north west,align=left] {Dichotomy, trichotomy of solutions\\ to ordinary differential equations};

\draw(12.2, -101.0) node[anchor=north west,align=left] {Global stability of solutions to\\ ordinary differential equations};

\draw(3, -102.19999999999999) node[anchor=north west,align=left] {Synchronization of solutions to\\ ordinary differential equations};

\draw(11.45, -102.19999999999999) node[anchor=north west,align=left] {Singular perturbations of \\ ordinary differential equations};

\draw(3, -103.39999999999999) node[anchor=north west,align=left] {Stability of solutions to \\ ordinary differential equations};

\draw(11.45, -103.39999999999999) node[anchor=north west,align=left] {Attractors of solutions to \\ ordinary differential equations};

\draw(3, -104.6) node[anchor=north west,align=left] {Perturbations of ordinary\\ differential equations};

\draw(25.5, -97.6) node[anchor=north west] { \large Control problems including ordinary differential equations};
\draw (25.5, -97.6) rectangle (44.65,-101.0);
\draw(26.5, -98.6) node[anchor=north west,align=left] {Chaos control for problems involving\\ ordinary differential equations};

\draw(36.2, -98.6) node[anchor=north west,align=left] {Control problems involving \\ ordinary differential equations};

\draw(26.5, -99.8) node[anchor=north west,align=left] {Stabilization of solutions to\\ ordinary differential equations};

\draw(34.95, -99.8) node[anchor=north west,align=left] {Bifurcation control of \\ ordinary differential equations};

\draw(25.5, -101.1) node[anchor=north west] { \large Differential equations in abstract spaces};
\draw (25.5, -101.1) rectangle (41.9,-104.0);
\draw(26.5, -102.1) node[anchor=north west,align=left] {Linear differential \\ equations in abstract spaces};

\draw(34.2, -102.1) node[anchor=north west,align=left] {Nonlinear differential \\ equations in abstract spaces};

\draw(26.5, -103.3) node[anchor=north west,align=left] {Evolution inclusions};

\draw(44.75, -97.6) node[anchor=north west] { \large Dynamic equations on time scales or measure chains};
\draw (44.75, -97.6) rectangle (60.35,-99.8);
\draw(45.75, -98.6) node[anchor=north west,align=left] {Dynamic equations on time\\ scales or measure chains};

\draw(68.85, -54.599999999999994) node[anchor=north west] { \large Measure and integration};
\draw (68.85, -54.599999999999994) rectangle (131.85,-66.39999999999999);
\draw(69.85, -55.599999999999994) node[anchor=north west] { \large Set functions and measures on spaces with additional structure};
\draw (69.85, -55.599999999999994) rectangle (104.5,-58.99999999999999);
\draw(70.85, -56.599999999999994) node[anchor=north west,align=left] {Integration theory via linear functionals (Radon measures, \\ Daniell integrals, etc.), representing set functions and measures};

\draw(87.8, -56.599999999999994) node[anchor=north west,align=left] {Set functions and measures and integrals in infinite-dimensional\\ spaces (Wiener measure, Gaussian measure, etc.)};

\draw(70.85, -57.8) node[anchor=north west,align=left] {Set functions and measures on topological groups\\ or semigroups, Haar measures, invariant measures};

\draw(83.55, -57.8) node[anchor=north west,align=left] {Set functions and measures on topological\\ spaces (regularity of measures, etc.)};

\draw(104.6, -55.599999999999994) node[anchor=north west] { \large Classical measure theory};
\draw (104.6, -55.599999999999994) rectangle (131.75,-62.8);
\draw(105.6, -56.599999999999994) node[anchor=north west,align=left] {Classes of sets (Borel fields, \(\sigma\)-rings,\\ etc.), measurable sets, Suslin sets, analytic sets};

\draw(118.8, -56.599999999999994) node[anchor=north west,align=left] {Measurable and nonmeasurable functions, sequences\\ of measurable functions, modes of convergence};

\draw(105.6, -57.8) node[anchor=north west,align=left] {Spaces of measures, convergence of measures};

\draw(117.05, -57.8) node[anchor=north west,align=left] {Measures on Boolean rings, measure algebras};

\draw(105.6, -58.49999999999999) node[anchor=north west,align=left] {Integration and disintegration of measures};

\draw(116.8, -58.49999999999999) node[anchor=north west,align=left] {Measures and integrals in product spaces};

\draw(105.6, -59.199999999999996) node[anchor=north west,align=left] {Real- or complex-valued set functions};

\draw(115.55, -59.199999999999996) node[anchor=north west,align=left] {Abstract differentiation theory,\\ differentiation of set functions};

\draw(105.6, -60.39999999999999) node[anchor=north west,align=left] {Integration with respect to \\ measures and other set functions};

\draw(114.3, -60.39999999999999) node[anchor=north west,align=left] {Hausdorff and packing measures};

\draw(122.5, -60.39999999999999) node[anchor=north west,align=left] {Length, area, volume, other\\ geometric measure theory};

\draw(105.6, -61.599999999999994) node[anchor=north west,align=left] {Contents, measures, outer\\ measures, capacities};

\draw(112.55, -61.599999999999994) node[anchor=north west,align=left] {Lifting theory};

\draw(116.75, -61.599999999999994) node[anchor=north west,align=left] {Fractals};

\draw(69.85, -59.099999999999994) node[anchor=north west] { \large Set functions, measures and integrals with values in abstract spaces};
\draw (69.85, -59.099999999999994) rectangle (94.0,-62.49999999999999);
\draw(70.85, -60.099999999999994) node[anchor=north west,align=left] {Set-valued set functions and measures; integration\\ of set-valued functions; measurable selections};

\draw(84.05, -60.099999999999994) node[anchor=north west,align=left] {Set functions, measures and integrals\\ with values in ordered spaces};

\draw(70.85, -61.3) node[anchor=north west,align=left] {Group- or semigroup-valued set \\ functions, measures and integrals};

\draw(79.8, -61.3) node[anchor=north west,align=left] {Vector-valued set functions,\\ measures and integrals};

\draw(69.85, -62.89999999999999) node[anchor=north west] { \large Measure-theoretic ergodic theory};
\draw (69.85, -62.89999999999999) rectangle (90.0,-66.3);
\draw(70.85, -63.89999999999999) node[anchor=north west,align=left] {One-parameter continuous families \\ of measure-preserving transformations};

\draw(80.8, -63.89999999999999) node[anchor=north west,align=left] {Measure-preserving transformations};

\draw(70.85, -65.1) node[anchor=north west,align=left] {General groups of \\ measure-preserving transformations};

\draw(80.05, -65.1) node[anchor=north west,align=left] {Entropy and other invariants};

\draw(90.1, -62.89999999999999) node[anchor=north west] { \large Miscellaneous topics in measure theory};
\draw (90.1, -62.89999999999999) rectangle (109.75,-65.3);
\draw(91.1, -63.89999999999999) node[anchor=north west,align=left] {Other connections with logic and set theory};

\draw(102.55, -63.89999999999999) node[anchor=north west,align=left] {Nonstandard measure theory};

\draw(91.1, -64.6) node[anchor=north west,align=left] {Fuzzy measure theory};

\draw(90.1, -65.39999999999999) node[anchor=north west,align=left] {History of measure and integration};

\draw(109.85, -62.89999999999999) node[anchor=north west,align=left] {Computational methods for problems \\ pertaining to measure and integration};

\draw(68.85, -66.49999999999999) node[anchor=north west] { \large Associative rings and algebras};
\draw (68.85, -66.49999999999999) rectangle (129.85,-102.6);
\draw(69.85, -67.49999999999999) node[anchor=north west] { \large Chain conditions, growth conditions, and other forms of finiteness for associative rings and algebras};
\draw (69.85, -67.49999999999999) rectangle (100.75,-72.09999999999998);
\draw(70.85, -68.49999999999999) node[anchor=north west,align=left] {Chain conditions on other classes of submodules, ideals,\\ subrings, etc.; coherence (associative rings and algebras)};

\draw(86.05, -68.49999999999999) node[anchor=north west,align=left] {Growth rate, Gelfand-Kirillov dimension};

\draw(70.85, -69.69999999999999) node[anchor=north west,align=left] {Chain conditions on annihilators \\ and summands: Goldie-type conditions};

\draw(80.55, -69.69999999999999) node[anchor=north west,align=left] {Finite rings and finite-dimensional\\ associative algebras};

\draw(70.85, -70.89999999999999) node[anchor=north west,align=left] {Artinian rings and modules \\ (associative rings and algebras)};

\draw(79.55, -70.89999999999999) node[anchor=north west,align=left] {Noetherian rings and modules \\ (associative rings and algebras)};

\draw(88.25, -70.89999999999999) node[anchor=north west,align=left] {Localization and \\ associative Noetherian rings};

\draw(100.85, -67.49999999999999) node[anchor=north west] { \large Homological methods in associative algebras};
\draw (100.85, -67.49999999999999) rectangle (129.75,-73.99999999999999);
\draw(101.85, -68.49999999999999) node[anchor=north west,align=left] {Homological conditions on associative rings (generalizations\\ of regular, Gorenstein, Cohen-Macaulay rings, etc.)};

\draw(117.55, -68.49999999999999) node[anchor=north west,align=left] {(Co)homology of rings and associative algebras\\ (e.g., Hochschild, cyclic, dihedral, etc.)};

\draw(101.85, -69.69999999999999) node[anchor=north west,align=left] {von Neumann regular rings and generalizations\\ (associative algebraic aspects)};

\draw(113.8, -69.69999999999999) node[anchor=north west,align=left] {Differential graded algebras and \\ applications (associative algebraic aspects)};

\draw(101.85, -70.89999999999999) node[anchor=north west,align=left] {Derived categories and associative algebras};

\draw(113.3, -70.89999999999999) node[anchor=north west,align=left] {Grothendieck groups, \(K\)-theory, etc.};

\draw(101.85, -71.59999999999998) node[anchor=north west,align=left] {Semihereditary and hereditary rings, \\ free ideal rings, Sylvester rings, etc.};

\draw(112.3, -71.59999999999998) node[anchor=north west,align=left] {Homological functors on modules (Tor,\\ Ext, etc.) in associative algebras};

\draw(101.85, -72.79999999999998) node[anchor=north west,align=left] {Syzygies, resolutions, complexes\\ in associative algebras};

\draw(110.55, -72.79999999999998) node[anchor=north west,align=left] {Homological dimension\\ in associative algebras};

\draw(69.85, -74.09999999999998) node[anchor=north west] { \large Associative rings and algebras arising under various constructions};
\draw (69.85, -74.09999999999998) rectangle (97.5,-82.99999999999999);
\draw(70.85, -75.09999999999998) node[anchor=north west,align=left] {Associative rings determined by universal properties \\ (free algebras, coproducts, adjunction of inverses, etc.)};

\draw(85.8, -75.09999999999998) node[anchor=north west,align=left] {Finite generation, finite presentability, \\ normal forms (diamond lemma, term-rewriting)};

\draw(70.85, -76.29999999999998) node[anchor=north west,align=left] {Torsion theories; radicals on module \\ categories (associative algebraic aspects)};

\draw(82.05, -76.29999999999998) node[anchor=north west,align=left] {Extensions of associative rings by ideals};

\draw(70.85, -77.49999999999999) node[anchor=north west,align=left] {Centralizing and normalizing extensions};

\draw(81.3, -77.49999999999999) node[anchor=north west,align=left] {Smash products of general Hopf actions};

\draw(70.85, -78.19999999999997) node[anchor=north west,align=left] {Associative rings of functions, \\ subdirect products, sheaves of rings};

\draw(80.55, -78.19999999999997) node[anchor=north west,align=left] {Rings arising from \\ noncommutative algebraic geometry};

\draw(70.85, -79.39999999999998) node[anchor=north west,align=left] {Deformations of associative rings};

\draw(79.8, -79.39999999999998) node[anchor=north west,align=left] {Endomorphism rings; matrix rings};

\draw(88.5, -79.39999999999998) node[anchor=north west,align=left] {Rings of differential operators\\ (associative algebraic aspects)};

\draw(70.85, -80.59999999999998) node[anchor=north west,align=left] {Quadratic and Koszul algebras};

\draw(78.8, -80.59999999999998) node[anchor=north west,align=left] {Ordinary and skew polynomial\\ rings and semigroup rings};

\draw(86.5, -80.59999999999998) node[anchor=north west,align=left] {Associative rings of \\ fractions and localizations};

\draw(70.85, -81.79999999999998) node[anchor=north west,align=left] {Universal enveloping \\ algebras of Lie algebras};

\draw(77.55, -81.79999999999998) node[anchor=north west,align=left] {Twisted and skew group\\ rings, crossed products};

\draw(84.0, -81.79999999999998) node[anchor=north west,align=left] {Leavitt path algebras};

\draw(89.94999999999999, -81.79999999999998) node[anchor=north west,align=left] {Group rings};

\draw(97.6, -74.09999999999998) node[anchor=north west] { \large Associative rings and algebras with additional structure};
\draw (97.6, -74.09999999999998) rectangle (124.25,-79.39999999999998);
\draw(98.6, -75.09999999999998) node[anchor=north west,align=left] {Valuations, completions, formal power series and \\ related constructions (associative rings and algebras)};

\draw(112.8, -75.09999999999998) node[anchor=north west,align=left] {Actions of groups and semigroups; invariant\\ theory (associative rings and algebras)};

\draw(98.6, -76.29999999999998) node[anchor=north west,align=left] {Topological and ordered rings and modules};

\draw(109.55, -76.29999999999998) node[anchor=north west,align=left] {Derivations, actions of Lie algebras};

\draw(98.6, -76.99999999999999) node[anchor=north west,align=left] {Rings with involution; Lie, Jordan\\ and other nonassociative structures};

\draw(108.05, -76.99999999999999) node[anchor=north west,align=left] {Filtered associative rings; \\ filtrational and graded techniques};

\draw(98.6, -78.19999999999997) node[anchor=north west,align=left] {Graded rings and modules \\ (associative rings and algebras)};

\draw(107.3, -78.19999999999997) node[anchor=north west,align=left] {Automorphisms and endomorphisms};

\draw(115.75, -78.19999999999997) node[anchor=north west,align=left] {“Super” (or “skew”) structure};

\draw(97.6, -79.49999999999999) node[anchor=north west] { \large Radicals and radical properties of associative rings};
\draw (97.6, -79.49999999999999) rectangle (118.75,-82.39999999999999);
\draw(98.6, -80.49999999999999) node[anchor=north west,align=left] {General radicals and associative rings};

\draw(108.8, -80.49999999999999) node[anchor=north west,align=left] {Jacobson radical, quasimultiplication};

\draw(98.6, -81.19999999999999) node[anchor=north west,align=left] {Prime and semiprime associative rings};

\draw(108.55, -81.19999999999999) node[anchor=north west,align=left] {Nil and nilpotent radicals, \\ sets, ideals, associative rings};

\draw(69.85, -83.1) node[anchor=north west] { \large Modules, bimodules and ideals in associative algebras};
\draw (69.85, -83.1) rectangle (96.25,-89.39999999999999);
\draw(70.85, -84.1) node[anchor=north west,align=left] {Structure and classification for modules, bimodules\\ and ideals (except as in 16Gxx), direct sum \\ decomposition and cancellation in associative algebras)};

\draw(85.3, -84.1) node[anchor=north west,align=left] {Module categories in associative algebras};

\draw(85.3, -84.8) node[anchor=north west,align=left] {Bimodules in associative algebras};

\draw(70.85, -85.8) node[anchor=north west,align=left] {Simple and semisimple modules, primitive\\ rings and ideals in associative algebras};

\draw(81.55, -85.8) node[anchor=north west,align=left] {Free, projective, and flat modules\\ and ideals in associative algebras};

\draw(70.85, -87.0) node[anchor=north west,align=left] {Injective modules, \\ self-injective associative rings};

\draw(79.55, -87.0) node[anchor=north west,align=left] {Ideals in associative algebras};

\draw(87.75, -87.0) node[anchor=north west,align=left] {Other classes of modules and\\ ideals in associative algebras};

\draw(70.85, -88.19999999999999) node[anchor=north west,align=left] {Infinite-dimensional simple\\ rings (except as in 16Kxx)};

\draw(78.3, -88.19999999999999) node[anchor=north west,align=left] {General module theory\\ in associative algebras};

\draw(96.35, -83.1) node[anchor=north west] { \large Representation theory of associative rings and algebras};
\draw (96.35, -83.1) rectangle (117.75,-87.69999999999999);
\draw(97.35, -84.1) node[anchor=north west,align=left] {Auslander-Reiten sequences (almost split\\ sequences) and Auslander-Reiten quivers};

\draw(108.05, -84.1) node[anchor=north west,align=left] {Representations of orders, lattices,\\ algebras over commutative rings};

\draw(97.35, -85.3) node[anchor=north west,align=left] {Representation type (finite, tame,\\ wild, etc.) of associative algebras};

\draw(106.8, -85.3) node[anchor=north west,align=left] {Representations of \\ associative Artinian rings};

\draw(97.35, -86.5) node[anchor=north west,align=left] {Representations of quivers\\ and partially ordered sets};

\draw(104.55, -86.5) node[anchor=north west,align=left] {Cohen-Macaulay modules\\ in associative algebras};

\draw(96.35, -87.79999999999998) node[anchor=north west,align=left] {History of associative rings and algebras};

\draw(69.85, -89.49999999999999) node[anchor=north west] { \large Rings with polynomial identity};
\draw (69.85, -89.49999999999999) rectangle (91.25,-94.09999999999998);
\draw(70.85, -90.49999999999999) node[anchor=north west,align=left] {Semiprime p.i. rings, rings embeddable\\ in matrices over commutative rings};

\draw(81.05, -90.49999999999999) node[anchor=north west,align=left] {Other kinds of identities (generalized\\ polynomial, rational, involution)};

\draw(70.85, -91.69999999999999) node[anchor=north west,align=left] {\(T\)-ideals, identities, varieties\\ of associative rings and algebras};

\draw(80.3, -91.69999999999999) node[anchor=north west,align=left] {Trace rings and invariant theory\\ (associative rings and algebras)};

\draw(70.85, -92.89999999999999) node[anchor=north west,align=left] {Functional identities \\ (associative rings and algebras)};

\draw(79.55, -92.89999999999999) node[anchor=north west,align=left] {Identities other than those of\\ matrices over commutative rings};

\draw(91.35, -89.49999999999999) node[anchor=north west] { \large Hopf algebras, quantum groups and related topics};
\draw (91.35, -89.49999999999999) rectangle (112.75,-93.09999999999998);
\draw(92.35, -90.49999999999999) node[anchor=north west,align=left] {Ring-theoretic aspects of quantum groups};

\draw(103.05, -90.49999999999999) node[anchor=north west,align=left] {Hopf algebras and their applications};

\draw(92.35, -91.19999999999999) node[anchor=north west,align=left] {Coalgebras and comodules; corings};

\draw(101.3, -91.19999999999999) node[anchor=north west,align=left] {Connections of Hopf \\ algebras with combinatorics};

\draw(92.35, -92.39999999999999) node[anchor=north west,align=left] {Yang-Baxter equations};

\draw(98.3, -92.39999999999999) node[anchor=north west,align=left] {Bialgebras};

\draw(69.85, -94.19999999999999) node[anchor=north west] { \large Conditions on elements};
\draw (69.85, -94.19999999999999) rectangle (90.25,-99.99999999999999);
\draw(70.85, -95.19999999999999) node[anchor=north west,align=left] {Center, normalizer (invariant elements)\\ (associative rings and algebras)};

\draw(81.3, -95.19999999999999) node[anchor=north west,align=left] {Divisibility, noncommutative UFDs};

\draw(70.85, -96.39999999999999) node[anchor=north west,align=left] {Idempotent elements \\ (associative rings and algebras)};

\draw(79.55, -96.39999999999999) node[anchor=north west,align=left] {Units, groups of units \\ (associative rings and algebras)};

\draw(70.85, -97.6) node[anchor=north west,align=left] {Generalizations of commutativity\\ (associative rings and algebras)};

\draw(79.55, -97.6) node[anchor=north west,align=left] {Generalized inverses \\ (associative rings and algebras)};

\draw(70.85, -98.79999999999998) node[anchor=north west,align=left] {Integral domains (associative\\ rings and algebras)};

\draw(78.8, -98.79999999999998) node[anchor=north west,align=left] {Ore rings, multiplicative\\ sets, Ore localization};

\draw(90.35, -94.19999999999999) node[anchor=north west] { \large Division rings and semisimple Artin rings};
\draw (90.35, -94.19999999999999) rectangle (109.25,-97.1);
\draw(91.35, -95.19999999999999) node[anchor=north west,align=left] {Finite-dimensional division rings};

\draw(100.3, -95.19999999999999) node[anchor=north west,align=left] {Brauer groups (algebraic aspects)};

\draw(91.35, -95.89999999999999) node[anchor=north west,align=left] {Infinite-dimensional and\\ general division rings};

\draw(90.35, -97.19999999999999) node[anchor=north west] { \large General and miscellaneous};
\draw (90.35, -97.19999999999999) rectangle (108.75,-99.39999999999999);
\draw(91.35, -98.19999999999999) node[anchor=north west,align=left] {Category-theoretic methods and results in\\ associative algebras (except as in 16D90)};

\draw(102.3, -98.19999999999999) node[anchor=north west,align=left] {Applications of logic\\ in associative algebras};

\draw(109.35, -94.19999999999999) node[anchor=north west] { \large Associative algebras and orders};
\draw (109.35, -94.19999999999999) rectangle (127.75,-97.1);
\draw(110.35, -95.19999999999999) node[anchor=north west,align=left] {Separable algebras (e.g., quaternion\\ algebras, Azumaya algebras, etc.)};

\draw(120.05, -95.19999999999999) node[anchor=north west,align=left] {Orders in separable algebras};

\draw(110.35, -96.39999999999999) node[anchor=north west,align=left] {Lattices over orders};

\draw(116.05, -96.39999999999999) node[anchor=north west,align=left] {Commutative orders};

\draw(69.85, -100.1) node[anchor=north west] { \large Computational aspects of associative rings};
\draw (69.85, -100.1) rectangle (86.25,-102.3);
\draw(70.85, -101.1) node[anchor=north west,align=left] {Computational aspects of \\ associative rings (general theory)};

\draw(80.05, -101.1) node[anchor=north west,align=left] {Gröbner-Shirshov bases};

\draw(86.35, -100.1) node[anchor=north west] { \large Local rings and generalizations};
\draw (86.35, -100.1) rectangle (101.5,-102.3);
\draw(87.35, -101.1) node[anchor=north west,align=left] {Noncommutative local and \\ semilocal rings, perfect rings};

\draw(95.55, -101.1) node[anchor=north west,align=left] {Quasi-Frobenius rings};

\draw(101.6, -100.1) node[anchor=north west] { \large Generalizations};
\draw (101.6, -100.1) rectangle (114.25,-102.5);
\draw(102.6, -101.1) node[anchor=north west,align=left] {\(\Gamma\) and fuzzy structures};

\draw(111.05, -101.1) node[anchor=north west,align=left] {Hyperrings};

\draw(102.6, -101.8) node[anchor=north west,align=left] {Near-rings};

\draw(105.8, -101.8) node[anchor=north west,align=left] {Semirings};

\draw(1, -106.0) node[anchor=north west] { \large Functions of a complex variable};
\draw (1, -106.0) rectangle (61.500000000000014,-136.4);
\draw(2, -107.0) node[anchor=north west] { \large Geometric function theory};
\draw (2, -107.0) rectangle (33.400000000000006,-116.9);
\draw(3, -108.0) node[anchor=north west,align=left] {Special classes of univalent and multivalent functions of one\\ complex variable (starlike, convex, bounded rotation, etc.)};

\draw(18.950000000000003, -108.0) node[anchor=north west,align=left] {Maximum principle, Schwarz’s lemma, Lindelöf \\ principle, analogues and generalizations; subordination};

\draw(3, -109.2) node[anchor=north west,align=left] {Zeros of polynomials, rational functions, and other\\ analytic functions of one complex variable (e.g.,\\ zeros of functions with bounded Dirichlet integral)};

\draw(16.45, -109.2) node[anchor=north west,align=left] {Coefficient problems for univalent and \\ multivalent functions of one complex variable};

\draw(3, -110.9) node[anchor=north west,align=left] {Quasiconformal mappings in the complex plane};

\draw(14.7, -110.9) node[anchor=north west,align=left] {Extremal problems for conformal and \\ quasiconformal mappings, variational methods};

\draw(3, -112.1) node[anchor=north west,align=left] {General theory of univalent and multivalent\\ functions of one complex variable};

\draw(14.45, -112.1) node[anchor=north west,align=left] {Quasiconformal mappings in \\ \(\mathbb{R}^n\), other generalizations};

\draw(3, -113.3) node[anchor=north west,align=left] {Extremal problems for conformal and \\ quasiconformal mappings, other methods};

\draw(13.2, -113.3) node[anchor=north west,align=left] {Conformal mappings of special domains};

\draw(23.15, -113.3) node[anchor=north west,align=left] {General theory of conformal mappings};

\draw(3, -114.5) node[anchor=north west,align=left] {Polynomials and rational \\ functions of one complex variable};

\draw(11.95, -114.5) node[anchor=north west,align=left] {Schwarz-Christoffel-type mappings};

\draw(20.9, -114.5) node[anchor=north west,align=left] {Kernel functions in one complex\\ variable and applications};

\draw(3, -115.7) node[anchor=north west,align=left] {Capacity and harmonic \\ measure in the complex plane};

\draw(10.7, -115.7) node[anchor=north west,align=left] {Covering theorems in \\ conformal mapping theory};

\draw(33.50000000000001, -107.0) node[anchor=north west] { \large Miscellaneous topics of analysis in the complex plane};
\draw (33.50000000000001, -107.0) rectangle (61.40000000000001,-110.4);
\draw(34.50000000000001, -108.0) node[anchor=north west,align=left] {Integration, integrals of Cauchy type, integral \\ representations of analytic functions in the complex plane};

\draw(49.70000000000001, -108.0) node[anchor=north west,align=left] {Boundary value problems in the complex plane};

\draw(34.50000000000001, -109.2) node[anchor=north west,align=left] {Approximation in the complex plane};

\draw(43.7, -109.2) node[anchor=north west,align=left] {Moment problems and interpolation\\ problems in the complex plane};

\draw(52.650000000000006, -109.2) node[anchor=north west,align=left] {Asymptotic representations\\ in the complex plane};

\draw(33.50000000000001, -110.5) node[anchor=north west] { \large Entire and meromorphic functions of one complex variable, and related topics};
\draw (33.50000000000001, -110.5) rectangle (60.900000000000006,-116.3);
\draw(34.50000000000001, -111.5) node[anchor=north west,align=left] {Functional equations in the complex plane, iteration and\\ composition of analytic functions of one complex variable};

\draw(49.45, -111.5) node[anchor=north west,align=left] {Value distribution of meromorphic functions\\ of one complex variable, Nevanlinna theory};

\draw(34.50000000000001, -112.7) node[anchor=north west,align=left] {Cluster sets, prime ends, boundary behavior};

\draw(45.95, -112.7) node[anchor=north west,align=left] {Representations of entire functions of one\\ complex variable by series and integrals};

\draw(34.50000000000001, -113.9) node[anchor=north west,align=left] {Special classes of entire functions of \\ one complex variable and growth estimates};

\draw(45.45, -113.9) node[anchor=north west,align=left] {Quasi-analytic and other classes \\ of functions of one complex variable};

\draw(34.50000000000001, -115.1) node[anchor=north west,align=left] {Meromorphic functions of one \\ complex variable, general theory};

\draw(43.2, -115.1) node[anchor=north west,align=left] {Entire functions of one complex\\ variable, general theory};

\draw(51.650000000000006, -115.1) node[anchor=north west,align=left] {Normal functions of one complex\\ variable, normal families};

\draw(2, -117.0) node[anchor=north west] { \large Riemann surfaces};
\draw (2, -117.0) rectangle (27.65,-123.0);
\draw(3, -118.0) node[anchor=north west,align=left] {Fuchsian groups and automorphic functions (aspects\\ of compact Riemann surfaces and uniformization)};

\draw(16.2, -118.0) node[anchor=north west,align=left] {Compact Riemann surfaces and uniformization};

\draw(3, -119.2) node[anchor=north west,align=left] {Ideal boundary theory for Riemann surfaces};

\draw(14.2, -119.2) node[anchor=north west,align=left] {Classification theory of Riemann surfaces};

\draw(3, -119.9) node[anchor=north west,align=left] {Teichmüller theory for Riemann surfaces};

\draw(13.45, -119.9) node[anchor=north west,align=left] {Harmonic functions on Riemann surfaces};

\draw(3, -120.6) node[anchor=north west,align=left] {Kleinian groups (aspects of compact\\ Riemann surfaces and uniformization)};

\draw(12.7, -120.6) node[anchor=north west,align=left] {Differentials on Riemann surfaces};

\draw(3, -121.8) node[anchor=north west,align=left] {Conformal metrics (hyperbolic,\\ Poincaré, distance functions)};

\draw(11.2, -121.8) node[anchor=north west,align=left] {Klein surfaces};

\draw(27.75, -117.0) node[anchor=north west] { \large Series expansions of functions of one complex variable};
\draw (27.75, -117.0) rectangle (51.65,-122.8);
\draw(28.75, -118.0) node[anchor=north west,align=left] {Random power series in one complex variable};

\draw(40.2, -118.0) node[anchor=north west,align=left] {Completeness problems, closure of a \\ system of functions of one complex variable};

\draw(28.75, -119.2) node[anchor=north west,align=left] {Dirichlet series, exponential series \\ and other series in one complex variable};

\draw(39.45, -119.2) node[anchor=north west,align=left] {Boundary behavior of power series in\\ one complex variable; over-convergence};

\draw(28.75, -120.4) node[anchor=north west,align=left] {Analytic continuation of \\ functions of one complex variable};

\draw(37.7, -120.4) node[anchor=north west,align=left] {Power series (including lacunary\\ series) in one complex variable};

\draw(28.75, -121.6) node[anchor=north west,align=left] {Continued fractions; \\ complex-analytic aspects};

\draw(2, -123.1) node[anchor=north west] { \large Generalized function theory};
\draw (2, -123.1) rectangle (24.65,-127.69999999999999);
\draw(3, -124.1) node[anchor=north west,align=left] {Other generalizations of analytic functions\\ (including abstract-valued functions)};

\draw(14.45, -124.1) node[anchor=north west,align=left] {Generalizations of Bers and Vekua type\\ (pseudoanalytic, \(p\)-analytic, etc.)};

\draw(3, -125.3) node[anchor=north west,align=left] {Functions of hypercomplex \\ variables and generalized variables};

\draw(12.45, -125.3) node[anchor=north west,align=left] {Non-Archimedean function theory};

\draw(3, -126.5) node[anchor=north west,align=left] {Finely holomorphic functions \\ and topological function theory};

\draw(11.45, -126.5) node[anchor=north west,align=left] {Discrete analytic functions};

\draw(24.75, -123.1) node[anchor=north west] { \large Analysis on metric spaces};
\draw (24.75, -123.1) rectangle (46.4,-125.5);
\draw(25.75, -124.1) node[anchor=north west,align=left] {Quasiconformal mappings in metric spaces};

\draw(36.45, -124.1) node[anchor=north west,align=left] {Geometric embeddings of metric spaces};

\draw(25.75, -124.8) node[anchor=north west,align=left] {Inequalities in metric spaces};

\draw(24.75, -125.6) node[anchor=north west,align=left] {Computational methods for problems \\ pertaining to functions of a complex variable};

\draw(24.75, -126.80000000000001) node[anchor=north west,align=left] {History of functions of a complex variable};

\draw(2, -127.8) node[anchor=north west] { \large Spaces and algebras of analytic functions of one complex variable};
\draw (2, -127.8) rectangle (22.1,-133.3);
\draw(3, -128.8) node[anchor=north west,align=left] {Nevanlinna spaces and Smirnov spaces};

\draw(12.7, -128.8) node[anchor=north west,align=left] {Spaces of bounded analytic \\ functions of one complex variable};

\draw(3, -130.0) node[anchor=north west,align=left] {Besov spaces and \(Q_p\)-spaces};

\draw(11.45, -130.0) node[anchor=north west,align=left] {Bergman spaces and Fock spaces};

\draw(3, -130.7) node[anchor=north west,align=left] {Algebras of analytic functions\\ of one complex variable};

\draw(11.2, -130.7) node[anchor=north west,align=left] {de Branges-Rovnyak spaces};

\draw(3, -131.9) node[anchor=north west,align=left] {Corona theorems};

\draw(7.449999999999999, -131.9) node[anchor=north west,align=left] {Zygmund spaces};

\draw(11.649999999999999, -131.9) node[anchor=north west,align=left] {Hardy spaces};

\draw(15.35, -131.9) node[anchor=north west,align=left] {Bloch spaces};

\draw(3, -132.6) node[anchor=north west,align=left] {BMO-spaces};

\draw(22.200000000000003, -127.8) node[anchor=north west] { \large Universal holomorphic functions of one complex variable};
\draw (22.200000000000003, -127.8) rectangle (41.85,-131.2);
\draw(23.200000000000003, -128.8) node[anchor=north west,align=left] {Universal functions of one complex variable};

\draw(34.650000000000006, -128.8) node[anchor=north west,align=left] {Universal Dirichlet series\\ in one complex variable};

\draw(23.200000000000003, -130.0) node[anchor=north west,align=left] {Compositional universality};

\draw(30.400000000000002, -130.0) node[anchor=north west,align=left] {Universal Taylor series\\ in one complex variable};

\draw(41.95, -127.8) node[anchor=north west] { \large General properties of functions of one complex variable};
\draw (41.95, -127.8) rectangle (60.85,-130.0);
\draw(42.95, -128.8) node[anchor=north west,align=left] {Monogenic and polygenic \\ functions of one complex variable};

\draw(51.900000000000006, -128.8) node[anchor=north west,align=left] {Inequalities in the complex plane};

\draw(2, -133.4) node[anchor=north west] { \large Function theory on the disc};
\draw (2, -133.4) rectangle (20.15,-136.3);
\draw(3, -134.4) node[anchor=north west,align=left] {Inner functions of one complex variable};

\draw(13.45, -134.4) node[anchor=north west,align=left] {Singular inner functions\\ of one complex variable};

\draw(3, -135.6) node[anchor=north west,align=left] {Blaschke products};

\draw(61.600000000000016, -106.0) node[anchor=north west] { \large Category theory; homological algebra};
\draw (61.600000000000016, -106.0) rectangle (120.85000000000002,-144.3);
\draw(62.600000000000016, -107.0) node[anchor=north west] { \large General theory of categories and functors};
\draw (62.600000000000016, -107.0) rectangle (93.25000000000003,-113.5);
\draw(63.600000000000016, -108.0) node[anchor=north west,align=left] {Limits and colimits (products, sums, directed limits, pushouts,\\ fiber products, equalizers, kernels, ends and coends, etc.)};

\draw(80.05000000000001, -108.0) node[anchor=north west,align=left] {Categories admitting limits (complete categories),\\ functors preserving limits, completions};

\draw(63.600000000000016, -109.2) node[anchor=north west,align=left] {Adjoint functors (universal constructions, \\ reflective subcategories, Kan extensions, etc.)};

\draw(76.05000000000001, -109.2) node[anchor=north west,align=left] {Factorization systems, substructures, \\ quotient structures, congruences, amalgams};

\draw(63.600000000000016, -110.4) node[anchor=north west,align=left] {Graphs, diagram schemes, precategories};

\draw(73.80000000000001, -110.4) node[anchor=north west,align=left] {Natural morphisms, dinatural morphisms};

\draw(63.600000000000016, -111.1) node[anchor=north west,align=left] {Epimorphisms, monomorphisms, special\\ classes of morphisms, null morphisms};

\draw(73.30000000000001, -111.1) node[anchor=north west,align=left] {Functor categories, comma categories};

\draw(83.00000000000001, -111.1) node[anchor=north west,align=left] {Definitions and generalizations\\ in theory of categories};

\draw(63.600000000000016, -112.3) node[anchor=north west,align=left] {Special properties of \\ functors (faithful, full, etc.)};

\draw(72.05000000000001, -112.3) node[anchor=north west,align=left] {Foundations, relations to\\ logic and deductive systems};

\draw(79.50000000000001, -112.3) node[anchor=north west,align=left] {Graded categories (general)};

\draw(93.35000000000002, -107.0) node[anchor=north west] { \large Categories and theories};
\draw (93.35000000000002, -107.0) rectangle (120.75000000000003,-111.6);
\draw(94.35000000000002, -108.0) node[anchor=north west,align=left] {Monads (= standard construction, triple or triad), algebras\\ for monads, homology and derived functors for monads};

\draw(109.80000000000003, -108.0) node[anchor=north west,align=left] {Categorical semantics of formal languages};

\draw(94.35000000000002, -109.2) node[anchor=north west,align=left] {Theories (e.g., algebraic \\ theories), structure, and semantics};

\draw(103.80000000000003, -109.2) node[anchor=north west,align=left] {Structured objects in a \\ category (group objects, etc.)};

\draw(112.00000000000003, -109.2) node[anchor=north west,align=left] {Sketches and generalizations};

\draw(94.35000000000002, -110.4) node[anchor=north west,align=left] {Eilenberg-Moore and Kleisli\\ constructions for monads};

\draw(101.80000000000003, -110.4) node[anchor=north west,align=left] {Accessible and locally\\ presentable categories};

\draw(108.00000000000003, -110.4) node[anchor=north west,align=left] {Equational categories};

\draw(93.35000000000002, -111.7) node[anchor=north west,align=left] {Computational methods for problems\\ pertaining to category theory};

\draw(93.35000000000002, -112.9) node[anchor=north west,align=left] {History of category theory};

\draw(62.600000000000016, -113.6) node[anchor=north west] { \large Higher categories and homotopical algebra};
\draw (62.600000000000016, -113.6) rectangle (89.00000000000003,-120.1);
\draw(63.600000000000016, -114.6) node[anchor=north west,align=left] {\((\infty,1)\)-categories (quasi-categories, Segal spaces,\\ etc.); \(\infty\)-topoi, stable \(\infty\)-categories};

\draw(78.80000000000001, -114.6) node[anchor=north west,align=left] {Categories of fibrations, relations to\\ \(K\)-theory, relations to type theory};

\draw(63.600000000000016, -115.8) node[anchor=north west,align=left] {Tricategories, weak \(n\)-categories,\\ coherence, semi-strictification};

\draw(73.55000000000001, -115.8) node[anchor=north west,align=left] {Localizations (e.g., simplicial \\ localization, Bousfield localization)};

\draw(63.600000000000016, -117.0) node[anchor=north west,align=left] {\(\infty\)-operads and higher algebra};

\draw(73.55000000000001, -117.0) node[anchor=north west,align=left] {Simplicial sets, simplicial objects};

\draw(63.600000000000016, -117.69999999999999) node[anchor=north west,align=left] {\((\infty, n)\)-categories and\\ \((\infty,\infty)\)-categories};

\draw(71.80000000000001, -117.69999999999999) node[anchor=north west,align=left] {Homotopical algebra, Quillen\\ model categories, derivators};

\draw(79.50000000000001, -117.69999999999999) node[anchor=north west,align=left] {2-categories, bicategories,\\ double categories};

\draw(63.600000000000016, -118.89999999999999) node[anchor=north west,align=left] {2-dimensional monad theory};

\draw(70.80000000000001, -118.89999999999999) node[anchor=north west,align=left] {Strict omega-categories,\\ computads, polygraphs};

\draw(77.50000000000001, -118.89999999999999) node[anchor=north west,align=left] {Categorification};

\draw(89.10000000000002, -113.6) node[anchor=north west] { \large Homological algebra in category theory, derived categories and functors};
\draw (89.10000000000002, -113.6) rectangle (113.25000000000003,-122.0);
\draw(90.10000000000002, -114.6) node[anchor=north west,align=left] {2-groups, crossed modules, crossed complexes};

\draw(101.80000000000003, -114.6) node[anchor=north west,align=left] {Derived categories, triangulated categories};

\draw(90.10000000000002, -115.3) node[anchor=north west,align=left] {Relative homological algebra, projective\\ classes (category-theoretic aspects)};

\draw(100.80000000000003, -115.3) node[anchor=north west,align=left] {Ext and Tor, generalizations, Künneth\\ formula (category-theoretic aspects)};

\draw(90.10000000000002, -116.5) node[anchor=north west,align=left] {\(A_{\infty}\)-categories, relations\\ with homological mirror symmetry};

\draw(99.80000000000003, -116.5) node[anchor=north west,align=left] {Chain complexes (category-theoretic\\ aspects), dg categories};

\draw(90.10000000000002, -117.69999999999999) node[anchor=north west,align=left] {Spectral sequences, hypercohomology};

\draw(99.55000000000003, -117.69999999999999) node[anchor=north west,align=left] {Graph complexes and graph homology};

\draw(90.10000000000002, -118.39999999999999) node[anchor=north west,align=left] {Nonabelian homological algebra\\ (category-theoretic aspects)};

\draw(98.30000000000003, -118.39999999999999) node[anchor=north west,align=left] {Resolutions; derived functors\\ (category-theoretic aspects)};

\draw(90.10000000000002, -119.6) node[anchor=north west,align=left] {Projectives and injectives\\ (category-theoretic aspects)};

\draw(97.80000000000003, -119.6) node[anchor=north west,align=left] {Homological dimension \\ (category-theoretic aspects)};

\draw(105.50000000000003, -119.6) node[anchor=north west,align=left] {Other (co)homology theories\\ (category-theoretic aspects)};

\draw(90.10000000000002, -120.8) node[anchor=north west,align=left] {Stable module categories};

\draw(96.80000000000003, -120.8) node[anchor=north west,align=left] {Simplicial modules and\\ Dold-Kan correspondence};

\draw(62.600000000000016, -122.1) node[anchor=north west] { \large Categorical algebra};
\draw (62.600000000000016, -122.1) rectangle (86.00000000000001,-127.39999999999999);
\draw(63.600000000000016, -123.1) node[anchor=north west,align=left] {Abelian categories, Grothendieck categories};

\draw(75.05000000000001, -123.1) node[anchor=north west,align=left] {Regular categories, Barr-exact categories};

\draw(63.600000000000016, -123.8) node[anchor=north west,align=left] {Protomodular categories, semi-abelian\\ categories, Mal’tsev categories};

\draw(73.55000000000001, -123.8) node[anchor=north west,align=left] {Preadditive, additive categories};

\draw(63.600000000000016, -125.0) node[anchor=north west,align=left] {Categorical embedding theorems};

\draw(71.80000000000001, -125.0) node[anchor=north west,align=left] {Definable subcategories and\\ connections with model theory};

\draw(63.600000000000016, -126.19999999999999) node[anchor=north west,align=left] {Localization of categories,\\ calculus of fractions};

\draw(71.05000000000001, -126.19999999999999) node[anchor=north west,align=left] {Torsion theories, radicals};

\draw(78.25000000000001, -126.19999999999999) node[anchor=north west,align=left] {Categorical Galois theory};

\draw(86.10000000000002, -122.1) node[anchor=north west] { \large Monoidal categories and operads};
\draw (86.10000000000002, -122.1) rectangle (108.75000000000003,-131.7);
\draw(87.10000000000002, -123.1) node[anchor=north west,align=left] {Traced monoidal categories, compact closed\\ categories, star-autonomous categories};

\draw(98.30000000000003, -123.1) node[anchor=north west,align=left] {Non-symmetric operads, multicategories,\\ generalized multicategories};

\draw(87.10000000000002, -124.3) node[anchor=north west,align=left] {Polycategories/dioperads, properads,\\ PROPs, cyclic operads, modular operads};

\draw(97.30000000000003, -124.3) node[anchor=north west,align=left] {String diagrams and graphical calculi};

\draw(87.10000000000002, -125.5) node[anchor=north west,align=left] {Categorical aspects of linear logic};

\draw(96.55000000000003, -125.5) node[anchor=north west,align=left] {Topological and simplicial operads};

\draw(87.10000000000002, -126.19999999999999) node[anchor=north west,align=left] {Fusion categories, modular tensor\\ categories, modular functors};

\draw(96.05000000000003, -126.19999999999999) node[anchor=north west,align=left] {Monoidal categories, \\ symmetric monoidal categories};

\draw(87.10000000000002, -127.39999999999999) node[anchor=north west,align=left] {Dagger categories, \\ categorical quantum mechanics};

\draw(95.05000000000003, -127.39999999999999) node[anchor=north west,align=left] {Algebraic operads, \\ cooperads, and Koszul duality};

\draw(87.10000000000002, -128.6) node[anchor=north west,align=left] {Braided monoidal categories\\ and ribbon categories};

\draw(94.55000000000003, -128.6) node[anchor=north west,align=left] {Categories of networks and\\ processes, compositionality};

\draw(87.10000000000002, -129.79999999999998) node[anchor=north west,align=left] {Bimonoidal, skew-monoidal,\\ duoidal categories};

\draw(94.30000000000003, -129.79999999999998) node[anchor=north west,align=left] {Species, Hopf monoids,\\ operads in combinatorics};

\draw(101.00000000000003, -129.79999999999998) node[anchor=north west,align=left] {Tannakian categories};

\draw(87.10000000000002, -131.0) node[anchor=north west,align=left] {Operads (general)};

\draw(92.05000000000003, -131.0) node[anchor=north west,align=left] {Globular operads};

\draw(62.600000000000016, -131.8) node[anchor=north west] { \large Categories in geometry and topology};
\draw (62.600000000000016, -131.8) rectangle (85.00000000000001,-138.8);
\draw(63.600000000000016, -132.8) node[anchor=north west,align=left] {Synthetic differential geometry, tangent\\ categories, differential categories};

\draw(74.30000000000001, -132.8) node[anchor=north west,align=left] {Goodwillie calculus and functor calculus};

\draw(63.600000000000016, -134.0) node[anchor=north west,align=left] {Presheaves and sheaves, stacks, descent\\ conditions (category-theoretic aspects)};

\draw(74.05000000000001, -134.0) node[anchor=north west,align=left] {Algebraic \(K\)-theory and \(L\)-theory\\ (category-theoretic aspects)};

\draw(63.600000000000016, -135.20000000000002) node[anchor=north west,align=left] {Abstract manifolds and fiber \\ bundles (category-theoretic aspects)};

\draw(73.30000000000001, -135.20000000000002) node[anchor=north west,align=left] {Categories of topological \\ spaces and continuous mappings};

\draw(63.600000000000016, -136.4) node[anchor=north west,align=left] {Local categories and functors};

\draw(71.55000000000001, -136.4) node[anchor=north west,align=left] {Frames and locales, pointfree\\ topology, Stone duality};

\draw(63.600000000000016, -137.60000000000002) node[anchor=north west,align=left] {Grothendieck groups \\ (category-theoretic aspects)};

\draw(71.30000000000001, -137.60000000000002) node[anchor=north west,align=left] {Grothendieck topologies\\ and Grothendieck topoi};

\draw(77.75000000000001, -137.60000000000002) node[anchor=north west,align=left] {Quantales};

\draw(85.10000000000002, -131.8) node[anchor=north west] { \large Special categories};
\draw (85.10000000000002, -131.8) rectangle (106.75000000000003,-137.10000000000002);
\draw(86.10000000000002, -132.8) node[anchor=north west,align=left] {Embedding theorems, universal categories};

\draw(96.80000000000003, -132.8) node[anchor=north west,align=left] {Categories of sets, characterizations};

\draw(86.10000000000002, -133.5) node[anchor=north west,align=left] {Groupoids, semigroupoids, semigroups,\\ groups (viewed as categories)};

\draw(96.05000000000003, -133.5) node[anchor=north west,align=left] {Categories of machines, automata};

\draw(86.10000000000002, -134.70000000000002) node[anchor=north west,align=left] {Preorders, orders, domains and\\ lattices (viewed as categories)};

\draw(94.55000000000003, -134.70000000000002) node[anchor=north west,align=left] {Categories of spans/cospans,\\ relations, or partial maps};

\draw(86.10000000000002, -135.9) node[anchor=north west,align=left] {Extensive, distributive,\\ and adhesive categories};

\draw(92.80000000000003, -135.9) node[anchor=north west,align=left] {Topoi};

\draw(62.600000000000016, -138.9) node[anchor=north west] { \large Categorical structures};
\draw (62.600000000000016, -138.9) rectangle (84.25000000000001,-144.20000000000002);
\draw(63.600000000000016, -139.9) node[anchor=north west,align=left] {Proarrow equipments, Yoneda structures,\\ KZ doctrines (lax idempotent monads)};

\draw(74.05000000000001, -139.9) node[anchor=north west,align=left] {Closed categories (closed monoidal \\ and Cartesian closed categories, etc.)};

\draw(63.600000000000016, -141.1) node[anchor=north west,align=left] {Internal categories and groupoids};

\draw(72.55000000000001, -141.1) node[anchor=north west,align=left] {Profunctors (= correspondences,\\ distributors, modules)};

\draw(63.600000000000016, -142.3) node[anchor=north west,align=left] {Enriched categories (over \\ closed or monoidal categories)};

\draw(71.80000000000001, -142.3) node[anchor=north west,align=left] {Actions of a monoidal \\ category, tensorial strength};

\draw(63.600000000000016, -143.5) node[anchor=north west,align=left] {Formal category theory};

\draw(69.80000000000001, -143.5) node[anchor=north west,align=left] {Fibered categories};

\draw(143.24999999999997, -1) node[anchor=north west] { \large Special functions};
\draw (143.24999999999997, -1) rectangle (202.49999999999997,-17.3);
\draw(144.24999999999997, -2) node[anchor=north west] { \large Hypergeometric functions};
\draw (144.24999999999997, -2) rectangle (174.39999999999998,-9.700000000000001);
\draw(145.24999999999997, -3) node[anchor=north west,align=left] {Orthogonal polynomials and functions in several variables\\ expressible in terms of special functions in one variable};

\draw(160.19999999999996, -3) node[anchor=north west,align=left] {Orthogonal polynomials and functions of hypergeometric\\ type (Jacobi, Laguerre, Hermite, Askey scheme, etc.)};

\draw(145.24999999999997, -4.2) node[anchor=north west,align=left] {Hypergeometric integrals and functions defined \\ by them (\(E\), \(G\), \(H\) and \(I\) functions)};

\draw(158.19999999999996, -4.2) node[anchor=north west,align=left] {Connections of hypergeometric functions \\ with groups and algebras, and related topics};

\draw(145.24999999999997, -5.4) node[anchor=north west,align=left] {Applications of hypergeometric functions};

\draw(155.94999999999996, -5.4) node[anchor=north west,align=left] {Appell, Horn and Lauricella functions};

\draw(145.24999999999997, -6.1000000000000005) node[anchor=north west,align=left] {Orthogonal polynomials and functions\\ associated with root systems};

\draw(154.94999999999996, -6.1000000000000005) node[anchor=north west,align=left] {Confluent hypergeometric functions,\\ Whittaker functions, \({}_1F_1\)};

\draw(164.39999999999998, -6.1000000000000005) node[anchor=north west,align=left] {Other hypergeometric functions \\ and integrals in several variables};

\draw(145.24999999999997, -7.300000000000001) node[anchor=north west,align=left] {Bessel and Airy functions, \\ cylinder functions, \({}_0F_1\)};

\draw(153.69999999999996, -7.300000000000001) node[anchor=north west,align=left] {Hypergeometric functions \\ associated with root systems};

\draw(161.39999999999998, -7.300000000000001) node[anchor=north west,align=left] {Generalized hypergeometric\\ series, \({}_pF_q\)};

\draw(145.24999999999997, -8.5) node[anchor=north west,align=left] {Other special orthogonal\\ polynomials and functions};

\draw(152.19999999999996, -8.5) node[anchor=north west,align=left] {Classical hypergeometric\\ functions, \({}_2F_1\)};

\draw(158.89999999999998, -8.5) node[anchor=north west,align=left] {Elliptic integrals as \\ hypergeometric functions};

\draw(165.59999999999997, -8.5) node[anchor=north west,align=left] {Spherical harmonics};

\draw(174.49999999999997, -2) node[anchor=north west] { \large Basic hypergeometric functions};
\draw (174.49999999999997, -2) rectangle (202.39999999999998,-10.0);
\draw(175.49999999999997, -3) node[anchor=north west,align=left] {Basic orthogonal polynomials and functions associated\\ with root systems (Macdonald polynomials, etc.)};

\draw(189.44999999999996, -3) node[anchor=north west,align=left] {Connections of basic hypergeometric functions\\ with quantum groups, Chevalley groups, \(p\)-adic\\ groups, Hecke algebras, and related topics};

\draw(175.49999999999997, -4.7) node[anchor=north west,align=left] {Orthogonal polynomials and functions in \\ several variables expressible in terms of \\ basic hypergeometric functions in one variable};

\draw(187.69999999999996, -4.7) node[anchor=north west,align=left] {Basic orthogonal polynomials and \\ functions (Askey-Wilson polynomials, etc.)};

\draw(175.49999999999997, -6.4) node[anchor=north west,align=left] {Bibasic functions and multiple bases};

\draw(185.19999999999996, -6.4) node[anchor=north west,align=left] {Other basic hypergeometric functions\\ and integrals in several variables};

\draw(175.49999999999997, -7.6000000000000005) node[anchor=north west,align=left] {\(q\)-gamma functions, \\ \(q\)-beta functions and integrals};

\draw(184.69999999999996, -7.6000000000000005) node[anchor=north west,align=left] {Basic hypergeometric functions\\ in one variable, \({}_r\phi_s\)};

\draw(193.14999999999998, -7.6000000000000005) node[anchor=north west,align=left] {Basic hypergeometric integrals\\ and functions defined by them};

\draw(175.49999999999997, -8.8) node[anchor=north west,align=left] {Basic hypergeometric functions\\ associated with root systems};

\draw(183.69999999999996, -8.8) node[anchor=north west,align=left] {Applications of basic \\ hypergeometric functions};

\draw(144.24999999999997, -10.1) node[anchor=north west] { \large Elementary classical functions};
\draw (144.24999999999997, -10.1) rectangle (169.14999999999998,-13.0);
\draw(145.24999999999997, -11.1) node[anchor=north west,align=left] {Incomplete beta and gamma functions (error \\ functions, probability integral, Fresnel integrals)};

\draw(158.69999999999996, -11.1) node[anchor=north west,align=left] {Exponential and trigonometric functions};

\draw(145.24999999999997, -12.3) node[anchor=north west,align=left] {Gamma, beta and polygamma functions};

\draw(154.69999999999996, -12.3) node[anchor=north west,align=left] {Higher logarithm functions};

\draw(169.24999999999997, -10.1) node[anchor=north west] { \large Other special functions};
\draw (169.24999999999997, -10.1) rectangle (193.64999999999998,-14.899999999999999);
\draw(170.24999999999997, -11.1) node[anchor=north west,align=left] {Lamé, Mathieu, and spheroidal wave functions};

\draw(181.94999999999996, -11.1) node[anchor=north west,align=left] {Mittag-Leffler functions and generalizations};

\draw(170.24999999999997, -11.799999999999999) node[anchor=north west,align=left] {Other functions coming from differential,\\ difference and integral equations};

\draw(181.19999999999996, -11.799999999999999) node[anchor=north west,align=left] {Special functions in characteristic\\ \(p\) (gamma functions, etc.)};

\draw(170.24999999999997, -13.0) node[anchor=north west,align=left] {Elliptic functions and integrals};

\draw(178.94999999999996, -13.0) node[anchor=north west,align=left] {Painlevé-type functions};

\draw(185.39999999999998, -13.0) node[anchor=north west,align=left] {Other functions defined\\ by series and integrals};

\draw(170.24999999999997, -14.2) node[anchor=north west,align=left] {Other wave functions};

\draw(144.24999999999997, -15.0) node[anchor=north west] { \large Computational aspects of special functions};
\draw (144.24999999999997, -15.0) rectangle (164.64999999999998,-17.2);
\draw(145.24999999999997, -16.0) node[anchor=north west,align=left] {Symbolic computation of special functions\\ (Gosper and Zeilberger algorithms, etc.)};

\draw(156.19999999999996, -16.0) node[anchor=north west,align=left] {Numerical approximation and \\ evaluation of special functions};

\draw(164.74999999999997, -15.0) node[anchor=north west,align=left] {History of special functions};

\draw(143.24999999999997, -17.400000000000002) node[anchor=north west] { \large Algebraic geometry};
\draw (143.24999999999997, -17.400000000000002) rectangle (202.0,-73.4);
\draw(144.24999999999997, -18.400000000000002) node[anchor=north west] { \large (Co)homology theory in algebraic geometry};
\draw (144.24999999999997, -18.400000000000002) rectangle (173.14999999999998,-26.1);
\draw(145.24999999999997, -19.400000000000002) node[anchor=north west,align=left] {Other algebro-geometric (co)homologies (e.g., \\ intersection, equivariant, Lawson, Deligne (co)homologies)};

\draw(160.45, -19.400000000000002) node[anchor=north west,align=left] {Differentials and other special sheaves; \\ D-modules; Bernstein-Sato ideals and polynomials};

\draw(145.24999999999997, -20.6) node[anchor=north west,align=left] {Derived categories of sheaves, dg categories,\\ and related constructions in algebraic geometry};

\draw(157.69999999999996, -20.6) node[anchor=north west,align=left] {Topological properties in algebraic geometry};

\draw(145.24999999999997, -21.800000000000004) node[anchor=north west,align=left] {Motivic cohomology; motivic homotopy theory};

\draw(156.69999999999996, -21.800000000000004) node[anchor=north west,align=left] {de Rham cohomology and algebraic geometry};

\draw(145.24999999999997, -22.500000000000004) node[anchor=north west,align=left] {Vanishing theorems in algebraic geometry};

\draw(155.94999999999996, -22.500000000000004) node[anchor=north west,align=left] {Classical real and complex \\ (co)homology in algebraic geometry};

\draw(145.24999999999997, -23.700000000000003) node[anchor=north west,align=left] {Homotopy theory and fundamental\\ groups in algebraic geometry};

\draw(153.69999999999996, -23.700000000000003) node[anchor=north west,align=left] {Sheaves in algebraic geometry};

\draw(161.64999999999998, -23.700000000000003) node[anchor=north west,align=left] {Étale and other Grothendieck\\ topologies and (co)homologies};

\draw(145.24999999999997, -24.900000000000002) node[anchor=north west,align=left] {Brauer groups of schemes};

\draw(151.94999999999996, -24.900000000000002) node[anchor=north west,align=left] {\(p\)-adic cohomology,\\ crystalline cohomology};

\draw(158.14999999999998, -24.900000000000002) node[anchor=north west,align=left] {Multiplier ideals};

\draw(173.24999999999997, -18.400000000000002) node[anchor=north west] { \large Families, fibrations in algebraic geometry};
\draw (173.24999999999997, -18.400000000000002) rectangle (201.89999999999998,-24.200000000000003);
\draw(174.24999999999997, -19.400000000000002) node[anchor=north west,align=left] {Applications of vector bundles and moduli spaces in mathematical\\ physics (twistor theory, instantons, quantum field theory)};

\draw(190.95, -19.400000000000002) node[anchor=north west,align=left] {Arithmetic ground fields (finite, \\ local, global) and families or fibrations};

\draw(174.24999999999997, -20.6) node[anchor=north west,align=left] {Structure of families \\ (Picard-Lefschetz, monodromy, etc.)};

\draw(183.69999999999996, -20.6) node[anchor=north west,align=left] {Formal methods and deformations\\ in algebraic geometry};

\draw(192.14999999999998, -20.6) node[anchor=north west,align=left] {Variation of Hodge structures\\ (algebro-geometric aspects)};

\draw(174.24999999999997, -21.800000000000004) node[anchor=north west,align=left] {Fine and coarse moduli spaces};

\draw(182.19999999999996, -21.800000000000004) node[anchor=north west,align=left] {Geometric Langlands program\\ (algebro-geometric aspects)};

\draw(189.64999999999998, -21.800000000000004) node[anchor=north west,align=left] {Algebraic moduli problems,\\ moduli of vector bundles};

\draw(174.24999999999997, -23.000000000000004) node[anchor=north west,align=left] {Stacks and moduli problems};

\draw(181.44999999999996, -23.000000000000004) node[anchor=north west,align=left] {Fibrations, degenerations\\ in algebraic geometry};

\draw(173.24999999999997, -24.300000000000004) node[anchor=north west,align=left] {History of algebraic geometry};

\draw(144.24999999999997, -26.200000000000003) node[anchor=north west] { \large Arithmetic problems in algebraic geometry; Diophantine geometry};
\draw (144.24999999999997, -26.200000000000003) rectangle (171.89999999999998,-32.900000000000006);
\draw(145.24999999999997, -27.200000000000003) node[anchor=north west,align=left] {Universal profinite groups (relationship to moduli\\ spaces, projective and moduli towers, Galois theory)};

\draw(158.94999999999996, -27.200000000000003) node[anchor=north west,align=left] {Zeta functions and related questions in algebraic\\ geometry (e.g., Birch-Swinnerton-Dyer conjecture)};

\draw(145.24999999999997, -28.400000000000002) node[anchor=north west,align=left] {Finite ground fields in algebraic geometry};

\draw(156.44999999999996, -28.400000000000002) node[anchor=north west,align=left] {Global ground fields in algebraic geometry};

\draw(145.24999999999997, -29.1) node[anchor=north west,align=left] {Perfectoid spaces and mixed characteristic};

\draw(156.44999999999996, -29.1) node[anchor=north west,align=left] {Local ground fields in algebraic geometry};

\draw(145.24999999999997, -29.800000000000004) node[anchor=north west,align=left] {Hasse principle, weak and strong \\ approximation, Brauer-Manin obstruction};

\draw(155.69999999999996, -29.800000000000004) node[anchor=north west,align=left] {Other nonalgebraically closed \\ ground fields in algebraic geometry};

\draw(145.24999999999997, -31.000000000000004) node[anchor=north west,align=left] {Applications to coding theory and\\ cryptography of arithmetic geometry};

\draw(154.69999999999996, -31.000000000000004) node[anchor=north west,align=left] {Arithmetic varieties and \\ schemes; Arakelov theory; heights};

\draw(163.64999999999998, -31.000000000000004) node[anchor=north west,align=left] {Positive characteristic ground\\ fields in algebraic geometry};

\draw(145.24999999999997, -32.2) node[anchor=north west,align=left] {Modular and Shimura varieties};

\draw(153.19999999999996, -32.2) node[anchor=north west,align=left] {Rigid analytic geometry};

\draw(159.64999999999998, -32.2) node[anchor=north west,align=left] {Rational points};

\draw(171.99999999999997, -26.200000000000003) node[anchor=north west] { \large Foundations of algebraic geometry};
\draw (171.99999999999997, -26.200000000000003) rectangle (198.39999999999998,-31.000000000000004);
\draw(172.99999999999997, -27.200000000000003) node[anchor=north west,align=left] {Fundamental constructions in algebraic geometry \\ involving higher and derived categories (homotopical \\ algebraic geometry, derived algebraic geometry, etc.)};

\draw(186.94999999999996, -27.200000000000003) node[anchor=north west,align=left] {Logarithmic algebraic geometry, log schemes};

\draw(186.94999999999996, -27.900000000000002) node[anchor=north west,align=left] {Generalizations (algebraic spaces, stacks)};

\draw(172.99999999999997, -28.900000000000002) node[anchor=north west,align=left] {Elementary questions in algebraic geometry};

\draw(184.19999999999996, -28.900000000000002) node[anchor=north west,align=left] {Geometry over the field with one element};

\draw(172.99999999999997, -29.6) node[anchor=north west,align=left] {Noncommutative algebraic geometry};

\draw(181.94999999999996, -29.6) node[anchor=north west,align=left] {Relevant commutative algebra};

\draw(189.64999999999998, -29.6) node[anchor=north west,align=left] {Varieties and morphisms};

\draw(172.99999999999997, -30.300000000000004) node[anchor=north west,align=left] {Schemes and morphisms};

\draw(144.24999999999997, -33.0) node[anchor=north west] { \large Cycles and subschemes};
\draw (144.24999999999997, -33.0) rectangle (169.89999999999998,-39.0);
\draw(145.24999999999997, -34.0) node[anchor=north west,align=left] {Intersection theory, characteristic classes, \\ intersection multiplicities in algebraic geometry};

\draw(158.19999999999996, -34.0) node[anchor=north west,align=left] {(Equivariant) Chow groups and rings; motives};

\draw(145.24999999999997, -35.2) node[anchor=north west,align=left] {Divisors, linear systems, invertible sheaves};

\draw(156.94999999999996, -35.2) node[anchor=north west,align=left] {Parametrization (Chow and Hilbert schemes)};

\draw(145.24999999999997, -35.9) node[anchor=north west,align=left] {Pencils, nets, webs in algebraic geometry};

\draw(156.19999999999996, -35.9) node[anchor=north west,align=left] {Applications of methods of algebraic\\ \(K\)-theory in algebraic geometry};

\draw(145.24999999999997, -37.1) node[anchor=north west,align=left] {Transcendental methods, Hodge \\ theory (algebro-geometric aspects)};

\draw(154.44999999999996, -37.1) node[anchor=north west,align=left] {Riemann-Roch theorems};

\draw(160.39999999999998, -37.1) node[anchor=north west,align=left] {Algebraic cycles};

\draw(165.09999999999997, -37.1) node[anchor=north west,align=left] {Torelli problem};

\draw(145.24999999999997, -38.3) node[anchor=north west,align=left] {Picard groups};

\draw(169.99999999999997, -33.0) node[anchor=north west] { \large Surfaces and higher-dimensional varieties};
\draw (169.99999999999997, -33.0) rectangle (194.64999999999998,-43.1);
\draw(170.99999999999997, -34.0) node[anchor=north west,align=left] {Vector bundles on surfaces and \\ higher-dimensional varieties, and their moduli};

\draw(183.19999999999996, -34.0) node[anchor=north west,align=left] {Mirror symmetry (algebro-geometric aspects)};

\draw(170.99999999999997, -35.2) node[anchor=north west,align=left] {Topology of surfaces (Donaldson \\ polynomials, Seiberg-Witten invariants)};

\draw(181.44999999999996, -35.2) node[anchor=north west,align=left] {Arithmetic ground fields for surfaces\\ or higher-dimensional varieties};

\draw(170.99999999999997, -36.4) node[anchor=north west,align=left] {\(K3\) surfaces and Enriques surfaces};

\draw(180.94999999999996, -36.4) node[anchor=north west,align=left] {Moduli, classification: analytic \\ theory; relations with modular forms};

\draw(170.99999999999997, -37.6) node[anchor=north west,align=left] {Hypersurfaces and algebraic geometry};

\draw(180.69999999999996, -37.6) node[anchor=north west,align=left] {Holomorphic symplectic \\ varieties, hyper-Kähler varieties};

\draw(170.99999999999997, -38.8) node[anchor=north west,align=left] {Families, moduli, \\ classification: algebraic theory};

\draw(179.69999999999996, -38.8) node[anchor=north west,align=left] {Automorphisms of surfaces and\\ higher-dimensional varieties};

\draw(170.99999999999997, -40.0) node[anchor=north west,align=left] {Singularities of surfaces or\\ higher-dimensional varieties};

\draw(178.69999999999996, -40.0) node[anchor=north west,align=left] {Rational and ruled surfaces};

\draw(186.14999999999998, -40.0) node[anchor=north west,align=left] {Elliptic surfaces, elliptic\\ or Calabi-Yau fibrations};

\draw(170.99999999999997, -41.2) node[anchor=north west,align=left] {Calabi-Yau manifolds \\ (algebro-geometric aspects)};

\draw(178.44999999999996, -41.2) node[anchor=north west,align=left] {Relationships with physics};

\draw(185.64999999999998, -41.2) node[anchor=north west,align=left] {Surfaces of general type};

\draw(170.99999999999997, -42.400000000000006) node[anchor=north west,align=left] {\(n\)-folds (\(n>4\))};

\draw(176.94999999999996, -42.400000000000006) node[anchor=north west,align=left] {Special surfaces};

\draw(181.64999999999998, -42.400000000000006) node[anchor=north west,align=left] {Fano varieties};

\draw(185.84999999999997, -42.400000000000006) node[anchor=north west,align=left] {\(3\)-folds};

\draw(189.29999999999998, -42.400000000000006) node[anchor=north west,align=left] {\(4\)-folds};

\draw(144.24999999999997, -43.2) node[anchor=north west] { \large Projective and enumerative algebraic geometry};
\draw (144.24999999999997, -43.2) rectangle (168.64999999999998,-48.300000000000004);
\draw(145.24999999999997, -44.2) node[anchor=north west,align=left] {Gromov-Witten invariants, quantum cohomology,\\ Gopakumar-Vafa invariants, Donaldson-Thomas\\ invariants (algebro-geometric aspects)};

\draw(157.19999999999996, -44.2) node[anchor=north west,align=left] {Projective techniques in algebraic geometry};

\draw(157.19999999999996, -44.900000000000006) node[anchor=north west,align=left] {Classical problems, Schubert calculus};

\draw(145.24999999999997, -45.900000000000006) node[anchor=north west,align=left] {Enumerative problems (combinatorial\\ problems) in algebraic geometry};

\draw(154.69999999999996, -45.900000000000006) node[anchor=north west,align=left] {Configurations and \\ arrangements of linear subspaces};

\draw(145.24999999999997, -47.1) node[anchor=north west,align=left] {Secant varieties, tensor rank,\\ varieties of sums of powers};

\draw(153.44999999999996, -47.1) node[anchor=north west,align=left] {Varieties of low degree};

\draw(159.89999999999998, -47.1) node[anchor=north west,align=left] {Adjunction problems};

\draw(168.74999999999997, -43.2) node[anchor=north west] { \large Birational geometry};
\draw (168.74999999999997, -43.2) rectangle (192.64999999999998,-48.7);
\draw(169.74999999999997, -44.2) node[anchor=north west,align=left] {Rationality questions in algebraic geometry};

\draw(181.19999999999996, -44.2) node[anchor=north west,align=left] {Ramification problems in algebraic geometry};

\draw(169.74999999999997, -44.900000000000006) node[anchor=north west,align=left] {Global theory and resolution of \\ singularities (algebro-geometric aspects)};

\draw(180.69999999999996, -44.900000000000006) node[anchor=north west,align=left] {Birational automorphisms, \\ Cremona group and generalizations};

\draw(169.74999999999997, -46.1) node[anchor=north west,align=left] {Embeddings in algebraic geometry};

\draw(178.44999999999996, -46.1) node[anchor=north west,align=left] {Coverings in algebraic geometry};

\draw(169.74999999999997, -46.800000000000004) node[anchor=north west,align=left] {Rational and birational maps};

\draw(177.44999999999996, -46.800000000000004) node[anchor=north west,align=left] {Arcs and motivic integration};

\draw(185.14999999999998, -46.800000000000004) node[anchor=north west,align=left] {Minimal model program (Mori\\ theory, extremal rays)};

\draw(169.74999999999997, -48.0) node[anchor=north west,align=left] {McKay correspondence};

\draw(144.24999999999997, -48.80000000000001) node[anchor=north west] { \large Algebraic groups};
\draw (144.24999999999997, -48.80000000000001) rectangle (168.14999999999998,-52.90000000000001);
\draw(145.24999999999997, -49.80000000000001) node[anchor=north west,align=left] {Classical groups (algebro-geometric aspects)};

\draw(156.94999999999996, -49.80000000000001) node[anchor=north west,align=left] {Other algebraic groups (geometric aspects)};

\draw(145.24999999999997, -50.500000000000014) node[anchor=north west,align=left] {Formal groups, \(p\)-divisible groups};

\draw(155.19999999999996, -50.500000000000014) node[anchor=north west,align=left] {Affine algebraic groups,\\ hyperalgebra constructions};

\draw(145.24999999999997, -51.70000000000001) node[anchor=north west,align=left] {Geometric invariant theory};

\draw(152.44999999999996, -51.70000000000001) node[anchor=north west,align=left] {Group actions on varieties\\ or schemes (quotients)};

\draw(159.64999999999998, -51.70000000000001) node[anchor=north west,align=left] {Group varieties};

\draw(164.09999999999997, -51.70000000000001) node[anchor=north west,align=left] {Group schemes};

\draw(168.24999999999997, -48.80000000000001) node[anchor=north west] { \large Local theory in algebraic geometry};
\draw (168.24999999999997, -48.80000000000001) rectangle (191.89999999999998,-53.60000000000001);
\draw(169.24999999999997, -49.80000000000001) node[anchor=north west,align=left] {Infinitesimal methods in algebraic geometry};

\draw(180.69999999999996, -49.80000000000001) node[anchor=north west,align=left] {Formal neighborhoods in algebraic geometry};

\draw(169.24999999999997, -50.500000000000014) node[anchor=north west,align=left] {Local cohomology and algebraic geometry};

\draw(179.69999999999996, -50.500000000000014) node[anchor=north west,align=left] {Local structure of morphisms in \\ algebraic geometry: étale, flat, etc.};

\draw(169.24999999999997, -51.70000000000001) node[anchor=north west,align=left] {Singularities in algebraic geometry};

\draw(178.69999999999996, -51.70000000000001) node[anchor=north west,align=left] {Deformations of singularities};

\draw(169.24999999999997, -52.40000000000001) node[anchor=north west,align=left] {Local deformation theory,\\ Artin approximation, etc.};

\draw(144.24999999999997, -53.7) node[anchor=north west] { \large Curves in algebraic geometry};
\draw (144.24999999999997, -53.7) rectangle (167.89999999999998,-62.300000000000004);
\draw(145.24999999999997, -54.7) node[anchor=north west,align=left] {Theta functions and curves; Schottky problem};

\draw(156.94999999999996, -54.7) node[anchor=north west,align=left] {Vector bundles on curves and their moduli};

\draw(145.24999999999997, -55.400000000000006) node[anchor=north west,align=left] {Families, moduli of curves (algebraic)};

\draw(155.44999999999996, -55.400000000000006) node[anchor=north west,align=left] {Coverings of curves, fundamental group};

\draw(145.24999999999997, -56.1) node[anchor=north west,align=left] {Families, moduli of curves (analytic)};

\draw(155.19999999999996, -56.1) node[anchor=north west,align=left] {Singularities of curves, local rings};

\draw(145.24999999999997, -56.800000000000004) node[anchor=north west,align=left] {Arithmetic ground fields for curves};

\draw(154.69999999999996, -56.800000000000004) node[anchor=north west,align=left] {Algebraic functions and function\\ fields in algebraic geometry};

\draw(145.24999999999997, -58.0) node[anchor=north west,align=left] {Special divisors on curves \\ (gonality, Brill-Noether theory)};

\draw(153.94999999999996, -58.0) node[anchor=north west,align=left] {Relationships between algebraic\\ curves and integrable systems};

\draw(145.24999999999997, -59.2) node[anchor=north west,align=left] {Riemann surfaces; Weierstrass\\ points; gap sequences};

\draw(153.19999999999996, -59.2) node[anchor=north west,align=left] {Relationships between \\ algebraic curves and physics};

\draw(160.89999999999998, -59.2) node[anchor=north west,align=left] {Jacobians, Prym varieties};

\draw(145.24999999999997, -60.400000000000006) node[anchor=north west,align=left] {Special algebraic curves\\ and curves of low genus};

\draw(151.94999999999996, -60.400000000000006) node[anchor=north west,align=left] {Dessins d’enfants theory};

\draw(158.64999999999998, -60.400000000000006) node[anchor=north west,align=left] {Automorphisms of curves};

\draw(145.24999999999997, -61.6) node[anchor=north west,align=left] {Plane and space curves};

\draw(151.44999999999996, -61.6) node[anchor=north west,align=left] {Elliptic curves};

\draw(167.99999999999997, -53.7) node[anchor=north west] { \large Computational aspects in algebraic geometry};
\draw (167.99999999999997, -53.7) rectangle (191.39999999999998,-59.0);
\draw(168.99999999999997, -54.7) node[anchor=north west,align=left] {Computational aspects of algebraic surfaces};

\draw(180.44999999999996, -54.7) node[anchor=north west,align=left] {Computational aspects of algebraic curves};

\draw(168.99999999999997, -55.400000000000006) node[anchor=north west,align=left] {Effectivity, complexity and computational\\ aspects of algebraic geometry};

\draw(179.94999999999996, -55.400000000000006) node[anchor=north west,align=left] {Computational real algebraic geometry};

\draw(168.99999999999997, -56.6) node[anchor=north west,align=left] {Computational algebraic geometry\\ over arithmetic ground fields};

\draw(177.69999999999996, -56.6) node[anchor=north west,align=left] {Computational aspects of \\ higher-dimensional varieties};

\draw(168.99999999999997, -57.800000000000004) node[anchor=north west,align=left] {Geometric aspects of \\ numerical algebraic geometry};

\draw(167.99999999999997, -59.10000000000001) node[anchor=north west] { \large Affine geometry};
\draw (167.99999999999997, -59.10000000000001) rectangle (189.14999999999998,-62.00000000000001);
\draw(168.99999999999997, -60.10000000000001) node[anchor=north west,align=left] {Affine spaces (automorphisms, embeddings,\\ exotic structures, cancellation problem)};

\draw(179.94999999999996, -60.10000000000001) node[anchor=north west,align=left] {Classification of affine varieties};

\draw(168.99999999999997, -61.30000000000001) node[anchor=north west,align=left] {Group actions on affine varieties};

\draw(177.94999999999996, -61.30000000000001) node[anchor=north west,align=left] {Affine fibrations};

\draw(182.89999999999998, -61.30000000000001) node[anchor=north west,align=left] {Jacobian problem};

\draw(144.24999999999997, -62.400000000000006) node[anchor=north west] { \large Tropical geometry};
\draw (144.24999999999997, -62.400000000000006) rectangle (167.39999999999998,-66.0);
\draw(145.24999999999997, -63.400000000000006) node[anchor=north west,align=left] {Combinatorial aspects of tropical varieties};

\draw(156.69999999999996, -63.400000000000006) node[anchor=north west,align=left] {Arithmetic aspects of tropical varieties};

\draw(145.24999999999997, -64.10000000000001) node[anchor=north west,align=left] {Geometric aspects of tropical varieties};

\draw(155.69999999999996, -64.10000000000001) node[anchor=north west,align=left] {Applications of tropical geometry};

\draw(145.24999999999997, -64.80000000000001) node[anchor=north west,align=left] {Foundations of tropical geometry\\ and relations with algebra};

\draw(167.49999999999997, -62.400000000000006) node[anchor=north west] { \large Abelian varieties and schemes};
\draw (167.49999999999997, -62.400000000000006) rectangle (190.14999999999998,-67.2);
\draw(168.49999999999997, -63.400000000000006) node[anchor=north west,align=left] {Complex multiplication and abelian varieties};

\draw(180.19999999999996, -63.400000000000006) node[anchor=north west,align=left] {Algebraic theory of abelian varieties};

\draw(168.49999999999997, -64.10000000000001) node[anchor=north west,align=left] {Analytic theory of abelian varieties;\\ abelian integrals and differentials};

\draw(178.44999999999996, -64.10000000000001) node[anchor=north west,align=left] {Theta functions and abelian varieties};

\draw(168.49999999999997, -65.30000000000001) node[anchor=north west,align=left] {Subvarieties of abelian varieties};

\draw(177.44999999999996, -65.30000000000001) node[anchor=north west,align=left] {Picard schemes, higher Jacobians};

\draw(168.49999999999997, -66.0) node[anchor=north west,align=left] {Algebraic moduli of abelian\\ varieties, classification};

\draw(175.94999999999996, -66.0) node[anchor=north west,align=left] {Arithmetic ground fields\\ for abelian varieties};

\draw(182.64999999999998, -66.0) node[anchor=north west,align=left] {Isogeny};

\draw(144.24999999999997, -67.3) node[anchor=north west] { \large Special varieties};
\draw (144.24999999999997, -67.3) rectangle (165.89999999999998,-73.3);
\draw(145.24999999999997, -68.3) node[anchor=north west,align=left] {Varieties defined by ring conditions \\ (factorial, Cohen-Macaulay, seminormal)};

\draw(155.69999999999996, -68.3) node[anchor=north west,align=left] {Homogeneous spaces and generalizations};

\draw(145.24999999999997, -69.5) node[anchor=north west,align=left] {Rational and unirational varieties};

\draw(154.44999999999996, -69.5) node[anchor=north west,align=left] {Rationally connected varieties};

\draw(145.24999999999997, -70.2) node[anchor=north west,align=left] {Compactifications; symmetric\\ and spherical varieties};

\draw(152.94999999999996, -70.2) node[anchor=north west,align=left] {Toric varieties, Newton \\ polyhedra, Okounkov bodies};

\draw(145.24999999999997, -71.39999999999999) node[anchor=north west,align=left] {Grassmannians, Schubert\\ varieties, flag manifolds};

\draw(152.19999999999996, -71.39999999999999) node[anchor=north west,align=left] {Low codimension problems\\ in algebraic geometry};

\draw(158.89999999999998, -71.39999999999999) node[anchor=north west,align=left] {Determinantal varieties};

\draw(145.24999999999997, -72.6) node[anchor=north west,align=left] {Complete intersections};

\draw(151.44999999999996, -72.6) node[anchor=north west,align=left] {Character varieties};

\draw(156.89999999999998, -72.6) node[anchor=north west,align=left] {Supervarieties};

\draw(161.09999999999997, -72.6) node[anchor=north west,align=left] {Linkage};

\draw(165.99999999999997, -67.3) node[anchor=north west] { \large Real algebraic and real-analytic geometry};
\draw (165.99999999999997, -67.3) rectangle (186.64999999999998,-70.39999999999999);
\draw(166.99999999999997, -68.3) node[anchor=north west,align=left] {Semialgebraic sets and related spaces};

\draw(176.94999999999996, -68.3) node[anchor=north west,align=left] {Real-analytic and semi-analytic sets};

\draw(166.99999999999997, -69.0) node[anchor=north west,align=left] {Topology of real algebraic varieties};

\draw(176.69999999999996, -69.0) node[anchor=north west,align=left] {Nash functions and manifolds};

\draw(166.99999999999997, -69.7) node[anchor=north west,align=left] {Real algebraic sets};

\draw(143.24999999999997, -73.5) node[anchor=north west] { \large Real functions};
\draw (143.24999999999997, -73.5) rectangle (201.64999999999998,-90.1);
\draw(144.24999999999997, -74.5) node[anchor=north west] { \large Functions of one variable};
\draw (144.24999999999997, -74.5) rectangle (175.79999999999995,-83.9);
\draw(145.24999999999997, -75.5) node[anchor=north west,align=left] {Nondifferentiability (nondifferentiable functions, \\ points of nondifferentiability), discontinuous derivatives};

\draw(160.45, -75.5) node[anchor=north west,align=left] {Differentiation (real functions of one variable): general\\ theory, generalized derivatives, mean value theorems};

\draw(145.24999999999997, -76.7) node[anchor=north west,align=left] {Continuity and related questions (modulus \\ of continuity, semicontinuity, discontinuities,\\ etc.) for real functions in one variable};

\draw(157.69999999999996, -76.7) node[anchor=north west,align=left] {Iteration of real functions in one variable};

\draw(157.69999999999996, -77.4) node[anchor=north west,align=left] {Fractional derivatives and integrals};

\draw(145.24999999999997, -78.4) node[anchor=north west,align=left] {Foundations: limits and generalizations,\\ elementary topology of the line};

\draw(155.94999999999996, -78.4) node[anchor=north west,align=left] {Classification of real functions; Baire\\ classification of sets and functions};

\draw(145.24999999999997, -79.6) node[anchor=north west,align=left] {Singular functions, Cantor functions,\\ functions with other special properties};

\draw(155.69999999999996, -79.6) node[anchor=north west,align=left] {Rate of growth of functions, orders\\ of infinity, slowly varying functions};

\draw(165.64999999999998, -79.6) node[anchor=north west,align=left] {Monotonic functions, generalizations};

\draw(145.24999999999997, -80.8) node[anchor=north west,align=left] {Convexity of real functions in\\ one variable, generalizations};

\draw(153.44999999999996, -80.8) node[anchor=north west,align=left] {Denjoy and Perron integrals,\\ other special integrals};

\draw(161.14999999999998, -80.8) node[anchor=north west,align=left] {Integrals of Riemann, \\ Stieltjes and Lebesgue type};

\draw(168.59999999999997, -80.8) node[anchor=north west,align=left] {Lipschitz (Hölder) classes};

\draw(145.24999999999997, -82.0) node[anchor=north west,align=left] {Functions of bounded \\ variation, generalizations};

\draw(152.44999999999996, -82.0) node[anchor=north west,align=left] {Absolutely continuous real\\ functions in one variable};

\draw(159.64999999999998, -82.0) node[anchor=north west,align=left] {One-variable calculus};

\draw(165.59999999999997, -82.0) node[anchor=north west,align=left] {Elementary functions};

\draw(145.24999999999997, -83.2) node[anchor=north west,align=left] {Antidifferentiation};

\draw(175.89999999999998, -74.5) node[anchor=north west] { \large Functions of several variables};
\draw (175.89999999999998, -74.5) rectangle (201.54999999999998,-80.3);
\draw(176.89999999999998, -75.5) node[anchor=north west,align=left] {Absolutely continuous real functions of several\\ variables, functions of bounded variation};

\draw(189.34999999999997, -75.5) node[anchor=north west,align=left] {Integral formulas of real functions of \\ several variables (Stokes, Gauss, Green, etc.)};

\draw(176.89999999999998, -76.7) node[anchor=north west,align=left] {Special properties of functions of \\ several variables, Hölder conditions, etc.};

\draw(188.09999999999997, -76.7) node[anchor=north west,align=left] {Continuity and differentiation questions};

\draw(176.89999999999998, -77.9) node[anchor=north west,align=left] {Integration of real functions of \\ several variables: length, area, volume};

\draw(187.34999999999997, -77.9) node[anchor=north west,align=left] {Implicit function theorems, Jacobians,\\ transformations with several variables};

\draw(176.89999999999998, -79.1) node[anchor=north west,align=left] {Convexity of real functions of \\ several variables, generalizations};

\draw(186.09999999999997, -79.1) node[anchor=north west,align=left] {Calculus of vector functions};

\draw(193.79999999999998, -79.1) node[anchor=north west,align=left] {Representation and \\ superposition of functions};

\draw(175.89999999999998, -80.4) node[anchor=north west] { \large Polynomials, rational functions in real analysis};
\draw (175.89999999999998, -80.4) rectangle (197.79999999999998,-82.80000000000001);
\draw(176.89999999999998, -81.4) node[anchor=north west,align=left] {Real polynomials: analytic properties, etc.};

\draw(188.34999999999997, -81.4) node[anchor=north west,align=left] {Real polynomials: location of zeros};

\draw(176.89999999999998, -82.10000000000001) node[anchor=north west,align=left] {Real rational functions};

\draw(175.89999999999998, -82.9) node[anchor=north west,align=left] {History of real functions};

\draw(144.24999999999997, -84.0) node[anchor=north west] { \large Inequalities in real analysis};
\draw (144.24999999999997, -84.0) rectangle (166.89999999999998,-88.6);
\draw(145.24999999999997, -85.0) node[anchor=north west,align=left] {Inequalities for sums, series and integrals};

\draw(156.69999999999996, -85.0) node[anchor=north west,align=left] {Inequalities involving derivatives and\\ differential and integral operators};

\draw(145.24999999999997, -86.2) node[anchor=north west,align=left] {Inequalities for trigonometric\\ functions and polynomials};

\draw(153.44999999999996, -86.2) node[anchor=north west,align=left] {Other analytical inequalities};

\draw(145.24999999999997, -87.4) node[anchor=north west,align=left] {Inequalities involving\\ other types of functions};

\draw(166.99999999999997, -84.0) node[anchor=north west] { \large Miscellaneous topics in real functions};
\draw (166.99999999999997, -84.0) rectangle (184.89999999999998,-90.0);
\draw(167.99999999999997, -85.0) node[anchor=north west,align=left] {Calculus of functions taking values\\ in infinite-dimensional spaces};

\draw(177.44999999999996, -85.0) node[anchor=north west,align=left] {Calculus of functions on \\ infinite-dimensional spaces};

\draw(167.99999999999997, -86.2) node[anchor=north west,align=left] {Constructive real analysis};

\draw(175.19999999999996, -86.2) node[anchor=north west,align=left] {\(C^\infty\)-functions,\\ quasi-analytic functions};

\draw(167.99999999999997, -87.4) node[anchor=north west,align=left] {Non-Archimedean analysis};

\draw(174.69999999999996, -87.4) node[anchor=north west,align=left] {Real analysis on time \\ scales or measure chains};

\draw(167.99999999999997, -88.6) node[anchor=north west,align=left] {Real-analytic functions};

\draw(174.44999999999996, -88.6) node[anchor=north west,align=left] {Set-valued functions};

\draw(167.99999999999997, -89.3) node[anchor=north west,align=left] {Nonstandard analysis};

\draw(173.69999999999996, -89.3) node[anchor=north west,align=left] {Fuzzy real analysis};

\draw(179.14999999999998, -89.3) node[anchor=north west,align=left] {Means};

\draw(144.24999999999997, -88.7) node[anchor=north west,align=left] {Computational methods for problems\\ pertaining to real functions};

\draw(143.24999999999997, -90.2) node[anchor=north west] { \large Commutative algebra};
\draw (143.24999999999997, -90.2) rectangle (200.99999999999997,-119.80000000000001);
\draw(144.24999999999997, -91.2) node[anchor=north west] { \large General commutative ring theory};
\draw (144.24999999999997, -91.2) rectangle (173.39999999999998,-96.5);
\draw(145.24999999999997, -92.2) node[anchor=north west,align=left] {General commutative ring theory and combinatorics \\ (zero-divisor graphs, annihilating-ideal graphs, etc.)};

\draw(159.44999999999996, -92.2) node[anchor=north west,align=left] {Characteristic \(p\) methods (Frobenius endomorphism)\\ and reduction to characteristic \(p\); tight closure};

\draw(145.24999999999997, -93.4) node[anchor=north west,align=left] {Associated graded rings of ideals (Rees ring,\\ form ring), analytic spread and related topics};

\draw(157.44999999999996, -93.4) node[anchor=north west,align=left] {Valuations and their generalizations\\ for commutative rings};

\draw(145.24999999999997, -94.60000000000001) node[anchor=north west,align=left] {Actions of groups on commutative\\ rings; invariant theory};

\draw(153.94999999999996, -94.60000000000001) node[anchor=north west,align=left] {Divisibility and factorizations\\ in commutative rings};

\draw(162.39999999999998, -94.60000000000001) node[anchor=north west,align=left] {Ideals and multiplicative ideal\\ theory in commutative rings};

\draw(145.24999999999997, -95.8) node[anchor=north west,align=left] {Graded rings};

\draw(173.49999999999997, -91.2) node[anchor=north west] { \large Arithmetic rings and other special commutative rings};
\draw (173.49999999999997, -91.2) rectangle (200.89999999999998,-97.7);
\draw(174.49999999999997, -92.2) node[anchor=north west,align=left] {Commutative rings defined by factorization \\ properties (e.g., atomic, factorial, half-factorial)};

\draw(188.19999999999996, -92.2) node[anchor=north west,align=left] {Commutative rings defined by monomial ideals; \\ Stanley-Reisner face rings; simplicial complexes};

\draw(174.49999999999997, -93.4) node[anchor=north west,align=left] {Euclidean rings and generalizations};

\draw(183.94999999999996, -93.4) node[anchor=north west,align=left] {Polynomial rings and ideals; rings\\ of integer-valued polynomials};

\draw(174.49999999999997, -94.60000000000001) node[anchor=north west,align=left] {Commutative rings defined by \\ binomial ideals, toric rings, etc.};

\draw(183.69999999999996, -94.60000000000001) node[anchor=north west,align=left] {Dedekind, Prüfer, Krull and Mori\\ rings and their generalizations};

\draw(192.39999999999998, -94.60000000000001) node[anchor=north west,align=left] {Other commutative rings defined\\ by combinatorial properties};

\draw(174.49999999999997, -95.8) node[anchor=north west,align=left] {Witt vectors and related rings};

\draw(182.69999999999996, -95.8) node[anchor=north west,align=left] {Formal power series rings};

\draw(189.64999999999998, -95.8) node[anchor=north west,align=left] {Rings with straightening\\ laws, Hodge algebras};

\draw(174.49999999999997, -97.0) node[anchor=north west,align=left] {Principal ideal rings};

\draw(180.44999999999996, -97.0) node[anchor=north west,align=left] {Seminormal rings};

\draw(185.14999999999998, -97.0) node[anchor=north west,align=left] {Cluster algebras};

\draw(189.84999999999997, -97.0) node[anchor=north west,align=left] {Valuation rings};

\draw(194.29999999999998, -97.0) node[anchor=north west,align=left] {Excellent rings};

\draw(144.24999999999997, -97.8) node[anchor=north west] { \large Homological methods in commutative ring theory};
\draw (144.24999999999997, -97.8) rectangle (170.64999999999998,-105.5);
\draw(145.24999999999997, -98.8) node[anchor=north west,align=left] {(Co)homology of commutative rings and algebras (e.g.,\\ Hochschild, André-Quillen, cyclic, dihedral, etc.)};

\draw(159.19999999999996, -98.8) node[anchor=north west,align=left] {Homological dimension and commutative rings};

\draw(145.24999999999997, -100.0) node[anchor=north west,align=left] {Derived categories and commutative rings};

\draw(155.94999999999996, -100.0) node[anchor=north west,align=left] {Local cohomology and commutative rings};

\draw(145.24999999999997, -100.7) node[anchor=north west,align=left] {Homological conjectures (intersection\\ theorems) in commutative ring theory};

\draw(155.19999999999996, -100.7) node[anchor=north west,align=left] {Torsion theory for commutative rings};

\draw(145.24999999999997, -101.89999999999999) node[anchor=north west,align=left] {Homological functors on modules of\\ commutative rings (Tor, Ext, etc.)};

\draw(154.44999999999996, -101.89999999999999) node[anchor=north west,align=left] {Deformations and infinitesimal \\ methods in commutative ring theory};

\draw(145.24999999999997, -103.1) node[anchor=north west,align=left] {Grothendieck groups, \\ \(K\)-theory and commutative rings};

\draw(154.44999999999996, -103.1) node[anchor=north west,align=left] {Syzygies, resolutions, \\ complexes and commutative rings};

\draw(145.24999999999997, -104.3) node[anchor=north west,align=left] {Hilbert-Samuel and Hilbert-Kunz\\ functions; Poincaré series};

\draw(170.74999999999997, -97.8) node[anchor=north west] { \large Chain conditions, finiteness conditions in commutative ring theory};
\draw (170.74999999999997, -97.8) rectangle (195.14999999999998,-101.2);
\draw(171.74999999999997, -98.8) node[anchor=north west,align=left] {Commutative rings and modules of finite \\ generation or presentation; number of generators};

\draw(184.44999999999996, -98.8) node[anchor=north west,align=left] {Commutative Noetherian rings and modules};

\draw(171.74999999999997, -100.0) node[anchor=north west,align=left] {Commutative Artinian rings and \\ modules, finite-dimensional algebras};

\draw(170.74999999999997, -101.3) node[anchor=north west] { \large Computational aspects and applications of commutative rings};
\draw (170.74999999999997, -101.3) rectangle (195.14999999999998,-105.39999999999999);
\draw(171.74999999999997, -102.3) node[anchor=north west,align=left] {Applications of commutative algebra (e.g., to\\ statistics, control theory, optimization, etc.)};

\draw(184.19999999999996, -102.3) node[anchor=north west,align=left] {Gröbner bases; other bases for ideals and\\ modules (e.g., Janet and border bases)};

\draw(171.74999999999997, -103.5) node[anchor=north west,align=left] {Solving polynomial systems; resultants};

\draw(181.94999999999996, -103.5) node[anchor=north west,align=left] {Computational homological algebra};

\draw(171.74999999999997, -104.2) node[anchor=north west,align=left] {Polynomials, factorization\\ in commutative rings};

\draw(144.24999999999997, -105.60000000000001) node[anchor=north west] { \large Theory of modules and ideals in commutative rings};
\draw (144.24999999999997, -105.60000000000001) rectangle (167.14999999999998,-112.60000000000001);
\draw(145.24999999999997, -106.60000000000001) node[anchor=north west,align=left] {Theory of modules and ideals in commutative\\ rings described by combinatorial properties};

\draw(156.69999999999996, -106.60000000000001) node[anchor=north west,align=left] {Structure, classification theorems for\\ modules and ideals in commutative rings};

\draw(145.24999999999997, -107.80000000000001) node[anchor=north west,align=left] {Module categories and commutative rings};

\draw(155.69999999999996, -107.80000000000001) node[anchor=north west,align=left] {Dimension theory, depth, related\\ commutative rings (catenary, etc.)};

\draw(145.24999999999997, -109.00000000000001) node[anchor=north west,align=left] {Projective and free modules \\ and ideals in commutative rings};

\draw(153.69999999999996, -109.00000000000001) node[anchor=north west,align=left] {Other special types of modules\\ and ideals in commutative rings};

\draw(145.24999999999997, -110.2) node[anchor=north west,align=left] {Linkage, complete intersections\\ and determinantal ideals};

\draw(153.69999999999996, -110.2) node[anchor=north west,align=left] {Injective and flat modules and\\ ideals in commutative rings};

\draw(145.24999999999997, -111.4) node[anchor=north west,align=left] {Torsion modules and \\ ideals in commutative rings};

\draw(152.69999999999996, -111.4) node[anchor=north west,align=left] {Cohen-Macaulay modules};

\draw(158.89999999999998, -111.4) node[anchor=north west,align=left] {Class groups};

\draw(167.24999999999997, -105.60000000000001) node[anchor=north west] { \large Finite commutative rings};
\draw (167.24999999999997, -105.60000000000001) rectangle (188.89999999999998,-107.30000000000001);
\draw(168.24999999999997, -106.60000000000001) node[anchor=north west,align=left] {Polynomials and finite commutative rings};

\draw(178.94999999999996, -106.60000000000001) node[anchor=north west,align=left] {Structure of finite commutative rings};

\draw(167.24999999999997, -107.4) node[anchor=north west] { \large Local rings and semilocal rings};
\draw (167.24999999999997, -107.4) rectangle (186.64999999999998,-110.30000000000001);
\draw(168.24999999999997, -108.4) node[anchor=north west,align=left] {Multiplicity theory and related topics};

\draw(178.44999999999996, -108.4) node[anchor=north west,align=left] {Special types (Cohen-Macaulay,\\ Gorenstein, Buchsbaum, etc.)};

\draw(168.24999999999997, -109.60000000000001) node[anchor=north west,align=left] {Regular local rings};

\draw(167.24999999999997, -110.4) node[anchor=north west] { \large Applications of logic to commutative algebra};
\draw (167.24999999999997, -110.4) rectangle (181.04999999999998,-112.10000000000001);
\draw(168.24999999999997, -111.4) node[anchor=north west,align=left] {Applications of logic to commutative algebra};

\draw(144.24999999999997, -112.7) node[anchor=north west] { \large Commutative ring extensions and related topics};
\draw (144.24999999999997, -112.7) rectangle (164.39999999999998,-119.7);
\draw(145.24999999999997, -113.7) node[anchor=north west,align=left] {Extension theory of commutative rings};

\draw(155.19999999999996, -113.7) node[anchor=north west,align=left] {Integral dependence in commutative\\ rings; going up, going down};

\draw(145.24999999999997, -114.9) node[anchor=north west,align=left] {Polynomials over commutative rings};

\draw(154.44999999999996, -114.9) node[anchor=north west,align=left] {Rings of fractions and \\ localization for commutative rings};

\draw(145.24999999999997, -116.10000000000001) node[anchor=north west,align=left] {Étale and flat extensions; \\ Henselization; Artin approximation};

\draw(154.44999999999996, -116.10000000000001) node[anchor=north west,align=left] {Completion of commutative rings};

\draw(145.24999999999997, -117.3) node[anchor=north west,align=left] {Morphisms of commutative rings};

\draw(153.44999999999996, -117.3) node[anchor=north west,align=left] {Integral closure of \\ commutative rings and ideals};

\draw(145.24999999999997, -118.5) node[anchor=north west,align=left] {Galois theory and \\ commutative ring extensions};

\draw(164.49999999999997, -112.7) node[anchor=north west] { \large Differential algebra};
\draw (164.49999999999997, -112.7) rectangle (183.39999999999998,-115.60000000000001);
\draw(165.49999999999997, -113.7) node[anchor=north west,align=left] {Commutative rings of differential\\ operators and their modules};

\draw(174.44999999999996, -113.7) node[anchor=north west,align=left] {Derivations and commutative rings};

\draw(165.49999999999997, -114.9) node[anchor=north west,align=left] {Modules of differentials};

\draw(164.49999999999997, -115.7) node[anchor=north west] { \large Topological rings and modules};
\draw (164.49999999999997, -115.7) rectangle (180.64999999999998,-118.8);
\draw(165.49999999999997, -116.7) node[anchor=north west,align=left] {Analytical algebras and rings};

\draw(173.44999999999996, -116.7) node[anchor=north west,align=left] {Complete rings, completion};

\draw(165.49999999999997, -117.4) node[anchor=north west,align=left] {Global topological rings};

\draw(172.19999999999996, -117.4) node[anchor=north west,align=left] {Power series rings};

\draw(165.49999999999997, -118.10000000000001) node[anchor=north west,align=left] {Henselian rings};

\draw(169.94999999999996, -118.10000000000001) node[anchor=north west,align=left] {Ordered rings};

\draw(173.89999999999998, -118.10000000000001) node[anchor=north west,align=left] {Real algebra};

\draw(164.49999999999997, -118.9) node[anchor=north west,align=left] {History of commutative algebra};

\draw(183.49999999999997, -112.7) node[anchor=north west] { \large Integral domains};
\draw (183.49999999999997, -112.7) rectangle (189.19999999999996,-114.4);
\draw(184.49999999999997, -113.7) node[anchor=north west,align=left] {Integral domains};

\draw(202.59999999999997, -1) node[anchor=north west] { \large Number theory};
\draw (202.59999999999997, -1) rectangle (259.95,-72.1);
\draw(203.59999999999997, -2) node[anchor=north west] { \large Discontinuous groups and automorphic forms};
\draw (203.59999999999997, -2) rectangle (233.44999999999996,-16.2);
\draw(204.59999999999997, -3) node[anchor=north west,align=left] {Dirichlet series in several complex variables associated\\ to automorphic forms; Weyl group multiple Dirichlet series};

\draw(219.79999999999995, -3) node[anchor=north west,align=left] {Automorphic forms on \(\mbox{GL}(2)\); Hilbert \\ and Hilbert-Siegel modular groups and their modular\\ and automorphic forms; Hilbert modular surfaces};

\draw(204.59999999999997, -4.7) node[anchor=north west,align=left] {Special values of automorphic \(L\)-series, periods\\ of automorphic forms, cohomology, modular symbols};

\draw(218.04999999999995, -4.7) node[anchor=north west,align=left] {Representation-theoretic methods; automorphic\\ representations over local and global fields};

\draw(204.59999999999997, -5.9) node[anchor=north west,align=left] {Holomorphic modular forms of integral weight};

\draw(216.29999999999995, -5.9) node[anchor=north west,align=left] {Siegel modular groups; Siegel and \\ Hilbert-Siegel modular and automorphic forms};

\draw(204.59999999999997, -7.1000000000000005) node[anchor=north west,align=left] {Fourier coefficients of automorphic forms};

\draw(215.54999999999995, -7.1000000000000005) node[anchor=north west,align=left] {Langlands \(L\)-functions; one variable\\ Dirichlet series and functional equations};

\draw(204.59999999999997, -8.3) node[anchor=north west,align=left] {Hecke-Petersson operators, differential\\ operators (several variables)};

\draw(215.04999999999995, -8.3) node[anchor=north west,align=left] {Hecke-Petersson operators, \\ differential operators (one variable)};

\draw(204.59999999999997, -9.5) node[anchor=north west,align=left] {Other groups and their modular and \\ automorphic forms (several variables)};

\draw(214.54999999999995, -9.5) node[anchor=north west,align=left] {Dedekind eta function, Dedekind sums};

\draw(224.24999999999997, -9.5) node[anchor=north west,align=left] {Structure of modular groups and \\ generalizations; arithmetic groups};

\draw(204.59999999999997, -10.700000000000001) node[anchor=north west,align=left] {Theta series; Weil representation;\\ theta correspondences};

\draw(213.79999999999995, -10.700000000000001) node[anchor=north west,align=left] {Modular and automorphic functions};

\draw(222.74999999999997, -10.700000000000001) node[anchor=north west,align=left] {Spectral theory; trace \\ formulas (e.g., that of Selberg)};

\draw(204.59999999999997, -11.900000000000002) node[anchor=north west,align=left] {Automorphic forms and their \\ relations with perfectoid spaces};

\draw(213.29999999999995, -11.900000000000002) node[anchor=north west,align=left] {Automorphic forms, one variable};

\draw(221.74999999999997, -11.900000000000002) node[anchor=north west,align=left] {Cohomology of arithmetic groups};

\draw(204.59999999999997, -13.100000000000001) node[anchor=north west,align=left] {\(p\)-adic theory, local fields};

\draw(213.04999999999995, -13.100000000000001) node[anchor=north west,align=left] {Modular correspondences, etc.};

\draw(220.99999999999997, -13.100000000000001) node[anchor=north west,align=left] {Forms of half-integer weight;\\ nonholomorphic modular forms};

\draw(204.59999999999997, -14.3) node[anchor=north west,align=left] {Relationship to Lie algebras\\ and finite simple groups};

\draw(212.29999999999995, -14.3) node[anchor=north west,align=left] {Congruences for modular and\\ \(p\)-adic modular forms};

\draw(219.74999999999997, -14.3) node[anchor=north west,align=left] {Relations with algebraic\\ geometry and topology};

\draw(226.44999999999996, -14.3) node[anchor=north west,align=left] {Modular forms associated\\ to Drinfel’d modules};

\draw(204.59999999999997, -15.5) node[anchor=north west,align=left] {Galois representations};

\draw(210.79999999999995, -15.5) node[anchor=north west,align=left] {Jacobi forms};

\draw(233.54999999999995, -2) node[anchor=north west] { \large Algebraic number theory: global fields};
\draw (233.54999999999995, -2) rectangle (259.84999999999997,-14.0);
\draw(234.54999999999995, -3) node[anchor=north west,align=left] {Integral representations related to algebraic \\ numbers; Galois module structure of rings of integers};

\draw(248.49999999999994, -3) node[anchor=north west,align=left] {Class numbers, class groups, discriminants};

\draw(234.54999999999995, -4.2) node[anchor=north west,align=left] {PV-numbers and generalizations; other \\ special algebraic numbers; Mahler measure};

\draw(245.49999999999994, -4.2) node[anchor=north west,align=left] {Class groups and Picard groups of orders};

\draw(234.54999999999995, -5.4) node[anchor=north west,align=left] {Other abelian and metabelian extensions};

\draw(244.99999999999994, -5.4) node[anchor=north west,align=left] {Quaternion and other division \\ algebras: arithmetic, zeta functions};

\draw(234.54999999999995, -6.6000000000000005) node[anchor=north west,align=left] {Polynomials (irreducibility, etc.)};

\draw(243.74999999999994, -6.6000000000000005) node[anchor=north west,align=left] {Zeta functions and \\ \(L\)-functions of function fields};

\draw(234.54999999999995, -7.800000000000001) node[anchor=north west,align=left] {Cyclotomic function fields (class\\ groups, Bernoulli objects, etc.)};

\draw(243.49999999999994, -7.800000000000001) node[anchor=north west,align=left] {Zeta functions and \\ \(L\)-functions of number fields};

\draw(234.54999999999995, -9.0) node[anchor=north west,align=left] {Other algebras and orders, and\\ their zeta and \(L\)-functions};

\draw(242.74999999999994, -9.0) node[anchor=north west,align=left] {Langlands-Weil conjectures,\\ nonabelian class field theory};

\draw(250.69999999999996, -9.0) node[anchor=north west,align=left] {\(K\)-theory of global fields};

\draw(234.54999999999995, -10.200000000000001) node[anchor=north west,align=left] {Cubic and quartic extensions};

\draw(242.24999999999994, -10.200000000000001) node[anchor=north west,align=left] {Distribution of prime ideals};

\draw(249.94999999999996, -10.200000000000001) node[anchor=north west,align=left] {Arithmetic theory of \\ algebraic function fields};

\draw(234.54999999999995, -11.400000000000002) node[anchor=north west,align=left] {Algebraic numbers; rings\\ of algebraic integers};

\draw(241.24999999999994, -11.400000000000002) node[anchor=north west,align=left] {Units and factorization};

\draw(247.69999999999996, -11.400000000000002) node[anchor=north west,align=left] {Adèle rings and groups};

\draw(253.89999999999995, -11.400000000000002) node[anchor=north west,align=left] {Cyclotomic extensions};

\draw(234.54999999999995, -12.600000000000001) node[anchor=north west,align=left] {Other analytic theory};

\draw(240.49999999999994, -12.600000000000001) node[anchor=north west,align=left] {Quadratic extensions};

\draw(246.19999999999996, -12.600000000000001) node[anchor=north west,align=left] {Other number fields};

\draw(251.64999999999995, -12.600000000000001) node[anchor=north west,align=left] {Totally real fields};

\draw(234.54999999999995, -13.3) node[anchor=north west,align=left] {Class field theory};

\draw(239.74999999999994, -13.3) node[anchor=north west,align=left] {Galois cohomology};

\draw(244.69999999999996, -13.3) node[anchor=north west,align=left] {Density theorems};

\draw(249.39999999999995, -13.3) node[anchor=north west,align=left] {Iwasawa theory};

\draw(253.59999999999997, -13.3) node[anchor=north west,align=left] {Galois theory};

\draw(233.54999999999995, -14.1) node[anchor=north west] { \large Polynomials and matrices};
\draw (233.54999999999995, -14.1) rectangle (252.69999999999996,-15.8);
\draw(234.54999999999995, -15.1) node[anchor=north west,align=left] {Matrices, determinants in number theory};

\draw(244.99999999999994, -15.1) node[anchor=north west,align=left] {Polynomials in number theory};

\draw(203.59999999999997, -16.299999999999997) node[anchor=north west] { \large Zeta and \(L\)-functions: analytic theory};
\draw (203.59999999999997, -16.299999999999997) rectangle (229.44999999999996,-23.299999999999997);
\draw(204.59999999999997, -17.299999999999997) node[anchor=north west,align=left] {Selberg zeta functions and regularized determinants;\\ applications to spectral theory, Dirichlet \\ series, Eisenstein series, etc. (explicit formulas)};

\draw(218.29999999999995, -17.299999999999997) node[anchor=north west,align=left] {Nonreal zeros of \(\zeta (s)\) and \(L(s, \chi)\); Riemann and other hypotheses};

\draw(204.59999999999997, -18.999999999999996) node[anchor=north west,align=left] {Other Dirichlet series and zeta functions};

\draw(215.54999999999995, -18.999999999999996) node[anchor=north west,align=left] {Relations with noncommutative geometry};

\draw(204.59999999999997, -19.699999999999996) node[anchor=north west,align=left] {Multiple Dirichlet series and \\ zeta functions and multizeta values};

\draw(214.04999999999995, -19.699999999999996) node[anchor=north west,align=left] {\(\zeta (s)\) and \(L(s, \chi)\)};

\draw(204.59999999999997, -20.9) node[anchor=north west,align=left] {Hurwitz and Lerch zeta functions};

\draw(213.29999999999995, -20.9) node[anchor=north west,align=left] {Relations with random matrices};

\draw(221.49999999999997, -20.9) node[anchor=north west,align=left] {Real zeros of \(L(s, \chi)\);\\ results on \(L(1, \chi)\)};

\draw(204.59999999999997, -22.099999999999998) node[anchor=north west,align=left] {Zeta and \(L\)-functions\\ in characteristic \(p\)};

\draw(211.29999999999995, -22.099999999999998) node[anchor=north west,align=left] {Tauberian theorems};

\draw(229.54999999999995, -16.299999999999997) node[anchor=north west] { \large Multiplicative number theory};
\draw (229.54999999999995, -16.299999999999997) rectangle (255.19999999999996,-24.0);
\draw(230.54999999999995, -17.299999999999997) node[anchor=north west,align=left] {Other results on the distribution of values \\ or the characterization of arithmetic functions};

\draw(242.99999999999994, -17.299999999999997) node[anchor=north west,align=left] {Primes represented by polynomials; other \\ multiplicative structures of polynomial values};

\draw(230.54999999999995, -18.499999999999996) node[anchor=north west,align=left] {Distribution functions associated with \\ additive and positive multiplicative functions};

\draw(242.74999999999994, -18.499999999999996) node[anchor=north west,align=left] {Asymptotic results on arithmetic functions};

\draw(230.54999999999995, -19.699999999999996) node[anchor=north west,align=left] {Asymptotic results on counting functions\\ for algebraic and topological structures};

\draw(241.24999999999994, -19.699999999999996) node[anchor=north west,align=left] {Rate of growth of arithmetic functions};

\draw(230.54999999999995, -20.9) node[anchor=north west,align=left] {Applications of automorphic functions\\ and forms to multiplicative problems};

\draw(240.49999999999994, -20.9) node[anchor=north west,align=left] {Distribution of integers with \\ specified multiplicative constraints};

\draw(230.54999999999995, -22.099999999999998) node[anchor=north west,align=left] {Generalized primes and integers};

\draw(238.99999999999994, -22.099999999999998) node[anchor=north west,align=left] {Applications of sieve methods};

\draw(246.94999999999996, -22.099999999999998) node[anchor=north west,align=left] {Primes in congruence classes};

\draw(230.54999999999995, -22.799999999999997) node[anchor=north west,align=left] {Distribution of integers\\ in special residue classes};

\draw(237.74999999999994, -22.799999999999997) node[anchor=north west,align=left] {Distribution of primes};

\draw(243.94999999999996, -22.799999999999997) node[anchor=north west,align=left] {Turán theory};

\draw(247.64999999999995, -22.799999999999997) node[anchor=north west,align=left] {Sieves};

\draw(203.59999999999997, -24.1) node[anchor=north west] { \large Probabilistic theory: distribution modulo \(1\); metric theory of algorithms};
\draw (203.59999999999997, -24.1) rectangle (228.99999999999997,-31.8);
\draw(204.59999999999997, -25.1) node[anchor=north west,align=left] {Normal numbers, radix expansions, Pisot \\ numbers, Salem numbers, good lattice points, etc.};

\draw(217.54999999999995, -25.1) node[anchor=north west,align=left] {General theory of distribution modulo \(1\)};

\draw(204.59999999999997, -26.3) node[anchor=north west,align=left] {Irregularities of distribution, discrepancy};

\draw(216.04999999999995, -26.3) node[anchor=north west,align=left] {Metric theory of other algorithms and \\ expansions; measure and Hausdorff dimension};

\draw(204.59999999999997, -27.5) node[anchor=north west,align=left] {Pseudo-random numbers; Monte Carlo methods};

\draw(215.79999999999995, -27.5) node[anchor=north west,align=left] {Harmonic analysis and almost periodicity\\ in probabilistic number theory};

\draw(204.59999999999997, -28.700000000000003) node[anchor=north west,align=left] {Metric theory of continued fractions};

\draw(214.29999999999995, -28.700000000000003) node[anchor=north west,align=left] {Diophantine approximation in\\ probabilistic number theory};

\draw(204.59999999999997, -29.900000000000002) node[anchor=north west,align=left] {Arithmetic functions in \\ probabilistic number theory};

\draw(212.04999999999995, -29.900000000000002) node[anchor=north west,align=left] {Well-distributed sequences\\ and other variations};

\draw(219.24999999999997, -29.900000000000002) node[anchor=north west,align=left] {Continuous, \(p\)-adic\\ and abstract analogues};

\draw(204.59999999999997, -31.1) node[anchor=north west,align=left] {Special sequences};

\draw(229.09999999999997, -24.1) node[anchor=north west] { \large Forms and linear algebraic groups};
\draw (229.09999999999997, -24.1) rectangle (254.24999999999997,-32.5);
\draw(230.09999999999997, -25.1) node[anchor=north west,align=left] {Analytic theory (Epstein zeta functions; \\ relations with automorphic forms and functions)};

\draw(242.54999999999995, -25.1) node[anchor=north west,align=left] {Quadratic forms over global rings and fields};

\draw(230.09999999999997, -26.3) node[anchor=north west,align=left] {Galois cohomology of linear algebraic groups};

\draw(241.79999999999995, -26.3) node[anchor=north west,align=left] {Quadratic forms over local rings and fields};

\draw(230.09999999999997, -27.0) node[anchor=north west,align=left] {General ternary and quaternary quadratic\\ forms; forms of more than two variables};

\draw(240.79999999999995, -27.0) node[anchor=north west,align=left] {Quadratic forms over general fields};

\draw(230.09999999999997, -28.200000000000003) node[anchor=north west,align=left] {Sums of squares and representations\\ by other particular quadratic forms};

\draw(239.54999999999995, -28.200000000000003) node[anchor=north west,align=left] {Quadratic spaces; Clifford algebras};

\draw(230.09999999999997, -29.400000000000002) node[anchor=north west,align=left] {Forms of degree higher than two};

\draw(238.54999999999995, -29.400000000000002) node[anchor=north west,align=left] {General binary quadratic forms};

\draw(230.09999999999997, -30.1) node[anchor=north west,align=left] {Algebraic theory of quadratic\\ forms; Witt groups and rings};

\draw(238.04999999999995, -30.1) node[anchor=north west,align=left] {Bilinear and Hermitian forms};

\draw(245.74999999999997, -30.1) node[anchor=north west,align=left] {Class numbers of quadratic\\ and Hermitian forms};

\draw(230.09999999999997, -31.300000000000004) node[anchor=north west,align=left] {\(K\)-theory of quadratic\\ and Hermitian forms};

\draw(237.04999999999995, -31.300000000000004) node[anchor=north west,align=left] {Forms over real fields};

\draw(243.24999999999997, -31.300000000000004) node[anchor=north west,align=left] {\(p\)-adic theory};

\draw(248.19999999999996, -31.300000000000004) node[anchor=north west,align=left] {Classical groups};

\draw(203.59999999999997, -32.6) node[anchor=north west] { \large Algebraic number theory: local and \(p\)-adic fields};
\draw (203.59999999999997, -32.6) rectangle (228.49999999999997,-39.300000000000004);
\draw(204.59999999999997, -33.6) node[anchor=north west,align=left] {Other analytic theory (analogues of beta and\\ gamma functions, \(p\)-adic integration, etc.)};

\draw(216.79999999999995, -33.6) node[anchor=north west,align=left] {Class field theory; \(p\)-adic formal groups};

\draw(204.59999999999997, -34.800000000000004) node[anchor=north west,align=left] {Zeta functions and \(L\)-functions};

\draw(213.79999999999995, -34.800000000000004) node[anchor=north west,align=left] {Ramification and extension theory};

\draw(204.59999999999997, -35.5) node[anchor=north west,align=left] {Non-Archimedean dynamical systems};

\draw(213.54999999999995, -35.5) node[anchor=north west,align=left] {Langlands-Weil conjectures,\\ nonabelian class field theory};

\draw(204.59999999999997, -36.7) node[anchor=north west,align=left] {\(K\)-theory of local fields};

\draw(212.29999999999995, -36.7) node[anchor=north west,align=left] {Prehomogeneous vector spaces};

\draw(219.99999999999997, -36.7) node[anchor=north west,align=left] {Integral representations};

\draw(204.59999999999997, -37.400000000000006) node[anchor=north west,align=left] {Algebras and orders, \\ and their zeta functions};

\draw(211.29999999999995, -37.400000000000006) node[anchor=north west,align=left] {Other nonanalytic theory};

\draw(217.99999999999997, -37.400000000000006) node[anchor=north west,align=left] {Galois cohomology};

\draw(222.94999999999996, -37.400000000000006) node[anchor=north west,align=left] {Galois theory};

\draw(204.59999999999997, -38.6) node[anchor=north west,align=left] {Polynomials};

\draw(228.59999999999997, -32.6) node[anchor=north west] { \large Diophantine equations};
\draw (228.59999999999997, -32.6) rectangle (252.74999999999997,-38.5);
\draw(229.59999999999997, -33.6) node[anchor=north west,align=left] {Quadratic and bilinear Diophantine equations};

\draw(241.29999999999995, -33.6) node[anchor=north west,align=left] {Counting solutions of Diophantine equations};

\draw(229.59999999999997, -34.300000000000004) node[anchor=north west,align=left] {Higher degree equations; Fermat’s equation};

\draw(240.79999999999995, -34.300000000000004) node[anchor=north west,align=left] {Cubic and quartic Diophantine equations};

\draw(229.59999999999997, -35.0) node[anchor=north west,align=left] {Diophantine equations in many variables};

\draw(240.04999999999995, -35.0) node[anchor=north west,align=left] {Multiplicative and norm form equations};

\draw(229.59999999999997, -35.7) node[anchor=north west,align=left] {Rational numbers as sums of fractions};

\draw(239.54999999999995, -35.7) node[anchor=north west,align=left] {\(p\)-adic and power series fields};

\draw(229.59999999999997, -36.4) node[anchor=north west,align=left] {Exponential Diophantine equations};

\draw(238.54999999999995, -36.4) node[anchor=north west,align=left] {Congruences in many variables};

\draw(229.59999999999997, -37.1) node[anchor=north west,align=left] {Linear Diophantine equations};

\draw(237.29999999999995, -37.1) node[anchor=north west,align=left] {Diophantine inequalities};

\draw(243.99999999999997, -37.1) node[anchor=north west,align=left] {Representation problems};

\draw(229.59999999999997, -37.800000000000004) node[anchor=north west,align=left] {The Frobenius problem};

\draw(235.54999999999995, -37.800000000000004) node[anchor=north west,align=left] {Thue-Mahler equations};

\draw(228.59999999999997, -38.6) node[anchor=north west,align=left] {History of number theory};

\draw(203.59999999999997, -39.4) node[anchor=north west] { \large Diophantine approximation, transcendental number theory};
\draw (203.59999999999997, -39.4) rectangle (227.69999999999996,-50.4);
\draw(204.59999999999997, -40.4) node[anchor=north west,align=left] {Approximation by numbers from a fixed field};

\draw(216.04999999999995, -40.4) node[anchor=north west,align=left] {Approximation in non-Archimedean valuations};

\draw(204.59999999999997, -41.1) node[anchor=north west,align=left] {Linear forms in logarithms; Baker’s method};

\draw(215.79999999999995, -41.1) node[anchor=north west,align=left] {Algebraic independence; Gel’fond’s method};

\draw(204.59999999999997, -41.8) node[anchor=north west,align=left] {Schmidt Subspace Theorem and applications};

\draw(215.54999999999995, -41.8) node[anchor=north west,align=left] {Number-theoretic analogues of methods in\\ Nevanlinna theory (work of Vojta et al.)};

\draw(204.59999999999997, -43.0) node[anchor=north west,align=left] {Homogeneous approximation to one number};

\draw(215.04999999999995, -43.0) node[anchor=north west,align=left] {Continued fractions and generalizations};

\draw(204.59999999999997, -43.699999999999996) node[anchor=north west,align=left] {Results involving abelian varieties};

\draw(214.04999999999995, -43.699999999999996) node[anchor=north west,align=left] {Approximation to algebraic numbers};

\draw(204.59999999999997, -44.4) node[anchor=north west,align=left] {Small fractional parts of \\ polynomials and generalizations};

\draw(213.04999999999995, -44.4) node[anchor=north west,align=left] {Transcendence (general theory)};

\draw(204.59999999999997, -45.6) node[anchor=north west,align=left] {Transcendence theory of \\ elliptic and abelian functions};

\draw(212.79999999999995, -45.6) node[anchor=north west,align=left] {Markov and Lagrange \\ spectra and generalizations};

\draw(220.24999999999997, -45.6) node[anchor=north west,align=left] {Simultaneous homogeneous \\ approximation, linear forms};

\draw(204.59999999999997, -46.8) node[anchor=north west,align=left] {Transcendence theory of \\ Drinfel’d and \(t\)-modules};

\draw(212.04999999999995, -46.8) node[anchor=north west,align=left] {Inhomogeneous linear forms};

\draw(219.24999999999997, -46.8) node[anchor=north west,align=left] {Irrationality; linear \\ independence over a field};

\draw(204.59999999999997, -48.0) node[anchor=north west,align=left] {Measures of irrationality\\ and of transcendence};

\draw(211.54999999999995, -48.0) node[anchor=north west,align=left] {Diophantine inequalities};

\draw(218.24999999999997, -48.0) node[anchor=north west,align=left] {Distribution modulo one};

\draw(204.59999999999997, -49.2) node[anchor=north west,align=left] {Transcendence theory of\\ other special functions};

\draw(211.04999999999995, -49.2) node[anchor=north west,align=left] {Metric theory};

\draw(227.79999999999995, -39.4) node[anchor=north west] { \large Exponential sums and character sums};
\draw (227.79999999999995, -39.4) rectangle (251.84999999999997,-43.0);
\draw(228.79999999999995, -40.4) node[anchor=north west,align=left] {Trigonometric and exponential sums, general};

\draw(240.24999999999994, -40.4) node[anchor=north west,align=left] {Gauss and Kloosterman sums; generalizations};

\draw(228.79999999999995, -41.1) node[anchor=north west,align=left] {Estimates on exponential sums};

\draw(236.74999999999994, -41.1) node[anchor=north west,align=left] {Jacobsthal and Brewer sums;\\ other complete character sums};

\draw(228.79999999999995, -42.3) node[anchor=north west,align=left] {Sums over arbitrary intervals};

\draw(236.74999999999994, -42.3) node[anchor=north west,align=left] {Estimates on character sums};

\draw(244.19999999999996, -42.3) node[anchor=north west,align=left] {Sums over primes};

\draw(248.89999999999995, -42.3) node[anchor=north west,align=left] {Weyl sums};

\draw(227.79999999999995, -43.1) node[anchor=north west] { \large Computational number theory};
\draw (227.79999999999995, -43.1) rectangle (250.69999999999996,-47.4);
\draw(228.79999999999995, -44.1) node[anchor=north west,align=left] {Computer solution of Diophantine equations};

\draw(239.99999999999994, -44.1) node[anchor=north west,align=left] {Evaluation of number-theoretic constants};

\draw(228.79999999999995, -44.800000000000004) node[anchor=north west,align=left] {Number-theoretic algorithms; complexity};

\draw(239.24999999999994, -44.800000000000004) node[anchor=north west,align=left] {Values of arithmetic functions; tables};

\draw(228.79999999999995, -45.5) node[anchor=north west,align=left] {Algebraic number theory computations};

\draw(238.49999999999994, -45.5) node[anchor=north west,align=left] {Calculation of integer sequences};

\draw(228.79999999999995, -46.2) node[anchor=north west,align=left] {Continued fraction calculations\\ (number-theoretic aspects)};

\draw(237.24999999999994, -46.2) node[anchor=north west,align=left] {Analytic computations};

\draw(243.19999999999996, -46.2) node[anchor=north west,align=left] {Factorization};

\draw(247.14999999999995, -46.2) node[anchor=north west,align=left] {Primality};

\draw(227.79999999999995, -47.5) node[anchor=north west] { \large Connections of number theory and logic};
\draw (227.79999999999995, -47.5) rectangle (249.94999999999996,-50.4);
\draw(228.79999999999995, -48.5) node[anchor=north west,align=left] {Ultraproducts (number-theoretic aspects)};

\draw(239.49999999999994, -48.5) node[anchor=north west,align=left] {Decidability (number-theoretic aspects)};

\draw(228.79999999999995, -49.2) node[anchor=north west,align=left] {Model theory (number-theoretic aspects)};

\draw(239.24999999999994, -49.2) node[anchor=north west,align=left] {Nonstandard arithmetic \\ (number-theoretic aspects)};

\draw(203.59999999999997, -50.5) node[anchor=north west] { \large Finite fields and commutative rings (number-theoretic aspects)};
\draw (203.59999999999997, -50.5) rectangle (226.24999999999997,-55.3);
\draw(204.59999999999997, -51.5) node[anchor=north west,align=left] {Structure theory for finite fields and \\ commutative rings (number-theoretic aspects)};

\draw(216.29999999999995, -51.5) node[anchor=north west,align=left] {Algebraic coding theory; cryptography\\ (number-theoretic aspects)};

\draw(204.59999999999997, -52.7) node[anchor=north west,align=left] {Other character sums and Gauss sums};

\draw(214.04999999999995, -52.7) node[anchor=north west,align=left] {Arithmetic theory of polynomial\\ rings over finite fields};

\draw(204.59999999999997, -53.9) node[anchor=north west,align=left] {Polynomials over finite fields};

\draw(212.79999999999995, -53.9) node[anchor=north west,align=left] {Finite upper half-planes};

\draw(219.49999999999997, -53.9) node[anchor=north west,align=left] {Exponential sums};

\draw(204.59999999999997, -54.6) node[anchor=north west,align=left] {Cyclotomy};

\draw(226.34999999999997, -50.5) node[anchor=north west] { \large Arithmetic algebraic geometry (Diophantine geometry)};
\draw (226.34999999999997, -50.5) rectangle (248.24999999999997,-59.6);
\draw(227.34999999999997, -51.5) node[anchor=north west,align=left] {\(L\)-functions of varieties over global\\ fields; Birch-Swinnerton-Dyer conjecture};

\draw(238.04999999999995, -51.5) node[anchor=north west,align=left] {Abelian varieties of dimension \(> 1\)};

\draw(227.34999999999997, -52.7) node[anchor=north west,align=left] {Varieties over finite and local fields};

\draw(237.54999999999995, -52.7) node[anchor=north west,align=left] {Curves over finite and local fields};

\draw(227.34999999999997, -53.4) node[anchor=north west,align=left] {Elliptic curves over global fields};

\draw(236.54999999999995, -53.4) node[anchor=north west,align=left] {Curves of arbitrary genus or \\ genus \(\ne 1\) over global fields};

\draw(227.34999999999997, -54.6) node[anchor=north west,align=left] {Elliptic curves over local fields};

\draw(236.29999999999995, -54.6) node[anchor=north west,align=left] {Drinfel’d modules; \\ higher-dimensional motives, etc.};

\draw(227.34999999999997, -55.8) node[anchor=north west,align=left] {Arithmetic aspects of \\ modular and Shimura varieties};

\draw(235.29999999999995, -55.8) node[anchor=north west,align=left] {Arithmetic aspects of dessins\\ d’enfants, Belyĭ theory};

\draw(227.34999999999997, -57.0) node[anchor=north west,align=left] {Varieties over global fields};

\draw(235.04999999999995, -57.0) node[anchor=north west,align=left] {Geometric class field theory};

\draw(227.34999999999997, -57.7) node[anchor=north west,align=left] {Complex multiplication and\\ moduli of abelian varieties};

\draw(234.79999999999995, -57.7) node[anchor=north west,align=left] {Polylogarithms and \\ relations with \(K\)-theory};

\draw(227.34999999999997, -58.900000000000006) node[anchor=north west,align=left] {Elliptic and modular units};

\draw(234.54999999999995, -58.900000000000006) node[anchor=north west,align=left] {Arithmetic mirror symmetry};

\draw(241.74999999999997, -58.900000000000006) node[anchor=north west,align=left] {Heights};

\draw(203.59999999999997, -59.7) node[anchor=north west] { \large Additive number theory; partitions};
\draw (203.59999999999997, -59.7) rectangle (225.49999999999997,-65.7);
\draw(204.59999999999997, -60.7) node[anchor=north west,align=left] {Applications of the Hardy-Littlewood method};

\draw(216.04999999999995, -60.7) node[anchor=north west,align=left] {Lattice points in specified regions};

\draw(204.59999999999997, -61.400000000000006) node[anchor=north west,align=left] {Goldbach-type theorems; other \\ additive questions involving primes};

\draw(214.04999999999995, -61.400000000000006) node[anchor=north west,align=left] {Inverse problems of additive \\ number theory, including sumsets};

\draw(204.59999999999997, -62.6) node[anchor=north west,align=left] {Partition identities; identities\\ of Rogers-Ramanujan type};

\draw(213.29999999999995, -62.6) node[anchor=north west,align=left] {Elementary theory of partitions};

\draw(204.59999999999997, -63.800000000000004) node[anchor=north west,align=left] {Waring’s problem and variants};

\draw(212.54999999999995, -63.800000000000004) node[anchor=north west,align=left] {Analytic theory of partitions};

\draw(204.59999999999997, -64.5) node[anchor=north west,align=left] {Partitions; congruences and\\ congruential restrictions};

\draw(225.59999999999997, -59.7) node[anchor=north west] { \large Sequences and sets};
\draw (225.59999999999997, -59.7) rectangle (246.99999999999997,-66.60000000000001);
\draw(226.59999999999997, -60.7) node[anchor=north west,align=left] {Bernoulli and Euler numbers and polynomials};

\draw(238.04999999999995, -60.7) node[anchor=north west,align=left] {Additive bases, including sumsets};

\draw(226.59999999999997, -61.400000000000006) node[anchor=north west,align=left] {Other combinatorial number theory};

\draw(235.54999999999995, -61.400000000000006) node[anchor=north west,align=left] {Special sequences and polynomials};

\draw(226.59999999999997, -62.1) node[anchor=north west,align=left] {Fibonacci and Lucas numbers and\\ polynomials and generalizations};

\draw(235.04999999999995, -62.1) node[anchor=north west,align=left] {Farey sequences; the \\ sequences \(1^k, 2^k, \dots\)};

\draw(226.59999999999997, -63.300000000000004) node[anchor=north west,align=left] {Binomial coefficients; \\ factorials; \(q\)-identities};

\draw(234.29999999999995, -63.300000000000004) node[anchor=north west,align=left] {Arithmetic combinatorics;\\ higher degree uniformity};

\draw(226.59999999999997, -64.5) node[anchor=north west,align=left] {Bell and Stirling numbers};

\draw(233.54999999999995, -64.5) node[anchor=north west,align=left] {Representation functions};

\draw(240.24999999999997, -64.5) node[anchor=north west,align=left] {Density, gaps, topology};

\draw(226.59999999999997, -65.2) node[anchor=north west,align=left] {Arithmetic progressions};

\draw(233.04999999999995, -65.2) node[anchor=north west,align=left] {Sequences (mod \(m\))};

\draw(238.99999999999997, -65.2) node[anchor=north west,align=left] {Automata sequences};

\draw(226.59999999999997, -65.9) node[anchor=north west,align=left] {Recurrences};

\draw(203.59999999999997, -66.7) node[anchor=north west] { \large Elementary number theory};
\draw (203.59999999999997, -66.7) rectangle (224.24999999999997,-72.0);
\draw(204.59999999999997, -67.7) node[anchor=north west,align=left] {Radix representation; digital problems};

\draw(214.79999999999995, -67.7) node[anchor=north west,align=left] {Multiplicative structure; Euclidean\\ algorithm; greatest common divisors};

\draw(204.59999999999997, -68.9) node[anchor=north west,align=left] {Arithmetic functions; related\\ numbers; inversion formulas};

\draw(212.54999999999995, -68.9) node[anchor=north west,align=left] {Other number representations};

\draw(204.59999999999997, -70.10000000000001) node[anchor=north west,align=left] {Power residues, reciprocity};

\draw(212.04999999999995, -70.10000000000001) node[anchor=north west,align=left] {Factorization; primality};

\draw(204.59999999999997, -70.8) node[anchor=north west,align=left] {Congruences; primitive\\ roots; residue systems};

\draw(210.79999999999995, -70.8) node[anchor=north west,align=left] {Continued fractions};

\draw(216.24999999999997, -70.8) node[anchor=north west,align=left] {Primes};

\draw(224.34999999999997, -66.7) node[anchor=north west] { \large Geometry of numbers};
\draw (224.34999999999997, -66.7) rectangle (242.49999999999997,-71.5);
\draw(225.34999999999997, -67.7) node[anchor=north west,align=left] {Mean value and transfer theorems};

\draw(234.04999999999995, -67.7) node[anchor=north west,align=left] {Automorphism groups of lattices};

\draw(225.34999999999997, -68.4) node[anchor=north west,align=left] {Lattice packing and covering\\ (number-theoretic aspects)};

\draw(233.04999999999995, -68.4) node[anchor=north west,align=left] {Quadratic forms (reduction\\ theory, extreme forms, etc.)};

\draw(225.34999999999997, -69.60000000000001) node[anchor=north west,align=left] {Relations with coding theory};

\draw(233.04999999999995, -69.60000000000001) node[anchor=north west,align=left] {Lattices and convex bodies\\ (number-theoretic aspects)};

\draw(225.34999999999997, -70.8) node[anchor=north west,align=left] {Products of linear forms};

\draw(232.04999999999995, -70.8) node[anchor=north west,align=left] {Nonconvex bodies};

\draw(236.74999999999997, -70.8) node[anchor=north west,align=left] {Minima of forms};

\draw(242.59999999999997, -66.7) node[anchor=north west] { \large Miscellaneous applications of number theory};
\draw (242.59999999999997, -66.7) rectangle (256.09999999999997,-68.4);
\draw(243.59999999999997, -67.7) node[anchor=north west,align=left] {Miscellaneous applications of number theory};

\draw(202.59999999999997, -72.19999999999999) node[anchor=north west] { \large Combinatorics};
\draw (202.59999999999997, -72.19999999999999) rectangle (259.09999999999997,-93.29999999999998);
\draw(203.59999999999997, -73.19999999999999) node[anchor=north west] { \large Graph theory};
\draw (203.59999999999997, -73.19999999999999) rectangle (234.49999999999997,-86.6);
\draw(204.59999999999997, -74.19999999999999) node[anchor=north west,align=left] {Isomorphism problems in graph theory (reconstruction \\ conjecture, etc.) and homomorphisms (subgraph embedding, etc.)};

\draw(220.79999999999995, -74.19999999999999) node[anchor=north west,align=left] {Edge subsets with special properties (factorization,\\ matching, partitioning, covering and packing, etc.)};

\draw(204.59999999999997, -75.39999999999999) node[anchor=north west,align=left] {Vertex subsets with special properties \\ (dominating sets, independent sets, cliques, etc.)};

\draw(217.79999999999995, -75.39999999999999) node[anchor=north west,align=left] {Fractional graph theory, fuzzy graph theory};

\draw(204.59999999999997, -76.6) node[anchor=north west,align=left] {Graph designs and isomorphic decomposition};

\draw(215.79999999999995, -76.6) node[anchor=north west,align=left] {Graph algorithms (graph-theoretic aspects)};

\draw(204.59999999999997, -77.29999999999998) node[anchor=north west,align=left] {Games on graphs (graph-theoretic aspects)};

\draw(215.54999999999995, -77.29999999999998) node[anchor=north west,align=left] {Directed graphs (digraphs), tournaments};

\draw(204.59999999999997, -77.99999999999999) node[anchor=north west,align=left] {Random graphs (graph-theoretic aspects)};

\draw(215.04999999999995, -77.99999999999999) node[anchor=north west,align=left] {Graph representations (geometric and\\ intersection representations, etc.)};

\draw(224.74999999999997, -77.99999999999999) node[anchor=north west,align=left] {Planar graphs; geometric and \\ topological aspects of graph theory};

\draw(204.59999999999997, -79.19999999999999) node[anchor=north west,align=left] {Coloring of graphs and hypergraphs};

\draw(213.79999999999995, -79.19999999999999) node[anchor=north west,align=left] {Small world graphs, complex \\ networks (graph-theoretic aspects)};

\draw(222.99999999999997, -79.19999999999999) node[anchor=north west,align=left] {Graphical indices (Wiener index,\\ Zagreb index, Randić index, etc.)};

\draw(204.59999999999997, -80.39999999999999) node[anchor=north west,align=left] {Extremal problems in graph theory};

\draw(213.54999999999995, -80.39999999999999) node[anchor=north west,align=left] {Eulerian and Hamiltonian graphs};

\draw(221.99999999999997, -80.39999999999999) node[anchor=north west,align=left] {Graphs and abstract algebra\\ (groups, rings, fields, etc.)};

\draw(204.59999999999997, -81.6) node[anchor=north west,align=left] {Graphs and linear algebra \\ (matrices, eigenvalues, etc.)};

\draw(212.54999999999995, -81.6) node[anchor=north west,align=left] {Applications of graph theory};

\draw(220.24999999999997, -81.6) node[anchor=north west,align=left] {Enumeration in graph theory};

\draw(204.59999999999997, -82.79999999999998) node[anchor=north west,align=left] {Structural characterization\\ of families of graphs};

\draw(212.04999999999995, -82.79999999999998) node[anchor=north west,align=left] {Signed and weighted graphs};

\draw(219.24999999999997, -82.79999999999998) node[anchor=north west,align=left] {Density (toughness, etc.)};

\draw(226.19999999999996, -82.79999999999998) node[anchor=north west,align=left] {Generalized Ramsey theory};

\draw(204.59999999999997, -83.99999999999999) node[anchor=north west,align=left] {Graph labelling (graceful\\ graphs, bandwidth, etc.)};

\draw(211.54999999999995, -83.99999999999999) node[anchor=north west,align=left] {Graph operations (line\\ graphs, products, etc.)};

\draw(217.99999999999997, -83.99999999999999) node[anchor=north west,align=left] {Random walks on graphs};

\draw(224.19999999999996, -83.99999999999999) node[anchor=north west,align=left] {Chemical graph theory};

\draw(204.59999999999997, -85.19999999999999) node[anchor=north west,align=left] {Distance in graphs};

\draw(209.79999999999995, -85.19999999999999) node[anchor=north west,align=left] {Graph polynomials};

\draw(214.74999999999997, -85.19999999999999) node[anchor=north west,align=left] {Paths and cycles};

\draw(219.44999999999996, -85.19999999999999) node[anchor=north west,align=left] {Flows in graphs};

\draw(223.89999999999998, -85.19999999999999) node[anchor=north west,align=left] {Expander graphs};

\draw(228.34999999999997, -85.19999999999999) node[anchor=north west,align=left] {Infinite graphs};

\draw(204.59999999999997, -85.89999999999999) node[anchor=north west,align=left] {Vertex degrees};

\draw(208.79999999999995, -85.89999999999999) node[anchor=north west,align=left] {Perfect graphs};

\draw(212.99999999999997, -85.89999999999999) node[anchor=north west,align=left] {Connectivity};

\draw(216.69999999999996, -85.89999999999999) node[anchor=north west,align=left] {Graph minors};

\draw(220.39999999999998, -85.89999999999999) node[anchor=north west,align=left] {Hypergraphs};

\draw(223.84999999999997, -85.89999999999999) node[anchor=north west,align=left] {Trees};

\draw(234.59999999999997, -73.19999999999999) node[anchor=north west] { \large Algebraic combinatorics};
\draw (234.59999999999997, -73.19999999999999) rectangle (258.99999999999994,-77.49999999999999);
\draw(235.59999999999997, -74.19999999999999) node[anchor=north west,align=left] {Combinatorial aspects of groups and algebras};

\draw(247.29999999999995, -74.19999999999999) node[anchor=north west,align=left] {Association schemes, strongly regular graphs};

\draw(235.59999999999997, -74.89999999999999) node[anchor=north west,align=left] {Combinatorial aspects of commutative algebra};

\draw(247.29999999999995, -74.89999999999999) node[anchor=north west,align=left] {Combinatorial aspects of algebraic geometry};

\draw(235.59999999999997, -75.6) node[anchor=north west,align=left] {Group actions on combinatorial structures};

\draw(246.54999999999995, -75.6) node[anchor=north west,align=left] {Symmetric functions and generalizations};

\draw(235.59999999999997, -76.29999999999998) node[anchor=north west,align=left] {Combinatorial aspects \\ of representation theory};

\draw(242.29999999999995, -76.29999999999998) node[anchor=north west,align=left] {Combinatorial aspects\\ of simplicial complexes};

\draw(234.59999999999997, -77.6) node[anchor=north west] { \large Extremal combinatorics};
\draw (234.59999999999997, -77.6) rectangle (258.75,-80.5);
\draw(235.59999999999997, -78.6) node[anchor=north west,align=left] {Probabilistic methods in extremal combinatorics, including\\ polynomial methods (combinatorial Nullstellensatz, etc.)};

\draw(250.79999999999995, -78.6) node[anchor=north west,align=left] {Transversal (matching) theory};

\draw(235.59999999999997, -79.8) node[anchor=north west,align=left] {Extremal set theory};

\draw(241.04999999999995, -79.8) node[anchor=north west,align=left] {Ramsey theory};

\draw(234.59999999999997, -80.6) node[anchor=north west,align=left] {Computational methods for problems\\ pertaining to combinatorics};

\draw(234.59999999999997, -81.79999999999998) node[anchor=north west,align=left] {History of combinatorics};

\draw(203.59999999999997, -86.69999999999999) node[anchor=north west] { \large Designs and configurations};
\draw (203.59999999999997, -86.69999999999999) rectangle (226.74999999999997,-93.19999999999999);
\draw(204.59999999999997, -87.69999999999999) node[anchor=north west,align=left] {Combinatorial aspects of finite geometries};

\draw(215.79999999999995, -87.69999999999999) node[anchor=north west,align=left] {Combinatorial aspects of difference sets\\ (number-theoretic, group-theoretic, etc.)};

\draw(204.59999999999997, -88.89999999999999) node[anchor=north west,align=left] {Combinatorial aspects of block designs};

\draw(214.79999999999995, -88.89999999999999) node[anchor=north west,align=left] {Combinatorial aspects of matrices\\ (incidence, Hadamard, etc.)};

\draw(204.59999999999997, -90.1) node[anchor=north west,align=left] {Combinatorial aspects of \\ tessellation and tiling problems};

\draw(213.29999999999995, -90.1) node[anchor=north west,align=left] {Combinatorial aspects of \\ matroids and geometric lattices};

\draw(204.59999999999997, -91.29999999999998) node[anchor=north west,align=left] {Other designs, configurations};

\draw(212.54999999999995, -91.29999999999998) node[anchor=north west,align=left] {Orthogonal arrays, Latin\\ squares, Room squares};

\draw(219.24999999999997, -91.29999999999998) node[anchor=north west,align=left] {Combinatorial aspects\\ of packing and covering};

\draw(204.59999999999997, -92.49999999999999) node[anchor=north west,align=left] {Triple systems};

\draw(208.79999999999995, -92.49999999999999) node[anchor=north west,align=left] {Polyominoes};

\draw(226.84999999999997, -86.69999999999999) node[anchor=north west] { \large Enumerative combinatorics};
\draw (226.84999999999997, -86.69999999999999) rectangle (246.19999999999996,-93.19999999999999);
\draw(227.84999999999997, -87.69999999999999) node[anchor=north west,align=left] {Factorials, binomial coefficients,\\ combinatorial functions};

\draw(237.04999999999995, -87.69999999999999) node[anchor=north west,align=left] {\(q\)-calculus and related topics};

\draw(227.84999999999997, -88.89999999999999) node[anchor=north west,align=left] {Permutations, words, matrices};

\draw(235.79999999999995, -88.89999999999999) node[anchor=north west,align=left] {Exact enumeration problems,\\ generating functions};

\draw(227.84999999999997, -90.1) node[anchor=north west,align=left] {Combinatorial inequalities};

\draw(235.04999999999995, -90.1) node[anchor=north west,align=left] {Combinatorial aspects \\ of partitions of integers};

\draw(227.84999999999997, -91.29999999999998) node[anchor=north west,align=left] {Combinatorial identities,\\ bijective combinatorics};

\draw(234.79999999999995, -91.29999999999998) node[anchor=north west,align=left] {Asymptotic enumeration};

\draw(240.99999999999997, -91.29999999999998) node[anchor=north west,align=left] {Partitions of sets};

\draw(227.84999999999997, -92.49999999999999) node[anchor=north west,align=left] {Umbral calculus};

\draw(202.59999999999997, -93.39999999999999) node[anchor=north west] { \large Group theory and generalizations};
\draw (202.59999999999997, -93.39999999999999) rectangle (257.09999999999997,-140.6);
\draw(203.59999999999997, -94.39999999999999) node[anchor=north west] { \large Structure and classification of infinite or finite groups};
\draw (203.59999999999997, -94.39999999999999) rectangle (230.49999999999997,-100.6);
\draw(204.59999999999997, -95.39999999999999) node[anchor=north west,align=left] {Free products of groups, free products with amalgamation,\\ Higman-Neumann-Neumann extensions, and generalizations};

\draw(219.54999999999995, -95.39999999999999) node[anchor=north west,align=left] {Residual properties and \\ generalizations; residually finite groups};

\draw(204.59999999999997, -96.6) node[anchor=north west,align=left] {Quasivarieties and varieties of groups};

\draw(214.79999999999995, -96.6) node[anchor=north west,align=left] {General structure theorems for groups};

\draw(204.59999999999997, -97.3) node[anchor=north west,align=left] {Groups with a \(BN\)-pair; buildings};

\draw(214.29999999999995, -97.3) node[anchor=north west,align=left] {Subgroup theorems; subgroup growth};

\draw(204.59999999999997, -97.99999999999999) node[anchor=north west,align=left] {Extensions, wreath products, \\ and other compositions of groups};

\draw(213.29999999999995, -97.99999999999999) node[anchor=north west,align=left] {Automorphisms of infinite groups};

\draw(221.99999999999997, -97.99999999999999) node[anchor=north west,align=left] {Chains and lattices of \\ subgroups, subnormal subgroups};

\draw(204.59999999999997, -99.19999999999999) node[anchor=north west,align=left] {Conjugacy classes for groups};

\draw(212.29999999999995, -99.19999999999999) node[anchor=north west,align=left] {Local properties of groups};

\draw(219.49999999999997, -99.19999999999999) node[anchor=north west,align=left] {Limits, profinite groups};

\draw(204.59999999999997, -99.89999999999999) node[anchor=north west,align=left] {Free nonabelian groups};

\draw(210.79999999999995, -99.89999999999999) node[anchor=north west,align=left] {Groups acting on trees};

\draw(216.99999999999997, -99.89999999999999) node[anchor=north west,align=left] {Maximal subgroups};

\draw(221.94999999999996, -99.89999999999999) node[anchor=north west,align=left] {Simple groups};

\draw(230.59999999999997, -94.39999999999999) node[anchor=north west] { \large Abstract finite groups};
\draw (230.59999999999997, -94.39999999999999) rectangle (256.99999999999994,-100.89999999999999);
\draw(231.59999999999997, -95.39999999999999) node[anchor=north west,align=left] {Finite solvable groups, theory of formations, Schunck\\ classes, Fitting classes, \(\pi\)-length, ranks};

\draw(245.54999999999995, -95.39999999999999) node[anchor=north west,align=left] {Special subgroups (Frattini, Fitting, etc.)};

\draw(231.59999999999997, -96.6) node[anchor=north west,align=left] {Automorphisms of abstract finite groups};

\draw(242.04999999999995, -96.6) node[anchor=north west,align=left] {Finite nilpotent groups, \(p\)-groups};

\draw(231.59999999999997, -97.3) node[anchor=north west,align=left] {Arithmetic and combinatorial problems\\ involving abstract finite groups};

\draw(241.54999999999995, -97.3) node[anchor=north west,align=left] {Sylow subgroups, Sylow properties,\\ \(\pi\)-groups, \(\pi\)-structure};

\draw(231.59999999999997, -98.49999999999999) node[anchor=north west,align=left] {Series and lattices of subgroups};

\draw(240.29999999999995, -98.49999999999999) node[anchor=north west,align=left] {Simple groups: sporadic groups};

\draw(248.49999999999997, -98.49999999999999) node[anchor=north west,align=left] {Simple groups: alternating \\ groups and groups of Lie type};

\draw(231.59999999999997, -99.69999999999999) node[anchor=north west,align=left] {Products of subgroups \\ of abstract finite groups};

\draw(238.54999999999995, -99.69999999999999) node[anchor=north west,align=left] {Finite simple groups \\ and their classification};

\draw(245.24999999999997, -99.69999999999999) node[anchor=north west,align=left] {Subnormal subgroups of\\ abstract finite groups};

\draw(203.59999999999997, -100.99999999999999) node[anchor=north west] { \large Special aspects of infinite or finite groups};
\draw (203.59999999999997, -100.99999999999999) rectangle (229.74999999999997,-113.19999999999999);
\draw(204.59999999999997, -101.99999999999999) node[anchor=north west,align=left] {Word problems, other decision problems, connections\\ with logic and automata (group-theoretic aspects)};

\draw(218.04999999999995, -101.99999999999999) node[anchor=north west,align=left] {Other groups related to topology or analysis};

\draw(204.59999999999997, -103.19999999999999) node[anchor=north west,align=left] {Representations of groups as \\ automorphism groups of algebraic systems};

\draw(215.29999999999995, -103.19999999999999) node[anchor=north west,align=left] {Ordered groups (group-theoretic aspects)};

\draw(204.59999999999997, -104.39999999999999) node[anchor=north west,align=left] {Fundamental groups and their \\ automorphisms (group-theoretic aspects)};

\draw(215.04999999999995, -104.39999999999999) node[anchor=north west,align=left] {Periodic groups; locally finite groups};

\draw(204.59999999999997, -105.59999999999998) node[anchor=north west,align=left] {Solvable groups, supersolvable groups};

\draw(214.54999999999995, -105.59999999999998) node[anchor=north west,align=left] {Formations of groups, Fitting classes};

\draw(204.59999999999997, -106.29999999999998) node[anchor=north west,align=left] {Associated Lie structures for groups};

\draw(214.29999999999995, -106.29999999999998) node[anchor=north west,align=left] {FC-groups and their generalizations};

\draw(204.59999999999997, -106.99999999999999) node[anchor=north west,align=left] {Cancellation theory of groups; \\ application of van Kampen diagrams};

\draw(213.79999999999995, -106.99999999999999) node[anchor=north west,align=left] {Derived series, central series,\\ and generalizations for groups};

\draw(204.59999999999997, -108.19999999999999) node[anchor=north west,align=left] {Asymptotic properties of groups};

\draw(213.04999999999995, -108.19999999999999) node[anchor=north west,align=left] {Automorphism groups of groups};

\draw(220.99999999999997, -108.19999999999999) node[anchor=north west,align=left] {Reflection and Coxeter groups\\ (group-theoretic aspects)};

\draw(204.59999999999997, -109.39999999999999) node[anchor=north west,align=left] {Algebraic geometry over \\ groups; equations over groups};

\draw(212.54999999999995, -109.39999999999999) node[anchor=north west,align=left] {Groups of finite Morley rank};

\draw(220.24999999999997, -109.39999999999999) node[anchor=north west,align=left] {Generalizations of solvable\\ and nilpotent groups};

\draw(204.59999999999997, -110.6) node[anchor=north west,align=left] {Hyperbolic groups and \\ nonpositively curved groups};

\draw(212.04999999999995, -110.6) node[anchor=north west,align=left] {Generators, relations, and\\ presentations of groups};

\draw(219.24999999999997, -110.6) node[anchor=north west,align=left] {Other classes of groups \\ defined by subgroup chains};

\draw(204.59999999999997, -111.79999999999998) node[anchor=north west,align=left] {Braid groups; Artin groups};

\draw(211.79999999999995, -111.79999999999998) node[anchor=north west,align=left] {Geometric group theory};

\draw(217.99999999999997, -111.79999999999998) node[anchor=north west,align=left] {Commutator calculus};

\draw(223.44999999999996, -111.79999999999998) node[anchor=north west,align=left] {Nilpotent groups};

\draw(204.59999999999997, -112.49999999999999) node[anchor=north west,align=left] {Engel conditions};

\draw(229.84999999999997, -100.99999999999999) node[anchor=north west] { \large Representation theory of groups};
\draw (229.84999999999997, -100.99999999999999) rectangle (254.24999999999997,-108.39999999999999);
\draw(230.84999999999997, -101.99999999999999) node[anchor=north west,align=left] {Representations of infinite symmetric groups};

\draw(242.54999999999995, -101.99999999999999) node[anchor=north west,align=left] {Representations of finite groups of Lie type};

\draw(230.84999999999997, -102.69999999999999) node[anchor=north west,align=left] {\(p\)-adic representations of finite groups};

\draw(242.29999999999995, -102.69999999999999) node[anchor=north west,align=left] {Integral representations of infinite groups};

\draw(230.84999999999997, -103.39999999999999) node[anchor=north west,align=left] {Projective representations and multipliers};

\draw(242.04999999999995, -103.39999999999999) node[anchor=north west,align=left] {Representations of finite symmetric groups};

\draw(230.84999999999997, -104.09999999999998) node[anchor=north west,align=left] {Integral representations of finite groups};

\draw(241.79999999999995, -104.09999999999998) node[anchor=north west,align=left] {Hecke algebras and their representations};

\draw(230.84999999999997, -104.79999999999998) node[anchor=north west,align=left] {Group rings of infinite groups and \\ their modules (group-theoretic aspects)};

\draw(241.29999999999995, -104.79999999999998) node[anchor=north west,align=left] {Ordinary representations and characters};

\draw(230.84999999999997, -105.99999999999999) node[anchor=north west,align=left] {Group rings of finite groups and their\\ modules (group-theoretic aspects)};

\draw(241.04999999999995, -105.99999999999999) node[anchor=north west,align=left] {Modular representations and characters};

\draw(230.84999999999997, -107.19999999999999) node[anchor=north west,align=left] {Applications of group representations\\ to physics and other areas of science};

\draw(240.79999999999995, -107.19999999999999) node[anchor=north west,align=left] {Representations of sporadic groups};

\draw(229.84999999999997, -108.49999999999999) node[anchor=north west] { \large Groupoids (i.e. small categories in which all morphisms are isomorphisms)};
\draw (229.84999999999997, -108.49999999999999) rectangle (252.34999999999997,-110.69999999999999);
\draw(230.84999999999997, -109.49999999999999) node[anchor=north west,align=left] {Groupoids (i.e. small categories in\\ which all morphisms are isomorphisms)};

\draw(229.84999999999997, -110.8) node[anchor=north west] { \large Probabilistic methods in group theory};
\draw (229.84999999999997, -110.8) rectangle (241.54999999999995,-112.5);
\draw(230.84999999999997, -111.8) node[anchor=north west,align=left] {Probabilistic methods in group theory};

\draw(203.59999999999997, -113.29999999999998) node[anchor=north west] { \large Linear algebraic groups and related topics};
\draw (203.59999999999997, -113.29999999999998) rectangle (227.49999999999997,-120.99999999999999);
\draw(204.59999999999997, -114.29999999999998) node[anchor=north west,align=left] {Structure theory for linear algebraic groups};

\draw(216.29999999999995, -114.29999999999998) node[anchor=north west,align=left] {Linear algebraic groups over finite fields};

\draw(204.59999999999997, -114.99999999999999) node[anchor=north west,align=left] {Linear algebraic groups over the \\ reals, the complexes, the quaternions};

\draw(214.54999999999995, -114.99999999999999) node[anchor=north west,align=left] {Quantum groups (quantized function\\ algebras) and their representations};

\draw(204.59999999999997, -116.19999999999999) node[anchor=north west,align=left] {Linear algebraic groups over \\ adèles and other rings and schemes};

\draw(213.79999999999995, -116.19999999999999) node[anchor=north west,align=left] {Linear algebraic groups over \\ global fields and their integers};

\draw(204.59999999999997, -117.39999999999998) node[anchor=north west,align=left] {Applications of linear \\ algebraic groups to the sciences};

\draw(213.29999999999995, -117.39999999999998) node[anchor=north west,align=left] {Linear algebraic groups over \\ local fields and their integers};

\draw(204.59999999999997, -118.59999999999998) node[anchor=north west,align=left] {Schur and \(q\)-Schur algebras};

\draw(212.79999999999995, -118.59999999999998) node[anchor=north west,align=left] {Representation theory for\\ linear algebraic groups};

\draw(219.74999999999997, -118.59999999999998) node[anchor=north west,align=left] {Cohomology theory for\\ linear algebraic groups};

\draw(204.59999999999997, -119.79999999999998) node[anchor=north west,align=left] {Linear algebraic groups\\ over arbitrary fields};

\draw(211.04999999999995, -119.79999999999998) node[anchor=north west,align=left] {Exceptional groups};

\draw(216.24999999999997, -119.79999999999998) node[anchor=north west,align=left] {Kac-Moody groups};

\draw(227.59999999999997, -113.29999999999998) node[anchor=north west] { \large Permutation groups};
\draw (227.59999999999997, -113.29999999999998) rectangle (250.99999999999997,-118.79999999999998);
\draw(228.59999999999997, -114.29999999999998) node[anchor=north west,align=left] {General theory for finite permutation groups};

\draw(240.29999999999995, -114.29999999999998) node[anchor=north west,align=left] {Finite automorphism groups of algebraic,\\ geometric, or combinatorial structures};

\draw(228.59999999999997, -115.49999999999999) node[anchor=north west,align=left] {Multiply transitive infinite groups};

\draw(238.04999999999995, -115.49999999999999) node[anchor=north west,align=left] {Multiply transitive finite groups};

\draw(228.59999999999997, -116.19999999999999) node[anchor=north west,align=left] {Subgroups of symmetric groups};

\draw(236.54999999999995, -116.19999999999999) node[anchor=north west,align=left] {Infinite automorphism groups};

\draw(228.59999999999997, -116.89999999999998) node[anchor=north west,align=left] {General theory for \\ infinite permutation groups};

\draw(236.04999999999995, -116.89999999999998) node[anchor=north west,align=left] {Characterization theorems\\ for permutation groups};

\draw(242.99999999999997, -116.89999999999998) node[anchor=north west,align=left] {Primitive groups};

\draw(228.59999999999997, -118.09999999999998) node[anchor=north west,align=left] {Symmetric groups};

\draw(227.59999999999997, -118.89999999999999) node[anchor=north west,align=left] {Computational methods for \\ problems pertaining to group theory};

\draw(227.59999999999997, -120.1) node[anchor=north west,align=left] {History of group theory};

\draw(203.59999999999997, -121.1) node[anchor=north west] { \large Semigroups};
\draw (203.59999999999997, -121.1) rectangle (226.64999999999998,-130.2);
\draw(204.59999999999997, -122.1) node[anchor=north west,align=left] {Varieties and pseudovarieties of semigroups};

\draw(216.04999999999995, -122.1) node[anchor=north west,align=left] {General structure theory for semigroups};

\draw(204.59999999999997, -122.8) node[anchor=north west,align=left] {Connections of semigroups with \\ homological algebra and category theory};

\draw(215.04999999999995, -122.8) node[anchor=north west,align=left] {Arithmetic theory of semigroups};

\draw(204.59999999999997, -124.0) node[anchor=north west,align=left] {Semigroup rings, multiplicative\\ semigroups of rings};

\draw(213.04999999999995, -124.0) node[anchor=north west,align=left] {Semigroups of transformations,\\ relations, partitions, etc.};

\draw(204.59999999999997, -125.19999999999999) node[anchor=north west,align=left] {Radical theory for semigroups};

\draw(212.54999999999995, -125.19999999999999) node[anchor=north west,align=left] {Representation of semigroups;\\ actions of semigroups on sets};

\draw(204.59999999999997, -126.39999999999999) node[anchor=north west,align=left] {Generalizations of semigroups};

\draw(212.54999999999995, -126.39999999999999) node[anchor=north west,align=left] {Free semigroups, generators\\ and relations, word problems};

\draw(204.59999999999997, -127.6) node[anchor=north west,align=left] {Ideal theory for semigroups};

\draw(212.04999999999995, -127.6) node[anchor=north west,align=left] {Semigroups in automata \\ theory, linguistics, etc.};

\draw(218.99999999999997, -127.6) node[anchor=north west,align=left] {Commutative semigroups};

\draw(204.59999999999997, -128.79999999999998) node[anchor=north west,align=left] {Mappings of semigroups};

\draw(210.79999999999995, -128.79999999999998) node[anchor=north west,align=left] {Orthodox semigroups};

\draw(216.24999999999997, -128.79999999999998) node[anchor=north west,align=left] {Regular semigroups};

\draw(221.44999999999996, -128.79999999999998) node[anchor=north west,align=left] {Inverse semigroups};

\draw(204.59999999999997, -129.5) node[anchor=north west,align=left] {Algebraic monoids};

\draw(226.74999999999997, -121.1) node[anchor=north west] { \large Connections of group theory with homological algebra and category theory};
\draw (226.74999999999997, -121.1) rectangle (248.94999999999996,-123.5);
\draw(227.74999999999997, -122.1) node[anchor=north west,align=left] {Homological methods in group theory};

\draw(237.19999999999996, -122.1) node[anchor=north west,align=left] {Cohomology of groups};

\draw(227.74999999999997, -122.8) node[anchor=north west,align=left] {Category of groups};

\draw(226.74999999999997, -123.6) node[anchor=north west] { \large Other groups of matrices};
\draw (226.74999999999997, -123.6) rectangle (248.89999999999998,-127.69999999999999);
\draw(227.74999999999997, -124.6) node[anchor=north west,align=left] {Fuchsian groups and their \\ generalizations (group-theoretic aspects)};

\draw(238.69999999999996, -124.6) node[anchor=north west,align=left] {Other matrix groups over finite fields};

\draw(227.74999999999997, -125.8) node[anchor=north west,align=left] {Unimodular groups, congruence \\ subgroups (group-theoretic aspects)};

\draw(237.19999999999996, -125.8) node[anchor=north west,align=left] {Other geometric groups, \\ including crystallographic groups};

\draw(227.74999999999997, -127.0) node[anchor=north west,align=left] {Other matrix groups over fields};

\draw(236.19999999999996, -127.0) node[anchor=north west,align=left] {Other matrix groups over rings};

\draw(203.59999999999997, -130.29999999999998) node[anchor=north west] { \large Abelian groups};
\draw (203.59999999999997, -130.29999999999998) rectangle (224.99999999999997,-136.29999999999998);
\draw(204.59999999999997, -131.29999999999998) node[anchor=north west,align=left] {Automorphisms, homomorphisms, \\ endomorphisms, etc. for abelian groups};

\draw(214.79999999999995, -131.29999999999998) node[anchor=north west,align=left] {Topological methods for abelian groups};

\draw(204.59999999999997, -132.49999999999997) node[anchor=north west,align=left] {Torsion-free groups, infinite rank};

\draw(213.79999999999995, -132.49999999999997) node[anchor=north west,align=left] {Torsion-free groups, finite rank};

\draw(204.59999999999997, -133.2) node[anchor=north west,align=left] {Torsion groups, primary groups\\ and generalized primary groups};

\draw(212.79999999999995, -133.2) node[anchor=north west,align=left] {Direct sums, direct products,\\ etc. for abelian groups};

\draw(204.59999999999997, -134.39999999999998) node[anchor=north west,align=left] {Extensions of abelian groups};

\draw(212.29999999999995, -134.39999999999998) node[anchor=north west,align=left] {Subgroups of abelian groups};

\draw(204.59999999999997, -135.1) node[anchor=north west,align=left] {Homological and categorical\\ methods for abelian groups};

\draw(212.04999999999995, -135.1) node[anchor=north west,align=left] {Finite abelian groups};

\draw(217.99999999999997, -135.1) node[anchor=north west,align=left] {Mixed groups};

\draw(225.09999999999997, -130.29999999999998) node[anchor=north west] { \large Foundations};
\draw (225.09999999999997, -130.29999999999998) rectangle (244.24999999999997,-133.7);
\draw(226.09999999999997, -131.29999999999998) node[anchor=north west,align=left] {Applications of logic to group theory};

\draw(236.04999999999995, -131.29999999999998) node[anchor=north west,align=left] {Metamathematical \\ considerations in group theory};

\draw(226.09999999999997, -132.49999999999997) node[anchor=north west,align=left] {Axiomatics and elementary\\ properties of groups};

\draw(203.59999999999997, -136.39999999999998) node[anchor=north west] { \large Other generalizations of groups};
\draw (203.59999999999997, -136.39999999999998) rectangle (219.99999999999997,-140.49999999999997);
\draw(204.59999999999997, -137.39999999999998) node[anchor=north west,align=left] {\(n\)-ary systems \((n\ge 3)\)};

\draw(212.79999999999995, -137.39999999999998) node[anchor=north west,align=left] {Ternary systems (heaps, \\ semiheaps, heapoids, etc.)};

\draw(204.59999999999997, -138.59999999999997) node[anchor=north west,align=left] {Sets with a single binary\\ operation (groupoids)};

\draw(211.54999999999995, -138.59999999999997) node[anchor=north west,align=left] {Loops, quasigroups};

\draw(204.59999999999997, -139.79999999999998) node[anchor=north west,align=left] {Fuzzy groups};

\draw(208.29999999999995, -139.79999999999998) node[anchor=north west,align=left] {Hypergroups};

\draw(260.05, -1) node[anchor=north west] { \large Topological groups, Lie groups};
\draw (260.05, -1) rectangle (314.0,-17.6);
\draw(261.05, -2) node[anchor=north west] { \large Lie groups};
\draw (261.05, -2) rectangle (290.65000000000003,-12.600000000000001);
\draw(262.05, -3) node[anchor=north west,align=left] {Representations of nilpotent and solvable Lie groups \\ (special orbital integrals, non-type I representations, etc.)};

\draw(278.0, -3) node[anchor=north west,align=left] {Representations of Lie and real algebraic \\ groups: algebraic methods (Verma modules, etc.)};

\draw(262.05, -4.2) node[anchor=north west,align=left] {Representations of Lie and linear algebraic\\ groups over real fields: analytic methods};

\draw(273.5, -4.2) node[anchor=north west,align=left] {Representations of Lie and linear algebraic\\ groups over global fields and adèle rings};

\draw(262.05, -5.4) node[anchor=north west,align=left] {Analysis on real and complex Lie groups};

\draw(272.5, -5.4) node[anchor=north west,align=left] {Infinite-dimensional Lie groups and \\ their Lie algebras: general properties};

\draw(262.05, -6.6000000000000005) node[anchor=north west,align=left] {Loop groups and related constructions,\\ group-theoretic treatment};

\draw(272.25, -6.6000000000000005) node[anchor=north west,align=left] {Continuous cohomologyof Lie groups};

\draw(281.45, -6.6000000000000005) node[anchor=north west,align=left] {Representations of Lie and linear\\ algebraic groups over local fields};

\draw(262.05, -7.800000000000001) node[anchor=north west,align=left] {Analysis on and representations \\ of infinite-dimensional Lie groups};

\draw(271.25, -7.800000000000001) node[anchor=north west,align=left] {Applications of Lie groups to the\\ sciences; explicit representations};

\draw(280.45, -7.800000000000001) node[anchor=north west,align=left] {Nilpotent and solvable Lie groups};

\draw(262.05, -9.0) node[anchor=north west,align=left] {Analysis on \(p\)-adic Lie groups};

\draw(271.0, -9.0) node[anchor=north west,align=left] {Discrete subgroups of Lie groups};

\draw(279.7, -9.0) node[anchor=north west,align=left] {Geometric Langlands program: \\ representation-theoretic aspects};

\draw(262.05, -10.200000000000001) node[anchor=north west,align=left] {General properties and \\ structure of complex Lie groups};

\draw(270.5, -10.200000000000001) node[anchor=north west,align=left] {General properties and \\ structure of other Lie groups};

\draw(278.45, -10.200000000000001) node[anchor=north west,align=left] {General properties and \\ structure of real Lie groups};

\draw(262.05, -11.400000000000002) node[anchor=north west,align=left] {Structure and representation\\ of the Lorentz group};

\draw(269.75, -11.400000000000002) node[anchor=north west,align=left] {Lie algebras of Lie groups};

\draw(276.95, -11.400000000000002) node[anchor=north west,align=left] {Semisimple Lie groups \\ and their representations};

\draw(283.90000000000003, -11.400000000000002) node[anchor=north west,align=left] {Local Lie groups};

\draw(290.75, -2) node[anchor=north west] { \large Locally compact groups and their algebras};
\draw (290.75, -2) rectangle (313.9,-9.200000000000001);
\draw(291.75, -3) node[anchor=north west,align=left] {Duality theorems for locally compact groups};

\draw(303.2, -3) node[anchor=north west,align=left] {Group algebras of locally compact groups};

\draw(291.75, -3.7) node[anchor=north west,align=left] {\(C^*\)-algebras and \(W^*\)-algebras\\ in relation to group representations};

\draw(301.7, -3.7) node[anchor=north west,align=left] {Kazhdan’s property (T), the Haagerup\\ property, and generalizations};

\draw(291.75, -4.9) node[anchor=north west,align=left] {Rigidity in locally compact groups};

\draw(300.95, -4.9) node[anchor=north west,align=left] {Representations of group algebras};

\draw(291.75, -5.6000000000000005) node[anchor=north west,align=left] {General properties and structure\\ of locally compact groups};

\draw(300.45, -5.6000000000000005) node[anchor=north west,align=left] {Induced representations \\ for locally compact groups};

\draw(291.75, -6.800000000000001) node[anchor=north west,align=left] {Unitary representations\\ of locally compact groups};

\draw(298.7, -6.800000000000001) node[anchor=north west,align=left] {Other representations \\ of locally compact groups};

\draw(305.65, -6.800000000000001) node[anchor=north west,align=left] {Ergodic theory on groups};

\draw(291.75, -8.0) node[anchor=north west,align=left] {Automorphism groups of\\ locally compact groups};

\draw(290.75, -9.3) node[anchor=north west] { \large Noncompact transformation groups};
\draw (290.75, -9.3) rectangle (309.9,-12.200000000000001);
\draw(291.75, -10.3) node[anchor=north west,align=left] {Groups as automorphisms of other structures};

\draw(303.2, -10.3) node[anchor=north west,align=left] {Measurable group actions};

\draw(291.75, -11.0) node[anchor=north west,align=left] {General theory of group\\ and pseudogroup actions};

\draw(298.2, -11.0) node[anchor=north west,align=left] {Homogeneous spaces};

\draw(261.05, -12.700000000000001) node[anchor=north west] { \large Topological and differentiable algebraic systems};
\draw (261.05, -12.700000000000001) rectangle (282.7,-17.5);
\draw(262.05, -13.700000000000001) node[anchor=north west,align=left] {Structure of general topological groups};

\draw(272.5, -13.700000000000001) node[anchor=north west,align=left] {Analysis on general topological groups};

\draw(262.05, -14.4) node[anchor=north west,align=left] {Structure of topological semigroups};

\draw(271.5, -14.4) node[anchor=north west,align=left] {Analysis on topological semigroups};

\draw(262.05, -15.100000000000001) node[anchor=north west,align=left] {Topological groupoids (including\\ differentiable and Lie groupoids)};

\draw(271.0, -15.100000000000001) node[anchor=north west,align=left] {Representations of general \\ topological groups and semigroups};

\draw(262.05, -16.3) node[anchor=north west,align=left] {Other topological algebraic \\ systems and their representations};

\draw(271.0, -16.3) node[anchor=north west,align=left] {Topological semilattices,\\ lattices and applications};

\draw(282.8, -12.700000000000001) node[anchor=north west] { \large Locally compact abelian groups (LCA groups)};
\draw (282.8, -12.700000000000001) rectangle (301.2,-14.900000000000002);
\draw(283.8, -13.700000000000001) node[anchor=north west,align=left] {Structure of group algebras of LCA groups};

\draw(294.75, -13.700000000000001) node[anchor=north west,align=left] {General properties and\\ structure of LCA groups};

\draw(282.8, -15.000000000000002) node[anchor=north west,align=left] {Computational methods for problems\\ pertaining to topological groups};

\draw(282.8, -16.200000000000003) node[anchor=north west,align=left] {History of topological groups};

\draw(301.3, -12.700000000000001) node[anchor=north west] { \large Compact groups};
\draw (301.3, -12.700000000000001) rectangle (306.5,-14.400000000000002);
\draw(302.3, -13.700000000000001) node[anchor=north west,align=left] {Compact groups};

\draw(260.05, -17.700000000000003) node[anchor=north west] { \large Potential theory};
\draw (260.05, -17.700000000000003) rectangle (313.05,-29.5);
\draw(261.05, -18.700000000000003) node[anchor=north west] { \large Two-dimensional potential theory};
\draw (261.05, -18.700000000000003) rectangle (287.95,-24.500000000000004);
\draw(262.05, -19.700000000000003) node[anchor=north west,align=left] {Potentials and capacity, harmonic measure, extremal\\ length and related notions in two dimensions};

\draw(275.5, -19.700000000000003) node[anchor=north west,align=left] {Biharmonic, polyharmonic functions and \\ equations, Poisson’s equation in two dimensions};

\draw(262.05, -20.900000000000002) node[anchor=north west,align=left] {Integral representations, integral operators,\\ integral equations methods in two dimensions};

\draw(274.0, -20.900000000000002) node[anchor=north west,align=left] {Boundary behavior (theorems of Fatou type, \\ etc.) of harmonic functions in two dimensions};

\draw(262.05, -22.1) node[anchor=north west,align=left] {Connections of harmonic functions with\\ differential equations in two dimensions};

\draw(272.75, -22.1) node[anchor=north west,align=left] {Boundary value and inverse problems for\\ harmonic functions in two dimensions};

\draw(262.05, -23.300000000000004) node[anchor=north west,align=left] {Harmonic, subharmonic, superharmonic\\ functions in two dimensions};

\draw(288.05, -18.700000000000003) node[anchor=north west] { \large Higher-dimensional potential theory};
\draw (288.05, -18.700000000000003) rectangle (312.95,-24.500000000000004);
\draw(289.05, -19.700000000000003) node[anchor=north west,align=left] {Integral representations, integral operators,\\ integral equations methods in higher dimensions};

\draw(301.5, -19.700000000000003) node[anchor=north west,align=left] {Connections of harmonic functions with \\ differential equations in higher dimensions};

\draw(289.05, -20.900000000000002) node[anchor=north west,align=left] {Potentials and capacities, extremal length\\ and related notions in higher dimensions};

\draw(300.25, -20.900000000000002) node[anchor=north west,align=left] {Boundary value and inverse problems for\\ harmonic functions in higher dimensions};

\draw(289.05, -22.1) node[anchor=north west,align=left] {Biharmonic and polyharmonic equations\\ and functions in higher dimensions};

\draw(299.0, -22.1) node[anchor=north west,align=left] {Harmonic, subharmonic, superharmonic\\ functions in higher dimensions};

\draw(289.05, -23.300000000000004) node[anchor=north west,align=left] {Boundary behavior of harmonic\\ functions in higher dimensions};

\draw(261.05, -24.6) node[anchor=north west] { \large Generalizations of potential theory};
\draw (261.05, -24.6) rectangle (284.7,-29.400000000000002);
\draw(262.05, -25.6) node[anchor=north west,align=left] {Pluriharmonic and plurisubharmonic functions};

\draw(273.75, -25.6) node[anchor=north west,align=left] {Potentials and capacities on other spaces};

\draw(262.05, -26.3) node[anchor=north west,align=left] {Harmonic, subharmonic, superharmonic\\ functions on other spaces};

\draw(271.75, -26.3) node[anchor=north west,align=left] {Fine potential theory; fine \\ properties of sets and functions};

\draw(262.05, -27.5) node[anchor=north west,align=left] {Other generalizations (nonlinear\\ potential theory, etc.)};

\draw(270.75, -27.5) node[anchor=north west,align=left] {Potential theory on Riemannian\\ manifolds and other spaces};

\draw(262.05, -28.700000000000003) node[anchor=north west,align=left] {Discrete potential theory};

\draw(269.0, -28.700000000000003) node[anchor=north west,align=left] {Martin boundary theory};

\draw(275.2, -28.700000000000003) node[anchor=north west,align=left] {Dirichlet forms};

\draw(284.8, -24.6) node[anchor=north west] { \large Potential theory on fractals and metric spaces};
\draw (284.8, -24.6) rectangle (299.2,-26.8);
\draw(285.8, -25.6) node[anchor=north west,align=left] {Potential theory on \\ fractals and metric spaces};

\draw(284.8, -26.900000000000006) node[anchor=north west,align=left] {Computational methods for problems\\ pertaining to potential theory};

\draw(284.8, -28.1) node[anchor=north west,align=left] {History of potential theory};

\draw(299.3, -24.6) node[anchor=north west] { \large Axiomatic potential theory};
\draw (299.3, -24.6) rectangle (307.7,-26.3);
\draw(300.3, -25.6) node[anchor=north west,align=left] {Axiomatic potential theory};

\draw(260.05, -29.6) node[anchor=north west] { \large Mathematical logic and foundations};
\draw (260.05, -29.6) rectangle (311.5,-66.6);
\draw(261.05, -30.6) node[anchor=north west] { \large General logic};
\draw (261.05, -30.6) rectangle (287.65000000000003,-40.2);
\draw(262.05, -31.6) node[anchor=north west,align=left] {Substructural logics (including relevance, entailment,\\ linear logic, Lambek calculus, BCK and BCI logics)};

\draw(276.25, -31.6) node[anchor=north west,align=left] {Modal logic (including the logic of norms)};

\draw(262.05, -32.800000000000004) node[anchor=north west,align=left] {Combinatory logic and lambda calculus};

\draw(272.0, -32.800000000000004) node[anchor=north west,align=left] {Foundations of classical theories\\ (including reverse mathematics)};

\draw(262.05, -34.0) node[anchor=north west,align=left] {Subsystems of classical logic \\ (including intuitionistic logic)};

\draw(270.75, -34.0) node[anchor=north west,align=left] {Probability and inductive logic};

\draw(279.2, -34.0) node[anchor=north west,align=left] {Fuzzy logic; logic of vagueness};

\draw(262.05, -35.2) node[anchor=north west,align=left] {Logics of knowledge and belief\\ (including belief change)};

\draw(270.25, -35.2) node[anchor=north west,align=left] {Classical propositional logic};

\draw(278.2, -35.2) node[anchor=north west,align=left] {Classical first-order logic};

\draw(262.05, -36.400000000000006) node[anchor=north west,align=left] {Other applications of logic};

\draw(269.5, -36.400000000000006) node[anchor=north west,align=left] {Abstract deductive systems};

\draw(276.7, -36.400000000000006) node[anchor=north west,align=left] {Logic of natural languages};

\draw(262.05, -37.1) node[anchor=north west,align=left] {Logic in computer science};

\draw(269.0, -37.1) node[anchor=north west,align=left] {Decidability of theories\\ and sets of sentences};

\draw(275.7, -37.1) node[anchor=north west,align=left] {Other nonclassical logic};

\draw(262.05, -38.300000000000004) node[anchor=north west,align=left] {Mechanization of proofs\\ and logical operations};

\draw(268.5, -38.300000000000004) node[anchor=north west,align=left] {Paraconsistent logics};

\draw(274.45, -38.300000000000004) node[anchor=north west,align=left] {Intermediate logics};

\draw(279.90000000000003, -38.300000000000004) node[anchor=north west,align=left] {Higher-order logic};

\draw(262.05, -39.5) node[anchor=north west,align=left] {Many-valued logic};

\draw(267.0, -39.5) node[anchor=north west,align=left] {Combined logics};

\draw(271.45, -39.5) node[anchor=north west,align=left] {Temporal logic};

\draw(275.65000000000003, -39.5) node[anchor=north west,align=left] {Type theory};

\draw(287.75, -30.6) node[anchor=north west] { \large Set theory};
\draw (287.75, -30.6) rectangle (311.4,-39.900000000000006);
\draw(288.75, -31.6) node[anchor=north west,align=left] {Other notions of set-theoretic definability};

\draw(300.2, -31.6) node[anchor=north west,align=left] {Nonclassical and second-order set theories};

\draw(288.75, -32.300000000000004) node[anchor=north west,align=left] {Cardinal characteristics of the continuum};

\draw(299.7, -32.300000000000004) node[anchor=north west,align=left] {Inner models, including constructibility,\\ ordinal definability, and core models};

\draw(288.75, -33.5) node[anchor=north west,align=left] {Other set-theoretic hypotheses and axioms};

\draw(299.7, -33.5) node[anchor=north west,align=left] {Axiom of choice and related propositions};

\draw(288.75, -34.2) node[anchor=north west,align=left] {Continuum hypothesis and Martin’s axiom};

\draw(299.2, -34.2) node[anchor=north west,align=left] {Generic absoluteness and forcing axioms};

\draw(288.75, -34.900000000000006) node[anchor=north west,align=left] {Other classical set theory (including\\ functions, relations, and set algebra)};

\draw(298.95, -34.900000000000006) node[anchor=north west,align=left] {Consistency and independence results};

\draw(288.75, -36.1) node[anchor=north west,align=left] {Other combinatorial set theory};

\draw(296.95, -36.1) node[anchor=north west,align=left] {Ordinal and cardinal numbers};

\draw(288.75, -36.800000000000004) node[anchor=north west,align=left] {Axiomatics of classical set\\ theory and its fragments};

\draw(296.2, -36.800000000000004) node[anchor=north west,align=left] {Theory of fuzzy sets, etc.};

\draw(303.4, -36.800000000000004) node[anchor=north west,align=left] {Applications of set theory};

\draw(288.75, -38.0) node[anchor=north west,align=left] {Other aspects of forcing\\ and Boolean-valued models};

\draw(295.7, -38.0) node[anchor=north west,align=left] {Ordered sets and their\\ cofinalities; pcf theory};

\draw(302.4, -38.0) node[anchor=north west,align=left] {Descriptive set theory};

\draw(288.75, -39.2) node[anchor=north west,align=left] {Determinacy principles};

\draw(294.95, -39.2) node[anchor=north west,align=left] {Partition relations};

\draw(300.4, -39.2) node[anchor=north west,align=left] {Large cardinals};

\draw(261.05, -40.300000000000004) node[anchor=north west] { \large Model theory};
\draw (261.05, -40.300000000000004) rectangle (284.40000000000003,-53.900000000000006);
\draw(262.05, -41.300000000000004) node[anchor=north west,align=left] {Logic with extra quantifiers and operators};

\draw(273.25, -41.300000000000004) node[anchor=north west,align=left] {Categoricity and completeness of theories};

\draw(262.05, -42.00000000000001) node[anchor=north west,align=left] {Interpolation, preservation, definability};

\draw(273.0, -42.00000000000001) node[anchor=north west,align=left] {Other classical first-order model theory};

\draw(262.05, -42.7) node[anchor=north west,align=left] {Ultraproducts and related constructions};

\draw(272.5, -42.7) node[anchor=north west,align=left] {Models of other mathematical theories};

\draw(262.05, -43.400000000000006) node[anchor=north west,align=left] {Second- and higher-order model theory};

\draw(272.0, -43.400000000000006) node[anchor=north west,align=left] {Classification theory, stability \\ and related concepts in model theory};

\draw(262.05, -44.6) node[anchor=north west,align=left] {Models of arithmetic and set theory};

\draw(271.5, -44.6) node[anchor=north west,align=left] {Model theory of finite structures};

\draw(262.05, -45.300000000000004) node[anchor=north west,align=left] {Basic properties of first-order\\ languages and structures};

\draw(270.5, -45.300000000000004) node[anchor=north west,align=left] {Quantifier elimination, model\\ completeness and related topics};

\draw(262.05, -46.50000000000001) node[anchor=north west,align=left] {Properties of classes of models};

\draw(270.5, -46.50000000000001) node[anchor=north west,align=left] {Models with special properties\\ (saturated, rigid, etc.)};

\draw(262.05, -47.7) node[anchor=north west,align=left] {Continuous model theory, model\\ theory of metric structures};

\draw(270.25, -47.7) node[anchor=north west,align=left] {Equational classes, universal\\ algebra in model theory};

\draw(262.05, -48.900000000000006) node[anchor=north west,align=left] {Nonclassical models \\ (Boolean-valued, sheaf, etc.)};

\draw(270.0, -48.900000000000006) node[anchor=north west,align=left] {Computable structure theory,\\ computable model theory};

\draw(262.05, -50.10000000000001) node[anchor=north west,align=left] {Applications of model theory};

\draw(269.75, -50.10000000000001) node[anchor=north west,align=left] {Model theory of denumerable\\ and separable structures};

\draw(277.2, -50.10000000000001) node[anchor=north west,align=left] {Abstract elementary \\ classes and related topics};

\draw(262.05, -51.300000000000004) node[anchor=north west,align=left] {Set-theoretic model theory};

\draw(269.25, -51.300000000000004) node[anchor=north west,align=left] {Other model constructions};

\draw(276.2, -51.300000000000004) node[anchor=north west,align=left] {Model theory of ordered\\ structures; o-minimality};

\draw(262.05, -52.5) node[anchor=north west,align=left] {Logic on admissible sets};

\draw(268.75, -52.5) node[anchor=north west,align=left] {Model-theoretic forcing};

\draw(275.2, -52.5) node[anchor=north west,align=left] {Model-theoretic algebra};

\draw(262.05, -53.2) node[anchor=north west,align=left] {Other infinitary logic};

\draw(268.25, -53.2) node[anchor=north west,align=left] {Abstract model theory};

\draw(284.5, -40.300000000000004) node[anchor=north west] { \large Proof theory and constructive mathematics};
\draw (284.5, -40.300000000000004) rectangle (306.9,-48.7);
\draw(285.5, -41.300000000000004) node[anchor=north west,align=left] {Cut-elimination and normal-form theorems};

\draw(296.2, -41.300000000000004) node[anchor=north west,align=left] {Recursive ordinals and ordinal notations};

\draw(285.5, -42.00000000000001) node[anchor=north west,align=left] {Relative consistency and interpretations};

\draw(296.2, -42.00000000000001) node[anchor=north west,align=left] {Provability logics and related \\ algebras (e.g., diagonalizable algebras)};

\draw(285.5, -43.2) node[anchor=north west,align=left] {Metamathematics of constructive systems};

\draw(295.95, -43.2) node[anchor=north west,align=left] {First-order arithmetic and fragments};

\draw(285.5, -43.900000000000006) node[anchor=north west,align=left] {Proof-theoretic aspects of linear \\ logic and other substructural logics};

\draw(295.2, -43.900000000000006) node[anchor=north west,align=left] {Constructive and recursive analysis};

\draw(285.5, -45.10000000000001) node[anchor=north west,align=left] {Proof theory, general (including\\ proof-theoretic semantics)};

\draw(294.2, -45.10000000000001) node[anchor=north west,align=left] {Other constructive mathematics};

\draw(285.5, -46.300000000000004) node[anchor=north west,align=left] {Functionals in proof theory};

\draw(292.95, -46.300000000000004) node[anchor=north west,align=left] {Intuitionistic mathematics};

\draw(300.15, -46.300000000000004) node[anchor=north west,align=left] {Second- and higher-order\\ arithmetic and fragments};

\draw(285.5, -47.50000000000001) node[anchor=north west,align=left] {Gödel numberings and \\ issues of incompleteness};

\draw(292.2, -47.50000000000001) node[anchor=north west,align=left] {Complexity of proofs};

\draw(297.9, -47.50000000000001) node[anchor=north west,align=left] {Structure of proofs};

\draw(284.5, -48.800000000000004) node[anchor=north west] { \large Philosophical aspects of logic and foundations};
\draw (284.5, -48.800000000000004) rectangle (303.4,-51.00000000000001);
\draw(285.5, -49.800000000000004) node[anchor=north west,align=left] {Logic in the philosophy of science};

\draw(294.7, -49.800000000000004) node[anchor=north west,align=left] {Philosophical and critical \\ aspects of logic and foundations};

\draw(284.5, -51.10000000000001) node[anchor=north west,align=left] {Computational methods for problems pertaining\\ to mathematical logic and foundations};

\draw(284.5, -52.300000000000004) node[anchor=north west,align=left] {History of mathematical\\ logic and foundations};

\draw(261.05, -54.0) node[anchor=north west] { \large Computability and recursion theory};
\draw (261.05, -54.0) rectangle (281.45,-66.5);
\draw(262.05, -55.0) node[anchor=north west,align=left] {Complexity of computation (including\\ implicit computational complexity)};

\draw(271.75, -55.0) node[anchor=north west,align=left] {Algorithmic randomness and dimension};

\draw(262.05, -56.2) node[anchor=north west,align=left] {Higher-type and set recursion theory};

\draw(271.75, -56.2) node[anchor=north west,align=left] {Turing machines and related notions};

\draw(262.05, -56.9) node[anchor=north west,align=left] {Recursive functions and \\ relations, subrecursive hierarchies};

\draw(271.5, -56.9) node[anchor=north west,align=left] {Other degrees and reducibilities in\\ computability and recursion theory};

\draw(262.05, -58.1) node[anchor=north west,align=left] {Word problems, etc. in \\ computability and recursion theory};

\draw(271.25, -58.1) node[anchor=north west,align=left] {Computability and recursion theory\\ on ordinals, admissible sets, etc.};

\draw(262.05, -59.3) node[anchor=north west,align=left] {Abstract and axiomatic \\ computability and recursion theory};

\draw(271.25, -59.3) node[anchor=north west,align=left] {Automata and formal grammars in\\ connection with logical questions};

\draw(262.05, -60.5) node[anchor=north west,align=left] {Theory of numerations, \\ effectively presented structures};

\draw(270.75, -60.5) node[anchor=north west,align=left] {Other Turing degree structures};

\draw(262.05, -61.7) node[anchor=north west,align=left] {Recursive equivalence types\\ of sets and structures, isols};

\draw(270.0, -61.7) node[anchor=north west,align=left] {Applications of computability\\ and recursion theory};

\draw(262.05, -62.900000000000006) node[anchor=north west,align=left] {Hierarchies of computability\\ and definability};

\draw(269.75, -62.900000000000006) node[anchor=north west,align=left] {Thue and Post systems, etc.};

\draw(262.05, -64.1) node[anchor=north west,align=left] {Recursively (computably) \\ enumerable sets and degrees};

\draw(269.5, -64.1) node[anchor=north west,align=left] {Undecidability and degrees\\ of sets of sentences};

\draw(262.05, -65.3) node[anchor=north west,align=left] {Computation over the \\ reals, computable analysis};

\draw(269.25, -65.3) node[anchor=north west,align=left] {Inductive definability};

\draw(281.55, -54.0) node[anchor=north west] { \large Algebraic logic};
\draw (281.55, -54.0) rectangle (300.45,-58.8);
\draw(282.55, -55.0) node[anchor=north west,align=left] {Logical aspects of Boolean algebras};

\draw(292.0, -55.0) node[anchor=north west,align=left] {Other algebras related to logic};

\draw(282.55, -55.7) node[anchor=north west,align=left] {Logical aspects of \\ Łukasiewicz and Post algebras};

\draw(290.5, -55.7) node[anchor=north west,align=left] {Logical aspects of lattices\\ and related structures};

\draw(282.55, -56.9) node[anchor=north west,align=left] {Cylindric and polyadic \\ algebras; relation algebras};

\draw(290.0, -56.9) node[anchor=north west,align=left] {Abstract algebraic logic};

\draw(282.55, -58.1) node[anchor=north west,align=left] {Categorical logic, topoi};

\draw(289.25, -58.1) node[anchor=north west,align=left] {Quantum logic};

\draw(281.55, -58.900000000000006) node[anchor=north west] { \large Nonstandard models};
\draw (281.55, -58.900000000000006) rectangle (300.45,-61.800000000000004);
\draw(282.55, -59.900000000000006) node[anchor=north west,align=left] {Nonstandard models in mathematics};

\draw(291.5, -59.900000000000006) node[anchor=north west,align=left] {Other applications of nonstandard\\ models (economics, physics, etc.)};

\draw(282.55, -61.10000000000001) node[anchor=north west,align=left] {Nonstandard models of arithmetic};

\draw(260.05, -66.7) node[anchor=north west] { \large Nonassociative rings and algebras};
\draw (260.05, -66.7) rectangle (309.25,-90.9);
\draw(261.05, -67.7) node[anchor=north west] { \large General nonassociative rings};
\draw (261.05, -67.7) rectangle (285.40000000000003,-75.60000000000001);
\draw(262.05, -68.7) node[anchor=north west,align=left] {Structure theory for nonassociative algebras};

\draw(273.75, -68.7) node[anchor=north west,align=left] {Automorphisms, derivations, other operators\\ (nonassociative rings and algebras)};

\draw(262.05, -69.9) node[anchor=north west,align=left] {Other \(n\)-ary compositions \((n \ge 3)\)};

\draw(273.25, -69.9) node[anchor=north west,align=left] {General theory of \\ nonassociative rings and algebras};

\draw(262.05, -71.10000000000001) node[anchor=north west,align=left] {Nonassociative division algebras};

\draw(270.75, -71.10000000000001) node[anchor=north west,align=left] {Noncommutative Jordan algebras};

\draw(262.05, -71.8) node[anchor=north west,align=left] {Radical theory (nonassociative\\ rings and algebras)};

\draw(270.25, -71.8) node[anchor=north west,align=left] {Free nonassociative algebras};

\draw(277.95, -71.8) node[anchor=north west,align=left] {Nonassociative algebras \\ satisfying other identities};

\draw(262.05, -73.0) node[anchor=north west,align=left] {Quadratic algebras (but not\\ quadratic Jordan algebras)};

\draw(269.5, -73.0) node[anchor=north west,align=left] {Gröbner-Shirshov bases \\ in nonassociative algebras};

\draw(276.7, -73.0) node[anchor=north west,align=left] {Power-associative rings};

\draw(262.05, -74.2) node[anchor=north west,align=left] {Ternary compositions};

\draw(267.75, -74.2) node[anchor=north west,align=left] {Composition algebras};

\draw(273.45, -74.2) node[anchor=north west,align=left] {Flexible algebras};

\draw(278.40000000000003, -74.2) node[anchor=north west,align=left] {Leibniz algebras};

\draw(262.05, -74.9) node[anchor=north west,align=left] {Valued algebras};

\draw(266.5, -74.9) node[anchor=north west,align=left] {Superalgebras};

\draw(285.5, -67.7) node[anchor=north west] { \large Lie algebras and Lie superalgebras};
\draw (285.5, -67.7) rectangle (309.15,-82.8);
\draw(286.5, -68.7) node[anchor=north west,align=left] {Automorphisms, derivations, other operators\\ for Lie algebras and super algebras};

\draw(297.95, -68.7) node[anchor=north west,align=left] {Homological methods in Lie (super)algebras};

\draw(286.5, -69.9) node[anchor=north west,align=left] {Kac-Moody (super)algebras; extended \\ affine Lie algebras; toroidal Lie algebras};

\draw(297.7, -69.9) node[anchor=north west,align=left] {Representations of Lie algebras and Lie\\ superalgebras, algebraic theory (weights)};

\draw(286.5, -71.10000000000001) node[anchor=north west,align=left] {Lie (super)algebras associated with other\\ structures (associative, Jordan, etc.)};

\draw(297.45, -71.10000000000001) node[anchor=north west,align=left] {Infinite-dimensional Lie (super)algebras};

\draw(286.5, -72.3) node[anchor=north west,align=left] {Lie algebras of linear algebraic groups};

\draw(296.95, -72.3) node[anchor=north west,align=left] {Coadjoint orbits; nilpotent varieties};

\draw(286.5, -73.0) node[anchor=north west,align=left] {Identities, free Lie (super)algebras};

\draw(296.2, -73.0) node[anchor=north west,align=left] {Universal enveloping (super)algebras};

\draw(286.5, -73.7) node[anchor=north west,align=left] {Quantum groups (quantized enveloping\\ algebras) and related deformations};

\draw(296.2, -73.7) node[anchor=north west,align=left] {Representations of Lie algebras and\\ Lie superalgebras, analytic theory};

\draw(286.5, -74.9) node[anchor=north west,align=left] {Solvable, nilpotent (super)algebras};

\draw(295.95, -74.9) node[anchor=north west,align=left] {Applications of Lie algebras and \\ superalgebras to integrable systems};

\draw(286.5, -76.10000000000001) node[anchor=north west,align=left] {Cohomology of Lie (super)algebras};

\draw(295.45, -76.10000000000001) node[anchor=north west,align=left] {Vertex operators; vertex operator\\ algebras and related structures};

\draw(286.5, -77.30000000000001) node[anchor=north west,align=left] {Applications of Lie \\ (super)algebras to physics, etc.};

\draw(295.2, -77.30000000000001) node[anchor=north west,align=left] {Lie bialgebras; Lie coalgebras};

\draw(286.5, -78.5) node[anchor=north west,align=left] {Lie algebras of vector fields\\ and related (super) algebras};

\draw(294.45, -78.5) node[anchor=north west,align=left] {Virasoro and related algebras};

\draw(286.5, -79.7) node[anchor=north west,align=left] {Hom-Lie and related algebras};

\draw(294.2, -79.7) node[anchor=north west,align=left] {Exceptional (super)algebras};

\draw(301.65, -79.7) node[anchor=north west,align=left] {Modular Lie (super)algebras};

\draw(286.5, -80.4) node[anchor=north west,align=left] {Structure theory for Lie\\ algebras and superalgebras};

\draw(293.7, -80.4) node[anchor=north west,align=left] {Graded Lie (super)algebras};

\draw(300.9, -80.4) node[anchor=north west,align=left] {Simple, semisimple, \\ reductive (super)algebras};

\draw(286.5, -81.6) node[anchor=north west,align=left] {Yang-Baxter equations \\ and Rota-Baxter operators};

\draw(293.45, -81.6) node[anchor=north west,align=left] {Color Lie (super)algebras};

\draw(300.4, -81.6) node[anchor=north west,align=left] {Poisson algebras};

\draw(305.1, -81.6) node[anchor=north west,align=left] {Root systems};

\draw(261.05, -82.9) node[anchor=north west] { \large Jordan algebras (algebras, triples and pairs)};
\draw (261.05, -82.9) rectangle (283.2,-90.80000000000001);
\draw(262.05, -83.9) node[anchor=north west,align=left] {Associated geometries of Jordan algebras};

\draw(272.75, -83.9) node[anchor=north west,align=left] {Associated manifolds of Jordan algebras};

\draw(262.05, -84.60000000000001) node[anchor=north west,align=left] {Identities and free Jordan structures};

\draw(272.0, -84.60000000000001) node[anchor=north west,align=left] {Division algebras and Jordan algebras};

\draw(262.05, -85.30000000000001) node[anchor=north west,align=left] {Structure theory for Jordan algebras};

\draw(271.75, -85.30000000000001) node[anchor=north west,align=left] {Simple, semisimple Jordan algebras};

\draw(262.05, -86.0) node[anchor=north west,align=left] {Idempotents, Peirce decompositions};

\draw(271.25, -86.0) node[anchor=north west,align=left] {Associated groups, \\ automorphisms of Jordan algebras};

\draw(262.05, -87.2) node[anchor=north west,align=left] {Exceptional Jordan structures};

\draw(270.0, -87.2) node[anchor=north west,align=left] {Finite-dimensional \\ structures of Jordan algebras};

\draw(262.05, -88.4) node[anchor=north west,align=left] {Jordan structures associated\\ with other structures};

\draw(269.75, -88.4) node[anchor=north west,align=left] {Radicals in Jordan algebras};

\draw(262.05, -89.60000000000001) node[anchor=north west,align=left] {Jordan structures on \\ Banach spaces and algebras};

\draw(269.25, -89.60000000000001) node[anchor=north west,align=left] {Applications of Jordan \\ algebras to physics, etc.};

\draw(276.2, -89.60000000000001) node[anchor=north west,align=left] {Super structures};

\draw(283.3, -82.9) node[anchor=north west] { \large Other nonassociative rings and algebras};
\draw (283.3, -82.9) rectangle (300.7,-86.5);
\draw(284.3, -83.9) node[anchor=north west,align=left] {(non-Lie) Hom algebras and topics};

\draw(293.25, -83.9) node[anchor=north west,align=left] {Mal’tsev rings and algebras};

\draw(284.3, -84.60000000000001) node[anchor=north west,align=left] {\((\gamma, \delta)\)-rings,\\ including \((1,-1)\)-rings};

\draw(291.75, -84.60000000000001) node[anchor=north west,align=left] {Right alternative rings};

\draw(284.3, -85.80000000000001) node[anchor=north west,align=left] {Lie-admissible algebras};

\draw(290.75, -85.80000000000001) node[anchor=north west,align=left] {Alternative rings};

\draw(295.7, -85.80000000000001) node[anchor=north west,align=left] {Genetic algebras};

\draw(283.3, -86.6) node[anchor=north west,align=left] {Computational methods for problems pertaining\\ to nonassociative rings and algebras};

\draw(283.3, -87.8) node[anchor=north west,align=left] {History of nonassociative rings and algebras};

\draw(260.05, -91.0) node[anchor=north west] { \large Order, lattices, ordered algebraic structures};
\draw (260.05, -91.0) rectangle (307.3,-106.7);
\draw(261.05, -92.0) node[anchor=north west] { \large Distributive lattices};
\draw (261.05, -92.0) rectangle (284.7,-96.8);
\draw(262.05, -93.0) node[anchor=north west,align=left] {Heyting algebras (lattice-theoretic aspects)};

\draw(273.75, -93.0) node[anchor=north west,align=left] {Post algebras (lattice-theoretic aspects)};

\draw(262.05, -93.7) node[anchor=north west,align=left] {De Morgan algebras, Łukasiewicz \\ algebras (lattice-theoretic aspects)};

\draw(271.75, -93.7) node[anchor=north west,align=left] {Structure and representation \\ theory of distributive lattices};

\draw(262.05, -94.9) node[anchor=north west,align=left] {Fuzzy lattices (soft \\ algebras) and related topics};

\draw(269.75, -94.9) node[anchor=north west,align=left] {Pseudocomplemented lattices};

\draw(277.2, -94.9) node[anchor=north west,align=left] {Other generalizations \\ of distributive lattices};

\draw(262.05, -96.1) node[anchor=north west,align=left] {Complete distributivity};

\draw(268.5, -96.1) node[anchor=north west,align=left] {Lattices and duality};

\draw(274.2, -96.1) node[anchor=north west,align=left] {Frames, locales};

\draw(278.65000000000003, -96.1) node[anchor=north west,align=left] {MV-algebras};

\draw(284.8, -92.0) node[anchor=north west] { \large Lattices};
\draw (284.8, -92.0) rectangle (307.2,-96.3);
\draw(285.8, -93.0) node[anchor=north west,align=left] {Continuous lattices and posets, applications};

\draw(297.5, -93.0) node[anchor=north west,align=left] {Lattice ideals, congruence relations};

\draw(285.8, -93.7) node[anchor=north west,align=left] {Representation theory of lattices};

\draw(294.75, -93.7) node[anchor=north west,align=left] {Complete lattices, completions};

\draw(285.8, -94.4) node[anchor=north west,align=left] {Structure theory of lattices};

\draw(293.5, -94.4) node[anchor=north west,align=left] {Generalizations of lattices};

\draw(285.8, -95.1) node[anchor=north west,align=left] {Free lattices, projective\\ lattices, word problems};

\draw(292.75, -95.1) node[anchor=north west,align=left] {Varieties of lattices};

\draw(298.7, -95.1) node[anchor=north west,align=left] {Topological lattices};

\draw(261.05, -96.9) node[anchor=north west] { \large Modular lattices, complemented lattices};
\draw (261.05, -96.9) rectangle (283.2,-99.80000000000001);
\draw(262.05, -97.9) node[anchor=north west,align=left] {Semimodular lattices, geometric lattices};

\draw(272.75, -97.9) node[anchor=north west,align=left] {Modular lattices, Desarguesian lattices};

\draw(262.05, -98.60000000000001) node[anchor=north west,align=left] {Complemented lattices, \\ orthocomplemented lattices and posets};

\draw(272.0, -98.60000000000001) node[anchor=north west,align=left] {Complemented modular lattices,\\ continuous geometries};

\draw(283.3, -96.9) node[anchor=north west] { \large Ordered sets};
\draw (283.3, -96.9) rectangle (305.2,-100.5);
\draw(284.3, -97.9) node[anchor=north west,align=left] {Combinatorics of partially ordered sets};

\draw(294.75, -97.9) node[anchor=north west,align=left] {Galois correspondences, closure \\ operators (in relation to ordered sets)};

\draw(284.3, -99.10000000000001) node[anchor=north west,align=left] {Generalizations of ordered sets};

\draw(292.75, -99.10000000000001) node[anchor=north west,align=left] {Algebraic aspects of posets};

\draw(284.3, -99.80000000000001) node[anchor=north west,align=left] {Partial orders, general};

\draw(290.75, -99.80000000000001) node[anchor=north west,align=left] {Total orders};

\draw(294.45, -99.80000000000001) node[anchor=north west,align=left] {Semilattices};

\draw(261.05, -100.6) node[anchor=north west] { \large Boolean algebras (Boolean rings)};
\draw (261.05, -100.6) rectangle (282.95,-104.69999999999999);
\draw(262.05, -101.6) node[anchor=north west,align=left] {Boolean algebras with additional \\ operations (diagonalizable algebras, etc.)};

\draw(273.25, -101.6) node[anchor=north west,align=left] {Structure theory of Boolean algebras};

\draw(262.05, -102.8) node[anchor=north west,align=left] {Chain conditions, complete algebras};

\draw(271.5, -102.8) node[anchor=north west,align=left] {Generalizations of Boolean algebras};

\draw(262.05, -103.5) node[anchor=north west,align=left] {Stone spaces (Boolean spaces)\\ and related structures};

\draw(270.0, -103.5) node[anchor=north west,align=left] {Ring-theoretic properties\\ of Boolean algebras};

\draw(276.95, -103.5) node[anchor=north west,align=left] {Boolean functions};

\draw(283.05, -100.6) node[anchor=north west] { \large Ordered structures};
\draw (283.05, -100.6) rectangle (301.2,-105.89999999999999);
\draw(284.05, -101.6) node[anchor=north west,align=left] {Ordered rings, algebras, modules};

\draw(292.75, -101.6) node[anchor=north west,align=left] {Ordered topological structures\\ (aspects of ordered structures)};

\draw(284.05, -102.8) node[anchor=north west,align=left] {BCK-algebras, BCI-algebras \\ (aspects of ordered structures)};

\draw(292.5, -102.8) node[anchor=north west,align=left] {Ordered semigroups and monoids};

\draw(284.05, -104.0) node[anchor=north west,align=left] {Ordered abelian groups, Riesz\\ groups, ordered linear spaces};

\draw(292.0, -104.0) node[anchor=north west,align=left] {Noether lattices};

\draw(296.7, -104.0) node[anchor=north west,align=left] {Ordered groups};

\draw(284.05, -105.19999999999999) node[anchor=north west,align=left] {Quantales};

\draw(261.05, -104.8) node[anchor=north west,align=left] {Computational methods for problems\\ pertaining to ordered structures};

\draw(261.05, -106.0) node[anchor=north west,align=left] {History of ordered structures};

\draw(260.05, -106.8) node[anchor=north west] { \large Linear and multilinear algebra; matrix theory};
\draw (260.05, -106.8) rectangle (307.3,-122.0);
\draw(261.05, -107.8) node[anchor=north west] { \large Basic linear algebra};
\draw (261.05, -107.8) rectangle (286.45,-121.89999999999999);
\draw(262.05, -108.8) node[anchor=north west,align=left] {Norms of matrices, numerical range, applications\\ of functional analysis to matrix theory};

\draw(274.75, -108.8) node[anchor=north west,align=left] {Quadratic and bilinear forms, inner products};

\draw(262.05, -110.0) node[anchor=north west,align=left] {Linear equations (linear algebraic aspects)};

\draw(273.5, -110.0) node[anchor=north west,align=left] {Canonical forms, reductions, classification};

\draw(262.05, -110.7) node[anchor=north west,align=left] {Applications of generalized inverses};

\draw(271.75, -110.7) node[anchor=north west,align=left] {Multilinear algebra, tensor calculus};

\draw(262.05, -111.39999999999999) node[anchor=north west,align=left] {Exterior algebra, Grassmann algebras};

\draw(271.75, -111.39999999999999) node[anchor=north west,align=left] {Inverse problems in linear algebra};

\draw(262.05, -112.1) node[anchor=north west,align=left] {Determinants, permanents, traces,\\ other special matrix functions};

\draw(271.0, -112.1) node[anchor=north west,align=left] {Other algebras built from modules};

\draw(262.05, -113.3) node[anchor=north west,align=left] {Matrix equations and identities};

\draw(270.5, -113.3) node[anchor=north west,align=left] {Linear inequalities of matrices};

\draw(262.05, -114.0) node[anchor=north west,align=left] {Vector spaces, linear \\ dependence, rank, lineability};

\draw(270.0, -114.0) node[anchor=north west,align=left] {Matrix exponential and \\ similar functions of matrices};

\draw(277.95, -114.0) node[anchor=north west,align=left] {Diagonalization, Jordan forms};

\draw(262.05, -115.2) node[anchor=north west,align=left] {Algebraic systems of matrices};

\draw(270.0, -115.2) node[anchor=north west,align=left] {Max-plus and related algebras};

\draw(277.95, -115.2) node[anchor=north west,align=left] {Inequalities involving \\ eigenvalues and eigenvectors};

\draw(262.05, -116.4) node[anchor=north west,align=left] {Matrices over function rings\\ in one or more variables};

\draw(269.75, -116.4) node[anchor=north west,align=left] {Linear transformations, \\ semilinear transformations};

\draw(276.95, -116.4) node[anchor=north west,align=left] {Theory of matrix inversion\\ and generalized inverses};

\draw(262.05, -117.6) node[anchor=north west,align=left] {Miscellaneous inequalities\\ involving matrices};

\draw(269.25, -117.6) node[anchor=north west,align=left] {Clifford algebras, spinors};

\draw(276.45, -117.6) node[anchor=north west,align=left] {Vector and tensor algebra,\\ theory of invariants};

\draw(262.05, -118.8) node[anchor=north west,align=left] {Matrix completion problems};

\draw(269.25, -118.8) node[anchor=north west,align=left] {Factorization of matrices};

\draw(276.2, -118.8) node[anchor=north west,align=left] {Commutativity of matrices};

\draw(262.05, -119.5) node[anchor=north west,align=left] {Applications of Clifford\\ algebras to physics, etc.};

\draw(269.0, -119.5) node[anchor=north west,align=left] {Linear preserver problems};

\draw(275.95, -119.5) node[anchor=north west,align=left] {Conditioning of matrices};

\draw(262.05, -120.69999999999999) node[anchor=north west,align=left] {Eigenvalues, singular \\ values, and eigenvectors};

\draw(268.75, -120.69999999999999) node[anchor=north west,align=left] {Matrix pencils};

\draw(286.55, -107.8) node[anchor=north west] { \large Special matrices};
\draw (286.55, -107.8) rectangle (307.2,-113.3);
\draw(287.55, -108.8) node[anchor=north west,align=left] {Toeplitz, Cauchy, and related matrices};

\draw(297.75, -108.8) node[anchor=north west,align=left] {Random matrices (algebraic aspects)};

\draw(287.55, -109.5) node[anchor=north west,align=left] {Matrices over special rings \\ (quaternions, finite fields, etc.)};

\draw(296.75, -109.5) node[anchor=north west,align=left] {Positive matrices and their \\ generalizations; cones of matrices};

\draw(287.55, -110.7) node[anchor=north west,align=left] {Boolean and Hadamard matrices};

\draw(295.5, -110.7) node[anchor=north west,align=left] {Hermitian, skew-Hermitian,\\ and related matrices};

\draw(287.55, -111.89999999999999) node[anchor=north west,align=left] {Sign pattern matrices};

\draw(293.5, -111.89999999999999) node[anchor=north west,align=left] {Matrices of integers};

\draw(299.2, -111.89999999999999) node[anchor=north west,align=left] {Orthogonal matrices};

\draw(287.55, -112.6) node[anchor=north west,align=left] {Matrix Lie algebras};

\draw(293.0, -112.6) node[anchor=north west,align=left] {Stochastic matrices};

\draw(298.45, -112.6) node[anchor=north west,align=left] {Fuzzy matrices};

\draw(286.55, -113.39999999999999) node[anchor=north west,align=left] {History of linear algebra};

\draw(260.05, -122.1) node[anchor=north west] { \large Field theory and polynomials};
\draw (260.05, -122.1) rectangle (306.05,-135.7);
\draw(261.05, -123.1) node[anchor=north west] { \large Real and complex fields};
\draw (261.05, -123.1) rectangle (284.45,-126.5);
\draw(262.05, -124.1) node[anchor=north west,align=left] {Fields related with sums of squares (formally\\ real fields, Pythagorean fields, etc.)};

\draw(274.0, -124.1) node[anchor=north west,align=left] {Polynomials in real and complex fields:\\ location of zeros (algebraic theorems)};

\draw(262.05, -125.3) node[anchor=north west,align=left] {Polynomials in real and \\ complex fields: factorization};

\draw(284.55, -123.1) node[anchor=north west] { \large General field theory};
\draw (284.55, -123.1) rectangle (305.95,-126.69999999999999);
\draw(285.55, -124.1) node[anchor=north west,align=left] {Finite fields (field-theoretic aspects)};

\draw(296.0, -124.1) node[anchor=north west,align=left] {Special polynomials in general fields};

\draw(285.55, -124.8) node[anchor=north west,align=left] {Polynomials in general \\ fields (irreducibility, etc.)};

\draw(293.5, -124.8) node[anchor=north west,align=left] {Hilbertian fields; Hilbert’s\\ irreducibility theorem};

\draw(285.55, -126.0) node[anchor=north west,align=left] {Equations in general fields};

\draw(293.0, -126.0) node[anchor=north west,align=left] {Skew fields, division rings};

\draw(300.45, -126.0) node[anchor=north west,align=left] {Field arithmetic};

\draw(261.05, -126.8) node[anchor=north west] { \large Connections between field theory and logic};
\draw (261.05, -126.8) rectangle (280.7,-129.2);
\draw(262.05, -127.8) node[anchor=north west,align=left] {Nonstandard arithmetic and field theory};

\draw(272.5, -127.8) node[anchor=north west,align=left] {Ultraproducts and field theory};

\draw(262.05, -128.5) node[anchor=north west,align=left] {Decidability and field theory};

\draw(270.0, -128.5) node[anchor=north west,align=left] {Model theory of fields};

\draw(280.8, -126.8) node[anchor=north west] { \large Field extensions};
\draw (280.8, -126.8) rectangle (299.7,-129.9);
\draw(281.8, -127.8) node[anchor=north west,align=left] {Separable extensions, Galois theory};

\draw(291.25, -127.8) node[anchor=north west,align=left] {Transcendental field extensions};

\draw(281.8, -128.5) node[anchor=north west,align=left] {Inseparable field extensions};

\draw(289.5, -128.5) node[anchor=north west,align=left] {Algebraic field extensions};

\draw(281.8, -129.2) node[anchor=north west,align=left] {Inverse Galois theory};

\draw(261.05, -130.0) node[anchor=north west] { \large Differential and difference algebra};
\draw (261.05, -130.0) rectangle (279.45,-132.4);
\draw(262.05, -131.0) node[anchor=north west,align=left] {\(p\)-adic differential equations};

\draw(271.0, -131.0) node[anchor=north west,align=left] {Abstract differential equations};

\draw(262.05, -131.7) node[anchor=north west,align=left] {Differential algebra};

\draw(267.75, -131.7) node[anchor=north west,align=left] {Difference algebra};

\draw(279.55, -130.0) node[anchor=north west] { \large Topological fields};
\draw (279.55, -130.0) rectangle (297.95,-133.8);
\draw(280.55, -131.0) node[anchor=north west,align=left] {General valuation theory for fields};

\draw(290.0, -131.0) node[anchor=north west,align=left] {Non-Archimedean valued fields};

\draw(280.55, -131.7) node[anchor=north west,align=left] {Formally \(p\)-adic fields};

\draw(287.75, -131.7) node[anchor=north west,align=left] {Topological semifields};

\draw(280.55, -132.4) node[anchor=north west,align=left] {Krasner-Tate algebras};

\draw(286.5, -132.4) node[anchor=north west,align=left] {Ordered fields};

\draw(290.7, -132.4) node[anchor=north west,align=left] {Normed fields};

\draw(280.55, -133.1) node[anchor=north west,align=left] {Valued fields};

\draw(261.05, -133.9) node[anchor=north west] { \large Homological methods (field theory)};
\draw (261.05, -133.9) rectangle (275.95,-135.6);
\draw(262.05, -134.9) node[anchor=north west,align=left] {Cohomological dimension of fields};

\draw(271.0, -134.9) node[anchor=north west,align=left] {Galois cohomology};

\draw(276.05, -133.9) node[anchor=north west,align=left] {Computational methods for \\ problems pertaining to field theory};

\draw(285.5, -133.9) node[anchor=north west] { \large Generalizations of fields};
\draw (285.5, -133.9) rectangle (293.6,-135.6);
\draw(286.5, -134.9) node[anchor=north west,align=left] {Near-fields};

\draw(289.95, -134.9) node[anchor=north west,align=left] {Semifields};

\draw(293.7, -133.9) node[anchor=north west,align=left] {History of field theory};

\draw(314.1, -1) node[anchor=north west] { \large K-Theory};
\draw (314.1, -1) rectangle (359.6,-22.099999999999998);
\draw(315.1, -2) node[anchor=north west] { \large Higher algebraic \(K\)-theory};
\draw (315.1, -2) rectangle (338.25,-6.3);
\draw(316.1, -3) node[anchor=north west,align=left] {Computations of higher \(K\)-theory of rings};

\draw(327.8, -3) node[anchor=north west,align=left] {Karoubi-Villamayor-Gersten \(K\)-theory};

\draw(316.1, -3.7) node[anchor=north west,align=left] {Higher symbols, Milnor \(K\)-theory};

\draw(325.55, -3.7) node[anchor=north west,align=left] {Negative \(K\)-theory, NK and Nil};

\draw(316.1, -4.4) node[anchor=north west,align=left] {Algebraic \(K\)-theory of spaces};

\draw(324.8, -4.4) node[anchor=north west,align=left] {\(K\)-theory and homology; \\ cyclic homology and cohomology};

\draw(316.1, -5.6) node[anchor=north west,align=left] {\(Q\)- and plus-constructions};

\draw(324.05, -5.6) node[anchor=north west,align=left] {Symmetric monoidal categories};

\draw(338.35, -2) node[anchor=north west] { \large \(K\)-theory in number theory};
\draw (338.35, -2) rectangle (359.5,-5.4);
\draw(339.35, -3) node[anchor=north west,align=left] {Étale cohomology, higher regulators, zeta \\ and \(L\)-functions (\(K\)-theoretic aspects)};

\draw(351.3, -3) node[anchor=north west,align=left] {Generalized class field theory\\ (\(K\)-theoretic aspects)};

\draw(339.35, -4.2) node[anchor=north west,align=left] {Symbols and arithmetic \\ (\(K\)-theoretic aspects)};

\draw(338.35, -5.5) node[anchor=north west,align=left] {History of \(K\)-theory};

\draw(315.1, -6.3999999999999995) node[anchor=north west] { \large Obstructions from topology};
\draw (315.1, -6.3999999999999995) rectangle (336.25,-9.299999999999999);
\draw(316.1, -7.3999999999999995) node[anchor=north west,align=left] {Finiteness and other obstructions in \(K_0\)};

\draw(327.8, -7.3999999999999995) node[anchor=north west,align=left] {Whitehead (and related) torsion};

\draw(316.1, -8.1) node[anchor=north west,align=left] {Obstructions to group actions\\ (\(K\)-theoretic aspects)};

\draw(324.05, -8.1) node[anchor=north west,align=left] {Surgery obstructions \\ (\(K\)-theoretic aspects)};

\draw(336.35, -6.3999999999999995) node[anchor=north west] { \large Topological \(K\)-theory};
\draw (336.35, -6.3999999999999995) rectangle (357.5,-10.5);
\draw(337.35, -7.3999999999999995) node[anchor=north west,align=left] {Riemann-Roch theorems, Chern characters};

\draw(347.8, -7.3999999999999995) node[anchor=north west,align=left] {\(J\)-homomorphism, Adams operations};

\draw(337.35, -8.1) node[anchor=north west,align=left] {Connective \(K\)-theory, cobordism};

\draw(346.55, -8.1) node[anchor=north west,align=left] {Twisted \(K\)-theory; \\ differential \(K\)-theory};

\draw(337.35, -9.3) node[anchor=north west,align=left] {Geometric applications of\\ topological \(K\)-theory};

\draw(344.3, -9.3) node[anchor=north west,align=left] {Equivariant \(K\)-theory};

\draw(315.1, -10.599999999999998) node[anchor=north west] { \large Steinberg groups and \(K_2\)};
\draw (315.1, -10.599999999999998) rectangle (334.5,-13.499999999999998);
\draw(316.1, -11.599999999999998) node[anchor=north west,align=left] {Central extensions and Schur multipliers};

\draw(326.8, -11.599999999999998) node[anchor=north west,align=left] {\(K_2\) and the Brauer group};

\draw(316.1, -12.299999999999997) node[anchor=north west,align=left] {Symbols, presentations\\ and stability of \(K_2\)};

\draw(322.8, -12.299999999999997) node[anchor=north west,align=left] {Excision for \(K_2\)};

\draw(334.6, -10.599999999999998) node[anchor=north west] { \large \(K\)-theory and operator algebras};
\draw (334.6, -10.599999999999998) rectangle (353.5,-12.999999999999998);
\draw(335.6, -11.599999999999998) node[anchor=north west,align=left] {\(K_0\) as an ordered group, traces};

\draw(345.05, -11.599999999999998) node[anchor=north west,align=left] {Kasparov theory (\(KK\)-theory)};

\draw(335.6, -12.299999999999997) node[anchor=north west,align=left] {Ext and \(K\)-homology};

\draw(341.8, -12.299999999999997) node[anchor=north west,align=left] {Index theory};

\draw(315.1, -13.599999999999998) node[anchor=north west] { \large Grothendieck groups and \(K_0\)};
\draw (315.1, -13.599999999999998) rectangle (333.75,-17.2);
\draw(316.1, -14.599999999999998) node[anchor=north west,align=left] {\(K_0\) of group rings and orders};

\draw(325.05, -14.599999999999998) node[anchor=north west,align=left] {Stability for projective modules};

\draw(316.1, -15.299999999999997) node[anchor=north west,align=left] {Efficient generation of modules};

\draw(324.55, -15.299999999999997) node[anchor=north west,align=left] {Frobenius induction, Burnside\\ and representation rings};

\draw(316.1, -16.5) node[anchor=north west,align=left] {\(K_0\) of other rings};

\draw(333.85, -13.599999999999998) node[anchor=north west] { \large \(K\)-theory of forms};
\draw (333.85, -13.599999999999998) rectangle (352.25,-16.5);
\draw(334.85, -14.599999999999998) node[anchor=north west,align=left] {Hermitian \(K\)-theory, relations\\ with \(K\)-theory of rings};

\draw(343.8, -14.599999999999998) node[anchor=north west,align=left] {Stability for quadratic modules};

\draw(334.85, -15.799999999999997) node[anchor=north west,align=left] {\(L\)-theory of group rings};

\draw(342.3, -15.799999999999997) node[anchor=north west,align=left] {Witt groups of rings};

\draw(315.1, -17.299999999999997) node[anchor=north west] { \large Whitehead groups and \(K_1\)};
\draw (315.1, -17.299999999999997) rectangle (332.75,-19.699999999999996);
\draw(316.1, -18.299999999999997) node[anchor=north west,align=left] {\(K_1\) of group rings and orders};

\draw(325.05, -18.299999999999997) node[anchor=north west,align=left] {Congruence subgroup problems};

\draw(316.1, -18.999999999999996) node[anchor=north west,align=left] {Stability for linear groups};

\draw(323.55, -18.999999999999996) node[anchor=north west,align=left] {Stable range conditions};

\draw(332.85, -17.299999999999997) node[anchor=north west] { \large \(K\)-theory in geometry};
\draw (332.85, -17.299999999999997) rectangle (350.5,-20.199999999999996);
\draw(333.85, -18.299999999999997) node[anchor=north west,align=left] {Algebraic cycles and motivic \\ cohomology (\(K\)-theoretic aspects)};

\draw(343.55, -18.299999999999997) node[anchor=north west,align=left] {Relations of \(K\)-theory\\ with cohomology theories};

\draw(333.85, -19.499999999999996) node[anchor=north west,align=left] {\(K\)-theory of schemes};

\draw(315.1, -20.299999999999997) node[anchor=north west] { \large Miscellaneous applications of \(K\)-theory};
\draw (315.1, -20.299999999999997) rectangle (328.3,-21.999999999999996);
\draw(316.1, -21.299999999999997) node[anchor=north west,align=left] {Miscellaneous applications of \(K\)-theory};

\draw(328.40000000000003, -20.299999999999997) node[anchor=north west,align=left] {Computational methods for \\ problems pertaining to \(K\)-theory};

\draw(314.1, -22.2) node[anchor=north west] { \large General algebraic systems};
\draw (314.1, -22.2) rectangle (358.0,-31.2);
\draw(315.1, -23.2) node[anchor=north west] { \large Algebraic structures};
\draw (315.1, -23.2) rectangle (337.40000000000003,-29.9);
\draw(316.1, -24.2) node[anchor=north west,align=left] {Structure theory of algebraic structures};

\draw(326.8, -24.2) node[anchor=north west,align=left] {Relational systems, laws of composition};

\draw(316.1, -24.9) node[anchor=north west,align=left] {Operations and polynomials in \\ algebraic structures, primal algebras};

\draw(326.05, -24.9) node[anchor=north west,align=left] {Subalgebras, congruence relations};

\draw(316.1, -26.1) node[anchor=north west,align=left] {Automorphisms and endomorphisms\\ of algebraic structures};

\draw(324.55, -26.1) node[anchor=north west,align=left] {Applications of universal\\ algebra in computer science};

\draw(316.1, -27.3) node[anchor=north west,align=left] {Fuzzy algebraic structures};

\draw(323.3, -27.3) node[anchor=north west,align=left] {Word problems (aspects\\ of algebraic structures)};

\draw(330.0, -27.3) node[anchor=north west,align=left] {Equational compactness};

\draw(316.1, -28.5) node[anchor=north west,align=left] {Heterogeneous algebras};

\draw(322.3, -28.5) node[anchor=north west,align=left] {Infinitary algebras};

\draw(327.75, -28.5) node[anchor=north west,align=left] {Finitary algebras};

\draw(332.70000000000005, -28.5) node[anchor=north west,align=left] {Partial algebras};

\draw(316.1, -29.2) node[anchor=north west,align=left] {Unary algebras};

\draw(337.5, -23.2) node[anchor=north west] { \large Varieties};
\draw (337.5, -23.2) rectangle (357.9,-27.299999999999997);
\draw(338.5, -24.2) node[anchor=north west,align=left] {Equational logic, Mal’tsev conditions};

\draw(348.45, -24.2) node[anchor=north west,align=left] {Products, amalgamated products, and\\ other kinds of limits and colimits};

\draw(338.5, -25.4) node[anchor=north west,align=left] {Congruence modularity, \\ congruence distributivity};

\draw(345.45, -25.4) node[anchor=north west,align=left] {Subdirect products and\\ subdirect irreducibility};

\draw(338.5, -26.6) node[anchor=north west,align=left] {Injectives, projectives};

\draw(344.95, -26.6) node[anchor=north west,align=left] {Lattices of varieties};

\draw(350.9, -26.6) node[anchor=north west,align=left] {Free algebras};

\draw(337.5, -27.4) node[anchor=north west] { \large Other classes of algebras};
\draw (337.5, -27.4) rectangle (355.9,-29.799999999999997);
\draw(338.5, -28.4) node[anchor=north west,align=left] {Natural dualities for classes of algebras};

\draw(349.45, -28.4) node[anchor=north west,align=left] {Axiomatic model classes};

\draw(338.5, -29.099999999999998) node[anchor=north west,align=left] {Categories of algebras};

\draw(344.7, -29.099999999999998) node[anchor=north west,align=left] {Quasivarieties};

\draw(315.1, -30.0) node[anchor=north west,align=left] {Computational methods for problems \\ pertaining to general algebraic systems};

\draw(325.55, -30.0) node[anchor=north west,align=left] {History of general algebraic systems};

\draw(314.1, -31.299999999999997) node[anchor=north west] { \large History and biography};
\draw (314.1, -31.299999999999997) rectangle (338.8,-46.699999999999996);
\draw(315.1, -32.3) node[anchor=north west] { \large History of mathematics and mathematicians};
\draw (315.1, -32.3) rectangle (338.70000000000005,-46.599999999999994);
\draw(316.1, -33.3) node[anchor=north west,align=left] {History of mathematics in the 17th century};

\draw(327.3, -33.3) node[anchor=north west,align=left] {History of mathematics in the 18th century};

\draw(316.1, -34.0) node[anchor=north west,align=left] {History of mathematics in the 19th century};

\draw(327.3, -34.0) node[anchor=north west,align=left] {History of mathematics in the 20th century};

\draw(316.1, -34.699999999999996) node[anchor=north west,align=left] {History of mathematics in the 21st century};

\draw(327.3, -34.699999999999996) node[anchor=north west,align=left] {History of mathematics in Ancient Babylon};

\draw(316.1, -35.4) node[anchor=north west,align=left] {Sociology (and profession) of mathematics};

\draw(327.05, -35.4) node[anchor=north west,align=left] {History of mathematics of the indigenous\\ cultures of Africa, Asia, and Oceania};

\draw(316.1, -36.599999999999994) node[anchor=north west,align=left] {History of mathematics of the indigenous\\ cultures of Europe (pre-Greek, etc.)};

\draw(326.8, -36.599999999999994) node[anchor=north west,align=left] {History of mathematics in Southeast Asia};

\draw(316.1, -37.8) node[anchor=north west,align=left] {History of mathematics in Ancient Egypt};

\draw(326.55, -37.8) node[anchor=north west,align=left] {Development of contemporary mathematics};

\draw(316.1, -38.5) node[anchor=north west,align=left] {Collected or selected works; \\ reprintings or translations of classics};

\draw(326.55, -38.5) node[anchor=north west,align=left] {History of mathematics at institutions\\ and academies (non-university)};

\draw(316.1, -39.699999999999996) node[anchor=north west,align=left] {History of mathematics of the \\ indigenous cultures of the Americas};

\draw(325.55, -39.699999999999996) node[anchor=north west,align=left] {History of mathematics in the 15th\\ and 16th centuries, Renaissance};

\draw(316.1, -40.9) node[anchor=north west,align=left] {Future perspectives in mathematics};

\draw(325.3, -40.9) node[anchor=north west,align=left] {General histories, source books};

\draw(316.1, -41.599999999999994) node[anchor=north west,align=left] {History of mathematics in \\ Paleolithic and Neolithic times};

\draw(324.55, -41.599999999999994) node[anchor=north west,align=left] {History of mathematics in China};

\draw(316.1, -42.8) node[anchor=north west,align=left] {History of mathematics in Japan};

\draw(324.55, -42.8) node[anchor=north west,align=left] {History of mathematics in India};

\draw(316.1, -43.5) node[anchor=north west,align=left] {History of mathematics in late\\ antiquity and medieval Europe};

\draw(324.3, -43.5) node[anchor=north west,align=left] {History of mathematics \\ in Ancient Greece and Rome};

\draw(331.5, -43.5) node[anchor=north west,align=left] {History of mathematics \\ in the Golden Age of Islam};

\draw(316.1, -44.699999999999996) node[anchor=north west,align=left] {Biographies, obituaries,\\ personalia, bibliographies};

\draw(323.3, -44.699999999999996) node[anchor=north west,align=left] {Ethnomathematics, general};

\draw(330.25, -44.699999999999996) node[anchor=north west,align=left] {History of mathematics\\ at specific universities};

\draw(316.1, -45.89999999999999) node[anchor=north west,align=left] {Schools of mathematics};

\draw(322.3, -45.89999999999999) node[anchor=north west,align=left] {Bibliographic studies};

\draw(328.25, -45.89999999999999) node[anchor=north west,align=left] {Historiography};

\end{tikzpicture}

\end{document}